\documentclass[12pt]{amsart}
\usepackage{templatestyle}
\usepackage{templatecommand}
\usepackage{templatetensor}


\title[Negative holomorphic sectional curvature and K\"ahler-Ricci flow]
{K\"ahler manifolds with negative holomorphic sectional curvature, K\"ahler-Ricci flow approach} 

\author
{Ryosuke Nomura}

\address{Graduate School of Mathematical Sciences, The University of Tokyo \endgraf
	3-8-1 Komaba, Meguro-ku, Tokyo, 153-8914, Japan.}

\email{nomu@ms.u-tokyo.ac.jp}

\thanks{Classification AMS 2010: 53C55, 
	32W20.
}

\keywords{holomorphic sectional curvature, K\"ahler-Ricci flow, , Monge-Amp\`ere equation.
}


\begin{document}

\begin{abstract}
	Recently, Wu-Yau and Tosatti-Yang established the connection between  the negativity of  holomorphic sectional curvatures and the positivity of canonical bundles for  compact K\"ahler manifolds. In this short note, we give anothe proof of their theorems by using the K\"ahler-Ricci flow.
\end{abstract}

\maketitle

\section{Introduction}

In this note, we provide a \krf  \ approach to the following two theorems, which represent the relationship between the negativity of the holomorphic sectional curvature and the positivity of the canonical bundle $K_X$ of a compact \kahler \ manifold $X$.

\begin{theorem}[{[\citet[Theorem 2]{WuYau16}, \citet[Corollary 1.3]{TosattiYang15}]}]\label{amplethm}
If $X$ admits a \kahler \ form with strictly negative holomorphic sectional curvature, then the canonical bundle $K_X$ is ample. In particular, $X$ is projective.
\end{theorem}


\begin{theorem}[{\citep[Theorem 1.1]{TosattiYang15}}]\label{nefthm}
If $X$ admits a \kahler \ form with semi-negative holomorphic sectional curvature, then the canonical bundle $K_X$ is nef.
\end{theorem}

The original proofs of both theorems are based on the following idea, in \citep{WuYau16}, constructing a \kahler \ form $\ome \in 2\pi c_1(K_X) + \varepsilon [\omh]$ satisfying  
\begin{align*}
\ric(\ome ) = - \ome + \varepsilon \omh,
\end{align*}
and considering the limiting behavior of $\ome $ as $\varepsilon \searrow 0$. Here, $\omh $ is a \kahler \ form whose holomorphic sectional curvature is (strictly/semi-) negative.
The objective of this note is to simplify the proofs by replacing $\omega_\varepsilon$ by the \krf \ $\omt$. 

These theorems are originated from the conjecture of Yau (see \citep[Conjecture 1.2]{2014arXiv1403.4210H}).
For a historical background, we refere [\citet{2014arXiv1403.4210H}, \citet{WuYau16}, \citet{WuYau16remark}, \citet{TosattiYang15}, \citet{2016arXiv160601381D}] and the references therein. 

We remark that  Diverio and Trapani \citep{2016arXiv160601381D} showed that the amplenss of $K_X$ can be obtained under the assumption that the holomorphic sectional curvature is semi-negative everywhere and strictly negative at one point.  
For the moment, we can only prove the above two theorems.

\noindent 
\textbf{Acknowledgment. }The author would like to thank to his supervisor Prof. Shigeharu Takayama for various comments.
This work is supported by the Program for Leading Graduate Schools, MEXT, Japan.

\section{Properties of the \krf }

In this section, we summarize the basic properties of the \krf \ which will be used later. For more detailed exposition, we refere the book \citep{BEG13IntrotoKRF}.
In the following argument, we will denote by $X$ a compact \kahler \  manifold of dimension $n$. 

\begin{definition}
A smooth family of \kahler \ forms $\{ \omt \}_{t\ge 0}$ is called the \krf \ (resp. the normalized \krf )\ if it satisfies the following equation:
\begin{align}\label{krfeq}
\begin{cases}
	\dt \omt
		&= -\ric(\omt) + \lambda \omt, \\[5pt]
	\omt|_{t=0}
		&=\omz, 
\end{cases}
\end{align}
where $\lambda =0$ (resp.  $\lambda =-1$). 
\end{definition}

By considering the cohomology class in $H^{1,1 }(X, \rr)$ of (\refs{krfeq}), $\omt$ belongs to $\alpha_t \in H^{1,1 }(X, \rr)$ which is defined as 
\begin{align}\label{cohom}
	\alpha_t  = 
		\begin{cases}
			[\omz ] + 2\pi t c_1(K_X) &\mosi \lambda =0, \\[3pt]
			e^{-t }[\omz ] + (1-e^{-t}) 2\pi  c_1(K_X) &\mosi \lambda =-1.
		\end{cases}
\end{align}
The optimal existence theorem for the \krf \ is stated as follows. 
\begin{theorem}[{[\citet{Cao85Deformation}, \citet{Tsuji89ExistenceDegeneKE}, \citet{TianZhang06KRFProjGenType}], see also \citep[3.3.1]{BEG13IntrotoKRF}}]\label{existthm}
For any \kahler \ form $\omz$, the \krf \ (resp. the normalized \krf) $\omt$ starting from $\omz$ exists uniquely for $t \in [0,T)$ and cannnot extend beyond $T$,  where $T$ is defined by
\begin{align}\label{maximaltime}
	T \deq \sup \{ t>0 \mid \alpha_t \mbox{ defined by (\refs{cohom}) is a \kahler \ class}  \},
\end{align}
and called the maiximal existence time.
In particular, $\omt$ exists for $t\in [0,\infty)$ if and only if $K_X$ is nef, i.e. $2\pi c_1(K_X)$ belongs to the closure of the \kahler \ cone of $X$.
\end{theorem}

A simple maximum principle argument shows that the scalar curvature  is uniformly lower bounded along the \krf \ \citep[3.2.2]{BEG13IntrotoKRF}.
The direct consequence is the volume bounds for the \krf .

\begin{proposition}[{\citep[3.2.3]{BEG13IntrotoKRF}}]\label{volprop}
The following volume bounds hold:
\begin{enua}
	\item For any \krf \ $\omt$, there exists a constant $C>0$ such that for all $t \in [0,T)$, $\omt ^n \le e^{Ct}\omz^n$ holds.
	\item For any normalized \krf \ $\omt$, there exists a constant $C>0$ such that for all $t \in [0,T)$, $\omt ^n \le C \omz^n$ holds.
\end{enua}
\end{proposition}


The next Proposition due to Royden \citep[Lemma]{Royden80} (see also \citep[Lemma 2.1]{WongWuYau12}) will be used to obtain the $C^2$-estimate. 
This is essentially based on the symmetry of the curvature tensor of the \kahler \ forms. 
\begin{proposition}\label{Roydenprop}
Let $\omh$ be a \kahler \ form on $X$, and denote by $\widehat{H} $ the holomorphic sectional curvature of $\omh$. Assume that there exists a non-negative constant $\kappa \ge 0$ such that for any tangent vector $\xi$, we have
\begin{align}\label{holsect}
\widehat{H} (\xi ) \le -\kappa |\xi |_{\omh}^4 \le 0.
\end{align}
Then, for any \kahler \ form $\omega$, we have
\begin{align*}
\guijb \guklb ?{\widehat{R}}_i\jbar k \lbar ? \le -\kappa \dfrac{n+1}{2n} \left( \trom(\omh )\right)^2 \le 0,
\end{align*} 
where $\omega = \ii \gijb \dzidzjb $ and $?{\widehat{R}}_i\jbar k \lbar ? $ is the curvature tensor of $\omh$.
\end{proposition}


We need the parabolic Schwarz lemma obtained by Song-Tian \citep{SongTian07KRFSurfKod} applied to the identity map (see also \citep[3.2.6]{BEG13IntrotoKRF}). This is a parabolic analogue of the Schwarz lemma due to Yau \citep{Yau78Schwarz}. 
\begin{proposition}\label{schwarzprop}
Let $\omt$ be the \krf \ (resp. the normalized \krf) and $\omh$ be an arbitrary \kahler \ form.
Then we have the following inequality:
\begin{align*}
\dalt \log \tromt(\omh ) 
	\le - \lambda
		+ \dfrac{ \guijb (t) \guklb (t) ?{\widehat{R}}_i\jbar k \lbar ? }{ \tromt (\omh ) },
\end{align*}
where $\lambda$ is in (\refs{krfeq}), $\omt = \ii \gijb(t) \dzidzjb $ and $?{\widehat{R}}_i\jbar k \lbar ? $ is the curvature tensor of $\omh$.
\end{proposition}


\section{Proof of Theorems  via \krf 
}
\begin{proof}[Proof of Theorem \refs{nefthm}]
By the assumption in Theorem \refs{nefthm}, there exists  a \kahler \ form $\omh$ whose holomorphic sectional curvature is semi-negative i.e. $\kappa= 0$ in (\refs{holsect}). Let $\omt $ be the \krf \ stating from arbitrary \kahler \ form $\omz$ on $X$.  By Theorem \refs{existthm}, the nefness of $K_X$ is equivalent to the long time existence of $\omt$. By definition of the maximal existence time (\refs{maximaltime}) and Theorem \refs{existthm}, it  is enough to show that if $\omt$ exists for $[0,T_0)$ with $T_0<\infty$, then $\alpha_{T_0}$ is  a \kahler \  class. 

By Proposition \refs{schwarzprop} and ,Proposition \refs{Roydenprop} we have 
\begin{align*}
\dalt \log \tromt(\omh ) \le \dfrac{ \guijb (t) \guklb (t) ?{\widehat{R}}_i\jbar k \lbar ? }{ \tromt (\omh ) } \le 0.
\end{align*}
Applying the maximum principle, we have $\tromt(\omh) \le \max_X \tromz(\omh )  \ddeq  C$ and therefore for all $t \in [0,T_0)$ we get 
\begin{align}\label{C2ineq}
	 \dfrac{1}{C}\omh \le \omt. 
\end{align}
Therefore, the limiting class $\alpha_{T_0}$ has positive intersection with any subvariety of $X$. 
By Demailly-P\u aun's characterization of the \kahler \ cone \citep[Main Theorem 0.1]{DemaillyPaun04}, the limiting class $\alpha_{T_0}$ is  \kahler. 
\end{proof}

\begin{remark}\label{higherrem}
The idea of avoiding higher order estimates by using the Demailly-P\u aun's theorem can be found in the proof of \citep[Theorem 1.1]{Zhang10ScalFTSingKRF}.

We can prove that $\omt$ converges to a smooth \kahler \ form as $t\rightarrow T_0$, in particular $\alpha_{T_0}$ is  \kahler.
In fact, by using (\refs{C2ineq}) and  Proposition \refs{volprop} (a), we get the uniform $C^2$-estimate for $\omt$:
\begin{align}\label{C2est}
	 \dfrac{1}{C}\omh \le \omt \le C^\prime \omh.
\end{align}
Thefore we obtain the higher order estimates (see for example \citep[3.2.16]{BEG13IntrotoKRF}), which guarantees the convergence.
\end{remark}

\begin{proof}[Proof of Theorem \refs{amplethm}]
By the assumption in Theorem \refs{amplethm}, there exists  a \kahler \ form $\omh$ whose holomorphic sectional curvature is strictly negative i.e. $\kappa>0$ in (\refs{holsect}). Let $\omt $ be the normalized \krf \ stating from arbitrary \kahler \ form $\omz$ on $X$.
By Theorem \refs{nefthm}, $K_X$ is nef, and therefore $\omt$ exists for $t \in [0,\infty)$.
It is enough to show that there exists a sequence $\{t_i \}\subset [0,\infty)$ such that $t_i $ tends to $\infty$ and $\omti$ converges to some \kahler \ form $\omega_\infty $ as $i \rightarrow \infty$. 
In fact, since $\alpha_t = [\omt ]$ converges to $2\pi c_1(K_X)$ as $t \rightarrow \infty$, the \kahler \ form $\omega_\infty$ represents $2\pi c_1(K_X)$, and
therefore $K_X$ is ample. 

By  Proposition \refs{schwarzprop} and Proposition \refs{Roydenprop}, we get 
\begin{align*}
\dalt \log \tromt(\omh ) 
	\le 1+\dfrac{ \guijb (t) \guklb (t) ?{\widehat{R}}_i\jbar k \lbar ? }{ \tromt (\omh ) }
	\le 1-\kappa \dfrac{n+1}{2n}\tromt(\omh ) 
\end{align*}
Applying the maximum principle, we have $\tromt(\omh) \le C$ where
\begin{align*}
 C \deq \max \left\{\dfrac{2n}{\kappa (n+1)}, \max_X \tromz(\omh)  \right\}>0.
\end{align*}
Therefore for any $t \in [0,\infty )$, we get  
\begin{align}\label{C2ineq2}
	 \dfrac{1}{C}\omh \le \omt.
\end{align}
Combining with Proposition \refs{volprop} (b), we get $C^2$-estimate as in (\refs{C2est}). Similar argument as in Remark \refs{higherrem}, we get higher order esimates for $\omt$. Therefore, by taking subsequence, $\omti$ converges to some \kahler \ form $\omega_\infty $ as $i \rightarrow \infty$. 
\end{proof}

\begin{remark}
The limiting \kahler \ form $\omega_\infty$ is the unique \ke \ metric with negative Ricci curvature i.e. $\ric(\omega_\infty)= -\,  \omega_\infty$ and, without passing to a subsequence, $\omt$ converges smoothly to $\ominf$ as $t \rightarrow \infty$.
In fact, a classical result due to Cao \citep{Cao85Deformation} show that under the assumption on the ampleness of $K_X$, any normalized \krf \ $\omega_t$ converges to the \ke \ metric. 
\end{remark}

\nocite{*} 
\bibliographystyle{amsalphaurlmod}
\bibliography{reference}
\end{document}

