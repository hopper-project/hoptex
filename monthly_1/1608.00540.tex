
\documentclass[12pt]{amsart}
\usepackage{amsmath,amssymb,amsthm}
\usepackage{color,colordvi,graphicx,verbatim,MnSymbol}
\usepackage[noend]{algorithmic}
\usepackage{filecontents,ulem}

\headheight=8pt       \topmargin=0pt
\textheight=611pt     \textwidth=456pt
\oddsidemargin=6pt   \evensidemargin=6pt
\newtheorem{theorem}{Theorem}
\newtheorem{lemma}[theorem]{Lemma}
\newtheorem{corollary}[theorem]{Corollary}
\newtheorem{proposition}[theorem]{Proposition}

\theoremstyle{definition}
\newtheorem{definition}[theorem]{Definition}
\newtheorem{example}[theorem]{Example}
\newtheorem{remark}[theorem]{Remark}
\newtheorem{algorithm}[theorem]{Algorithm}

 

\copyrightinfo{}{}
\title{Trace test} 
\author[A.~Leykin]{Anton Leykin}
\address{School of Mathematics\\
         Georgia Institute of Technology\\
         686 Cherry Street\\
         Atlanta, GA 30332-0160 USA\\
         USA}
\email{leykin@math.gatech.edu}
\urladdr{http://people.math.gatech.edu/~aleykin3}
\author[F.~Sottile]{Frank Sottile}
\address{Frank Sottile\\
         Department of Mathematics\\
         Texas A\&M University\\
         College Station\\
         Texas \ 77843\\
         USA}
\email{sottile@math.tamu.edu}
\urladdr{http://www.math.tamu.edu/~sottile}
\thanks{Research of Leykin supported in part by NSF grant DMS-1151297}
\thanks{Research of Sottile supported in part by NSF grant DMS-1501370}
\subjclass{65H10}
\keywords{trace test, witness set, numerical algebraic geometry}
\begin{document}

\begin{abstract}
We give a brief derivation of the trace test to verify completeness of a partial witness set of an irreducible variety in affine or
projective space.
We then consider the trace test for subvarieties of products of projective spaces
working with multihomogeneous witness sets.
We show how a dimension reduction based on Bertini's Theorem
leads to a practical trace test in this case involving a curve in a small-dimensional affine space.
\end{abstract}
\maketitle

\section*{Introduction}
The fundamental data structure in numerical algebraic geometry~\cite{SW05} is a witness set, which is a
general linear section of an affine or projective variety~\cite{SV}.
A key algorithm is {{{\color{blue}{{\sl {numerical irreducible decomposition}}}}}} of a witness set for a reducible
variety.
This was introduced in~\cite{SVW_decomposition}; see~\cite[Ch.~15]{SW05} for a complete treatment.
That algorithm computes the {{{\color{blue}{{\sl {witness set partition}}}}}}---its blocks are witness sets
of each irreducible component.
It uses two subroutines.
{{{\color{blue}{{\sl {Monodromy break-up}}}}}} follows points of a witness set as the linear section varies along a
loop and gives an {\it a priori} finer partition into orbits of the computed monodromy
permutations~\cite{SVW_monodromy}.
A block $W$ of this orbit partition is verified to be a block of the witness set partition
using the {{{\color{blue}{{\sl {trace test}}}}}}~\cite{SVW_trace}:
Following its points as the linear section varies affinely,
$W$ is a block of the witness set partition if and only if the sum of the
points from $W$ (its {{{\color{blue}{{\sl {trace}}}}}}) also moves affinely.

Witness sets for subvarieties of products of projective spaces (multihomogeneous varieties) were introduced
in~\cite{HR15}.
Example~20 of~\cite{HR15} shows that the trace of a multihomogeneous witness set may be nonlinear.
Consequently, they propose to apply the trace test to a witness set from the ambient projective space of the Segre
embedding.
We observe in Remark~\ref{remark:Segre} that this may involve very large witness sets.
We propose an alternative method to certify irreducible components, using a dimension reduction that
sidesteps this potential bottleneck followed by a reduction to the ordinary trace test in an affine patch
on the product of projective spaces.

In \S\ref{S:affine} we present a simple derivation of the usual trace test in affine space.
While containing the same essential ideas as in~\cite{SVW_trace}, our derivation is shorter,
and we believe significantly clearer.
In \S\ref{S:product} we discuss multihomogeneous witness sets and numerical irreducible decomposition.
In \S\ref{S:dimension} we propose a modified trace test for multihomogeneous varieties that exploits a
reduction in dimension.
All proofs are placed in \S\ref{S:proofs} to streamline the exposition.

\section{Trace in an affine space}\label{S:affine}

Let ${{\color{blue}{V}}}\subset{{\mathbb{C}}}^n$ be an irreducible algebraic variety of dimension $m>0$.
We restrict to $m>0$, for if $m=0$, then $V$ is a single point.
Let $(x,y)$ be coordinates for ${{\mathbb{C}}}^n$ with $x\in{{\mathbb{C}}}^{n-m}$ and $y\in{{\mathbb{C}}}^m$.
The polynomials defining $V$ generate a prime ideal $I$ in the polynomial ring ${{\mathbb{C}}}[x,y]$.
We assume that $V$ is in general position with respect to these coordinates.
In particular, the projection $\pi$ of $V$ to ${{\mathbb{C}}}^m$ is a branched cover with a fiber of
${{\color{blue}{d}}}=\deg V$ points outside the ramification locus ${{\color{blue}{\Delta}}}\subset{{\mathbb{C}}}^m$.

Let ${{\color{blue}{\ell}}}\subset{{\mathbb{C}}}^m$ be a general line parameterized by $t\in{{\mathbb{C}}}$, so that
${{\color{blue}{L}}}:={{\mathbb{C}}}^{n-m}\times\ell$ is a general affine subspace of dimension $n{-}m{+}1$ with coordinates
$(x,t)$.
The intersection ${{\color{blue}{C}}}:=V\cap L$ is an irreducible curve of degree $d$ by Bertini's
Theorem~(see Theorem~\ref{thm:bertini}) and the projection $\pi\colon C\to\ell$ is a degree $d$ cover over
$\ell\smallsetminus\Delta$.

\begin{proposition}\label{prop:generic-projection-to-P2}
 Let $C\subset {{\mathbb{P}}}^n$, $n\geq 2$, be a curve. Let $\alpha:{{\mathbb{P}}}^n{\dashedrightarrow} {{\mathbb{P}}}^2$ be a generic projection.
 Then $C$ is irreducible if and only if $\alpha(C)$ is irreducible.
\end{proposition}

Since $V$ and $L$ are in general position, Proposition~\ref{prop:generic-projection-to-P2} implies that
the projection of $C$ to the  $(x_i,t)$-coordinate plane
is an irreducible curve given by a single polynomial {{\color{blue}{$f(x_i,t)$}}} of degree $d$ with all
monomials up to degree $d$ having nonzero coefficients.

Normalize $f$ so that its coefficient of $x_i^d$ is $1$, and then
extend scalars from ${{\mathbb{C}}}$ to ${{\mathbb{C}}}(t)$.
Then $f\in {{\mathbb{C}}}(t)[x_i]$ is a monic irreducible polynomial in $x_i$.
The negative sum of its roots is the coefficient of $x^{d-1}$ in $f$, which is an affine function
of $t$.
Equivalently,
 \[
   \operatorname{tr}_{K/{{\mathbb{C}}}(t)}(x_i)\ =\ c_0 t+c_1\,, \quad(\mbox{for some }c_0,c_1\in {{\mathbb{C}}})\,,
 \]
where $K$ is a finite extension of ${{\mathbb{C}}}(t)$ containing the roots of $f$.
A function of $t$ of the form $c_0 t+c_1$ where $c_0,c_1$ are constants is an
{{{\color{blue}{{\sl {affine function}}}}}}.
We deduce the following.

\begin{proposition}\label{Prop:one}
 The sum in ${{\mathbb{C}}}^{n-m}$ of the points in a fiber of $C$ over $t\in\ell\smallsetminus\Delta$ is an affine
 function of $t$.
\end{proposition}

The converse to this holds.

\begin{proposition}\label{Prop:two}
 No proper subset of the points in a fiber of $C$ over $t\in\ell\smallsetminus\Delta$ has sum that is an affine
 function of $t$.
\end{proposition}

\begin{remark}
 We generalize the situation of Proposition~\ref{Prop:two}.
 A {{{\color{blue}{{\sl {pencil}}}}}} of linear spaces is a family {{\color{blue}{$M_t$}}} for $t\in{{\mathbb{C}}}$ of linear spaces that depends
 affinely on the parameter $t$.
 Each $M_t$ is the span of a linear space $L$ and a point $t$ on a line $\ell$ that is disjoint from $L$.

 Suppose that $V\subset{{\mathbb{P}}}^n$ is a subvariety of dimension $m$ and that $M_t$ for $t\in{{\mathbb{C}}}$ is a general pencil of
 linear subspaces of codimension $m$ with $V\cap M_0$ transverse.
 Let $\Delta\subset{{\mathbb{C}}}$ be the finite set of points $t$ such that the intersection $V\cap M_t$ is not transverse.
 Given any path $\gamma\colon[0,1]\to{{\mathbb{C}}}\smallsetminus\Delta$ with $\gamma(0)=0$ and any $v\in V\cap
 M_0$, we may analytically continue $v$ along $\gamma$ to obtain a path $v(\gamma(s))$ for $s\in[0,1]$
 with $v(\gamma(s))\in V\cap M_{\gamma(s)}$.

 Given any subset $W$ of $V\cap M_0$, we will say that the sum of the points in $W$ is
 {{{\color{blue}{{\sl {an affine function of $t$}}}}}} if for a nonconstant path $\gamma\colon[0,1]\to{{\mathbb{C}}}\smallsetminus\Delta$ with
 $\gamma(0)=0$, the sum of the points $w(\gamma(s))$ is an affine function of $\gamma(s)$.
 This is independent of choice of path and of a general pencil.
\end{remark}

\begin{remark}\label{remark:trace-test}
 These propositions lead to the trace test.
 Let $V\subset{{\mathbb{P}}}^n$ (or $\subset{{\mathbb{C}}}^n$) be a possibly reducible variety of dimension $m$ and $M$ a general
 linear space of codimension $m$ so that $W=V\cap M$ is a witness set for $V$.
 Using, for example, monodromy, we have a subset $\emptyset\neq W'\subset W$ whose points lie in a single
 component $V'$ of $V$ and we want to test if $W'=V'\cap M$.
 We will call such a set $W'$ a {{{\color{blue}{{\sl {partial witness set}}}}}} for $V'$.
 To test if $W'=V'\cap M$, consider a general pencil $M_t$ for $t\in{{\mathbb{C}}}$ of codimension $m$ planes in
 ${{\mathbb{P}}}^n$ with $M=M_0$ and we test if the sum of the points of $W'$ is an affine function of $t$.
 By Proposition~\ref{Prop:two}, $W'=V'\cap M$ if and only if it passes this {{{\color{blue}{{\sl {trace test}}}}}}.
\end{remark}

\begin{remark}\label{remark:trace-image}
 Let $U$ be an arbitrary variety and $\phi: U{\dashedrightarrow} {{\mathbb{P}}}^n$ be a rational map with image $V=\overline{\phi(U)}$.
 As obtaining defining equations for $V$ may not be practical, working with a witness set $V\cap M$ may not be feasible.
Instead one may work with the preimage $\phi^{-1}(V\cap M)$ producing a proxy for the witness set $V\cap M$. 
A {{{\color{blue}{{\sl {partial proxy witness set}}}}}} is a finite subset of $\phi^{-1}(V\cap M)$. (It is {{{\color{blue}{{\sl {complete}}}}}} if its image is a complete witness set.)

We can, in particular, employ the trace test for the image working with proxy witness sets for $V\cap M_t$ in Remark~\ref{remark:trace-test}.
\end{remark}

Hauenstein and Sommese use this general observation in \cite{WitnessProj} to provide a detailed description of how proxy witness sets can be computed and used to get witness sets of images of subvarieties under a linear map ${{\mathbb{P}}}^m \to {{\mathbb{P}}}^n$.

\section{Traces in a product of projective spaces}\label{S:product}

Suppose that $V\subset{{\mathbb{C}}}^{n_1}\times{{\mathbb{C}}}^{n_2}$ is an irreducible variety of dimension $m>0$.
Letting $z^{(i)}$ be coordinates for ${{\mathbb{C}}}^{n_i}$ for $i=1,2$, $V$ is defined by polynomials
$F(z^{(1)},z^{(2)})$ which generate a prime ideal.
Separately homogenizing these polynomials in each set $z^{(i)}$ of variables give bihomogeneous polynomials
that define the closure $\overline{V}$ of $V$ in the product ${{\mathbb{P}}}^{n_1}\times{{\mathbb{P}}}^{n_2}$ of projective spaces.
Let us also write $V$ for this closure.

Then $V$ has a {{{\color{blue}{{\sl {multidegree}}}}}}~\cite[Ch.~19]{Harris}.
This is a set of nonnegative integers $d_{m_1,m_2}$ where $m_1+m_2=m$ with $0\leq m_i\leq n_i$ for
$i=1,2$ that has the following geometric meaning.
Given general linear subspaces $M_i\subset{{\mathbb{P}}}^{n_i}$ of codimension $m_i$ for $i=1,2$ with
$m_1+m_2=m$, the number of points in the intersection $V\cap (M_1\times M_2)$ is $d_{m_1,m_2}$.
These multidegrees are log-concave in that for every $1\leq m_1\leq m{-}1$, we have
 \begin{equation}\label{Eq:log-concave}
   d_{m_1,m_2}^2\ \geq\ d_{m_1-1,m_2+1}\cdot d_{m_1+1,m_2-1}\,.
 \end{equation}
These inequalities of Khovanskii and Tessier are explained in~\cite[Ex.~1.6.4]{Laz}.

Following~\cite{HR15}, a {{{\color{blue}{{\sl {multihomogeneous witness set}}}}}} of dimension $(m_1,m_2)$ with $m_1+m_2=\dim V$
for an irreducible variety $V$ is the set ${{\color{blue}{W_{m_1,m_2}}}}:=V\cap(M_1\times M_2)$, where for $i=1,2$,
$M_i\subset{{\mathbb{P}}}^{n_i}$ is a general linear subspace of codimension $m_i$.
More formally, the witness set is a triple consisting of the points $W_{m_1,m_2}$,
equations for a variety that has $V$ as a component, and equations for  $M_1$ and for $M_2$.
A {{{\color{blue}{{\sl {witness collection}}}}}} is the list of witness sets $W_{m_1,m_2}$ for all $m_1+m_2=m$.

\begin{remark}\label{R:WitnessProj}
 If $\pi_2(V)$ has dimension $m_2$, then the multihomogeneous witness set $V\cap(M_1\times M_2)$ is a (proxy) witness
 set for the image $\pi_2(V)$ of $V$ in the sense of~\cite{WitnessProj} and Remark~\ref{remark:trace-image}.
\end{remark}

Suppose that $V$ is reducible and we have a multihomogeneous witness set $W_{m_1,m_2}=V\cap(M_1\times M_2)$ for
$V$.
This is a disjoint union of multihomogeneous witness sets for the irreducible components of $V$ that have non zero
$(m_1,m_2)$-mulltidegree.
We may similarly have a witness collection for $V$.
We consider the problem of decomposing a witness collection into witness collections for the components
of $V$.
For every irreducible component $V'$ of $V$ it is possible to obtain a {{{\color{blue}{{\sl {partial witness collection}}}}}}
$W'_{m_1,m_2}$ for $m_1+m_2=m$ and then---much like in the affine/projective setting---use the monodromy
action and the membership test in order to build up a (complete) witness collection.
We seek a practical trace test to verify that a partial witness collection is, in fact, complete.
That is, if we have equality  $W'_{m_1,m_2}=V'\cap(M_1\times M_2)$ for each partial witness set $W'_{m_1,m_2}$
for $V'$.

By Example 20 of~\cite{HR15}, the trace of a multihomogeneous witness set as the linear subspaces $M_1$ and
$M_2$ vary in pencils is not multilinear.
The trace test for subvarieties of products of projective spaces described in~\cite{HR15} uses
{{{\color{blue}{{\sl {Segre embedding}}}}}}
$\sigma\colon{{\mathbb{P}}}^{n_1}\times{{\mathbb{P}}}^{n_2} \to {{\mathbb{P}}}^{(n_1+1)(n_2+1)-1}$ to construct the proxy witness sets as in Remark~\ref{remark:trace-image} (with $\phi = \sigma$). Since $\sigma$ is an isomorphism, proxy witness sets are simply preimages of witness sets (in constrast, for instance, with~\cite{ProjW} where extra work is needed, since the preimage of a witness point may not be 0-dimensional).

\begin{remark}\label{remark:Segre}
 Multihomogeneous witness sets are typically significantly smaller than Segre witness sets.
 If $V\subset {{\mathbb{P}}}^{n_1}\times{{\mathbb{P}}}^{n_2}$ is a subvariety with multidegrees $d_{m_1,m_2}$, then the degree of its
 image under the Segre embedding is
\[
   \deg(V)\ =\ \sum_{m_1+m_2=m} d_{m_1,m_2}\frac{m!}{m_1! m_2!}\,,
\]
 which is Exercise 19.2 in~\cite{Harris} and follows from the cohomological interpretation of multidegree given
 there.

 This is significantly (even exponentially) larger that the union of the multihomogeneous witness sets for $V$.
 Thus a witness set for the image of $V$ under the Segre embedding (a {{{\color{blue}{{\sl {Segre witness set}}}}}} in~\cite{HR15})
 involves significantly more points than any of its multihomogeneous witness sets.
 \end{remark}

\begin{example}
  For instance, the graph $V\subset{{\mathbb{P}}}^m\times{{\mathbb{P}}}^m$ of a general linear map has multidegrees
  $(1,\dotsc,1)$ with sum $m{+}1$, but its image under the Segre map has degree $2^m$.
  Similarly, if $V$ is the closure of the graph of the the standard Cremona transformation
  $[x_0,\dotsc,x_m]\mapsto[1/x_0,\dotsc,1/x_m]$, then its multidegrees are $d_{i,m-i}=\binom{m}{i}$
  with sum $2^m$   and its degree under the Segre embedding is $\binom{2m}{m}=\sum \binom{n}{i}^2$, which is
  considerably larger.
\end{example}

  These examples suggest that one should try to develop algorithms that work directly with multihomogeneous
  witness sets $W_{m_1,m_2}$ for $m_1+m_2=m$ and---as the graph of Cremona suggests---also involve as few of
  these as possible.

  Algorithm~\ref{algorithm:multi-trace-test} does exactly that while avoiding Segre embedding.

\section{Dimension reduction}\label{S:dimension}

While stronger versions are available, the following variant of Bertini's theorem is particularly useful in
numerical algebraic geometry.
It follows, for instance, from~\cite[Thm.~6.3~(4)]{jouanolou}.

\begin{theorem}[Bertini's Theorem]\label{thm:bertini}
Let $V$ be a variety and $\phi:V{\dashedrightarrow} {{\mathbb{P}}}^n$ be a rational map such that $\dim\phi(V) \geq 2$.

Then $V$ is irreducible if and only if $V\cap \phi^{-1}(H)$ for a generic hypersurface $H\subset{{\mathbb{P}}}^n$.
\end{theorem}

In the beginning of \S\ref{S:affine} we used Bertini's theorem to slice a projective variety $V\subset{{\mathbb{P}}}^n$, $\dim V \geq 2$, with a general linear subspace.
This reduced the dimensions of the variety and the ambient space, but did not alter its degree or irreducible decomposition.

A similar dimension reduction procedure is slightly more involved for subvarieties of a product of projective spaces.

\begin{proposition}\label{prop:reduction-to-curve}
 Let $V \subset {{\mathbb{P}}}^{n_1}\times{{\mathbb{P}}}^{n_2}$ be an irreducible variety and $d_{m_1,m_2}(V)\neq 0$ a nonzero multidegree
 with $1\leq m_1,m_2$.
 For $i=1,2$, let $M'_i$ be a general linear subspace of\/ ${{\mathbb{P}}}^{n_i}$ of codimension $m_i{-}1$.
 Then ${{\color{blue}{V'}}}:=V\cap(M'_1\times M'_2)$ is irreducible, has dimension $2$, and multidegrees
\[
  d_{0,2}(V')\ =\ d_{m_1-1,m_2+1}\,,\  \
  d_{1,1}(V')\ =\ d_{m_1,m_2}\,,\  \quad\mbox{\rm and}\quad
  d_{2,0}(V')\ =\ d_{m_1+1,m_2-1}\,.
\]
 We have several overlapping cases.
\begin{enumerate}
\item[(1)] If $d_{0,2}(V')=d_{2,0}(V')=0$, then $\pi_1(V')$ and $\pi_2(V')$ are both curves, $V'$ is their product,
        and $V$ is the product of its projections $\pi_1(V)\subset{{\mathbb{P}}}^{n_1}$ and  $\pi_2(V)\subset{{\mathbb{P}}}^{n_2}$.
\item[(2a)] If $d_{0,2}(V')=0$ then $\pi_1(V')$ is an irreducible curve and $V'$ is fibered over $\pi_1(V')$ by curves.
      Also, $\pi_1(V)$ is irreducible of dimension $m_1$ and the map $V\to\pi_1(V)$ is a fiber bundle.
      If $d_{2,0}(V')=0$, then the same holds {\it mutatis mutandis}.
\item[(2b)] One of  $d_{2,0}(V')$ or $d_{0,2}(V')$ is non-zero.
      Suppose that $d_{2,0}(V')\neq 0$.
      Then $\pi_1(V')$ is two-dimensional, and for a general hyperplane $H\subset{{\mathbb{P}}}^{n_1}$,
      $W \cap (H \times {{\mathbb{P}}}^{n_2})$ is an irreducible curve $C$ with $d_{1,0}(C)=d_{2,0}(V')$ and
      $d_{0,1}(C)=d_{1,1}(V')$.
\end{enumerate}
\end{proposition}

Case (1) can be distinguished from cases (2a) and (2b) as follows. It suffices to look at the linear maps induced by projections $\pi_i$, $i=1,2$,
on the tangent space of $V'$ at any general point. We are in case (1) if and only if both of these are degenerate.

Note that case (1) reduces to the analysis of projections $\pi_i(V')$,
otherwise it is possible to use Bertini's theorem to slice one more time (preserving irreducibility and the multidegrees)
in order to reduce a two-dimensional subvariety $V'$ to a curve $C$.

The following proposition is not a part of our multihomogeneous trace test. We will use it in the upcoming arguments only to provide better intuition to the reader.
It is the multihomogeneous counterpart of the reduction to a planar curve stated in Proposition~\ref{prop:generic-projection-to-P2}.

\begin{proposition}\label{prop:generic-projection-to-P1xP1}
Let $C\subset {{\mathbb{P}}}^{n_1}\times{{\mathbb{P}}}^{n_2}$ be a curve.
Let $\alpha_i\colon{{\mathbb{P}}}^{n_i}{\dashedrightarrow} {{\mathbb{P}}}^1$ be a generic linear projection for $i=1,2$.
Then $C$ is irreducible if and only if $(\alpha_1\times\alpha_2)(C) \subset {{\mathbb{P}}}^1\times{{\mathbb{P}}}^1$ is irreducible.
\end{proposition}

Now that we have reduced to the case of a curve $C \subset {{\mathbb{P}}}^1\times{{\mathbb{P}}}^1$, we could
use a trace test via Segre embedding ${{\mathbb{P}}}^1\times{{\mathbb{P}}}^1\to {{\mathbb{P}}}^3$ as discussed in Remark~\ref{remark:Segre}.
However, there is a more direct way that uses the trace test in an affine space of lower dimension.

\begin{example}\label{Ex:P1xP1}
Let us consider the trace test for a curve $C$ in ${{\mathbb{P}}}^1\times{{\mathbb{P}}}^1$.
Let $x_0,x_1$ and $y_0,y_1$ be homogeneous coordinates on two copies of ${{\mathbb{P}}}^1$,
the curve $C$ is given by a bihomogeneous polynomial $f=f(x,y)$ of degree $d$.
Note that $d_{1,0}(C)+d_{0,1}(C) = d$.

Let witness sets  $W_{1,0}$ and $W_{0,1}$ be cut out by linear forms $\ell^{(1)}=\ell^{(1)}(x)$ and $\ell^{(2)}=\ell^{(2)}(y)$.
Pick generic affine charts given by $h^{(1)}=1$ and $h^{(2)}=1$ where $h^{(1)}=h^{(1)}(x)$ and $h^{(2)}=h^{(2)}(y)$ are linear forms.
Take generic  linear forms $\ell'^{(1)}=\ell'^{(1)}(x)$ and $\ell'^{(2)}=\ell'^{(2)}(y)$ and consider the multihomogeneous quadric $g = h^{(2)}\ell'^{(1)} + h^{(1)}\ell'^{(2)} + h^{(1)}h^{(2)}$.

One can follow the homotopy $$(1-t)\ell^{(1)}\ell^{(2)} + tg,\quad t\in[0,1]$$ to obtain $d$ points of $C\cap{{\mathbb{V} \! \left({g}\right)}}$ from $W_{1,0}$ and $W_{0,1}$.
In the specified charts the dehomogenization of $g$ defines a generic hyperplane in ${{\mathbb{C}}}^1\times{{\mathbb{C}}}^1 \simeq {{\mathbb{C}}}^2$, therefore, providing a witness set for the curve in ${{\mathbb{C}}}^2$.

This gives a way to use the trace test in ${{\mathbb{C}}}^2$ to verify completeness of a pair of partial witness set in
bidimensions $(1,0)$ and $(0,1)$.
Note that it does not use the Segre embedding $\sigma({{\mathbb{P}}}^1\times{{\mathbb{P}}}^1)\subset{{\mathbb{P}}}^3$, but works directly on
the affine patch ${{\mathbb{C}}}^2$ of ${{\mathbb{P}}}^1\times{{\mathbb{P}}}^1$.
\end{example}

\begin{remark}\label{remark:trace-test-for-curve}
It is not essential to reduce to a curve in ${{\mathbb{P}}}^1\times{{\mathbb{P}}}^1$.
The construction and argument of Example~\ref{Ex:P1xP1} go through for an irreducible curve ${{\mathbb{P}}}^{n_1}\times{{\mathbb{P}}}^{n_2}$
with the trace test performed in an affine patch ${{\mathbb{C}}}^{n_1+n_2} \simeq {{\mathbb{C}}}^{n_1}\times{{\mathbb{C}}}^{n_2}$.
\end{remark}
To summarize, we give a high-level description of an algorithm for the trace test for a {{{\color{blue}{{\sl {collection of partial
multihomogeneous witness sets}}}}}}.
The details of a numerical irreducible decomposition algorithm that uses this trace test shall be given elsewhere.

Let us fix\vspace{-3pt}
\begin{description}
\item[dimension] an integer $m$, the dimension of a witnessed component;
\item[affine charts] for $i=1,2$, linear forms $h^{(i)}$ defining affine charts $h^{(i)}=1$ in ${{\mathbb{P}}}^{n_i}$;
\item[slices] for $i=1,2$, for $j=1,\ldots,m$, linear forms $\ell^{(i)}_j$ defining hyperplanes in ${{\mathbb{P}}}^{n_i}$;
\end{description}
We denote by $L_{m_1,m_2}$ the linear system $\{h^{(1)}-1,\ell^{(1)}_1,\ldots,\ell^{(1)}_{m_1},h^{(2)}-1,\ell^{(2)}_1,\ldots,\ell^{(2)}_{m_2}\}$.
This is mildly technical; note that the system $L_{m_1,m_2}$ defines a product $M_1\times M_2$ in an affine
chart of ${{\mathbb{P}}}^{n_1}\times{{\mathbb{P}}}^{n_2}$.

\begin{algorithm}[Multihomogeneous Trace Test]  \label{algorithm:multi-trace-test}
\label{algorithm:monodromy-breakup}
\ \newline
\noindent{\sc Input:} \vspace{-3pt} \begin{description}
  \item[equations] a multihomogeneous polynomial system $F$;
  \item[a partial witness collection]  partial witness sets $W_{m_1,m_2}$ where $m_1=0,\ldots,m$ and $m_2=m-m_1$
representing an irreducible component $V\subset{{\mathbb{V} \! \left({F}\right)}}$, i.e., $W_{m_1,m_2} \subset V\cap{{\mathbb{V} \! \left({L_{m_1,m_2}}\right)}}$.
\end{description}
\noindent{\sc Output:} a boolean value = $\text{the witness collection is complete}$.
\begin{algorithmic}[1]
\IF{$W_{m_1,m_2}=\emptyset$ for all $m_1=0,\ldots,m$ but one}
  \IF{both projections $\pi_i$ to the factors ${{\mathbb{P}}}^{i}$ are locally degenerate at an available witness point}
    \RETURN (both trace tests for the projections to ${{\mathbb{P}}}^{n_i}$ for $i=1,2$ pass) \AND (the unique nonempty set of witness points equals the product of its projections to the factors).
  \ELSE \RETURN \FALSE
  \ENDIF
\ELSE
  \FOR{$m_1 = 0,\ldots, m-1$}
  \IF{the trace test in ${{\mathbb{C}}}^{n_1+n_2}$ described in Example~\ref{Ex:P1xP1} and
    Remark~\ref{remark:trace-test-for-curve} (after tracking $W_{m_1,m_2}$ and $W_{m_1+1,m_2-1}$ along the
    deformation from $\ell^{(1)}_{m_1+1}\ell^{(2)}_{m_2}$ to a general affine linear function on
    ${{\mathbb{C}}}^{n_1+n_2}$) does not pass} \RETURN \FALSE \ENDIF
  \ENDFOR
  \RETURN \TRUE
\ENDIF
\end{algorithmic}
\end{algorithm}

We presented all results in this section for subvarieties of a product of {\em two} projective spaces for the sake of clarity. All arguments generalize to a product of arbitrarily many factors.

\section{Proofs}\label{S:proofs}
The proofs appear not in order, since the proof of Proposition~\ref{prop:generic-projection-to-P2} is standard and is provided for better understanding of the proof of Proposition~\ref{prop:generic-projection-to-P1xP1}.

Every surjective linear map ${{\mathbb{P}}}^n{\dashedrightarrow} {{\mathbb{P}}}^{n-1}$, $n>1$, is the projection from a point
$p\in{{\mathbb{P}}}^n\simeq \operatorname{Proj}({{\mathbb{C}}}^{n+1})$.
Namely, it is the projectivization $\alpha_p: {{\mathbb{P}}}^n{\dashedrightarrow} {{\mathbb{P}}}^{n-1}$ of the quotient map
${{\mathbb{C}}}^{n+1}\to{{\mathbb{C}}}^{n+1}/{{\mathbb{C}}} p \simeq{{\mathbb{C}}}^n$.
Note that this is a rational map not defined at $p$.

\begin{proof}[Proof of Proposition~\ref{prop:generic-projection-to-P2}]
Consider the incidence variety of triples
$(p,c,c')\subset {{\mathbb{P}}}^n \times C\times C$, where
$p, c, c'$ are collinear.   Projecting to
$C\times C$ shows that this is three dimensional,
and the projection to ${{\mathbb{P}}}^n$ is dense in the secant variety of $C$.

When $n=3$, observe that the map to ${{\mathbb{P}}}^3$ is either
(1) not dense, so projecting from a point not in
     its closure sends $C$ birationally onto a plane
     curve, or
(2) dense.  In this case, a general point in the
     image has finitely many preimages, so projecting
     from such a point gives a plane curve $C'$ that
     is birational to $C$, but with finitely many
     points of self-intersection.

For $n>3$ we are always in case (1).

Thus, we are always able to reduce the ambient dimension by one until $n=2$.
\end{proof}

\begin{proof}[Proof of Proposition~\ref{prop:generic-projection-to-P1xP1}]
W.l.o.g., assume that $n=n_1\geq n_2$. Note that using any inclusion ${{\mathbb{P}}}^{n_2}\hookrightarrow{{\mathbb{P}}}^{n_1}$ we can map $C$ birationally to a curve in ${{\mathbb{P}}}^n\times{{\mathbb{P}}}^n$ and replace $\alpha_2$ with a generic linear map ${{\mathbb{P}}}^n{\dashedrightarrow}{{\mathbb{P}}}^1$ it factors through.

For $(p,q) \in {{\mathbb{P}}}^n\times{{\mathbb{P}}}^n$ consider the product of projection-from-a-point maps
$$\alpha_p\times \alpha_q:{{\mathbb{P}}}^n\times{{\mathbb{P}}}^n {\dashedrightarrow} {{\mathbb{P}}}^{n-1}\times{{\mathbb{P}}}^{n-1}.$$

Consider the incidence variety of triples
$(s,c,c')\in ({{\mathbb{P}}}^n\times{{\mathbb{P}}}^n) \times C\times C$,
where $s=(s_1,s_2)$, $c=(c_1,c_2)$, and $c'=(c'_1,c'_2)$
such that $s_i,c_i,c'_i$ are colinear in ${{\mathbb{P}}}^{n_i}$ for each $i=1,2$.

Projecting to $C\times C$, this has fibers isomorphic to ${{\mathbb{P}}}^1\times {{\mathbb{P}}}^1$, therefore, it is four-dimensional.
The projection to ${{\mathbb{P}}}^n\times{{\mathbb{P}}}^n$ is dense in a generalized secant variety of dimension four.

When $n=2$, observe that the projection to ${{\mathbb{P}}}^2\times{{\mathbb{P}}}^2$ is either
(1) not dense, so ${\alpha_p\times \alpha_q}$ for a point $(p,q)$ not in
     its closure sends $C$ birationally onto its image, or
(2) dense.  In this case, a general point  $(p,q) \in{{\mathbb{P}}}^n\times{{\mathbb{P}}}^n$
     has finitely many preimages. This implies tha the map
${\alpha_p\times \alpha_q}$ is one-to-one on $C$ with the exception of finitely many points, whose images are self-intersections of the curve $(\alpha_p\times \alpha_q)(C)$.

For $n>2$ we are always in case (1).

Thus, we are always able to reduce $n$ by one until $n=1$.
\end{proof}

\begin{proof}[Proof of Proposition~\ref{Prop:two}]
 Let $W$ be a subset of the fiber $C_t$ of $C$ over
 $t\in\ell\smallsetminus\Delta$ whose sum is an affine function of $t$.
 Given a path $\gamma\colon[0,1]\to\ell\smallsetminus\Delta$ with $\gamma(0)=t$, we may analytically
 continue any point $w\in C_t$ along $\gamma$ to get a path $w(\gamma(s))$ for
 $s\in[0,1]$.
 By assumption, the sum of the points $w(\gamma(s))$ for $w\in W$ is an affine function of $\gamma(s)$.

 Following points of $C_t$ along a loop $\gamma$ in $\ell\smallsetminus\Delta$ based at $t$
 ($t=\gamma(0)=\gamma(1)$) gives a permutation of $C_t$.
 By our assumption of general position
 and~\cite[Lemma on page 111]{Arbarello-Cornalba-Griffiths-Harris},
 every permutation of $C_t$ is obtained by some loop $\gamma$.

 Suppose that $W$ is a proper subset of $C_t$.
 Then there is a point $u\in W$ and a point $v\in C_t\smallsetminus W$, hence $u\neq v$.
 Let $\gamma$ be a loop in $\ell\smallsetminus\Delta$ based at $t$ whose permutation interchanges $u$ and
 $v$ and fixes the other points of $C_t$.
 In particular, $u(\gamma(1))=v$.
 Since $\gamma(0)=\gamma(1)$, our assumption on $W$ implies that
\[
   \sum_{w\in W} w(\gamma(0))\ =\ \sum_{w\in W} w(\gamma(1))\,.
\]
Taking the difference gives $0=u(\gamma(1))-u(\gamma(0))$ so that $u=v$, a contradiction.
\end{proof}

\begin{proof}[Proof of Proposition~\ref{prop:reduction-to-curve}]
Note that projections $\pi_i:{{\mathbb{P}}}^{n_1}\times {{\mathbb{P}}}^{n_2}\to{{\mathbb{P}}}^{n_i}$, for $i=1,2$, satisfy the assumptions on the map $\phi$ in Theorem~\ref{thm:bertini}.
Applying the Theorem $m_i-1$ times for $\pi_i$, for $i=1,2$, gives the proof of the first part of the conclusion.

The rest of the conclusion follows from the case analysis: in the case $\dim\pi_1(V')\geq 2$, one more application of Theorem~\ref{thm:bertini} for the map $\pi_1$ proves the statement.
\end{proof}

\bibliographystyle{amsplain}
\providecommand{\bysame}{\leavevmode\hbox to3em{\hrulefill}\thinspace}
\providecommand{\MR}{\relax\ifhmode\unskip\space\fi MR }
\providecommand{\MRhref}[2]{  \href{http://www.ams.org/mathscinet-getitem?mr=#1}{#2}
}
\providecommand{\href}[2]{#2}
\begin{thebibliography}{10}

\bibitem{Arbarello-Cornalba-Griffiths-Harris}
E.~Arbarello, M.~Cornalba, P.A. Griffiths, and J.~Harris, \emph{Geometry of
  algebraic curves. {V}ol. {I}}, Grundlehren der Mathematischen Wissenschaften,
  vol. 267, Springer-Verlag, New York, 1985.

\bibitem{Harris}
Joe Harris, \emph{Algebraic geometry}, Graduate Texts in Mathematics, vol. 133,
  Springer-Verlag, New York, 1992.

\bibitem{HR15}
J.D. Hauenstein and J.I. Rodriguez, \emph{Numerical irreducible decomposition
  of multiprojective varieties}, 2015, {\tt arXiv:1507.07069}.

\bibitem{WitnessProj}
J.D. Hauenstein and A.J. Sommese, \emph{Witness sets of projections}, Appl.
  Math. Comput. \textbf{217} (2010), no.~7, 3349--3354.

\bibitem{jouanolou}
Jean-Pierre Jouanolou, \emph{Th\'eor\`emes de {B}ertini et applications},
  Progress in Mathematics, vol.~42, Birkh\"auser Boston, Inc., Boston, MA,
  1983.

\bibitem{Laz}
Robert Lazarsfeld, \emph{Positivity in algebraic geometry. {I}}, Ergebnisse der
  Mathematik und ihrer Grenzgebiete. 3. Folge. A Series of Modern Surveys in
  Mathematics [Results in Mathematics and Related Areas. 3rd Series. A Series
  of Modern Surveys in Mathematics], vol.~48, Springer-Verlag, Berlin, 2004,
  Classical setting: line bundles and linear series.

\bibitem{SV}
A.J. Sommese and J.~Verschelde, \emph{Numerical homotopies to compute generic
  points on positive dimensional algebraic sets}, J. Complexity \textbf{16}
  (2000), no.~3, 572--602.

\bibitem{SVW_decomposition}
A.J. Sommese, J.~Verschelde, and C.W. Wampler, \emph{Numerical decomposition of
  the solution sets of polynomial systems into irreducible components}, SIAM
  J.\ Numer.\ Anal. \textbf{38} (2001), no.~6, 2022--2046.

\bibitem{SVW_monodromy}
\bysame, \emph{Using monodromy to decompose solution sets of polynomial systems
  into irreducible components}, Applications of algebraic geometry to coding
  theory, physics and computation ({E}ilat, 2001), NATO Sci. Ser. II Math.
  Phys. Chem., vol.~36, Kluwer Acad. Publ., Dordrecht, 2001, pp.~297--315.

\bibitem{SVW_trace}
\bysame, \emph{Symmetric functions applied to decomposing solution sets of
  polynomial systems}, SIAM J. Numer. Anal. \textbf{40} (2002), no.~6,
  2026--2046 (2003).

\bibitem{SW05}
A.J. Sommese and C.W. Wampler, \emph{The numerical solution of systems of
  polynomials}, World Scientific Publishing Co. Pte. Ltd., Hackensack, NJ,
  2005.

\end{thebibliography}

\end{document}

