

\documentclass[12pt]{amsart}

\usepackage{
amsfonts,
latexsym,
amssymb,
amsbsy,
enumerate,
mathrsfs,
nicefrac}

 
\newtheorem{theorem}{Theorem}
\newtheorem{lemma}[theorem]{Lemma}
\newtheorem{thm}[theorem]{Theorem}
\newtheorem{prop}[theorem]{Proposition} 
\newtheorem{proposition}[theorem]{Proposition} 
\newtheorem{cor}[theorem]{Corollary} 
\newtheorem{corollary}[theorem]{Corollary} 
\newtheorem{fact}[theorem]{Fact} 
\newtheorem*{claim}{Claim} 
\newtheorem*{claim1}{Claim 1} 
\newtheorem*{claim2}{Claim 2} 
\newtheorem*{theorem*}{Theorem}
\newtheorem*{corollary*}{Corollary}
\newtheorem*{proposition*}{Proposition}

\theoremstyle{definition}
\newtheorem{definition}[theorem]{Definition}
\newtheorem{definitions}[theorem]{Definitions}
\newtheorem{df}[theorem]{Definition}
\newtheorem{xca}[theorem]{Exercise}
\newtheorem*{definition*}{Definition}

\newtheorem{conj}[theorem]{Conjecture} 
\newtheorem{prob}[theorem]{Problem} 
\newtheorem{problem}[theorem]{Problem} 
\newtheorem{problems}[theorem]{Problems} 

\theoremstyle{remark}
\newtheorem{remark}[theorem]{Remark}
\newtheorem{rem}[theorem]{Remark} 
\newtheorem{example}[theorem]{Example}
\newtheorem{examples}[theorem]{Examples}
\newtheorem{notations}[theorem]{Notations} 
\newtheorem{notation}[theorem]{Notation} 
\newtheorem{convention}[theorem]{Convention}
\newtheorem{construction}[theorem]{Construction} 
\newtheorem*{disclaimer*}{Disclaimer}

\begin{document}
\title[A surreal limit]{A surreal limit \\ (surrealist landscape with figures)} 

\author{Paolo Lipparini} 
\address{Dipartimento di Matematica\\Viale della Ricerca Scientifica\\
Universit\`a di Roma ``Tor Virgolettata'' \\I-00133 ROME ITALY}
\urladdr{http://www.mat.uniroma2.it/\textasciitilde lipparin}

\keywords{Surreal number, real number, limit, sign expansion, string convergence} 

\subjclass[2010]{03H05; 40A05, 12J15}

\thanks{Work performed under the auspices of G.N.S.A.G.A. Work partially supported by PRIN 2012 ``Logica, Modelli e Insiemi''.}

\begin{abstract}
We note that if a sequence of real 
numbers converges to some limit, then 
the sequence of the corresponding strings  in the   surreal 
$+,-$ sign expansion 
representation converges, for  a natural notion
of string convergence,
to the string corresponding to the  limit,
modulo an infinitesimal.
This would be obviously false if we
were considering, as strings,  decimal or binary representations, 
instead. The string limit 
of a possibly transfinite
sequence of surreal numbers is always defined and, restricted 
to increasing sequences of ordinals, corresponds to 
taking the supremum.
A transfinite sum can be defined 
using the string limit   
and this sum agrees with the
representation of a surreal in 
Conway normal form.
\end{abstract} 

\maketitle  

\bigskip 

If we compute, say, the number $e$ by means of the usual series expansion,
we get the following sequence of partial sums:
\[
1, \ \, 2, \ \, 2.5, \ \, 2.666\dots, \ \, 2.7083\dots, \ \,  2.7166\dots,  \ \,  2.7180\dots, \ \,  2.7182\dots
 \] 
 
Since the digits in the above decimal expressions eventually stabilize, we get 
confident that $2.718\dots$  approximates the actual decimal expansion of $e$
(of course, this is not a proof!).
Though a similar argument goes well for many limits,
it is not always the case that digits eventually stabilize
 in converging sequences. For example, consider the sequence
\[
1.99, \ \ 2.01, \ \ 1.999, \ \  2.001, \ \ 1.9999, \ \  2.0001, \ \  \dots, 
 \] 
 which obviously converges to $2$. In this case, no digit at all stabilizes.

The aim of this note is to observe that, instead, if 
we represent real numbers 
  as sequences
of $+$'s and $-$'s by means of  the surreal 
 sign expansion, and we consider a converging (in the classical sense
of real analysis) sequence 
of real numbers, then the $+$'s and $-$'s which stabilize
give the surreal sign expansion of the limit, 
possibly with the difference of  an infinitesimal.
This shows that the apparently odd and strange surreal 
way of representing numbers, in particular, reals, by
 means of a sequence of $+$'s and  $-$'s is, in a sense, actually 
a more natural way,  in comparison with  the usual decimal or binary expansions.

So far, the above result is just slightly more 
than a mere curiosity,
but a few arguments presented below suggest at least some
remote eventuality of further  developments.
See, e.~g., Gonshor \cite{G} or Siegel \cite{S} for details and history
about surreal numbers, in particular, for details about 
the sign expansion.

Notice that, again in the surreal sense, the sign expansion of  an ordinal
is a sequence of $+$'s. If we have an increasing sequence of ordinals,
then every place in the sequence is stable from some point on, hence
a construction similar to the above one furnishes the limit, that is, the supremum,
of the sequence of ordinals. 

The above ideas can be carried over in general,
dealing with sequences of 
transfinite strings of symbols, or, even more generally,
sequences of labeled linearly ordered sets. In this sense, 
we shall introduce  a rather general notion
in Definition \ref{def} below. However, for a while we shall be content
with the particular case of countable sequences of
surreal numbers. 
For notational convenience, 
we shall identify a surreal number with its 
sign expansion, that is, its representation as 
an ordinal-indexed  sequence of $+$'s and  $-$'s.
Thus a surreal number is a function $s: \alpha_s \to \{ +, - \}  $,
where $\alpha_s$ is an ordinal depending on $s$.
If $\beta< \alpha $,  $s( \beta )$ will be sometimes called
\emph{place $\beta$ (in the representation) of $s$}, and places
$ \geq \alpha_s$ will be dubbed \emph{undefined}.   

\begin{definition} {\label}{def1}
If $( s_n ) _{ n < \omega } $  is a sequence of surreal numbers,
we define the \emph{s-limit
  of 
$( s_n ) _{ n < \omega } $},
in symbols, $\operatorname{slim}_{n < \omega} s_n  $
as the surreal $s$ such that place $\gamma$ 
 in the sign expansion of $s$ is defined if and only if there is $m< \omega $
such that, for every $n \geq m$, the sign expansions of
the $s_n$'s are identical up to place $\gamma$ included.      
If this is the case, place  $\gamma$ 
of $s$ is set to be equal to the corresponding place of 
$s_m$ (hence also of the  $s_n$'s which follow).
Notice that, by construction,
if $s(\gamma)$ is defined, then
$s(\gamma')$ is defined, too, for every $\gamma' < \gamma $,
hence the definition is well posed. 
 \end{definition}   

The s-limit of a sequence  is always defined;
possibly, it is the empty sequence. 
Notice also that, as far as we meet a place
at which the values of the $s_n$'s eventually oscillate,
we impose that the sign expansion of the s-limit 
 stops at that point. This is necessary 
if we want Theorem \ref{thm} below to become true. 
A  reason in support of this choice
will be presented in Remark \ref{canrepr} below,
suggesting that the definition might be interpreted 
in a way consistent with the ``surreal philosophy''. 

Since a real number is
(or can be considered as) a surreal number, 
we have also the notion of an s-limit of a sequence of reals.
The main observation of the present note
is the curious fact that 
if a sequence of real numbers has a limit in the sense of classical 
real analysis, then the limit and the s-limit coincide,
modulo an infinitesimal.

In the proof of the following theorem 
we shall freely use the Berlekamp's Sign-Expansion Rule
and the characterization of surreals born at day $ \omega$. See, e.~g., 
\cite[VIII, 2]{S} for full details. Recall that, under 
the sign expansion representation, the
surreal order is obtained by comparing
the first difference. This is quite similar to 
the lexicographic order, but formally different; 
in the surreal sense undefined is considered to be between 
$-$ and  $+$, rather than before them.  
Notice that the following theorem 
deals only with surreals born at most at day $ \omega$.
We shall always consider \emph{addition} $+$ in the surreal sense 
(symbols overlapping never causes confusion).
Of course, when restricted to real numbers, surreal addition
coincides with usual addition.

\begin{theorem} {\label}{thm}
If $( r_n) _{n < \omega } $ 
is a sequence of real numbers
and 
$\lim_{n\to\infty}  r_n $
exists, possibly equal to $ \infty= \omega$
 or $ -\infty = -\omega$,
then
$\lim_{n\to\infty}  r_n  = \operatorname{slim}_{n < \omega } r_n + \varepsilon $,
where $\varepsilon$ is either
$0$, or 
$ 1/ \omega $, or
$-1/ \omega $.
  \end{theorem}

 \begin{proof}
Let $r=\lim_{n\to\infty}  r_n$.

First, suppose that  $r=0$.
The conclusion is trivial
if   infinitely many 
$r_n$'s are equal to $0$,
because of the definition of the s-limit. 
If all but finitely many 
$r_n$'s are
strictly positive, 
then the first sign in the surreal representation
is eventually $+$.
Since the sequence converges in the classical sense,
then, for every $m < \omega$,
we have $r_n \leq 1/2^m$  eventually, that is, 
the first $+$ sign is eventually followed by at least $m$ minuses.   
This means that  $\operatorname{slim}_{n < \omega } r_n = +---\dots = 1/ \omega $.
If all but finitely many 
$r_n$'s are
strictly negative, 
the symmetric argument gives 
$\operatorname{slim}_{n < \omega } r_n = -1/ \omega $. 
In the remaining case we have
infinitely many positive $r_n$'s, hence 
infinitely many occurrences of $+$ at the first place, and
 infinitely many negative $r_n$'s, hence 
infinitely many occurrences of $-$ at the first place.
Thus the first place does not eventually stabilize and,
according to
Definition \ref{def1},
$\operatorname{slim}_{n < \omega } r_n =0$,
being the empty sequence.
 
The case when $r$ is a dyadic rational is similar.
A number $r'$  greater than $r$ and sufficiently close to $r$ 
has the form $r+---\dots$, where juxtaposition denotes string concatenation
and, as far as $r'$ approaches $r$, after the string $r+$  
we get a larger and larger number
of minuses,   possibly followed by a plus and other 
signs.
Thus if $( r_n) _{n < \omega } $ converges to 
$r$ from above,  
$\operatorname{slim}_{n < \omega } r_n = r+---\dots = r+  \nicefrac{1}{ \omega}  $.
Symmetrically, 
if $( r_n) _{n < \omega } $ converges to 
$r$ from below,  
$\operatorname{slim}_{n < \omega } r_n  = r-  \nicefrac{1}{ \omega} $.
In the remaining cases, either  $r_n=r$ for infinitely many $n$'s,
or there are infinitely many  $n$'s such that  
$r_n<r$ and there are infinitely many  $n$'s such that  
$r_n>r$. In each of the above cases 
$\operatorname{slim}_{n < \omega } r_n =r$.

If $r$ is real and not dyadic, 
then its sign expansion is infinite and
 neither eventually $+$ nor eventually $-$.
Suppose that $r>0$, the case $r<0$ being treated symmetrically.
Since $r$ is not dyadic, then, in particular, it is not an integer.
Letting $[r]$ denote the integer part of $r$,
we have that $|r-r_n| < \min(r-[r], [r]+1-r)$,
for sufficiently large $n$, 
hence, from some point on, the integer part of  $r_n$
 is the same as the integer part of $r$; moreover, $r_n$
ha a binary point, too, hence these
parts of the sign expansion eventually stabilize. 
What remain are  the fractional parts, 
which are computed like the binary expansion, except possibly for 
a last sign/digit. Since the sign 
expansion of $r$ is neither eventually $+$ nor eventually $-$,
then, for every $ m < \omega$, both a $+$ and a $-$ occur
after the $m^{\rm th}$ place of the fractional part of $r$.
Let $q > m$ be such that both  a $+$ and a $-$ occur
between the  $m+1^{\rm th}$ place and
the $q^{\rm th}$ place of the fractional part of $r$. If 
 $|r-r_n| < 2^q$, then
$r$ and $r_n$ have the same fractional part up to the
$m^{\rm th}$ place.
Since $m$ is arbitrary,
the fractional part of the sequence  
$( r_n) _{n < \omega } $  eventually stabilizes to the
value of the fractional part of $r$.
In conclusion, the whole sign expansions stabilize
to the sign expansion of $r$.  
 
Notice that if $r$ is not  a dyadic rational, the same argument works
for the binary expansions, as well. Hence,
in the sense of the theorem,
the sign expansion representation has
 some
actual advantage 
over the binary one
 only for the countable set
of dyadic rationals.

Finally, the cases when $r=\infty$ or  
$r=-\infty$ are trivial.
 \end{proof}  

The proof of Theorem \ref{thm} gives a bit more.
A surreal $s$ is \emph{finite}  if $ -\omega < s < \omega $.
Any finite surreal can be expressed uniquely as a sum
$s= R(s) + \varepsilon(s) $,
where  
$ R(s)\in \mathbb R $ and $  \varepsilon(s)$ is infinitesimal.
Then the proof of Theorem \ref{thm} together
with  Berlekamp's  Rule and easy facts about
the sign expansion shows that
if $( s_n) _{n < \omega } $ 
is a sequence of finite surreal numbers
and 
$\lim_{n\to\infty}  R(s_n) $
exists and is finite,
then
$\lim_{n\to\infty}  R(s_n)  = R(\operatorname{slim}_{n < \omega } s_n )$.

Of course, the notion of limit in the classical sense
and the notion of s-limit do not coincide, in general,
even modulo infinitesimals.
Indeed, the latter limit is always defined, while this is not
necessarily the case for the former.
As we mentioned during the proof, Theorem \ref{thm}
shows that  the sign expansion has some real advantage
 over the binary expansion only in the case of dyadic numbers.
However, independently from Theorem \ref{thm}, the sign expansion has the advantage of  representing
each real \emph{uniquely},
while the choice of, say, the binary representation
$1.000\dots$ in place of $0.111\dots$ for the natural number $1$ 
might be perceived as somewhat
arbitrary.  

The eventual presence  of an infinitesimal in
Theorem \ref{thm} does not seem to 
be a serious drawback.
Anyone  interested only in
real numbers would surely feel free to ignore it.
In a sense, the presence of  $\pm 1/ \omega $ has some use,
since it tells us when the sequence converges from above or from below.
However, notice the asymmetry between the cases of dyadic  and nondyadic
limits, since the infinitesimal can appear only in the former case.

Another notion of limit
for $ \omega$-indexed sequences of surreal numbers
is known and useful. It is usually denoted as the \emph{Limit},
with upper-case L, and is obtained by taking componentwise the
limits (in the sense of real analysis)
 of the coefficients in the Conway normal
representation of the elements of the sequence.
This Limit is not always defined:
 the componentwise real limits should
always exist and be finite, and the result should
give an actual surreal number
(if the result contains an ascending 
sequence of exponents of $ \omega$, 
it is not a surreal number).

Though Theorem \ref{thm} shows that 
the Limit and the s-limit give quite close
results when taking the limit of
a sequence of reals, on the other hand,
for arbitrary surreals, the two limits could turn out to be
quite remote.
For example, $\operatorname{Lim}_{n\to\infty}   \frac{ \omega}{n} = 0 $,
while $\operatorname{slim}_{n < \omega } \frac{ \omega}{n} = \sqrt{ \omega } $.
This last s-limit
might appear unnatural, but notice that 
$\sqrt{ \omega }$ is ``multiplicatively  halfway'' 
between $1=  \frac{ \omega}{ \omega }$ 
and $ \omega$.
However, $\operatorname{Lim}_{n\to\infty}   (\frac{ \omega}{n}+1) = 1 $,
while $\operatorname{slim}_{n < \omega } (\frac{ \omega}{n} +1) $ is still 
$  \sqrt{ \omega } $. This shows that 
the s-limit of a finite sum is not always
the sum of the s-limits.
In another direction,
$\operatorname{Lim}_{n\to\infty}    \omega^n = 0 $,
while
 $\operatorname{slim}_{n < \omega }   \omega^n = \omega ^ \omega  $,
which seems much closer to intuition and corresponds
to a general rule we shall describe 
below for taking s-limits of ordinals.
As a final example, 
$\operatorname{Lim}_{n\to\infty}   ( \omega - n ) $ is undefined,
and
 $\operatorname{slim}_{n < \omega }   (\omega - n) =   \frac{ \omega}{ 2} $,
again, halfway between $0= \omega - \omega $ and $ \omega$. 

Probably, there is not a unique notion of limit for a sequence of surreals
which is good
for every purpose; in each particular case one should choose the 
most appropriate
notion.

Taking s-limits of surreal numbers is not always a monotone operation.
For example, if  $a_ n= n $ and $b_n = \omega -1$, for every $n < \omega$,
we have $a_n < b_n$, for every $n< \omega $, 
but  $ \operatorname{slim} _{n< \omega } a_n = \omega > \omega -1 = 
\operatorname{slim} _{n< \omega } b_n  $.
Of course, a similar situation
should occur for every notion of limit defined on 
\emph{all} sequences of surreals,
just assuming that the limit of a constant sequence gives its
constant value.
Then if the limit of some sequence is greater than all
the elements of the sequence, you can also
find a surreal in between the limit
and all the members of the sequence, and the same argument as above applies.

It is not always the case that the s-limit
of a subsequence coincides with the s-limit of the sequence.
If $a_n= n$, for $n$ even, and
$a_n = \omega -1$, for $n $ odd, then
$ \operatorname{slim} _{n< \omega } a_n = \omega $,
but $ \operatorname{slim} _{n< \omega, \text{ $n$ odd}} a_n = \omega -1 $. 
However, the s-limit of a nondecreasing sequence
gives a result $\geq$  than all the elements of the sequence; moreover,
a subsequence of a nondecreasing sequence has the same s-limit.
Since the proofs of the above facts work as well
for sequences indexed by ordinals $> \omega$,
we shall present the general result.
Notice that Definition \ref{def1}
can be naturally extended to deal
with arbitrary ordinal-indexed
sequences; just consider those initial
segments of sign expansions which are eventually constant.
The reader can find full details
in Definition \ref{def} below.   

\begin{prop} {\label}{prop} 
Suppose that  $( s _ \beta  ) _{ \beta  < \alpha  } $
and  $( t _ \zeta  ) _{ \zeta  < \eta } $  are
nondecreasing sequences
of surreals with s-limits, respectively, $s$ and $t$. Then
  \begin{enumerate}[(a)]
    \item
 $ s_ \delta  \leq s$,
for every $ \delta   <  \alpha $.
\item
If $( t _ \zeta  ) _{ \zeta  < \eta } $ is 
(an order-preserving rearrangement of) a cofinal subsequence of
 $( s _ \beta  ) _{ \beta  < \alpha  } $, then $t=s$.
\item 
Suppose that  $( s _ \beta  ) _{ \beta  < \alpha  } $
and  $( t _ \zeta  ) _{ \zeta  < \eta } $ are \emph{chained},
in the sense that,
for every $ \beta < \alpha $, there is 
$ \zeta  < \eta$ such that 
 $s_ \beta \leq t_ \zeta $ 
and conversely.
Then $s=t$. 
   \end{enumerate}
\end{prop} 

\begin{proof} 
Let us say that two surreals
$\gamma'$-agree if  their sign expansions 
 agree up to place $\gamma'$ included. 

(a) Let $ \delta   <  \alpha $,
we want to show that 
 $ s_ \delta  \leq s$.
If, for every ordinal $\gamma$, 
place $\gamma$ in the subsequence
 $( s _ \beta  ) _{ \delta \leq \beta  < \alpha  } $ is constant,
 then the subsequence itself is constant, the definition 
of the s-limit gives 
$ s_ \delta  = s$ and we are done.
Otherwise, let $\gamma$ be the first ordinal 
such that place $\gamma$ in the subsequence
 $( s _ \beta  ) _{  \delta \leq \beta  < \alpha  } $ is not  constant.
By the definition of $\gamma$, all
the $s_\beta$'s $\gamma'$-agree, for every $\gamma' < \gamma $
and $\beta \geq \delta $,
hence,   
again by the definition of the s-limit, $s$ and $ s_ \delta $
$\gamma'$-agree, for every $\gamma' < \gamma $.  
Since the $s_\beta$'s $\gamma'$-agree, for every $\gamma' < \gamma $
and $\beta \geq \delta $,
and the sequence is nondecreasing, the only possible
transitions at place $\gamma$ are from $-$ to undefined or 
$+$ and from undefined to $+$. By the definition of $\gamma$,
at least one transition occurs and, since there is a finite number
of possible transitions
and no cycle is possible, place $\gamma$ eventually stabilizes
to some value, which will be the value  in $s$. Thus  $ s_ \delta < s$.

(b) Suppose by contradiction that
$t\not =s$
and
let $\gamma$ be 
the first place at which they disagree.
Hence at least one between 
$t( \gamma )$ and 
$s ( \gamma )$ is defined.
Suppose that $t( \gamma )$ is defined;
the other case is similar and easier.
By the definition of the s-limit, there is some
$\bar {\zeta}$ such that 
  $t _ \zeta   $'s $\gamma$-agree with $t$,   
for every $\zeta \geq \bar {\zeta}$.
If 
$t _{\bar {\zeta}} \leq u \leq t _ \zeta  $ 
and $t _{\bar {\zeta}} $ and $  t _ \zeta  $
$\gamma$-agree, then they $\gamma$-agree with $u$.  
Since 
$( t _ \zeta  ) _{ \zeta  < \eta } $ is 
cofinal in 
 $( s _ \beta  ) _{ \beta  < \alpha  } $
and
 $( s _ \beta  ) _{ \beta  < \alpha  } $ is nondecreasing,
then the $s_ \beta $'s eventually $\gamma$-agree with 
$t _{\bar {\zeta}}$, hence with $t$. Then the definition of s-limit  
gives that $s$  $\gamma$-agrees with 
$t$, a contradiction.

(c) Consider a new sequence made by all the elements
of both sequences, ordered in such a way that the new big
sequence is still nondecreasing.
Since the original sequences are chained,
 they are both cofinal in the big sequence
(except perhaps for the trivial case in which all
 the sequences are eventually constant).
If $u$ is the  s-limit of the big sequence, then, by (b),
$u=s$ and  $u=t$, thus $s=t$.  
\end{proof}   

As we hinted at the beginning,
the s-limit of a non decreasing ordinal-indexed sequence
of ordinals is their supremum. In general, 
the s-limit of an ordinal-indexed sequence of ordinals is
their inferior limit, i. e., 
the supremum of the set of  those ordinals $\alpha$ such that 
the  members of the sequence 
 are eventually $\geq \alpha$.

Since an addition operation is defined among 
 surreal numbers, any notion of limit
entails the definition of a series.
If $( s _n ) _{ n < \omega } $  is a sequence of surreals,
 let $\sum^s _{n < \omega } s_n$ be
 $\operatorname{slim}_{n < \omega } S_n$,
where  $S_n$ denotes the partial sum
$s_0 + s_1 + \dots + s _{n-1} $.   
By Proposition \ref{prop}(c), 
if all the $s_n$'s are nonnegative, then  
$\sum^s _{n < \omega } s_n$ is invariant under permutations,
since the corresponding partial sums are chained in the sense of
\ref{prop}(c).  
Here we are using the general commutative-associative property 
of $+$ and the fact that the nonnegative surreals 
form an ordered monoid.

Since
  we can define the s-limit 
of every ordinal-indexed sequence of surreals, the above
 s-sum of length $ \omega$ can be extended to the transfinite.
Details go as follows.
The s-sum of the empty sequence is $0$.
If $\sum^s _{ \alpha < \beta  } s_ \alpha $
has been already constructed, let 
 $\sum^s _{ \alpha < \beta +1  } s_ \alpha 
= s _{ \beta} + \sum^s _{ \alpha < \beta  } s_ \alpha $.
Finally, if
$\beta$ is limit, let
 $\sum^s _{ \alpha < \beta  } s_ \alpha 
= \operatorname{slim} _{ \beta' < \beta }  \sum^s _{ \alpha < \beta'  } s_ \alpha $.
In the case when all the $s_ \alpha $'s
are ordinals, 
 the above iterated natural sum has been studied elsewhere.
Notice that, when restricted to ordinals, $\sum^s$ is different from the usual
transfinite ordinal sum $\sum$; though the limiting process is the same,
the successor steps in defining $\sum^s$ 
correspond to taking the natural ordinal sum, while in $\sum$
the usual noncommutative ordinal sum is used.

Invariance of  $\sum^s $ under permutations
does not extend beyond $ \omega$; actually, invariance fails already
at stage $ \omega+1$. Just take $s_0=0$
and all the other $s_ \alpha $'s to be $1$. Then
$\sum^s _{ \alpha < \omega +1 } s_ \alpha = \omega +1$,
but if we permute $s_0$ with $s_ \omega $, we get
$ \omega$ instead. 
It will be probably interesting to consider transfinite s-sums
 in the case when all the $s_ \alpha $'s are equal. 

The s-sum equals  the surreal sum in many cases,
notably, 
it follows immediately from \cite[Theorem 5.12]{G} that 
$\sum ^s_{ \alpha < \beta  } \omega ^{s_ \alpha} r_ \alpha
=
  \sum_{ \alpha < \beta  } \omega ^{s_ \alpha} r_ \alpha  $,
provided the latter sum represents the Conway normal form of some surreal.
Thus the s-sum provides another more concrete interpretation
for the formal series defining normal forms.

One could try to extend the classical series expansions
of real analysis to infinite numbers by using the s-sum,
for example, by considering $f(s) = \sum^s _{n < \omega } \frac{s^n}{n!} $.
 Though $f( \omega )$ gives the expected value $ \omega ^ \omega $, which is equal
to  $ \exp( \omega )$
in the sense of the surreal exponentiation, 
on the other hand,   $f( \omega +1) = \omega ^ \omega $,
too,
hence series expansions through the s-sum generally
give unwanted results.
Actually, as follows from
the proof of Theorem \ref{thm}, if $e^ r = d$ is dyadic $>1$,
then $f(r) = d- \nicefrac{1}{ \omega } \not= d$,
hence series obtained by using the s-sum
do not always assume  the exact wanted value even for 
 real numbers.
However, there is perhaps  the possibility
of modifying the s-limit and hence the s-sum 
in order 
to make things work better,
but this is still to be developed.
See the last sentence in Remark \ref{rmk}. 

\begin{remark} {\label}{canrepr}  
The \emph{canonical representation} of a surreal 
$s$ is   $\{ F \mid G\}$,
where $F$, $G$, respectively,  are the sets of those surreals 
 which are initial segments of $s$ and are
$<s$, respectively, $>s$.
If $(s_ \beta ) _{ \beta  < \alpha } $ 
is a sequence of surreals and 
$\{ F_ \beta  \mid G_ \beta \} _{ \beta < \alpha } $
are their respective canonical representations,
then a representation
of $\operatorname{slim} _{\beta  < \alpha } s_ \beta  $ is  
 $\{ F \mid G\}$,
where
$F = \bigcup _{ \beta < \alpha }  \bigcap _{ \beta ' \geq \beta } F _{ \beta '}  $
and 
$G = \bigcup _{ \beta < \alpha }  \bigcap _{ \beta ' \geq\beta } G _{ \beta '}  $,
in words, we take as representatives only those elements
which are eventually in the lower, respectively, upper sets.
 
The same remark holds if we start considering, as
another representation, those
$F$ and  $G$ which are  the sets of those surreals 
born strictly before $s$ and are
$<s$, respectively, $>s$.
\end{remark}  

As we mentioned, Definition \ref{def1} can be naturally extended in order to
deal with  sequences of strings
of arbitrary symbols, not just  $+$ and  $-$. 
Actually, we shall present a generalization in which
we take into account 
linear orders, not only
well-orders.

\begin{definition} {\label}{def}    
 Let $A$ be any set and $L$, $ M$ be linearly ordered
sets. We shall consider sequences of the form $( a_ \ell) _{\ell \in L} $,
where each $a_\ell$ is a function from some initial segment
of $M$ to $A$. Here both the empty set and $M$ itself are considered
to be initial segments.

It is useful to visualize the 
$a_\ell$'s as rows in an infinite $L \times M$  matrix
with possibly empty entries.
In this sense,
$a_\ell$ is the $\ell^{\rm th}$ row
of the matrix, and
$a_ \ell(m)$ is the element in the  $m^{\rm th}$ column 
of the
 $\ell^{\rm th}$ row.
(Warning: in the case when $L$ or $M$ is an ordinal, 
the above terminology might be misleading, since, say, $0$ is the 
$1^{\rm st}$ ordinal.)

 We define the \emph{s-limit
  of 
 $( a_ \ell) _{\ell \in L} $},
in symbols, $\operatorname{slim}_{\ell \in L} a_ \ell  $ 
 as follows.

If $L$ has a maximum $\bar \ell$, then  
we set 
$\operatorname{slim}_{\ell \in L} a_ \ell  =  a_{\bar \ell}$. 

If $L$ has no maximum, then
$\operatorname{slim}_{\ell \in L} a_ \ell $ is the function $  a$
given by the following prescriptions.
If $m \in M$, we declare $a(m)$ to be defined   
in case there is some $\ell (m) \in L$ such that,
for every $m' \leq m$ and every $\ell, \ell' \geq \ell(m)$,
we have   
 $ a_{\ell}(m') = a_{\ell\/'}(m')$.
If this is the case, we let 
$a(m) = a_{\ell(m)}(m) $.
It is immediate from the definition 
that the domain $\operatorname{dom} (a)$ of $a$ is an initial 
(possibly empty) segment of $M$.
In particular, the s-limit is \emph{always} and uniquely
defined. 

Under the above matrix visualization,
$a(m)$ is defined if there is some $\ell (m) $ such that
all the columns before (and including) the $m^{\rm th}$ column
are eventually constant from the $\ell (m) ^{\rm th}$ row on.
Notice that we could have declared 
$a(m)$ to be defined just in case 
the $m^{\rm th}$ column is eventually
constant. This would give a different definition of a limit;
this latter definition has the drawback that
it does not imply that 
$\operatorname{dom} (a)$ is an initial segment of $M$.
To avoid the trouble, we can define
another notion of limits of strings, call it $\operatorname{slim}^\diamond$, 
by declaring $a(m)$ to be defined  if the 
 $m^{\rm th}$ column is eventually
constant \emph{and} all the preceding columns 
are eventually constant, too (the difference with  $\operatorname{slim}$ is that 
here we make no assumption  about  the points from which 
the columns become constant).

We believe $\operatorname{slim}$ to be more natural than 
$\operatorname{slim}^\diamond$. For sure, 
though the version of Proposition \ref{prop}(a)
 holds for $\operatorname{slim}^\diamond$ with the same proof,
the analogues of
Proposition \ref{prop}(b)(c) do not hold for    
$\operatorname{slim}^\diamond$.
Consider the following increasing 
sequence of strings of length $ \omega+1$:
$a_0 =-----\dots - $, $a_1=+----\dots + $, $a_3=   ++---\dots - $
$ a_4 = +++--\dots + $, \dots
(the main point is that  signs in the last place alternate).
If $( b _n ) _{ n < \omega } $   is the subsequence consisting of 
the strings with odd index, then
$\operatorname{slim} _{n  < \omega } a_n = \operatorname{slim}^ \diamond _{n  < \omega } a_n
= \operatorname{slim} _{n  < \omega } b_n = \omega $,
but 
 $\operatorname{slim}^ \diamond _{n  < \omega } b_n = \omega - 1 \not= \operatorname{slim}^ \diamond _{n  < \omega } a_n$;
 in particular, $\operatorname{slim}^ \diamond $ and $\operatorname{slim}$ differ on
$( b _n ) _{ n < \omega } $. 
The counterexample to  \ref{prop}(c) is obtained by considering
also the  subsequence consisting of 
the strings with even index.
However, for many  arguments in this note
the two definitions would turn out  to be essentially equivalent.

Notice that in the above definitions  we simply discard 
those columns for which the entries are not eventually constant; actually,
we discard all further columns which follow a column as above.
In the case when we do not have to discard columns, we shall speak
 of a full  limit.
Formally, we say that
$a$ is the
\emph{full limit
  of 
 $( a_ \ell) _{\ell \in L} $}
if $\operatorname{slim}_{\ell \in L} a_ \ell  =  a$
and, in addition,  for every $m \in M$,
if there is some $\ell (m) \in L$
such that $a_\ell (m)$ is defined, for 
every $\ell \geq \ell (m) $,
then   $a(m)$ is defined, too. 
In other words, the s-limit $a$ of a sequence 
is the full limit of the sequences truncated at 
$\operatorname{dom} (a)$, where $\operatorname{dom} (a)$ is the largest possible 
initial segment of $M$ such that the truncated sequences
do admit a full limit. 
In the above example, 
$ \operatorname{slim} _{n  < \omega } b_n $
is not a full limit, though, in the sense of 
$ \operatorname{slim} ^ \diamond$,
$ \operatorname{slim}^ \diamond _{n  < \omega } b_n $   
 would actually be a full limit.
\end{definition}

Notice that if $L = \omega $, then 
$\operatorname{slim}_{n \in \omega } a_ n $ 
is invariant under permutations of the 
$a_n$'s.
For arbitrary $L$, 
$\operatorname{slim}$ satisfies a nice continuity property.
If $L$ is partitioned into convex subsets as $( L_i) _{i \in I} $,
then $I$ inherits from $L$ the structure of a linearly ordered set, 
and then   $\operatorname{slim}_{\ell \in L} a_ \ell = \operatorname{slim}_{i \in I} \operatorname{slim}_{\ell \in L_i} a_ \ell $,
for every sequence $a_ \ell $.
The s-limit behaves only partially well with respect to 
string concatenation, that we shall denote by juxtaposition.
Though $\operatorname{slim}_{\ell \in L} ba_ \ell =b \operatorname{slim}_{\ell \in L} a_ \ell $ 
for all strings, it is not necessarily always the case
that $\operatorname{slim}_{\ell \in L} a_ \ell b = (\operatorname{slim}_{\ell \in L} a_ \ell) b $:
just take $L= \omega $, $a_n$ of length $n$, for $n \in \omega$,  
and $b$ of length $1$. However,
  $\operatorname{slim}_{\ell \in L} a_ \ell b_ \ell = (\operatorname{slim}_{\ell \in L} a_ \ell)
\operatorname{slim}_{\ell \in L} b_ \ell $
holds when all the $a_ \ell $'s have the same length and 
$\operatorname{slim}_{\ell \in L} a_ \ell$ is a full limit.

If $a= \operatorname{slim}_{\ell \in L} a_ \ell  $ 
then either the $a_\ell$'s restricted to   $\operatorname{dom} (a)$ are
eventually constant, or  $\operatorname{cf} \operatorname{dom} (a) = \operatorname{cf} L$.  
The s-limit of a sequence is equal to the s-limit 
of some subsequence cofinal in $L$, but, in general, 
as we showed before Proposition \ref{prop}, 
it is not necessarily
the case that every cofinal subsequence has the same s-limit.
However, if $\operatorname{slim}_{\ell \in L} a_ \ell$ is a full limit,
then every cofinal subsequence has the same limit.

\begin{remark} {\label}{rmk}   
One can introduce a shorthand for the surreal sign expansion.
We can consider a surreal number as a sequence of nonzero 
signed ordinals. As in the standard case,
$0$ is represented by the empty sequence. 
If the first sign in the expansion of the surreal $s$
is $+$, and we have exactly $\alpha$ consecutive $+$'s at the beginning,
the first element of the shorthand is $ \alpha $; then if we have a
certain number $\beta$ of consecutive $-$'s, the second element of the shorthand 
is $-\beta$, and so on. In such a shorthand, ordinals and negated ordinals
alternate. Let $\operatorname{sh}(s)$ denote 
the shorthand of $s$ in the above sense.
 Taking Definition \ref{def} literally
would give us strange results,
such as 
$\operatorname{slim} _{n < \omega } \operatorname{sh}(n) = 0 $.
However, we can 
adapt the definition 
by 
taking place by place the inferior limit
for places which consist eventually of positive ordinals,
taking the superior limit 
for places which consist eventually of negative ordinals,
and considering a place undefined in the limit if it
is not of the above kind, with 
the usual convention that  also all the  places which follow
should be considered undefined.
Let us denote by $\operatorname{slim}^*$ this modified limit
acting on shorthands. It gives results different from $\operatorname{slim}$.
For example, the s-limit of the sequence
$+-$, $++--$, $+++---$, \dots, is $++++\dots = \omega $,
while the s-limit$^*$ is $++++\dots ----\dots = \nicefrac{\omega}{2} $.

Of course, in the general sense of Definition \ref{def},
$\operatorname{slim}^*$ can be defined when $A$ is a complete  lattice. 
Maybe there are useful variations on the above limit,
say, using still different representations of surreal numbers.
This has still to be investigated.
  \end{remark}

As is the case for most notions of convergence,
Definition \ref{def} can be extended to the situation
when we work modulo some filter.
Under the notations in 
Definition \ref{def} 
and if $F$ is a filter over $L$, we let
the \emph{$F$-limit} 
$F $-$\lim a_ \ell$  be the function $a$ such that    
$a(m)$ is defined in case there is $X \in F$ such that, 
for every $m' \leq m$ and every $\ell, \ell' \in X$, 
we have $a_ \ell (m')= a _{\ell'}  (m')$.
If this is the case, we let $a(m) = a_ \ell (m)$,
for some $\ell\in X$. Notice that 
the definition of the s-limit in Definition \ref{def}
is the particular case of the above $F $-limit
when $F$ is the unbounded filter over $L$.
The definition in the present remark looks particularly promising, since
if $A$ is finite, $F$ is an ultrafilter and, for every 
$ m < \omega$, the $a_ \ell$'s are eventually of length $\geq m$,
then    $F $-$\lim a_ \ell$ has length $ \geq \omega$.
We could have introduced a variant of the above 
definition (in the same spirit as of $\operatorname{slim}^\diamond$)
  by saying that $a(m)$ is defined 
if, for every $m' \leq m$, 
 there is some $X_{m'} \in F$ such that 
 $a_ \ell (m')= a _{\ell'}  (m')$,
for every $\ell, \ell' \in X_{m'}$.

 
\smallskip

In conclusion, at least from some point of view,
 the s-limit appears to be quite unnatural.
For example, the s-limit of any sequence with infinitely many positive numbers
and infinitely many negative numbers, no matter their size,
 is always $0$.
This is however essentially a consequence of the pleasant fact that the 
s-limit
is \emph{always} defined.
On the other hand, besides having a very natural order/string-theoretical
definition and a fine interpretation in the surreal spirit,
as explained in Remark \ref{canrepr}, 
 the s-limit coincides with---or, better, incorporates---
classical notions of limits in some  significant cases.
This suggests that the s-limit might deserve further study.

\smallskip 

This is a preliminary version, it might contain
inaccuracies (to be precise, it is more likely to contain
inaccuracies than planned subsequent versions).
We have not yet performed a completely accurate 
search in order to 
check whether some of the results presented here are 
already known. Credits for already known results
should go to the original discoverers. 

\smallskip 

The list of references below is not intended to represent
the historical development of the subject.
It is not intended that, whenever we quote a result, 
the result should necessarily be attributed to the author of the 
quoted book.
The reader looking for historical remarks, credits
 for original contributions
and sources is
invited to consult the listed books and, possibly,
further  references 
there.

\begin{thebibliography}{1}

\bibitem[G]{G} H. Gonshor, 
   \emph{An introduction to the theory of surreal numbers},
   London Mathematical Society Lecture Note Series, \textbf{110}, Cambridge
   University Press, Cambridge, 1986. 

\bibitem[S]{S}
A. N. Siegel, \emph{Combinatorial game theory}, Graduate
   Studies in Mathematics, \textbf{146}, American Mathematical Society, Providence,
   RI, 2013.

\end{thebibliography} 

\end{document}

