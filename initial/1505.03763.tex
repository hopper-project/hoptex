\documentclass[11pt, reqno]{amsart}

\usepackage{amssymb,latexsym,amsmath,amsfonts}
\usepackage{mathrsfs}
\usepackage{graphicx}

\hoffset = -40pt
\voffset = -10pt
\textwidth = 6.0in
\textheight = 8.45in
\numberwithin{equation}{section}

\theoremstyle{definition}
\newtheorem{definition}{Definition}[section]
\newtheorem{fact}[definition]{Fact}

\theoremstyle{remark}
\newtheorem{remark}[definition]{Remark}

 \theoremstyle{plain}
\newtheorem{theorem}[definition]{Theorem}
\newtheorem{result}[definition]{Result}
\newtheorem{lemma}[definition]{Lemma}
\newtheorem{proposition}[definition]{Proposition}
\newtheorem{example}[definition]{Example}
\newtheorem{corollary}[definition]{Corollary}

 

\begin{document}

\title[The three-point Pick interpolation problem]{The three-point Pick-Nevanlinna interpolation problem on the polydisc}

\author{Vikramjeet Singh Chandel}
\address{Department of Mathematics, Indian Institute of Science, Bangalore 560012, India}
\email{abelvikram@math.iisc.ernet.in}

\thanks{This work is supported by a UGC Centre for
Advanced Study grant and by a scholarship from the IISc}

\keywords{Pick--Nevanlinna interpolation, irreducible inner functions, polydiscs, rational inner functions}
\subjclass[2010]{Primary: 32A17, 32F45; Secondary: 46E20, 30J10}

\begin{abstract}
We give a characterization for the existence of an interpolant that is a rational inner function on 
the unit polydisc ${\mathbb{D}}^n,$ $n\geq 2,$ for prescribed three-point Pick--Nevanlinna data. Our approach reduces
the search for a three-point interpolant to finding a single rational inner function that satisfies a type of
positivity condition and arises from a polynomial of a very special form.
One of the key tools to achieve this is a pair of factorization results for rational inner functions,
which might be of independent interest. 
\end{abstract}
\maketitle

\section{Introduction and statement of results}\label{S:intro}
The problem alluded to in the title of this work is the following
(in this work, ${\mathbb{D}}$ will denote the open unit disc with centre $0\in{\mathbb{C}}$):
\begin{itemize}
\item[$(*)$] Let $X_1,\ldots,X_{N}$ be distinct points in ${\mathbb{D}}^n$ and let $w_1,\ldots,w_{N}\in{\mathbb{D}}.$
Characterize those data $\{(X_j,w_j):1\leq j\leq N\}$ for which there exists a holomorphic function
$F:{\mathbb{D}}^n\longrightarrow{\mathbb{D}}$ such that $F(X_j)=w_j, \ j=1,\ldots,N.$
\end{itemize}
\noindent This, in the case $n=1$ was solved by Pick in $1916$ and the properties 
of an interpolant $F,$ whenever it exists, were studied by Nevanlinna. Sarason's proof 
\cite{sarason:ftp67} opened up a new paradigm 
for approaching $(*)$ for $n\geq 2.$ This approach led to Agler's solution to {\em a version of $(*)$},
characterizing those $\{(X_j,w_j):1\leq j\leq N\},$ for any $n\geq 2,$ that admit
an interpolant in the Schur--Agler class. This stems from Agler's solution \cite{ag:ftp88} of $(*)$ for $n=2.$
\smallskip

Agler's solution to $(*)$ for $n=2$ relies on And\^{o}'s inequality \cite{ando:ftp63}
(see also the article \cite{agmac:ftp99} by Agler--McCarthy). However, for $n\geq 3,$ the
Schur--Agler class is {\em strictly} smaller than the class $\{F\in H^{\infty}({\mathbb{D}}^n):\sup_{{\mathbb{D}}^n}|F|\leq 1\}$
(i.e., the Schur class). There have thus been many articles in the last couple of decades that have dwelt on the problem
$(*).$ We shall not cite all of these works: we refer the reader to the articles \cite{cw:ftp94},
\cite{ham:ftp013} and to the works listed in the references therein. However, despite all the results obtained so far,
where matters stand at this moment, one is faced with two difficulties:
\begin{itemize}
\item[$(i)$] The currently known characterizations for a Schur-class interpolant are not amenable to any computational
procedure of checking or search.
\item[$(ii)$] One has little knowledge of the structure of the interpolant $F\in{\mathcal{O}}({\mathbb{D}}^n;{\mathbb{D}})$ whenever it exists.
\end{itemize}
\smallskip

A class of functions in which one may look for an interpolant for the data $\{(X_j,w_j):1\leq j\leq N\}$
(or, alternatively, conclude that there is no such interpolant in this class) is the class of 
rational inner functions on ${\mathbb{D}}^n.$ This would certainly address the concern $(ii)$ above: there is a lot
that one knows about the structure of rational inner functions. We shall recall some of these properties at
the beginning of Section~\ref{S:prelim-res}. It turns out that replacing the Schur class by the class of rational
inner functions in the statement $(*)$ allows us\,---\,for the three-point interpolation problem,
at any rate\,---\,to use aspects of the strategy {\em originally} used to solve $(*)$ for
$n=1.$ Specifically: we are able to exploit the fact that the respective automorphism groups act transitively on ${\mathbb{D}}^n$
and ${\mathbb{D}}.$
\smallskip

Our result, albeit only for $N=3,$ also addresses the concern $(i)$ above to an extent. The problem
of determining whether a function interpolates three pairs of points is reduced to finding a single rational inner function
that satisfies a certain positivity condition, and which arises from a polynomial of a very special form.
To make this precise, we shall need some notations and terminology. Given a polynomial
$Q\in {\mathbb{C}}[z_1,\dots, z_n]$, recall that the {\em support of $Q$} is the set
\[
 {\sf supp}(Q) = \big\{\alpha\in {\mathbb{N}^n} : \ \text{\small $\dfrac{\partial Q}{\partial z^\alpha}(0)$}\neq 0\big\}.
\]
Writing $Q(z) =\sum_{j=0}^{d}\sum_{|\alpha|=j} a_{\alpha}z^{\alpha}$, define
(we use standard multi-index notation here)
\begin{align*}
 \widetilde{Q}(z)&:=\sum_{j=0}^{d}\sum_{|\alpha|=j}\overline{a_{\alpha}}z^{\alpha},\\
 \widetilde{Q}\left(\frac{1}{z}\right)&:=\sum_{j=0}^{d}\sum_{|\alpha|=j}\overline{a_{\alpha}}
 \frac{1}{z^{\alpha}},\\
 \nu(Q)&:=(\nu_1(Q),\ldots,\nu_n(Q)),
\end{align*}
where $\nu_j(Q)$ denotes the
degree of the polynomial $Q(a_1,\ldots,a_{j-1},\zeta,a_{j+1},\ldots,a_n)\in{\mathbb{C}}[\zeta]$ for a {\em generic}
$(a_1,\ldots,a_{j-1},a_{j+1},\ldots,a_n)\in{\mathbb{C}}^{n-1}.$ We say that the polynomial $Q$ is {\em deficient
in degree} if the multi-index $\nu(Q)\notin {\sf supp}(Q)$ (our terminology stems from the fact that the
latter property is equivalent to $|\nu(Q)| > d$). We are now in a position to state our main theorem.
One final note: given $a\in{\mathbb{D}},$ $\psi_a$ will denote the automorphism
\begin{equation}
\psi_a(z)=\frac{z-a}{1-\bar{a}z}, \ \ \ z\in{\mathbb{D}}.\label{E:aaut}
\end{equation}

\begin{theorem}\label{T:mainThm}
Let $X_1,X_2,X_3$ be three distinct points in ${\mathbb{D}}^n, \ n\geq 2,$ and let $w_1,w_2,w_3\in{\mathbb{D}}.$ There exists
a rational inner function $F$ on ${\mathbb{D}}^n$ such that $F(X_j)=w_j, \ j=1,2,3,$ if and only if there exists a 
rational inner function $H$ on ${\mathbb{D}}^n$ such that
\[
 w_j'/H(X_j')\in\overline{\mathbb{D}}\;\;\;\text{for $j=1,2$},
\]
and is of either one of the following forms:
\[
 H(z) = \begin{cases}
 		z_j \ \ \text{for some $j:\,1\leq j\leq n$}, &\!\!\!\text{OR} \\
 		z^{\nu(Q)}\widetilde{Q}(\frac{1}{z})/Q(z), &{ \ }
 		\end{cases}
\]
where $Q$ is an irreducible polynomial having no zeros in ${\mathbb{D}}^n$ and is deficient in degree,
and there exists an integer $l\in\{1,2,\ldots,n\}$ such that the $2\times 2$ matrix
\begin{equation}
{\begin{bmatrix}
\dfrac{1-({w_j'}/{H(X_j')})(\overline{{w_k'}/{H(X_k')}})}{1-X_{j,\,l}'\overline{X'}_{k,\,l}}
\end{bmatrix}}_{j,k=1}^{2}\label{E:mainmat}
\end{equation}
 is positive semi-definite. Here $w_j':=\psi_{w_3}(w_j),$ $X_j':=\Psi_{X_3}(X_j), \ j=1,2,$ and we write
 $X_j=(X_{j,\,1},\ldots,X_{j,\,n}).$ Furthermore, if the latter conditions hold true, then:
\begin{itemize}
\item[$a)$] If the matrix in \eqref{E:mainmat} is zero, then $\exists c\in\partial{\mathbb{D}}$ such that
 $F={\psi}^{-1}_{w_3}\circ(cH)\circ\Psi_{X_3}$ is the desired interpolant.
\item[$b)$] If the rank of the matrix in \eqref{E:mainmat} is $r, \ r=1,2,$ then there is a
 Blaschke product $B$ of degree $r$ such that $F={\psi}^{-1}_{w_3}\circ((B\circ \pi_l)H)\circ\Psi_{X_3}$ is the desired interpolant
(here, $\pi_l$ denotes the projection onto the $l$-th coordinate, $l$ as introduced above).
\end{itemize}
\end{theorem}
\smallskip

Under the constraint $N=3,$ the above theorem characterizes the existence of interpolants
for a problem that is, in a sense, {\em less constrained} than Agler's version of $(*).$ This is because the Schur--Agler class
has positive codimension, relative to even the topology of local uniform convergence, in ${\mathcal{O}}({\mathbb{D}}^n),$ while the class of 
all rational inner functions on ${\mathbb{D}}^n$ is dense in the set ${\mathcal{O}}({\mathbb{D}}^n;{\mathbb{D}})$ with the latter topology. This is a consequence of
Carath{\'e}odory's Theorem (see \cite[Theorem 5.5.1]{rudin:ftp69}) and the examples in
\cite[Section 5]{ana:ftp012}.
\smallskip

The alert reader will surmise that the idea of the proof of Theorem~\ref{T:mainThm} is really
the Schur algorithm. Indeed, there are no univariate polynomials that are deficient in degree, owing to
which the matrix in \eqref{E:mainmat} will, for $n = 1,$ be a matrix that the reader will recognize.
However, the point of interest is that 3-point interpolation is determined by the outcome of a
search through a meagre class of rational inner functions. But, the details behind this 
observation are not entirely trivial. Indeed, this work is as much a study of certain properties
of rational inner functions on ${\mathbb{D}}^n$ as it is about Theorem~\ref{T:mainThm}.
The former is the content of Section~\ref{S:prelim-res}.
The proof of Theorem~\ref{T:mainThm} is given in Section~\ref{S:Thm}.
\medskip

\section{Some results about rational inner functions on ${\mathbb{D}}^n$}\label{S:prelim-res}

In this section, we shall present a few results concerning the rational inner functions 
on the polydisc $\mathbb{D}^n,$ which will rely on the properties of the polynomial ring
${\mathbb{C}}[z_1,\dots, z_n].$ We shall make use of the notation introduced prior to 
Theorem~\ref{T:mainThm}. These notations help us to present the following 
important discussion about rational inner functions on ${\mathbb{D}}^n.$

\begin{fact}\label{SS:rif}
An {\em inner function} on ${\mathbb{D}}^n$ is a function $f\in H^{\infty}({\mathbb{D}}^n)$ such that
$\lim_{r\to 1^{-}}|f(rw)|=1$ for almost every $w\in\mathbb{T}^n.$ A {\em rational inner function} on ${\mathbb{D}}^n$
is an inner function that is rational. It is elementary to see that, given a polynomial
$Q\in{\mathbb{C}}[z_1,\dots,z_n],$ any function of the form
\[ f(z)=\frac{Az^{\beta}\widetilde{Q}(\frac{1}{z})}{Q(z)},\]
where 
\begin{itemize}
\item $Z(Q)\cap{\mathbb{D}}^n=\emptyset,$
\item $z^{\beta}\widetilde{Q}(\frac{1}{z})$ is a polynomial,
\item $A$ is a unimodular constant,
\end{itemize}
is a rational inner function. Here, and in what follows, $Z(Q)$ denotes the zero set of $Q.$
Moreover, it is a fact \cite[Theorem 5.2.5]{rudin:ftp69} that every rational inner function on ${\mathbb{D}}^n$
has the above form.
\end{fact}
\smallskip

The next two results are central to proving Theorem~\ref{T:mainThm}, but are also of independent interest.

\begin{proposition}\label{P:factprop}
Let $f$ be a nonconstant rational inner function of the form
${z^{\nu(Q)}\widetilde{Q}(\frac{1}{z})}/{Q(z)},$ where
$Q$ is a nonconstant polynomial in ${\mathbb{C}^n}$ such that $Z(Q)\cap{\mathbb{D}}^n=\emptyset.$ Then:
\begin{itemize}
\item[$(a)$] There exist a nonconstant polynomial ${\mathcal{Q}}$ with $Z({\mathcal{Q}})\cap{\mathbb{D}}^n=\emptyset$
and a unimodular constant $C$ such that $f$ can also be expressed as
\begin{equation}
 f(z) = C\,\frac{z^{\nu({\mathcal{Q}})}\widetilde{\mathcal{Q}}(\frac{1}{z})}{{\mathcal{Q}}(z)}, \label{E:altform}
\end{equation}
and such that the numerator and the denominator of the above expression have no (nonconstant) irreducible
polynomial factors in common.

\item[$(b)$] There exist rational
inner functions $f_1,f_2\in{\mathcal{O}}({\mathbb{D}}^n),$ both nonunits in ${\mathcal{O}}({\mathbb{D}}^n),$
such that $f=f_1f_2$ in ${\mathbb{D}}^n$ if and only if ${\mathcal{Q}}$ is reducible in ${\mathbb{C}}[z_1,z_2,\ldots,z_n].$ 
\end{itemize}
\end{proposition}
\smallskip

We call a nonconstant rational inner function $f$ on ${\mathbb{D}}^n$ an {\em irreducible inner function} (resp., {\em reducible})
if we cannot (resp., can) express it as $f=gh,$ where $g$ and $h$ are rational inner functions and nonunits
in ${\mathcal{O}}({\mathbb{D}}^n).$
We now have the following corollary to Proposition~\ref{P:factprop}.

\begin{corollary}\label{C:irdefpol}
Let $f$ be an irreducible rational inner function such that $f(0)=0.$ Then, either
 $f(z) = z_j$ for some $j\in \{1,\dots, n\}$, or it has the form (modulo scaling by a unimodular constant)
\[
 f(z) = z^{\nu({\mathcal{Q}})}\widetilde{\mathcal{Q}}\left(\frac{1}{z}\right)\big/{\mathcal{Q}}(z),
\]
where ${\mathcal{Q}}$ is an irreducible polynomial having no zeros in ${\mathbb{D}}^n$ and is deficient in degree.
\end{corollary}
\smallskip

The corollary is immediate from Proposition \ref{P:factprop} and Fact~\ref{SS:rif} 
once we realize that the numerator of the rational inner function given by \eqref{E:altform}
{\em cannot} vanish at $0$ if $\nu({\mathcal{Q}})\in {\sf supp}({\mathcal{Q}})$.
We shall not write down the (essentially trivial, in view of Proposition \ref{P:factprop}) proof of this
corollary.
\smallskip

The proof of Proposition \ref{P:factprop} depends on a few lemmas. The first of these
states a simple factorization property associated to 
$Q$ and $z^{\nu(Q)}\widetilde{Q}(\frac{1}{z}).$

\begin{lemma}\label{L:conirr}
Let $Q$ be a nonconstant polynomial such that $Q(0)\not =0.$
Then $z^{\nu(Q)}\widetilde{Q}(\frac{1}{z})$ is irreducible 
in ${\mathbb{C}}[z_1,z_2,\ldots,z_n]$ if and only if $Q$ is irreducible in ${\mathbb{C}}[z_1,z_2,\ldots,z_n].$
\end{lemma}
\begin{proof}
Assume that $z^{\nu(Q)}\widetilde{Q}(\frac{1}{z})$ is reducible. Then there exist
nonconstant polynomials $P_1,P_2$ such that
\begin{equation}
z^{\nu(Q)}\widetilde{Q}\left(\frac{1}{z}\right)=P_1(z)P_2(z)\label{E:red}
\end{equation}
for all $z\in{\mathbb{C}^n}.$ Also, we have 
\begin{equation}
\nu_{j}(P_1)+\nu_{j}(P_2)=\nu_{j}(P_1P_2)
=\nu_{j}\left(z^{\nu(Q)}\widetilde{Q}\left(\frac{1}{z}\right)\right)
=\nu_{j}(Q).\label{E:deg}
\end{equation}
The last equality in \eqref{E:deg} follows from the assumption that $Q(0)\neq 0.$
Let ${\Omega}:={\mathbb{C}^n}\setminus(\cup_{j=1}^{n}\{z\in{\mathbb{C}^n}:z_j=0\})$ and define
$\Theta(z):=(\frac{1}{z_1},\frac{1}{z_2},\ldots,\frac{1}{z_n})$ for all $z\in{\Omega}.$
Note that $\Theta\in\text{Aut}({\Omega}),$ whence, by the surjectivity of $\Theta,$
the fact that $\Theta=\Theta^{-1},$ and from \eqref{E:red} we have
\[
\left(\Theta(z)\right)^{\nu(Q)}\widetilde{Q}(z)=\left[P_1\circ\Theta(z)\right]
\left[P_2\circ\Theta(z)\right]\;\;\;\text{for all $z\in{\Omega}$.}
\]
From the fact that ${\mathbb{C}^n}\setminus{\Omega}$ is nowhere dense in ${\mathbb{C}^n}$ and from \eqref{E:deg} we have
\[
\widetilde{Q}(z)=z^{\nu(Q)}P_1\left(\frac{1}{z}\right)P_2\left(\frac{1}{z}\right)
=\left[z^{\nu(P_1)}P_1\left(\frac{1}{z}\right)\right]
\left[z^{\nu(P_2)}P_2\left(\frac{1}{z}\right)\right]\;\;\;\text{for all $z\in{\mathbb{C}}^n$.}
\]
So $\widetilde{Q}$ and, therefore, $Q$ is reducible in ${\mathbb{C}}[z_1,\ldots,z_n].$
\smallskip

Let us now assume that $Q$ is reducible. Then there exist nonconstant
polynomials $Q_1,Q_2$ such that
\[
\widetilde{Q}(z)=\widetilde{Q}_1(z)\widetilde{Q}_2(z)
\]
for all $z\in{\mathbb{C}^n}.$
It is elementary to see that $\nu(Q)=\nu(Q_1)+\nu(Q_2).$ It follows\,---\,owing to the fact that
 the set $\cup_{j=1}^{n}\{z\in{\mathbb{C}^n}:z_j=0\}$ is nowhere dense\,---\,that
\begin{equation}
z^{\nu(Q)}\widetilde{Q}\left(\frac{1}{z}\right)=
\left[z^{\nu(Q_1)}\widetilde{Q}_1\left(\frac{1}{z}\right)\right]
\left[z^{\nu(Q_2)}\widetilde{Q}_2\left(\frac{1}{z}\right)\right]\;\;\;\text{for all $z\in{\mathbb{C}^n}.$}
\label{E:red1}
\end{equation}
The assumption $Q(0)\neq 0$ implies that $Q_{j}(0)\not=0,\,\,j=1,2.$ Thus we have 
\[
\nu\left(z^{\nu(Q_j)}\widetilde{Q}_j\left(\frac{1}{z}\right)\right)=\nu(Q_j)\not=(0,\ldots,0).
\]
Hence neither of the factors on the right hand side of \eqref{E:red1} is a constant.
Thus $z^{\nu(Q)}\widetilde{Q}\left(\frac{1}{z}\right)$ is reducible.
\end{proof}
\smallskip

We know that ${\mathcal{O}}({\mathbb{D}}^n)$ is an integral domain. The next lemma concludes that the
nonconstant rational inner functions on ${\mathbb{D}}^n$ are nonunits in this ring. While the result itself is unsurprising,
it does need a few lines of justification. To this end,
we shall need a definition. A set $E\subset {\mathbb{C}}^n$ is called a {\em determining set for
polynomials} if, for every polynomial $p\in {\mathbb{C}}[z_1,\dots, z_n]$, $p(z) = 0$ for every
$z\in E$ implies that $p\equiv 0$.

\begin{lemma}\label{L:nu}
Let $f$ be a nonconstant rational inner function, and write
\begin{equation}
f(z)=\frac{Az^{\beta}\widetilde{Q}(\frac{1}{z})}{Q(z)},\label{E:RI}
\end{equation}
where $Q$ is nonconstant, and  $A,$ $\beta$ and $Q$ have exactly the meanings and properties
stated under the heading ``Fact~\ref{SS:rif}'' above. 
Then $f$ has a zero in ${\mathbb{D}}^n.$ In particular, the numerator of \eqref{E:RI} has a zero in ${\mathbb{D}}^n.$
\end{lemma}

\begin{proof}
Let $d:=\text{deg}(Q).$ Let $Q(z):=\Sigma_{j=0}^{d}\Sigma_{|\alpha|=j} c_{\alpha}z^{\alpha}.$
Define
\[
S_1:=\{z\in\mathbb{T}^n:\Sigma_{|\alpha|=d} c_{\alpha}z^{\alpha}=0\}.
\]
$\mathbb{T}^{n}\setminus{S_1}$ is an open subset of $\mathbb{T}^n$
and has full measure in $\mathbb{T}^n.$
Fix $z\in\mathbb{T}^n\setminus{S_1}$, and write
$z=(e^{i{\theta}_1},e^{i{\theta}_2},\ldots,e^{i{\theta}_n}).$
Then $Q({\zeta} z),$ viewed as a polynomial in ${\zeta}\in{\mathbb{C}},$ has the factorization ($B_{z}$ being independent of $\zeta\in{\mathbb{C}}$)
\begin{equation}
Q({\zeta} z)= B_z\prod_{j=1}^{d}(1-a_{z,j}{\zeta}),\label{E:RIDN}
\end{equation}
where we have used the hypothesis $Z(Q)\cap\mathbb{D}^n=\emptyset.$ Owing to this,
we also have $a_{z,j}\in\overline{\mathbb{D}},\,\,1\leq j\leq d.$
Then from \eqref{E:RI} and \eqref{E:RIDN} we have
\begin{equation}
f({\zeta} z)=AB_ze^{i\langle\beta,{\theta}\rangle}\left[{\zeta}^{|\beta|-d}\prod_{j=1}^{d}
\frac{{\zeta}-\bar{a}_{z,j}}{1-a_{z,j}{\zeta}}\right],\label{E:Bla}
\end{equation}
where ${\theta}:=({\theta}_1,\ldots,{\theta}_n).$
For each $j$ such that $|a_{z,j}|=1,$ we have ${\zeta}-\bar{a}_{z,j}=-\bar{a}_{z,j}(1-a_{z,j}{\zeta}).$
Thus, whenever $|a_{z,j}|=1,$ the associated factor in \eqref{E:Bla} is
understood to be the constant $-\bar{a}_{z,j}.$ Therefore, from \eqref{E:Bla} it is clear that
$f({\zeta} z)$ is a finite Blaschke product of degree at most $|\beta|$ for all
$z\in\mathbb{T}^n\setminus{S_1}.$
\smallskip

\noindent{\bf Claim.} {\em There exists a $z\in\mathbb{T}^n\setminus{S_1}$ such that $f({\zeta} z)$ is
a finite Blaschke product of positive degree.}

\noindent Suppose this is not true, i.e., for each $z\in\mathbb{T}^n\setminus{S_1},$
$f({\zeta} z)$ is a  Blaschke product of degree $0$. It is standard to see that
$\mathbb{T}^n\setminus{S_1}$ contains a compact determining set for polynomials. 
Using a result by Rudin \cite[Theorem~5.2.2]{rudin:ftp69} we get that $f$ is constant, which is a contradiction.
Hence the claim.
\smallskip

Now from the claim and the fact that the range set of a finite Blaschke product of positive degree is 
${\mathbb{D}},$ we know that $0\in \text{Range}(f),$ whence the result.
\end{proof}
\smallskip

We now have all the tools to present the proof of the Proposition \ref{P:factprop}.

\begin{proof}[{\bf Proof of the Proposition \ref{P:factprop}}]
In this proof, all ring-theoretic assertions made {\em without any further qualification will be
for the ring ${\mathbb{C}}[z_1,\ldots,z_n].$}
\smallskip

Write $P(z):=z^{\nu(Q)}\widetilde{Q}(\frac{1}{z}).$
If $Q$ is irreducible then, from Lemma \ref{L:conirr}, $P$ is irreducible. Hence, $P$ and $Q$ are relatively 
prime to each other (since $f$ is nonconstant, $P$ cannot be a scaling of $Q$ owing Lemma \ref{L:nu}).
Hence $(a)$ follows in this case with ${\mathcal{Q}} = Q$.
\smallskip

Next, suppose that $Q$ is reducible and let $Q = \prod_{i=1}^{k}Q_i$ be the unique (up to units) 
factorization of $Q$ into irreducible nonunit factors. Then proceeding as in the proof of Lemma \ref{L:conirr}
leading up to \eqref{E:red1}, we have
\begin{equation}
z^{\nu(Q)}\widetilde{Q}\left(\frac{1}{z}\right)=
\prod_{i=1}^{k} z^{\nu(Q_i)}\widetilde{Q}_{i}\left(\frac{1}{z}\right).\label{E:fact}
\end{equation}
Observe that if the function ${z^{\nu(Q_i)}\widetilde{Q}_i(\frac{1}{z})}/Q_i(z)$ is nonconstant, then
by applying Lemma \ref{L:nu} to the latter function we see that $z^{\nu({Q_i})}\widetilde{Q}_{i}(\frac{1}{z})$
has a zero in ${\mathbb{D}}^n.$ It is therefore nonconstant,
hence a nonunit in ${\mathbb{C}}[z_1,\ldots,z_n].$ However, if ${z^{\nu(Q_i)}\widetilde{Q}_i(\frac{1}{z})}/Q_i(z)$
is a constant function, then, as $Q_i$ itself is nonconstant,
$z^{\nu({Q_i})}\widetilde{Q}_{i}(\frac{1}{z}),$ in this case as well, is nonconstant.
Since $Z(Q)\cap{\mathbb{D}}^n=\emptyset,$ $Q_{i}(0)\not=0$ for each $i.$ 
Thus, by Lemma \ref{L:conirr}, each factor on the right-hand side of \eqref{E:fact} is irreducible.
Hence \eqref{E:fact} gives the unique factorization of $P$. 
\smallskip

Let us define the set
\[
 \mathcal{S}:=\{i\in \{1,\dots, k\}\,:\,{z^{\nu(Q_i)}\widetilde{Q}_i(\tfrac{1}{z})}/Q_i(z) \ \text{is a constant function}\}.
\]
It is elementary to see that if, for each $i\in \mathcal{S}$, we set
\[
 \lambda_i \equiv \frac{z^{\nu(Q_i)}\widetilde{Q}_i(\frac{1}{z})}{Q_i(z)},
\]
then $|\lambda_i| = 1$. Let us now define
\[
 C := \prod_{i\in \mathcal{S}}\lambda_i, \qquad\quad
 {\mathcal{Q}} := \prod_{i\in \{1,\dots, k\}\setminus\mathcal{S}}\!\!Q_i.
\]
The argument that leads to \eqref{E:fact} shows us that
\begin{equation}
 f(z) = C\,\frac{z^{\nu({\mathcal{Q}})}\widetilde{\mathcal{Q}}(\frac{1}{z})}{{\mathcal{Q}}(z)}. \label{E:newform}
\end{equation}
Clearly, $\mathcal{S}\neq \{1,\dots, k\}$, since $f$ is nonconstant. Hence ${\mathcal{Q}}$ is nonconstant. 
\smallskip

We must establish that the numerator and denominator of \eqref{E:newform} do not have any
common factors. To this end, 
write $\mathcal{P}(z):=z^{\nu({\mathcal{Q}})}\widetilde{\mathcal{Q}}(\frac{1}{z}).$  
Every irreducible element in a unique factorization domain is a prime element.
Using this we conclude that if
${{\sf gcd}}(\mathcal{P},{\mathcal{Q}})\not =1$ then there exist $i_0,\,j_0\in (\{1,\dots, k\}\!\setminus\!\mathcal{S})$ such
that 
\begin{equation}
cz^{\nu(Q_{i_0})}\widetilde{Q}_{i_0}\left(\frac{1}{z}\right)=Q_{j_0}(z),\label{E:eqcon}
\end{equation}
where $c$ is a non-zero constant (the argument that follows \eqref{E:fact} establishes that
$z^{\nu(Q_{i_0})}\widetilde{Q}_{i_0}$ is an irreducible factor of $\mathcal{P}$). 
As $i_0\notin \mathcal{S}$, we have seen that
polynomial on the left-hand side  of \eqref{E:eqcon} has a zero in ${\mathbb{D}}^n,$
whence the right-hand side must also have a zero 
in ${\mathbb{D}}^n.$ This implies that $Q$ has a zero in ${\mathbb{D}}^n$, which is a contradiction.
This establishes $(a)$.
\smallskip

Suppose ${\mathcal{Q}}$ is reducible in ${\mathbb{C}}[z_1,\ldots, z_n].$ Then there exist $q_1, q_2\in
{\mathbb{C}}[z_1,\ldots, z_n]$ which are nonunits such that ${\mathcal{Q}}=q_1q_2.$ As $Z({\mathcal{Q}})\cap{\mathbb{D}}^n=\emptyset,$
we have $Z(q_i)\cap{\mathbb{D}}^n=\emptyset,\,\,i=1,2.$ We also have $\nu({\mathcal{Q}})=\nu(q_1)+\nu(q_2).$ Thus,
appealing again to the argument leading up to \eqref{E:red1}, we have
\[
\frac{z^{\nu({\mathcal{Q}})}\widetilde{\mathcal{Q}}(\frac{1}{z})}{{\mathcal{Q}}(z)}=\frac{z^{\nu(q_1)}\widetilde{q}_1(\frac{1}{z})}{q_1(z)}
\frac{z^{\nu(q_2)}\widetilde{q}_2(\frac{1}{z})}{q_2(z)}.
\]
Note that, by our construction of ${\mathcal{Q}}$, we can apply
Lemma \ref{L:nu} to the factors on the right-hand side of the above equation to infer that they are nonunits
of ${\mathcal{O}}({\mathbb{D}}^n)$. These factors are also rational inner. This gives us one of the implications in $(b)$.
\smallskip

Now assume there exist $f_1,f_2,$ rational inner and nonunits in ${\mathcal{O}}({\mathbb{D}}^n)$
such that $f\equiv f_1f_2.$ Owing to Fact~\ref{SS:rif}, we can write
\[
f_i(z) = A_i z^{\beta_i-\nu(Q_i)}\,\frac{z^{\nu(Q_i)}\widetilde{Q}_i(\frac{1}{z})}{Q_i(z)}, \ \ \ 
i=1, 2,
\]
where $A_i,$ $\beta_i$ and $Q_i$ have the meanings and properties
stated in Fact~\ref{SS:rif}. In view of $(a)$, we can assume without loss of generality that
the numerator and the denominator of the right-hand side of the above expression do not
have any common (nonconstant) factors (note that we do {\bf not} require $Q_i$ to be
nonconstant to assert this). This assumption will be in effect for the remainder of this proof.
\smallskip

Put $P_i(z)= A_i z^{\beta_i}\widetilde{Q}_i(\frac{1}{z})$, $i = 1,2.$
Appealing to Lemma~\ref{L:nu} if $Q_i$ is nonconstant, else to the fact that $f_i$ is nonconstant,
we deduce that $P_1$ and $P_2$ have zeros in ${\mathbb{D}}^n$. 
Let $C$ and $\mathcal{P}$ be as in \eqref{E:newform}. We have
\begin{equation}
C\,\frac{\mathcal{P}}{\mathcal{Q}}=\frac{P_1P_2}{Q_1Q_2}=\frac{p_1p_2}{q_1q_2}, \label{maineqid}
\end{equation}
where $p_1$ and $q_2$ are obtained by cancelling any common factors that $P_1$ and $Q_2$ might
have; and defining the pair $p_2$ and $q_1$ analogously.  
We observe, at this point, that any such nonconstant common factor cannot vanish in ${\mathbb{D}}^n.$ Hence,
$p_1$ and $p_2$ must have zeros in ${\mathbb{D}}^n$ and are nonunits in ${\mathbb{C}}[z_1,\ldots,z_n].$  
Now \eqref{maineqid} gives us
\begin{equation*}
C\mathcal{P}q_1q_2 = {\mathcal{Q}} p_1p_2.
\end{equation*}
Hence $p_1p_2|\mathcal{P}q_1q_2.$ As ${{\sf gcd}}(p_1p_2, q_1q_2)=1,$ we have $p_1p_2|\mathcal{P},$
whence $\mathcal{P}$ is reducible. Hence from Lemma \ref{L:conirr}, ${\mathcal{Q}}$ is reducible. This establishes $(b)$.
\end{proof}
\medskip

\section{The proof of Theorem~\ref{T:mainThm}}\label{S:Thm}

In this section we present the proof of Theorem~\ref{T:mainThm}. Before that we
need to state a result about the positivity of certain quadratic forms. The proof of the
result is found in Garnett \cite[Theorem~2.2]{garnett:ftp07}. Here, given $a\in{\mathbb{D}},$ $\psi_a$
is as described in Section~\ref{S:intro}.

\begin{result}\label{R:quadposi}
Let $\{(a_j,b_j)\in{\mathbb{D}}\times{\mathbb{D}}: 1\leq j\leq n\},$ where $a_j$'s are distinct.
Let $a_j'=\psi_{a_n}(a_j)$ and $b_j'=\psi_{b_n}(b_j),\,\,1\leq j\leq n.$
Consider the quadratic form:
\begin{equation*}
Q_n(t_1,t_2,\ldots,t_n):=\sum_{j,k=1}^{n}\frac{1-b_j\bar{b}_k}{1-a_j\bar{a}_k}t_j\bar{t}_k.
\end{equation*}
Let $Q_n'$ be the quadratic form obtained from $Q_n$ by replacing $a_j$ with $a_j'$ and $b_j$ with $b_j'.$ Then
\begin{equation*}
Q_n\geq 0 \iff Q_n'\geq 0.
\end{equation*} 
Moreover if we take $a_n=0=b_n$ in $Q_n$, and consider the quadratic form:
\begin{equation*}
\widetilde{Q}_{n-1}(s_1,s_2,\ldots,s_{n-1})=\sum_{j,k=1}^{n-1}\frac{1-({b_j}/{a_j})(\overline{{b_k}/{a_k}})}{1-a_j\bar{a}_k}
s_j\bar{s}_k,
\end{equation*}
then
\begin{equation*}
Q_n\geq 0 \iff \widetilde{Q}_{n-1}\geq 0 \\\ \text{(taking $a_n=0=b_n$ in $Q_n$)}.
\end{equation*}
\end{result}
\smallskip

In the remainder of this section, we will use expressions of the form
``a function that interpolates the data $(X_1,\ldots,X_N;w_1,\ldots,w_N)$''
to signify the existence of a function, in the stated class, that maps the data in the manner described by $(*).$
The following lemma is also a key tool in our proof of Theorem~\ref{T:mainThm}.
\begin{lemma}\label{L:twointer}
Let $(X_1,w_1),(X_2,w_2)\in{\mathbb{D}}^n\times{\mathbb{D}}.$ There exists a holomorphic map in ${\mathcal{O}}({\mathbb{D}}^n,{\mathbb{D}})$
interpolating the data $(X_1,X_2;w_1,w_2)$ if and only if
\begin{equation*}
C_{{\mathbb{D}}^n}(X_1,X_2)\geq C_{\mathbb{D}}(w_1,w_2),
\end{equation*}
where $C_{{\mathbb{D}}^n}$ and $C_{\mathbb{D}}$ denote the Carath{\'e}odory distance on ${\mathbb{D}}^n$ and ${\mathbb{D}}$ respectively.
\end{lemma}
\smallskip

The above lemma is a standard application of the Schwarz lemma.
We now have all the tools to present the proof of Theorem~\ref{T:mainThm}.

\begin{proof}[{\bf Proof of Theorem~\ref{T:mainThm}}] Let $F\in{\mathcal{O}}({\mathbb{D}}^n)$ denote a rational inner function
(if it exists) that interpolates the data $(X_1,X_2,X_3;w_1,w_2,w_3).$ Let $\Psi_{X_3}
\in\text{Aut}({\mathbb{D}}^n)$ be defined as $\Psi_{X_3}\equiv(\psi_{X_{3,1}},\ldots,\psi_{X_{3,n}}),$
where we write $X_3:=(X_{3,1},\ldots,X_{3,n}).$ Then the interpolant $F$ exists if and only if
$\widetilde{F}:=\psi_{w_3}\circ F\circ{\Psi_{X_3}^{-1}},$ which is a rational inner function on ${\mathbb{D}}^n,$
interpolates the data $(X_1',X_2',0;w_1',w_2',0),$ where $X_1',X_2',w_1'$ and $w_2'$ are as stated in the theorem.
\smallskip

\noindent{\bf Claim.} {\em The interpolant $\widetilde{F}$ exists if and only if there exist $H,G,$
both rational inner functions on ${\mathbb{D}}^n,$ with $H$ having the form described in Theorem~\ref{T:mainThm}, 
such that $G$ interpolates $(X_1',X_2';{w_1'}/{H(X_1')},{w_2'}/{H(X_2')}),$ and such that 
${w_j'}/{H(X_j')}\in\overline{\mathbb{D}}$ for $j=1,2.$}

\noindent The ``if'' part of the above claim is easy to prove. Assume that $G,H$ exist as in the claim. Then take
$\widetilde{F}=GH,$ which has all the desired properties.
\smallskip

To see the ``only if'' part we consider two cases.
In what follows, the adjectives {\em irreducible} and {\em reducible},
applied to $\widetilde{F},$ are as defined prior to Corollary \ref{C:irdefpol}.
\smallskip

\noindent{\bf Case 1.} {\em The interpolant $\widetilde{F}$ is irreducible.}
 
\noindent In this case we take $H=\widetilde{F}$ and $G\equiv 1.$
Note that both are rational inner functions. That $H$ has the form described in Theorem~\ref{T:mainThm}
follows from Corollary \ref{C:irdefpol}.
\smallskip

\noindent{\bf Case 2.} {\em $\widetilde{F}$ is reducible.}
 
\noindent Since $\widetilde{F}$ is reducible, and $\widetilde{F}(0)=0,$ there exist an irreducible rational inner function
$H$ such that $H(0)=0,$ and a rational inner function $G$ such that $\widetilde{F}=GH.$ In view of
Corollary~\ref{C:irdefpol}, $G$ and $H$ have the properties claimed. 
\smallskip

This establishes our Claim.
\smallskip

Let us look closely at the situation in Case~2. Since $X_j'\in{\mathbb{D}}^n$ for $j=1,2,$
we have $|{w_j'}/{H(X_j')}|=|G(X_j')|<1,\ j=1,2.$
We have used here the fact that $G$ is nonconstant. We have from Lemma~\ref{L:twointer} that the 
existence of $G$ and $H$ as in our Claim leads to
\begin{equation}
C_{{\mathbb{D}}^n}(X_1',X_2')\geq C_{\mathbb{D}}\left(\frac{w_1'}{H(X_1')},\frac{w_2'}{H(X_2')}\right). \label{E:card}
\end{equation}
As $C_{{\mathbb{D}}^n}(X_1',X_2')=\max\{C_{\mathbb{D}}(X_{1,\,j}',X_{2,\,j}'):1\leq j\leq n\},$
the inequality \eqref{E:card} is equivalent to 
\begin{equation}
C_{\mathbb{D}}(X_{1,\,l}',X_{2,\,l}')\geq C_{\mathbb{D}}\left(\frac{w_1'}{H(X_1')},\frac{w_2'}{H(X_2')}\right)
\ \ \ \text{for some $l,1\leq l\leq n.$}\label{E:cardco}
\end{equation}
Writing the expression for $C_{\mathbb{D}},$ a simple matricial trick (see \cite[page 7]{garnett:ftp07})
shows that the inequality \eqref{E:cardco} is equivalent to
\begin{equation}
{\begin{bmatrix}
\dfrac{1-({w_j'}/{H(X_j')})(\overline{{w_k'}/{H(X_k')}})}{1-X_{j,\,l}'\overline{X'}_{k,\,l}}
\end{bmatrix}}_{j,k=1}^{2}
\geq 0.\label{E:mainm}
\end{equation}
\smallskip

The interpolation criterion in Theorem \ref{T:mainThm} is stated in terms of a quadratic form because \eqref{E:card} does not
make sense in Case~1. In Case~1 the existence of the interpolant $\widetilde{F}$ implies that the interpolant $G$ is 
the constant $1,$ whence $w_j'/H(X_j')=1,\ j=1,2.$ Trivially, the matrix in \eqref{E:mainm} is positive semi-definite.
This completes the proof of the ``only if'' part of the theorem.
\smallskip

Let us denote the matrix in \eqref{E:mainm} by $M_l.$ In view of the chain of equivalences discussed above,
we would be done if we can produce a rational inner function $G$ with the properties stated in our Claim.
So we assume that $M_l$ is positive semi-definite (which tacitly assumes the existence of the function $H$
with the properties stated above). It is a classical fact (see \cite[Theorem 6.15]{agmac:ftp02}, for instance) 
that there exists a finite Blaschke product $B$ (which includes the case when $B$ is a unimodular constant)
that interpolates the data $(X_{1,\,l}',X_{2,\,l}';w_1'/H(X_1'),w_2'/H(X_2')).$ Take $G=B\circ \pi_l,$ where $\pi_l$ 
denotes the projection onto the $l$-th coordinate. This $G$ satisfies all the properties as in the above Claim.
\smallskip

Suppose, now, that the condition in Theorem~\ref{T:mainThm} holds true. Then it is easy to see that the matrix in 
\eqref{E:mainm} is the zero matrix if and only if $w_1'/H(X_1')=w_2'/H(X_2')=c\in\partial{\mathbb{D}}.$ It follows from
the discussion in the previous paragraph that
$\widetilde{F}=\psi_{w_3}\circ F\circ\Psi^{-1}_{X_3}=cH,$ and $(a)$ follows from this. When the rank $r\geq 1$,
we refer to the full force of \cite[Theorem 6.15]{agmac:ftp02}: this gives the degree of the Blaschke product $B$
mentioned in the previous paragraph. 
Arguing as before,  $\widetilde{F}=(B\circ\pi_l)H,$ and we are done.
\end{proof}
\medskip

\begin{remark}
Using Result \ref{R:quadposi} it is possible to replace the positive semi-definiteness of the matrix in \eqref{E:mainm} by
the positive semi-definiteness of a $3\times 3$ matrix where, in the denominator of each entry $X_{j,\,l}, \ j=1, 2, 3,$ appear.
We also observe that the proof of our main theorem, together with the above discussion that ``inflates'' condition \eqref{E:mainm}
into a condition on a $3\times 3$ matrix, suggests a generalization of the main theorem for $N$ points, $N\geq3,$
involving the positivity of an $N\times N$ matrix. We would proceed by deflating the   
$N$-point data set to an equivalent $(N-1)$-point data set, which sets up an inductive scheme as
in the Schur algorithm. But this results in a condition that is too unwieldy to be useful.
We will not present this generalization here as the associated technicalities only obscure the main idea underlying this work.

\end{remark}
\medskip

\noindent{{\bf Acknowledgements.} The author wishes to thank Gautam Bharali for the many useful discussions during
the course of this work.}

\begin{thebibliography}{1}
\bibitem{ag:ftp88} Jim Agler, Interpolation, unpublished manuscript, 1988.

\bibitem{agmac:ftp02} Jim Agler $\&$ John E. McCarthy,\,Pick Interpolation and Hilbert Function Spaces,
\,American Mathematical Society, 2002.

\bibitem{agmac:ftp99} Jim Agler $\&$ John E. McCarthy, {\em Nevanlinna-Pick interpolation on the bidisk}, J. Reine Angew. Math.
{\bf 506} (1999), 191-204.

\bibitem{ando:ftp63} T. And\^{o}, {\em On a pair of commutative contractions}, Acta Sci. Math. (Szeged) {\bf 24} (1963), 88-90.

\bibitem{cw:ftp94} B.J. Cole and J. Wermer, {\em Pick interpolation, von Neumann inequalities, and hyperconvex sets},
Complex Potential Theory, Kluwer Acad. Publ., Dordrecht, 1994, 89-129.

\bibitem{ana:ftp012} Anatolii Grinshpan, Dmitry S. Kaliuzhnyi--Verbovetskyi, Hugo J. Woerdeman,
{\em Norm-constrained determinantal representations of multivariable polynomials}, Complex Anal. Oper. Theory 7 (2013),
no. 3, 635-654.

\bibitem{garnett:ftp07} John B. Garnett,\,Bounded Analytic Functions,\,Springer Verlag, 2007.

\bibitem{ham:ftp013} R. Hamilton, {\em Pick interpolation in several variables}, Proc. Amer. Math. Soc. {\bf 141} (2013),
no. 6, 2097-2103.

\bibitem{rudin:ftp69} Walter Rudin,\,Function Theory in Polydiscs,\,W.A. Benjamin, 1969.

\bibitem{sarason:ftp67} D. Sarason, {\em Generalized interpolation in $H^{\infty}$}, Trans. Amer. Math. Soc.
{\bf 127} (1967), 179-203.

\end{thebibliography}

\end{document}

