
\usepackage{amsmath,amsthm,amsfonts,latexsym,amssymb,amscd,color}
\usepackage{chemarr}
\pagestyle{headings}
\setlength{\textwidth}{36true pc}
\setlength{\headheight}{8true pt} 
\setlength{\oddsidemargin}{0 truept}
\setlength{\evensidemargin}{0 truept}
\setlength{\textheight}{572true pt}
\newtheorem{thm}{Theorem}[section] 
\newtheorem{cor}[thm]{Corollary}
\newtheorem{lm}[thm]{Lemma}
\newtheorem{prp}[thm]{Proposition}
\newtheorem{clm}[thm]{Claim}
\newtheorem*{clm*}{Claim}
\theoremstyle{definition}
\newtheorem{df}[thm]{Definition}
\newtheorem{dfs}[thm]{Definitions}
\newtheorem{exmp}[thm]{Example}
\newtheorem{exmps}[thm]{Examples}
\newtheorem{remark}[thm]{Remark}
\newtheorem{prb}[thm]{Problem}
\newtheorem{quest}[thm]{Question}
\numberwithin{equation}{section}

 

\newenvironment{spf}{\noindent{\it Proof.}}{\hfill\rule{1.3mm}{3mm}\medskip}
 

\newenvironment{cpf}{\noindent{\it Proof of Claim.}\ }{\hfill\rule{1.3mm}{3mm}}

 
                                           
 
                                           
  
                                           
 
     

                                           

                                           

 
 
 
 
 
         
         

  
   
 

 
  
     

 

 
 

 

  

\begin{document}

\title{Varieties whose finitely generated members are free}

\author{Keith A. Kearnes}
\address[Keith Kearnes]{Department of Mathematics\\
University of Colorado\\
Boulder, CO 80309-0395\\
USA}
\email{keith.kearnes@colorado.edu}
\author{Emil W. Kiss}
\address[Emil W. Kiss]{
Lor\'{a}nd E{\"o}tv{\"o}s University\\
Department of Algebra and Number Theory\\
H--1117 Budapest, P\'{a}zm\'{a}ny P\'{e}ter s\'{e}t\'{a}ny 1/c.\\
Hungary}
\email{ewkiss@cs.elte.hu}
\author{\'Agnes Szendrei}
\address[\'Agnes Szendrei]{Department of Mathematics\\
University of Colorado\\
Boulder, CO 80309-0395\\
USA}
\email{agnes.szendrei@colorado.edu}
\thanks{This material is based upon work supported by
the Hungarian National Foundation for Scientific Research (OTKA)
grant no.\ K77409, K83219, and K104251.
}
\subjclass{08B20 (08A05, 03C35)}
\keywords{}

\begin{abstract}
  We prove that a
  variety of algebras whose finitely
  generated members are free
  must be
  
  definitionally equivalent to
  the variety of sets, the variety of pointed sets, a variety of vector spaces
  over a division ring, or a variety of affine vector spaces over a
  division ring.
\end{abstract}

\maketitle

\section{Introduction}\label{intro_sec}
In this paper we address a MathOverflow question, \cite{campion},
which asks for a description of the varieties
where every algebra is free, as well as a description of the varieties
satisfying the weaker requirement that
every finitely generated algebra is free.

Steven Givant classified the varieties where every algebra is free
in \cite{givant}.
He proved that they
are precisely those definitionally equivalent to
\begin{itemize}
\item the variety of sets,
\item the variety of pointed sets,
\item a variety of vector spaces over a division ring, or
\item a variety of affine spaces over a division ring.
\end{itemize}
In this paper we use different techniques
to classify the varieties where every
finitely generated algebra is free.
Our result is that if the finitely generated members of
$\mathcal V$ are free, then $\mathcal V$
must also be one of these types of varieties
(sets, pointed sets, vector spaces or affine spaces).
Hence, if the finitely generated algebras
in $\mathcal V$ are free, then all algebras in $\mathcal V$ are free.

We conclude the paper by pointing out that
for each positive integer $n$ there are varieties
whose $(n+1)$-generated algebras are not all free, but whose 
members generated by at most $n$ elements are free.

\section{Abelian and affine algebras}
Please refer to \cite{freese-mckenzie, hobby-mckenzie, kearnes-kiss}
for elaboration of the introductory remarks of this section.

An algebra ${{\mathbf{\uppercase{a}}}}$
is \emph{abelian} if it satisfies the \emph{term condition},
which is the assertion that if $t({{\mathbf{{x}}}},{{\mathbf{{y}}}})$ is a term
in the language, ${{\mathbf{{a}}}}, {{\mathbf{{b}}}}, {{\mathbf{{u}}}}$ and ${{\mathbf{{v}}}}$
are tuples of elements of $A$, and
\[
t^{{{\mathbf{\uppercase{a}}}}}(\underline{{{\mathbf{{a}}}}},{{\mathbf{{u}}}}) = t^{{{\mathbf{\uppercase{a}}}}}(\underline{{{\mathbf{{a}}}}},{{\mathbf{{v}}}}), 
\]
then $t^{{{\mathbf{\uppercase{a}}}}}(\underline{{{\mathbf{{b}}}}},{{\mathbf{{u}}}}) =
t^{{{\mathbf{\uppercase{a}}}}}(\underline{{{\mathbf{{b}}}}},{{\mathbf{{v}}}})$. This property
is the same as the property that the diagonal of ${{\mathbf{\uppercase{a}}}}\times {{\mathbf{\uppercase{a}}}}$
is the class of a congruence.

An algebra ${{\mathbf{\uppercase{b}}}}$ is \emph{affine} if it is polynomially equivalent
to a module. This means that there is a ring $R$ and a left
module structure on the universe $B$ of ${{\mathbf{\uppercase{b}}}}$ such that
the polynomial operations of ${{\mathbf{\uppercase{b}}}}$ coincide with the
${}_RB$-module polynomial operations. (A polynomial
operation of an algebra ${{\mathbf{\uppercase{b}}}}$ is an operation
obtained from a term operation by substituting constants for
some of the variables, i.e. $p({{\mathbf{{x}}}}) = t^{{{\mathbf{\uppercase{a}}}}}({{\mathbf{{x}}}},{{\mathbf{{a}}}})$.)

A variety is abelian or affine if its members are.
It is a fact that affine algebras and varieties are abelian,
but the converse is false, e.g. unary varieties are abelian
but not affine.

Abelian varieties that are not affine are poorly understand
at present. If $\mathcal V$ is a \underline{locally finite} variety that 
is abelian but not affine, then it can be proved that $\mathcal V$
contains a very ``bad'' or ``structureless'' algebra, i.e.
one that is a definitionally equivalent
to a matrix power of a nontrivial set or a pointed set.
The procedure for proving this is to first exploit
the nonaffineness assumption to construct
a finite ``strongly abelian'' algebra ${{\mathbf{\uppercase{s}}}}\in \mathcal V$,
and then to examine a minimal subvariety of $\operatorname{\mathsf{H}}\operatorname{\mathsf{S}}\operatorname{\mathsf{P}}({{\mathbf{\uppercase{s}}}})$.
The structure of such minimal subvarieties are determined
by the classification theorem for minimal abelian varieties,
which can be found in 
\cite{kkv1} and \cite{szendrei}. Namely, a minimal subvariety
of a variety generated by a finite strongly abelian algebra
is definitionally
equivalent to a matrix power of the variety of sets
or the variety of pointed sets.

These arguments fail at the very first step
for nonlocally finite varieties: it is not known of the
algebra ${{\mathbf{\uppercase{s}}}}$ is strongly abelian. In this section we examine
the construction of ${{\mathbf{\uppercase{s}}}}$
and identify some ``strongly abelian--like''
properties of ${{\mathbf{\uppercase{s}}}}$.

First, a congruence $\theta\in\operatorname{Con}({{\mathbf{\uppercase{a}}}})$ is \emph{strongly abelian}
if it satisfies the strong term condition,
which is the assertion that if $t({{\mathbf{{x}}}},{{\mathbf{{y}}}})$ is a term
in the language, ${{\mathbf{{a}}}}, {{\mathbf{{b}}}}, {{\mathbf{{u}}}}, {{\mathbf{{v}}}}, {{\mathbf{{w}}}}$
are tuples of elements of $A$ with ${{\mathbf{{a}}}}$ and ${{\mathbf{{b}}}}$
$\theta$-related coordinatewise and
${{\mathbf{{u}}}}, {{\mathbf{{v}}}}, {{\mathbf{{w}}}}$
$\theta$-related coordinatewise, and
\[
t^{{{\mathbf{\uppercase{a}}}}}({{\mathbf{{a}}}},\underline{{{\mathbf{{u}}}}}) = t^{{{\mathbf{\uppercase{a}}}}}({{\mathbf{{b}}}},\underline{{{\mathbf{{v}}}}}), 
\]
then
$t^{{{\mathbf{\uppercase{a}}}}}({{\mathbf{{a}}}},\underline{{{\mathbf{{w}}}}}) = t^{{{\mathbf{\uppercase{a}}}}}({{\mathbf{{b}}}},\underline{{{\mathbf{{w}}}}})$.

Now suppose that ${{\mathbf{\uppercase{a}}}}$ is abelian
and $\theta\in\operatorname{Con}({{\mathbf{\uppercase{a}}}})$ is strongly abelian.
The construction we are concerned with is the following one:
Let ${{\mathbf{\uppercase{a}}}}(\theta)$ be the subalgebra of ${{\mathbf{\uppercase{a}}}}\times {{\mathbf{\uppercase{a}}}}$
supported by (the graph of) $\theta$.
Let $\Delta$ be the congruence on ${{\mathbf{\uppercase{a}}}}(\theta)$ 
generated by $D\times D$ where $D = \{(a,a)\;|\;a\in A\}$
is the diagonal. 
$D$ is a $\Delta$-class, because ${{\mathbf{\uppercase{a}}}}$ is abelian.
Let ${{\mathbf{\uppercase{s}}}} = {{\mathbf{\uppercase{s}}}}_{{{\mathbf{\uppercase{a}}}},\theta} := {{\mathbf{\uppercase{a}}}}(\theta)/\Delta$.
Let $0 = D/\Delta\in S$.

If ${{\mathbf{\uppercase{a}}}}$ was a finite member of an abelian variety,
then one could prove that the resulting algebra
${{\mathbf{\uppercase{s}}}}$ would be a strongly abelian
member of the variety (meaning that
all of its congruences would be strongly
abelian). Without finiteness we do not know how to prove this.
However, we can prove the following.

\begin{lm}\label{nonaffine}
  Let $\mathcal V$ be an abelian variety, and suppose that
  $\theta$ is a nontrivial strongly abelian congruence on some
  ${{\mathbf{\uppercase{a}}}}\in\mathcal V$. Let ${{\mathbf{\uppercase{s}}}} = {{\mathbf{\uppercase{s}}}}_{{{\mathbf{\uppercase{a}}}},\theta}$
  and let $0 = D/\Delta\in S$. The following are true:
  \begin{enumerate}
    \item ${{\mathbf{\uppercase{s}}}}$ has more than one element.
  \item $\{0\}$ is a 1-element subuniverse of ${{\mathbf{\uppercase{s}}}}$.
  \item ${{\mathbf{\uppercase{s}}}}$ has ``Property P'': for every $n$-ary polynomial
    $p({{\mathbf{{x}}}})$ of ${{\mathbf{\uppercase{s}}}}$ and every tuple ${{\mathbf{{s}}}}\in S^n$
    \[
p({{\mathbf{{s}}}})=0\quad\textrm{implies}\quad p({{\mathbf{{0}}}})=0,
\]
where ${{\mathbf{{0}}}} = (0,0,\ldots,0)$.
  \item Whenever $t(x_1,\ldots,x_n)$, 
    is a $\mathcal V$-term, and
    \[
    \mathcal V\models t({{\mathbf{{x}}}})=
    t({{\mathbf{{y}}}})
    \]
    where ${{\mathbf{{x}}}}$ and ${{\mathbf{{y}}}}$ are tuples of not necessarily
    distinct variables which differ in the $i$th position,
    then the term operation $t^{{{\mathbf{\uppercase{s}}}}}(x_1,\ldots,x_n)$ is
    independent of its $i$th variable.
  \item ${{\mathbf{\uppercase{s}}}}$ has a congruence $\sigma$ such that the
    algebra ${{\mathbf{\uppercase{s}}}}/\sigma$ satisfies (1)--(4) of this lemma,
    and ${{\mathbf{\uppercase{s}}}}/\sigma$ also
    has a compatible partial order $\leq$ such that
$0\leq s$ for every $s\in (S/\sigma)$.
    \end{enumerate}
  \end{lm}

\begin{proof}
  
  [Item (1)]
    Since ${{\mathbf{\uppercase{a}}}}$ is abelian, the diagonal 
  $D$ is the class of a congruence on ${{\mathbf{\uppercase{a}}}}\times {{\mathbf{\uppercase{a}}}}$,
  namely the congruence generated by $D\times D$. 
  This congruence restricts to ${{\mathbf{\uppercase{a}}}}(\theta)$ to have $D$ as a class.
  Since $\theta$ is nontrivial, it properly contains $D$,
  so the congruence $\Delta$ of
  ${{\mathbf{\uppercase{a}}}}(\theta)$ generated by $D\times D$ is proper. Equivalently, 
  ${{\mathbf{\uppercase{s}}}} = {{\mathbf{\uppercase{a}}}}(\theta)/\Delta$ is nontrivial.
  
\bigskip

  [Item (2)]
Since $D$ is a subuniverse of ${{\mathbf{\uppercase{a}}}}(\theta)$, $\{D/\Delta\}=\{0\}$
is a subuniverse of ${{\mathbf{\uppercase{s}}}}$.  

\bigskip

[Item (3)] To show that ${{\mathbf{\uppercase{s}}}}$ has Property P,
choose ${{\mathbf{{s}}}}\in S^n$ and $p({{\mathbf{{x}}}})$ such that $p({{\mathbf{{s}}}})=0$. Our goal
is to show that $p({{\mathbf{{0}}}}) = 0$.

Express $p({{\mathbf{{x}}}})$ as $t^{{{\mathbf{\uppercase{s}}}}}({{\mathbf{{x}}}},{{\mathbf{{u}}}})$
for some tuple ${{\mathbf{{u}}}}$ with coordinates in  $S$.
Also, express $s_i$ and $u_j$ 
as $s_i = (a_i,b_i)/\Delta$ and $u_j = (v_j,w_j)/\Delta$ where
$(a_i,b_i), (v_i,w_i)\in\theta$. Then $p({{\mathbf{{s}}}})=0$ may be expressed as
$t^{{{\mathbf{\uppercase{a}}}}(\theta)}(({{\mathbf{{a}}}},{{\mathbf{{b}}}}),({{\mathbf{{v}}}},{{\mathbf{{w}}}}))\in D$,
or 
\[
t^{{{\mathbf{\uppercase{a}}}}}(\underline{{{\mathbf{{a}}}}},{{\mathbf{{v}}}})=t^{{{\mathbf{\uppercase{a}}}}}(\underline{{{\mathbf{{b}}}}},{{\mathbf{{w}}}}).
\]
By the strong term condition we derive that
$t^{{{\mathbf{\uppercase{a}}}}}(\underline{{{\mathbf{{a}}}}},{{\mathbf{{v}}}})=t^{{{\mathbf{\uppercase{a}}}}}(\underline{{{\mathbf{{a}}}}},{{\mathbf{{w}}}})$
holds,
which may be expressed as
$t^{{{\mathbf{\uppercase{a}}}}(\theta)}(({{\mathbf{{a}}}},{{\mathbf{{a}}}}),({{\mathbf{{v}}}},{{\mathbf{{w}}}}))\in D$,
or $p({{\mathbf{{0}}}})=p(({{\mathbf{{a}}}},{{\mathbf{{a}}}})/\Delta)=0$.

\bigskip

[Item (4)]
Assume for simplicity that $i=1$ in the statement of (4),
that is ${\mathcal V}\models t(x,{{\mathbf{{w}}}})=t(y,{{\mathbf{{z}}}})$.
By specializing if necessary we may assume further
that $w_j, z_j\in\{x,y\}$ for all $j$. Our goal is to show that
$t^{{{\mathbf{\uppercase{a}}}}}(x_1,\ldots,x_n)$ is independent of its first variable.

\begin{clm}
  For any $s\in S$, $t^{{{\mathbf{\uppercase{s}}}}}(s,0,0,\ldots,0)=0$.
\end{clm}

{\noindent{\it Proof of Claim.}\ }
The identity $t(x,{{\mathbf{{w}}}})=t(y,{{\mathbf{{z}}}})$ may be written
symbolically as
\[t((x,y),({{\mathbf{{w}}}},{{\mathbf{{z}}}}))\in D,\]
where
$(x,y)$ and each $(w_j,z_j)$ belong to the set
$\{(x,y), (y,x), (x,x), (y,y)\}$.

Choose $s\in S$ and represent it as
$s = (a,b)/\Delta$ for some pair $(a,b)\in\theta$.
Each of the pairs $(a,b), (b,a), (a,a), (b,b)$
belongs to $\theta$, so we may substitute $a$'s and $b$'s
for $x$'s and $y$'s to obtain that
\[
t^{{{\mathbf{\uppercase{a}}}}(\theta)}((a,b),({{\mathbf{{c}}}},{{\mathbf{{d}}}}))\in D,
\]
where each $(c_j,d_j)$ is one of the elements of
$\{(a,b), (b,a), (a,a), (b,b)\}$
Factoring by $\Delta$ yields
\begin{equation}\label{weird}
t^{{{\mathbf{\uppercase{s}}}}}((a,b)/\Delta,({{\mathbf{{c}}}},{{\mathbf{{d}}}})/\Delta) = 
t^{{{\mathbf{\uppercase{s}}}}}(s,\underline{({{\mathbf{{c}}}},{{\mathbf{{d}}}})/\Delta}) = 0.
\end{equation}

We apply Property P to (\ref{weird}) to change
the underlined values to $0$. We obtain that
$t^{{{\mathbf{\uppercase{s}}}}}(s,{{\mathbf{{0}}}})=0$, as desired.
{\hfill\rule{1.3mm}{3mm}}

\bigskip

Now, for arbitrary $s\in S$, apply the term condition to
\[
t^{{{\mathbf{\uppercase{s}}}}}(s,\underline{{{\mathbf{{0}}}}}) = t^{{{\mathbf{\uppercase{s}}}}}(0,\underline{{{\mathbf{{0}}}}}) =0
\]
to obtain
$t^{{{\mathbf{\uppercase{s}}}}}(s,\underline{{{\mathbf{{u}}}}}) = t^{{{\mathbf{\uppercase{s}}}}}(0,\underline{{{\mathbf{{u}}}}})$
for any ${{\mathbf{{u}}}}$. This is what it means for $t^{{{\mathbf{\uppercase{s}}}}}(x_1,\ldots,x_n)$
to be independent of its first variable.

\bigskip

[Item (5)]
Let $R$ be the reflexive compatible relation on ${{\mathbf{\uppercase{s}}}}$
generated by $\{0\}\times S$. Property P asserts exactly that
$(x,0)\in R$ implies $x=0$. The transitive closure $R^*$ of $R$
also has this property. The symmetrization $\sigma:= R^*\cap (R^*)^{\cup}$
is a congruence on ${{\mathbf{\uppercase{s}}}}$ and $\leq := R^*/\sigma$ is a compatible
partial order on the quotient. This partial order
contains $(\{0\}\times S)/(\sigma\times\sigma)$, so
$0\leq s$ for every $s\in (S/\sigma)$.

Note that ${{\mathbf{\uppercase{s}}}}/\sigma$ satisfies all of the earlier properties:
(1)
The quotient ${{\mathbf{\uppercase{s}}}}/\sigma$ is nontrivial, since
${{\mathbf{\uppercase{s}}}}$ is nontrivial and $\{0\}$ is a singleton class
of $\sigma$. (2)
$\{0\}/\sigma$ is a singleton subuniverse of the quotient.
(3) Property P is easily derivable from a
lower bounded compatible order:
$0\leq p({{\mathbf{{0}}}})\leq p({{\mathbf{{s}}}})$ for any ${{\mathbf{{s}}}}$,
so $p({{\mathbf{{s}}}})=0$ implies $p({{\mathbf{{0}}}})=0$.
(4) The assumption and conclusion of part (4) of the
Lemma statement are preserved when taking quotients.
\end{proof}

The properties that have been proved for ${{\mathbf{\uppercase{s}}}} = {{\mathbf{\uppercase{s}}}}_{{{\mathbf{\uppercase{a}}}},\theta}$
and its quotient ${{\mathbf{\uppercase{s}}}}/\sigma$
prevent $\mathcal V$ from being affine. For example, no nontrivial
affine algebra can satisfy Property P:
let $p(x) = x-s$ for some $s\in S\setminus\{0\}$.
Then $p(s)=0$ while $p(0)\neq 0$. In fact,
this polynomial has no fixed points at all.

Similarly, an affine algebra has no compatible reflexive relations
other than equivalent relations. If the compatible partial order
in (5) was an equivalence relation, then it would be discrete.
For the discrete order to have a least element $0$, the
underlying set could have only one element, contrary to item (1).

Also, it is not hard to show that any variety 
containing an algebra ${{\mathbf{\uppercase{s}}}}/\sigma$ satisfying
the property described in item (4) cannot satisfy 
any nontrivial idempotent Maltsev condition, while
affine varieties satisfy strong idempotent Maltsev conditions.

We will call an algebra of the form ${{\mathbf{\uppercase{s}}}} = {{\mathbf{\uppercase{s}}}}_{{{\mathbf{\uppercase{a}}}},\theta}$
an \emph{affine obstruction}.

\begin{thm}\label{affine}
  The following are equivalent for an abelian variety
  $\mathcal V$.
  \begin{enumerate}
  \item $\mathcal V$ is not affine.
  \item $\mathcal V$ satisfies no nontrivial idempotent Maltsev condition.
  \item $\mathcal V$ contains an algebra ${{\mathbf{\uppercase{a}}}}$ that has a nonzero
    strongly abelian congruence, $\theta$.
  \item $\mathcal V$ contains an affine obstruction.
    \end{enumerate}
  \end{thm}

\begin{proof}
  [$(1)\Rightarrow(2)$] (First proof.)
  We argue the contrapositive, so assume that $\mathcal V$ satisfies
  a nontrivial idempotent Maltsev condition.
  By Theorem 4.16~(2) of \cite{kearnes-kiss}, congruence
  lattices of algebras
  in $\mathcal V$ omit pentagons with certain specified
  abelian intervals. Since $\mathcal V$ is abelian,
  all intervals in congruence lattices of members are abelian.
  Hence there are no pentagons in congruence lattices of
  members of $\mathcal V$, which means that $\mathcal V$ is congruence
  modular. In this context it is known that
  abelian varieties are affine (see \cite{freese-mckenzie}).

[$(1)\Rightarrow(2)$] (Second proof.)
  Again we argue the contrapositive, so assume that $\mathcal V$ satisfies
  a nontrivial idempotent Maltsev condition.
  By Theorem 3.21 of \cite{kearnes-kiss}, $\mathcal V$ has a join term.
  It acts as a semilattice operation on blocks of any rectangular tolerance
  of an algebra in $\mathcal V$. Since there are no
  nontrivial abelian semilattices, it follows that
  rectangular tolerances in $\mathcal V$ are
  trivial. (This fact can also be deduced from Corollary~5.15 of
  \cite{kearnes-kiss}.) Now by Theorem~5.25 of \cite{kearnes-kiss},
  it follows that $\mathcal V$ satisfies an idempotent Maltsev condition
  that fails in the variety of semilattices. By Theorem~4.10
  of \cite{kearnes-szendrei}, $\mathcal V$ is affine.

[$(2)\Leftrightarrow(3)$]
This is part of Theorem~3.13 of \cite{kearnes-kiss}.

[$(3)\Rightarrow(4)$]
If $\mathcal V$ contains an algebra ${{\mathbf{\uppercase{a}}}}$ with a strongly
abelian congruence $\theta$, then it contains
${{\mathbf{\uppercase{s}}}} = {{\mathbf{\uppercase{s}}}}_{{{\mathbf{\uppercase{a}}}},\theta} :={{\mathbf{\uppercase{a}}}}(\theta)/\Delta$.

[$(4)\Rightarrow(1)$]
Here it suffices to prove that an affine obstruction
for $\mathcal V$ prevents $\mathcal V$ from being
affine. This was explained right after the proof
of Lemma~\ref{nonaffine}.
\end{proof}

\section{Varieties whose finitely generated members are free}
In this section we investigate the class of varieties 
whose finitely generated members are free. 
This class
of varieties is closed under definitional equivalence.
The symbol
$\mathcal V$ will be used only to denote some nontrivial member
of this class.
We shall divide our analysis of this class
into two cases: the subclass of 
varieties with no $0$-ary function symbols
versus 
the subclass of varieties with at least one $0$-ary function symbol.

We shall prove that if the
finitely generated members of $\mathcal V$ are free,
then $\mathcal V$ must be definitionally
equivalent to the variety of sets, pointed sets, vector
spaces over a division ring, or affine spaces over a division
ring. It is obvious that each of these varieties
has the property that its finitely generated members are free.

\subsection{Varieties without constants}\label{subsection1}

If $\mathcal V$ has no $0$-ary function symbols, then
${{\mathbf{\uppercase{f}}}}_{\mathcal V}(\emptyset)$ is empty. ${{\mathbf{\uppercase{f}}}}_{\mathcal V}(1)$
is the only candidate for the $1$-element
algebra in $\mathcal V$. Hence $\mathcal V$ is idempotent.

\begin{thm}\label{idempotent}
  Assume that the finitely generated algebras in $\mathcal V$ are free.
  If $\mathcal V$ is idempotent, then it is definitionally equivalent
  to the variety of sets or to a variety of affine
  spaces over a division ring.
  \end{thm}

\begin{proof}
  It follows from the standard proofs of Magari's Theorem that
  every nontrivial variety has a {\it finitely generated} simple member.
  A free algebra ${{\mathbf{\uppercase{f}}}}_{\mathcal V}(X)$ over $X = \{x_1,x_2,\ldots\}$
  cannot be simple if $|X|>2$, since there are
    noninjective homomorphisms
    $\varepsilon_i\colon {{\mathbf{\uppercase{f}}}}_{\mathcal V}(X)\to {{\mathbf{\uppercase{f}}}}_{\mathcal V}(y,z)$ 
defined on generators by 
  \begin{equation}\label{kernels}
    x_j\mapsto
    \begin{cases}
  y & \textrm{if $j=i$}\\
  z & \textrm{else}.
  \end{cases}   
  \end{equation}
  If $\mathcal V$ is idempotent, then ${{\mathbf{\uppercase{f}}}}_{\mathcal V}(X)$
  cannot be simple for $|X|<2$, either. Thus, in our situation
  ${{\mathbf{\uppercase{f}}}}_{\mathcal V}(2)$ is the only candidate for a finitely generated
  simple member of $\mathcal V$.

  Let $\mathcal M$ be a minimal subvariety of $\mathcal V$.
  $\mathcal M$ also must contain a finitely generated simple algebra,
  and ${{\mathbf{\uppercase{f}}}}_{\mathcal V}(2)$ is the only one in $\mathcal V$
  up to isomorphism, so
  $\mathcal M$ must contain (and be generated by)
  ${{\mathbf{\uppercase{f}}}}_{\mathcal V}(2)$. 
  Every finitely generated algebra in $\mathcal M$ is finitely
  generated in $\mathcal V$, hence free in $\mathcal V$,
  hence free over the same free generating set in $\mathcal M$.
  This shows that $\mathcal M$ is also a variety whose finitely generated
  algebras are free. Also, ${{\mathbf{\uppercase{f}}}}_{\mathcal M}(2)={{\mathbf{\uppercase{f}}}}_{\mathcal V}(2)$.

   According to Corollary 2.10 of \cite{kearnes},
  any minimal idempotent variety, like $\mathcal M$,
  is definitionally equivalent to the
  variety of sets, the variety of semilattices, a variety
  of affine modules over a simple ring, or is congruence distributive.
  
  The variety of semilattices does not have
  the property that its finitely generated members are free.

  A minimal, congruence distributive, idempotent
  variety also does not have the property
  that its finitely generated generated members are free,
  as we now explain. If otherwise, then since
  ${{\mathbf{\uppercase{f}}}}_{\mathcal M}(x,y)\times {{\mathbf{\uppercase{f}}}}_{\mathcal M}(x,y)$
  is finitely generated (by $\{x,y\}\times \{x,y\}$), it must be
  isomorphic to ${{\mathbf{\uppercase{F}}}}_{\mathcal M}(m)$ for some $m$.
  ${{\mathbf{\uppercase{f}}}}_{\mathcal M}(x,y)\times {{\mathbf{\uppercase{f}}}}_{\mathcal M}(x,y)$ is not trivial or simple,
  so $m>2$.   The homomorphisms $\{\varepsilon_i\}_{i=1}^m$
    described in (\ref{kernels})
  (with subscript $\mathcal M$ in place of $\mathcal V$)
  map ${{\mathbf{\uppercase{f}}}}_{\mathcal M}(m)$
  onto the simple algebra ${{\mathbf{\uppercase{f}}}}_{\mathcal M}(2)$,
  and $\varepsilon_i$ has kernel
  different from that of $\varepsilon_j$ when $i\neq j$.
  Thus ${{\mathbf{\uppercase{f}}}}_{\mathcal M}(m)$ has at least $m$ distinct coatoms in
  its congruence lattice
  of the form $\ker(\varepsilon_i)$.
  From this it follows that 
  ${{\mathbf{\uppercase{f}}}}_{\mathcal M}(x,y)\times {{\mathbf{\uppercase{f}}}}_{\mathcal M}(x,y)\cong {{\mathbf{\uppercase{f}}}}_{\mathcal M}(m)$,
  $m>2$,
  has at least
  $3$ coatoms in its congruence lattice.
  But in a congruence distributive variety, the square of a simple
  algebra has exactly two coatoms in its congruence lattice.

  Now consider the case where $\mathcal M$ is a
  variety of affine (left) modules over some ring $R$.
  One realization of ${{\mathbf{\uppercase{f}}}}_{\mathcal M}(2)$
  has universe $R$, generators $0, 1\in R$, and term operations
  of the form
  \[
r_1x_1+\cdots + r_hx_h,\quad r_i\in R, \quad \sum r_i = 1.
\]
Each left ideal of $R$ induces a congruence on this algebra.
Since ${{\mathbf{\uppercase{f}}}}_{\mathcal M}(2)$ is simple, $R$ can have no nontrivial
proper left ideals, hence $R$ must be a division ring.

We have thus far argued that if $\mathcal V$ has the property that
its finitely generated members are free, and $\mathcal M$ is a minimal
subvariety of $\mathcal V$, then $\mathcal M$ is definitionally
equivalent to the variety
of sets or a variety of affine modules over a division ring.
  We now argue that $\mathcal V = \mathcal M$. If this is not
  the case, then there is a finitely generated
  algebra in $\mathcal V\setminus\mathcal M$,
  which we may assume is ${{\mathbf{\uppercase{a}}}}:={{\mathbf{\uppercase{f}}}}_{\mathcal V}(m)$.
  By its very definition, ${{\mathbf{\uppercase{a}}}}$ has an $m$-element generating set
  that is minimal under inclusion as a generating set.
  Now let ${{\mathbf{\uppercase{b}}}}$ be the $m$-generated free algebra in $\mathcal M$.
  ${{\mathbf{\uppercase{b}}}}$ also has an $m$-element minimal generating set.
  ${{\mathbf{\uppercase{b}}}}\in{\mathcal M}$ so ${{\mathbf{\uppercase{b}}}}\in {\mathcal V}$, and
  ${{\mathbf{\uppercase{b}}}}$ cannot be isomorphic to ${{\mathbf{\uppercase{a}}}}$,
  so ${{\mathbf{\uppercase{b}}}} \cong {{\mathbf{\uppercase{f}}}}_{\mathcal V}(n)$
  for some $n\neq m$. This implies that ${{\mathbf{\uppercase{b}}}}$
  has an $n$-element minimal generating set as well as an
  $m$-element minimal generating set.
  But $\mathcal M$ is definitionally
  equivalent to the variety of sets or the variety
  of affine spaces, so it is not possible for ${{\mathbf{\uppercase{b}}}}$ to have minimal
  generating sets of different cardinalities.
  We conclude that $\mathcal V = \mathcal M$.
    \end{proof}

\subsection{Varieties with constants}\label{subsection2}

We still assume that 
$\mathcal V$ is a nontrivial variety whose finitely generated
members are free. In this subsection we also assume
that $\mathcal V$ has $0$-ary function symbols in its language.
In this situation, 
${{\mathbf{\uppercase{f}}}}_{\mathcal V}(\emptyset)$ must
be the $1$-element algebra in $\mathcal V$,
so there is only one constant up to equivalence.
We will assume that there is exactly one constant in the language and
use $0$ to denote it. In any algebra ${{\mathbf{\uppercase{a}}}}\in\mathcal V$
the set $\{0\}$ is the unique 1-element subuniverse of ${{\mathbf{\uppercase{a}}}}$.

In the situation we are in now, with a constant,
it is now ${{\mathbf{\uppercase{f}}}}_{\mathcal V}(1)$ rather than
${{\mathbf{\uppercase{f}}}}_{\mathcal V}(2)$ that is the only candidate
for the finitely generated simple algebra of $\mathcal V$.
To see this, note that when $m$ is greater than $1$, then 
${{\mathbf{\uppercase{f}}}}_{\mathcal V}(x_1,\ldots,x_m)$ has at least
three distinct kernels of homomorphisms onto ${{\mathbf{\uppercase{f}}}}_{\mathcal V}(x)$,
namely the kernels of the homomorphisms defined on generators by
  \begin{enumerate}
  \item $x_1\mapsto 0$; $x_2, \ldots, x_m\mapsto x$, 
  \item $x_1\mapsto x$; $x_2, \ldots, x_m\mapsto 0$, and
  \item $x_1, x_2, \ldots, x_m\mapsto x$.
  \end{enumerate}
  To see that the kernels of these homomorphisms are distinct,
  it suffices to note that they restrict differently to
  the set $\{0,x_1,\ldots,x_m\}\subseteq F_{\mathcal V}(x_1,\ldots,x_m)$.
  Thus ${{\mathbf{\uppercase{f}}}}_{\mathcal V}(m)$ cannot be simple when $m>1$, nor
  can it be simple when $m=0$,
  hence ${{\mathbf{\uppercase{f}}}}_{\mathcal V}(1)$ is the 
  finitely generated simple member of $\mathcal V$. This argument
  also shows that, if $m>1$, then ${{\mathbf{\uppercase{f}}}}_{\mathcal V}(m)$ has at least
  $3$ coatoms in its congruence lattice. We record these observations as:

  \begin{lm}\label{3coatoms}
    If $\mathcal V$ is a nontrivial variety with
    at least one $0$-ary function symbol in its language, and
    all finitely generated members of $\mathcal V$ are free, then
    \begin{enumerate}
    \item ${{\mathbf{\uppercase{f}}}}_{\mathcal V}(0)$ has one element.
    \item ${{\mathbf{\uppercase{f}}}}_{\mathcal V}(1)$ is simple.
    \item ${{\mathbf{\uppercase{f}}}}_{\mathcal V}(m)$ has at least $3$ distinct coatoms in
      its congruence lattice for every finite $m>1$.      $\Box$
      \end{enumerate} 
  \end{lm}

  Later we will need to remember that, from part (3) of this lemma,
  any finitely generated, nontrivial, nonsimple member of $\mathcal V$
  has at least 3 distinct coatoms in its congruence lattice.

Suppose that ${{\mathbf{\uppercase{a}}}}\in \mathcal V$ and $a\in A\setminus \{0\}$.
Then there is a homomorphism ${{\mathbf{\uppercase{f}}}}_{\mathcal V}(x)\to {{\mathbf{\uppercase{a}}}}\colon x\mapsto a$,
which cannot be constant. By the
simplicity of ${{\mathbf{\uppercase{f}}}}_{\mathcal V}(x)$, this homomorphism
must be an isomorphism. This shows
that $a$ is a free generator of the subalgebra $\langle a\rangle\leq {{\mathbf{\uppercase{a}}}}$.
We record this as:

    \begin{lm}\label{freely}
    If $\mathcal V$ is a nontrivial variety with
    a $0$-ary function symbol, and
    all finitely generated members of $\mathcal V$ are free, then
    any nonzero element of any $\mathcal V$
    {freely} generates a subalgebra isomorphic to ${{\mathbf{\uppercase{f}}}}_{\mathcal V}(x)$. $\Box$
      \end{lm}

\begin{lm}\label{abelian}
    If $\mathcal V$ is a nontrivial variety with
    a $0$-ary function symbol, and
    all finitely generated members of $\mathcal V$ are free, then
    ${{\mathbf{\uppercase{f}}}}_{\mathcal V}(x)$ is abelian.
  \end{lm}

\begin{proof}
  In this proof we will abbreviate ${{\mathbf{\uppercase{f}}}}_{\mathcal V}(x)$ by ${{\mathbf{\uppercase{f}}}}$.
  
  Let ${{\mathbf{\uppercase{a}}}}$ be the subalgebra of
  ${{\mathbf{\uppercase{f}}}}\times {{\mathbf{\uppercase{f}}}}$ that is generated
  by $(0,x)$ and $(x,0)$. Let $\eta_1, \eta_2\in\operatorname{Con}({{\mathbf{\uppercase{a}}}})$
  be the restrictions to ${{\mathbf{\uppercase{a}}}}$ of the 
  coordinate projection kernels. Observe that the $\eta_1$-class
  of $0^{{{\mathbf{\uppercase{a}}}}} = (0,0)$ is the set $\{0\}\times F$, which is a
  subuniverse of ${{\mathbf{\uppercase{a}}}}$ that supports a subalgebra isomorphic to ${{\mathbf{\uppercase{f}}}}$.
  Similarly, the $\eta_2$-class of $0^{{{\mathbf{\uppercase{a}}}}}$,
  $F\times \{0\}$, is the universe of a simple subalgebra of ${{\mathbf{\uppercase{a}}}}$.

  The congruence $\operatorname{Cg}((0,0),(0,x))$ is contained in
  $\eta_1$ and both have $F\times \{0\}$ for a transversal.
  Consequently $\eta_1 = \operatorname{Cg}((0,0),(0,x))$.

  Since $\eta_1$ is principal there
  is a congruence $\mu$ that is maximal among congruences
  strictly below $\eta_1$. $\operatorname{\mathbf{Con}}({{\mathbf{\uppercase{a}}}}/\mu)$ contains
  a $3$-element maximal chain $0 = \mu/\mu\prec \eta_1/\mu\prec 1$.
  We apply Lemma~\ref{3coatoms}~(3) to ${{\mathbf{\uppercase{a}}}}/\mu$: the algebra ${{\mathbf{\uppercase{a}}}}/\mu$
  is nontrivial, nonsimple, and a quotient of the $2$-generated
  algebra ${{\mathbf{\uppercase{a}}}}$, so it is finitely generated. The lemma guarantees
  that $\operatorname{\mathbf{Con}}({{\mathbf{\uppercase{a}}}}/\mu)$ has at least $3$ coatoms. The congruence
  $\eta_1/\mu$ is a coatom, but there must be at least two
  others, say $\alpha, \beta\in\operatorname{\mathbf{Con}}({{\mathbf{\uppercase{a}}}}/\mu)$.

  Since $\alpha, \beta$ and $(\eta_1/\mu)$ are incomparable congruences,
  and $(\eta_1/\mu)$ is an atom in $\operatorname{\mathbf{Con}}({{\mathbf{\uppercase{a}}}}/\mu)$,
  we have $\alpha\wedge (\eta_1/\mu) = 0 = \beta\wedge (\eta_1/\mu)$.
  We also have
  \[
(\alpha\vee\beta)\wedge (\eta_1/\mu) = 
1 \wedge (\eta_1/\mu) = \eta_1/\mu,
\]
so the interval $[0,\eta_1/\mu]$ is a meet semidistributivity failure
in $\operatorname{\mathbf{Con}}({{\mathbf{\uppercase{a}}}}/\mu)$. It follows that $\eta_1/\mu$ is abelian.

Recall that $\{0\}\times F$ is a subuniverse of ${{\mathbf{\uppercase{a}}}}$
that is an $\eta_1$-class.
The congruence $\mu$ is smaller than $\eta_1 = \operatorname{Cg}((0,0),(0,x))$,
so it does not contain $\{0\}\times F$ entirely within a class.
Since the subuniverse supported by $\{0\}\times F$
is isomorphic to ${{\mathbf{\uppercase{f}}}}={{\mathbf{\uppercase{f}}}}_{\mathcal V}(1)$, and therefore simple,
$\mu$ restricts
trivially to this set. This implies that
$(\{0\}\times F)/\mu$
is a class of $\eta_1/\mu$ 
that supports a subalgebra
isomorphic to ${{\mathbf{\uppercase{f}}}}$. Since $\eta_1/\mu$ is abelian, it follows
that ${{\mathbf{\uppercase{f}}}}$ is abelian too.
  \end{proof}

\begin{lm}\label{injective}
    If $\mathcal V$ is a nontrivial variety with
    a $0$-ary function symbol, and
    all finitely generated members of $\mathcal V$ are free, 
    then the nonconstant unary polynomial operations
    of ${{\mathbf{\uppercase{f}}}}={{\mathbf{\uppercase{f}}}}_{\mathcal V}(x)$ are injective.
\end{lm}

\begin{proof}
  We first show that the nonconstant unary term operations
  act injectively on ${{\mathbf{\uppercase{f}}}}$.
  
  Let $\eta_1, \eta_2, \Delta\in\operatorname{\mathbf{Con}}({{\mathbf{\uppercase{f}}}}\times {{\mathbf{\uppercase{f}}}})$ be
  the coordinate projection kernels and the congruence obtained
  from collapsing the diagonal. Suppose that $r, s, t\in F$,
  and that $rs=rt$ while $s\neq t$.

  The element $(s,t)\in F\times F$ freely generates a subalgebra
  of ${{\mathbf{\uppercase{f}}}}\times {{\mathbf{\uppercase{f}}}}$ that is
  isomorphic to ${{\mathbf{\uppercase{F}}}}_{\mathcal V}(1)$, hence it is
  a simple subalgebra we denote ${{\mathbf{\uppercase{T}}}}$.
  We have $(s,t)\not\!\!\Delta (0,0)$, so $\Delta|_T$ is trivial.
  But $(rs,rt) \Delta (0,0)$, so $(rs,rt)=(0,0)$. This shows that if
  $s\neq t$ and $rs=rt$, then $rs=0=rt$.
  At least one of $s$ and $t$ is not $0$,
  and the situation between $s$ and $t$ has been symmetric up to this point,
  so assume that $s\neq 0$.

  Choose $u\in F$ arbitrarily. The element $(s, u)$ generates a
  simple subalgebra ${{\mathbf{\uppercase{u}}}}$ of ${{\mathbf{\uppercase{f}}}}\times {{\mathbf{\uppercase{f}}}}$, since $s\neq 0$.
  We have $(s,u)\not\!\!\eta_1(0,0)$, so $\eta_1|_U$ is trivial.
  But $(rs,ru) \eta_1 (0,0)$, so $(rs,ru)=(0,0)$. This shows that if
  $s\neq t$ and $rs=rt$ and $u$ is arbitrary, then $ru=0$, which
  shows that $r$ is constant or acts injectively.

  Now we generalize our conclusion from unary
  term operations to unary polynomial operations of ${{\mathbf{\uppercase{f}}}}$.
  
  
Assume that $p(x)=t^{{{\mathbf{\uppercase{f}}}}}(x,{{\mathbf{{u}}}})$
for some term $t$ and some tuple ${{\mathbf{{u}}}}$.
If $p(a)=p(b)$, then
\[
t^{{{\mathbf{\uppercase{f}}}}}(a,\underline{{{\mathbf{{u}}}}}) = t^{{{\mathbf{\uppercase{f}}}}}(b,\underline{{{\mathbf{{u}}}}}).
\]
This is equivalent to 
\[
t^{{{\mathbf{\uppercase{f}}}}}(a,\underline{{{\mathbf{{0}}}}}) = t^{{{\mathbf{\uppercase{f}}}}}(b,\underline{{{\mathbf{{0}}}}}),
\]
according to the term condition.
This shows that the unary polynomial $p(x) = t^{{{\mathbf{\uppercase{f}}}}}(x,{{\mathbf{{u}}}})$
has the same kernel as the ``twin'' unary term operation
$t^{{{\mathbf{\uppercase{f}}}}}(x,{{\mathbf{{0}}}})$. But such kernels have been shown to
be trivial or universal in the first part of this proof,
so they remain so here. I.e., any nonconstant unary
polynomial operation acts injectively on ${{\mathbf{\uppercase{f}}}}$.
  \end{proof}

Now we are prepared to prove the main result of this subsection.

\begin{thm}
  Let $\mathcal V$ be a nontrivial variety with a $0$-ary function symbol
  such that every finitely generated member of $\mathcal V$ is free.
  $\mathcal V$ is definitionally equivalent to either the variety of
  pointed sets or a variety of vector spaces over a division ring.  
\end{thm}

\begin{proof}
  Let $\mathcal M$ be a minimal subvariety of $\mathcal V$.
  By the same argument we used in Theorem~\ref{idempotent},
  $\mathcal M$ also has the property that its finitely generated
  algebras are free. We first prove the theorem for $\mathcal M$,
  then lift the result to $\mathcal V$, as we did in Theorem~\ref{idempotent}.

  All the lemmas proved for $\mathcal V$ in this subsection
  hold for $\mathcal M$. In particular, the unique finitely generated
  simple algebra up to isomorphism is ${{\mathbf{\uppercase{f}}}}_{\mathcal M}(1) = {{\mathbf{\uppercase{f}}}}_{\mathcal V}(1)$.
  By the minimality of $\mathcal M$,
  ${\mathcal M}=\operatorname{\mathsf{H}}\operatorname{\mathsf{S}}\operatorname{\mathsf{P}}({{\mathbf{\uppercase{f}}}}_{\mathcal M}(1))$, and the free algebras
  of $\mathcal M$ therefore lie in $\operatorname{\mathsf{S}}\operatorname{\mathsf{P}}({{\mathbf{\uppercase{f}}}}_{\mathcal M}(1))$. This
  latter class contains all the free algebras of $\mathcal M$,
  hence contains all of the finitely generated members of $\mathcal M$,
  hence generates $\mathcal M$ as a universal class:
\begin{equation}\label{Mvariety}
  {\mathcal M} =
  \operatorname{\mathsf{S}}\operatorname{\mathsf{P}}_U(\operatorname{\mathsf{S}}\operatorname{\mathsf{P}}({{\mathbf{\uppercase{f}}}}_{\mathcal M}(1)))=\operatorname{\mathsf{S}}\operatorname{\mathsf{P}}\operatorname{\mathsf{P}}_U({{\mathbf{\uppercase{f}}}}_{\mathcal M}(1)).
  \end{equation}
  By Lemma~\ref{abelian},
  ${{\mathbf{\uppercase{f}}}}_{\mathcal M}(1)$ is abelian, hence from (\ref{Mvariety})
  we deduce that $\mathcal M$ is an abelian variety.

  As a first case, assume that $\mathcal M$ is affine. 
  Since it has a $0$-ary function symbol naming a subuniverse,
  $\mathcal M$ is definitionally equivalent to a variety of
  left $R$-modules for some ring $R$. One realization of 
  ${{\mathbf{\uppercase{f}}}}_{\mathcal M}(1)$ has universe $R$, generator $1$, and term operations
  of the form
  \[
r_1x_1+\cdots + r_hx_h,\quad r_i\in R.
\]
Each left ideal of $R$ induces a congruence on this algebra.
Since ${{\mathbf{\uppercase{f}}}}_{\mathcal M}(1)$ is simple, $R$ can have no nontrivial
proper left ideals, hence $R$ must be a division ring.

For the remaining case we may assume, from Theorem~\ref{affine},
that $\mathcal M$ has
an affine obstruction. The element referred to as
$0$ in Lemma~\ref{nonaffine} 
must be the element named by our constant $0$, since
in any member of $\mathcal M$ the element named $0$ is the only
singleton subuniverse. Any subalgebra of an affine obstruction
which contains $0$ inherits properties (2)--(5)
of Lemma~\ref{nonaffine}. Since
any nontrivial $1$-generated algebra in $\mathcal V$ is isomorphic
to ${{\mathbf{\uppercase{f}}}}_{\mathcal M}(1)$, according to Lemma~\ref{freely},
we conclude that ${{\mathbf{\uppercase{f}}}}_{\mathcal M}(1)$ has Property P.

\begin{clm}\label{claim}
${{\mathbf{\uppercase{f}}}}_{\mathcal M}(1)$ has size $2$.
  \end{clm}

{\noindent{\it Proof of Claim.}\ }
Assume otherwise that there are distinct nonzero
elements $a, b\in F_{\mathcal M}(1)$. The congruence
$\operatorname{Cg}(a,b)$ is nonzero, hence by the simplicity of
${{\mathbf{\uppercase{f}}}}_{\mathcal M}(1)$
there is a unary polynomial $p(x)$
of ${{\mathbf{\uppercase{f}}}}_{\mathcal M}(1)$ such that $p(a)=0\neq p(b)$, or the same
with $a$ and $b$ interchanged. But $p(a)=0$ implies
$p(0)=0$, by Property P, showing that $(a,0)$
is a nontrivial pair in $\ker(p)$. On the other hand
$(a,b)$ is a pair not in $\ker(p)$. This contradicts
Lemma~\ref{injective}, which establishes that unary polynomials
of ${{\mathbf{\uppercase{f}}}}_{\mathcal M}(1)$ are constant or injective.
  {\hfill\rule{1.3mm}{3mm}}

  \bigskip
  
  Claim~\ref{claim}, together with earlier information, yields that
  ${{\mathbf{\uppercase{f}}}}_{\mathcal M}(1)$ is a 2-element, nonaffine, abelian
  algebra with a singleton subalgebra named by a constant.
  There is one such algebra up to definitional equivalent,
  namely the 2-element pointed set. (The simplest way to affirm this
  is to refer to Post's classification of 2-element algebras,
  but one doesn't need a result of such depth to make this conclusion.)

  Since $\mathcal M$ is generated by ${{\mathbf{\uppercase{f}}}}_{\mathcal M}(1)$, which
  is equivalent to a pointed set, it follows that ${\mathcal M}$
  is definitionally equivalent to the variety of pointed sets
  in the case we are considering.

  We have shown that $\mathcal M$ is definitionally equivalent
  to a variety of vector spaces over a division ring or the variety
  of pointed sets. 
  We now argue that ${\mathcal V}={\mathcal M}$ using the same type
  of argument used in Theorem~\ref{idempotent}.

  If ${\mathcal V}\neq {\mathcal M}$,
  there is a finitely generated
  algebra in $\mathcal V\setminus\mathcal M$,
  which we may assume is ${{\mathbf{\uppercase{a}}}}:={{\mathbf{\uppercase{f}}}}_{\mathcal V}(m)$.
  ${{\mathbf{\uppercase{a}}}}$ has an $m$-element generating set
  that is minimal under inclusion as a generating set.
  Now let ${{\mathbf{\uppercase{b}}}}$ be the $m$-generated free algebra in $\mathcal M$.
  ${{\mathbf{\uppercase{b}}}}$ also has an $m$-element minimal generating set.
  ${{\mathbf{\uppercase{b}}}}\in{\mathcal M}$, so ${{\mathbf{\uppercase{b}}}}\in {\mathcal V}$, and
  ${{\mathbf{\uppercase{b}}}}$ cannot be isomorphic to ${{\mathbf{\uppercase{a}}}}$,
  so ${{\mathbf{\uppercase{b}}}} \cong {{\mathbf{\uppercase{f}}}}_{\mathcal V}(n)$
  for some $n\neq m$. This implies that ${{\mathbf{\uppercase{b}}}}$
  has an $n$-element minimal generating set as well as an
  $m$-element minimal generating set.
  But there does not exist a vector space nor a pointed set
  that has minimal generating sets of different cardinalities.
  We conclude that $\mathcal V = \mathcal M$.  
\end{proof}

\begin{exmps}
  An $(n+1)$-ary (first variable) \emph{semiprojection} 
  on a set $A$ is an $n$-ary operation $s(x_0,x_1,\ldots,x_n)$
  on $A$ such that for any ${{\mathbf{{a}}}}\in A^{n+1}$ we have
  \[
s(a_0,a_1,\ldots,a_n) = a_0
\]
whenever $a_i=a_j$ for some $i\neq j$.
Starting with any variety $\mathcal V$ we can add an
$(n+1)$-ary function symbol $s$ to the language and define
${\mathcal V}_s$ to be the variety of all
$\mathcal V$-algebras expanded by a semiprojection.

The added semiprojection operation acts like first projection
on any algebra in ${\mathcal V}_s$ that has cardinality
at most $n$. Hence any algebra of size at most $n$
in ${\mathcal V}_s$ is definitionally equivalent to an algebra
in $\mathcal V$.

If $\mathcal V$ is the variety of sets, then every algebra
in ${\mathcal V}_s$ that is generated by at most $n$ elements
will be definitionally equivalent to a set, hence will
be free. But ${\mathcal V}_s$ contains an $(n+1)$-generated
algebra definitionally equivalent to a set which is not free in
${\mathcal V}_s$.

Similarly, if $\mathcal V$ is the variety of vector spaces over
the $2$-element field, and $s$ is a $(2^n+1)$-ary semiprojection,
then the semiprojection acts like first projection
on any algebra in ${\mathcal V}_s$ of generated by at most $n$
elements. Again, all algebras in ${\mathcal V}_s$
that are generated by at most $n$ elements will be free,
but there will be $(n+1)$-generated algebras in ${\mathcal V}_s$
that are not free.
\end{exmps}

These examples show that we really do need to assume that
\emph{all}
finitely generated algebras are free to obtain our results.

\bibliographystyle{plain}
\begin{thebibliography}{10}

\bibitem{campion}
  Tim  Campion,    {\tt http://mathoverflow.net/questions/157974/},
  Feb 18 '14 at 21:15.

\bibitem{freese-mckenzie}
Ralph Freese and Ralph McKenzie, 
{\sl Commutator theory for congruence modular varieties.} 
London Mathematical Society Lecture Note Series, 125. 
Cambridge University Press, Cambridge, 1987. 

\bibitem{givant}
Steven Givant, 
{\it Universal Horn classes categorical or free in power.}
Ann.\ Math.\ Logic {\bf 15} (1978), no. 1, 1--53.

\bibitem{hobby-mckenzie}
David  Hobby and Ralph McKenzie, 
  {\sl The structure of finite algebras.}
  Contemporary Mathematics, {\bf 76}.
  American Mathematical Society, Providence, RI, 1988.

  \bibitem{kearnes}
Keith A.  Kearnes, 
    {\it Almost all minimal idempotent varieties are congruence modular.}
    Algebra Universalis {\bf 44} (2000), no. 1-2, 39--45.
  
\bibitem{kearnes-kiss}
Keith Kearnes and Emil W.\ Kiss, 
{\sl The Shape of Congruence Lattices.}
Mem.\ Amer.\ Math.\ Soc.\ {\bf 222} (2013), no. 1046.

\bibitem{kkv1}
  Keith A.\ Kearnes and Emil W.\ Kiss and Matthew A.\ Valeriote, 
  {\it Minimal sets and varieties.}
  Trans.\ Amer.\ Math.\ Soc.\ {\bf 350} (1998), no. 1, 1--41.

\bibitem{kearnes-szendrei}
Keith Kearnes and \'Agnes Szendrei, 
  {\it The relationship between two commutators.}
  Internat.\ J.\ Algebra Comput.\ {\bf 8} (1998), no. 4, 497--531.

\bibitem{szendrei}
  \'Agnes Szendrei, 
{\it Strongly abelian minimal varieties.}
Acta Sci.\ Math.\ (Szeged) {\bf 59} (1994), no. 1-2, 25--42. 
  
  
\end{thebibliography}
\end{document}

