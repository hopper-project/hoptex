
\documentclass[a4paper,final]{article}
\usepackage[british]{babel}
\pdfminorversion=7
\pdfobjcompresslevel=5

\usepackage{amsmath, amssymb, amsthm, amsfonts}
\usepackage{ifdraft}
\usepackage{microtype}
\usepackage{url}
\DeclareMathAlphabet{\mathpzc}{OT1}{pzc}{m}{it}
\usepackage[utf8]{inputenc}
\usepackage{aliascnt}
\usepackage[colorlinks=true,linkcolor=blue]{hyperref}
\usepackage{tikz}
\usetikzlibrary{arrows,automata}
\usepackage{mathtools}
\usepackage{enumerate}
\usepackage[all]{xy}
\usepackage{paralist}
\usepackage{extarrows}

\usepackage[notref,notcite,color]{showkeys}

\ifdraft{

}
{

}

\makeatletter

\makeatother

\title{On the penetration distance in Garside monoids}
\author{{{Volker Gebhardt}}, {{Stephen Tawn}}}

\hypersetup{pdftitle={On the penetration distance in Garside monoids},
            pdfauthor={Volker Gebhardt, Stephen Tawn},
            pdfkeywords={},
            pdfsubject={},
            pdfdisplaydoctitle={true}}

\let\emptyset\varnothing
\let\interior\mathring

\let\le\leqslant
\let\leq\leqslant
\let\ge\geqslant
\let\geq\geqslant

\let\epsilon\varepsilon

\let\zs=\bowtie

\theoremstyle{plain}
\newtheorem{theorem}{Theorem}

\newaliascnt{lemma}{theorem}
\newtheorem{lemma}[lemma]{Lemma}
\aliascntresetthe{lemma}
\providecommand*{\lemmaautorefname}{Lemma}

\newaliascnt{proposition}{theorem}
\newtheorem{proposition}[proposition]{Proposition}
\aliascntresetthe{proposition}
\providecommand*{\propositionautorefname}{Proposition}

\newaliascnt{corollary}{theorem}
\newtheorem{corollary}[corollary]{Corollary}
\aliascntresetthe{corollary}
\providecommand*{\corollaryautorefname}{Corollary}

\newaliascnt{conjecture}{theorem}
\newtheorem{conjecture}[conjecture]{Conjecture}
\aliascntresetthe{conjecture}
\providecommand*{\conjectureautorefname}{Conjecture}

\theoremstyle{remark}

\newaliascnt{claim}{theorem}
\newtheorem{claim}[claim]{Claim}
\aliascntresetthe{claim}
\providecommand*{\claimautorefname}{Claim}

\newtheorem*{claim*}{Claim}
\newtheorem*{remark}{Remark}

\theoremstyle{definition}

\newaliascnt{definition}{theorem}
\newtheorem{definition}[definition]{Definition}
\aliascntresetthe{definition}
\providecommand*{\definitionautorefname}{Definition}

\newaliascnt{example}{theorem}
\newtheorem{example}[example]{Example}
\aliascntresetthe{example}
\providecommand*{\exampleautorefname}{Example}

\newaliascnt{notation}{theorem}
\newtheorem{notation}[notation]{Notation}
\aliascntresetthe{notation}
\providecommand*{\notationautorefname}{Notation}

\addto\extrasbritish{
  
  
  
  
  
  
  
  
  
}

\makeatletter

\makeatother

\begin{document}

\maketitle
{\xdef\@thefnmark{}\@footnotetext}{Both authors acknowledge support under Australian Research Council's Discovery Projects funding scheme (project number DP1094072). Volker Gebhardt acknowledges support under the Spanish Project MTM2010-19355.}
{\xdef\@thefnmark{}\@footnotetext}{MSC-class: 20F36, 43A07 (Primary) 20F69, 60B15 (Secondary)}

\begin{abstract}
We prove that the exponential growth rate of the regular language of
penetration sequences is smaller than the growth rate of the regular language
of normal form words, if the acceptor of the regular language of normal form
words is strongly connected.
Moreover, we show that the latter property is satisfied for all irreducible
Artin monoids of spherical type, extending a result by Caruso.

   Apart from establishing that the expected value of the penetration distance
$pd(x,y)$ in irreducible Artin monoids of spherical type is bounded
independently of the length of $x$, if $x$ is chosen uniformly among all
elements of given canonical length and $y$ is chosen uniformly among all
atoms, our results also give an affirmative answer to a question posed by
Dehornoy.
\end{abstract}

\section{Introduction}

Random walks on discrete infinite groups in general are a subject that has received substantial interest; see for instance~\cite{KaimanovichVershik,Woess2000} and the references therein.
Random walks on braid groups in particular have many applications in areas as diverse as organic chemistry, nanotechnology, fluid dynamics or astrophysics~\cite{MR1246471,MR1434066,Woess2000,MR1891810}.

Random walks on the 3-strand braid group were analysed in~\cite{MairesseMatheus}.
A complete understanding of random walks on braid groups on a larger number of strands, let alone more general classes of groups such as Artin--Tits groups or Garside groups, has not been achieved yet.
\medskip

A topic closely related to random walks on Garside groups is the behaviour of the Garside normal forms of random elements:  Fixing a position in the normal form, one obtains an induced distribution on the set of simple elements (that is, on the symmetric group in the case of braids) which can be studied.
In~\cite{GT13}, the authors studied these induced distributions and observed convergence.
More precisely, experimental data suggests that, except for an initial and a final region whose lengths are uniformly bounded, the distributions of the factors of the normal form of sufficiently long random braids depend neither on the position in the normal form nor on the lengths of the random braids.

Another, related, observation made for random braids in~\cite{GT13} was the fact that the expected value of the \emph{penetration distance} ${\mathrm{pd}}(x,y)$, that is, the number of factors in the normal form of~$x$ that are modified when computing the normal form of the product~$x\cdot y$, is bounded above independently of the length of~$x$.

For specific distributions of~$x$ and~$y$, we related the behaviour of the expected value ${\mathbb{E}}[{\mathrm{pd}}]$ of the penetration distance to the growth rates~$\alpha$ and~$\beta$ of two regular languages (cf.\ \autoref{S:BackgroundPenetration}).  
More precisely, we showed that, if $x$ is chosen uniformly among all elements of canonical length~$k$ and~$y$ is chosen uniformly among the atoms, then ${\mathbb{E}}[{\mathrm{pd}}]$ is bounded if $\alpha<\beta$ holds~\cite[Theorem~4.7]{GT13}.
Experimental data strongly suggested that the latter condition is satisfied for all irreducible Artin monoids of spherical type.
\medskip

In this paper, we relate the condition $\alpha<\beta$ to certain structural properties of the lattice of simple elements (more precisely, to the connectedness of the acceptor of the regular language of normal forms) and prove that this condition is satisfied for all irreducible Artin monoids of spherical type.

Apart from establishing that the expected value of the penetration distance is bounded in the sense of~\cite[Theorem~4.7]{GT13} for irreducible Artin monoids of spherical type, our results also answer a question posed by Dehornoy in~\cite{DehornoyJCTA07} in the affirmative.

We also prove that the expected value of the penetration distance is unbounded for {Zappa--Sz{\'e}p}{} products of 
irreducible Artin monoids of spherical type.
\medskip

The structure of the paper is as follows.
In \autoref{S:BackgroundGarside}, we recall the basic concepts of normal forms in Garside groups; experts may skip this section.
In \autoref{S:BackgroundPenetration}, we recall the regular languages defined in~\cite{GT13} to study the expected value of the penetration distance.
\autoref{S:BackgroundZappaSzep} recalls the notion of {Zappa--Sz{\'e}p}{} products.

In \autoref{S:Essential} we define the notions of \emph{essential} simple elements and \emph{essential transitivity}, which will be used in \autoref{S:GrowthRates} to compare the growth rates~$\alpha$ and~$\beta$.
\autoref{S:ArtinMonoids} establishes that irreducible Artin monoids of spherical type are essentially transitive and have the property that all proper simple elements are essential, and thus that the results of the preceding sections can be applied to this class of monoids.

\section{Background}\label{S:Background}

In order to fix notation, we briefly recall the main concepts used in the paper.
The material in \autoref{S:BackgroundGarside} is rather well-known, so experts may skip that section.

\subsection{Garside monoids and Garside normal form}\label{S:BackgroundGarside}

We refer to \cite{DehornoyParis99,Dehornoy02,GarsideBook} for details.

Let $M$ be a monoid.
The monoid $M$ is called \emph{left-cancellative} if for any $x,y,y'$ in $M$, the equality $xy = xy'$ implies $y = y'$.
Similarly, $M$ is called \emph{right-cancellative} if for any $x,y,y'$ in $M$, the equality $yx = y'x$ implies $y = y'$.

For $x,y\in M$, we say that $x$ is a \emph{left-divisor} or \emph{prefix} of $y$, writing $x {\preccurlyeq}_M y$, if there exists an element $u\in M$ such that $y = xu$.  If the monoid is obvious, we simply write $x {\preccurlyeq} y$ to reduce clutter.
Similarly, we say that $x$ is a \emph{right-divisor} or \emph{suffix} of $y$,
writing $y{\succcurlyeq}_M x$ or $y{\succcurlyeq} x$, if there exists $u\in M$ such that $y = ux$.
If $M$ does not contain any non-trivial invertible elements, then the relation ${\preccurlyeq}$ is a partial order if $M$ is left-cancellative, and the relation ${\succcurlyeq}$ is a partial order if $M$ is right-cancellative.

An element $a\in M{\mathbin{\raisebox{0.25ex}{$\smallsetminus$}}}\{{\mathbf{1}}\}$ is called an \emph{atom} if whenever $a = uv$ for $u,v\in M$, either $u = {\mathbf{1}}$ or $v = {\mathbf{1}}$ holds.  The existence of atoms implies that $M$ does not contain any non-trivial invertible elements.
The monoid $M$ is said to be
\emph{atomic} if it is generated by its set ${\mathcal{A}}$ of atoms and if for every
element $x\in M$ there is an upper bound on the length of
decompositions of $x$ as a product of atoms, that is, if
$||x||_{\mathcal{A}} := \sup\{ k\in{\mathbb{N}} : x=a_1\cdots a_k \text{ with }a_1,\ldots,a_k\in{\mathcal{A}} \} < \infty$.

An element $d\in M$ is called \emph{balanced}, if the set of its left-divisors is equal to the set of its right-divisors.  In this case, we write $\operatorname{Div}(d)$ for the set of (left- and right-) divisors of $d$.

\begin{definition}
  A \emph{quasi-Garside structure} is a pair $(M, \Delta)$ where $M$
  is a monoid and $\Delta$ is an element of $M$ such that
  \begin{enumerate}[ (a)] \itemsep 0em \vspace{-0.5\topskip}
  \item $M$ is cancellative and atomic,
  \item the prefix and suffix relations are lattice orders, that is, for any pair of elements there exist unique least common upper bounds and unique greatest common lower bounds with respect to ${\preccurlyeq}$ respectively ${\succcurlyeq}$,
  \item $\Delta$ is balanced, and
  \item $M$ is generated by the divisors of $\Delta$.
  \end{enumerate}
  If the set of divisors of $\Delta$ is finite then we say that $(M, \Delta)$ is a \emph{Garside structure}.

  A monoid $M$ is a (quasi)-Garside monoid if there exists a
  \emph{(quasi)-Garside element} $\Delta \in M$ such that $(M, \Delta)$ is a
  (quasi)-Garside structure.
  The elements of $\operatorname{Div}(\Delta)$ are called \emph{simple elements}.  (Note that the set of simple elements depends on the choice of the Garside element.)
\end{definition}

\begin{remark}
  If $(M,\Delta)$ is a quasi-Garside structure in the above sense, then in the terminology of \cite{GarsideBook}, the set $\operatorname{Div}(\Delta)$ forms a bounded Garside family for the monoid $M$.
\end{remark}

\begin{notation}
If $M$ is a left-cancellative atomic monoid, then least common upper bounds and greatest common lower bounds are unique if they exist.  In this situation, we will write $x {\vee} y$ for the ${\preccurlyeq}$-least common upper bound of $x,y\in M$ if it exists, and we write $x {\wedge} y$ for their ${\preccurlyeq}$-greatest common lower bound if it exists.
If $x,y\in M$ admit a ${\preccurlyeq}$-least common upper bound, we define $x{\backslash} y$ as the unique element of $M$ satisfying $x(x{\backslash} y)=x{\vee} y$.

If $(M,\Delta)$ is a Garside structure, we write ${\mathcal{D}}_M$ for the set of simple elements $\operatorname{Div}(\Delta)$, and we define the set of \emph{proper} simple elements as ${{\mathcal{D}}^{\!{}^{\circ}\!}}_M = {\mathcal{D}}_M {\mathbin{\raisebox{0.25ex}{$\smallsetminus$}}} \{ {\mathbf{1}}, \Delta \}$, where ${\mathbf{1}}$ is the identity element of~$M$.  To avoid clutter, we will usually drop the subscript if there is no danger of confusion.
For $x\in{\mathcal{D}}$, there exist a unique elements ${\partial} x={\partial}_M x\in{\mathcal{D}}$ and ${\widetilde{\partial}} x = {\widetilde{\partial}}_M x\in {\mathcal{D}}$ such that $x {\partial} x=\Delta = ({\widetilde{\partial}} x) x$.  We define inductively ${\partial}^{k+1} x = {\partial}({\partial}^k x)$ and ${\widetilde{\partial}}^{k+1} x = {\widetilde{\partial}}({\widetilde{\partial}}^k x)$ for $k\in{\mathbb{N}}$.  As ${\partial} ({\widetilde{\partial}} x) = x = {\widetilde{\partial}}({\partial} x)$ for any $x\in{\mathcal{D}}$, we can define ${\partial}^k = {\widetilde{\partial}}^{-k}$ for any $k\in{\mathbb{Z}}$.

Clearly, ${\partial}^k x\in{{\mathcal{D}}^{\!{}^{\circ}\!}}$ iff $x\in{{\mathcal{D}}^{\!{}^{\circ}\!}}$.
Moreover, for any $x,y\in{\mathcal{D}}$, one has $x{\preccurlyeq} y$ iff ${\partial} x {\succcurlyeq} {\partial} y$ iff ${\partial}^2 x {\preccurlyeq} {\partial}^2 y$.

Given a set $X$ we will write $X^* = {\bigcup}_{i = 0}^{\infty} X^i$ for
the set of strings of elements of $X$.  We will write $\epsilon$ for
the empty string and separate the letters of a string with dots, for
example we will write $a {\mathbin{.}} b {\mathbin{.}} a \in \{a,b\}^*$.

Given a (quasi)-Garside structure $(M, \Delta)$ we can define the \emph{left
normal form} of an element by repeatedly extracting the
${\preccurlyeq}$-GCD of the element and $\Delta$.  More precisely, the
normal form of $x \in M$ is the unique word ${\mathrm{NF}}(x) = x_1 {\mathbin{.}} x_2
{\mathbin{.}} \cdots {\mathbin{.}} x_\ell$ in $({\mathcal{D}}{\mathbin{\raisebox{0.25ex}{$\smallsetminus$}}}\{{\mathbf{1}}\})^*$ such that $x = x_1 x_2 \cdots
x_\ell$ and $x_i = \Delta {\wedge} x_i x_{i+1} \cdots x_\ell$ for $i=1,\ldots,\ell$, or equivalently, ${\partial} x_{i-1} {\wedge} x_i={\mathbf{1}}$ for $i=2,\ldots,\ell$.
We write $x_1|x_2|\cdots|x_\ell$ for the word $x_1 {\mathbin{.}} x_2 {\mathbin{.}} \cdots {\mathbin{.}} x_\ell$
together with the proposition that this word is in normal form.

If $x_1|x_2|\cdots|x_\ell$ is the normal form of $x\in M$, we define the \emph{infimum} of $x$ as
$\inf(x) = \max\{ i\in\{1,\ldots,\ell\} : x_i = \Delta \}$, the \emph{supremum} of $x$ as $\sup(x) = \ell$, and the \emph{canonical length} of $x$ as ${\mathrm{cl}}(x) = \sup(x)-\inf(x)$.  Note that $\inf(x)$ is the largest integer~$i$ such that $\Delta^i{\preccurlyeq} x$ holds, and $\sup(x)$ is the smallest integer~$i$ such that $x{\preccurlyeq} \Delta^i$ holds.

The operation ${\partial}$ can be extended to all of $M$ by defining ${\partial} x$ to be the unique element such that $x {\partial} x = \Delta^{\sup(x)}$.
If ${\mathrm{NF}}(x) = x_1|x_2 \cdots|x_\ell$ is the normal form of~$x$ and $\inf(x)=k$, then
${\mathrm{NF}}({\partial} x) = {\partial} x_\ell| {\partial}^3 x_{\ell-1}| \cdots |{\partial}^{2(\ell-k)+1} x_{k+1}$.

Let ${\mathcal{L}}={\mathcal{L}}_M$ be the language on the set ${{\mathcal{D}}^{\!{}^{\circ}\!}}$ of proper simple elements
consisting of all words in normal form, and write ${\mathcal{L}}^{(n)}={\mathcal{L}}_M^{(n)}$ for
the subset consisting of words of length $n$:
\[
  {\mathcal{L}} := \bigcup_{n\in{\mathbb{N}}} {\mathcal{L}}^{(n)} \quad\text{where}\quad
  {\mathcal{L}}^{(n)} := \big\{ 
      x_1 {\mathbin{.}} \cdots {\mathbin{.}} x_n \in \big({{\mathcal{D}}^{\!{}^{\circ}\!}}\big)^* :
      \forall i,\ {\partial} x_i {\wedge} x_{i+1} = {\mathbf{1}} 
    \big\}
\]

We also define ${\overline{\mathcal{L}}} := {\overline{\mathcal{L}}}_M := \bigcup_{n\in{\mathbb{N}}} {\overline{\mathcal{L}}}^{(n)}$, where
\[
  {\overline{\mathcal{L}}}^{(n)} := {\overline{\mathcal{L}}}_M^{(n)} := \big\{ 
      x_1 {\mathbin{.}} \cdots {\mathbin{.}} x_n \in ({\mathcal{D}}{\mathbin{\raisebox{0.25ex}{$\smallsetminus$}}}\{{\mathbf{1}}\})^* :
      \forall i,\ {\partial} x_i {\wedge} x_{i+1} = {\mathbf{1}} 
    \big\} \;.
\]
\end{notation}

\begin{definition} {} {}
  Given a Garside monoid $M$ with Garside element $\Delta$ and an integer $k\ge1$, let $M(k)$
  denote the same monoid but with the Garside structure given by the
  Garside element $\Delta^k$.
\end{definition}

As the partial orders ${\preccurlyeq}$ and ${\succcurlyeq}$ on $M$ and $M(k)$ are identical, so are the lattice operations ${\vee}$, ${\wedge}$ and ${\backslash}$.  It is obvious from the definitions that one has
${\partial}_{M(k)} x = {\partial}_M x \Delta^m$, where $m = k\big\lceil\frac{\sup(x)}{k}\big\rceil-\sup(x)$.

\begin{lemma} \label{lemma:non-minimal-normal-form}
  If $w$ is a word in $M(k)$-normal form, then replacing each letter of~$w$
  with the word for its $M$-normal form yields a word in $M$-normal form.
\end{lemma}
\begin{proof}
  Suppose that we have $M(k)$-simple elements $x, y \in {\mathcal{D}}_{M(k)}$ with $M$-normal forms
  \begin{align*}
    {\mathrm{NF}}_M(x) &= x_1 |x_2 |\cdots |x_k \\
    {\mathrm{NF}}_M(y) &= y_1 |y_2 |\cdots |y_k
  \end{align*}
  It is sufficient to prove that $x_k | y_1$ in the original Garside structure of $M$.

  Let $m = {\partial}_M x_k {\wedge} y_1$.  We have that $m {\preccurlyeq} y_1
  {\preccurlyeq} y$ and $m {\preccurlyeq} {\partial}_M x_k {\preccurlyeq}
  {\partial}_{M(k)} x$.  Hence $m {\preccurlyeq} {\partial}_{M(k)} x {\wedge}
  y = {\mathbf{1}}$ and so ${\partial}_M x_k {\wedge} y_1 = {\mathbf{1}}$ as required.
\end{proof}

\begin{remark}
The map from \autoref{lemma:non-minimal-normal-form} does not take ${\mathcal{L}}_{M(k)}$ to ${\mathcal{L}}_M$, as proper
simple elements of $M(k)$ may have non-zero infimum in $M$.
\end{remark}

\begin{lemma} \label{lemma:normal-form-non-minimal}
  Given a word $x_0 | x_1 |\cdots| x_l$ in $M$-normal form, let $x_j={\mathbf{1}}$ for $j>l$ and define 
  $y_i = x_{ik} x_{ik + 1} \cdots x_{(i+1)k - 1}$ for $i=0,\ldots,\lceil\frac{l}{k}\rceil$.
  
  Then $y_0 {\mathbin{.}} y_1 {\mathbin{.}}\cdots {\mathbin{.}} y_{\lceil\frac{l}{k}\rceil}$ is in $M(k)$-normal form.
\end{lemma}
\begin{proof}
One has $\Delta {\wedge} {\partial}_{M(k)} y_{i-1} {\wedge} y_i = {\partial}_M x_{ik-1} {\wedge} x_{ik} = {\mathbf{1}}$ for $i=1,\ldots,\lceil\frac{l}{k}\rceil$ and $x_{ik}\neq{\mathbf{1}}$, hence $y_i\neq{\mathbf{1}}$, for $i=0,\ldots,\lceil\frac{l}{k}\rceil$.
\end{proof}

\subsection{Penetration distance and penetration sequences}\label{S:BackgroundPenetration}

Throughout this section, let $M$ be a Garside monoid with Garside element $\Delta$, set of atoms ${\mathcal{A}}$, and set of proper simple elements ${{\mathcal{D}}^{\!{}^{\circ}\!}}$.

In \cite{GT13}, we investigated the \emph{penetration distance}, that is, the number of factors in the normal form of an element $x\in M$ that undergo a non-trivial change when $x$ is multiplied on the right by an element $y\in M$:

\begin{definition}[{\cite[Definition 3.2]{GT13}}]
  For $x,y\in M$, the \emph{penetration distance} for the product
  $xy$ is
  \[
  {\mathrm{pd}}(x,y) = {\mathrm{cl}}(x) - \max\big\{ i \in \{0,\ldots,{\mathrm{cl}}(x)\} :
        x\Delta^{-\inf(x)}  \wedge \Delta^i
        = xy\Delta^{-\inf(xy)} \wedge \Delta^i \big\} \,.
  \]
\end{definition}

For certain probability distributions for $x$ and $y$, we calculated the expected value of ${\mathrm{pd}}(x,y)$ by analysing the regular languages ${\mathcal{L}}$ and ${\textsc{PSeq}} = {\textsc{PSeq}}_M = \bigcup_{k\in{\mathbb{N}}} {\textsc{PSeq}}_M^{(k)}$, where
${\textsc{PSeq}}^{(k)}={\textsc{PSeq}}_M^{(k)}$ denotes the set of all \emph{penetration sequences} of length $k$:

\begin{definition}[{\cite[Definition 4.2]{GT13}}]
  A word $(s_k, m_k) {\mathbin{.}}\cdots {\mathbin{.}}(s_2, m_2) {\mathbin{.}}(s_1, m_1)$ in
  $\left({{\mathcal{D}}^{\!{}^{\circ}\!}} \times {{\mathcal{D}}^{\!{}^{\circ}\!}}\right)^*$ is a \emph{penetration
    sequence of length $k$} if $m_1 \preccurlyeq \partial s_1$ holds, and if one has
  $s_i m_i \ne \Delta$, $\partial s_{i+1} \wedge s_i = {\mathbf{1}}$, and
  $m_{i+1} = \partial s_{i+1} \wedge s_{i} m_{i}$ for $i=1,\ldots,k-1$.
\end{definition}

\begin{notation}\label{N:GrowthRates}
 The regular languages ${\mathcal{L}}_M$, ${\overline{\mathcal{L}}}_M$ and ${\textsc{PSeq}}_M$ are factorial (that is, closed under taking subwords).  By \cite[Corollary 4]{Shur08}, there exist constants $p_M,q_M,r_M\in{\mathbb{N}}$ and $\alpha_M,\beta_M,\gamma_M\in \{0\}\cup [1,\infty[$ such that one has
 \[
   \big|{\textsc{PSeq}}^{(k)}\big|\in\Theta(k^{p_M}{\alpha_M}^k) \;,\;\;
   \big|{\mathcal{L}}^{(k)}\big|\in\Theta(k^{q_M}{\beta_M}^k)\; \text{ and }\;
   \Big|{\overline{\mathcal{L}}}^{(k)}\Big|\in\Theta(k^{r_M}{\gamma_M}^k) \;.
 \]
\end{notation}

One of the key results of \cite{GT13} was that, if $x\in{\mathcal{L}}^{(k)}$ and $y\in{\mathcal{A}}_M$ are chosen with uniform probability, the expected value of the penetration distance is uniformly bounded (that is, there exists a bound that is independent of $k$) if one has $\alpha_M < \beta_M$~\cite[Theorem 4.7]{GT13}.

\subsection{{Zappa--Sz{\'e}p}{} products}\label{S:BackgroundZappaSzep}

{Zappa--Sz{\'e}p}{} products of Garside monoids were considered in \cite{Picantin01} (there called ``crossed products'') and \cite{Zappa-Szep}.  The notion of {Zappa--Sz{\'e}p}{} products generalises direct and semidirect products; the fundamental property being the existence of unique decompositions of the elements of a monoid as products of elements of two submonoids.

\begin{definition}[{\cite[Definition~1]{Zappa-Szep}}]\label{D:ZappaSzepProduct}
  Let $K$ be a monoid with two submonoids~$G$ and~$H$.  We say that~$K$ is
  the (internal) \emph{{Zappa--Sz{\'e}p}{} product} of~$G$ and~$H$, written
  $K=G\zs H$, if for every $k\in K$ there exist unique $g_1, g_2\in
  G$ and $h_1, h_2 \in H$ such that $g_1 h_1 = k = h_2 g_2$.
\end{definition}

It was shown in \cite{Zappa-Szep} that the {Zappa--Sz{\'e}p}{} product $K=G\zs H$ is a Garside monoid if and only if both~$G$ and~$H$ are Garside monoids~\cite[Theorem~31, Theorem~34]{Zappa-Szep}.

If Garside structures for $K$, $G$ and $H$ are chosen in a compatible way, then normal forms in~$K$ can be completely described in terms of normal forms in~$G$ and in~$H$:

\begin{definition}[{\cite[Definition~37]{Zappa-Szep}}]\label{D:GarsideZappaSzepProduct}
  A {Zappa--Sz{\'e}p}{} product $K=G\zs H$ is called a \emph{Garside {Zappa--Sz{\'e}p}{} product} if $K$ is a Garside monoid (and hence $G$ and $H$ are also Garside monoids) and the Garside elements $\Delta_K$, $\Delta_G$ and $\Delta_H$ of $K$, $G$ and $H$, respectively, are chosen such that
  $\Delta_K = \Delta_G\Delta_H$.
\end{definition}

\begin{theorem}[{\cite[Theorem~28, Theorem~38, Corollary~52]{Zappa-Szep}}]\label{T:ZappaSzep}
  Suppose that\linebreak $K = G \zs H$ is a Garside {Zappa--Sz{\'e}p}{} product.
  
  Then the following hold.
  \begin{enumerate} \itemsep 0em \vspace{-0.5\topskip}
   \item
     The map $G \times H \to K$ given by $(g,h) \mapsto g {\vee} h$ is a poset isomorphism
     $(G, {\preccurlyeq}_G)\times(H, {\preccurlyeq}_H) \to (K, {\preccurlyeq}_K)$.

     Similarly, the map $G \times H \to K$ given by $(g,h) \mapsto g {\mathbin{\widetilde{\vee}}} h$ is a poset isomorphism
     $(G, {\succcurlyeq}_G)\times(H, {\succcurlyeq}_H) \to (K, {\succcurlyeq}_K)$.
   \item
     For all $g \in G$ and $h\in H$, one has
     \begin{align*}
       {\inf}_K(g {\vee} h) &= \min({\inf}_G(g), {\inf}_H(h)) \\
        {\sup}_K(g {\vee} h) &= \max({\sup}_G(g), {\sup}_H(h))
        \;..
     \end{align*}
   \item
     The map $\psi {\colon\!} {\overline{\mathcal{L}}}_G \times {\overline{\mathcal{L}}}_H \to {\overline{\mathcal{L}}}_K$ given by
     \[
        \psi\big(g_1 | g_2 | \cdots | g_m \,,\, h_1 | h_2 | \cdots | h_n\big)
          = {\mathrm{NF}}\big( (g_1 g_2 \cdots g_m) {\vee} (h_1 h_2 \cdots h_n) \big)
     \]
     is a bijection.
  \end{enumerate}
\end{theorem}

\section{Essential simple elements}\label{S:Essential}

Throughout this section, let~$M$ be a Garside monoid with Garside element~$\Delta$,
set of atoms~${\mathcal{A}}$ and set of proper simple elements~${{\mathcal{D}}^{\!{}^{\circ}\!}}$, and let~${\mathcal{L}}$ be the words over the alphabet~${{\mathcal{D}}^{\!{}^{\circ}\!}}$ that are in normal form.

By the acceptor graph~${\Gamma}$ of~${\mathcal{L}}$ we shall mean the labelled directed graph
with vertex set~${{\mathcal{D}}^{\!{}^{\circ}\!}}$ and, for $x,y\in{{\mathcal{D}}^{\!{}^{\circ}\!}}$, an edge labelled~$y$ from~$x$ to~$y$ iff
${\partial} x {\wedge} y = {\mathbf{1}}$.  Paths in this graph are in 1-to-1 correspondence with words in normal form.  This graph can be made into a deterministic finite state automaton (DFA) accepting~${\mathcal{L}}$ by adding an initial vertex~${\mathbf{1}}_\Gamma$ with an edge labelled~$y$ from~${\mathbf{1}}_\Gamma$ to~$y$ for each $y\in{{\mathcal{D}}^{\!{}^{\circ}\!}}$.

\begin{definition}
  Say that a simple element $x \in {{\mathcal{D}}^{\!{}^{\circ}\!}}$ is \emph{essential} if
  for all $K \in {\mathbb{N}}$ there exists a word $x_1 |x_2 |\cdots |x_n \in
  {\mathcal{L}}$ such that $x = x_i$ for some $K < i < n-K$.
\end{definition}
Let ${\mathpzc{Ess}} = {\mathpzc{Ess}}_M$ denote the set of all essential simple elements and
${\mathcal{L}}_{\mathpzc{Ess}}$ be the restriction of ${\mathcal{L}}$ to essential simple
elements.
\[ {\mathcal{L}}_{\mathpzc{Ess}} := {\mathcal{L}} {\cap} {\mathpzc{Ess}}^* \]  

\begin{example}
  In $M = \langle a, b {\boldsymbol{\mid}} a\,b\,a = b^2\rangle^+$ one has ${\partial}
  b {\wedge} x = b^2 {\wedge} x \ne {\mathbf{1}}$ for all proper simple elements
  $x\in{{\mathcal{D}}^{\!{}^{\circ}\!}}$, whence $b$ can only occur as the final canonical
  factor and $b^2$ can only occur as the first canonical factor in a
  word in normal form.  That is, the simple elements $b$ and $b^2$ are
  not essential.
\end{example}

\begin{proposition} \label{prop:complement-essential}
  A proper simple element $x\in{{\mathcal{D}}^{\!{}^{\circ}\!}}$ is essential if and only if
  its complement ${\partial} x$ is essential.
\end{proposition}
\begin{proof}
  Suppose that $x\in{{\mathcal{D}}^{\!{}^{\circ}\!}}$ is essential and that we are given
  $K\in{\mathbb{N}}$.  As $x$ is essential there exists a word 
  \[ w_n |\cdots |w_2| w_1| x| y_1| y_2| \cdots |y_m \in {\mathcal{L}} \]
  with $n, m > K$.  As for any $s,t\in{\mathcal{D}}$ and any $k\in{\mathbb{Z}}$ one has ${\partial} s{\wedge} t={\mathbf{1}}$ iff ${\partial}({\partial}^{2k+1} t){\wedge} {\partial}^{2k+3} s={\mathbf{1}}$, the word
  \[ {\partial}^{-2m+1} y_m {\mathbin{.}} \cdots {\mathbin{.}} {\partial}^{-3} y_2 {\mathbin{.}}
     {\partial}^{-1} y_1 {\mathbin{.}} {\partial} x {\mathbin{.}} {\partial}^3 w_1 {\mathbin{.}}
     {\partial}^5 w_2 {\mathbin{.}} \cdots {\mathbin{.}} {\partial}^{2n+1} w_n \]
  is also in normal form and hence lies in ${\mathcal{L}}$.  Therefore
  ${\partial} x$ is essential.

  Applying the same argument with ${\widetilde{\partial}} = {\partial}^{-1}$, we deduce
  that if ${\partial} x$ is essential then $x$ is essential.
\end{proof}

\begin{lemma} \label{lemma:non-minimal-essential}
  A simple element in $M(k)$ is essential if and only if its normal
  form in $M$ is a length $k$ word of essential simple elements.
  \[ {\mathpzc{Ess}}_{M(k)} = {\mathcal{L}}_{{\mathpzc{Ess}},M}^{(k)} \]
\end{lemma}
\begin{proof}
  First we will show that the $M$-normal form of every
  $M(k)$-essential element is a length $k$ word of $M$-essential
  elements.  So suppose that $x\in{\mathpzc{Ess}}_{M(k)}$ is $M(k)$-essential
  and that ${\mathrm{NF}}_M(x) = x_1 |x_2 |\cdots |x_k$ is the $M$-normal form of
  $x$.  As $x$ is essential, for every $K \in {\mathbb{N}}$ there exists a word
  \[ w_K |\cdots| w_2| w_1|x|y_1| y_2 |\cdots| y_K \in {\mathcal{L}}_{M(k)}. \] 
  By \autoref{lemma:non-minimal-normal-form} we can replace each $w_i$
  and each $y_i$ by their $M$-normal forms to produce a word in
  $M$-normal form.
  \begin{align*}
    {\mathrm{NF}}_M(w_i) &= w_{i,1} |w_{i,2} |\cdots| w_{i,k} \\
    {\mathrm{NF}}_M(y_i) &= y_{i,1} |y_{i,2} |\cdots| y_{i,k} 
  \end{align*}
  As each $w_i$ and $y_i$ are proper $M(k)$-simple elements we have
  that, for $i < K$, $w_{i,j}$ and $y_{i,j}$ are proper $M$-simple
  elements.  Hence
  \begin{align*}
  w_{K-1,1} |\cdots |w_{K-1,k}|\cdots|
  w_{1,1} |\cdots |w_{1,k}| 
  x_1 |\cdots |x_k| 
   & y_{1,1} |\cdots |y_{1,k}|\cdots \\
   & \cdots |y_{K-1,1} |\cdots| y_{K-1,k}
  \end{align*}
  is a word in ${\mathcal{L}}_M$ and so each $x_i$ is essential.

  It remains to show that every $x_1 |x_2 |\cdots| x_k \in
  {\mathcal{L}}_{{\mathpzc{Ess}},M}^{(k)}$ defines a $M(k)$-essential simple element.
  Using \autoref{lemma:normal-form-non-minimal} it is clear that $x_1
  x_2 \cdots x_k$ is a proper $M(k)$-simple element.  As $x_1$ and
  $x_k$ are essential, for any $K\in{\mathbb{N}}$ there exist words
  $w_{kK} |\cdots| w_2 |w_1 \in {\mathcal{L}}_M$ and $y_1| y_2| \cdots| y_{kK} \in {\mathcal{L}}$
  such that $w_1 | x_1$ and $x_k | y_1$.  By
  \autoref{lemma:normal-form-non-minimal}, if we let
  \begin{align*}
    \bar w_i &= w_{iK} w_{iK-1} \cdots w_{(i-1)K + 1} \\
    \bar x   &= x_1 x_2 \cdots x_k \\
    \bar y_i &= y_{(i-1)K+1} y_{(i-1)K+2} \cdots y_{iK}
  \end{align*}
  then 
  \[
  \bar w_K {\mathbin{.}}\cdots{\mathbin{.}} \bar w_2{\mathbin{.}} \bar w_1{\mathbin{.}} \bar x{\mathbin{.}} \bar y_1{\mathbin{.}} \bar y_2{\mathbin{.}} \cdots {\mathbin{.}}\bar y_K
  \]
  is in $M(k)$-normal form.  Moreover, because each $w_i$ and each
  $y_i$ are proper $M$-simple elements, we know that each letter of
  this word is a proper $M(k)$-simple element.  Therefore this word
  is in ${\mathcal{L}}_{M(k)}$ and hence $\bar x$ is essential.
\end{proof}

\begin{proposition}
  If ${\mathpzc{Ess}} = \emptyset$ then $M = {\mathbb{N}}$.
\end{proposition}
\begin{proof}
  First note that ${\mathcal{L}}$ must be finite:  Otherwise, there would be word $w \in {\mathcal{L}}$ whose length is longer than the number of states in the automaton accepting~${\mathcal{L}}$ whence, by the pumping lemma, there would exist words $x$, $y$, $z$ such that $w = x{\mathbin{.}} y{\mathbin{.}} z$, $|y| > 1$ and $x{\mathbin{.}} y^i{\mathbin{.}} z \in {\mathcal{L}}$ for all $i$; in particular, every letter of $y$ would be essential in contradiction to the hypothesis.
  
  By \autoref{lemma:non-minimal-essential}, if ${\mathpzc{Ess}}_M$ is empty then ${\mathpzc{Ess}}_{M(k)}$ is empty for all $k$.  Hence, passing to $M(k)$, where $k$ is the length of the longest word contained in ${\mathcal{L}}$, we may assume that all the words in ${\mathcal{L}}$ have length 1.
  
  For every $s \in {{\mathcal{D}}^{\!{}^{\circ}\!}}$ and every $a \in{\mathcal{A}}$ we have $sa \in {\mathcal{D}}$, as otherwise, $s|a$ would be a word of length $2$ in ${\mathcal{L}}$.
  Choosing any maximal element $s \in {{\mathcal{D}}^{\!{}^{\circ}\!}}$, we have $s a = \Delta$ for all atoms $a \in {\mathcal{A}}$, hence, by cancellativity, all the atoms must be equal.
  In other words there is a single atom and so $M = {\mathbb{N}}$.
\end{proof}

\begin{proposition}
  If ${\mathpzc{Ess}} \ne \emptyset$ then $|{\mathpzc{Ess}}| > 1$.
\end{proposition}
\begin{proof}
  Suppose ${\mathpzc{Ess}} = \{ s \}$.

  As $s$ is essential there exist arbitrarily long words containing
  $s$.  So there is a word $w \in {\mathcal{L}}$ whose length is longer than
  the number of states in the automaton accepting ${\mathcal{L}}$ hence, by
  the pumping lemma, there exist words $x$, $y$, $z$ such that $w =
  x{\mathbin{.}} y{\mathbin{.}} z$, $|y| > 1$ and $x{\mathbin{.}} y^i{\mathbin{.}} z \in {\mathcal{L}}$ for all $i$.  

  This means that every letter of $y$ is essential.  Therefore $y$ is
  a power of $s$ and so $s|s$, that is, ${\partial} s {\wedge} s
  = {\mathbf{1}}$.

  By \autoref{prop:complement-essential}, ${\partial} s$ is
  essential, hence ${\partial} s = s$.  Therefore, ${\partial} s
  {\wedge} s = s$, which is a contradiction.
\end{proof}

\begin{definition}
  Say that the language ${\mathcal{L}}$ is \emph{essentially $k$-transitive}
  if ${\mathpzc{Ess}} \ne \emptyset$ and for all $x, y \in {\mathpzc{Ess}}$ there exists a word of
  length less than or equal to $k+1$ in~${\mathcal{L}}_{\mathpzc{Ess}}$ which starts with
  $x$ and ends with $y$.

  Say that ${\mathcal{L}}$ is essentially transitive if it is essentially
  $k$-transitive for some $k$.
\end{definition}

{}

\begin{remark} 
  The language ${\mathcal{L}}$ is essentially transitive if and only if the
  acceptor graph ${\Gamma}$ has exactly one non-singleton strongly connected component (that is, exactly one strongly connected component containing at least one edge).
\end{remark}

\begin{proposition}
  For any pair of Garside monoids $G$ and $H$, each containing at
  least one proper simple element, their free product amalgamated over their Garside elements,
  $G*_{\Delta_G = \Delta_H} H$, is a Garside monoid for which
  every proper simple element is essential and whose language of
  normal forms is essentially $2$-transitive.
\end{proposition}
\begin{proof}
  The amalgamated product $G*_{\Delta_G = \Delta_H} H$ is a Garside
  monoid \cite[Prop.~5.3]{DehornoyParis99}\cite{Froehle07} whose set of
  proper simple elements is the disjoint union of the proper simple
  elements of $G$ and $H$.\footnote{Free products amalgamated over
    standard parabolic subgroups are considered for Artin groups in
    \cite{DehornoyGodelle13} and for preGarside groups in
    \cite{GodelleParis12}.}

  Now consider $u,v \in {{\mathcal{D}}^{\!{}^{\circ}\!}}$.
  If $u$ and $v$ lie in different factors then $u|v$ is in normal form.
  If $u$ and $v$ lie in the same factor, then choosing any proper simple
  element $w$ from the other factor yields a word $u| w| v$ that is in
  normal form.
\end{proof}

\begin{lemma}\label{lemma_essential_transitivity_framed}
  If the language of normal forms of $M(k)$ is essentially transitive
  then the language of normal forms of $M$ is essentially transitive
\end{lemma}
\begin{proof}
  Let $x,y \in {\mathpzc{Ess}}_M$.  Since $x$ and $y$ are essential, there exist elements $x_1,\ldots,x_{k-1}$ and 
  $y_2,\ldots,y_k$ in ${{\mathcal{D}}^{\!{}^{\circ}\!}}_M$ such that $\bar x = x_1 {\mathbin{.}} \cdots{\mathbin{.}} x_{k-1} {\mathbin{.}} x$ and
  $\bar y = y {\mathbin{.}} y_2{\mathbin{.}}  \cdots{\mathbin{.}} y_k$.  
  By \autoref{lemma:non-minimal-essential}, $\bar x, \bar y \in {\mathpzc{Ess}}_{M(k)}$.
  Now as the language of normal forms in $M(k)$ is
  essentially transitive, we can find a path connecting $\bar x$ to
  $\bar y$ and then, by \autoref{lemma:non-minimal-normal-form},
  taking the $M$-normal form of each letter gives a path in $M$ from
  $x$ to $y$.
\end{proof}

\begin{remark}
The converse of \autoref{lemma_essential_transitivity_framed} does not hold.  For example, consider the monoid $M=\langle a, b {\boldsymbol{\mid}} a^2=b^2\rangle^+$ (see \autoref{example:aa=bb}).
$M$ is essentially transitive, but $M(2)$ is not:  We have ${\mathpzc{Ess}}_M = \{a,b\}$ and $a|b$ as well as $b|a$.  However, ${\mathpzc{Ess}}_{M(2)} = \{ab,ba\}$ and ${\mathcal{L}}_{M(2)} = (ab)^* \cup (ba)^*$.
\end{remark}

\begin{lemma}
  If the language of normal forms of $M$ is essentially transitive then
  for all $k_0$ there exists $k \ge k_0$ such that the language of
  normal forms of $M(k)$ is essentially transitive.
\end{lemma}
\begin{proof}
  First note that $M(k)$ is essentially transitive if and only if for
  each pair $x$, $y$ of $M$-essential elements there exists a
  path $x|x_1|x_2|\cdots|x_l|y$ such that $l$ is a multiple of $k$.

  For each pair $x$, $y$ of $M$-essential elements choose
  $z\in{\mathpzc{Ess}}_M$ together with paths $x|x_1|x_2|\cdots|x_l|z$,
  $z|z_1|z_2|\cdots|z_m|z$ and $z|y_1|y_2|\cdots|y_n|y$.

  As the set of essential elements is finite, we can choose an integer $k\ge k_0$ such that, for each pair $x$, $y$ the integer $m$ is coprime to $k$.
  Then for each pair $x$, $y$ there exists $p$ such that $l + pm + n = 0 \mod{k}$.
\end{proof}

\begin{definition}
 A word $x_1|\cdots|x_k\in{\mathcal{L}}^{(k)}$ is called \emph{rigid}, if the word $x_k|x_1$ is in normal form.
 Let ${\mathcal{L}}_{\mathpzc{Rig}}^{(k)}$ denote the set of rigid words in ${\mathcal{L}}^{(k)}$.
\end{definition}

\begin{theorem}[{\cite[Proposition~4.1]{Caruso13}}] \label{PercentageOfRigid}
  If the language ${\mathcal{L}}$ is essentially transitive then, for sufficiently large~$l$, the proportion of rigid elements in the ball of radius $l$, that is in ${\bigcup}_{k\le l}{\mathcal{L}}^{(k)}$, is bounded below by a positive constant.
\end{theorem}
\begin{proof}
  {}
  {}
  A word $x_1|\cdots|x_k \in {\mathcal{L}}$ is rigid if and only if the word $x_1|\cdots|x_k{\mathbin{.}} x_1$ is in normal form, that is, traces out a cycle in ${\Gamma}$.
  
  By the hypotheses there exists a constant $D$ such that, given any word
  $w = x_1|\cdots|x_\ell \in {\mathcal{L}}^{(\ell)}$, there are an integer $c<D$ and $x_{\ell+1},\ldots,x_{\ell+c} \in {{\mathcal{D}}^{\!{}^{\circ}\!}}$ such that
  $x_1|\cdots|x_\ell|x_{\ell+1}|\cdots|x_{\ell+c}|x_1 \in {\mathcal{L}}^{(\ell)}$ or, in other words,
  such that any $w \in {\mathcal{L}}^{(\ell)}$ can be extended to a rigid word $w' \in {\mathcal{L}}^{(\ell')}$ for some $\ell'\in\{\ell,\ldots,\ell+D-1\}$.
  For fixed $\ell$, the map $w\mapsto w'$ is clearly injective, so
  $\Big|{\bigcup}_{k=\ell}^{\ell+D-1} {\mathcal{L}}_{\mathpzc{Rig}}^{(k)}\Big| \ge \big|{\mathcal{L}}^{(\ell)}\big|$.

  Now let
  $
     a_d := \Big| {\bigcup}_{k=(d-1)D+1}^{dD} {\mathcal{L}}_{\mathpzc{Rig}}^{(k)} \Big|
     \quad\text{ and }\quad
     b_d := \Big| {\bigcup}_{k=(d-1)D+1}^{dD} {\mathcal{L}}^{(k)} \Big|
     \;.
  $
  As there are essential elements, the language ${\mathcal{L}}$ is infinite, and thus $\big|{\mathcal{L}}^{(k)}\big|\in\Theta(k^{q}{\beta}^k)$ with $\beta\ge 1$.
  Thus, we have $a_d, b_d\ge 1$ for all values of $d$, and there exists a constant $C>0$ such that $a_d > \frac1C b_d$ and $1+b_{d+1} < C b_d$ hold for sufficiently large~$d$.
  Hence,
  \[
    \frac{\,\Big|{\bigcup}_{k=0}^l {\mathcal{L}}_{\mathpzc{Rig}}^{(k)}\Big|\,}{\,\Big|{\bigcup}_{k=0}^l {\mathcal{L}}^{(k)}\Big|\,}
    \ge \frac{a_1+\cdots+a_{\lfloor\frac{l}{D}\rfloor}}
             {1+b_1+\cdots+b_{\lfloor\frac{l}{D}\rfloor}+b_{\lfloor\frac{l}{D}\rfloor+1}}
    > \frac1{2C(C+1)} 
  \]
  holds for sufficiently large $l$.
\end{proof}

\begin{example} \label{example:aa=bb}
  Under the hypotheses of \autoref{PercentageOfRigid}, it is not necessarily true that the percentage of rigid elements in ${\mathcal{L}}^{(k)}$ is bounded below by a positive constant for sufficiently large values of~$k$.  As an example, consider the monoid $M=\langle a, b {\boldsymbol{\mid}} a^2 = b^2 \rangle^+$.
{}

  \hfill
  \parbox[c]{0.45\textwidth}{
    \begin{xy}
      0;<3em,0em>:<0em,3em>::
      
      (1,-0.6)*+{\text{Hasse diagram}};
      (1,0)*+{\mathbf{1}}="e";
      (0,1)*+{a}="a";
      (2,1)*+{b}="b";
      (1,2)*+{\Delta}="aa";
      
      {\ar@{->}^{a} "e";"a"};
      {\ar@{->}_{b} "e";"b"};
      {\ar@{->}^{a} "a";"aa"};
      {\ar@{->}_{b} "b";"aa"};
    \end{xy}
  } \hfill
  \parbox[c]{0.45\textwidth}{
    \begin{xy}
      0;<3em,0em>:<0em,3em>::
      
      (0.5,-0.75)*+{\Gamma};
      (0,0)*+{a}="a";
      (1,0)*+{b}="b";
      
      {\ar@/^/ "a";"b"};
      {\ar@/^/ "b";"a"};
    \end{xy}
  } \hfill {}
  
  Clearly, the language ${\mathcal{L}}$ is essentially transitive.
  (And, moreover, every proper simple element is essential.)
  However, for odd $k=2m+1$, one has
  ${\mathcal{L}}^{(k)} = \{ a{\mathbin{.}}(b{\mathbin{.}} a)^m , b{\mathbin{.}}(a{\mathbin{.}} b)^m \}$, and so there are no rigid elements of length~$k$.
\end{example}

\medskip

Being essentially transitive is a rather strong property.  In particular, it implies that the monoid in question cannot be decomposed as a {Zappa--Sz{\'e}p}{} product, but not all monoids that are $\zs$-indecomposable are essentially transitive.

\begin{proposition}\label{product-not-transitive}
  If $M = G \zs H$ for two submonoids~$G$ and~$H$, then~$M$ is not essentially transitive.
\end{proposition}
\begin{proof}
  By \cite[Theorem~31]{Zappa-Szep},~$G$ and~$H$ are parabolic submonoids of~$K$.
  The corresponding Garside elements~$\Delta_G$ of~$G$ and~$\Delta_H$ of~$H$ are proper simple elements of~$K$.
  Moreover, these elements are essential as, for all~$k\in{\mathbb{N}}$, the words~$\Delta_G^k$
  and~$\Delta_H^k$ are in normal form.
  Yet, by \cite[Corollary~45]{Zappa-Szep}, there cannot be a normal form word connecting~$\Delta_G$ to~$\Delta_H$.
\end{proof}

\begin{definition}\label{D:DeltaPure}
  For $y\in M$, let $\Delta_y := {\bigvee}\{ x{\backslash} y : x \in M \}$.
  The monoid $M$ is called \emph{$\Delta$-pure} if $\Delta_a=\Delta_b$ holds for any $a,b\in{\mathcal{A}}$.
\end{definition}

\begin{theorem}[{\cite[Proposition~4.7]{Picantin01}\cite[Theorem~36]{Zappa-Szep}}]\label{T:DeltaPure}
  A Garside monoid $M$ is $\Delta$-pure if and only if it is $\zs$-indecom\-posable.
\end{theorem}

\begin{corollary}\label{TransitiveImpliesDeltaPure}
  If $M$ is essentially transitive then it is $\Delta$-pure.
\end{corollary}
\begin{proof}
  The claim follows with \autoref{product-not-transitive} and \autoref{T:DeltaPure}.
\end{proof}

The following example shows that the converse to \autoref{product-not-transitive} and \autoref{TransitiveImpliesDeltaPure} does not hold, that is, there exists a Garside monoid that is $\Delta$-pure, hence $\zs$-indecomposable, but not essentially transitive.

\begin{example}\label{example:aa=bb:2-framed}
  Consider the monoid $M = \langle a, b {\boldsymbol{\mid}} a^2 = b^2 \rangle^+$, with the Garside structure given by the Garside element $\Delta = a^4$.
  The lattice of simple elements and the acceptor for the regular language of words in normal form are shown in \autoref{F:aa=bb:2-framed}.

  \begin{figure}
  \hfill
  \parbox[c]{0.45\textwidth}{
    \begin{xy}
      0;<3em,0em>:<0em,3em>::
      
      (1,-0.6)*+{\text{Hasse diagram}};
      (1,0)*+{\mathbf{1}}="e";
      (0,1)*+{a}="a";
      (2,1)*+{b}="b";
      (-1,2)*+{ab}="ab";
      (1,2)*+{a^2}="aa";
      (3,2)*+{ba}="ba";
      (0,3)*+{a^3}="aaa";
      (2,3)*+{b^3}="bbb";
      (1,4)*+{\Delta}="Delta";
      
      {\ar@{->}^{a} "e";"a"};
      {\ar@{->}_{b} "e";"b"};
      {\ar@{->}_{a} "a";"aa"};
      {\ar@{->}^{b} "a";"ab"};
      {\ar@{->}^{b} "b";"aa"};
      {\ar@{->}_{a} "b";"ba"};
      {\ar@{->}^{b} "ab";"aaa"};
      {\ar@{->}_{a} "aa";"aaa"};
      {\ar@{->}^{b} "aa";"bbb"};
      {\ar@{->}_{a} "ba";"bbb"};
      {\ar@{->}^{a} "aaa";"Delta"};
      {\ar@{->}_{b} "bbb";"Delta"};
    \end{xy}
  } \hfill
  \parbox[c]{0.45\textwidth}{
    \begin{xy}
      0;<3em,0em>:<0em,3em>::
      
      (1,-1)*+{\Gamma};
      (0,0)*+{ab}="ab";
      (2,0)*+{ba}="ba";
      (1,1)*+{a^2}="aa";
      (0,1)*+{a}="a";
      (2,1)*+{b}="b";
      (0,2)*+{b^3}="bbb";
      (2,2)*+{a^3}="aaa";
      
      {\ar@{->} "ab";"a"};
      {\ar@{->} "ba";"b"};
      {\ar@{->} "bbb";"a"};
      {\ar@{->} "aaa";"b"};
      {\ar@(l,d) "ab";"ab"};
      {\ar@(d,r) "ba";"ba"};
    \end{xy}
  } \hfill {}
  \caption{The lattice of simple elements and the acceptor for the regular language of words in normal form for the monoid from \autoref{example:aa=bb:2-framed}.}
  \label{F:aa=bb:2-framed}
  \end{figure}

  We see that the acceptor graph is not connected and so $M$ cannot be
  essentially transitive.

  Now consider $\Delta_a = {\bigvee} \{ x {\backslash} a : x \in M \}$.
  If $x$ has $a$ as a prefix then $x {\backslash} a = {\mathbf{1}}$, so we can
  restrict our attention to elements which do not have $a$ as a
  prefix.  This means that $x = (ba)^k$ or $x = (ba)^k b$ for some
  $k$.  In the first case $(ba)^k {\vee} a = (ba)^k a$, so $x {\backslash} a
  = a$.  In the second case $(ba)^k b {\vee} a = (ba)^k b^2$, so $x
  {\backslash} a = b$.  Therefore $\Delta_a = a {\vee} b = a^2$.  Similarly,
  $\Delta_b = a^2$.  Hence $M$ is $\Delta$-pure, and thus $\zs$-indecomposable.
\end{example}

\section{Growth rates}\label{S:GrowthRates}

Throughout this section, let~$M$ be a Garside monoid with Garside element~$\Delta$ and set of atoms~${\mathcal{A}}$, such that the set ${{\mathcal{D}}^{\!{}^{\circ}\!}}=\operatorname{Div}(\Delta){\mathbin{\raisebox{0.25ex}{$\smallsetminus$}}}\{{\mathbf{1}},\Delta\}$ of proper simple elements is finite.
Recall that~$\alpha_M$ is the exponential growth rate of the regular language~${\textsc{PSeq}}_M$ and that~$\beta_M$ is the exponential growth rate of the regular language~${\mathcal{L}}_M$.

The main aim of this section is to show that $\alpha_M < \beta_M$ holds if the language~${\mathcal{L}}_M$ is essentially transitive and every element of~${{\mathcal{D}}^{\!{}^{\circ}\!}}$ is essential.
In particular, by \cite[Theorem~4.7]{GT13}, the expectation ${\mathbb{E}}_{\nu_k\times\mu_{\mathcal{A}}}[{\mathrm{pd}}]$ of the penetration distance ${\mathrm{pd}}(x,a)$ is bounded independently of~$k$ if~$\nu_k$ is the uniform distribution on~${\mathcal{L}}^{(k)}$ and~$\mu_{\mathcal{A}}$ is the uniform distribution on the set~${\mathcal{A}}$.

Moreover, we will show that the expectation of the penetration distance ${\mathbb{E}}_{\nu_k\times\mu_{\mathcal{A}}}[{\mathrm{pd}}]$, with~$\nu_k$ and~$\mu_{\mathcal{A}}$ as above, diverges if~$M$ is the Garside {Zappa--Sz{\'e}p}{} product of two Garside monoids~$G$ and~$H$, such that $\beta_G,\beta_H>1$ holds, the languages~${\mathcal{L}}_G$ and~${\mathcal{L}}_H$ are essentially transitive, and all proper simple elements of~$G$ respectively~$H$ are essential.

\begin{theorem}\label{T:BoundedExpectedPD}
If every proper simple element of~$M$ is essential and the language~${\mathcal{L}}_M$ of normal forms is essentially transitive, then one has $\alpha_M<\beta_M$.
\end{theorem}
\begin{proof}
As there is only one monoid, we drop the subscript $M$.

Consider the acceptor ${\Gamma}$ of ${\mathcal{L}} \subseteq ({{\mathcal{D}}^{\!{}^{\circ}\!}})^*$, whose adjacency matrix is given~by
\[
 {\Gamma}_{s_1,s_2} = \begin{cases}
                         1 & \partial s_2\wedge s_1 = {\mathbf{1}} \\
                         0 & \text{otherwise}
                       \end{cases}
\]
for $s_1,s_2\in{{\mathcal{D}}^{\!{}^{\circ}\!}}$.  The growth rate $\beta$ of ${\mathcal{L}}$ is the Perron--Frobenius eigenvalue of the non-negative matrix $({\Gamma}_{s_1,s_2})_{s_1,s_2\in{{\mathcal{D}}^{\!{}^{\circ}\!}}}$.
Let $x = (x_s)_{s\in{{\mathcal{D}}^{\!{}^{\circ}\!}}}$ be an eigenvector for the eigenvalue $\beta$ of ${\Gamma}$.

The acceptor ${\Pi}$ of ${\textsc{PSeq}} \subseteq \mathcal{P}^*$, where $\mathcal{P}=\{(s,m)\in{{\mathcal{D}}^{\!{}^{\circ}\!}}\times{{\mathcal{D}}^{\!{}^{\circ}\!}} : sm\preccurlyeq\Delta\}$, has the adjacency matrix given by
\[
 {\Pi}_{(s_1,m_1),(s_2,m_2)} = \begin{cases}
                         1 & \partial s_2\wedge s_1 = {\mathbf{1}}
                             \,\text{ and }\, s_1m_1\ne \Delta
                             \,\text{ and }\, m_2 = \partial s_2\wedge s_1m_1 \\
                         0 & \text{otherwise}
                       \end{cases}
\]
for $(s_1,m_1),(s_2,m_2)\in\mathcal{P}$.  The growth rate $\alpha$ of ${\textsc{PSeq}}$ is the Perron--Frobenius eigenvalue of the non-negative matrix $\big({\Pi}_{(s_1,m_1),(s_2,m_2)}\big)_{(s_1,m_1),(s_2,m_2)\in\mathcal{P}}$, whence one has
\[
   \alpha = \inf_{z\in({\mathbb{R}}^+)^{\mathcal{P}}} \;
               \max_{t\in\mathcal{P}} \;
                  \frac{({\Pi} z)_t}{z_t}
\]
by~\cite[Theorem~3.1]{TamWu89}.
In order to prove the theorem, it is thus sufficient to construct a vector
$y = (y_t)_{t\in\mathcal{P}}$ such that, for any 
$t\in\mathcal{P}$, one has $y_t>0$ and $(\Pi y)_t < \beta y_t$.

To do this, consider
$\widetilde{\mathcal{P}} = \{(s,m)\in{{\mathcal{D}}^{\!{}^{\circ}\!}}\times({{\mathcal{D}}^{\!{}^{\circ}\!}}\cup\{{\mathbf{1}}\}) : sm\preccurlyeq\Delta\}$,
and define a directed graph~$\widetilde{\Pi}$ with vertex set~$\widetilde{\mathcal{P}}$
via its adjacency matrix given by
\[
 \widetilde{\Pi}_{(s_1,m_1),(s_2,m_2)} = \begin{cases}
                         1 & \partial s_2\wedge s_1 = {\mathbf{1}} \,\text{ and }\, m_2 = \partial s_2\wedge s_1m_1 \\
                         0 & \text{otherwise}
                       \end{cases}
\]
for $(s_1,m_1),(s_2,m_2)\in\widetilde{\mathcal{P}}$.
Observe that~$\widetilde{\Pi}$ has~${\Pi}$ as a subgraph and that, locally,~$\widetilde{\Pi}$ resembles the graph~${\Gamma}$:
The edges ending in the vertex $(s_1,m_1)\in\widetilde{\mathcal{P}}$ of $\widetilde{\Pi}$ are in bijection to the edges ending in the vertex~$s_1$ of~${\Gamma}$.  More precisely, for given $s_1$, $s_2$ and $m_1$, there exists an~$m_2$ such that there is an edge $(s_2,m_2) \to (s_1,m_1)$ in~$\widetilde{\Pi}$, if and only if there is an edge $s_2 \to s_1$ in~${\Gamma}$, and if so,~$m_2$ is uniquely determined, that is, there exists exactly one such edge.
(See \autoref{F:Perron-Frobenius-aa=bb} and \autoref{F:Perron-Frobenius-aba=bab}.)

\begin{figure}
  \hfill
  \parbox[b]{0.3\textwidth}{
    \begin{tikzpicture}[->,auto,node distance=2cm, semithick]
      \node (1) at (1,0) {${\mathbf{1}}$};
      \node (a) at (0,1) {$a$};
      \node (b) at (2,1) {$b$};
      \node (D) at (1,2) {$\Delta$};

      \path (1) edge node               {$a$} (a);
      \path (1) edge node [below right] {$b$} (b);
      \path (a) edge node               {$a$} (D);
      \path (b) edge node [above right] {$b$} (D);
      
      \node at (1,-1) {Hasse diagram};
    \end{tikzpicture}
  } \hfill
  \parbox[b]{0.3\textwidth}{
    \begin{tikzpicture}[->,auto,node distance=2cm, semithick]
      \node (a) at (0,0) {$a$};
      \node (b) at (2,0) {$b$};

      \path (a) edge [bend left=20] (b);
      \path (b) edge [bend left=20] (a);
      
      \node at (1,-1) {${\Gamma}$};
    \end{tikzpicture}
  } \hfill
  \parbox[b]{0.3\textwidth}{    
    \begin{tikzpicture}[->,auto,node distance=2cm, semithick]
      \node [red] (a_1) at (0,0) {$(a,{\mathbf{1}})$};
      \node [red] (b_1) at (2,0) {$(b,{\mathbf{1}})$};
      \node       (a_a) at (0,1.5) {$(a,a)$};
      \node       (b_b) at (2,1.5) {$(b,b)$};

      \path (a_1) edge [bend left=20,red] (b_1);
      \path (b_1) edge [bend left=20,red] (a_1);
      \path (b_b) edge [bend left=20,red] (a_a);
      \path (a_a) edge [bend left=20,red] (b_b);
      
      \node at (1,-1) {${\Pi}$\qquad
                       \color{red}{$\widetilde{\Pi} {\mathbin{\raisebox{0.25ex}{$\smallsetminus$}}} {\Pi}$}};
    \end{tikzpicture}
  } \hfill {}
  \caption{Lattice of simple elements and digraphs $\Gamma$, ${\Pi}$ and $\widetilde{\Pi}$ for the monoid $M=\langle a,b {\boldsymbol{\mid}} a^2=b^2 \rangle^+$.  Vertices and edges in red are contained in $\widetilde{\Pi} \smallsetminus {\Pi}$.  (That is, the digraph~${\Pi}$ consists of the vertices $(a,a)$, and $(b,b)$ without any edges.)}
  \label{F:Perron-Frobenius-aa=bb}
\end{figure}

\begin{figure}
  \hfill
  \parbox[b]{0.3\textwidth}{
    \begin{tikzpicture}[->,auto,node distance=2cm, semithick]
      \node (1)  at (1,0) {${\mathbf{1}}$};
      \node (a)  at (0,1) {$a$};
      \node (b)  at (2,1) {$b$};
      \node (ab) at (0,2) {$ab$};
      \node (ba) at (2,2) {$ba$};
      \node (D)  at (1,3) {$\Delta$};

      \path (1)  edge node               {$a$} (a);
      \path (1)  edge node [below right] {$b$} (b);
      \path (a)  edge node               {$b$} (ab);
      \path (b)  edge node               {$a$} (ba);
      \path (ab) edge node               {$a$} (D);
      \path (ba) edge node [above right] {$b$} (D);
      
      \node at (1,-1) {Hasse diagram};
    \end{tikzpicture}
  } \hfill
  \parbox[b]{0.3\textwidth}{
    \begin{tikzpicture}[->,auto,node distance=2cm, semithick]
      \node (a)  at (0,1) {$a$};
      \node (ba) at (2,1) {$ba$};
      \node (b)  at (2,0) {$b$};
      \node (ab) at (0,0) {$ab$};

      \path (a)  edge [loop above,left]  (a);
      \path (b)  edge [loop below,right] (b);
      \path (ba) edge [bend left=10]     (ab);
      \path (ab) edge [bend left=10]     (ba);
      \path (a)  edge                    (ab);
      \path (ab) edge                    (b);
      \path (b)  edge                    (ba);
      \path (ba) edge                    (a);
      
      \node at (1,-1) {${\Gamma}$};
    \end{tikzpicture}
  } \hfill
  \parbox[b]{0.3\textwidth}{
    \begin{tikzpicture}[->,auto,node distance=2cm, semithick]
      \node [red] (a_1)  at (0,2) {$(a,{\mathbf{1}})$};
      \node [red] (ba_1) at (2,1) {$(ba,{\mathbf{1}})$};
      \node [red] (b_1)  at (2,0) {$(b,{\mathbf{1}})$};
      \node [red] (ab_1) at (0,1) {$(ab,{\mathbf{1}})$};
      \node (a_b)  at (2,2) {$(a,b)$};
      \node (b_a)  at (0,0) {$(b,a)$};

      \path (a_1)  edge [loop above,left,red]  (a_1);
      \path (b_1)  edge [loop below,right,red] (b_1);
      \path (ba_1) edge [bend left=10,red]     (ab_1);
      \path (ab_1) edge [bend left=10,red]     (ba_1);
      \path (a_1)  edge [red]                  (ab_1);
      \path (ab_1) edge [red]                  (b_1);
      \path (b_1)  edge [red]                  (ba_1);
      \path (ba_1) edge [red]                  (a_1);
      \path (ab_1) edge [red]                  (b_a);
      \path (b_1)  edge [red]                  (b_a);
      \path (ba_1) edge [red]                  (a_b);
      \path (a_1)  edge [red]                  (a_b);
      
      \node (ab_a) at (0,3.5) {$(ab,a)$};
      \node (b_ab) at (2,3.5) {$(b,ab)$};
      \node (a_ba) at (0,4.5) {$(a,ba)$};
      \node (ba_b) at (2,4.5) {$(ba,b)$};

      \path (ba_b) edge [bend left=10,red]     (ab_a);
      \path (ab_a) edge [bend left=10,red]     (ba_b);
      \path (ab_a) edge [red]                   (b_ab);
      \path (ba_b) edge [red]                   (a_ba);
      \path (b_ab) edge [red]                   (ba_b);
      \path (a_ba) edge [red]                   (ab_a);
      \path (a_ba) edge [loop above,left,red]  (a_ba);
      \path (b_ab) edge [loop below,right,red] (b_ab);

      \node at (1,-1) {${\Pi}$\qquad
                       \color{red}{$\widetilde{\Pi} {\mathbin{\raisebox{0.25ex}{$\smallsetminus$}}} {\Pi}$}};
    \end{tikzpicture}
  } \hfill {}
  \caption{Lattice of simple elements and digraphs $\Gamma$, ${\Pi}$ and $\widetilde{\Pi}$ for the monoid $M=\langle a,b {\boldsymbol{\mid}} a\,b\,a = b\,a\,b \rangle^+$.  Vertices and edges in red are contained in $\widetilde{\Pi} \smallsetminus {\Pi}$.  (That is, the digraph~${\Pi}$ consists of the vertices $(a,b)$, $(b,a)$, $(b.ab)$, $(a,ba)$, $(ab,a)$ and $(ba,b)$ without any edges.)}
  \label{F:Perron-Frobenius-aba=bab}
\end{figure}

Now define a vector $y = (y_t)_{t\in\mathcal{P}}$ by setting $y_{(s,m)}=x_s$ for $(s,m)\in\mathcal{P}$ and a vector $\widetilde{y} = (\widetilde{y}_t)_{t\in\widetilde{\mathcal{P}}}$ by setting $\widetilde{y}_{(s,m)}=x_s$ for $(s,m)\in\widetilde{\mathcal{P}}$.

Clearly, $\widetilde{y}$ is an eigenvector to the eigenvalue $\beta$ of $\widetilde{\Pi}$:
\[
   (\widetilde{\Pi}\widetilde{y})_{(s_1,m_1)}
     = \!\!\!\!
   \sum_{\substack{
               (s_2,m_2)\in\widetilde{\mathcal{P}} \\[0.5ex]
               \partial s_2\wedge s_1 = {\mathbf{1}} \\[0.5ex]
               m_2=\partial s_2\wedge s_1 m_1
             }} \!\!\!\! \widetilde{y}_{(s_2,m_2)}
     = \!\!
   \sum_{\substack{
               s_2\in{{\mathcal{D}}^{\!{}^{\circ}\!}} \\[0.5ex]
               \partial s_2\wedge s_1 = {\mathbf{1}}
             }} \!\!\!\! x_{s_2}
     =
   ({\Gamma} x)_{s_1}
     =
   \beta \cdot x_{s_1}
     =
   \beta \cdot \widetilde{y}_{(s_1,m_1)} .
\]

So it remains to show that for $t\in\mathcal{P}\subseteq\widetilde{\mathcal{P}}$ one has $y_t > 0$ and
$({\Pi} y)_t < (\widetilde{\Pi}\widetilde{y})_t$.\medskip

\begin{claim*}
For $(s,m)\in\mathcal{P}\subseteq\widetilde{\mathcal{P}}$ one has $y_{(s,m)} > 0$.
\end{claim*}
\noindent
As every $s\in{{\mathcal{D}}^{\!{}^{\circ}\!}}$ is essential and ${\mathcal{L}}$ is essentially transitive,
for any $s_1,s_2\in{{\mathcal{D}}^{\!{}^{\circ}\!}}$ there exists a positive integer $\ell$ such that $(\Gamma^\ell)_{s_1,s_2}>0$, whence $x_{s_2}>0$ implies $(\Gamma^\ell x)_{s_1}>0$, as $\Gamma$ and $x$ are non-negative.  As $x$ is an eigenvector of $\Gamma$ and $x_{s_2}>0$ must hold for at least one $s_2\in{{\mathcal{D}}^{\!{}^{\circ}\!}}$, we have $x_{s_1}>0$ for every $s_1\in{{\mathcal{D}}^{\!{}^{\circ}\!}}$ and thus $y_{(s,m)}=x_{s}>0$ for every $(s,m)\in\mathcal{P}$, showing the claim.
\medskip

\begin{claim*}
For $(s,m)\in\mathcal{P}\subseteq\widetilde{\mathcal{P}}$ one has
$({\Pi} y)_{(s_1,m_1)} < (\widetilde{\Pi}\widetilde{y})_{(s_1,m_1)}$.
{}
\end{claim*}
\noindent
We obtain ${\Pi}$ from $\widetilde{\Pi}$ by
\begin{inparaenum}[1.)]
 \item removing all edges ending in $(s_1,m_1)$ if $m_1={\mathbf{1}}$ or $s_1m_1=\Delta$; and
 \item removing the edge $(s_2,m_2)\to (s_1,m_1)$ if $m_2={\mathbf{1}}$.
\end{inparaenum}

So it is sufficient to show that for all $(s_1,m_1)\in\mathcal{P}$ such that $s_1m_1\ne\Delta$, there exists $s_2\in{{\mathcal{D}}^{\!{}^{\circ}\!}}$ such that $\partial s_2\wedge s_1m_1={\mathbf{1}}$ and $y_{(s_2,m_2)}=x_{s_2}>0$.
The latter holds as all proper simple elements are essential, and thus $s_1m_1\ne\Delta$ implies the existence of an essential $s_2$ such that $s_2|s_1m_1$, and since $x_{s_2}>0$ holds.
\end{proof}

\begin{remark}
\autoref{T:BoundedExpectedPD} shows that the hypotheses of \cite[Theorem~4.8]{GT13} cannot be satisfied.
\end{remark}

\begin{corollary}\label{C:BoundedExpectedPD}
Assume that every proper simple element of $M$ is essential and that the language ${\mathcal{L}}_M$ of normal forms is essentially transitive, let $\nu_{k}$ be the uniform probability measure on ${\mathcal{L}}_M^{(k)}$, and let $\mu_{\mathcal{A}}$ be the uniform probability distribution on the set ${\mathcal{A}}$ of atoms of $M$.

The expected value $\mathbf{E}_{\nu_{k} \times \mu_{\mathcal{A}}}[{\mathrm{pd}}]$ of the penetration distance with respect to $\nu_k\times \mu_{\mathcal{A}}$ is uniformly bounded (that is, bounded independently of $k$).
\end{corollary}

\begin{proof}
The claim follows with \autoref{T:BoundedExpectedPD} and \cite[Theorem 4.7]{GT13}.
\end{proof}

\smallskip\noindent
We now turn to the analysis of growth rates of Garside {Zappa--Sz{\'e}p}{} products.

\begin{lemma}\label{L:Sums}
For $c>1$ and $m\in{\mathbb{N}}$ the following hold:
\begin{enumerate} \itemsep 0em \vspace{-0.5\topskip}
\item One has $\sum_{j=0}^{k-1} j^m c^j \in \Theta(k^m c^k)$.
\item One has $\sum_{j=0}^{k-1} j^m \in \Theta(k^{m+1})$.
\end{enumerate}
\end{lemma}

\begin{proof}
The second claim holds by Bernoulli's formula~\cite[p.\,283]{concrete}.  For the first claim, observe that for any $k\ge2$ one has
\[
\frac1{2^{-m}c}
 \le \frac{1}{k^m c^k} (k-1)^m c^{k-1}
 \le \frac{1}{k^m c^k} \sum_{j=0}^{k-1} j^m c^j
 <   \sum_{j=1}^k c^{-j}
 <   \frac{1}{c-1}  \;.
\]
\end{proof}

\begin{lemma}\label{L:GrowthRates}
With the notation of \autoref{N:GrowthRates} the following hold:
\begin{enumerate} \itemsep 0em \vspace{-0.5\topskip}
\item One has $\beta_M=0$ if and only if $\gamma_M=1$ and $r_M=0$ hold.
\item One has $\beta_M=1$ if and only if $\gamma_M=1$ and $r_M\ge1$ hold.
      Moreover, in this case, one has $q_M=r_M-1$.
\item One has $\beta_M>1$ if and only if $\gamma_M>1$ holds.
      Moreover, in this case, one has $\beta_M=\gamma_M$ and $q_M=r_M$.
\end{enumerate}
\end{lemma}
\begin{proof}
As ${\overline{\mathcal{L}}}^{(k)} = \bigsqcup_{j=0}^k \Delta^{k-j} {\mathcal{L}}^{(j)}$ holds, one has
$\Big|{\overline{\mathcal{L}}}^{(k)}\Big| = \sum_{j=0}^k \big|{\mathcal{L}}^{(j)}\big|$.
Firstly observe that $\beta_M=0$ holds if and only if one has $\big|{\mathcal{L}}^{(k)}\big|=0$ for sufficiently large $k$.  The latter happens if and only if $\Big|{\overline{\mathcal{L}}}^{(k)}\Big|$ is eventually constant, which is equivalent to $\gamma_M=1$ and $r_M=0$.  So the first claim is shown.

Using \autoref{N:GrowthRates} and \autoref{L:Sums}, we obtain
\[
  k^{r_G} {\gamma_M}^k \in \Theta\Bigg( \sum_{j=0}^k j^{q_M} {\beta_M}^j \Bigg)
    = \begin{cases}
       \Theta\big( k^{q_M} {\beta_M}^k \big) & \text{if $\beta_M>1$} \\[1.0ex]
       \Theta\big( k^{q_M+1} \big)         & \text{if $\beta_M=1$} \;,
      \end{cases}
\]
which implies the remaining claims.
\end{proof}

\begin{proposition}\label{P:GrowthRatesProduct}
Assume that $M=G\zs H$ is a Garside {Zappa--Sz{\'e}p}{} product, let $\beta_M,\gamma_M$ and $q_M,r_M$ be as in~\autoref{N:GrowthRates}, and let $\beta_G,\gamma_G,\beta_H,\gamma_H\in\{0\}\cup[1,\infty[$ and $q_G,r_G,q_H,r_H\in{\mathbb{N}}$ the corresponding constants for~$G$ respectively~$H$.

The following table gives $\beta_M,\gamma_M,q_M,r_M$ in terms of $\beta_G,\gamma_G,\beta_H,\gamma_H$ and $q_G,r_G,q_H,r_H$:
\vspace*{-2ex}

{\tiny
\[
\begin{array}{@{}c@{\,}||@{\,}c@{\,}|@{\,}c@{\,}|@{\,}c@{}}
                          & \beta_H=0             & \beta_H=1                 & \beta_H>1                   \\
                          & \gamma_H=1, r_H=0     & \gamma_H=1, r_H=q_H+1     & \gamma_H=\beta_H, r_H=q_H
\rule[-1.2ex]{0pt}{2.5ex}\\ \hline\hline\rule{0pt}{2.5ex}
\beta_G=0                 & \beta_M=1, q_M=0      & \beta_M=1, q_M=q_H+1      & \beta_M=\beta_H, q_M=q_H+1  \\
\gamma_G=1, r_G=0         & \gamma_M=1, r_M=1     & \gamma_M=1, r_M=r_H+1     & \gamma_M=\gamma_H, r_M=r_H+1
\rule[-1.2ex]{0pt}{2.5ex}\\ \hline\rule{0pt}{2.5ex}
\beta_G=1                 & \beta_M=1, q_M=q_G+1  & \beta_M=1, q_M=q_G+q_H+2  & \beta_M=\beta_H, q_M=q_G+q_H+2    \\
\gamma_G=1, r_G=q_G+1     & \gamma_M=1, r_M=r_G+1 & \gamma_M=1, r_M=r_G+r_H+1 & \gamma_M=\gamma_H, r_M=r_G+r_H+1
\rule[-1.2ex]{0pt}{2.5ex}\\ \hline\rule{0pt}{2.5ex}
\beta_G>1                 & \beta_M=\beta_G, q_M=q_G+1  & \beta_M=\beta_G, q_M=q_G+q_H+2  & \beta_M=\beta_G\beta_H, q_M=q_G+q_H \\
\gamma_G=\beta_G, r_G=q_G & \gamma_M=\gamma_G, r_M=r_G+1 & \gamma_M=\gamma_G, r_M=r_G+r_H+1 & \gamma_M=\gamma_G\gamma_H, r_M=r_G+r_H \\
\end{array}
\]
}
\end{proposition}
\begin{proof}
First note that, by \autoref{L:GrowthRates}, the cases in the table are correct and exhaustive.

By \autoref{T:ZappaSzep}, for any $x\in {\overline{\mathcal{L}}}_M^{(k)}$ there are unique words $g_x\in {\overline{\mathcal{L}}}_G^{(k_G)}$ and $h_x\in {\overline{\mathcal{L}}}_H^{(k_H)}$ such that $k=\max\{k_G,k_H\}$.  Moreover, the map $x\mapsto (g_x,h_x)$ is a bijection from ${\overline{\mathcal{L}}}_M$ to ${\overline{\mathcal{L}}}_G \times {\overline{\mathcal{L}}}_H$.
Hence one has
\[
  \big|{\overline{\mathcal{L}}}_M^{(k)}\big| = \big|{\overline{\mathcal{L}}}_G^{(k)}\big| \cdot \big|{\overline{\mathcal{L}}}_H^{(k)}\big|
      + \sum_{j=0}^{k-1} \big|{\overline{\mathcal{L}}}_G^{(k)}\big| \cdot \big|{\overline{\mathcal{L}}}_H^{(j)}\big|
      + \sum_{j=0}^{k-1} \big|{\overline{\mathcal{L}}}_G^{(j)}\big| \cdot \big|{\overline{\mathcal{L}}}_H^{(k)}\big|
\]
and thus
\begin{align*}
  {\gamma_M}^k k^{r_M} \in \Theta\Bigg(
    {\gamma_G}^k k^{r_G} {\gamma_H}^k k^{r_H}
    + {\gamma_G}^k k^{r_G} & \bigg(\sum_{j=0}^{k-1} {\gamma_H}^j j^{r_H}\bigg) \\
    & + \bigg(\sum_{j=0}^{k-1} {\gamma_G}^j j^{r_G}\bigg) {\gamma_H}^k k^{r_H}
  \Bigg)  \;.
\end{align*}
The claimed equalities for $\gamma_M$ and $r_M$ are easily verified using \autoref{L:Sums}, and the claimed equalities for $\beta_M$ and $q_M$ then follow with \autoref{L:GrowthRates}.
\end{proof}

\begin{corollary}\label{C:GrowthRatesProduct}
Assume that $M=G\zs H$ is a Garside {Zappa--Sz{\'e}p}{} product, and let $\beta_G$ and $\beta_H$ be the exponential growth rates of the regular languages~${\mathcal{L}}_G$ respectively~${\mathcal{L}}_H$.

Then $\big|{\mathcal{L}}_M^{(k)}\big| \in \Theta\Big( \big|{\mathcal{L}}_G^{(k)}\big| \cdot \big|{\mathcal{L}}_H^{(k)}\big| \Big)$ holds if and only if $\beta_G>1$ and $\beta_H>1$.
\end{corollary}
\begin{proof}
Using \autoref{N:GrowthRates}, $\big|{\mathcal{L}}_M^{(k)}\big| \in \Theta\Big( \big|{\mathcal{L}}_G^{(k)}\big| \cdot \big|{\mathcal{L}}_H^{(k)}\big| \Big)$ is equivalent to $\beta_M=\beta_G\beta_H$ and $q_M=q_G+q_H$.
The claim then follows with \autoref{P:GrowthRatesProduct}.
\end{proof}

\begin{notation}\label{N:PSeqProduct}
Assume that $M=G\zs H$ is a Garside {Zappa--Sz{\'e}p}{} product.  Given $g=g_1|\cdots| g_k\in {\mathcal{L}}_G^{(k)}$ and $h=h_1|\cdots |h_k\in {\mathcal{L}}_H^{(k)}$,
consider the normal form $g'_1 h'_1|\cdots| g'_k h'_k \in {\mathcal{L}}_M^{(k)}$ of
$g_1\cdots g_k {\vee} h_1\cdots h_k$,
and define
\[
   \pi_{g,h} := (g'_1 h'_1, m_1){\mathbin{.}}\cdots{\mathbin{.}} (g'_k h'_k, m_k)
   \;,
\]
where $m_k=\partial_H(h'_k)$ (that is, $h'_k m_k=\Delta_H$) and $m_i = \partial_M(g'_i h'_i) \wedge_M g'_{i+1} h'_{i+1} m_{i+1}$ for $i=1,\ldots,k-1$.
\end{notation}

\begin{lemma}\label{L:PSeqProduct}
In the situation of \autoref{N:PSeqProduct}, one has $\pi_{g,h} \in {\textsc{PSeq}}_M^{(k)}$.
\end{lemma}
\begin{proof}
Using \cite[Lemma~39]{Zappa-Szep}, we have $\Delta_H {\preccurlyeq} g'_i h'_i m_i \neq \Delta_M$ and $m_i\ne {\mathbf{1}}$ for all $i$ by induction, so $\pi_{x,y}\in{\textsc{PSeq}}_M^{(k)}$.
\end{proof}

\begin{proposition}\label{P:PSeqProduct}
Assume that $M=G\zs H$ is a Garside {Zappa--Sz{\'e}p}{} product, let $\alpha_M,\beta_M$ and $p_M,q_M$ be as in~\autoref{N:GrowthRates}, and let $\alpha_G,\beta_G,\alpha_H,\beta_H\in\{0\}\cup[1,\infty[$ and $p_G,q_G,p_H,q_H\in{\mathbb{N}}$ the corresponding constants for~$G$ respectively~$H$.

Then one has the following:
\begin{enumerate} \itemsep 0em \vspace{-0.5\topskip}
\item If one has $\beta_G,\beta_H>0$, then $\alpha_M=\beta_M$ holds.
\item If $\beta_G,\beta_H>1$, then $\alpha_M=\beta_M$ and $p_M=q_M$ hold.
\end{enumerate}
\end{proposition}
\begin{proof}
By \autoref{L:PSeqProduct}, we have a map ${\mathcal{L}}_G^{(k)} \times {\mathcal{L}}_H^{(k)} \to {\textsc{PSeq}}_M^{(k)}$ given by the assignment $(g,h)\mapsto \pi_{g,h}$.
This map is injective by \autoref{T:ZappaSzep}, so we have
$\big|{\textsc{PSeq}}_M^{(k)}\big| \ge \big|{\mathcal{L}}_M^{(k)}\big| \cdot \big|{\mathcal{L}}_N^{(k)}\big|$.
Thus $k^{q_G+q_H} (\beta_G\beta_H)^k \in O(k^{p_M} {\alpha_M}^k) \subseteq O(k^{q_M} {\beta_M}^k)$, where the final inclusion holds by~\cite[Corollary~4.4]{GT13}.

If $\beta_G,\beta_H>0$, then we have $\beta_M=\beta_G\beta_H$ by \autoref{P:GrowthRatesProduct}, and thus obtain $\alpha_M=\beta_M$.
Similarly, if $\beta_G,\beta_H>1$, then we have $\beta_M=\beta_G\beta_H$ and $q_M=q_G+q_H$ by \autoref{P:GrowthRatesProduct}, and thus obtain $\alpha_M=\beta_M$ and $p_M=q_M$.
\end{proof}

\begin{notation}
Assume that $M$ is a Garside monoid with set of proper simple elements ${{\mathcal{D}}^{\!{}^{\circ}\!}}$.
For $s\in {{\mathcal{D}}^{\!{}^{\circ}\!}}$ and $k\ge 1$, we define
\[
  {\mathcal{L}}_M^{(k)}(s) := {\mathcal{L}}_M^{(k)} \cap ({{\mathcal{D}}^{\!{}^{\circ}\!}})^*{\mathbin{.}} s
    = \{ s_1{\mathbin{.}}\cdots{\mathbin{.}} s_k\in {\mathcal{L}}_M^{(k)} : s_k = s \}
  \;.
\]
\end{notation}

\begin{lemma}[{\cite[Lemma~4.10]{GT13}}]\label{L:RestrictedNF}
If $M$ is a Garside monoid such that all proper simple elements of $M$ are essential and the language ${\mathcal{L}}_M$ is essentially transitive, then one has
$\big|{\mathcal{L}}_M^{(k)}(s)\big| \in \Theta\Big(\big|{\mathcal{L}}_M^{(k)}\big|\Big)$ for all $s\in{{\mathcal{D}}^{\!{}^{\circ}\!}}$.
\end{lemma}

\begin{theorem}\label{T:PSeqProduct}
Assume that $M=G\zs H$ is a Garside {Zappa--Sz{\'e}p}{} product, that all proper simple elements of~$H$ are essential, that the language~${\mathcal{L}}_H$ is essentially transitive, and that the exponential growth rates~$\beta_G$ of~${\mathcal{L}}_G$  and~$\beta_H$ of~${\mathcal{L}}_H$ satisfy $\beta_G,\beta_H>1$.

If $\nu_{k}$ is the uniform probability measure on~${\mathcal{L}}_M^{(k)}$ and~$\mu_{\mathcal{A}}$ is the uniform probability distribution on the set~${\mathcal{A}}$ of atoms of~$M$, then the expected value
$\mathbf{E}_{\nu_{k} \times \mu_{\mathcal{A}}}[{\mathrm{pd}}]$ diverges, that is, one has $\lim_{k\to\infty} \mathbf{E}_{\nu_{k} \times \mu_{\mathcal{A}}}[{\mathrm{pd}}] = \infty$.
\end{theorem}
\begin{proof}
For any atom $a\in{\mathcal{A}}\cap H$, any $g\in{\mathcal{L}}_G^{(k)}$ and any $h\in{\mathcal{L}}_H^{(k)}({\widetilde{\partial}}_H a)$, 
the sequence $\pi_{g,h}$ defined in \autoref{N:PSeqProduct} is a penetration sequence establishing ${\mathrm{pd}}(x_{(g,h)},a)\ge k$ for some $x_{(g,h)}\in M$.  Moreover, the map $(g,h)\mapsto x_{(g,h)}$ is injective.

Using \autoref{L:RestrictedNF} and \autoref{C:GrowthRatesProduct}, we have
\begin{align*}
 \mathbf{E}_{\nu_{k} \times \mu_{\mathcal{A}}}[{\mathrm{pd}}]
   & \ge \sum_{a\in{\mathcal{A}}\cap H} k
                 \frac{\big|{\mathcal{L}}_G^{(k)}\big|\cdot \big|{\mathcal{L}}_H^{(k)}({\widetilde{\partial}}_H a)\big|}
                      {\big|{\mathcal{L}}_M^{(k)}\big|\cdot \big|{\mathcal{A}}\big|}
     \in \Theta(k)
   \;,
\end{align*}
proving $\lim_{k\to\infty}\mathbf{E}_{\nu_{k} \times \mu_{\mathcal{A}}}[{\mathrm{pd}}] = \infty$ as claimed.
\end{proof}

\begin{example}\label{E:UnboundedExpectedPD}
We see in particular that essential transitivity is necessary for the statement of \autoref{T:BoundedExpectedPD}:

Consider $M=G\times H$, where $G=H={\mathsf{{A}}}_2=\langle a,b {\boldsymbol{\mid}} a\,b\,a=b\,a\,b\rangle^+$.
The lattice of simple elements of $G=H$ and the acceptor~$\Gamma$ of ${\mathcal{L}}_G={\mathcal{L}}_H$ are shown in \autoref{F:UnboundedExpectedPD}.
One sees that all proper simple elements of~$G$, respectively of~$H$, and thus of~$M$, are essential and the languages~${\mathcal{L}}_G$ and~${\mathcal{L}}_H$ are essentially transitive, and it is easy to check that $\beta_G=\beta_H=2$.

Hence, by \autoref{T:PSeqProduct},~$M$ has unbounded expected penetration distance.
As every proper simple element of~$M$ is essential, so~$M$ satisfies all the hypotheses of \autoref{T:BoundedExpectedPD}, except for essential transitivity.

  \begin{figure}
  \hfill
  \parbox[c]{0.45\textwidth}{
    \begin{xy}
      0;<3em,0em>:<0em,3em>::
      
      (1,-0.6)*+{\text{Hasse diagram}};
      (1,0)*+{\mathbf{1}}="e";
      (0,1)*+{a}="a";
      (2,1)*+{b}="b";
      (0,2)*+{ab}="ab";
      (2,2)*+{ba}="ba";
      (1,3)*+{\Delta}="Delta";
      
      {\ar@{->}^{a} "e";"a"};
      {\ar@{->}_{b} "e";"b"};
      {\ar@{->}^{b} "a";"ab"};
      {\ar@{->}_{a} "b";"ba"};
      {\ar@{->}^{a} "ab";"Delta"};
      {\ar@{->}_{b} "ba";"Delta"};
    \end{xy}
  } \hfill
  \parbox[c]{0.45\textwidth}{
    \begin{xy}
      0;<3em,0em>:<0em,3em>::
      
      (1,-0.75)*+{\Gamma};
      (0,2)*+{a}="a";
      (2,0)*+{b}="b";
      (0,0)*+{ab}="ab";
      (2,2)*+{ba}="ba";
      
      {\ar@(u,l) "a";"a"};
      {\ar@{->} "a";"ab"};
      {\ar@{->} "ab";"b"};
      {\ar@/_/ "ab";"ba"};
      {\ar@(d,r) "b";"b"};
      {\ar@{->} "b";"ba"};
      {\ar@{->} "ba";"a"};
      {\ar@/_/ "ba";"ab"};
    \end{xy}
  } \hfill {}
  \caption{The lattice of simple elements and the acceptor for the regular language of words in normal form for the monoid from \autoref{E:UnboundedExpectedPD}.}
  \label{F:UnboundedExpectedPD}
  \end{figure}
\end{example}

{}

\section{Artin monoids}\label{S:ArtinMonoids}

The aim of this section is to determine the essential simple elements
of Artin monoids of spherical type and to determine when these monoids
are essentially transitive.

In \cite[Lemma~3.4]{Caruso13}, Caruso shows that, in our terminology,
the language of normal forms of a spherical Artin monoid of type
${\mathsf{{A}}}$ is essentially 5-transitive.  In Lemmas
\ref{connecting-atoms} to \ref{rev-u} we will generalize Caruso's
construction and reproduce this result in \autoref{type-A-ess-trans}.
We then go on to generalize this result to all irreducible spherical
Artin monoids.

{}

\begin{definition}
  Suppose that $M$ is a Garside monoid.  For $x \in M$ the
  \emph{starting set} of $x$ is the set of atoms which are prefixes of
  $x$.  Similarly, the \emph{finishing set} of $x$ is the set of
  atomic suffixes of $x$.
  \begin{align*}
    {S}(x) &= \{ a \in {\mathcal{A}} | a {\preccurlyeq} x \} \\
    {F}(x) &= \{ a \in {\mathcal{A}} | x {\succcurlyeq} a \}
  \end{align*}
\end{definition}

For Artin monoids of spherical type, we have the following result connecting
the normal form condition to the starting and finishing sets.

\begin{theorem}[{\cite[Lemma 4.2]{Charney1993}}]\label{Artin-start-finish}
 If $M$ is an Artin monoid of spherical type with set of atoms ${\mathcal{A}}$ and $s\in{\mathcal{D}}$, then ${S}({\partial} s) = {\mathcal{A}}{\mathbin{\raisebox{0.25ex}{$\smallsetminus$}}}{F}(s)$.
\end{theorem}

\begin{corollary}\label{Artin-normal-form}
  Suppose that $M$ is an Artin monoid of spherical type.  Then, for $x, y \in
  {\mathcal{D}}$, $x | y$ if and only if ${F}(x) \supseteq {S}(y)$. \qed
\end{corollary}
\begin{proof}
We have $x|y$ if and only if ${\partial} x {\wedge} y = {\mathbf{1}}$.  The latter is equivalent to ${S}(\partial x){\cap} {S}(y)=\emptyset$, so the claim follows with \autoref{Artin-start-finish}.
\end{proof}

\begin{proposition}\label{Artin-ProperSimplesEssential}
  Suppose that $M$ is an Artin monoid of spherical type.  Then every proper
  simple element is essential, ${\mathpzc{Ess}} = {{\mathcal{D}}^{\!{}^{\circ}\!}}$.
\end{proposition}
\begin{proof}
  Given any $x \in {{\mathcal{D}}^{\!{}^{\circ}\!}}$ pick an atom $a \in {F}(x)$ and let
  $x_{i+1} = {\mathop{\widetilde{\bigvee}}} {S}(x_i)$, where $x_0 = x_1$ then we have
  \[ \cdots | x_2 | x_1 | x | a | a | \cdots. \]
  Note that $x_i \ne \Delta$ as $\Delta$ is the only element whose
  starting set or finishing set consists of all of the atoms.
\end{proof}

For the rest of this section we will assume that $M$ is an Artin monoid of spherical type.

\begin{proposition}
  If $M$ is reducible then it is not essentially transitive.
\end{proposition}
\begin{proof}
  If $M$ is reducible then $M = M_1 \times M_2 \times \cdots \times
  M_k$ where the $M_i$ are irreducible.  Hence the proposition follows by
  \autoref{product-not-transitive}.
\end{proof}

\begin{lemma} \label{connecting-atoms}
  Suppose that $a$ and $b$ are two atoms which lie in the same
  connected component of the Coxeter graph.  Then there exist
  simple elements $x$ and $y$ such that ${S}(x) = \{ a \}$,
  ${F}(x) = \{ b \}$, ${S}(y) = {\mathcal{A}}{\mathbin{\raisebox{0.25ex}{$\smallsetminus$}}}\{a\}$ and ${F}(y)
  = {\mathcal{A}}{\mathbin{\raisebox{0.25ex}{$\smallsetminus$}}}\{b\}$.
\end{lemma}
\begin{proof}
  Suppose that we have an embedded path $a=a_1
  \overset{i_1}{\text{ --- }} a_2 \overset{i_2}{\text{ --- }} \cdots
  \overset{i_{k-1}}{\text{ --- }} a_k=b$ in the Coxeter graph of $M$.
  Let $x = a_1 a_2 \cdots a_k$.  There are no subwords which match any
  of the relations, hence ${S}(x) = \{ a \}$ and ${F}(x) = \{ b
  \}$.  Furthermore, the only way $x$ can be written as a product is
  as $x = (a_1 \cdots a_p)(a_{p+1} \cdots a_k)$, which can never be in
  normal form.  Hence $x$ has canonical length 1, i.e.\ it is a simple
  element.

  Let $y = {\partial} (a_k \cdots a_2 a_1)$, then ${S}(y) = {\mathcal{A}}
  {\mathbin{\raisebox{0.25ex}{$\smallsetminus$}}} {F}(a_k \cdots a_2 a_1) = {\mathcal{A}} {\mathbin{\raisebox{0.25ex}{$\smallsetminus$}}} \{a\}$ and,
  similarly, ${F}(y) = {\mathcal{A}} {\mathbin{\raisebox{0.25ex}{$\smallsetminus$}}} {S}(a_k \cdots a_2 a_1) =
  {\mathcal{A}} {\mathbin{\raisebox{0.25ex}{$\smallsetminus$}}} \{b\}$.  
\end{proof}

Suppose that $1 \text{ --- } 2 \text{ ---} \cdots \text{--- } k-1$ is
a subgraph of the Coxeter graph of $M$, so we have a standard
parabolic subgroup of type ${\mathsf{{A}}}_{k-1}$.  In this situation we
have a map from ${\mathsf{{A}}}_{k-1}$ to the symmetric group on the set
$\{ 1, 2, \ldots, k \}$ given by mapping each atom $i$ to the
transposition $(i, i+1)$.  This map is a bijection when we restrict to
the set of simple elements of this submonoid.

Suppose that $a \in {\mathcal{A}}_{{\mathsf{{A}}}_{k-1}}$ and $x \in
{\mathcal{D}}_{{\mathsf{{A}}}_{k-1}}$ are an atom and a simple element of this
submonoid.  Let $\pi$ be the permutation induced by $x$.  Then we have
that $a \in {S}(x)$ if and only if $\pi(a+1) < \pi(a)$, and $a \in
{F}(X)$ if and only if $\pi^{-1}(a+1) < \pi^{-1}(a)$.

\begin{lemma}[\cite{Caruso13}]\label{one-to-many}
  Suppose that $1 \text{ --- } 2 \text{ ---} \cdots \text{--- } k-1$ is
  a subgraph of the Coxeter graph of $M$. Let $u = u(1,2,\ldots,k-1)$ be the
  braid which corresponds to the permutation
  \[ \pi_u = \left(\begin{array}{cccccccc}
      1 & 
      2 & 
      \cdots & 
      \left\lfloor \frac{k}{2} \right\rfloor &
      \left\lfloor \frac{k}{2} + 1 \right\rfloor &
      \left\lfloor \frac{k}{2} + 2 \right\rfloor &
      \cdots & 
      k \\
      2 &
      4 &
      \cdots & 
      2\left\lfloor \frac{k}{2} \right\rfloor &
      1 &
      3 &
      \cdots & 
      2\left\lceil \frac{k}{2} \right\rceil - 1
    \end{array}\right)
    \;.
  \]
  Then one has
  \[ 
  {S}(u) = \left\{\left\lfloor\frac{k}{2}\right\rfloor\right\}
  \qquad
  {F}(u) = \left\{1,3,\ldots,2\left\lfloor\frac{k}{2}\right\rfloor-1\right\}.
  \]
\end{lemma}
\begin{proof}
  It is clear that the only atom $a \in
  {\mathcal{A}}_{{\mathsf{{A}}}_{k-1}}$ for which $\pi_u(a+1) < \pi_u(a)$ is~$a=\left\lfloor\frac{k}{2}\right\rfloor$, hence ${S}(u) = \left\{\left\lfloor\frac{k}{2}\right\rfloor\right\}$.  Similarly,
  $\pi_u^{-1}(a+1) < \pi_u^{-1}(a)$ if and only if~$a$ is odd, hence
  ${F}(u)$ consists of all the odd atoms.
\end{proof}

\begin{lemma}[\cite{Caruso13}]\label{rev-u}
  Suppose that $1 \text{ --- } 2 \text{ ---} \cdots \text{--- } k-1$, where $k>2$, is
  a subgraph of the Coxeter graph of $M$, let $u$ be the element defined in \autoref{one-to-many},
  and let 
  \[ 
  v = v(1,2,\ldots,k) = (\operatorname{rev} u) \cdot D
  \;,
  \]
  where $D = {\bigvee} \left\{ a \in {\mathcal{A}}_M {\boldsymbol{\mid}} a \ne \left\lfloor\tfrac{k}{2}\right\rfloor\right\}$.
  Then $v$ is a simple element and its finishing set
  contains every atom except possibly $\left\lfloor\frac{k}{2}\right\rfloor$,
  that is, ${F}(v) \supseteq {\mathcal{A}}_M {\mathbin{\raisebox{0.25ex}{$\smallsetminus$}}}
  \left\{\left\lfloor\frac{k}{2}\right\rfloor\right\}$.
\end{lemma}
\begin{proof}
  As ${S}(u) = \left\{\left\lfloor\frac{k}{2}\right\rfloor\right\}$
  we have that ${F}(\operatorname{rev} u) =
  \left\{\left\lfloor\frac{k}{2}\right\rfloor\right\}$ hence
  ${S}({\partial}\operatorname{rev} u) =
  {\mathcal{A}}{\mathbin{\raisebox{0.25ex}{$\smallsetminus$}}}\left\{\left\lfloor\frac{k}{2}\right\rfloor\right\}$.  Therefore
  ${\bigvee}\left({\mathcal{A}}{\mathbin{\raisebox{0.25ex}{$\smallsetminus$}}}\{\left\lfloor\frac{k}{2}\right\rfloor\}\right)
  {\preccurlyeq} {\partial}\operatorname{rev} u$ and so $v$ is simple.

  The element $D$ is a Garside element of a standard parabolic subgroup of $M$, thus balanced, whence one has
  ${\mathcal{A}}_M {\mathbin{\raisebox{0.25ex}{$\smallsetminus$}}} \left\{\left\lfloor\frac{k}{2}\right\rfloor\right\} \subseteq {F}(D) \subseteq {F}(v)$.
\end{proof}

\begin{proposition}[\cite{Caruso13}] \label{type-A-ess-trans}
  Suppose that $M$ is the Artin monoid of type ${\mathsf{{A}}}_{n-1}$.\, where $n>2$.
  \[
    \begin{xy}
      0;<3em,0em>:<0em,3em>::
      
      (1,0)*+{1}="1";
      (2,0)*+{2}="2";
      (3,0)*+{3}="3";
      (3.75,0)="4";
      (5,0)="5";
      (6,0)*+{n-1}="6";
      
      {\ar@{-}     "1";"2"};
      {\ar@{-}     "2";"3"};
      {\ar@{-}     "3";"4"};
      {\ar@{..}    "4";"5"};
      {\ar@{-}     "5";"6"};
    \end{xy}
  \]
  Then $M$ is essentially 5-transitive.
\end{proposition}
\begin{proof}
  By \autoref{Artin-ProperSimplesEssential}, one has ${\mathpzc{Ess}}={{\mathcal{D}}^{\!{}^{\circ}\!}}\neq\emptyset$.
  Suppose $x, y \in {{\mathcal{D}}^{\!{}^{\circ}\!}}$.  We will construct elements $x_1, x_2,
  x_3, x_4 \in {{\mathcal{D}}^{\!{}^{\circ}\!}}$ that satisfy $x|x_1|x_2|x_3|x_4|y$.

  Suppose that $a$ is an atom in the finishing set of $x$.  By
  \autoref{connecting-atoms}, there exists $x_1$ such that
  ${S}(x_1) = \{ a \}$ and ${F}(x_1) =
  \left\{\left\lfloor\frac{n}{2}\right\rfloor\right\}$.  Let $x_2 = u(1,2,\ldots,n-1)$,
  so by \autoref{one-to-many} we have ${S}(x_2) =
  \left\lfloor\frac{n}{2}\right\rfloor$.  So by
  \autoref{Artin-normal-form} we have $x|x_1|x_2$.

  Similarly, suppose that $b$ is an atom not in the starting set of
  $y$.  By \autoref{connecting-atoms}, there exists $x_4$ such that
  ${F}(x_4) = {\mathcal{A}} {\mathbin{\raisebox{0.25ex}{$\smallsetminus$}}} \{b\}$ and ${S}(x_4) = {\mathcal{A}}
  {\mathbin{\raisebox{0.25ex}{$\smallsetminus$}}} \left\{\left\lfloor\frac{n}{2}\right\rfloor\right\}$.  Let
  $x_3 = v(1,2,\ldots,n-1)$, so by \autoref{rev-u} we have
  ${F}(x_3)\supseteq{\mathcal{A}}{\mathbin{\raisebox{0.25ex}{$\smallsetminus$}}}\left\{\left\lfloor\frac{n}{2}\right\rfloor\right\}$, whence
  we have $x_3|x_4|y$ by \autoref{Artin-normal-form}.

  It remains to show that $x_2|x_3$, or equivalently that ${F}(x_2)
  \supseteq {S}(x_3)$.  We have ${F}(x_2) =
  \left\{1,3,\ldots,2\left\lfloor\frac{k}{2}\right\rfloor-1\right\}$ by \autoref{one-to-many}.
  
  Consider the permutation induced by $x_3=v$.  The permutation
  induced by $\operatorname{rev} u$ takes the set of even numbered strings to
  $\{1,\ldots,\lfloor\frac{n}{2}\rfloor\}$ and the set of odd numbered
  strings to $\{\lfloor\frac{n}{2}+1\rfloor,\ldots,n\}$, and
  ${\bigvee}\left\{ a \in {\mathcal{A}}_M | a \ne
    \left\lfloor\frac{k}{2}\right\rfloor\right\}$ consist of a
  half-twist on both of these subsets.  Hence, for all $i$ we
  have
  \[ (2i\pm 1)\cdot x_3 < (2i)\cdot x_3. \]
  Therefore ${S}(x_3) = \{ 1, 3, \ldots
  2\left\lfloor\frac{k}{2}\right\rfloor - 1\} = {F}(x_2)$, i.e.\ we
  have $x_2|x_3$.
\end{proof}

{}

\begin{proposition}
  Suppose that $M$ is the Artin monoid of type ${\mathsf{{B}}}_n$, where $n>2$.
  \[
    \begin{xy}
      0;<3em,0em>:<0em,3em>::
      
      (0,0)*+{0}="0";
      (1,0)*+{1}="1";
      (2,0)*+{2}="2";
      (3,0)*+{3}="3";
      (3.75,0)="4";
      (5,0)="5";
      (6,0)*+{n-1}="6";
      
      {\ar@{-}^{4}  "0";"1"};
      {\ar@{-}     "1";"2"};
      {\ar@{-}     "2";"3"};
      {\ar@{-}     "3";"4"};
      {\ar@{..}    "4";"5"};
      {\ar@{-}     "5";"6"};
    \end{xy}
  \]
  Then $M$ is essentially 5-transitive.
\end{proposition}
\begin{proof}
  By \autoref{Artin-ProperSimplesEssential}, one has ${\mathpzc{Ess}}={{\mathcal{D}}^{\!{}^{\circ}\!}}\neq\emptyset$.
  As in the proof of the previous proposition, by
  \autoref{connecting-atoms}, it suffices to construct elements $x_2,
  x_3 \in {{\mathcal{D}}^{\!{}^{\circ}\!}}$ such that ${S}(x_2) =
  \left\{\left\lfloor\frac{n}{2}\right\rfloor\right\}$, $x_2 | x_3$
  and ${F}(x_3) \supseteq {\mathcal{A}} {\mathbin{\raisebox{0.25ex}{$\smallsetminus$}}}
  \left\{\left\lfloor\frac{n}{2}\right\rfloor\right\}$.

  Let $x_2 = u(1,2, \ldots, n-1)$ and $x_3 = v(1,2,\ldots,n-1)$.  

  Simple elements in ${\mathsf{{B}}}_n$ correspond to signed permutations
  of $\{1,\ldots,n\}$.  The atoms $i = 1, 2, \ldots n-1$ give the
  transposition $(i,i+1)(-i,-i-1)$ and $0$ is the transposition
  $(1,-1)$ which changes the sign of $1$.

  The starting set can be computed from the induced signed
  permutation \cite[Proposition 8.1.2]{CombCox}.
  \[
    {S}(x_3) = \{ i \in {\mathcal{A}} {\boldsymbol{\mid}} i\cdot x_3 > (i+1)\cdot x_3\}
  \]
  Where for $0$ we use $0\cdot x_3 = 0$.

  The signed permutation induced by $\operatorname{rev} u$ takes every odd number to
  a number greater than $\frac{n}{2}$, and every even number to a
  number less than or equal to $\frac{n}{2}$.  The action of
  $\Delta_{{\mathcal{A}}{\mathbin{\raisebox{0.25ex}{$\smallsetminus$}}}\{\lfloor\frac{n}{2}\rfloor\}}$ performs a
  half twist of the numbers greater than $\frac{n}{2}$ and changes the
  sign of the numbers less than or equal to $\frac{n}{2}$.  Hence, for
  all $i$ we have
  \[ (2i\pm 1)\cdot x_3 > (2i)\cdot x_3. \] 
  Therefore ${S}(x_3) = \{ 1, 3, \ldots
  2\left\lfloor\frac{n}{2}\right\rfloor - 1\} = {F}(x_2)$, i.e.\ we
  have $x_2|x_3$.
\end{proof}

\begin{proposition}
  Suppose that $M$ is the Artin monoid of type ${\mathsf{{D}}}_n$, where $n>2$.
  \[
    \begin{xy}
      0;<3em,0em>:<0em,3em>::
      
      (0,1)*+{0}="0";
      (0,-1)*+{1}="1";
      (1,0)*+{2}="2";
      (2,0)*+{3}="3";
      (3,0)*+{4}="4";
      (3.75,0)="5";
      (5,0)="6";
      (6,0)*+{n-1}="7";
      
      {\ar@{-}  "0";"2"};
      {\ar@{-}  "1";"2"};
      {\ar@{-}  "2";"3"};
      {\ar@{-}  "3";"4"};
      {\ar@{-} "4";"5"};
      {\ar@{..}  "5";"6"};
      {\ar@{-}  "6";"7"};
    \end{xy}
  \]
  Then $M$ is essentially 5-transitive.
\end{proposition}
\begin{proof}
  By \autoref{Artin-ProperSimplesEssential}, one has ${\mathpzc{Ess}}={{\mathcal{D}}^{\!{}^{\circ}\!}}\neq\emptyset$.
  As in the proof of the previous propositions, it suffices to
  construct elements $x_2, x_3 \in {{\mathcal{D}}^{\!{}^{\circ}\!}}$ such that ${S}(x_2) =
  \left\{\left\lfloor\frac{n}{2}\right\rfloor\right\}$, $x_2 | x_3$
  and ${F}(x_3) \supseteq {\mathcal{A}} {\mathbin{\raisebox{0.25ex}{$\smallsetminus$}}}
  \left\{\left\lfloor\frac{n}{2}\right\rfloor\right\}$.

  Let $x_2 = u(1,2,\ldots,n-1)$ and $x_3 = v(1,2,\ldots,n-1)$.

  Simple elements in ${\mathsf{{B}}}_n$ correspond to signed permutations
  of $\{1,\ldots,n\}$.  The atoms $i = 1, 2, \ldots n-1$ give the
  transposition $(i,i+1)(-i,-i-1)$ and $0$ is the transposition
  $(1,-1)$ which changes the sign of $1$.

  Simple elements in ${\mathsf{{D}}}_n$ correspond to the subset of signed
  permutations of $\{1,\ldots,n\}$ which consists of all elements that
  change the sign of an even number of numbers.  As in the type
  ${\mathsf{{B}}}_n$ case, the atoms $i = 1, 2, \ldots n-1$ give the
  transposition $(i,i+1)(-i,-i-1)$, but now $0$ is the transposition
  $(1,-2)(2,-1)$.

  The starting set can be computed from the induced signed
  permutation \cite[Proposition 8.2.2]{CombCox}.
  \[
    {S}(x_3) = \{ i \in {\mathcal{A}} {\boldsymbol{\mid}} i\cdot x_3 > (i+1)\cdot x_3\}
  \]
  Where for 0 we use $0\cdot x_3 = -2\cdot x_3$.

  The signed permutation induced by $\operatorname{rev} u$ takes every odd number to
  a number greater than~$\frac{n}{2}$, and every even number to a
  number less than or equal to~$\frac{n}{2}$.{}{}
  The action of
  $\Delta_{{\mathcal{A}}{\mathbin{\raisebox{0.25ex}{$\smallsetminus$}}}\{\lfloor\frac{n}{2}\rfloor\}}$ performs a
  half twist of the numbers greater than $\frac{n}{2}$ and changes the
  sign of the numbers less than or equal to $\frac{n}{2}$.  Hence, for
  all $i$ we have
  \[ (2i\pm 1)\cdot x_3 > (2i)\cdot x_3. \] 
  Therefore ${S}(x_3) = \{ 1, 3, \ldots
  2\left\lfloor\frac{n}{2}\right\rfloor - 1\} = {F}(x_2)$, i.e.\ we
  have $x_2|x_3$.
\end{proof}

\begin{proposition}
  Suppose that $M$ is the Artin monoid of type ${\mathsf{{E}}}_6$.
  \[
    \begin{xy}
      0;<3em,0em>:<0em,3em>::
      
      (0,0)*+{1}="1";
      (1,0)*+{2}="2";
      (2,0)*+{3}="3";
      (3,0)*+{4}="4";
      (4,0)*+{5}="5";
      (2,1)*+{6}="6";
      
      {\ar@{-}     "1";"2"};
      {\ar@{-}     "2";"3"};
      {\ar@{-}     "3";"4"};
      {\ar@{-}     "4";"5"};
      {\ar@{-}     "3";"6"};
    \end{xy}
  \]
  Then $M$ is essentially 5-transitive.
\end{proposition}
\begin{proof}
  By \autoref{Artin-ProperSimplesEssential}, one has ${\mathpzc{Ess}}={{\mathcal{D}}^{\!{}^{\circ}\!}}\neq\emptyset$.
  Following the same pattern as previous proofs, it suffices to
  construct elements $x_2, x_3 \in {{\mathcal{D}}^{\!{}^{\circ}\!}}$ such that ${S}(x_2) =
  \{3\}$, $x_2 | x_3$ and ${F}(x_3) \supseteq {\mathcal{A}} {\mathbin{\raisebox{0.25ex}{$\smallsetminus$}}} \{3\}$.

  Let $x_2 = u(1,2,3,4,5) = 324135$ and $x_3 = v(1,2,3,4,5) = 1352431214546$.
  By direct computation we have the following starting and finishing sets.
  \begin{align*}
    {S}(x_2) &= \{ 3 \} &
    {F}(x_2) &= \{ 1, 3, 5 \} \\
    {S}(x_3) &= \{ 1, 3, 5 \} &
    {F}(x_3) &= {\mathcal{A}}{\mathbin{\raisebox{0.25ex}{$\smallsetminus$}}}\{ 3 \}
  \end{align*}
\end{proof}

\begin{proposition}
  Suppose that $M$ is the Artin monoid of type ${\mathsf{{E}}}_7$.
  \[
    \begin{xy}
      0;<3em,0em>:<0em,3em>::
      
      (0,0)*+{1}="1";
      (1,0)*+{2}="2";
      (2,0)*+{3}="3";
      (3,0)*+{4}="4";
      (4,0)*+{5}="5";
      (5,0)*+{6}="6";
      (2,1)*+{7}="7";
      
      {\ar@{-}     "1";"2"};
      {\ar@{-}     "2";"3"};
      {\ar@{-}     "3";"4"};
      {\ar@{-}     "4";"5"};
      {\ar@{-}     "5";"6"};
      {\ar@{-}     "3";"7"};
    \end{xy}
  \]
  Then $M$ is essentially 5-transitive.
\end{proposition}
\begin{proof}
  By \autoref{Artin-ProperSimplesEssential}, one has ${\mathpzc{Ess}}={{\mathcal{D}}^{\!{}^{\circ}\!}}\neq\emptyset$.
  Again, it suffices to construct elements $x_2, x_3 \in {{\mathcal{D}}^{\!{}^{\circ}\!}}$
  such that one has ${S}(x_2) = \{3\}$, $x_2 | x_3$ and ${F}(x_3) \supseteq {\mathcal{A}}
  {\mathbin{\raisebox{0.25ex}{$\smallsetminus$}}} \{3\}$.
  
  We define $x_2 = u(1,2,3,4,5,6) = 324135$ and $x_3 = v(1,2,3,4,5,6) = 135243 121 456454 7$.
  By direct computation we have the following starting and finishing sets.
  \begin{align*}
    {S}(x_2) &= \{ 3 \} &
    {F}(x_2) &= \{ 1, 3, 5 \} \\
    {S}(x_3) &= \{ 1, 3, 5 \} &
    {F}(x_3) &= {\mathcal{A}}{\mathbin{\raisebox{0.25ex}{$\smallsetminus$}}}\{ 3 \}
  \end{align*}
\end{proof}

\begin{proposition}
  Suppose that $M$ is the Artin monoid of type ${\mathsf{{E}}}_8$.
  \[
    \begin{xy}
      0;<3em,0em>:<0em,3em>::
      
      (0,0)*+{1}="1";
      (1,0)*+{2}="2";
      (2,0)*+{3}="3";
      (3,0)*+{4}="4";
      (4,0)*+{5}="5";
      (5,0)*+{6}="6";
      (6,0)*+{7}="7";
      (2,1)*+{8}="8";
      
      {\ar@{-}     "1";"2"};
      {\ar@{-}     "2";"3"};
      {\ar@{-}     "3";"4"};
      {\ar@{-}     "4";"5"};
      {\ar@{-}     "5";"6"};
      {\ar@{-}     "6";"7"};
      {\ar@{-}     "3";"8"};
    \end{xy}
  \]
  Then $M$ is essentially 5-transitive.
\end{proposition}
\begin{proof}
  By \autoref{Artin-ProperSimplesEssential}, one has ${\mathpzc{Ess}}={{\mathcal{D}}^{\!{}^{\circ}\!}}\neq\emptyset$.
  Again, it suffices to construct elements $x_2, x_3 \in {{\mathcal{D}}^{\!{}^{\circ}\!}}$
  such that ${S}(x_2) = \{4\}$, $x_2 | x_3$ and ${F}(x_3) \supseteq {\mathcal{A}}
  {\mathbin{\raisebox{0.25ex}{$\smallsetminus$}}} \{4\}$.

  Let $x_2 = u(1,2,3,4,5,6,7) = 4352461357$ and $x_3 =
  v(1,2,3,4,5,6,7) = 1357246354 1238123121 567565$.  By direct
  computation we have the following starting and finishing sets.
  \begin{align*}
    {S}(x_2) &= \{ 4 \} &
    {F}(x_2) &= \{ 1, 3, 5, 7 \} \\
    {S}(x_3) &= \{ 1, 3, 5, 7 \} &
    {F}(x_3) &= {\mathcal{A}}{\mathbin{\raisebox{0.25ex}{$\smallsetminus$}}}\{ 4 \}
  \end{align*}
\end{proof}

\begin{proposition}
  Suppose that $M$ is the Artin monoid of type ${\mathsf{{F}}}_4$.
  \[
    \begin{xy}
      0;<3em,0em>:<0em,3em>::
      
      (0,0)*+{1}="1";
      (1,0)*+{2}="2";
      (2,0)*+{3}="3";
      (3,0)*+{4}="4";
      
      {\ar@{-}     "1";"2"};
      {\ar@{-}^{4}  "2";"3"};
      {\ar@{-}     "3";"4"};
    \end{xy}
  \]
  Then $M$ is essentially 4-transitive.
\end{proposition}
\begin{proof}
  By \autoref{Artin-ProperSimplesEssential}, one has ${\mathpzc{Ess}}={{\mathcal{D}}^{\!{}^{\circ}\!}}\neq\emptyset$.
  By \autoref{connecting-atoms}, it suffices to construct an element
  $x_2 \in {{\mathcal{D}}^{\!{}^{\circ}\!}}$ such that ${S}(x_2) = \{3\}$ and ${F}(x_2) =
  {\mathcal{A}} {\mathbin{\raisebox{0.25ex}{$\smallsetminus$}}} \{2\}$.

  Let $x_2 = 321343$.  By direct computation we have
  the required starting and finishing sets.
\end{proof}

\begin{proposition}
  Suppose that $M$ is the Artin monoid of type ${\mathsf{{H}}}_3$.
  \[
    \begin{xy}
      0;<3em,0em>:<0em,3em>::
      
      (0,0)*+{1}="1";
      (1,0)*+{2}="2";
      (2,0)*+{3}="3";
      
      {\ar@{-}^{5}  "1";"2"};
      {\ar@{-}     "2";"3"};
    \end{xy}
  \]
  Then $M$ is essentially 4-transitive.
\end{proposition}
\begin{proof}
  By \autoref{Artin-ProperSimplesEssential}, one has ${\mathpzc{Ess}}={{\mathcal{D}}^{\!{}^{\circ}\!}}\neq\emptyset$.
  By \autoref{connecting-atoms}, it suffices to construct an element
  $x_2 \in {{\mathcal{D}}^{\!{}^{\circ}\!}}$ such that ${S}(x_2) = \{2\}$ and ${F}(x_2) =
  {\mathcal{A}} {\mathbin{\raisebox{0.25ex}{$\smallsetminus$}}} \{2\}$.

  Let $x_2 = 213$.  By direct computation we have the required
  starting and finishing sets.
\end{proof}

\begin{proposition}
  Suppose that $M$ is the Artin monoid of type ${\mathsf{{H}}}_4$.
  \[
    \begin{xy}
      0;<3em,0em>:<0em,3em>::
      
      (0,0)*+{1}="1";
      (1,0)*+{2}="2";
      (2,0)*+{3}="3";
      (3,0)*+{4}="4";
      
      {\ar@{-}^{5}  "1";"2"};
      {\ar@{-}     "2";"3"};
      {\ar@{-}     "3";"4"};
    \end{xy}
  \]
  Then $M$ is essentially 5-transitive.
\end{proposition}
\begin{proof}
  By \autoref{connecting-atoms}, it suffices to construct elements
  $x_2, x_3 \in {{\mathcal{D}}^{\!{}^{\circ}\!}}$ such that ${S}(x_2) = \{3\}$, $x_2 |
  x_3$ and ${F}(x_3) = {\mathcal{A}} {\mathbin{\raisebox{0.25ex}{$\smallsetminus$}}} \{3\}$.

  Let $x_2 = 324$ and $x_3 = 243121214$.  By direct
  computation we have the following starting and finishing sets.
  \begin{align*}
    {S}(x_2) &= \{ 3 \} &
    {F}(x_2) &= \{ 2, 4 \} \\
    {S}(x_3) &= \{ 2, 4 \} &
    {F}(x_3) &= {\mathcal{A}}{\mathbin{\raisebox{0.25ex}{$\smallsetminus$}}}\{ 3 \}
  \end{align*}
\end{proof}

\begin{proposition}
  Suppose that $M$ is the Artin monoid of type
  ${\mathsf{{I}}}_2(p)$, where $p \ge 3$.
  \[
    \begin{xy}
      0;<3em,0em>:<0em,3em>::
      
      (0,0)*+{a}="1";
      (1,0)*+{b}="2";
      
      {\ar@{-}^{p}  "1";"2"};
    \end{xy}
  \]
  Then $M$ is essentially 2-transitive.
\end{proposition}
\begin{proof}
  We have $p\ge3$ by assumption, so $ab$ and $ba$ are distinct proper simple elements.  The proper simple elements fall into one of four types:\smallskip

  \begin{tabular}{l@{\hspace{1ex}}l@{\qquad}l@{\hspace{1ex}}l}
   (1) & $b(ab)^*=(ba)^*b$ & (2) & $a(ba)^*b$ \\[0.5ex]
   (3) & $b(ab)^*a$        & (4) & $a(ba)^*=(ab)^*a$
  \end{tabular}
  \smallskip

  Writing $(i)$ and $(i')$ for arbitrary simple elements of type $i$, we have\smallskip

  \begin{tabular}{l@{\qquad}l@{\qquad}l@{\qquad}l}
  $(1)|(1')$    & $(1)|ba|(2')$ & $(1)|(3')$    & $(1)|ba|(4')$ \\[0.5ex]
  $(2)|(1')$    & $(2)|ba|(2')$ & $(2)|(3')$    & $(2)|ba|(4')$ \\[0.5ex]
  $(3)|ab|(1')$ & $(3)|(2')$    & $(3)|ab|(3')$ & $(3)|(4')$    \\[0.5ex]
  $(4)|ab|(1')$ & $(4)|(2')$    & $(4)|ab|(3')$ & $(4)|(4')$
  \end{tabular}
  \smallskip

  \noindent and thus $2$-transitivity.
\end{proof}

Combining the classification of Artin monoids of spherical type with the above
propositions we have the following theorem.

\begin{theorem}\label{Artin-EssentiallyTransitive}
  Let $M$ be an Artin monoid of spherical type with more than one atom.
  
  The language of normal forms in $M$ is essentially transitive if and only if~$M$ is irreducible.  Moreover,
  if the language is essentially transitive then it is essentially $5$-transitive. \qed
\end{theorem}

\begin{corollary}\label{C:BoundedExpectedPD-Artin}
Let $M$ be an irreducible Artin monoid of spherical type, let $\nu_{k}$ be the uniform probability measure on ${\mathcal{L}}_M^{(k)}$, and let $\mu_{\mathcal{A}}$ be the uniform probability distribution on the set ${\mathcal{A}}$ of atoms of $M$.

The expected value $\mathbf{E}_{\nu_{k} \times \mu_{\mathcal{A}}}[{\mathrm{pd}}]$ of the penetration distance with respect to $\nu_k\times \mu_{\mathcal{A}}$ is uniformly bounded (that is, bounded independently of $k$).
\end{corollary}
\begin{proof}
\autoref{C:BoundedExpectedPD}, \autoref{Artin-ProperSimplesEssential} and \autoref{Artin-EssentiallyTransitive} imply the claim.
\end{proof}

Recall that $\beta_M$ is the exponential growth rate of the regular language~${\mathcal{L}}_M^{(k)}$.

\begin{lemma}\label{Artin-Growth}
If $M$ is an irreducible Artin monoid of spherical type with more than one atom, then one has $\beta_M>1$.
\end{lemma}
\begin{proof}
As there is only one monoid, we drop the subscript $M$.

Consider two atoms~$a\neq b$ of~$M$.  As~${\mathcal{L}}_M$ is essentially transitive, there exist $s_1,\ldots,s_k,t_1,\ldots,t_l\in{{\mathcal{D}}^{\!{}^{\circ}\!}}$ such that $a|s_1|\cdots|s_k|b|t_1|\cdots t_l|a$.  Moreover, we have $a|a$ by \autoref{Artin-normal-form}.
Thus, one has ${\mathcal{L}}^{\left(N(k+l+2)\right)} \ge 2^N$, showing the claim.
\end{proof}

\begin{corollary}\label{C:UnboundedExpectedPD-ZappaSzep-Artin}
Let $M=G\zs H$, where $G$ and $H$ are irreducible Artin monoids of spherical type with more than one atom, let $\nu_{k}$ be the uniform probability measure on~${\mathcal{L}}_M^{(k)}$, and let~$\mu_{\mathcal{A}}$ be the uniform probability distribution on the set~${\mathcal{A}}$ of atoms of~$M$.

The expected value
$\mathbf{E}_{\nu_{k} \times \mu_{\mathcal{A}}}[{\mathrm{pd}}]$ diverges, that is, $\lim_{k\to\infty} \mathbf{E}_{\nu_{k} \times \mu_{\mathcal{A}}}[{\mathrm{pd}}] = \infty$.
\end{corollary}
\begin{proof}
\autoref{T:PSeqProduct}, \autoref{Artin-ProperSimplesEssential}, \autoref{Artin-EssentiallyTransitive} and \autoref{Artin-Growth} imply the claim.
\end{proof}

Using the terminology of this paper, Dehornoy asked in~\cite[Question~3.13]{DehornoyJCTA07} whether for the braid monoid, that is the Artin monoid of type~${\mathsf{{A}}}_n$, one has $|{\mathcal{L}}(s)^{(k)}|\in\Theta\big({\mathcal{L}}^{(k)}\big)$ for all~$s\in{{\mathcal{D}}^{\!{}^{\circ}\!}}$.  The answer is affirmative for all irreducible Artin monoids of spherical type:

\begin{corollary}\label{C:Dehornoy}
If $M$ is an irreducible Artin monoid of spherical type and $s\in{{\mathcal{D}}^{\!{}^{\circ}\!}}$, then one has
$|{\mathcal{L}}(s)^{(k)}|\in\Theta\big({\mathcal{L}}^{(k)}\big)$.
\end{corollary}
\begin{proof}
The claim follows from \autoref{L:RestrictedNF}, \autoref{Artin-ProperSimplesEssential} and
\autoref{Artin-EssentiallyTransitive}.
\end{proof}

\bibliographystyle{alpha-sjt}
\bibliography{bibliography}

\bigskip
\noindent
\begin{minipage}[t]{0.45\textwidth}
\noindent\textbf{Volker Gebhardt}\\
\noindent E-mail: \texttt{v.gebhardt@uws.edu.au}
\end{minipage}
\hfill
\begin{minipage}[t]{0.49\textwidth}
\noindent\textbf{Stephen Tawn}\\
\noindent E-mail: \texttt{stephen@tawn.co.uk}\\
\noindent URL: \url{http://www.stephentawn.info}
\end{minipage}
\medskip
\begin{center}
University of Western Sydney\\
Centre for Research in Mathematics\\
Locked Bag 1797, Penrith NSW 2751, Australia\\
\noindent URL: \url{http://www.uws.edu.au/crm}
\end{center}

\end{document}

