
\documentclass[11pt,reqno]{amsart}
\usepackage{graphicx, enumerate, url}
\usepackage[margin=1in]{geometry}
\usepackage{amssymb,mathrsfs}
\usepackage{color}
\newtheorem{theorem}{Theorem}[section]

\newtheorem{proposition}[theorem]{Proposition}
\newtheorem{lemma}[theorem]{Lemma}
\newtheorem{corollary}[theorem]{Corollary}
\newtheorem{conjecture}[theorem]{Conjecture}

\theoremstyle{definition}
\newtheorem{definition}[theorem]{Definition}
\newtheorem{example}[theorem]{Example}
\newtheorem{problem}[theorem]{Problem}
\newtheorem{question}[theorem]{Question}

\theoremstyle{remark}
\newtheorem{remark}[theorem]{Remark}

\begin{document}
\title{The Computational Complexity of Duality}
\author[S.~Friedland]{Shmuel~Friedland}
\address{Department of Mathematics, Statistics and Computer Science,  University of Illinois, Chicago}
\email{friedlan@uic.edu}
\author[L.-H.~Lim]{Lek-Heng~Lim}
\address{Computational and Applied Mathematics Initiative, Department of Statistics,
University of Chicago}
\email{lekheng@galton.uchicago.edu}
\begin{abstract}
We show that for any given norm ball or proper cone, weak membership in its dual ball or dual cone  is polynomial-time reducible to  weak membership in the given ball or cone. A consequence is that the weak membership or membership problem for a ball or cone is NP-hard if and only if the corresponding problem for the dual ball or cone is NP-hard. In a similar vein, we show that computation of the dual norm of a given norm is polynomial-time reducible to computation of the given norm. This extends to convex functions satisfying a polynomial growth condition: for such a given function, computation of its Fenchel dual/conjugate is polynomial-time reducible to computation of the given function. Hence the computation of a norm or a convex function of polynomial-growth is NP-hard if and only if the computation of its dual norm or Fenchel dual is NP-hard. We discuss implications of these results on the weak membership problem for a symmetric convex body and its polar dual, the polynomial approximability of Mahler volume, and the weak membership problem for the epigraph of a convex function with polynomial growth and that of its Fenchel dual.
\end{abstract}

\keywords{dual norm, dual cone, Fenchel dual, NP-hard, weak membership, approximation}

\subjclass[2010]{15B48, 52A41, 65F35, 90C46, 90C60}

\maketitle

\section{Introduction}

In convex optimization, we often encounter problems that involve one of the following notions of duality. For convex sets: (i) norm balls and their polar duals, (ii) proper cones and their dual cones; for convex functions:
(iii) norms and their dual norms; (iv) functions and their Fenchel duals. The main goal of this article is to establish the equivalence between the polynomial-time computability or NP-hardness of these objects and their duals.

We will first show in Section~\ref{sec:norm}  that the weak membership problem for a norm ball is NP-hard (resp.\ is polynomial-time) if and only if the weak membership problem for its dual norm ball is NP-hard (resp.\ is polynomial-time). For readers unfamiliar with the notion, NP-hardness of \emph{weak} membership is a \emph{stronger} statement than NP-hardness of membership, i.e., the latter is implied by the former. Since every symmetric convex compact set with nonempty interior is a norm ball, the result applies to such objects and their polar duals as well.

In Section~\ref{sec:approx} we show that the approximation of a norm to arbitrary precision is NP-hard (resp.\ is polynomial time) if and only if  weak membership  in the unit ball of the norm is NP-hard (resp.\ is NP-hard).  A consequence is that if the weak membership problem for a norm ball is polynomial-time decidable, then its Mahler volume is polynomial-time approximable. In fact, computation of Mahler volume is polynomial-time reducible to the weak membership problem for a norm ball.

In Section~\ref{sec:cones}, we establish an analogue of our norm ball result for proper cones, showing that  the weak membership problem for such a cone  is NP-hard (resp.\ is polynomial-time) if and only if the weak membership problem for its dual cone can be decided is NP-hard (resp.\ is polynomial-time).

We conclude by showing in Section~\ref{sec:fenchel} that for convex functions that satisfy a polynomial-growth condition, its Fenchel dual must also satisfy the same condition with possibly different constants. A consequence of this is that such a function is polynomial-time approximable to arbitrary precision if and only if its Fenchel dual is also  polynomial-time approximable to arbitrary precision. On the other hand, such a function is NP-hard to approximate if and only if its Fenchel dual is NP-hard to approximate.

\section{Weak membership, weak validity, and polynomial-time reducibility}\label{sec:weak}

We introduce some basic terminologies based on \cite[Chapter~2]{GLS88}) with some natural extensions for our context. 
Let $B(x,\delta)$ denote the closed Euclidean norm ball of radius $\delta > 0$ centered at $x$ in $\mathbb{R}^n$.
For any $\delta>0$ and any $K \subseteq \mathbb{R}^n$,  we define respectively a `thickened' $K$ and a `shrunkened' $K$ by
\begin{equation}\label{eq:thick-shrunk}
S(K,\delta):=\bigcup_{x\in K} B(x,\delta)\quad \text{and}\quad S(K,-\delta):=\{x\in K: B(x,\delta)\subseteq K\}.
\end{equation}
Note that if $K$ has no interior point, then $S(K,-\delta) = \varnothing$.

\begin{definition}\label{def:mem}
Let $K \subseteq \mathbb{R}^n$ be a convex set with nonempty interior.
\begin{enumerate}[\upshape (i)]
\item The \emph{membership problem} (\textsc{mem}) for $K$ is: Given $x\in\mathbb{Q}^n$, determine if $x$ is in $K$.

\item The \emph{weak membership problem} (\textsc{wmem}) for $K$ is: Given $x\in\mathbb{Q}^n$ and a rational $\delta>0$,  assert that $x\in S(K,\delta)$ or $x\notin S(K,-\delta)$.  

\item The \emph{weak validity problem} (\textsc{wval}) problem for $K$ is: Given $c\in\mathbb{Q}^n$ and rational $\gamma,\varepsilon>0$,
either assert that $c^\mathsf{T} x\le \gamma +\varepsilon$ for all $x\in S(K,-\varepsilon)$, or assert that $c^\mathsf{T} x \ge \gamma -\varepsilon$ for some $x\in S(K,\varepsilon)$.

\item The \emph{weak optimization problem} (\textsc{wopt}) problem for $K$ is: Given $c\in\mathbb{Q}^n$ and a rational $\varepsilon>0$,
either find $y \in\mathbb{Q}^n$ such that  $y\in S(K,\varepsilon)$ and $c^\mathsf{T} x\le c^\mathsf{T} y +\varepsilon$ for all $x\in S(K,-\varepsilon)$, or assert that $S(K,\varepsilon) = \varnothing$.
\end{enumerate}
\end{definition}

For the benefit of readers unfamiliar with these notions, we highlight that in our weak membership problem, there are  $x$'s that satisfy both $x\in S(K,\delta)$ and $x\notin S(K,-\delta)$ simultaneously. So if we can ascertain \textsc{mem}, we can ascertain \textsc{wmem}, but not conversely. A consequence is that if \textsc{wmem} problem for $K$ is NP-hard, then \textsc{mem} for $K$ is also NP-hard.

There will be occasions, particularly in Section~\ref{sec:cones}, when we have to discuss weak membership and weak validity of a convex set $K \subseteq \mathbb{R}^n$ of  positive codimension, i.e., contained in an affine subspace of dimension $< n$. As a subset of $\mathbb{R}^n$, $K$ will have no interior points and the \textsc{wmem} and \textsc{wval} as defined above would make little sense. With this in mind, we introduce the following variant of Definition~\ref{def:mem} that makes use of the interior of $K$ relative to $H$, an affine subspace of minimal dimension that contains $K$, i.e., $H$ is the affine hull of $K$.  We start by defining
\[
S_H(K,-\delta):=\{x\in K: B(x,\delta)\cap H\subseteq K\}.
\]
Note that if $K \ne \varnothing$, then there exists $\varepsilon >0$ such that $S_H(K,-\delta) \ne \varnothing$ for each $\delta \in (0,\varepsilon)$, even if $K$ has no interior point. If $K$ has nonempty interior, then $H = \mathbb{R}^n$ and $S_H(K,-\delta) =S(K,-\delta)$.

\begin{definition}\label{def:relmem}
Let $K \subseteq \mathbb{R}^n$ be a convex set and let $H = \operatorname{aff}(K)$ be its affine hull.
\begin{enumerate}[\upshape (i)]
\item The weak membership problem (\textsc{wmem}) for $K$ relative to $H$ is: Given $x\in\mathbb{Q}^n$ and a rational number $\delta>0$,  assert that $x\in S(K,\delta)\cap H $ or  $x\notin S_H(K,-\delta)$.  

\item The weak validity problem (\textsc{wval}) problem for $K$ relative to $H$ is: Given $c\in\mathbb{Q}^n$ and rational numbers $\gamma,\varepsilon>0$,
either assert that $c^\mathsf{T} x\le \gamma +\varepsilon$ for all $x\in S_H(K,-\varepsilon)$, or assert that $c^\mathsf{T} x \ge \gamma -\varepsilon$ for some $x\in S(K,\varepsilon) \cap H$.
\end{enumerate}
\end{definition}

An implicit assumption throughout this article is that when we study the computational complexity of \textsc{wmem} and \textsc{wval} problems for a convex set $K \subseteq \mathbb{R}^n$ with nonempty interior, we assume that we know a point $a \in \mathbb{Q}^n$ and a rational $r > 0$ such that the Euclidean norm ball $B(a,r) \subseteq K$. This mild assumption guarantees that $K$ is `centered' in the sense of \cite[Definition~2.1.16]{GLS88} and is needed whenever we invoke Yudin--Nemirovski Theorem \cite[Theorem~4.3.2]{GLS88}.

Recall that a problem $\mathscr{P}$ is said to be \emph{polynomial-time reducible}  \cite[p.~28]{GLS88} to a problem $\mathscr{Q}$ if there is a polynomial-time algorithm $A_\mathscr{P}$ for solving $\mathscr{P}$ by making a polynomial number of oracle calls to an algorithm $A_\mathscr{Q}$  for solving $\mathscr{Q}$. This notion of polynomial-time reducibility is also called Cook  or Turing reducibility and will be the one used throughout our article. There is also a more restrictive notion of polynomial-time reducibility that allows only a single oracle call to  $A_\mathscr{Q}$ called Karp or many-one reducibility.

Note that if $A_\mathscr{Q}$ is a polynomial-time algorithm for $\mathscr{Q}$, then $A_\mathscr{P}$ is a polynomial-time algorithm for $\mathscr{P}$. Consequently, if $\mathscr{Q}$ is computable in polynomial-time, then so is $\mathscr{P}$. On the other hand, if $\mathscr{P}$ is NP-hard, then so is $\mathscr{Q}$.

We say that $\mathscr{P}$ and $\mathscr{Q}$ are \emph{polynomial-time inter-reducible} if $\mathscr{P}$ is polynomial-time reducible to $\mathscr{Q}$ and $\mathscr{Q}$ is polynomial-time reducible to $\mathscr{P}$.  The polynomial-time inter-reducibility of two problems $\mathscr{P}$ and $\mathscr{Q}$ implies that they are in the same time-complexity class\footnote{Assuming that the complexity class is defined by polynomial-time inter-reducibility.} whatever it may be. Nevertheless, in this article we will restrict ourselves to just polynomial-time computability and NP-hardness, the two most oft-used cases in optimization.

\section{Weak membership in dual norm balls}\label{sec:norm}

Our techniques for this section relies on tools introduced in \cite[Chapter 4]{GLS88} and are inspired by \cite[Section~6.1]{Gu02}.

Let $\nu:\mathbb{R}^n\to [0,\infty)$ be a norm and denote the closed ball and open ball centered at $a \in \mathbb{R}^n$ of radius $r > 0$ with respect to  the norm $\nu$ by
\[
B_\nu(a,r):=\{x\in \mathbb{R}^n : \nu(x-a)\le r\}\quad\text{and}\quad
B^\circ_\nu(a,r):=\{x\in \mathbb{R}^n : \nu(x-a) < r\}
\]
respectively. For the special case $a = 0$ and $r =1$, we write $B_\nu:=B_\nu(0,1)$ and  $B^\circ_\nu:=B^\circ_\nu(0,1)$  for the closed  and open unit balls.  
For the special case $\nu = \|\cdot\|$, the Euclidean norm on $\mathbb{R}^n$, we write $B(a,r):=B_{\|\cdot \|}(a,r)$ and  $B^\circ(a,r):=B^\circ_{\|\cdot \|}(a,r)$, dropping the subscript.
Since all norms on $\mathbb{R}^n$ are equivalent, it follows that there exist 
constants $K_\nu\ge k_\nu>0$ such that 
\begin{equation}\label{mnuequiv}
k_\nu\|x\|\le \nu(x)\le K_\nu\|x\|
\end{equation}
for all $x\in\mathbb{R}^{n}$. There is no loss of generality in assuming that $k_\nu$ and $K_\nu$ are rational\footnote{If not just pick a smaller $k_\nu$ or a larger $K_\nu$ that is rational.} and we may denote the number of bits required to specify them by $\langle k_\nu \rangle$ and $ \langle K_\nu \rangle$ respectively. We write 

Recall that the \emph{dual norm} of $\nu$, denoted $\nu^*$, is given by 
\[
\nu^*(x)=\max\{\lvert y^\mathsf{T} x \rvert : \nu(y)\le 1\}
\]
for every $x\in\mathbb{R}^{n}$. Hence
\begin{equation}\label{constdualnrm}
 \frac{1}{K_\nu}\|x\|\le \nu^*(x)\le \frac{1}{k_\nu}\|x\|
\end{equation}
for all  $x\in\mathbb{R}^{n}$.

Observe first that $B(0,1/K_\nu)\subseteq B_\nu\subseteq B(0,1/k_\nu)$.
Hence
\[
\langle B_\nu \rangle:=\langle n \rangle+\langle k_\nu \rangle+\langle K_\nu \rangle
\]
may be regarded as the encoding length of $B_\nu$ in number of bits. In other words, for any given norm, we may always specify its unit norm ball in a finite number of bits using the procedure outlined above.

The main result of this section is the polynomial-time inter-reducibility between a norm and its dual.
\begin{theorem}\label{polduality}  
Let $\nu$ be a norm and $\nu^*$ be its dual norm. The \textsc{wmem} problem for the unit ball of $\nu^*$  is polynomial-time reducible to
the \textsc{wmem} problem for the unit ball of $\nu$.
\end{theorem}

We will prove this result via two intermediate lemmas. 
A key step in our proof depends on the Yudin--Nemirovski Theorem \cite[Theorem~4.3.2]{GLS88}, which may be stated as follows.
\begin{theorem}[Yudin--Nemirovski]\label{thm:YN}
The \textsc{wval} problem for $B_\nu$ is polynomial-time reducible to the \textsc{wmem} problem for $B_{\nu}$. More generally this holds for any convex set with nonempty interior $K \subseteq \mathbb{R}^n$ for which we have knowledge of $a \in \mathbb{Q}^n$ and $0< r \le R \in \mathbb{Q}$ such that $B(a,r) \subseteq K \subseteq B(0,R)$.
\end{theorem}
The original Yudin--Nemirovski Theorem is in fact stronger than the version stated here, allowing the weak violation problem \textsc{wviol} to be reduced to \textsc{wmem}. Nevertheless in this article we will only require the weaker result with \textsc{wval} in place of \textsc{wviol}.

For a compact set $K\subset \mathbb{R}^n$ and $c\in \mathbb{R}^n$, we define
\[
\max(K,c):=\max \{c^\mathsf{T} x  :  x \in K\}
\]
denote the maximum of the linear optimization problem over $K$. In particular, observe that 
\[
\nu(x)=\max(B_{\nu^*},x).
\]
\begin{lemma}\label{auxineq}  Let $\nu$ be a norm on $\mathbb{R}^n$ and $\delta>0$.  Then we have inclusions
\begin{gather}
(1+k_\nu\delta)B_\nu\subseteq S(B_\nu,\delta)\subseteq (1+K_\nu\delta)B_\nu,\label{auxineq3}
\\
(1-K_\nu\delta)B_\nu\subseteq S(B_\nu,-\delta)\subseteq (1-k_\nu\delta)B_\nu,\label{auxineq4}
\end{gather}
whenever  $K_\nu\delta< 1$, and the inequalities
\begin{gather}
\label{auxineq1} \left(1-\frac{\delta}{k_\nu}\right)\nu(x) \le \max \bigl(S(B_{\nu^*},-\delta),x\bigr)\le 
\left(1-\frac{\delta}{K_\nu}\right)\nu(x),\\
\label{auxineq2}
\left(1+\frac{\delta}{K_\nu}\right)\nu(x)\le \max \bigl(S(B_{\nu^*},\delta),x\bigr)\le \left(1+\frac{\delta}{k_\nu}\right)\nu(x),
\end{gather}
whenever $\delta/k_\nu< 1$.
\end{lemma}
\begin{proof}  To prove \eqref{auxineq3}, observe that
\[
k_\nu B_\nu\subseteq B(0,1)\subseteq K_\nu B_\nu,\qquad  k_\nu B^\circ_\nu\subseteq B^\circ(0,1)\subseteq K_\nu B^\circ_\nu,
\]
and thus
\[
B_\nu(x,k_\nu \delta) \subseteq B(x,\delta) \subseteq B_\nu(x,K_\nu \delta),\qquad
B^\circ_\nu(x,k_\nu \delta) \subseteq B^\circ(x,\delta) \subseteq B^\circ_\nu(x,K_\nu \delta).
\]
Also, $ \bigcup_{x\in B_\nu} B_\nu(x, r) =B_\nu(0,1+r)$ by the defining properties of a norm. Hence
\[
S(B_\nu,\delta) =\bigcup_{x\in B_\nu} B(x,\delta) \subseteq \bigcup_{x\in B_\nu} B_\nu(x, K_\nu\delta) =B_\nu(0,1+K_\nu\delta).
\]
On the other  hand,
\[
S(B_\nu,\delta) =\bigcup_{x\in B_\nu} B(x,\delta) \supseteq \bigcup_{x\in B_\nu} B_\nu(x, k_\nu\delta)  =B_\nu(0, 1+k_\nu\delta).
\]

To prove  \eqref{auxineq4}, let $T=\bigcup_{x\, : \,\nu(x)=1} B^\circ(x,\delta)$ and so $S(B_\nu,-\delta)=B_\nu\setminus T$.
Let 
\[T_1=\bigcup_{x\, : \,\nu(x)=1}B^\circ_\nu(x, K_\nu\delta), \qquad T_2=\bigcup_{x\, : \, \nu(x)=1} 
B^\circ_\nu(x, k_\nu\delta).
\]
Since $T_1\supseteq T$ and $T_2\subseteq T$, we obtain
\[
S(B_\nu,-\delta)\supseteq B_\nu\setminus T_1 = (1-K_\nu\delta)B_\nu,
\qquad
S(B_\nu,-\delta)\subseteq B_\nu\setminus T_2 =(1-k_\nu\delta)B_\nu.
\]
The last two inequalities follow from the first two inclusions and \eqref{constdualnrm}.
\end{proof}

\begin{lemma}\label{wmemwval} Let $k_\nu\ge 2$.  Then the solution to \textsc{wval} problem for $B_{\nu^*}$ gives the solution to \textsc{wmem} problem for $B_\nu$.
\end{lemma}
\begin{proof} Let $x\in\mathbb{Q}^n$ and  $\delta\in (0, \frac{1}{2})\cap \mathbb{Q}$.  We choose $\gamma=1$.
Suppose that $x^\mathsf{T} y\le 1+ \delta$ for all $y\in S(B_{\nu^*},-\delta)$.  Then $\max\bigl(S(B_{\nu^*},-\delta),x\bigr)\le 1+\delta$  and we  deduce from \eqref{auxineq1}  that
\[
\nu(x)\le \frac{1+\delta}{1-\delta/k_\nu}.
\]
Since $k_\nu\ge 2$, it follows  that
\[
\frac{1+\delta}{1-\delta/k_\nu}\le
1+k_\nu\delta.
\]
It follows from \eqref{auxineq3} that $x\in S(B_\nu,\delta)$.

Suppose that $x^\mathsf{T} y> 1- \delta$ for some $y\in S(B_{\nu^*},\delta)$. Then $\max\bigl(S(B_{\nu},\delta),\delta\bigr)>1-\delta$ and we deduce from \eqref{auxineq2}  that
\[
\nu(x)> \frac{1-\delta}{1+ \delta/k_\nu}.
\]
As straightforward calculation shows that
\[
\frac{1-\delta}{1+ \delta/k_\nu }\ge 1-k_\nu\delta.
\]
It follows from \eqref{auxineq4} that $x\notin S(B_\nu,-\delta)$.   
\end{proof} 

\begin{proof}[Proof of Theorem~\ref{polduality}]
We observe that the assumption $k_\nu\ge 2$ in Lemma~\ref{wmemwval} is not restrictive. Let $r\ge 2/k_\nu$.  Then a new norm defined by $ \nu_r(x)=r\nu(x)$ would satisfy the assumption.
Now note that $x\in B_\nu$ if and only if $\frac{1}{r}x\in B_{\nu_r}$.  With this observation,
Theorem~\ref{polduality} follows from
\[
\text{\textsc{wmem} for }\nu^* \; \Rightarrow \; 
\text{\textsc{wval} for }\nu^* \; \Rightarrow \; 
\text{\textsc{wmem} for }\nu \; \Rightarrow \; 
\text{\textsc{wval} for }\nu \; \Rightarrow \; 
\text{\textsc{wmem} for }\nu^*.
\]
Here $\mathscr{P} \Rightarrow \mathscr{Q}$ means that $\mathscr{Q}$ is polynomial-time reducible to $\mathscr{P}$. Yudin--Nemirovski Theorem gives the first and third reductions whereas Lemma~\ref{wmemwval} gives the second and last reductions.
\end{proof}

Since taking dual of a dual norm gives us back the original norm, we have the following corollary.
\begin{corollary}
The \textsc{wmem} problem for the unit ball of a norm $\nu$  is polynomial-time decidable (resp.\ NP-hard)
if and only if the \textsc{wmem} problem for the unit ball of the dual norm $\nu^*$ is polynomial-time decidable (resp.\ NP-hard).
\end{corollary}

Since every centrally symmetric compact convex set with nonempty interior is a norm ball for some norm and its \emph{polar dual} is exactly the norm ball for the corresponding dual norm, we immediately have the following.
\begin{corollary}\label{cor:cscc}
Let $C$ be a  centrally symmetric compact convex set with nonempty interior in $\mathbb{R}^n$ and
\[
C^* = \{ x \in \mathbb{R}^n : x^\mathsf{T} y \le 1 \}
\]
be its polar dual. Then \textsc{wmem} in $C$ is polynomial-time inter-reducible to the \textsc{wmem} in $C^*$. In particular, if one is polynomial-time decidable  (resp.\ NP-hard), then so is the other.
\end{corollary}

While our discussion above is over $\mathbb{R}$, it is easy to extend it to $\mathbb{C}$ since $\mathbb{C}^{n}$ maybe identified with $\mathbb{R}^{2n}\equiv \mathbb{R}^{n}\times \mathbb{R}^{n}$, where $z = x+\sqrt{-1}y\in \mathbb{C}^n$ is identified with $(x,y)\in\mathbb{R}^{n}\times \mathbb{R}^{n}$.  
A norm $\nu:\mathbb{C}^n\to [0,\infty)$ induces a norm $\tilde \nu:\mathbb{R}^{2n}\to [0,\infty)$ via
$\tilde \nu\bigl((x,y)\bigr):=\nu(x+\sqrt{-1}y)$ and we may identify $\nu$ with $\tilde \nu$.  In particular, the Hermitian norm on $\mathbb{C}^n$ 
gives exactly the Euclidean norm on $\mathbb{R}^{2n}$.  Hence for the purpose of this article, it suffices to consider norms over real vector spaces.
 
\section{Approximation of dual norms}\label{sec:approx}

In this section we  show that for a given norm $\nu:\mathbb{R}^n \to [0,\infty)$ satisfying 
\eqref{mnuequiv} for $k_\nu, K_\nu \in \mathbb{Q}$, \textsc{wmem} in $B_\nu$ with respect to  $\delta \in \mathbb{Q}$ is polynomial-time inter-reducible with a $\delta$-approximation of the norm $\nu$.

\begin{definition}\label{def:approx}
Let  $\nu:\mathbb{R}^n \to [0,\infty)$ be a norm satisfying 
\eqref{mnuequiv} for $k_\nu, K_\nu \in \mathbb{Q}$.
We define an \emph{approximation problem} (\textsc{approx}) for $\nu$ as follows.
Let $\delta \in \mathbb{Q}$ and $\delta > 0$.  Given any $x\in \mathbb{Q}^n$ with $1/2 < \|x\| < 3/2$, compute an approximation $\omega(x) \in \mathbb{Q}$ such that
\begin{equation}\label{approxnu}
\omega(x) -\delta<\nu(x)<\omega(x) +\delta.
\end{equation}
We call $\omega$ a \emph{$\delta$-approximation} of $\nu$.
\end{definition}
The requirement that $1/2 < \|x\| < 3/2$ is not restrictive since we may always scale any given $x$ to meet this condition in polynomial-time. Note that an approximation problem has $n+\langle \delta \rangle+\langle K_\nu \rangle+\langle k_\nu \rangle $ input bits. If we say that such  a problem can be solved in polynomial time, we mean time polynomial in this number of input bits.

\begin{theorem}\label{nuapproxwm}
Let  $\nu:\mathbb{R}^n \to [0,\infty)$ be a norm satisfying 
\eqref{mnuequiv} for $k_\nu, K_\nu \in \mathbb{Q}$. Then the following problems are polynomial-time inter-reducible:
\begin{enumerate}[\upshape (i)]
\item The approximation problem for $\nu$.
\item The weak membership problem for $B_\nu$.
\end{enumerate}

\end{theorem}
\begin{proof}  Let us use (i) as an oracle and solve (ii).  Let $x\in\mathbb{Q}^n$ and a rational $\delta>0$ be given.
If $\|x\|\le 1/K_\nu$, then $\nu(x)\le 1$, and so $x\in S(B_\nu,\delta)$.
If $\|x\|\ge 1/k_\nu$, then $\nu(x)\ge 1$, and so $x\notin S(B_\nu,-\delta)$. 

It remains to check the case $\|x\|\in (1/K_\nu, 1/k_\nu)$. Let $r\in (2\|x\|/3, 2\|x\|)\cap \mathbb{Q}$ and let $y:= x/ r$. Observe that $\nu(y)\in (k_\nu/2,3K_\nu/2)$. Now let  $\varepsilon=k_\nu^2\delta/4$ and $\omega(y)$ be an $\varepsilon$-approximation of $\nu(y)$. Assume first that 
\[
r\omega(y)\le 1+k_\nu\delta -\frac{2 \varepsilon}{k_\nu}=1+\frac{k_\nu\delta}{2}.
\]
Then
\[
\nu(x)= r\nu(y)< r(\omega(y)+\varepsilon)< r\omega(y)+\frac{2}{k_\nu}\varepsilon\le 1+k_\nu\delta,
\]
and \eqref{auxineq3} yields that $x\in S(B_\nu,\delta)$. Assume now that
\[
r\omega(y)> 1+\frac{k_\nu\delta}{2}.
\]
Then
\[
\nu(x)>r(\omega(y)-\varepsilon)\ge r\omega(y)-\frac{2\varepsilon}{k_\nu}>1+\frac{k_\nu\delta}{2}-\frac{k_\nu\delta}{2}=1
\]
and so $x\notin S(B_\nu,-\delta)$. This shows that we may decide weak membership in $B_\nu$ with a $\delta$-approximation to $\nu$. In fact we just need one oracle call to \textsc{approx}.

Let us use (ii) as an oracle and solve (i).   Let $x \in \mathbb{Q}^n$ where $\|x\|\in(1/2, 3/2)$ and a rational $\delta > 0$ be given. Again, observe that $\nu(x)\in [a_1,b_1]$, where $a_1= k_\nu/2$ and $b_1=3K_\nu/2$.  
Suppose that for an integer $i\ge 1$ we showed that $\nu(x)\in [a_i,b_i]$.
Let
\begin{equation}\label{defr}
r=\frac{a_i+b_i}{2},\qquad \varepsilon=\frac{b_i-a_i}{2K_\nu(b_i+a_i)},
\end{equation}
and consider $y=x/r$. Assume first that $y\in S(B_\nu,\varepsilon)$.  Then the right inclusion in \eqref{auxineq3} yields
$\nu(y)\le 1+K_\nu\varepsilon$ and thus
\[
\nu(x)=r\nu(y)\le \frac{a_i+b_i}{2}(1+K_\nu\varepsilon)= \frac{3}{4} b_i+\frac{1}{4}a_i.
\]
In this case we set $a_{i+1}=a_i$ and $b_{i+1}= 3b_i/4+ a_i/4$. Assume now that $y\notin S(B_\nu,-\varepsilon)$.  Then the left inclusion in \eqref{auxineq4} yields
\[
\nu(x)> r(1-K_\nu\varepsilon)= \frac{1}{4}b_i+\frac{3}{4}a_i.
\]
In this case we set  $a_{i+1}= b_i/4+3a_i/4$ and  $b_{i+1}=b_i$.

In either case, we obtain that $\nu(x)\in [a_{i+1},b_{i+1}]$.
Clearly, the sequence of intervals $\{[a_i,b_i] : i\in\mathbb{N} \}$ is nested and their successive lengths decrease by a factor of $3/4$.
Let $m$ be the smallest integer such that
\[b_m-a_m=
\left(\frac{3}{4}\right)^{m-1}(b_1-a_1)<2\delta.
\]
Then $m$ is polynomial in 
$\langle K_\nu \rangle+\langle k_\nu \rangle+\langle \delta \rangle$.  
Setting $\omega(x):=(a_m+b_m)/2$,  we obtain a $\delta$-approximation of $\nu(x)$.  This shows that we may determine a $\delta$-approximation to $\nu$ with $m$ oracle calls to  \textsc{wmem} in $B_\nu$.
\end{proof}

\begin{corollary}\label{cor:polduality}
A norm is polynomial-time approximable (resp.\ NP-hard to approximate) if and only if its dual norm is polynomial-time approximable (resp.\ NP-hard to approximate).
\end{corollary}

We end this section with a word about \emph{Mahler volume} \cite{BM87}. For any norm $\nu:\mathbb{R}^n\to [0,\infty)$, let 
$\operatorname{Vol}_n(B_{\nu})$ denote the volume of its unit ball $B_\nu$.  The Mahler volume of $\nu$ is defined as
\[
M(\nu):=\operatorname{Vol}_n(B_{\nu})\operatorname{Vol}_n(B_{\nu^*}).
\]
A particularly nice property of the Mahler volume is that it is invariant under \emph{any} invertible linear transformation, regardless of whether it is volume-preserving or not.
 
\begin{corollary}
If the weak membership problem in $B_{\nu}$ is polynomial-time decidable, then $M(\nu)$ is polynomial-time approximable.
\end{corollary}
\begin{proof}
If the \textsc{wmem} in $B_\nu$ is polynomial-time decidable, then it follows from \cite{DFK91} that there exist polynomial-time algorithms to approximate $\operatorname{Vol}_n(B_{\nu})$ to any given error $\varepsilon>0$. By Corollary~\ref{cor:polduality}, the \textsc{wmem} in $B_{\nu^*}$ is also polynomial-time 
decidable and thus the same holds for  $\operatorname{Vol}_n(B_{\nu^*})$.
\end{proof}
Mahler volume is more commonly defined for a centrally symmetric compact convex set but as we mentioned before Corollary~\ref{cor:cscc}, this is equal to a 
unit norm ball for an appropriate choice of norm.

\section{Weak membership in dual cones}\label{sec:cones}

In this section, we move our discussion from balls to cones. While every ball is, by definition, a norm ball, a (proper) cone may not be a norm cone, 
i.e., of the form $\{ x\in \mathbb{R}^n : \| Ax \| \le c^\mathsf{T}x \}$ for some norm $\|\cdot\|$ and $A \in \mathbb{R}^{n \times n}$, $c \in \mathbb{R}^n$.  
So the results in this section would not in general follow from the previous sections.

Let $K\subset \mathbb{R}^n$ be a \emph{proper cone} in $\mathbb{R}^n$, i.e., $K$ is a closed convex pointed\footnote{By pointed, we mean that 
$K \cap (-K)=\{0\} $.} cone with non-empty interior. Then its \emph{dual cone},
\[
K^*:=\{x\in\mathbb{R}^n : y^\mathsf{T} x\ge 0 \;\text{for every}\; y\in K\},
\]
is also a proper cone \cite{Roc}. We assume that both $K$ and $K^*$ can be encoded in a finite number of bits with encoding length $\langle K \rangle$ and $\langle K^* \rangle$ respectively. The main result of this section is an analogue of Theorem~\ref{polduality} for such cones: The weak membership problem for $K^*$ is polynomial-time reducible to the weak membership problem for $K$.

It is well-known that deciding \textsc{mem} for the cone of copositive matrices is NP-hard \cite{MK87}. This result  has recently been  extended \cite{DG14}: \textsc{wmem} in the cone of copositive matrices and \textsc{wmem} in its dual cone, the cone of completely positive matrices, are both NP-hard problems. Our result in this section generalizes this to arbitrary proper cones.

We first recall a well-known result regarding the interior points of $K^*$.
\begin{lemma}\label{interptK'}  Let $K\subseteq \mathbb{R}^n$ be a closed convex cone.  Let $b$ be an interior point of $K^*$, i.e., 
$b+z\in K^*$ for all $z\in B(0,\varepsilon_b)$ for some $\varepsilon_b >0$.  Then
\begin{equation} \label{interptK'1}  
b^\mathsf{T} x\ge \varepsilon_b \|x\|
\end{equation}
for every $x\in K$.
\end{lemma}
\begin{proof}  Let $x\in K\setminus\{0\}$.  Then $c:=b- \varepsilon_b x/ \|x\|\in K^*$.  Hence $c^\mathsf{T} x\ge 0$, which implies \eqref{interptK'1}.
\end{proof}

We now discuss the notion of \textsc{wmem} in $K$.  Recall that $x\in K\setminus\{0\}$ if and only if $tx\in K$ for each $t>0$.  Hence it make sense to define \textsc{wmem} in $K$ for $x\in \mathbb{Q}^n$
having Euclidean norm $1$.  Since $\|x\|$ may be not a rational number it make sense to define a \textsc{wmem} problem for $x\in \mathbb{Q}^n, \frac{1}{2} <\|x\|<1$.
 
Let $a\in\mathbb{Q}^n$ and $b\in\mathbb{Q}^n$ be in the interior of $K$ and $K^*$ respectively. By Lemma~\ref{interptK'},
\begin{equation}\label{defPab}
P_b:=\{x\in K:b^\mathsf{T} x=1\}, \quad P^*_a=\{y\in K^*: a^\mathsf{T}y=1\}
\end{equation} 
are compact convex sets of dimension $n-1$.  Hence the sets $P_b-a$ and $P_a^*-b$
are full-dimensional compact convex sets in the orthogonal complements of $\operatorname{span}(b)$ and $\operatorname{span}(a)$ respectively.
In what follows we assume that we have knowledge of positive $\rho_a, \rho_b \in \mathbb{Q}$ such that the $(n-1)$-dimensional balls $B(0,\rho_b)$ and $B(0,\rho_a)$ 
contain  the sets $P_b-a$ and $P_a^*-b$ respectively. While $\rho_a, \rho_b$ do not appear explicitly in our proofs, they are needed implicitly when we invoke the Yudin--Nemirovski Theorem.

In fact $P_b$ and $P_a^*$ are compact convex sets of maximal dimension in the affine hyperplanes 
\[
H_b:=\{z\in\mathbb{R}^n: b^\mathsf{T} z=1\}, \quad  H_a:=\{z\in\mathbb{R}^n: a^\mathsf{T} z=1\}
\]
respectively. We may also view $H_b$ and $H_a$ as the affine hulls of $P_b$ and $P_a^*$ respectively. Given any $x\ne 0$, observe that $x\in K$ if and only if $x/(b^\mathsf{T} x) \in P_b$.
Thus the membership problem for $K$ is equivalent to the membership problem for $P_b$. We show in the following that this extends,  in an appropriate sense, to weak membership as well.

\begin{lemma}\label{eqWMEMconePb}
Let $x \in \mathbb{Q}^n$ with $1/2 <\|x\|<1$ and $b \in \mathbb{Q}^n$ with $b^\mathsf{T} x> 0$. Then the following problems are polynomial-time inter-reducible:
\begin{enumerate}[\upshape (i)]
\item  Decide weak membership of $x$ in $K$.
\item Decide weak membership of  $y:= x/(b^\mathsf{T} x)$ in $P_b$ relative to $H_b$.
\end{enumerate}
\end{lemma}
\begin{proof}
Suppose that $0< \delta < b^\mathsf{T} x/(2\|b\|)$.   Let $z\in\mathbb{R}^n$ and $\|z\|\le \delta$. 
Clearly,
\[
b^\mathsf{T} (x+z)=b^\mathsf{T} x +b^\mathsf{T}z\ge b^\mathsf{T} x - \|b\|\|z\|\ge \frac{1}{2}b^\mathsf{T} x >0.
\]
In the following, we let $y:= x/(b^\mathsf{T} x)$ and $u:=(x+z)/\bigl(b^\mathsf{T} (x+z)\bigr)\in H_b$.

Suppose that we can solve (i), i.e., for any rational $\delta>0$ and  $x\in \mathbb{Q}^n$ with $1/2<\|x\|<1$  we can decide whether $x\in S(K,\delta)$ or $x\notin S(K,-\delta)$.  Let $\varepsilon > 0$ be rational and choose $\delta$ rational so that
\[
\frac{(b^\mathsf{T} x)^2}{8\|b\|}\varepsilon < \delta < \frac{(b^\mathsf{T} x)^2}{4\|b\|}\varepsilon.
\]

Consider first the case $x\notin S(K,-\delta)$. There exists $z\in \mathbb{R}^n$, $\|z\|\le \delta$ such that $x+z\notin K$. So $u\notin P_b$.  Since
\[
u-y=\frac{1}{(b^\mathsf{T} x)(b^\mathsf{T} (x+z))}[(b^\mathsf{T} (x+z))x - (b^\mathsf{T} x)(x+z)]
=\frac{1}{(b^\mathsf{T} x)(b^\mathsf{T} (x+z))}[(b^\mathsf{T} z)x - (b^\mathsf{T} x)z],
\]
we obtain
\[
\|u-y\|\le \frac{2}{(b^\mathsf{T} x)^2}(2\|b\|\|x\|\|z\|)\le  \frac{4\|b\|\delta}{(b^\mathsf{T} x)^2} < \varepsilon.
\]
Hence $y\notin S_{H_b}(P_b,-\varepsilon)$.

Consider now the case $x\in S(K,\delta)$. There exists $z\in \mathbb{R}^n$, $\|z\|\le \delta$ such that $x+z\in K$.
The same line of argument as above yields that $x\in S(P_b,\varepsilon) \cap H_b$. Together the two cases show that if we can decide \textsc{wmem} in $K$ with inputs $x$, $\delta$, then we can decide \textsc{wmem} in $P_b$ relative to $H_b$ with inputs $y$, $\varepsilon$.

Suppose we can solve (ii), i.e.,  for any rational $\varepsilon > 0$ and $x\in \mathbb{Q}^n$ with $1/2 <\|x\|<1$, $b^\mathsf{T} x>0$, we can decide whether $y\in S(P_b,\varepsilon)\cap H_b $ or $y\notin S_{H_b}(P_b,-\varepsilon)$.  

Let $x\in \mathbb{Q}^n$ with $1/2 <\|x\|<1$. We start by excluding the trivial case when $b^\mathsf{T} x\le 0$.  By Lemma~\ref{interptK'},  $x\notin K$ and thus $x\notin S(K,-\delta)$ for any $\delta >0$. So we may assume henceforth that $b^\mathsf{T} x> 0$. Let $\delta > 0$ be rational  and set $\varepsilon:=\delta/(b^\mathsf{T} x)$.

Consider first the case $y\notin S_{H_b}(P_b,-\varepsilon)$.  There exists $v\in H_b\setminus{P_b}$ such that $\|v-y\|\le \varepsilon$.  Let $z=(b^\mathsf{T} x)(v-y)$.  So
\[
\|z\|\le (b^\mathsf{T} x)\varepsilon =\delta.
\]
Hence $(b^\mathsf{T} x)v=x+z\notin K$ and so $x\notin S(K,-\delta)$.  

Consider now the case $y\in S(P_b,\varepsilon) \cap H_b$. The same line of argument as above yields  that  $x\in S(K,\delta)$.  Together the two cases show that if we can decide \textsc{wmem} in $P_b$ relative to $H_b$ with inputs $y$, $\varepsilon$, then we can decide \textsc{wmem} in $K$ with inputs $x$, $\delta$.
\end{proof}

Lemma~\ref{eqWMEMconePb} may be viewed as a compactification result: We transform a problem involving a noncompact object $K$ to a problem involving a compact object $P_b$. The motivation is so that we may apply the Yudin--Nemirovski Theorem later. 

\begin{theorem}\label{WMEMcone}
Let $K\subset \mathbb{R}^n$ be a proper cone and $K^*$ be its dual.  Let $a \in \mathbb{Q}^n$ and $b\in\mathbb{Q}^n$ be  interior points of $K$ and $K^*$ respectively that satisfy $b^\mathsf{T} a=1$.  Then the \textsc{wmem} problem for $K^*$ is polynomial-time reducible to the \textsc{wmem} problem for $K$.
\end{theorem}
\begin{proof} 
Note that such a pair of $a$ and $b$ must exist for any proper cone. Let $a,b\in \mathbb{Q}^n$ be interior points contained in balls of radii $\varepsilon_a$, $\varepsilon_b >0$ within $K^*$, $K$
respectively. So $b^\mathsf{T} a > 0$.  If $b^\mathsf{T} a=1$, we are done. Otherwise set $a' = a/(b^\mathsf{T} a)\in \mathbb{Q}^n$.  Then $b^\mathsf{T}a'=1$ and $a'$ is contained in a ball of radius
$\varepsilon_{a'}=\varepsilon_a/(b^\mathsf{T}a)$ within $K^*$.

By Lemma~\ref{eqWMEMconePb}, we just need to show that the \textsc{wmem} problem for $P_a^*$ relative to $H_a$ is polynomial-time reducible to the \textsc{wmem} problem for $P_b$ relative to $H_b$. Since $b^\mathsf{T} a=1$, $H_b -a =b^{\perp}$, the orthogonal complement of $b$, and can be identified with $\mathbb{R}^{n-1}$ by an orthogonal change of coordinates.  We set $K_b:= P_b-a$, a compact closed set in $\mathbb{R}^{n-1}$ containing the origin $0\in \mathbb{R}^{n-1}$.  Moreover $B(0,\varepsilon_a)\subset K_b$, where $B(0,\varepsilon_a)$ here is an $(n-1)$-dimensional ball in $\mathbb{R}^{n-1}$. It is enough to show that the \textsc{wmem} problem for $P_a^*$ relative to $H_a$ is polynomial-time reducible to the \textsc{wmem} problem\footnote{When we refer to the \textsc{wmem} or \textsc{wval} problem for $K_b$, we mean its \textsc{wmem} or \textsc{wval} problem as a subset of $\mathbb{R}^{n-1}$.} for $K_b$. We would also need to invoke the fact that the \textsc{wval} problem for $K_b$ is polynomial-time reducible to the \textsc{wmem} problem for $K_b$ by the Yudin--Nemirovski Theorem.

Let $c\in \mathbb{Q}^n\cap H_a$.  Given a rational $\delta>0$ we need to decide whether $c\notin S_{H_a}(P_a^*,-\delta)$ or $c\in S(P_a^*,\delta)\cap H_a$.  Let $\varepsilon>0$ be rational with
\begin{equation}\label{condeps}
\varepsilon <\min\left\{\frac{1}{4(1+\|c\|)}, \frac{\delta}{4(1+\|c\|)(\|b-c\|)} \right\},
\end{equation}
where $\delta/0 :=\infty$ if $b = c$.  It follows from \eqref{condeps} that
\begin{equation}\label{condeps1}
\tau:=(1+\|c\|)\varepsilon\le \frac{1}{4},\qquad \left\|c-\frac{c+\tau b}{1+\tau}\right\| \le \delta,\qquad \left\|c - \frac{c-2\tau b}{(1-2\tau}\right\|\le \delta.
\end{equation}

Observe that $c$ defines a linear functional $b^{\perp} \to \mathbb{R}$, $x\mapsto c^\mathsf{T} x$.  Consider the \textsc{wval} problem for $K_b$ with $\gamma=-c^\mathsf{T} a$:
Either $c^\mathsf{T}x\ge -c^\mathsf{T} a -\varepsilon$ for all $x\in S(K_b,-\varepsilon)$ or  $c^\mathsf{T}x\le -c^\mathsf{T}a +\varepsilon$ for some $x\in S(K_b,\varepsilon)$.
We will show that in the first case $c\in S(P_a^*,\delta)\cap H_a $ and in the second case $c\not\in S_{H_a}(P_a^*,-\delta)$ for a corresponding $\delta >0$.

Consider first the case $c^\mathsf{T}x\ge -c^\mathsf{T} a -\varepsilon$ for all $x\in S(K_b,-\varepsilon)$, or, equivalently, $c^\mathsf{T}y \ge -\varepsilon$ for all $y = x + a \in S_{H_b}(P_b,-\varepsilon)$. 
We claim that $c^\mathsf{T}y\ge -(1+\|c\|)\varepsilon$ for all $y\in P_b$. This holds for $y \in  S(P_b,-\varepsilon)$ since  $c^\mathsf{T}y \ge -\varepsilon \ge -(1+\|c\|)\varepsilon$. For 
$y\in P_b\setminus S(P_b,-\varepsilon)$, there exists $x\in S(P_b,-\varepsilon)$ such that $\|y-x\|\le \varepsilon$.
Thus $c^\mathsf{T} y=c^\mathsf{T}x+ c^\mathsf{T}(y-x)\ge -\varepsilon -\|c\|\|y-x\|=-(1+\|c\|)\varepsilon$.
Then for any $ y\in P_b$,
\[
\frac{1}{1+\tau}(c+\tau b)^\mathsf{T} y\ge 0 \quad \Rightarrow\quad \frac{1}{1+\tau}(c+\tau b)\in P_a^*.
\] 
By the middle inequality in \eqref{condeps1}, we obtain $c\in S(P_a^*,\delta)\cap H_a$.

Consider now the case $c^\mathsf{T}x\le -c^\mathsf{T}a +\varepsilon$ for some $x\in S(K_b,\varepsilon)$, or, equivalently, $c^\mathsf{T} y \le \varepsilon$ for some $y = x + a\in S(P_b,\varepsilon)\cap H_b$.  Hence there exists $z\in P_b$ such that $\|z - y\|\le \varepsilon$ and so $c^\mathsf{T} z=c^\mathsf{T} y +c^\mathsf{T}(z- y)\le (1+\|c\|)\varepsilon=\tau<1/4$ by the left inequality in \eqref{condeps1}.  Then
\[\frac{1}{1-2 \tau}(c-2\tau b)^\mathsf{T} z\le - \tau \quad \Rightarrow \quad \frac{1}{1-2 \tau}(c-2\tau b) \not\in P_a^*.\] 
By the right inequality in \eqref{condeps1}, we obtain $c\not\in S_{H_a}(P_a^*,-\delta)$.
\end{proof}

\section{Approximation of Fenchel duals}\label{sec:fenchel}

Let $C\subseteq \mathbb{R}^n$ and $f: C \to \mathbb{R}$. Since the epigraph of $f$,
$\operatorname{epi}(f)=\{(x,t)\in C \times \mathbb{R} : f(x)\le t\}$, is in general noncompact, we introduce the following variant that preserves all essential features of the epigraph but has the added advantage of facilitating complexity theoretic discussions. For any $\alpha\in\mathbb{R}$, we let
\[
\operatorname{epi}_\alpha(f)=\{(x,t)\in C \times (-\infty, \alpha] :  f(x)\le t\}
\]
and call this the \emph{$\alpha$-epigraph} of $f$.
Clearly $f$ is a convex function if and only if $\operatorname{epi}_\alpha(f)$ is a convex set for all $\alpha \in \mathbb{R}$.

\begin{definition}\label{defncompfepi}  Let $C\subseteq \mathbb{R}^n$ be a bounded set with nonempty interior.  Let $f: C\to\mathbb{R}$ be a bounded function.  We define the following \emph{approximation problems} (\textsc{approx}).
\begin{enumerate}[\upshape (i)]
\item Approximation problem for $f$: Given any $x\in \mathbb{Q}^n\cap C$ and any rational $\varepsilon >0$, find an $\omega(x)$ 
such that $\omega(x)-\varepsilon< f(x)<\omega(x)+\varepsilon$.
\item  Approximation problem for $\mu:=\inf_{x\in C} f(x)$: Given any rational $\varepsilon>0$, find $\mu(\varepsilon)\in\mathbb{Q}$ such $\mu-\varepsilon<\mu(\varepsilon)<\mu+\varepsilon$. 
\end{enumerate}
\end{definition}
(i) is of course a generalization of Definition~\ref{def:approx} from norms to a more general function. We will show that (i) and (ii) are polynomial-time inter-reducible. For this purpose, we will need a useful corollary \cite[Corollary~4.3.12]{GLS88} of the Yudin--Nemirovski Theorem (cf.\ Theorem~\ref{thm:YN})
with the \textsc{wopt} problem in place of the \textsc{wval} problem.

\begin{corollary}[Yudin--Nemirovski]\label{cor:YN}
Let $C \subseteq \mathbb{R}^n$ be a compact convex set with nonempty interior for which we have knowledge of $a \in \mathbb{Q}^n$ and $0< r \le R \in \mathbb{Q}$ such that $B(a,r) \subseteq C \subseteq B(0,R)$. Then the \textsc{wopt} problem for $C$ is polynomial-time reducible to the \textsc{wmem} problem for $C$.
\end{corollary}

We will rely on this to show that for a convex function $f : C \to \mathbb{R}$, the approximation problem for $\inf_{x \in C} f(x)$ is polynomial-time reducible to the  approximation problem for $f$.

\begin{lemma}\label{mincomput}
Let $C\subseteq \mathbb{R}^n$ be a compact convex set with nonempty interior where \textsc{mem}  in $C$ can be checked in polynomial time. Let $f:C \to \mathbb{R}$ be a continuous convex functions with $\lvert f(x) \rvert \le \alpha$ for some rational $\alpha > 0$.
Suppose that there exists a rational $\delta>0$ such that  
\begin{equation}\label{mindeltacond}
\mu:=\min_{x\in C} f(x)=\min_{x\in S(C,-\delta)} f(x) .
\end{equation}
Then the approximation problem for $\mu$ is polynomial-time reducible to the approximation problem for $f$. (Note that we require knowledge of the values of both $\alpha$ and $\delta$, not just of their existence.)
\end{lemma}
\begin{proof}
We will show that \textsc{wopt} in $\operatorname{epi}_{2\alpha}(f)$ yields a solution to \textsc{approx} for $\mu$. The result then follows from two polynomial-time reductions: \textsc{wopt} in $\operatorname{epi}_{2\alpha}(f)$ can be reduced to \textsc{wmem} in $\operatorname{epi}_{2\alpha}(f)$, \textsc{wmem} in $\operatorname{epi}_{2\alpha}(f)$ can be reduced to \textsc{approx} for $f$.

As $f$ is a continuous convex function and $C$ is compact with nonempty interior, $C' :=\operatorname{epi}_{2\alpha}(f)$ is a compact convex set with interior in $\mathbb{R}^{n+1}$.
We claim that the \textsc{wmem} in $C'$ is polynomial-time reducible to the approximation problem for $f$.  Let $\varepsilon\in \mathbb{Q}$ with $0 < \varepsilon < \alpha$  and $(x,t)\in\mathbb{Q}^{n+1}$. If $x\notin C$  or $t>2\alpha$, then $(x,t)\notin C'$ and so $(x,t)\notin S(C',-\varepsilon)$. Now suppose $x\in C$ and $t\le 2\alpha$. An oracle call to the approximation problem for $f$ gives us $\omega(x)$ with $\omega(x)-\varepsilon<f(x)<\omega(x)+\varepsilon$. If $t\ge \omega(x)$, then as $(x,t)+(0,\varepsilon)\in C'$, it follows 
that $(x,t)\in S(C',\varepsilon)$.  If $t<\omega(x)$, then as $(x,t)-(0,\varepsilon)\notin C'$, it follows that $(x,t)\notin S(C',-\varepsilon)$.

By Corollary~\ref{cor:YN}, \textsc{wopt} in $C'$ is polynomial-time reducible to \textsc{wmem} in $C'$.  Therefore given $\varepsilon \in \mathbb{Q}$ with $0 < \varepsilon < \min(\alpha,\delta)$ and $\gamma=(0,\dots,0,-1)\in \mathbb{Z}^{n+1}$, by an oracle call to \textsc{wmem} in $C'$,
we may find $(y,s)\in S(C',\varepsilon)$ such that 
\[
\gamma^\mathsf{T}(x,t)=-t\le \gamma^\mathsf{T}(y,s)+\varepsilon=-s+\varepsilon
\]
for all $(x,t)\in S(C',-\varepsilon)$. We claim that $s = \mu(\varepsilon)$, the required approximation to $\mu$.
Since $\varepsilon\ge \delta$, it follows that $S(C',-\varepsilon)\supseteq S(C',-\delta)$.  The assumption \eqref{mindeltacond} ensures that $(x^\star, \mu)\in S(C,-\delta)$ where $f(x^\star) = \mu$.
Hence we deduce that $s\le \mu+\varepsilon$, i.e., $\mu\ge s-\varepsilon$.   As $(y,s)\in S(C',\varepsilon)$, it follow that there exists $(x',t')\in C'$ such that 
$t'\ge f(x')$ and $\lvert t'-s\rvert \le \varepsilon$. So $s\ge t'-\varepsilon\ge \mu-\varepsilon$. Thus
$\mu-\varepsilon \le s \le \mu+\varepsilon$, but starting with $2\varepsilon$ in place of $\varepsilon$ allows us to replace `$\le$' by `$<$' as required in Definition~\ref{defncompfepi}(iii).

\end{proof}
 
We now turn to the computational complexity of \emph{Fenchel dual} \cite{Roc}. Our results here require that $f$ be defined on all of $\mathbb{R}^n$.  Recall that for a function $f : \mathbb{R}^n \to \mathbb{R}$, its Fenchel dual is defined to be the function $f^*: \mathbb{R}^n\to (-\infty,\infty]$,
\[
f^*(y):=\sup_{x\in \mathbb{R}^n} y^\mathsf{T} x -f (x).
\]
The Fenchel dual is also known as the \emph{Fenchel conjugate} and the map $f \mapsto f^*$ is sometimes  called the \emph{Legendre transform}. It is well-known that $f^*$ is always a convex function, being the pointwise supremum of a family of affine functions  $y \mapsto y^\mathsf{T}x -f(x)$.  It is  also well-known that if $f$ is convex, then $f^{**}=f$.

Suppose that given any inputs $x \in \mathbb{Q}^n$ and $0 < \varepsilon \in \mathbb{Q}$, we can compute $f(x)$ to within precision $\varepsilon$  in polynomial-time. What can be we say about the complexity of computing $f^*(y)$ for an input $y \in \mathbb{Q}^n$ to a certain precision?  We will see that if $f$ is not convex, then the computation of $f^*$ can be NP-hard at least for some $y$. However, when $f$ is convex and satisfies certain growth conditions, computing $f^*$ is a problem that is polynomial-time reducible to computing $f$. Furthermore $f^*$ would satisfy the same growth conditions so that computing $f$ and computing $f^*$ are in fact polynomial-time inter-reducible.

Let $g : \mathbb{R}^n \times \mathbb{R}^n \times \mathbb{R}^n \to \mathbb{R}$, $(x,y,z) \mapsto \sum_{i,j,k=1}^n a_{ijk} x_i y_j z_k$ be a multilinear function. Let $D=\{(x,y,z) \in \mathbb{R}^{3n}: \|x\|\le 1,\; \|y \| \le 1, \; \|z \| \le 1\}$. We define a nonconvex function $f$ as follows:  For $(x,y,z)\in D$, $f(x,y,z):=-g(x,y,z)$.  For $(x,y,z)\notin D$, let $t=1/\max(\|x\|,\|y\|,\|z\|)$ and set $f(x,y,z):=-g(tx,ty,tz)$. It is trivial to compute $f$ for any $(x,y,z) \in \mathbb{R}^{3n}$ but $f^*(0)=\max_{(x,y,z)\in D} g(x,y,z)$ is NP-hard to approximate in general \cite[Theorem~10.2]{HL13}.

In what follows let $f : \mathbb{R}^n \to \mathbb{R}$ be a  continuous convex function. We will assume that $f$ satisfies the following growth condition:
\begin{equation}\label{growthcondf1}
k_f\|x\|^s\le f(x)\le K_f\|x\|^t \quad \text{whenever} \quad \|x\|\ge r.
\end{equation} 
for some constants $0<k_f \le K_f$,  $1<s \le t$, and $r > 0$ depending on $f$.
We now show that $f^*$ must satisfy similar growth conditions
\begin{equation}\label{growthcondf*}
k_{f^*}\|y\|^{s'}\le f^*(y)\le K_{f^*}\|y\|^{t'}  \quad  \text{whenever} \quad \|y\|\ge r',
\end{equation}
but with possibly different constants.  
\begin{lemma}\label{equivgrowthcond}
Let $f:\mathbb{R}^n\to \mathbb{R}$ be a convex function and let $f^*:\mathbb{R}^n\to (-\infty,\infty]$ be its Fenchel dual.
Then $f$ satisfies \eqref{growthcondf1} if and only if $f^*$ satisfies \eqref{growthcondf*}.
\end{lemma}
\begin{proof}
For $\|x\|\ge r$, the lower bound in  \eqref{growthcondf1} and $y^\mathsf{T}x\le \|y\|\|x\|$ give
\begin{equation}\label{basineq}
y^\mathsf{T}x - f(x)\le \|y\|\|x\|-k_f\|x\|^s=\|x\|(\|y\| - k_f\|x\|^{s-1}).
\end{equation}
Observe that for $z \in [0, \infty)$, the maximum of $h(z):=\|y\| z- k_f z^s$ is attained at
\[
z^\star=\left(\frac{\|y\|}{k_f s}\right)^{1/(s-1)},
\]
with maximum value
\[
h(z^\star)=\frac{s-1}{s}\|y\|z^\star=\frac{s-1}{s (k_f s)^{1/(s-1)}} \|y\|^{s/(s-1)}.
\]

Let $\mu:=\min_{\lVert x \rVert \le r} f(x)$.  Then
\[
\max_{\lVert x \rVert \le r} y^\mathsf{T} x -f (x)\le \|y\| r-\mu.
\]
Combine this with \eqref{basineq} and we obtain
\[
f^*(y)\le \max\left(\|y\| r-\mu, \frac{s-1}{s (k_f s)^{1/(s-1)}} \|y\|^{s/(s-1)}\right).
\]
This last inequality yields the upper bound in \eqref{growthcondf*}
with 
\[
K_{f^*}= \frac{s-1}{s (k_f s)^{1/(s-1)}},\qquad t'=\frac{s}{s-1},\qquad r'\ge r_1,
\]
for a corresponding $r_1$  that depends on $k_f,s,r,\mu$.  More precisely, either $r_1=0$ or $r_1$ is the unique positive solution of
\[
r_1r-\mu=\frac{s-1}{s (k_f s)^{1/(s-1)}} r_1^{s/(s-1)}.
\]

To deduce the lower bound in \eqref{growthcondf*}, let $y$ be such that
\[
\|y\| \ge r^{t-1} K_f t.
\]
Choose $x = cy$ such that
\[
\|x\|= \left( \frac{\|y\|}{K_f t} \right)^{1/(t-1)}.
\]
It follows that $\| x \| \ge r $ and so  the upper bound in \eqref{growthcondf1} yields $f^*(y)\ge \|y\|\|x\|-K_f\|x\|^t $.
Hence we have the lower bound in \eqref{growthcondf*} with
\[
k_{f^*} = \frac{t-1}{t (K_f t)^{1/(t-1)}}, \qquad s'=\frac{t}{t-1}. \qedhere
\]
\end{proof}

\begin{theorem}\label{compcpmplexf}
Let  $f:\mathbb{R}^n\to  \mathbb{R}$ be a convex function satisfying  
\eqref{growthcondf1}.  Then the approximation problem for $f^*$ is polynomial-time reducible to the approximation problem for $f$.
\end{theorem}
\begin{proof}
We will compute an approximation of $f^*(y)$ with oracle calls to approximations of $f(x)$.

Suppose first that $y=0$ and we need to compute an approximation of $f^*(0)=\min_{x\in\mathbb{R}^n} -f(x)$.  By the lower bound in \eqref{growthcondf1}, there is some $\rho_0=\rho(r, k_f,s)\in \mathbb{Q}\cap(0,\infty)$ such that $-f(x)<-f(0)$ whenever $\|x\|\ge \rho_0$.   Hence 
\[
f^*(0)=\max_{\|x\|\le \rho_0}-f(x)=-\min_{\|x\|\le R(0)} f(x)=\min_{\|x\|\le \rho_0+1} f(x).
\]
Let $C=B(0,\rho_0+1)$. Since \textsc{mem} in a Euclidean ball $B(0,\rho)$ is clearly polynomial-time decidable, the conditions of Lemma~\ref{mincomput} are satisfied.  Hence \textsc{approx} for $f^*(0)$ is polynomial-time reducible to \textsc{approx} for $f$.

Suppose now that $y\ne 0$.  Clearly $f^*(y)\ge -f(0)$.  Let $\rho>r$, where $r$ is as in \eqref{growthcondf1}.  Let $f^*_\rho(y):=\max_{\|x\|=\rho} y^\mathsf{T}x -f(x)$.
As $y^\mathsf{T}x\le \|y\| \|x\|$, the lower bound in \eqref{growthcondf1} gives
\[
f^*_{\rho}(y)\le \|x\|(\|y\|-k\|x\|^{s-1})= \rho(\|y\|-k\rho^{s-1}).
\]
Hence there exists $\rho_1  = \rho(\|y\|,k_f,s)\in \mathbb{Q}\cap (r,\infty)$ such that $-f(0)> f^*_{\rho}$.  Hence
\[f^*(y)=-\min_{\|x\|\le \rho_1} f(x)-y^\mathsf{T}x=-\min_{\|x\|\le \rho_1+1} f(x)-y^\mathsf{T}x.\]
Let $0 \ne y\in \mathbb{Q}^n$ and $C=B(0,\rho_1+1)$.  Then the conditions of Lemma~\ref{mincomput} are satisfied.  Hence \textsc{approx} for $f^*(y)$ is polynomial-time reducible to \textsc{approx} for $f$.
\end{proof}

Since $f^{**} = f$ for a convex function and by Lemma~\ref{growthcondf1}, $f$ and $f^*$ both satisfy the polynomial growth condition if either one does, we obtain the following.
\begin{corollary}
Let  $f:\mathbb{R}^n\to  \mathbb{R}$ be a convex function satisfying  
\eqref{growthcondf1}.  The approximation problem for $f^*$ is polynomial-time computable (resp.\ NP-hard) if and only if the approximation problem for $f$ is polynomial-time computable (resp.\ NP-hard).
\end{corollary}

\section*{Acknowledgment}

Both authors would like to thank Lev~Reyzin for helpful discussions. We first learned about  the work in \cite{DG14} and that the problem of complexity of dual cones  was opened from Shuzhong~Zhang, this is gratefully acknowledged.
SF's work is partially supported by NSF DMS-1216393.  
LH's work is partially supported by AFOSR FA9550-13-1-0133, DARPA D15AP00109, NSF IIS 1546413, DMS 1209136, and DMS 1057064.

\bibliographystyle{plain}
\begin{thebibliography}{1}
\bibitem{BM87} J.~Bourgain and V.~D.~Milman, ``New volume ratio properties for convex symmetric bodies in $\mathbb{R}^n$,'' \emph{Invent.\ Math.}, \textbf{88} (1987), no.~2, pp.~319--340.

\bibitem{DG14} P.~J.~C.~Dickinson and L.~Gijben, ``On the computational complexity of membership problems for the completely positive cone and its dual,'' \emph{Comput.\ Optim.\ Appl.}, \textbf{57} (2014), no.~2, pp.~403--415.

\bibitem{DFK91} M.~Dyer, A.~Frieze, and R.~Kannan, ``A random polynomial-time algorithm for approximating the volume of convex bodies,'' \emph{J.\ Assoc.\ Comput.\ Mach.}, \textbf{38} (1991), no.~1, pp.~1--17. 

\bibitem{GLS88} M.~Gr\"{o}tschel, L.~Lov\'{a}sz, and A.~Schrijver, \emph{Geometric Algorithms and Combinatorial Optimization}, 2nd Ed., Algorithms and Combinatorics, \textbf{2}, Springer-Verlag, Berlin, 1993.

\bibitem{Gu02}  L.~Gurvits, ``Classical deterministic complexity of Edmonds problem and quantum entanglement,'' \emph{Proc.\ ACM Symp.\ Theory Comput.} (STOC), \textbf{35}, pp.~10--19, ACM Press,  New York, NY, 2003.

\bibitem{HL13} C.~J. Hillar and L.-H. Lim, ``Most tensor problems are NP-hard,''  \emph{J.\ Assoc.\ Comput.\ Mach.}, \textbf{60} (2013), no.~6, Art.~45, 39 pp.

\bibitem{MK87}K.~G.~Murty and S.~N.~Kabadi, ``Some NP-complete problems in quadratic and nonlinear programming,'' \emph{Math.\ Programming}, \textbf{39} (1987), no.~2, pp.~117--129.

\bibitem{Roc}  R.~T.~Rockafellar, \emph{Convex Analysis},  Princeton Mathematical Series, \textbf{28}, Princeton University Press, Princeton, NJ, 1970.

 \end{thebibliography}

\end{document}

