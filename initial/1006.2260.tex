
\documentclass[11pt]{amsart}

\usepackage{amssymb}
\usepackage{enumerate}
\usepackage{amsfonts}
\usepackage{amsmath}
\usepackage{mathrsfs}
\allowdisplaybreaks[1]

\newcounter{fig}

                               

\newtheorem{theorem}{Theorem}
\theoremstyle{plain}
\newtheorem{acknowledgement}{Acknowledgement}
\newtheorem{corollary}{Corollary}
\newtheorem{assumption}{Assumption}
\newtheorem{definition}{Definition}
\newtheorem{example}{Example}
\newtheorem{exercise}{Exercise}
\newtheorem{lemma}{Lemma}
\newtheorem{notation}{Notation}
\newtheorem{proposition}{Proposition}
\newtheorem{remark}{Remark}
\newtheorem{solution}{Solution}
\setlength{\textwidth}{6.5in}
\setlength{\oddsidemargin}{0in}
\setlength{\evensidemargin}{0in}
\setlength{\textheight}{9in}
\setlength{\topmargin}{0.0in}
\setlength{\headheight}{0.12in}

 
 
 
 
 
 
 
 
 

 
 
 

 
 

 
 

 
 
 

 

 
 
 
 

 
 
 
 
 
 
 
 
 

 
 
 
 
 
 
 
 

 
 
 
 
 
 
 
 
 
 
 
 

 

 

\begin{document} 
\title{Measures on Semilattices}
{\author{Gianluca Cassese}
                     \address{Universit\`{a} Milano Bicocca}
                     \email{gianluca.cassese@unimib.it}
                     \curraddr{Department of Statistics, Building U7, Room 2097, via Bicocca 
                               degli Arcimboldi 8, 20126 Milano - Italy}} \date \today \subjclass[2000]{Primary: 28A05, 60G05.} 
\keywords{Lattice, Modular set functions, Extension of measures, Non additive integral.}

\begin{abstract} 
Every vector-valued, semiadditive set function on a semilattice of sets extends uniquely 
to an additive set function on the generated ring. 
\end{abstract}

\maketitle

\section{Introduction and Notation} 
In this paper, $\Omega$ is a given, non empty set and ${\mathscr{A}}_0$ a collection of 
subsets of $\Omega$ closed with respect to one set operation, either $\cup$ or 
$\cap$, a semilattice (of sets). Moreover, $Y$ is a given, real vector space
and $\Phi_0:{\mathscr{A}}_0\to Y$ a vector valued set function on ${\mathscr{A}}_0$. ${\mathscr{A}}_1$ 
designates the lattice generated by ${\mathscr{A}}_0$, i.e. the smallest collection of subsets 
of $\Omega$ which contains ${\mathscr{A}}_0$ and is closed with respect to $\cup$ and 
$\cap$; ${\mathscr{A}}$ is the generated ring. In this paper we address the problem of
finding necessary and sufficient conditions for the existence of an extension
of $\Phi_0$ to ${\mathscr{A}}_1$ and to ${\mathscr{A}}$. These conditions are established in Theorems 
\ref{th extension} and \ref{th pettis}. 

The measure extension problem is, of course, a time honored one and, among 
the many contributions, it includes the major one of Horn and Tarski 
\cite[Theorem 1.21]{horn tarski}. Their strategy is to construct a linear functional 
associated with the given set function and to exploit the results on the extension of 
linear functionals, ultimately the Hahn Banach Theorem. The whole approach is 
therefore functional analytic and relies on the notion of partial  measure, developed 
in full generality in \cite[chapter 3]{rao}. Our strategy is different not only in the relative 
generality of the assumptions on the initial collection of sets, a semilattice, but also 
because it follows a measure theoretic way and obtains a fully explicit description of 
the resulting extension. We exploit rather the deep study of the properties of the 
M\"{o}bius function as illustrated by Rota \cite{rota}.

After introducing in section \ref{sec properties} the main properties involved -- 
semimodularity and semiadditivity -- and providing some examples, we show
in section \ref{sec general} that any set function on a semilattice is associated with 
a functional defined on an auxiliary space. This functional has in fact some nice
mathematical properties which we study in Theorem \ref{th pre norberg}, some
of which depend on those of the original set function. Exploiting this construction
we obtain a version of Jordan decomposition for real valued set functions on 
semilattices. The results obtained in this section find a rather speedy application 
in the following section \ref{sec pettis} to the problem of additive extensions of
set functions. We prove Theorems \ref{th extension} and \ref{th pettis} which give 
a complete characterization of this problem. These results are the generalized in 
section \ref{sec fubini} to product spaces. Section \ref{sec order} illustrates a particularly 
interesting special case which had originally been studied by Norberg and that has 
later become widely used in the theory of set-indexed stochastic processes. Eventually, 
section \ref{sec choquet} gives some results on measurability and integrability with 
respect to a set function defined on an arbitrary collection of sets.

We find convenient the following notation. When $S$ is a set, ${\vert S\vert}$ is its cardinality
and $2^S$ and ${\mathbf{1}_{S}}$ indicate the collection of all of its subsets and its indicator function. 
If $N\in{\mathbb{N}}$ we write $[N]=\{1,\ldots,N\}$. If $\Sigma\subset2^S$, then ${\mathscr{S}}(\Sigma)$ 
designates the collection of $\Sigma$ simple functions on $S$, ${\mathfrak{B}}(\Sigma)$ its closure 
with respect to the topology of uniform convergence and $ba(\Sigma)$ (resp. $ca(\Sigma)$) 
the family of bounded, finitely (resp. countably) additive set functions on $\Sigma$. In general 
we omit to refer to $\Sigma$ whenever $\Sigma=2^S$. If $\lambda$ is a set function on 
$\Sigma$, the $\lambda$ integral of $f$, if well defined, is denoted by $\lambda(f)$ or 
$\int fd\lambda$, according to notational convenience.

Let $X$ be the collection of all finite subsets of a given set $\mathcal X$. For each $b\subset X$  
we write
\begin{equation}
\label{nu}
\nu(b)=(-1)^{1+{\vert b\vert}}\qquad b\subset X
\end{equation}
If $\mathcal X=\{[N]:N\in{\mathbb{N}}\}$, we write $\nu(j)$ rather than $\nu([j])$.
If $a\subset b$, one easily concludes that $\nu(b)=\mu(a,b)\nu(a)$ where $\mu$ is the 
M\"{o}bius function on $X$, see \cite[Corollary, p. 345]{rota}. We thus 
deduce from \cite[Propositions 1 and 2, p. 344]{rota} that $\nu$ is the unique set 
function on $X$ satisfying satisfying: (\textit{i}) 
\begin{equation}
\label{Mobius function}
\nu({\varnothing})=-1\quad\text{and}\quad\sum_{a\subset x\subset b}\nu(x)=0
\qquad a,b\in X,\ a\subsetneq b
\end{equation}
and (\textit{ii}) if $f:X\to{\mathbb{R}}$ and $F(a,b)=\sum_{\{y:a\subset y\subset b\}}f(y)$ 
then the following inversion formulas hold
\begin{equation}
\label{Mobius inversion}
f(b)=\nu(b)\sum_{\{y:a\subset y\subset b\}}F(a,y)\nu(y)\quad \text{and}\quad
f(a)=\nu(a)\sum_{\{y:a\subset y\subset b\}}F(y,b)\nu(y)
\end{equation}

We provide an application in the following Lemma that will be useful in the sequel.

\begin{lemma}
\label{lemma combinatorial} 
Let ${\mathfrak a}$ and ${\mathfrak b}$ be finite sets, ${\mathfrak c}={\mathfrak a}\times{\mathfrak b}$ and denote by $K_j({\mathfrak a},{\mathfrak b})$ 
the number of subsets of ${\mathfrak c}$ which have cardinality $j$ and in which each $a\in{\mathfrak a}$ 
and $b\in{\mathfrak b}$ appear at least once. Then,
\begin{equation} 
\label{K}
\sum_{0<j\le{\vert {\mathfrak c}\vert}}\nu(j)K_j({\mathfrak a},{\mathfrak b})=\nu({\mathfrak a})\nu({\mathfrak b})
\end{equation}
\end{lemma}

\begin{proof} 
Consider forming subsets of ${\mathfrak c}$ of cardinality $j$ by using either (\textit{i}) 
\textit{at most} or (\textit{ii}) \textit{exactly} $n$ elements of ${\mathfrak a}$ and $m$
 elements of ${\mathfrak b}$. This can be done in a number of ways, $q_j(n,m)$ or $k_j(n,m)$
respectively. By classical combinatorial formulas we know that
\begin{align}
\label{q}
q_j(n,m)
=
{nm\choose j}{{\vert {\mathfrak a}\vert}\choose n}{{\vert {\mathfrak b}\vert}\choose m}
=
\sum_{\substack{0\le t\le n\{\emptyset}\le s\le m}}k_j(t,s)
=
\sum_{\substack{0<t\le n\{\emptyset}<s\le m}}k_j(t,s)
\end{align}
which can be inverted, via \eqref{Mobius inversion}, to give
\begin{align*}
\label{k}
k_j({\vert {\mathfrak a}\vert},{\vert {\mathfrak b}\vert})
&=
\nu({\vert {\mathfrak a}\vert})\nu({\vert {\mathfrak b}\vert})
\sum_{\substack{0\le n\le{\vert {\mathfrak a}\vert}\{\emptyset}\le m\le{\vert {\mathfrak b}\vert}}}\nu(n)\nu(m)q_j(n,m)
=
\nu({\vert {\mathfrak a}\vert})\nu({\vert {\mathfrak b}\vert})\sum_{\substack{0\le n\le{\vert {\mathfrak a}\vert}\{\emptyset}\le m\le{\vert {\mathfrak b}\vert}}}
(-1)^{n+m}{nm\choose j}{{\vert {\mathfrak a}\vert}\choose n}{{\vert {\mathfrak b}\vert}\choose m}
\end{align*}
We conclude from $K_j({\mathfrak a},{\mathfrak b})=k_j({\vert {\mathfrak a}\vert},{\vert {\mathfrak b}\vert})$ and 
$\sum_{0<j\le N}\nu(j){N\choose j}=1$ that
\begin{align*}
\sum_{0<j\le{\vert {\mathfrak c}\vert}}\nu(j)K_j({\mathfrak a},{\mathfrak b})
&=
\nu({\vert {\mathfrak a}\vert})\nu({\vert {\mathfrak b}\vert})\sum_{\substack{0<n\le{\vert {\mathfrak a}\vert}\{\emptyset}<m\le{\vert {\mathfrak b}\vert}}}
(-1)^{n+m}{{\vert {\mathfrak a}\vert}\choose n}{{\vert {\mathfrak b}\vert}\choose m}
\sum_{0<j\le nm}\nu(j){nm\choose j}=
\nu({\mathfrak a})\nu({\mathfrak b})
\end{align*}
\end{proof}

\section{Semimodular and Semiadditive Set Functions} 
\label{sec properties}

Several authors, including Bachman 
and Sultan \cite{bachman sultan} and Bhaskara Rao and Bhaskara Rao \cite{rao}, 
have studied set functions on lattices of sets\footnote{Some authors, e.g. Halmos \cite[p. 25]{halmos} and Bogachev \cite[p. 75]
{bogachev}, include the property ${\emptyset}\in{\mathscr{A}}$ in the definition of a lattice. Remark 
that a lattice of sets, endowed witht he order induced by inclusion, is also a lattice in 
the sense of ordered spaces with $\vee=\cup$ and $\wedge=\cap$. However, a family 
of sets may be given a lattice structure different from the one induced by sets operations 
(see the next section). To avoid confusion, in this paper we shall refer to a lattice of sets 
simply as a lattice while the corresponding functional analytic concept will be referred to 
as an order lattice.}
and the possibility of extending them to a larger class, typically the generated algebra or 
ring. This possibility hinges on the following property: we speak of $\Phi:{\mathscr{A}}_1\to Y$ as 
being strongly additive, in symbols $\Phi\in sa({\mathscr{A}}_1,Y)$, if (\textit{i}) $\Phi$ is modular, 
i.e.\footnote{Pettis 
\cite{pettis} calls this property $2$-additivity.} 
\begin{equation} 
\label{modular}
\Phi(A\cup B)+\Phi(A\cap B)=\Phi(A)+\Phi(B)\qquad A,B\in{\mathscr{A}}_1 
\end{equation}
and (\textit{ii}) it satisfies
\begin{equation} 
\label{sa}
\text{either}\qquad{\varnothing}\notin{\mathscr{A}}\quad\text{or}\qquad\Phi({\varnothing})={\emptyset} 
\end{equation}

Simple induction shows that \eqref{modular} is equivalent to either one of the following 
two properties\footnote{Chung \cite[p. 329]{chung} credits Poincar\'{e} for being the 
first to obtain \eqref{modular cap}.}:
\begin{subequations}
\label{semimodular}
\begin{equation} 
\label{modular cap}
\Phi\left(\bigcup_{n\in[N]}A_n\right)
=\sum_{{\varnothing}\ne b\subset[N]}\nu(b) \Phi\left(\bigcap_{n\in b}A_n\right) 
\qquad A_1,\ldots,A_N\in{\mathscr{A}}_1
\end{equation} 
\begin{equation} 
\label{modular cup}
\Phi\left(\bigcap_{n\in[N]}A_n\right)
=\sum_{{\varnothing}\ne b\subset[N]}\nu(b)\Phi\left(\bigcup_{n\in b}A_n\right) 
\qquad A_1,\ldots,A_N\in{\mathscr{A}}_1
\end{equation} 
\end{subequations}
It is also easy to see that each 
$\Phi\in sa({\mathscr{A}}_1,Y)$ is additive and that, if ${\mathscr{A}}_1$ is a ring, then the converse 
is also true. The case of modular or strongly additive set functions on lattices of 
sets was studied in depth by Pettis \cite{pettis}. 

We are interested in defining similar properties when the underlying family ${\mathscr{A}}_0$ 
is just a semilattice. Important examples of semilattices are, of course, ideals 
and filters; a $\cap$-semilattice is known in probability as a $\pi$ system. 

\begin{definition}
A set function $\Phi_0:{\mathscr{A}}_0\to Y$ is semimodular if either (\textit{i}) ${\mathscr{A}}_0$ is a 
$\cap$-semilattice and \eqref{modular cap} holds for every finite collection 
$A_1,\ldots,A_N\in{\mathscr{A}}_0$ such that $\bigcup_{n=1}^NA_n\in{\mathscr{A}}_0$ or (\textit{ii}) 
${\mathscr{A}}_0$ is a $\cup$-semilattice and \eqref{modular cup} holds for every finite 
collection $A_1,\ldots,A_N\in{\mathscr{A}}_0$ such that $\bigcap_{n=1}^NA_n\in{\mathscr{A}}_0$. 
\end{definition}

There exists of course a perfect symmetry between the two versions
of this property 

\begin{lemma}
\label{lemma conjugate}
${\mathscr{A}}_0$ is a semilattice if and only if so is ${\mathscr{A}}_0^c=\{A^c:A\in{\mathscr{A}}_0\}$. 
$\Phi_0:{\mathscr{A}}_0\to Y$ is semi modular if and only if so are its conjugate $\Phi_0^c$
and its translation $\Phi_0^y$ by $y\in Y$, i.e. the set functions 
\begin{align}
\label{conjugate} 
\Phi_0^c(A^c)&=-\Phi_0(A)&\Phi_0^y(A)&=\Phi_0(A)+y& A\in{\mathscr{A}}_0 
\end{align}
\end{lemma} 
\begin{proof} 
The first claim is obvious. Assume that ${\mathscr{A}}_0$ is a $\cap$-lattice and 
$\Phi_0:{\mathscr{A}}_0\to Y$ is semimodular. If $A_1\ldots,A_N,\bigcup_{n=1}^NA_n\in{\mathscr{A}}_0$, 
then, we conclude
\begin{align*}
\Phi^c_0\left(\bigcap_{n=1}^NA^c_n\right)
&=
\Phi_0\left(\bigcup_{n=1}^NA_n\right)
=
\sum_{{\varnothing}\ne b\subset[N]}\nu(b)\Phi_0\left(\bigcap_{n\in b}A_n\right)
=
\sum_{{\varnothing}\ne b\subset[N]}\nu(b)\Phi^c_0\left(\bigcup_{n\in b}A^c_n\right)
\end{align*} 
The fact that $\Phi_0$ is semimodular follows from \eqref{Mobius function}.
The poof for the case in which ${\mathscr{A}}_0$ is a $\cup$-semilattice is identical.
\end{proof}

Thus, by Lemma \ref{lemma conjugate} we will often just treat the case of
$\cap$-semilattices

We shall see later on that there are special cases of semilattices on which all set 
functions are semimodular. 

We extend to semilattices the property of strong additivity as follows:

\begin{definition}
$\Phi_0:{\mathscr{A}}_0\to Y$ is semiadditive, in symbols $\Phi_0\in sa_0({\mathscr{A}}_0,Y)$, if it may 
be extended as a semimodular set function to the semilattice ${\mathscr{A}}_0\cup\{{\varnothing}\}$ 
with $\Phi_0({\varnothing})={\emptyset}$. 
\end{definition}

Necessary and sufficient conditions for the existence of such 
an extension depend on the structure of ${\mathscr{A}}_0$. If ${\mathscr{A}}_0$ is $\cap$ closed $\Phi_0$ 
is semiadditive if and only if it satisfies \eqref{sa}; if ${\mathscr{A}}_0$ is $\cup$ closed, 
semiadditivity is equivalent to 
\begin{equation}
\label{sa cup}
\sum_{{\varnothing}\ne b\subset[N]}\nu(b)\Phi_0\left(\bigcup_{n\in b}A_n\right)={\emptyset}
\qquad A_1,\ldots,A_N\in{\mathscr{A}}_0,\ \bigcap_{n=1}^NA_n={\varnothing} 
\end{equation} 
In either case a semimodular set function on a semilattice is semiadditive whenever 
$\bigcap_{n=1}^NA_n\neq{\varnothing}$ for all $A_1,\ldots,A_N\in{\mathscr{A}}_0$. When dealing
with semiadditive set functions it will be tacitly assumed that ${\varnothing}\in{\mathscr{A}}_0$.

We end this section with some examples of semimodular ans semiadditive set functions.
\begin{example}
Let ${\mathscr{A}}_0$ the collection of all the finite dimensional subspaces of a vector space $X$ 
and define $\Phi_0:{\mathscr{A}}_0\to{\mathbb{R}}$ by letting $\Phi_0(A)=\dim(A)$. Then, ${\mathscr{A}}_0$ is a 
$\cap$-semilattice including ${\varnothing}$ and, by Grasmann formula, $\Phi_0$ is semiadditive.
\end{example}

\begin{example}
Let $H$ be a finite subset of $\Omega$ and define $\Phi_0^H:{\mathscr{A}}_0\to{\mathbb{N}}$ by letting 
\begin{equation*}
\Phi_0^H(A)=\frac{{\vert {A\cap H}\vert}}{{\vert H\vert}}\qquad A\in{\mathscr{A}}_0
\end{equation*}
Then $\Phi_0^H\in sa_0({\mathscr{A}}_0)$. Moreover, if ${{{\left\langle {{H}}_{{{n}}}\right\rangle_{{{n}}\in {\mathbb{N}}} }}}$ is an increasing
sequence of subsets of $\Omega$ we can define $\Phi_0=\operatorname*{LIM}_n\Phi_0^{H_n}$,
where $\operatorname*{LIM}$ denotes the Banach limit. Then, $\Phi_0\in sa_0({\mathscr{A}}_0)$.
\end{example}

\begin{example}
\label{ex dirac}
Let ${\mathscr{A}}_0$ be a $\cup$-semilattice and $Y={\mathbb{R}}^\Omega$. Define 
$\Phi_0:{\mathscr{A}}_0\cup\{{\varnothing}\}\to Y$ by letting $\Phi_0(A)={\mathbf{1}_{{A}}}$. Then, 
since
\begin{equation}
\label{lattice representation}
{\mathbf{1}_{{\bigcap_nA_n}}}=\sum_{{\varnothing}\ne b\subset[N]}\nu(b){\mathbf{1}_{{\bigcup_{n\in b}A_n}}}
\end{equation}
$\Phi_0\in sa_0({\mathscr{A}}_0)$ and the same is true of $\delta_x$, the Dirac 
measure sitting at $x$, for all $x\in X$. 
\end{example}

In the next section we shall shortly study set functions on semilattices of sets
with no additional property. We shall see, in particular, that each such function is 
associated with a functional defined on an auxiliary space that possesses several
good mathematical properties and that makes the extension of $\Phi_0$ fairly
simple.

\section{General Set Functions on Semilattices}
\label{sec general}

For reasons of clarity we assume in this section that ${\mathscr{A}}_0$ is a $\cap$-semilattice 
although all results obtained carry over to $\cup$-semilattices with obvious changes. 

Let $s_0({\mathscr{A}}_0)$ be the collection of all finite sequences from ${\mathscr{A}}_0$, including the
empty one, and considered independently from order, i.e. in which two finite 
sequences obtained by permutation are identified\footnote{One should avoid confusing the empty sequence in $s_0({\mathscr{A}}_0)$ with the
empty set in ${\mathscr{A}}_0$ although both will be denoted, as usual, by ${\varnothing}$. Thus, if
${\varnothing}\in{\mathscr{A}}_0$ then $\{{\varnothing}\}$ is a (non empty) element in $s_0({\mathscr{A}}_0)$.}. 
We can endow $s_0({\mathscr{A}}_0)$ with an order relation by writing
\begin{align}
\label{order}
{\mathfrak a}\ge{\mathfrak b}
\quad\text{whenever for each }B\in{\mathfrak b}\text{ there exists }A\in{\mathfrak a}\text{ such that } 
B\subset A
\end{align}
The relation so defined is clearly reflexive and transitive, although not antisymmetric.
We can define some binary operations on $s_0({\mathscr{A}}_0)$. In particular for 
${\mathfrak a},{\mathfrak b}\in s_0({\mathscr{A}}_0)$ we denote by ${\mathfrak a}{\overset\circ\cup}{\mathfrak b}$ the sequence obtained by 
chaining ${\mathfrak a}$ and ${\mathfrak b}$ together. We also let ${\mathfrak a}{\overset\circ\cap}{\mathfrak b}=\{A\cap B:A\in{\mathfrak a},B\in{\mathfrak b}\}$\footnote{Observe that ${\mathfrak a}\cap{\mathfrak b}$ is an admissible binary operation and that in fact
${\mathfrak a}\cap{\mathfrak b}\subset{\mathfrak a}{\overset\circ\cap}{\mathfrak b}$.}.
It is clear that these operations are associative, commutative and satisfy the distributive 
property. Moreover, if ${\mathfrak c}\in s_0({\mathscr{A}}_0)$ then ${\mathfrak c}\ge{\mathfrak a},{\mathfrak b}$ implies 
${\mathfrak c}\ge{\mathfrak a}{\overset\circ\cup}{\mathfrak b}\ge{\mathfrak a},{\mathfrak b}$ while ${\mathfrak c}\le{\mathfrak a},{\mathfrak b}$ implies
${\mathfrak c}\le{\mathfrak a}{\overset\circ\cap}{\mathfrak b}\le{\mathfrak a},{\mathfrak b}$. Thus, moving to the quotient space, ${\mathfrak a}{\overset\circ\cup}{\mathfrak b}$
may be written as ${\mathfrak a}\vee{\mathfrak b}$ and ${\mathfrak a}{\overset\circ\cap}{\mathfrak b}$ as ${\mathfrak a}\wedge{\mathfrak b}$.

We start noting that each $\Phi_0:{\mathscr{A}}_0\to Y$ is associated with a 
functional on $s_0({\mathscr{A}}_0)$, defined implicitly 
as:
\begin{equation}
\label{T}
T({\mathfrak a})
=
\sum_{{\varnothing}\ne\alpha\subset{\mathfrak a}}\nu(\alpha)\Phi_0\left(\bigcap_{A\in\alpha}A\right)
\qquad
{\mathfrak a}\in s_0({\mathscr{A}}_0)
\end{equation}
At times, to avoid confusion, we will write $T$ in \eqref{T} as $T_{\Phi_0}$.
Indeed the right hand side of \eqref{T} is invariant with respect to permutations;
moreover, $T({\varnothing})={\emptyset}$. Other properties of $T$ are proved hereafter.

\begin{theorem} 
\label{th pre norberg} 
Let $\Phi_0:{\mathscr{A}}_0\to Y$ and $T:s_0({\mathscr{A}}_0)\to Y$ be related via \eqref{T}.
Then,
\begin{equation}
\label{decomposition}
T({\mathfrak a}{\overset\circ\cup}{\mathfrak b})
=
T({\mathfrak a})+T({\mathfrak b})-T({\mathfrak a}{\overset\circ\cap}{\mathfrak b})
\qquad{\mathfrak a},{\mathfrak b}\in s_0({\mathscr{A}}_0)
\end{equation}
or, equivalently,
\begin{equation}
\label{modular T}
\sum_{n=1}^NT({\mathfrak a}_n)
=
\sum_{k=1}^NT
\left(\underset{\{b\subset[N]:{\vert b\vert}=k\}}{\overset\circ\bigcup}\ \underset{i\in b}{\overset\circ\bigcap}{\mathfrak a}_i\right)
\qquad
{\mathfrak a}_1,\ldots,{\mathfrak a}_N\in s_0({\mathscr{A}}_0)
\end{equation}
Moreover,
\begin{equation}
\label{parsimony}
T({\mathfrak a}{\overset\circ\cup}{\mathfrak b})=T({\mathfrak a})
\qquad{\mathfrak a},{\mathfrak b}\in s_0({\mathscr{A}}_0),\ {\mathfrak a}\ge{\mathfrak b}
\end{equation}
If $B\in{\mathscr{A}}_0$, ${\mathfrak a}=\{A_1,\ldots,A_N\}$, ${\mathfrak a}'=\{A_1,\ldots,A_{N-1},A_N\cap B\}$
and ${\mathfrak a}''=\{A_N\cap A_1,\ldots,A_N\cap B\}$ then
\begin{equation}
\label{associative}
T({\mathfrak a}')
=
T({\mathfrak a})+T({\mathfrak a}'')-\Phi_0(A_N)
\end{equation}
\end{theorem}

\begin{proof} 
Any nonempty subset of ${\mathfrak a}{\overset\circ\cup}{\mathfrak b}$ may be written as $a{\overset\circ\cup} b$ with $a\subset{\mathfrak a}$, 
$b\subset{\mathfrak b}$ and $a{\overset\circ\cup} b\ne{\varnothing}$. Denote by $\Gamma_j(\alpha,\beta)$ the subset 
of $\alpha{\overset\circ\cap}\beta$ consisting of exactly $j$ intersections $A\cap B$ in which all 
$A\in\alpha$ and $B\in\beta$ are included. The cardinality of $\Gamma_j(\alpha,\beta)$ 
was denoted by $K_j(\alpha,\beta)$ in Lemma \ref{lemma combinatorial}. Observe that 
if $\gamma\in\Gamma_j(\alpha,\beta)$ then 
$\bigcap_{C\in\gamma}C=\bigcap_{A\in\alpha}A\cap\bigcap_{B\in\beta}C$. 
Thus, given that $\nu(\alpha{\overset\circ\cup}\beta)=(-1)^{1+{\vert \alpha\vert}+{\vert \beta\vert}}=-\nu(\alpha)\nu(\beta)$, \eqref{decomposition} 
follows from:
\begin{align}\label{long}
\nonumber
T({\mathfrak a}{\overset\circ\cup}{\mathfrak b})
&=
\sum_{\substack{\alpha\subset{\mathfrak a},\ \beta\subset{\mathfrak b}\\{\varnothing}\ne\alpha{\overset\circ\cup}\beta}}
\nu(\alpha{\overset\circ\cup}\beta)\Phi_0\left(\bigcap_{A\in\alpha}A\cap\bigcap_{B\in\beta}B\right)
\\
\nonumber
&=
T({\mathfrak a})+T({\mathfrak b})
-\sum_{\substack{{\varnothing}\ne\alpha\subset{\mathfrak a}\\ {\varnothing}\ne\beta\subset{\mathfrak b}}}
\nu(\alpha)\nu(\beta)\Phi_0\left(\bigcap_{A\in\alpha}A\cap\bigcap_{B\in\beta}B\right)
\\
&=
T({\mathfrak a})+T({\mathfrak b})-
\sum_{\substack{{\varnothing}\ne\alpha\subset{\mathfrak a}\\ {\varnothing}\ne\beta\subset{\mathfrak b}}}\ 
\sum_{0<j\le{\vert \alpha\vert}{\vert \beta\vert}}\nu(j)
K_j(\alpha,\beta)\Phi_0\left(\bigcap_{A\in\alpha}A\cap\bigcap_{B\in\beta}B\right)
&(\text{by }\eqref{K})
\\
\nonumber
&=
T({\mathfrak a})+T({\mathfrak b})-
\sum_{\substack{{\varnothing}\ne\alpha\subset{\mathfrak a}\\ {\varnothing}\ne\beta\subset{\mathfrak b}}}\ 
\sum_{0<j\le{\vert \alpha\vert}{\vert \beta\vert}}
\sum_{\gamma\in\Gamma_j(\alpha,\beta)}
\nu(\gamma)\Phi_0\left(\bigcap_{C\in\gamma}C\right)\\
\nonumber
&=
T({\mathfrak a})+T({\mathfrak b})-
\sum_{{\varnothing}\ne\gamma\subset{\mathfrak a}{\overset\circ\cap}{\mathfrak b}}\nu(\gamma)\Phi_0\left(\bigcap_{C\in\gamma}C\right)\\
\nonumber
&=
T({\mathfrak a})+T({\mathfrak b})-T({\mathfrak a}{\overset\circ\cap}{\mathfrak b})
\end{align}
This proves \eqref{decomposition}. Suppose that \eqref{modular T}, which is clearly true
for $N=1$, holds up to some integer $N-1$. Let
$
{\mathfrak a}_N[k]
=
{\overset\circ\bigcup}_{\{b\subset[N]:{\vert b\vert}=k\}}\ {\overset\circ\bigcap}_{i\in b}{\mathfrak a}_i
$
so that 
$
{\overset\circ\bigcup}_{\{\{N\}\subset b\subset[N]:{\vert b\vert}=k\}}\ {\overset\circ\bigcap}_{i\in b}{\mathfrak a}_i
=
{\mathfrak a}_{N-1}[k-1]{\overset\circ\cap}{\mathfrak a}_N
$.
But then for $1<k<N$ we have
\begin{align*}
T({\mathfrak a}_N[k])
&=
T
\left(\left(\underset{\{\{N\}\subset b\subset[N]:{\vert b\vert}=k\}}{\overset\circ\bigcup}\  \underset{i\in b}{\overset\circ\bigcap}\ {\mathfrak a}_i\right){\overset\circ\cup}\left(\underset{\{b\subset[N-1]:{\vert b\vert}=k\}}{\overset\circ\bigcup}\ \underset{i\in b}{\overset\circ\bigcap}\ {\mathfrak a}_i\right)\right)\\
&=
T\left(\left({\mathfrak a}_{N-1}[k-1]{\overset\circ\cap}{\mathfrak a}_N\right){\overset\circ\cup}{\mathfrak a}_{N-1}[k]\right)\\
&=
T\left({\mathfrak a}_{N-1}[k-1]{\overset\circ\cap}{\mathfrak a}_N\right)
+T\left({\mathfrak a}_{N-1}[k]\right)
-T\left({\mathfrak a}_{N-1}[k-1]{\overset\circ\cap}{\mathfrak a}_N{\overset\circ\cap}{\mathfrak a}_{N-1}[k]\right)
&(\text{by }\eqref{decomposition})
\\
&=
T\left({\mathfrak a}_{N-1}[k-1]{\overset\circ\cap}{\mathfrak a}_N\right)
+T\left({\mathfrak a}_{N-1}[k]\right)
-T\left({\mathfrak a}_{N-1}[k]{\overset\circ\cap}{\mathfrak a}_N\right)
\end{align*}
from which follows
\begin{align*}
\sum_{k=1}^NT({\mathfrak a}_N[k])
&=
T({\mathfrak a}_N[1])+T({\mathfrak a}_N[N])
+\sum_{k=2}^{N-1}T({\mathfrak a}_{N-1}[k])\\
&\quad+T\left({\mathfrak a}_{N-1}[1]{\overset\circ\cap}{\mathfrak a}_N\right)
-T\left({\mathfrak a}_{N-1}[N-1]{\overset\circ\cap}{\mathfrak a}_N]\right)\\
&=
T({\mathfrak a}_N[1])
+T\left({\mathfrak a}_{N-1}[1]{\overset\circ\cap}{\mathfrak a}_N\right)
+\sum_{k=2}^{N-1}T({\mathfrak a}_{N-1}[k])\\
&=
T({\mathfrak a}_{N-1}[1])+T({\mathfrak a}_N)
+\sum_{k=2}^{N-1}T({\mathfrak a}_{N-1}[k])
&(\text{by }\eqref{decomposition})\\
&=
T({\mathfrak a}_N)+\sum_{k=1}^{N-1}T({\mathfrak a}_{N-1}[k])\\
&=
\sum_{n=1}^NT({\mathfrak a}_n)
&(\text{by induction})
\end{align*}
Thus, \eqref{decomposition} implies \eqref{modular T} while the converse is obvious.

To prove \eqref{parsimony}, observe that
\begin{equation*}
T({\mathfrak b})
-\sum_{{\varnothing}\ne\alpha\subset{\mathfrak a},\ {\varnothing}\ne\beta\subset{\mathfrak b}}\nu(\alpha)\nu(\beta)
\Phi_0\left(\bigcap_{A\in\alpha}A\cap\bigcap_{B\in\beta}B\right)
=
\sum_{\alpha\subset{\mathfrak a},\ {\varnothing}\ne\beta\subset{\mathfrak b}}\nu(\alpha)\nu(\beta)
\Phi_0\left(\bigcap_{A\in\alpha}A\cap\bigcap_{B\in\beta}B\right)
\end{equation*}
Assume that ${\mathfrak b}\le{\mathfrak a}$, fix $B_0\in\beta\subset{\mathfrak b}$ and let $A_0\in{\mathfrak a}$ be 
such that $B_0\subset A_0$. If $\alpha\subset{\mathfrak a}$ 
and $A_0\notin\alpha$ then 
\begin{equation*}
\Phi_0\left(\bigcap_{A\in\alpha}A\cap\bigcap_{B\in\beta}B\right)
=\Phi_0\left(\bigcap_{A\in\alpha\cup\{A_0\}}A\cap\bigcap_{B\in\beta}B\right)
\end{equation*}
but $\nu(\alpha)=-\nu(\alpha\cup\{A_0\})$ so that all terms containing $\beta$ 
will vanish. We thus obtain \eqref{parsimony}. 

Let ${\mathfrak a}'$ and ${\mathfrak a}''$ be as in the claim and write $A'_n=A_n$, $A''_n=A_N\cap A_n$ 
for $n<N$ and $A'_N=A''_N=A_N\cap B$. Then,
\begin{align*}
T({\mathfrak a}')-T({\mathfrak a})
&=
\sum_{\{N\}\subset b\subset[N]}\nu(b)
\left\{\Phi_0\left(\bigcap_{n\in b}A'_n\right)-\Phi_0\left(\bigcap_{n\in b}A_n\right)\right\}\\
&=
\sum_{\{N\}\subset b\subset[N]}\nu(b)\Phi_0\left(\bigcap_{n\in b}A''_n\right)
+
\sum_{b\subset[N-1]}\nu(b)\Phi_0\left(\bigcap_{n\in b}A''_n\right)\\
&=
T({\mathfrak a}'')-\Phi_0(A_N)
\end{align*}
\end{proof}

It follows from \eqref{parsimony} that $T({\mathfrak a})=T({\mathfrak b})$ whenever 
${\mathfrak a},{\mathfrak b}\in s_0({\mathscr{A}}_0)$ are in the same equivalence class defined by \eqref{order},
i.e. when each $A\in{\mathfrak a}$ is contained in some $B\in{\mathfrak b}$ and viceversa. Thus 
$T$ may be regarded as a map defined on the quotient space 
$s_0({\mathscr{A}}_0)\slash\sim$ of $s_0({\mathscr{A}}_0)$. Observe that the equivalence class of
each ${\mathfrak a}\in s_0({\mathscr{A}}_0)$ contains only one element, ${\mathfrak a}_*$, consisting of non 
nested sets. Thus the family $s_*({\mathscr{A}}_0)=\{{\mathfrak a}_*:{\mathfrak a}\in s_0({\mathscr{A}}_0)\}$ contains a 
representative for each equivalence class of $s_0({\mathscr{A}}_0)$ and is in fact a lattice,
with ${\mathfrak a}_*\vee{\mathfrak b}_*=({\mathfrak a}{\overset\circ\cup}{\mathfrak b})_*$ and ${\mathfrak a}_*\wedge{\mathfrak b}_*=({\mathfrak a}{\overset\circ\cap}{\mathfrak b})_*$. 
By \eqref{decomposition}, $T$ is in fact a modular function in restriction to 
$s_*({\mathscr{A}}_0)$, i.e. a function such that
\begin{equation}
\label{T modular}
T({\mathfrak a}_*)+T({\mathfrak b}_*)=T({\mathfrak a}_*\vee{\mathfrak b}_*)+T({\mathfrak a}_*\wedge{\mathfrak b}_*)
\qquad {\mathfrak a},{\mathfrak b}\in s_0({\mathscr{A}}_0)
\end{equation}
We show that some of the properties of $\Phi_0$ are in fact easy consequences of 
properties which hold for $T$.

On some special semilattices the semimodular property simplifies considerably.

\begin{corollary}
\label{cor groemer}
Let ${\mathscr{A}}_0$ be such that every ${\mathfrak a}\in s_0({\mathscr{A}}_0)$ with ${\vert {\mathfrak a}\vert}>1$ and 
$\bigcup_{A\in{\mathfrak a}}A\in{\mathscr{A}}_0$ splits as ${\mathfrak a}={\mathfrak a}_1{\overset\circ\cup}{\mathfrak a}_2$ with ${\varnothing}\ne{\mathfrak a}_1,{\mathfrak a}_2$ 
and $\bigcup_{A\in{\mathfrak a}_1}A,\bigcup_{A\in{\mathfrak a}_2}A\in{\mathscr{A}}_0$. Then $\Phi_0$ is semimodular 
if and only if
\begin{equation}
\label{2 additive}
\Phi(A_1\cup A_2)=\Phi_0(A_1)+\Phi(A_2)-\Phi_0(A_1\cap A_2)
\qquad
A_1,A_2,A_1\cup A_2\in{\mathscr{A}}_0
\end{equation}
\end{corollary}

\begin{proof}
Let ${\mathfrak a}\in s_0({\mathscr{A}}_0)$ have minimal cardinality among those collections 
for which \eqref{modular cap} fails i.e. for which $\bigcup_{A\in{\mathfrak a}}A\in{\mathscr{A}}_0$
but $\Phi_0(\bigcup_{A\in{\mathfrak a}}A)\ne T({\mathfrak a})$. Then, necessarily, ${\vert {\mathfrak a}\vert}>1$.
Let ${\mathfrak a}_1,{\mathfrak a}_2$ be as in the claim. By \eqref{decomposition},
\begin{align*}
T({\mathfrak a})=T({\mathfrak a}_1)+T({\mathfrak a}_2)-T({\mathfrak a}_1{\overset\circ\cap}{\mathfrak a}_2)
\end{align*}
Moreover, ${\vert {{\mathfrak a}_1}\vert},{\vert {{\mathfrak a}_2}\vert}<{\vert {\mathfrak a}\vert}$ imply
$T({\mathfrak a}_1)=\Phi_0(\bigcup_{A\in{\mathfrak a}_1}A)$, $T({\mathfrak a}_2)=\Phi_0(\bigcup_{A\in{\mathfrak a}_2}A)$ and,
from \eqref{long},
\begin{align*}
T({\mathfrak a}_1{\overset\circ\cap}{\mathfrak a}_2)
&=
\sum_{\substack{{\varnothing}\ne\alpha\subset{\mathfrak a}_1\\{\varnothing}\ne\beta\subset{\mathfrak a}_2}}
\Phi_0\left(\bigcap_{A\in\alpha}A\cap\bigcap_{B\in\beta}B\right)
=
\sum_{{\varnothing}\ne\alpha\subset{\mathfrak a}_1}
\Phi_0\left(\bigcap_{A\in\alpha}A\cap\bigcup_{B\in{\mathfrak a}_2}B\right)
=
\Phi_0\left(\bigcup_{A\in{\mathfrak a}_1}A\cap\bigcup_{B\in{\mathfrak a}_2}B\right)
\end{align*}
Thus 
$T({\mathfrak a})=
\Phi_0\left(\bigcup_{A\in{\mathfrak a}_1}A\right)+\Phi_0\left(\bigcup_{A\in{\mathfrak a}_2}B\right)
-\Phi_0\left(\bigcup_{A\in{\mathfrak a}_1}A\cap\bigcup_{A\in{\mathfrak a}_2}B\right)
$ 
and, under \eqref{2 additive}, $T({\mathfrak a})=\Phi_0\left(\bigcup_{A\in{\mathfrak a}}A\right)$. By 
negative induction, $\Phi_0$ is semimodular. The converse is obvious.
\end{proof}

\begin{example}
\label{ex groemer}
Let $\mathscr H\subset2^\Omega$ be given and assume that $\lambda:\mathscr H\to{\mathbb{R}}_+$
is a given, monotonic set function on $\mathscr H$. Define ${\mathscr{A}}_0$ to be the collection
of all subsets of the sets in $\mathscr H$ and define
$\Phi_0(A)=\inf\{\lambda(H):H\in\mathscr H,\ A\subset H\}$. It is clear that ${\mathscr{A}}_0$ is a 
$\cap$-semilattice; moreover $N>1$ and $\bigcup_{1\le n\le N}A_n\in{\mathscr{A}}_0$ imply
$\bigcup_{1\le n\le N-1}A_n\in{\mathscr{A}}_0$ so that the conditions of Corollary \ref{cor groemer}
are met. Then \eqref{2 additive} provides a criterion for semimodularity.
\end{example}

Another property of special interest was introduced by Choquet \cite{choquet} and 
requires $Y$ to be an ordered vector space. 

\begin{definition}
\label{def monotone}
Let $Y$ be partially ordered. Then $\Phi_0:{\mathscr{A}}_0\to Y$  is completely monotone if
\begin{equation}
\label{monotone}
\Phi_0(A)
\ge
\sum_{{\varnothing}\ne\alpha\subset{\mathfrak a}}\nu(\alpha)\Phi_0\left(A\cap\bigcap_{A\in\alpha}A\right)
\quad\text{for all}\quad
{\mathfrak a}\in s_0({\mathscr{A}}_0),\ A\in{\mathscr{A}}_0
\end{equation}
\end{definition}

Choquet lists a number of reasons for which this further property is important in 
the study of capacities. In particular, it is clear that complete monotonicity implies 
that $\Phi_0$ is monotone, but the converse need not be true. Moreover, given that 
$s_0({\mathscr{A}}_0)$ contains the empty sequence, a completely monotone set function is
necessarily positive. From our standpoint, the following is an interesting characterization:

\begin{lemma}
\label{lemma choquet}
$\Phi_0$ is completely monotone if and only if $T$ is positive and monotone, i.e.
\begin{equation}
T({\mathfrak a})\ge T({\mathfrak b})\ge{\emptyset}
\qquad{\mathfrak a},{\mathfrak b}\in s_0({\mathscr{A}}_0),\ {\mathfrak a}\ge{\mathfrak b}
\end{equation}
\end{lemma}

\begin{proof}
By \eqref{decomposition},
\begin{align*}
T(A_1,\ldots,A_N)
&=
T(A_1,\ldots,A_{N-1})+\Phi_0(A_N)-T(A_N\cap A_1,\ldots,A_N\cap A_{N-1})
\end{align*}
so that, if $\Phi_0$ is completely monotone, then $T({\mathfrak a})\ge T({\mathfrak b})$ 
whenever ${\mathfrak a}$ is obtained from ${\mathfrak b}$ by including additional terms. ${\mathfrak a}\ge{\mathfrak b}$   
and \eqref{parsimony} imply that for any $B\in{\mathfrak b}$, 
${\emptyset} \le T(\{B\})\le T({\mathfrak b})\le T({\mathfrak a}{\overset\circ\cup}{\mathfrak b})=T({\mathfrak a})$. 
Conversely, if $T$ is monotone, then \eqref{monotone} follows from 
$\{A\cap A_1,\ldots,A\cap A_N\}\le\{A\}$.
\end{proof}

We are interested in conditions that allow to associate to any $\Phi_0:{\mathscr{A}}_0\to Y$ 
a completely monotone set function. This will be achieved by establishing a version
of the classical Jordan decomposition which is however not obvious under the current
general assumptions on ${\mathscr{A}}_0$ and $\Phi_0$. To this end we proceed under the 
assumption that $Y={\mathbb{R}}$ although assuming that $Y$ is a complete lattice would be 
enough. 

Define the mapping $\chi_{\mathfrak a}:{\mathscr{A}}_0\to{\mathbb{R}}$ by letting $\chi_{\mathfrak a}(A)=1$ if $\{A\}\le{\mathfrak a}$ 
or $0$ otherwise. The next step is to extend the functional $T$ to the rational vector 
space generated such functions i.e. the space
\begin{equation}
\label{L(A)}
{\mathscr L({\mathscr{A}}_0)}=\left\{\sum_{n=1}^Nt_n\chi_{{\mathfrak a}_n}:
t_1,\ldots,t_n\in{\mathbb{Q}},\ {\mathfrak a}_1,\ldots,{\mathfrak a}_n\in s_0({\mathscr{A}}_0),\ N\in{\mathbb{N}}\right\}
\end{equation}
Each element in ${\mathscr L({\mathscr{A}}_0)}$ is then a real valued function on ${\mathscr{A}}_0$. Suppose
that $\sum_{n=1}^N\chi_{{\mathfrak a}_n}(A)\ge k$. This is equivalent to saying that there exists
$b\subset[N]$ with ${\vert b\vert}=k$ such that $A\in\alpha$ or,
in yet other terms, that 
\begin{equation*}
A\in\underset{\{b\subset[N]:{\vert b\vert}=k\}}{\overset\circ\bigcup}\ {\overset\circ\bigcap}_{n\in b}{\mathfrak a}_n={\mathfrak a}_N[k]
\end{equation*}
and so $\sum_{n=1}^N\chi_{{\mathfrak a}_n}=\sum_{k=1}^N{\mathbf{1}_{{{\mathfrak a}[k]}}}$. But then
\eqref{modular T} implies
\begin{equation*}
\sum_{n=1}^NT({\mathfrak a}_n)
=
\sum_{k=1}^NT\left({\mathfrak a}[k]\right)
\end{equation*}
Suppose that  $\sum_{n=1}^N\chi_{{\mathfrak a}_n}=\sum_{i=1}^I\chi_{{\mathfrak b}_i}$ with, say 
$I\ge N$. For $I\ge n>N$ put ${\mathfrak a}_n={\varnothing}$. Then necessarily ${\mathfrak a}[i]={\mathfrak b}[i]$ for 
$i=1,\ldots,I$
so that
\begin{equation}
\label{partial}
\sum_{n=1}^NT({\mathfrak a}_n)
=
\sum_{k=1}^IT\left({\mathfrak a}[k]\right)
=
\sum_{k=1}^IT\left({\mathfrak b}[k]\right)
=
\sum_{i=1}^IT({\mathfrak b}_i)
\end{equation}

\begin{theorem}
\label{th T extension}
There exists a unique extension of $T$ to a linear functional $\hat T$ on ${\mathscr L({\mathscr{A}}_0)}$ and
this is defined by letting
\begin{equation}
\label{T ext}
\hat T\left(\sum_{n=1}^Nt_n\chi_{{\mathfrak a}_n}\right)
=
\sum_{n=1}^Nt_nT({\mathfrak a}_n)
\qquad
\sum_{n=1}^Nt_n\chi_{{\mathfrak a}_n}\in{\mathscr L({\mathscr{A}}_0)}
\end{equation}
Moreover, $\hat T$ is positive if and only if $\Phi_0$ is completely monotone.
\end{theorem}

\begin{proof}
Throughout the proof let $f=\sum_{n=1}^Nt_n\chi_{{\mathfrak a}_n}\in{\mathscr L({\mathscr{A}}_0)}$. First, assume
that the coefficients of $f$ are positive and let $t_n=\frac{p_n}{q_n}$ with $p_n,q_n\in{\mathbb{N}}$.
Let also $\sum_{k=1}^Ks_k\chi_{{\mathfrak b}_n}$ be another representation of $f$ with
$s_k=\frac{u_k}{w_k}$ and $u_k,w_k\in{\mathbb{N}}$. Then, letting $q=q_1q_2\ldots q_N$ and 
$w=w_1w_2\ldots w_K$
\begin{equation}
\label{>0}
\sum_{n=1}^Nt_nT({\mathfrak a}_n)
=
\frac{1}{qw}\sum_{n=1}^N\sum_{j=1}^{t_nqw}T({\mathfrak a}_n)
=
\frac{1}{qw}\sum_{k=1}^K\sum_{j=1}^{s_kqw}T({\mathfrak b}_k)
=
\sum_{k=1}^Ks_kT({\mathfrak b}_k)
\end{equation}
because of \eqref{partial} and of the fact that
\begin{equation*}
\sum_{n=1}^N\sum_{j=1}^{t_nqw}\chi_{{\mathfrak a}_n}
=
\sum_{k=1}^K\sum_{j=1}^{s_kqw}\chi_{{\mathfrak b}_k}
\end{equation*}
Second, if the coefficients of $f$ are arbitrary and  $\sum_{k=1}^Ks_k\chi_{{\mathfrak b}_k}$ is 
another representation of $f$, then
\begin{equation*}
\sum_{n=1}^Nt^+_n\chi_{{\mathfrak a}_n}+\sum_{k=1}^Ks^-_k\chi_{{\mathfrak b}_k}
=
\sum_{n=1}^Nt^-_n\chi_{{\mathfrak a}_n}+\sum_{k=1}^Ks^+_k\chi_{{\mathfrak b}_k}
\end{equation*}
so that, by \eqref{>0}, indeed \eqref{T ext} defines a functional $\hat T$ on ${\mathscr L({\mathscr{A}}_0)}$ 
which is also clearly linear. In general, if $\{f_1<\ldots<f_J\}$ is the range of $f$,
$f_0=0$ and
\begin{equation*}
{\mathfrak a}_f[j]
=
\underset{\{b\subset[N]:\sum_{n\in b}t_n\ge f_j\}}{\overset\circ\bigcup}\underset{n\in b}{\overset\circ\bigcap}{\mathfrak a}_n
\end{equation*}
we obtain the representation 
\begin{equation}
\label{representation}
f=\sum_{j=1}^J(f_j-f_{j-1})\chi_{{\mathfrak a}_f[j]}
\quad\text{and}\quad
\hat T(f)=\sum_{j=1}^J(f_j-f_{j-1})T({\mathfrak a}_f[j])
\end{equation}
If $f\ge0$ and $T$ is monotone then $\hat T(f)\ge0$ by \eqref{representation}. If
viceversa $\hat T$ is positive and ${\mathfrak b}\le{\mathfrak a}$, then $\chi_{\mathfrak a}-\chi_{\mathfrak b}\in{\mathscr L({\mathscr{A}}_0)}_+$ and 
$0\le\hat T(\chi_{\mathfrak a}-\chi_{\mathfrak b})=T({\mathfrak a})-T({\mathfrak b})$ so that $T$ is monotone and $\Phi_0$
completely monotone. 
\end{proof}

\begin{definition}
$\Phi_0$ is said to be of locally bounded variation if for each $g\in{\mathscr L({\mathscr{A}}_0)}$ there exists $y(g)\in Y$ 
such that
\begin{equation}
\label{abs}
{\left\vert {\hat T(f)}\right\vert}\le y(g)
\qquad f\in{\mathscr L({\mathscr{A}}_0)},\ {\vert f\vert}\le{\vert g\vert}
\qquad g\in{\mathscr L({\mathscr{A}}_0)}
\end{equation}
$\Phi_0$ is said to be of bounded variation if \eqref{abs} holds with some $y\in Y$
independent of $g\in{\mathscr L({\mathscr{A}}_0)}$.
\end{definition}

We eventually obtain the classical decomposition of Jordan.

\begin{theorem}
\label{th BV}
$\Phi_0:{\mathscr{A}}\to{\mathbb{R}}$ is of (locally) bounded variation if and only if it decomposes as
\begin{equation}
\label{jordan}
\Phi_0=\Phi_0^+-\Phi_0^-
\end{equation}
where (i) $\Phi_0^+,\Phi_0^-$ are completely monotone and of (locally) bounded variation 
and (ii) $v(\Phi_0)=\Phi_0^++\Phi_0^-\le\Psi_0^++\Psi_0^-$ for any pair of completely
monotone set functions $\Psi_0^++\Psi_0^-:{\mathscr{A}}_0\to Y$ of (locally) finite variation that 
meets \eqref{jordan}. 
\end{theorem}

\begin{proof}
Given \eqref{abs}, we conclude from \cite[Theorem 1.10]{aliprantis} that $\hat T$
admits a lattice decomposition $\hat T=\hat T^+-\hat T^-$ with $\hat T^+,\hat T^-$ 
positive linear functionals on ${\mathscr L({\mathscr{A}}_0)}$. Let $\Phi_0^+,\Phi_0^-$ be the set functions on 
${\mathscr{A}}_0$ defined by letting 
\begin{equation*}
\Phi_0^+(A)=\hat T^+(\{A\}))
\quad\text{and}\quad
\Phi_0^-(A)=\hat T^-(\{A\}))
\qquad A\in{\mathscr{A}}_0
\end{equation*}
It follows from \eqref{T} and \eqref{T ext} that $\hat T_{\Phi_0^+}=\hat T^+$
so that $\Phi_0^+$ is completely monotone, and so is $\Phi_0^-$. The decomposition
\eqref{jordan} is then just a consequence of the lattice decomposition for $\hat T$.
Let $\Psi_0^+-\Psi_0^-$ be another decomposition of $\Phi_0$ into the difference of two
completely monotone set functions of finite variation. Then,
\begin{align*}
\hat T_{\Psi_0^+}
=
(\hat T+\hat T_{\Psi_0^-})\vee0
\ge\hat T\vee0
=\hat T^+
=\hat T_{\Phi_0^+}
\end{align*}
and the claim follows.
\end{proof}

\section{Additive Extensions}
\label{sec pettis}

The main result of this section is a necessary and sufficient condition for extending 
semimodular, vector-valued set functions from a semilattice of sets to the generated 
ring. 

Given their repeated use, we state hereafter some well known facts. 

\begin{lemma} 
\label{lemma semilattice} 
${\mathscr{A}}_1$ consists of finite intersections (resp. unions) of sets from ${\mathscr{A}}_0$ 
while each $A\in{\mathscr{A}}$ is the disjoint union $\bigcup_{n=1}^NB_n\backslash C_n$ where 
$B_n,C_n\in{\mathscr{A}}_1$ and $C_n\subset B_n$ for $n=1,\ldots,N$. Moreover, 
${\mathscr{S}}({\mathscr{A}}_0)={\mathscr{S}}({\mathscr{A}})$.
\end{lemma}

\begin{proof} 
The representation of the members of ${\mathscr{A}}_1$ is an easy fact; for ${\mathscr{A}}$ see 
\cite[p. 26]{halmos} or \cite[theorem 1.1.9, p. 4]{rao}. From \eqref{lattice representation} 
one easily establishes ${\mathscr{S}}({\mathscr{A}}_1)={\mathscr{S}}({\mathscr{A}}_0)$. Given the representation 
$\bigcup_{n=1}^NB_n\cap C_n^c$ of each $A\in{\mathscr{A}}$ as in the claim, we obtain 
${\mathbf{1}_{{A}}}=\sum_{n=1}^N({\mathbf{1}_{{B_n}}}-{\mathbf{1}_{{C_n}}})$ so that ${\mathscr{S}}({\mathscr{A}})={\mathscr{S}}({\mathscr{A}}_1)$. 
\end{proof}

The following is the main result of this section. 

\begin{theorem} 
\label{th extension} 
$\Phi_0$ is semimodular if and only if there exists a necessarily unique, modular set 
function $\Phi_1:{\mathscr{A}}_1\to Y$ such that ${\left.{\Phi_1}\right\vert {{\mathscr{A}}_0}}=\Phi_0$. 
\end{theorem}

\begin{proof} 
By Lemma \ref{lemma conjugate} we just consider the case of a $\cap$-semilattice.
It is obvious that if $\Phi_1:{\mathscr{A}}_1\to Y$ is modular and ${{\Phi_1}\vert {{\mathscr{A}}_0}}=\Phi_0$
then $\Phi_0$ is semimodular. Conversely, assume that $\Phi_0$ is semimodular. We 
start showing that
\begin{equation}
\label{semimodular parsimony}
T({\mathfrak a}{\overset\circ\cup}{\mathfrak b})=T({\mathfrak a})
\qquad 
{\mathfrak a},{\mathfrak b}\in s_0({\mathscr{A}}_0),\ \bigcup_{B\in{\mathfrak b}}B\subset\bigcup_{A\in{\mathfrak a}}A
\end{equation}
Indeed, for any $\beta\subset{\mathfrak b}$ we have $\bigcup_{A\in{\mathfrak a}}(A\cap\bigcap_{B\in\beta}B)=\bigcap_{B\in\beta}B$ so that 
\begin{align*}
T({\mathfrak a}{\overset\circ\cap}{\mathfrak b})
&=
\sum_{\substack{{\varnothing}<\alpha\le{\mathfrak a}\\ {\varnothing}<\beta\le{\mathfrak b}}}\nu(\alpha)\nu(\beta)
\Phi_0\left(\bigcap_{A\in\alpha}A\cap\bigcap_{B\in\beta}B\right)
&(\text{by }\eqref{long})\\
&=
\sum_{{\varnothing}<\beta\le{\mathfrak b}}\nu(\beta)\Phi_0\left(\bigcap_{B\in\beta}B\right)
&(\text{by }\eqref{modular cap})\\
&=
T({\mathfrak b})
\end{align*}
and \eqref{semimodular parsimony} follows from \eqref{decomposition}.
Thus, writing
\begin{equation} 
\label{k} 
\Phi_1\left(\bigcup_{A\in{\mathfrak a}}A\right)=T({\mathfrak a})
\qquad {\mathfrak a}\in s_0({\mathscr{A}}_0) 
\end{equation} 
implicitly defines an extension of $\Phi_0$ to ${\mathscr{A}}_1$. Moreover, this is 
modular since
\begin{equation*}
\bigcup_{A\in{\mathfrak a}}A\cap\bigcap_{B\in{\mathfrak b}}B
=
\bigcup_{A\in{\mathfrak a},B\in{\mathfrak b}}A\cap B
=
\bigcup_{C\in{\mathfrak a}{\overset\circ\cap}{\mathfrak b}}C
\end{equation*}
so that
\begin{align*}
\Phi_1\left(\bigcup_{A\in{\mathfrak a}}A\cap\bigcap_{B\in{\mathfrak b}}B\right)
&=
T({\mathfrak a}{\overset\circ\cap}{\mathfrak b})\\
&=
T({\mathfrak a})+T({\mathfrak b})-T({\mathfrak a}{\overset\circ\cup}{\mathfrak b})
&(\text{by }\eqref{decomposition})
\\
&=
\Phi_0\left(\bigcup_{A\in{\mathfrak a}}A\right)
+\Phi_0\left(\bigcup_{B\in{\mathfrak b}}B\right)
-\Phi_0\left(\bigcup_{A\in{\mathfrak a}}A\cup\bigcup_{B\in{\mathfrak b}}B\right)
\end{align*}
Given that ${\mathscr{A}}_1$ is a lattice, this is equivalent to the statement that $\Phi_1$ is 
semimodular. Eventually, if $\Psi$ is another semimodular extension of $\Phi_0$ to 
${\mathscr{A}}_1$ then necessarilty
\begin{equation*}
\Psi\left(\bigcup_{A\in{\mathfrak a}}A\right)
=
\sum_{{\varnothing}<\alpha\le{\mathfrak a}}\nu(\alpha)\Phi_0\left(\bigcap_{A\in\alpha}A\right)
=
\Phi_1\left(\bigcup_{A\in{\mathfrak a}}A\right)
\end{equation*}
\end{proof}

The following is a minor generalization of a classical result of Pettis 
\cite[Theorem 1.2, p. 188]{pettis}.

\begin{theorem}[Pettis]
\label{th pettis}
$\Phi_0$ is semimodular if and only if there exists a necessarily unique $\Phi\in sa({\mathscr{A}},Y)$ 
satisfying
\begin{equation}
\label{pettis}
\Phi(A\backslash B)=\Phi_0(A)-\Phi_0(B)\qquad A,B\in{\mathscr{A}}_0,\ B\subset A
\end{equation}
Moreover, (i) $\Phi$ extends $\Phi_0$ if and only if $\Phi_0\in sa_0({\mathscr{A}}_0,Y)$ and (ii)
if $\bar{\mathscr{A}}$ is any algebra containing ${\mathscr{A}}_0$ then $\Phi_0\in sa({\mathscr{A}}_0,Y)$ has an additive 
extension to $\bar{\mathscr{A}}$ and this will be unique up to the choice of $\bar\Phi(X)$ if $\bar{\mathscr{A}}$ 
is the algebra generated by ${\mathscr{A}}_0$.
\end{theorem}
\begin{proof}
If $\Phi_0$ is semimodular let $\Phi_1$ be its modular extension to the generated lattice 
${\mathscr{A}}_1$. Let ${\mathscr{A}}_2=\{A\backslash B:A,B\in{\mathscr{A}}_1,\ B\subset A\}$, a $\cap$-lattice. If 
$A\backslash B,\ C\backslash D\in{\mathscr{A}}_2$ and $A\backslash B=C\backslash D$ then, 
since $\Phi_1$ is modular, $\Phi_1(A\cup D)=\Phi_1(A)+\Phi_1(D)-\Phi_1(A\cap D)$ 
so that the equalities $A\cup D=C\cup B$ and $A\cap D=C\cap B$ imply 
$\Phi_1(A)-\Phi_1(B)=\Phi_1(C)-\Phi_1(D)$. We are then free to define 
$\Phi_2:{\mathscr{A}}_2\to Y$ by letting 
\begin{equation}
\Phi_2(A\backslash B)=\Phi_1(A)-\Phi_1(B)\qquad A\backslash B\in{\mathscr{A}}_1
\end{equation}
Clearly ${\varnothing}\in{\mathscr{A}}_2$ and $\Phi_2({\varnothing})={\emptyset}$. By Lemma \ref{lemma semilattice}, ${\mathscr{A}}$ 
is the lattice generated by ${\mathscr{A}}_2$. Suppose $A_n\backslash B_n,A\backslash B\in{\mathscr{A}}_2$ for 
$n=1,\ldots,N$ and $\bigcup_{n=1}^NA_n\backslash B_n=A\backslash B$. Then, 
\begin{equation*}
A\backslash B=\bigcup_{n=1}^N\bar A_n\left\backslash\bigcap_{n=1}^N\bar B_n\right.
\quad\text{where}\quad
\bar A_n=B\cup(A\cap A_n),\ \bar B_n=B\cup(A\cap B_n)
\end{equation*}
Thus, $\bar A_n\backslash\bar B_n=A_n\backslash B_n$ and therefore
\begin{align*}
\Phi_2(A\backslash B)
&=
\Phi_1\left(\bigcup_{n=1}^N\bar A_n\right)-\Phi_1\left(\bigcup_{n=1}^N\bar B_n\right)\\
&=
\sum_{{\varnothing}\ne b\subset[N]}\nu(b)\left[\Phi_1\left(\bigcap_{n\in b}\bar A_n\right)
-\Phi_1\left(\bigcup_{n\in b}\bar B_n\right)\right]\\
&=
\sum_{{\varnothing}\ne b\subset[N]}\nu(b)\Phi_2
\left(\bigcap_{n\in b}(\bar A_n\backslash\bar B_n)\right)\\
&=
\sum_{{\varnothing}\ne b\subset[N]}\nu(b)\Phi_2\left(\bigcap_{n\in b}(A_n\backslash B_n)\right)
\end{align*}
Thus, $\Phi_2\in sa_0({\mathscr{A}}_2,Y)$ and admits, by Theorem \ref{th extension}, an extension 
$\Phi\in sa({\mathscr{A}},Y)$  satisfying \eqref{pettis}. Suppose that the same is true of $\Psi$ and fix 
$A,B\in{\mathscr{A}}_0$. If ${\mathscr{A}}_0$ is a $\cap$-lattice, write
$
\Psi(A)-\Psi(A\cap B)=\Phi(A)-\Phi(A\cap B)
$
(otherwise $\Psi(A\cup B)-\Psi(B)=\Phi(A\cup B)-\Phi(B)$). Given that both $\Psi$ 
and $\Phi$ are modular on ${\mathscr{A}}_1$, this equality is easily extended first to all $B\in{\mathscr{A}}_1$ 
for fixed $A\in{\mathscr{A}}_0$ and then to all $A,B\in{\mathscr{A}}_1$. But then 
${{\Psi}\vert {{\mathscr{A}}_2}}={{\Phi}\vert {{\mathscr{A}}_2}}$. Uniqueness then follows from Theorem 
\ref{th extension}. Claims (\textit{i}) and (\textit{ii}) are obvious.
\end{proof}

An extension of Theorem \ref{th extension} may be obtained, at least for the case
in which $\Phi_0$ is real valued and monotonic, by exploiting Corollary \ref{cor groemer}
and the comments following it. Starting with a monotonic set function on any collection
$\mathscr H$ of subsets we may get an extension to the $\cap$-semilattice of all
the subsets of sets in $\mathscr H$ which will be semimodular, and thus admit
a modular extension to the generated lattice, if and only if \eqref{2 additive} is
met.

Another implication of Theorems \ref{th extension} and \ref{th pettis} is that if 
$\Phi_0$ is a semimodular set function then there always exist a semiadditive 
set function obtained from it by translation (see \eqref{conjugate}). 

\begin{corollary}
\label{corollary semiadditive}
If $\Phi_0$ is semimodular then admits a unique semiadditive translation.
\end{corollary}

\begin{proof}
Let $\Phi_1$ be the modular extension of $\Phi_0$ to ${\mathscr{A}}_1$ and write 
$y=-\Phi_1({\varnothing})$, if ${\varnothing}\in{\mathscr{A}}_1$, or $y={\emptyset}$ otherwise and denote, as in 
\eqref{conjugate}, translations by superscripts. By construction,
$\Phi_ 1^y\in sa({\mathscr{A}}_1,Y)$ while $\Phi_0^y\in sa_0({\mathscr{A}}_0,Y)$ because of 
$\Phi_0^y={{\Phi_1^y}\vert {{\mathscr{A}}_0}}$. If $\Phi_0^z$ were another semiadditive 
translation of $\Phi_0$ then, by Theorem \ref{th pettis}, $\Phi_0^y$ and 
$\Phi_0^z$ would have the same additive extension to ${\mathscr{A}}$. Thus, if $A\in{\mathscr{A}}_0$ 
one has $0=\Phi_0^y(A)-\Phi_0^z(A)=y-z$.
\end{proof}

Theorem \ref{th pettis} also implies a finitely additive version of Dynkin's lemma:

\begin{corollary} 
\label{corollary Dynkin} 
Two finitely additive probabilities agree on a semilattice if and only if they agree on the 
generated algebra.
\end{corollary}

The existence of an additive extension to the generated ring, established in Theorem 
\ref{th pettis} for semiadditive set functions on a semilattice, allows to associate this 
class of set functions with the family of linear functionals on the vector space of simple 
functions. It is easy to see that this relationship is isomorphic and in principle may be 
exploited to obtain even more general measure extensions. This approach was inaugurated 
by Horn and Tarski \cite[Definition 1.6, p. 471]{horn tarski} who first introduced  the 
notion of \textit{partial measure}, later generalized to that of \textit{real partial charge} 
in \cite[Definition 3.2.1, p. 64]{rao}. A real-valued set function $\lambda$ on some, 
arbitrary collection $\mathscr B\subset 2^X$ is a real partial charge if 
$\sum_{n=1}^N{\mathbf{1}_{{C_n}}}=\sum_{k=1}^K{\mathbf{1}_{{D_k}}}$ with $C_n,D_k\in\mathscr B$ implies 
$\sum_{n=1}^N\lambda(C_n)=\sum_{k=1}^K\lambda(D_k)$ and, if so, it may be 
extended to a real-valued, additive set function on any algebra containing $\mathscr B$, 
\cite[Theorem 3.2.5, p. 65]{rao}. In other words real partial charges may be extended 
with no restriction on the initial collection of sets, while we had to assume that ${\mathscr{A}}_0$ is 
a semilattice. However, this condition belongs more to analysis than to measure theory 
and in several cases it is rather difficult to apply. 

\section{Product Spaces}
\label{sec fubini}

Of some interest is also a version of Theorem \ref{th extension} for product 
spaces. If $\Phi:{{\mathscr{A}}_{{}}\times\mathscr B_{{}}}\to Y$, $A\in{\mathscr{A}}$ and $B\in\mathscr B$ then we define 
$\Phi^A:\mathscr B\to Y$ and $\Phi^B:{\mathscr{A}}\to Y$ by letting
\begin{equation*}
\Phi^A(B)=\Phi^B(A)=\Phi(A\times B)\qquad A\in{\mathscr{A}},\ B\in\mathscr B
\end{equation*}
$\Phi$ is said to be separately in a given class if $\Phi^B$ and $\Phi^A$ belong 
to that class for each $A\in{\mathscr{A}}$ and $B\in\mathscr B$.

\begin{lemma}
\label{lemma fubini}
Let ${\mathscr{A}}_0$ and $\mathscr B_0$ be semilattices of sets generating the lattices 
${\mathscr{A}}_1$ and $\mathscr B_1$ and the rings ${\mathscr{A}}$ and $\mathscr B$, respectively. 
Let also $\Phi_0:{{\mathscr{A}}_{{0}}\times\mathscr B_{{0}}}\to Y$. Then:
\begin{enumerate}[(i)]
\item $\Phi_0$ is separately semimodular if and only if there exists a unique separately 
modular set function $\Phi_1:{{\mathscr{A}}_{{1}}\times\mathscr B_{{1}}}\to Y$ such that ${{\Phi_1}\vert {{{\mathscr{A}}_{{0}}\times\mathscr B_{{0}}}}}=\Phi_0$;
\item $\Phi_0$ is separately semiadditive if and only if there exists a unique separately 
additive set function $\Phi:{{\mathscr{A}}_{{}}\times\mathscr B_{{}}}\to Y$ such that ${{\Phi}\vert {{{\mathscr{A}}_{{0}}\times\mathscr B_{{0}}}}}=\Phi_0$.
\end{enumerate}
\end{lemma}

\begin{proof}
If $\Phi_0$ is separately semiadditive, then we may define 
$\bar\Phi_0:({\mathscr{A}}_0\cup\{{\varnothing}\})\times(\mathscr B_0\cup\{{\varnothing}\})\to Y$
by letting $\bar\Phi(A\times B)=\Phi_0(A\times B)$ if $A,B\ne{\varnothing}$
or $\bar\Phi_0({\varnothing})={\emptyset}$ otherwise. It is the clear that $\bar\Phi_0$ is again
semimodular. To avoid notational complications we will identify $\Phi_0$
and $\bar\Phi_0$ and assume ${\varnothing}\in{\mathscr{A}}_0,\mathscr B_0$. Thus the extension
of $\Phi_0$ to ${\mathscr{A}}_1\times\mathscr B_1$ as a semimodular (resp. semiadditive)
set function will be exactly the same.

We prove the claim just for the case in which ${\mathscr{A}}_0$ is a $\cup$-semilattice and 
$\mathscr B_0$ a $\cap$-semilattice but it will be clear that the argument remains 
unchanged for any other choice. For each $B\in\mathscr B_0$ extend $\Phi^B$ to 
a semimodular set function on ${\mathscr{A}}_1$ and define the set function 
$\Phi_*:{\mathscr{A}}_1\times\mathscr B_0\to Y$ obtained from this by letting $B$ vary across 
$\mathscr B_0$. We claim that $\Phi_*$ is separately semimodular. In fact, let 
$B,B_1,\ldots,B_N\in\mathscr B_0$ be such that $B=\bigcup_{n=1}^NB_n$ and 
fix $A\in{\mathscr{A}}_1$. By Lemma \ref{lemma semilattice} there exist 
$A_1,\ldots,A_M\in{\mathscr{A}}_0$ such that $A=\bigcap_{m=1}^MA_n$. Then,
\begin{align*}
\Phi_*(A\times B)
&=
\sum_{{\varnothing}\ne b\subset [M]}\nu(b)\Phi_0\left(\bigcup_{m\in b}A_m\times B\right)\\
&=
\sum_{{\varnothing}\ne a\subset [N]}\nu(a)\sum_{{\varnothing}\ne b\subset [M]}\nu(b)
\Phi_0\left(\bigcup_{m\in b}A_m\times\bigcap_{n\in a} B_n\right)\\
&=
\sum_{{\varnothing}\ne a\subset [N]}\nu(a)\Phi_*\left(A\times\bigcap_{n\in a} B_n\right)
\end{align*}
so that $\Phi_*$ is semimodular in its second coordinate while modular in the first one, 
by construction; moreover it is unique with these properties. A further application of this 
same argument proves the existence of the extension $\Phi_1$ of $\Phi_0$ to
${\mathscr{A}}_1\times\mathscr B_1$ with the desired properties. If $\Psi_1:{{\mathscr{A}}_{{1}}\times\mathscr B_{{1}}}\to Y$ is 
another extension with these same properties, then necessarily 
${{\Psi_1}\vert {{\mathscr{A}}_1\times\mathscr B}}={{\Phi_1}\vert {{\mathscr{A}}_1\times\mathscr B}}$ and,
by semimodularity in the second coordinate, $\Psi_1=\Phi_1$. Fix now $B\in\mathscr B_1$, 
consider the set function on ${\mathscr{A}}$ associated with $\Phi_1^B$ via \eqref{pettis} and obtain 
from this a set function $\Phi_{**}:{\mathscr{A}}\times\mathscr B_1\to Y$. As above, the linear nature 
of the relationship linking $\Phi_{**}$ to $\Phi_1$ implies that 
$\Phi^A_{**}:\mathscr B_1\to Y$ is modular. A further application of this same procedure 
concludes the proof of existence in (\textit{ii}). As above, uniqueness follows from 
Theorem \ref{th pettis} applied coordinatewise.
\end{proof}

\begin{theorem} 
\label{th fubini} 
Let ${\mathscr{A}}_0$, $\mathscr B_0$ be semilattices and $\mathscr C$ the ring generated by 
${{\mathscr{A}}_{{0}}\times\mathscr B_{{0}}}$. $\Phi_0:{{\mathscr{A}}_{{0}}\times\mathscr B_{{0}}}\to Y$ is separately semiadditive if and only if there exists 
$\Phi\in sa(\mathscr C,Y)$ such that ${{\Phi}\vert {{{\mathscr{A}}_{{0}}\times\mathscr B_{{0}}}}}=\Phi_0$.
\end{theorem}

\begin{proof}
Define the set function $\Phi_*:({\mathscr{A}}_0\cup{\varnothing})\times(\mathscr B_0\cup{\varnothing})\to Y$ 
by letting $\Phi_*(A\times B)={\emptyset}$ if either $A={\varnothing}$ or $B={\varnothing}$ and 
$\Phi_*(A\times B)=\Phi_1(A\times B)$ if $A\times B\in{{\mathscr{A}}_{{0}}\times\mathscr B_{{0}}}$. Given that $\Phi_1$ 
is semiadditive on both coordinates, this is easily seen to be the unique separately 
semiadditive extension of $\Phi_1$ to $({\mathscr{A}}_0\cup{\varnothing})\times(\mathscr B_0\cup{\varnothing})$. 
To avoid additional notation we simply assume ${\varnothing}\in{\mathscr{A}}_0$ and ${\varnothing}\in\mathscr B_0$ 
and $\Phi_0({\varnothing})={\emptyset}$. Let ${\mathscr{A}}$ and $\mathscr B$ be the rings generated by ${\mathscr{A}}_0$ 
and $\mathscr B_0$ respectively. By Lemma \ref{lemma fubini} we obtain a unique 
extension $\Phi$ to ${{\mathscr{A}}_{{}}\times\mathscr B_{{}}}$ that is separately additive. Let ${\mathscr C}_0$ be the collection of 
all finite unions $\bigcup_{n=1}^NA_n\times B_n$ from ${{\mathscr{A}}_{{}}\times\mathscr B_{{}}}$ where the sets 
$A_1,\ldots,A_N$ are pairwise disjoint. Let $C,\bar C\in{\mathscr C}_0$ be given, with 
$C=\bigcup_{j=1}^JA_j\times B_j$ and 
$\bar C=\bigcup_{k=1}^K\bar A_k\times\bar B_k$. Writing
\begin{equation} 
\label{union} 
C\cup\bar C
=
\bigcup_{j,k}[(A_j\cap\bar A_k)\times(B_j\cup\bar B_k)]
\cup\bigcup_{j=1}^J\left[\left(A_j\backslash\bigcup_{k=1}^K\bar A_k\right)\times B_j\right]
\cup\bigcup_{k=1}^K\left[\left(\bar A_k\backslash\bigcup_{j=1}^JA_j\right)\times\bar 
B_k\right]
\end{equation} 
shows that ${\mathscr C}_0$ is a $\cup$-lattice. If $C=\bar C$, then $A_j\cap\bar A_k\neq{\varnothing}$ 
implies $B_j=\bar B_k$ so that, $ \sum_{j=1}^J\Phi(A_j\times B_j) 
=\sum_{j=1}^J\sum_{k=1}^K\Phi((A_j\cap\bar A_k)\times B_j)
=\sum_{k=1}^K\sum_{j=1}^J\Phi((A_j\cap\bar A_k)\times\bar B_k)
=\sum_{k=1}^K\Phi(\bar A_k\times\bar B_k) $. $\Phi_*:{\mathscr C}_0\to Y$ may thus be 
unambiguously defined by letting
\begin{equation} 
\Phi_*(C)=\sum_{j=1}^J\Phi(A_j\times B_j) \qquad 
C=\bigcup_{j=1}^JA_j\times B_j\in{\mathscr C}_0 
\end{equation} 

To show that $\Phi_*\in sa_0({\mathscr C}_0,Y)$, fix $C_1,\ldots, C_J,\bigcap_{j=1}^JC_j\in{\mathscr C}_0$ 
with $C_j=\bigcup_{k=1}^{K_j} A^j_k\times B^j_k$. Given that ${\mathscr{A}}$ is a ring one may 
form a collection $A_1,\ldots,A_N\in{\mathscr{A}}$ such that 
$A_k^j=\bigcup_{\{n:A_n\subset A_k^j\}}A_n$ for all $j,k$. Set conventionally 
$A_0^j=B_0^j={\varnothing}$. If $1\le n\le N$ and $1\le j\le J$ there is then at most one 
integer $k$ such that $A_n\subset A_k^j$. Define $k_n^j$ to be such integer, if it exists, 
or else put $k_n^j=0$. Then,
\begin{align*}
\sum_{{\varnothing}\ne b\subset[J]}\nu(b)\Phi_*\left(\bigcup_{j\in b}C_j\right)
&=\sum_{{\varnothing}\ne b\subset[J]}\nu(b)\Phi_*\left(\bigcup_{n=1}^N\bigcup_{j\in b}A_n
\times B^j_{k_n^j}\right)\\
&=\sum_{{\varnothing}\ne b\subset[J]}\nu(b)\Phi_*\left(\bigcup_{n=1}^NA_n\times
\left(\bigcup_{j\in b} B^j_{k_n^j}\right)\right)\\
&=\sum_{n=1}^N\sum_{{\varnothing}\ne b\subset[J]}\nu(b)\Phi\left(A_n\times
\left(\bigcup_{j\in b} B^j_{k_n^j}\right)\right)\\
&=\sum_{n=1}^N\Phi\left(A_n\times\left(\bigcap_{j=1}^J B^j_{k_n^j}\right)\right)\\
&=\Phi_*\left(\bigcup_{n=1}^NA_n\times\left(\bigcap_{j=1}^J B^j_{k_n^j}\right)\right)\\
&=\Phi_*\left(\bigcap_{j=1}^JC_j\right)
\end{align*} 
i.e. $\Phi_*\in sa_0({\mathscr C}_0,Y)$. The claim thus follows from Theorem \ref{th pettis}.
\end{proof}
One may remark that Theorem \ref{th fubini} is somehow more general than Theorem 
\ref{th extension} since the given initial collection, ${{\mathscr{A}}_{{0}}\times\mathscr B_{{0}}}$, is not even a semilattice 
nor is the set function $\Phi_0$ semimodular. Of course, Theorem \ref{th fubini} may 
be proved for any arbitrary but finite number of coordinates.

\section{Order Semilattices}
\label{sec order}

By an order semilattice we mean a partially ordered set closed with respect to either
lattice operation, $\wedge$ or $\vee$. In particular, in this section $X$ will be a
$\wedge$-semilattice endowed with the order $\ge$. In this section we will briefly 
consider a special case of a semilattice on which the semimodular property is particularly 
easy. This class was originally introduced by Norberg \cite{norberg} who proved the
first version of the following Corollary \ref{cor norberg}. His ideas have been
widely applied in the literature on set-indexed stochastic processes.

Let us introduce a class of order intervals and its corresponding family
\begin{align}
\label{order interval}
{\ ]-\infty,x]}&=\{y\in X:y\le x\}
&\text{and}&
&{\mathscr{A}}_0(X)&=\{{\ ]-\infty,x]}:x\in X\}
\end{align}
To avoid possible confusions, let us mention, anticipating on the next section, that
if $X$ consists of functions $x:\Omega\to{\mathbb{R}}$, it is the possible to associate with $x$
a differently defined interval, namely 
\begin{equation}
\label{stoch interval}
]]-\infty,x]]=\left\{(\omega,r)\in\Omega\times{\mathbb{R}}:r\le x(\omega)\right\}
\end{equation}
It should be clear that \eqref{order interval} and \eqref{stoch interval} have little
to do with one another.

${\mathscr{A}}_0(X)$ is clearly a $\cap$-semilattice. It turns out that all semilattices 
isomorphic to ${\mathscr{A}}_0(X)$ with $X$ an order lattice have special properties.

\begin{theorem}
\label{th iso}
Let ${\mathscr{A}}_0$ be a $\cap$-semilattice. The following are equivalent:
\begin{enumerate}[(i)]
\item\label{all} 
each $\Phi_0:{\mathscr{A}}_0\to Y$ is semimodular;
\item\label{iproperty} 
if $A_1,\ldots,A_N\in{\mathscr{A}}_0$ then 
\begin{equation}
\label{property}
\bigcup_{n=1}^NA_n\in{\mathscr{A}}_0
\quad\text{if and only if}\quad
\bigcup_{n=1}^NA_n=A_{n_0}
\quad\text{for some}\quad1\le n_0\le N
\end{equation}
\item\label{isomorphism} 
there exists a $\wedge$-semilattice $X$ such that ${\mathscr{A}}_0$ and ${\mathscr{A}}_0(X)$ are
lattice isomorphic.
\end{enumerate}
\end{theorem}

\begin{proof}
{{(\textit{\ref{{{all}}}})}$\Leftrightarrow${(\textit{\ref{{{iproperty}}}})}}. 
Let $A_1,\ldots,A_N,\bigcup_{n=1}^NA_n\in{\mathscr{A}}_0$ and $y\ne{\emptyset}$ and define 
$\Phi_0:{\mathscr{A}}_0\to Y$ implicitly by letting $\Phi_0(A)=y$ if 
$\bigcup_{n=1}^NA_n\subset A$ or else $\Phi_0(A)={\emptyset}$, for all $A\in{\mathscr{A}}_0$. 
It is then obvious that $\Phi_0(\bigcup_{n=1}^NA_n)=y$ while, if {(\textit{\ref{{iproperty}}})} fails, 
$\sum_{{\varnothing}< b\le[N]}\Phi_0(\bigcap_{n\in b}A_n)={\emptyset}$ so that \eqref{modular cap} 
fails too and $\Phi_0$ is not semimodular. Conversely, assume {(\textit{\ref{{iproperty}}})}. Then 
{(\textit{\ref{{all}}})} follows from \eqref{parsimony}.

{{(\textit{\ref{{{iproperty}}}})}$\Leftrightarrow${(\textit{\ref{{{isomorphism}}}})}}. 
Assume {(\textit{\ref{{isomorphism}}})}, denote the lattice isomorphism by $\tau$ and let 
$A_1,\ldots,A_N\in{\mathscr{A}}_0$ be as in {(\textit{\ref{{iproperty}}})}. There exist then
$x_1,\ldots,x_N\in X$ such that ${\ ]-\infty,{x_n}]}=\tau(A_n)$; moreover,
$\bigcup_{n=1}^N{\ ]-\infty,{x_n}]}=\bigcup_{n=1}^N\tau(A_n)=\tau(\bigcup_{n=1}^NA_n)$ 
so that $\bigcup_{n=1}^N{\ ]-\infty,{x_n}]}\in{\mathscr{A}}_0^X$, i.e. ${\ ]-\infty,{x}]}=\bigcup_{n=1}^N{\ ]-\infty,{x_n}]}$
for some $x\in X$. Thus, $x\ge x_n$ for $n=1,\ldots,N$ and $x\in{\ ]-\infty,{x_{n_0}}]}$ for some 
$n_0$ so that $x=x_{n_0}$. We conclude that $\bigcup_{n=1}^NA_n=\tau^{-1}(\bigcup_{n=1}^N{\ ]-\infty,{x_n}]})=\tau^{-1}({\ ]-\infty,{x_{n_0}}]})=A_{n_0}$.
Conversely, let $X$ be the set ${\mathscr{A}}_0$ endowed with the partial order induced
by inclusion so that ${\mathscr{A}}_0(X)=\{{\ ]-\infty,A]}:A\in{\mathscr{A}}_0\}$. Define $\tau(A)={\ ]-\infty,A]}$, i.e. 
$\tau(A)=\{B\cap A:B\in{\mathscr{A}}_0\}$. Observe that ${\ ]-\infty,A]}={\ ]-\infty,B]}$ if and only if $A=B$,
so that $\tau$ is a bijection. If $A_1,\ldots,A_n\in{\mathscr{A}}_0$, then 
$\tau(\bigcap_{n=1}^NA_n)$ is the family of those $B\in{\mathscr{A}}_0$ such that 
$B\subset\bigcap_{n=1}^NA_n$ and coincides therefore with 
$\bigcap_{n=1}^N\tau(A_n)$. If, moreover, $\bigcup_{n=1}^NA_n\in{\mathscr{A}}_0$, then, 
by {(\textit{\ref{{iproperty}}})}, $A_{n_0}=\bigcup_{n=1}^NA_n$ for some $1\le n_0\le N$ so that 
$\tau(\bigcup_{n=1}^NA_n)=\tau(A_{n_0})=\bigcup_{n=1}^N\tau(A_n)$.
\end{proof}

We obtain as a Corollary this result of Norberg \cite[pp. 6-9]{norberg}.

\begin{corollary}[Norberg]
\label{cor norberg}
Let $X$ be a $\wedge$-lattice and denote by $\mathfrak F(X,Y)$ the collection 
of functions $F:X\to Y$ such that $x,x'\in X$ and $x\sim x'$ imply $F(x)=F(x')$. 
$\mathfrak F(X,Y)$ is linearly isomorphic to $sa_0({\mathscr{A}}_0,Y)$ via the identity 
\begin{align} 
\label{bijection}
\Phi({\ ]-\infty,{x}]})=F(x)\qquad x\in X
\end{align}
\end{corollary}

\begin{proof} 
As noted above, ${\mathscr{A}}_0(X)$ is a $\cap$-lattice and ${\ ]-\infty,{x}]}={\ ]-\infty,{x'}]}$ if and only if 
$x\sim x'$. If $F\in\mathfrak F(X,Y)$, then the right hand side of \eqref{bijection} 
implicitly defines a set function $\Phi:{\mathscr{A}}_0\to Y$ which is semiadditive by Theorem 
\ref{th iso} and the fact that ${\varnothing}\notin{\mathscr{A}}_0(X)$. The converse is obvious.
\end{proof}

Corollary \ref{cor norberg} shows, similarly to the Stiltjes construction, how to 
associate additive set functions with \textit{any} point function, even if of unbounded 
variation. It is based on the fact that for all order lattices ${\mathscr{A}}_0(X)$ satisfies property
\eqref{property}. 

In the literature on set-indexed stochastic processes (see e.g. \cite{dozzi}) property 
\eqref{property} is taken as a starting assumption and used to obtain an \textit{additive}
extension of processes\footnote{I wish to express my gratitude to an anonymous referee who has called 
my attention on the role of property \eqref{property} in this literature, a fact that 
I had, quite surprisingly, completely overlooked. It is then clear, after Theorem 
\ref{th iso} that this extension is perfectly feasible.}. 
Theorem \ref{th iso} shows, however, that this assumption is perfectly adequate 
only in the framework of order semilattices, which may limit considerably the setting
adopted. If, for example, ${\mathscr{A}}_0$ is a lattice (of sets), then \eqref{property} fails 
unless ${\mathscr{A}}_0$ is linearly ordered (by inclusion), a fact which excludes many 
situations of interest. A noteworthy example is the collection of all stochastic intervals 
of the form ${\ ]]-\infty,t]]}$, with $t$ a stopping time on some given filtration, which is a 
non linearly ordered lattice and in fact violates property \eqref{property}. The 
construction of the Dol\'eans-Dade measure associated to a process seems thus 
to require more stringent assumptions.

\section{Functionals on Vector Lattices}
\label{sec choquet}

In this section we provide an application of the results obtained above by developing 
a definition of measurability and of integrability with respect to an arbitrary set function 
$\mu:\mathscr B\to{\mathbb{R}}$ when $\mathscr B$ is an arbitrary family of subsets of $\Omega$
with ${\varnothing}\in\mathscr B$ and $\mu({\varnothing})=0$. 
Among the contributions given to a general theory of integration without additivity, one 
should mention the deep work of Choquet \cite{choquet} on capacities and the 
representation theorem of Greco \cite{greco} characterizing functional that may be written 
as such integrals. One should also mention Denneberg \cite{denneberg}. Our interest for 
this theory lies only in the frequent application of the results obtained above.

\begin{definition}
A function $f:\Omega\to{\mathbb{R}}_+$ is said to be $\mu$-measurable if 
$\{f>t\}\in\mathscr B$ for all $t\in{\mathbb{R}}_+$ and if the distribution function
$\mu_f:{\mathbb{R}}_+\to{\mathbb{R}}$ defined implicitly as
\begin{align}
\label{m+,m-}
\mu_f(t)=\mu(f>t)
\qquad t>0
\end{align}
is Borel measurable. $f:\Omega\to{\mathbb{R}}$ is said to be  $\mu$ measurable if
$f^+$ and $f^-$ are $\mu$-measurable.
\end{definition}

One should compare this definition to that of strong measurability given by 
Denneberg \cite[p. 49]{denneberg}. Observe that if $f$ is measurable
then so are $-f$, $af$ and $f\wedge a$ for $a\in{\mathbb{R}}$. However, $f\vee a$ need not
be measurable we do not assume $\Omega\in\mathscr B$ as well as the sum
of two measurable functions.

The definition of the integral $\int fd\mu$ follows that proposed by Choquet 
\cite[]{choquet}.

\begin{definition}
A function $f:\Omega\to{\mathbb{R}}$ is said to be $\mu$-integrable, in symbols 
$f\in L^1(\mu)$, if $\mu_{f^+},\mu_{f^-}\in L^1(dt)$. The integral of $f$ with 
respect to $\mu$ is then defined to be
\begin{equation}
\label{integral}
\int fd\mu=\int_{{\mathbb{R}}_+}\mu(f^+>t)dt-\int_{{\mathbb{R}}_+}\mu(f^->t)dt
\end{equation}
\end{definition}
This definition should be compared to that of the symmetric integral given by
Denneberg \cite[Chapter 6]{denneberg}. In fact in this literature it is more common
to define
\begin{equation}
\int fd\mu=\int_{{\mathbb{R}}_+}\mu(f^+>t)dt-\int_{{\mathbb{R}}_+}[\mu(\Omega)-\mu(f>-t)]dt
\end{equation}
which however requires $\Omega\in\mathscr B$ and $\mu(\Omega)\in{\mathbb{R}}$.

We list next some useful properties:

\begin{proposition}
\label{pro integral}
Let $\mu:\mathscr B\to{\mathbb{R}}$. If $f\in L^1(\mu)$ and $\lambda\in{\mathbb{R}}$ then 
$f\wedge\lambda,\lambda f\in L^1(\mu)$ and 
\begin{equation}
\label{int property}
\mu(f)
=
\lim_n\{\mu(f^+\wedge n)-\mu(f^-\wedge n)\}
\quad\text{and}\quad
\mu(-f)=-\mu(f)
\end{equation}
If $\mathscr B$ is an algebra and $\mu\in ba(\mathscr B)$ then the integral 
\eqref{integral} coincides with the finitely additive integral.
\end{proposition}

\begin{proof}
Remark that $\mu(f\wedge\lambda>t)$ coincides with $\mu(f>t)$ on $t\le\lambda$
ans is $0$ otherwise; analogously, $\mu(f\wedge\lambda<-t)$ is the same as
$\mu(f<-t)$ on $\lambda>-t$ and $\mu(\Omega)$ on $\lambda\le-t$. Analogous
conclusions hold for $f\vee\lambda$ and $\lambda f$. We deduce measurability
and integrability of $f\wedge\lambda,f\vee\lambda,\lambda f$. Moreover,
\begin{equation*}
\mu(f^+\wedge n)=\int_{{\mathbb{R}}_+}\mu(f\wedge n>t)dt=\int_0^n\mu(f>t)dt
\end{equation*}
and $\mu(f^-\wedge n)=\int_0^n\mu(f<-t)dt$ which imply \eqref{int property}.
$\mu(-f)=-\mu(f)$ is also clear from the definition. Assume that $\mathscr B$
is an algebra of subsets of $\Omega$ and that $\mu\in ba(\mathscr B)$. If
$f\in L^1(\mu)$ is simple, then \eqref{integral} is an obvious implication of Fubini
and the fact that the algebra generated by $f$ is finite. We then prove the claim
for $f$ bounded and, with no loss of generality, $f$ and $\mu$ positive. Let then 
$f$ be bounded by some real number $k>0$. Its integral $\int fd\mu$ is the 
limit of a sequence $\int f_nd\mu$ where ${{{\left\langle {{f}}_{{{n}}}\right\rangle_{{{n}}\in {\mathbb{N}}} }}}$ belongs to ${\mathscr{S}}({\mathscr{A}})$ is 
such that $f_n\le f_{n+1}\le f$ and converges to $f$ in $\lambda$-measure see 
\cite[4.5.7]{rao}. Thus,
\begin{align*}
\int fd\mu=\lim_n\int_0^k\mu(f_n\ge t)dt=\int_0^k\lim_n\mu(f_n\ge t)dt
\le\int_0^k\mu(f\ge t)dt
\end{align*}
On the other hand, 
\begin{align*}
\int_0^k\mu(f\ge t)dt
&\le
\int_0^k\mu(f_n\ge t-\varepsilon)dt+k\mu^*({\vert {f-f_n}\vert}>\varepsilon)\\
&\le
\int_{-\varepsilon}^k\mu(f_n\ge t)dt+k\mu^*({\vert {f-f_n}\vert}>\varepsilon)\\
&\le
\int_0^k\mu(f_n\ge t)dt+\varepsilon+k\mu^*({\vert {f-f_n}\vert}>\varepsilon)
\end{align*}
Eventually, if $f\in L^1(\mu)$,
\begin{align*}
\int fd\mu
=
\lim_n\int (f\wedge n)d\mu=\lim_n\int_{{\mathbb{R}}_+}\mu(f\wedge n\ge t)dt
=
\lim_n\int_0^n\mu(f\ge t)dt=\int_{{\mathbb{R}}_+}\mu(f\ge t)dt
\end{align*}
\end{proof}

Although $\mu$ need not be additive we show that it is necessarily so \textit{locally}.
If $f:\Omega\to{\mathbb{R}}_+$, let 
\begin{align}
\label{A+(f)}
{\mathscr{A}}_0(f)=\{\{f>t\}:t>0\}\cup\{{\varnothing}\}
\end{align}
and, in the general case,
\begin{equation}
\label{A0(f)}
{\mathscr{A}}_0(f)=\{A^+\cup A^-:A^+\in{\mathscr{A}}_0(f^+),A^-\in{\mathscr{A}}_0(f^-)\}
\end{equation}
Let moreover
${\mathscr{A}}(f)$ be the ring generated by ${\mathscr{A}}_0(f)$. 

\begin{proposition}
\label{pro local}
$f\in L^1(\mu)$ if and only if there exists $\nu_f\in sa({\mathscr{A}}(f))$ such that 
$\nu_f(f>t)=\mu(f>t)$, $\nu_f(f<-t)=\mu(f<-t)$ for all $t\in{\mathbb{R}}_+$, $f\in L^1(\nu_f)$
and
\begin{equation}
\int fd\mu=\int fd\nu_f
\end{equation}
\end{proposition}

\begin{proof}
Define 
$\nu_{f,0}:{\mathscr{A}}_0(f)\to{\mathbb{R}}$ by letting
\begin{align*}
\nu_{f,0}(A^+\cup A^-)=\mu(A^+)+\mu(A^-)
\qquad A^+\in{\mathscr{A}}_0(f^+),A^-\in{\mathscr{A}}_0(f^-)
\end{align*}
Observe that ${\mathscr{A}}_0(f)$ is a $\cap$-semilattice which includes ${\varnothing}$ and possesses 
property \eqref{property} so that $\nu_{f,0}\in sa_0({\mathscr{A}}_0(f))$. Thus by Theorem 
\ref{th pettis} $\nu_{f,0}$ admits a unique additive extension $\nu_f$ to ${\mathscr{A}}(f)$. 
We conclude that $\int fd\mu=\int_{{\mathbb{R}}_+}\nu_f(f>t)dt-\int_{{\mathbb{R}}_+}\nu_f(f<-t)dt$ and 
this definition coincides with the usual finitely additive one by Proposition \ref{pro integral}.
\end{proof}

\begin{thebibliography}{99} 
\bibitem{aliprantis} C. D. Aliprantis, O. Burkinshaw (1985), \textit{Positive Operators},
Academic Press, London.
\bibitem{bachman sultan} G. Bachman, A. Sultan (1980), \textit{On Regular Extensions 
of Measures}, Pacific J. Math., \textbf{86}, 389-395.

\bibitem{rao} K. P. S. Bhaskara Rao, M. Bhaskara Rao (1983), \textit{Theory of Charges}, 
Academic Press, London.

\bibitem{bogachev} V. I. Bogachev (2007), \textit{Measure Theory}, Springer-Verlag, 
Berlin - Heidelberg.

\bibitem{choquet} G. Choquet (1954), \textit{Theory of Capacities}, Ann. Inst. Fourier,
\textit{5}, 131-295.

\bibitem{chung} K. L. Chung (1941), \textit{On the Probability of the Occurrence of at 
Least $m$ Events among $n$ Arbitrary Events}, Ann. Math. Stat. \textbf{12}, 328-338.

\bibitem{denneberg} D. Denneberg (1998), \textit{Non Additive Measure and Integral},
Dordrecht, Kluwer.

\bibitem{dozzi} Dozzi M., B. G. Ivanoff, E. Merzbach (1994), \textit{Doob-Meyer 
Decomposition for Set-Indexed Submartingales}, J. Theor. Prob. \textbf 7, 499-525.

\bibitem{bible} N. Dunford, J. Schwartz (1988), \textit{Linear Operators. Part I}, Wiley, 
New York.

\bibitem{greco} G. H. Greco (1982), \textit{Sulla Rappresentazione di Funzionali
Mediante Integrali}, Rend. Sem. Mat. Univ. Padova \textbf{66}, 21-42.

\bibitem{halmos} P. R. Halmos (1974), \textit{Measure Theory}, Springer-Verlag, New 
York-Heidelberg-Berlin.

\bibitem{horn tarski} A. Horn, A. Tarski (1948), \textit{Measures in Boolean Algebras}, 
Trans. Amer. Math. Soc., \textbf{64}, 467-497.

\bibitem{norberg} T. Norberg (1989), \textit{Stochastic Integration on Lattices}, Tech. 
Rep. N. 1989-08, Chalmers Univ. Tech. and Univ. G\"{o}teborg.

\bibitem{pettis} B. J. Pettis (1951), \textit{On the Extension of Measures}, Ann. Math., 
\textbf{54}, 186-197.

\bibitem{rota} G.-C. Rota (1964), \textit{On the Foundations of Combinatorial Theory I. 
Theory of M\"{o}bius Functions}, Z. Wahrsch. Verw. Geb., \textbf{2}, 340-368.
\end{thebibliography}

\end{document}

