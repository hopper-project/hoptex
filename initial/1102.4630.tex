

\documentclass[12pt,reqno]{amsart}
\usepackage{graphicx}
\usepackage{amscd}
\usepackage{amsmath}
\usepackage{epsfig}
\usepackage{amsfonts}
\usepackage{amssymb}

\setcounter{MaxMatrixCols}{10}

\providecommand{\U}[1]{\protect\rule{.1in}{.1in}}
\providecommand{\U}[1]{\protect\rule{.1in}{.1in}}
\textheight=8.9in \textwidth=7in \headheight=8pt \topmargin=0pt
\oddsidemargin=-.25in \evensidemargin=-.25in
\parskip=6pt plus 2pt minus 2pt
\allowdisplaybreaks
\newtheorem{theorem}{Theorem}
\theoremstyle{plain}
\newtheorem{acknowledgement}{Acknowledgement}
\newtheorem{algorithm}{Algorithm}
\newtheorem{axiom}{Axiom}
\newtheorem{case}{Case}
\newtheorem{claim}{Claim}
\newtheorem{conclusion}{Conclusion}
\newtheorem{condition}{Condition}
\newtheorem{conjecture}{Conjecture}
\newtheorem{corollary}{Corollary}
\newtheorem{criterion}{Criterion}
\newtheorem{definition}{Definition}
\newtheorem{example}{Example}
\newtheorem{exercise}{Exercise}
\newtheorem{lemma}{Lemma}
\newtheorem{notation}{Notation}
\newtheorem{problem}{Problem}
\newtheorem{proposition}{Proposition}
\newtheorem{remark}{Remark}
\newtheorem{solution}{Solution}
\newtheorem{summary}{Summary}
\numberwithin{equation}{section}

\begin{document}
\title[The Riccati System]{The Riccati System and a Diffusion-Type Equation}
\author{Erwin Suazo}
\address{Department of Mathematical Sciences, University of Puerto Rico,
Mayaguez, Call Box 9000, Puerto Rico 00681--9000.}
\email{erwin.suazo@upr.edu}
\author{Sergei K. Suslov}
\address{School of Mathematical and Statistical Sciences \& Mathematical,
Computational and Modeling Sciences Center, Arizona State University, Tempe,
AZ 85287--1804, U.S.A.}
\email{sks@asu.edu}
\urladdr{http://hahn.la.asu.edu/\symbol{126}suslov/index.html}
\author{Jos\'{e} M. Vega-Guzm\'{a}n}
\address{Mathematical, Computational and Modeling Sciences Center, Arizona
State University, Tempe, AZ 85287--1904, U.S.A.}
\email{jmvega@asu.edu}
\date{\today }
\subjclass{Primary 35C05, 35K15, 42A38. Secondary 35A08, 80A99.}
\keywords{Diffusion-type equations, Green's function, fundamental solution,
autonomous and nonautonomous Burgers equations, Fokker--Planck equation,
Riccati equation and Riccati-type system, Ermakov-type system and
Pinney-type solution.}

\begin{abstract}
We discuss a method of constructing solution of the initial value problem
for diffusion-type equations in terms of solutions of certain Riccati and
Ermakov-type systems. A nonautonomous Burgers-type equation is also
considered.
\end{abstract}

\maketitle

\section{Introduction}

A goal of this note, complementary to our recent paper \cite{SuazoSusVega10}, is to elaborate on the Cauchy initial value problem for a class of
nonautonomous and inhomogeneous diffusion-type equations on $\mathbb{R}
.$ A corresponding nonautonomous Burgers-type equation is also analyzed as a
by-product. Here, we use explicit transformations to the standard forms and
emphasize natural relations with certain Riccati and Ermakov-type systems,
which seem are missing in the available literature. Similar methods are
applied to the corresponding Schr\"{o}dinger equation (see, for example, 
\cite{Cor-Sot:Lop:Sua:Sus}, \cite{Cor-Sot:Sua:Sus}, \cite{Cor-Sot:Sua:SusInv}, \cite{Cor-Sot:Sus}, \cite{Cor-Sot:SusDPO}, \cite{Lan:Lop:Sus}, \cite{Lan:Sus}, \cite{Lop:Sus}, \cite{Me:Co:Su}, \cite{Suaz:Sus}, \cite{Suslov10}, \cite{Suslov11} and references therein). A group theoretical approach to a
similar class of partial differential equations is discussed in Refs.~\cite{GagWint93}, \cite{Miller77} and \cite{Rosen76}.

For an introduction to fundamental solutions for parabolic equations, see
chapter one of the book by Friedman \cite{Friedman64}. Among numerous
applications, we only elaborate here on an important role of fundamental
solutions in probability theory \cite{Craddock09}, \cite{KaratzShreve}.
Consider an It\^{o} diffusion $X=\left\{ X_{t}:t\geq 0\right\} $ which
satisfies the stochastic differential equation\begin{equation}
dX_{t}=b\left( X_{t},t\right) \ dt+\sigma \left( X_{t},t\right) \
dW_{t},\qquad X_{0}=x,  \label{SDE}
\end{equation}in which $W=\left\{ W_{t}:t\geq 0\right\} $ is a standard Wiener process.
The existence and uniqueness of solutions of (\ref{SDE}) depends on the
coefficients $b$ and $\sigma .$ (See Ref.~\cite{KaratzShreve} for conditions
of unique strong solution to (\ref{SDE}).) If the equation (\ref{SDE}) has a
unique solution, then the expectations\begin{equation}
u\left( x,t\right) =E_{x}\left[ \phi \left( X_{t}\right) \right] =E\left[
\phi \left( X_{t}\right) |X_{0}=x\right]  \label{Expect}
\end{equation}are solutions of the Cauchy problem\begin{equation}
u_{t}=\frac{1}{2}\sigma ^{2}\left( x,t\right) u_{xx}+b\left( x,t\right)
u_{x},\qquad u\left( x,0\right) =\phi \left( x\right) .  \label{SDECauchy}
\end{equation}This PDE is known as Kolmogorov forward equation \cite{Craddock09}, \cite{KaratzShreve}. Thus if $p\left( x,y,t\right) $ is the appropriate
fundamental solution of (\ref{SDECauchy}), then one can compute the given
expectations according to\begin{equation}
E_{x}\left[ \phi \left( X_{t}\right) \right] =\int_{\Omega }p\left(
x,y,t\right) \phi \left( y\right) \ dy.  \label{SDEExpect}
\end{equation}In this context, the fundamental solution is known as the probability
transition density for the process and\begin{equation}
\int_{\Omega }p\left( x,y,t\right) \ dy=1.  \label{norm}
\end{equation}See also Refs.~\cite{Albev:Roz10} and \cite{Kamb10} for applications to
stochastic differential equations related to Fokker--Planck and Burgers
equations.

\section{Transformation to the Standard Form}

We present the following result.

\begin{lemma}
The nonautonomous and inhomogeneous diffusion-type equation\begin{equation}
\frac{\partial u}{\partial t}=a\left( t\right) \frac{\partial ^{2}u}{\partial x^{2}}-\left( g\left( t\right) -c\left( t\right) x\right) \frac{\partial u}{\partial x}+\left( d\left( t\right) +f\left( t\right) x-b\left(
t\right) x^{2}\right) u,  \label{heat}
\end{equation}where $a,b,c,d,f,g$ are suitable functions of time $t$\ only, can be reduced
to the standard autonomous form\begin{equation}
\frac{\partial v}{\partial \tau }=\frac{\partial ^{2}v}{\partial \xi ^{2}}
\label{standard}
\end{equation}with the help of the following substitution:\begin{align}
u\left( x,t\right) & =\frac{1}{\sqrt{\mu \left( t\right) }}e^{\alpha \left(
t\right) x^{2}+\delta \left( t\right) x+\kappa \left( t\right) }v\left( \xi
,\tau \right) ,  \label{substitution} \\
\xi & =\beta \left( t\right) x+\varepsilon \left( t\right) ,\quad \tau
=\gamma \left( t\right) .  \notag
\end{align}Here, $\mu ,\alpha ,\beta ,\gamma ,\delta ,\varepsilon ,\kappa $ are
functions of $t$\ that satisfy\begin{equation}
\frac{\mu ^{\prime }}{2\mu }+2a\alpha +d=0  \label{mu}
\end{equation}and\begin{align}
& \frac{d\alpha }{dt}+b-2c\alpha -4a\alpha ^{2}=0,  \label{alpha} \\
& \frac{d\beta }{dt}-\left( c+4a\alpha \right) \beta =0,  \label{beta} \\
& \frac{d\gamma }{dt}-a\beta ^{2}=0,  \label{gamma} \\
& \frac{d\delta }{dt}-\left( c+4a\alpha \right) \delta =f-2\alpha g,
\label{delta} \\
& \frac{d\varepsilon }{dt}+\left( g-2a\delta \right) \beta =0,
\label{epsilon} \\
& \frac{d\kappa }{dt}+g\delta -a\delta ^{2}=0.  \label{kappa}
\end{align}
\end{lemma}

Equation (\ref{alpha}) is called the \textit{Riccati nonlinear differential
equation} \cite{Reid72}, \cite{Wa}, \cite{Wh:Wa} and we shall refer to the
system (\ref{alpha})--(\ref{kappa}) as a \textit{Riccati-type system}.

The substitution (\ref{mu}) reduces the nonlinear Riccati equation (\ref{alpha}) to the second order linear equation\begin{equation}
\mu ^{\prime \prime }-\tau \left( t\right) \mu ^{\prime }-4\sigma \left(
t\right) \mu =0,  \label{charequation}
\end{equation}where\begin{equation}
\tau \left( t\right) =\frac{a^{\prime }}{a}+2c-4d,\quad \sigma \left(
t\right) =ab+cd-d^{2}+\frac{d}{2}\left( \frac{a^{\prime }}{a}-\frac{d^{\prime }}{d}\right) ,  \label{SigTau}
\end{equation}which shall be referred to as a \textit{characteristic equation} \cite{SuazoSusVega10}.

It is also known \cite{SuazoSusVega10} that the diffusion-type equation (\ref{heat}) has a particular solution of the form\begin{equation}
u=\frac{1}{\sqrt{\mu \left( t\right) }}e^{\alpha \left( t\right) x^{2}+\beta
\left( t\right) xy+\gamma \left( t\right) y^{2}+\delta \left( t\right)
x+\varepsilon \left( t\right) y+\kappa \left( t\right) },
\label{partsolution}
\end{equation}provided that the time dependent functions $\mu ,\alpha ,\beta ,\gamma
,\delta ,\varepsilon ,\kappa $ satisfy the Riccati-type system (\ref{mu})--(\ref{kappa}).

A group theoretical approach to a similar class of partial differential
equations is discussed in Refs.~\cite{GagWint93}, \cite{Miller77} and \cite{Rosen76}.

\section{Fundamental Solution}

By the \textit{superposition principle} one can solve (formally) the Cauchy
initial value problem for the diffusion-type equation (\ref{heat}) subject
to initial data $u\left( x,0\right) =\varphi \left( x\right) $ on the entire
real line $-\infty <x<\infty $ in an integral form\begin{equation}
u\left( x,t\right) =\int_{-\infty }^{\infty }K_{0}\left( x,y,t\right) \
\varphi \left( x\right) dy  \label{CIVP}
\end{equation}with the \textit{fundamental solution} (heat kernel) \cite{SuazoSusVega10}:\begin{equation}
K_{0}\left( x,y,t\right) =\frac{1}{\sqrt{2\pi \mu _{0}\left( t\right) }}\
e^{\alpha _{0}\left( t\right) x^{2}+\beta _{0}\left( t\right) xy+\gamma
_{0}\left( t\right) y^{2}+\delta _{0}\left( t\right) x+\varepsilon
_{0}\left( t\right) y+\kappa _{0}\left( t\right) },  \label{heatkernel}
\end{equation}where a particular solution of the Riccati-type system (\ref{alpha})--(\ref{kappa}) is given by 
\begin{equation}
\alpha _{0}\left( t\right) =-\frac{1}{4a\left( t\right) }\frac{\mu
_{0}^{\prime }\left( t\right) }{\mu _{0}\left( t\right) }-\frac{d\left(
t\right) }{2a\left( t\right) },  \label{A0}
\end{equation}\begin{equation}
\beta _{0}\left( t\right) =\frac{h\left( t\right) }{\mu _{0}\left( t\right) },\quad h\left( t\right) =\exp \left( \int_{0}^{t}\left( c\left( s\right)
-2d\left( s\right) \right) \ ds\right) ,  \label{B0}
\end{equation}\begin{align}
\gamma _{0}\left( t\right) & =\frac{d\left( 0\right) }{2a\left( 0\right) }-\frac{a\left( t\right) h^{2}\left( t\right) }{\mu _{0}\left( t\right) \mu
_{0}^{\prime }\left( t\right) }-4\int_{0}^{t}\frac{a\left( s\right) \sigma
\left( s\right) h\left( s\right) }{\left( \mu _{0}^{\prime }\left( s\right)
\right) ^{2}}\ ds  \label{C0} \\
& =\frac{d\left( 0\right) }{2a\left( 0\right) }-\frac{1}{2\mu _{1}\left(
0\right) }\frac{\mu _{1}\left( t\right) }{\mu _{0}\left( t\right) },
\label{C1}
\end{align}\begin{equation}
\delta _{0}\left( t\right) =\frac{h\left( t\right) }{\mu _{0}\left( t\right) 
}\ \ \int_{0}^{t}\left[ \left( f\left( s\right) +\frac{d\left( s\right) }{a\left( s\right) }g\left( s\right) \right) \mu _{0}\left( s\right) +\frac{g\left( s\right) }{2a\left( s\right) }\mu _{0}^{\prime }\left( s\right) \right] \ \frac{ds}{h\left( s\right) },  \label{D0}
\end{equation}\begin{align}
\varepsilon _{0}\left( t\right) & =-\frac{2a\left( t\right) h\left( t\right) 
}{\mu _{0}^{\prime }\left( t\right) }\delta _{0}\left( t\right)
-8\int_{0}^{t}\frac{a\left( s\right) \sigma \left( s\right) h\left( s\right) 
}{\left( \mu _{0}^{\prime }\left( s\right) \right) ^{2}}\left( \mu
_{0}\left( s\right) \delta _{0}\left( s\right) \right) \ ds  \label{E0} \\
& \quad +2\int_{0}^{t}\frac{a\left( s\right) h\left( s\right) }{\mu
_{0}^{\prime }\left( s\right) }\left[ f\left( s\right) +\frac{d\left(
s\right) }{a\left( s\right) }g\left( s\right) \right] \ ds,  \notag
\end{align}\begin{align}
\kappa _{0}\left( t\right) & =-\frac{a\left( t\right) \mu _{0}\left(
t\right) }{\mu _{0}^{\prime }\left( t\right) }\delta _{0}^{2}\left( t\right)
-4\int_{0}^{t}\frac{a\left( s\right) \sigma \left( s\right) }{\left( \mu
_{0}^{\prime }\left( s\right) \right) ^{2}}\left( \mu _{0}\left( s\right)
\delta _{0}\left( s\right) \right) ^{2}\ ds  \label{K0} \\
& \quad +2\int_{0}^{t}\frac{a\left( s\right) }{\mu _{0}^{\prime }\left(
s\right) }\left( \mu _{0}\left( s\right) \delta _{0}\left( s\right) \right) \left[ f\left( s\right) +\frac{d\left( s\right) }{a\left( s\right) }g\left(
s\right) \right] \ ds  \notag
\end{align}with $\delta \left( 0\right) =g\left( 0\right) /\left( 2a\left( 0\right)
\right) ,$ $\varepsilon \left( 0\right) =-\delta \left( 0\right) ,$ $\kappa
\left( 0\right) =0.$ Here, $\mu _{0}$ and $\mu _{1}$ are the so-called 
\textit{standard solutions} of the characteristic equation (\ref{charequation}) subject to the\ following initial data\begin{equation}
\mu _{0}\left( 0\right) =0,\quad \mu _{0}^{\prime }\left( 0\right) =2a\left(
0\right) \neq 0\qquad \mu _{1}\left( 0\right) \neq 0,\quad \mu _{1}^{\prime
}\left( 0\right) =0.  \label{standarddata}
\end{equation}Solution (\ref{A0})--(\ref{K0}) shall be referred to as a \textit{fundamental solution} of the Riccati-type system (\ref{alpha})--(\ref{kappa}); see (\ref{AsA0})--(\ref{AsF0}) and (\ref{AsK0}) for the corresponding
asymptotics.

\begin{lemma}
The Riccati-type system (\ref{mu})--(\ref{kappa}) has the following
(general) solution:\begin{align}
& \mu \left( t\right) =-2\mu \left( 0\right) \mu _{0}\left( t\right) \left(
\alpha \left( 0\right) +\gamma _{0}\left( t\right) \right) ,  \label{MKernel}
\\
& \alpha \left( t\right) =\alpha _{0}\left( t\right) -\frac{\beta
_{0}^{2}\left( t\right) }{4\left( \alpha \left( 0\right) +\gamma _{0}\left(
t\right) \right) },  \label{AKernel} \\
& \beta \left( t\right) =-\frac{\beta \left( 0\right) \beta _{0}\left(
t\right) }{2\left( \alpha \left( 0\right) +\gamma _{0}\left( t\right)
\right) },  \label{BKernel} \\
& \gamma \left( t\right) =\gamma \left( 0\right) -\frac{\beta ^{2}\left(
0\right) }{4\left( \alpha \left( 0\right) +\gamma _{0}\left( t\right)
\right) }  \label{CKernel}
\end{align}and\begin{align}
\delta \left( t\right) & =\delta _{0}\left( t\right) -\frac{\beta _{0}\left(
t\right) \left( \delta \left( 0\right) +\varepsilon _{0}\left( t\right)
\right) }{2\left( \alpha \left( 0\right) +\gamma _{0}\left( t\right) \right) 
},  \label{DKernel} \\
\varepsilon \left( t\right) & =\varepsilon \left( 0\right) -\frac{\beta
\left( 0\right) \left( \delta \left( 0\right) +\varepsilon _{0}\left(
t\right) \right) }{2\left( \alpha \left( 0\right) +\gamma _{0}\left(
t\right) \right) },  \label{EKernel} \\
\kappa \left( t\right) & =\kappa \left( 0\right) +\kappa _{0}\left( t\right)
-\frac{\left( \delta \left( 0\right) +\varepsilon _{0}\left( t\right)
\right) ^{2}}{4\left( \alpha \left( 0\right) +\gamma _{0}\left( t\right)
\right) }  \label{FKernel}
\end{align}in terms of the fundamental solution (\ref{A0})--(\ref{K0}) subject to
arbitrary initial data $\mu \left( 0\right) ,$ $\alpha \left( 0\right) ,$ $\beta \left( 0\right) ,$ $\gamma \left( 0\right) ,$ $\delta \left( 0\right)
, $ $\varepsilon \left( 0\right) ,$ $\kappa \left( 0\right) .$
\end{lemma}

\begin{proof}
Use (\ref{partsolution})--(\ref{heatkernel}), uniqueness of the solution and
the elementary integral:\begin{equation}
\int_{-\infty }^{\infty }e^{-ay^{2}+2by}\ dy=\sqrt{\frac{\pi }{a}}\
e^{b^{2}/a},\quad a>0.  \label{Gauss}
\end{equation}Computational details are left to the reader.
\end{proof}

\begin{remark}
It is worth noting that our transformation (\ref{substitution}), combined
with the standard heat kernel \cite{NiPDE}:\begin{equation}
K_{0}\left( \xi ,\eta ,\tau \right) =\frac{1}{\sqrt{4\pi \left( \tau -\tau
_{0}\right) }}\exp \left[ -\frac{\left( \xi -\eta \right) ^{2}}{4\left( \tau
-\tau _{0}\right) }\right]  \label{heatstandard}
\end{equation}for the diffusion equation (\ref{standard}) and (\ref{MKernel})--(\ref{FKernel}), allows one to derive the fundamental solution (\ref{heatkernel})
of the diffusion-type equation (\ref{heat}) from a new perspective.
\end{remark}

\begin{lemma}
Solution (\ref{MKernel})--(\ref{FKernel}) implies:\begin{align}
& \mu _{0}=\frac{2\mu }{\mu \left( 0\right) \beta ^{2}\left( 0\right) }\left( \gamma -\gamma \left( 0\right) \right) ,  \label{M0} \\
& \alpha _{0}=\alpha _{0}\left( t\right) -\frac{\beta ^{2}}{4\left( \gamma
-\gamma \left( 0\right) \right) },  \label{AA0} \\
& \beta _{0}=\frac{\beta \left( 0\right) \beta }{2\left( \gamma -\gamma
\left( 0\right) \right) },  \label{BB0} \\
& \gamma _{0}=-\alpha \left( 0\right) -\frac{\beta ^{2}\left( 0\right) }{4\left( \gamma -\gamma \left( 0\right) \right) }  \label{CC0}
\end{align}and\begin{align}
\delta _{0}& =\delta -\frac{\beta \left( \varepsilon -\varepsilon \left(
0\right) \right) }{2\left( \gamma -\gamma \left( 0\right) \right) },
\label{DD0} \\
\varepsilon _{0}& =-\delta \left( 0\right) +\frac{\beta \left( 0\right)
\left( \varepsilon -\varepsilon \left( 0\right) \right) }{2\left( \gamma
-\gamma \left( 0\right) \right) },  \label{EE0} \\
\kappa _{0}& =\kappa -\kappa \left( 0\right) -\frac{\left( \varepsilon
-\varepsilon \left( 0\right) \right) ^{2}}{4\left( \gamma -\gamma \left(
0\right) \right) },  \label{FF0}
\end{align}which gives the following asymptotics\begin{align}
& \alpha _{0}\left( t\right) =-\frac{1}{4a\left( 0\right) t}-\frac{c\left(
0\right) }{4a\left( 0\right) }+\frac{a^{\prime }\left( 0\right) }{8a^{2}\left( 0\right) }+\mathcal{O}\left( t\right) ,  \label{AsA0} \\
& \beta _{0}\left( t\right) =\frac{1}{2a\left( 0\right) t}-\frac{a^{\prime
}\left( 0\right) }{4a^{2}\left( 0\right) }+\mathcal{O}\left( t\right) ,
\label{AsB0} \\
& \gamma _{0}\left( t\right) =-\frac{1}{4a\left( 0\right) t}+\frac{c\left(
0\right) }{4a\left( 0\right) }+\frac{a^{\prime }\left( 0\right) }{8a^{2}\left( 0\right) }+\mathcal{O}\left( t\right) ,  \label{AsC0} \\
& \delta _{0}\left( t\right) =\frac{g\left( 0\right) }{2a\left( 0\right) }+\mathcal{O}\left( t\right) ,\qquad \varepsilon _{0}\left( t\right) =-\frac{g\left( 0\right) }{2a\left( 0\right) }+\mathcal{O}\left( t\right) ,
\label{AsDE0} \\
& \kappa _{0}\left( t\right) =\mathcal{O}\left( t\right)  \label{AsF0}
\end{align}as $t\rightarrow 0^{+}.$
\end{lemma}

(The proof is left to the reader.)

These formulas allows to establish a required asymptotic of the fundamental
solution (\ref{heatkernel}):\begin{align}
K_{0}\left( x,y,t\right) & \sim \frac{1}{\sqrt{4\pi a\left( 0\right) t}}\exp \left[ -\frac{\left( x-y\right) ^{2}}{4a\left( 0\right) t}\right]
\label{AsK0} \\
& \times \exp \left[ \frac{a^{\prime }\left( 0\right) }{8a^{2}\left(
0\right) }\left( x-y\right) ^{2}-\frac{c\left( 0\right) }{4a\left( 0\right) }\left( x^{2}-y^{2}\right) \right] \exp \left[ \frac{g\left( 0\right) }{2a\left( 0\right) }\left( x-y\right) \right] .  \notag
\end{align}(Here, $f\sim g$ as $t\rightarrow 0^{+},$ if $\lim_{t\rightarrow
0^{+}}\left( f/g\right) =$ $1.$ The proof is left to the reader.)

By a direct substitution one can verify that the right hand sides of (\ref{MKernel})--(\ref{FKernel}) satisfy the Riccati-type system (\ref{mu})--(\ref{kappa}) and that the asymptotics (\ref{AsA0})--(\ref{AsF0}) result in the
continuity with respect to initial data:\begin{equation}
\lim_{t\rightarrow 0^{+}}\mu \left( t\right) =\mu \left( 0\right) ,\quad
\lim_{t\rightarrow 0^{+}}\alpha \left( t\right) =\alpha \left( 0\right)
,\quad \text{etc.\label{lims}}
\end{equation}The transformation property (\ref{MKernel})--(\ref{FKernel}) allows one to
find solution of the initial value problem in terms of the fundamental
solution (\ref{A0})--(\ref{K0}) and may be referred to as a \textit{nonlinear superposition principle} for the Riccati-type system.

\section{Eigenfunction Expansion and Ermakov-type System}

With the help of transformation (\ref{substitution}) one can reduce the
diffusion equation (\ref{heat}) to another convenient form\begin{equation}
\frac{\partial v}{\partial \tau }=\frac{\partial ^{2}v}{\partial \xi ^{2}}+\xi ^{2}v,  \label{harmonic}
\end{equation}which allows to find solution of the Cauchy initial value problem in terms
of an eigenfunction expansion similar to the case of the corresponding Schr\"{o}dinger in Refs.~\cite{Lan:Lop:Sus} and \cite{Suslov10}. This method
requires an extension the Riccati-type system (\ref{alpha})--(\ref{kappa})
to a more general Ermakov-type system \cite{Lan:Lop:Sus}, which is
integrable in quadratures once again in terms of solutions of the
characteristic equation (\ref{charequation}). Further details are left to
the reader.

\section{Nonautonomous Burgers Equation}

The nonlinear equation\begin{align}
& \frac{\partial v}{\partial t}+a\left( t\right) \left( v\frac{\partial v}{\partial x}-\frac{\partial ^{2}v}{\partial x^{2}}\right) -c\left( t\right)
\left( x\frac{\partial v}{\partial x}+v\right) +g\left( t\right) \frac{\partial v}{\partial x}  \label{NABurgers} \\
& \qquad =2\left( 2b\left( t\right) x-f\left( t\right) \right) ,  \notag
\end{align}when $a=1$ and $b=c=f=g=0,$ is known as \textit{Burgers' equation} \cite{Bateman15}, \cite{Burgers48}, \cite{Cole50}, \cite{Hopf50}, \cite{Kadom:Karp71}, \cite{Sach87}, \cite{Whitham}\textit{\ }and we shall refer
to (\ref{NABurgers}) as a nonautonomous \textit{Burgers-type equation}.

\begin{lemma}
The following identity holds\begin{align}
& v_{t}+a\left( vv_{x}-v_{xx}\right) +\left( g-cx\right) v_{x}-cv+2\left(
f-2bx\right)  \notag \\
& \qquad =-2\left( \frac{u_{t}-Qu}{u}\right) _{x},  \label{BurgersIdentity}
\end{align}if\begin{equation}
v=-2\frac{u_{x}}{u}\qquad \left( \text{The \textit{Cole--Hopf transformation}}\right)  \label{ColeHopf}
\end{equation}and\begin{equation}
Qu=au_{xx}-\left( g-cx\right) u_{x}+\left( d+fx-bx^{2}\right) u
\label{HOperator}
\end{equation}($a,$ $b,$ $c,$ $d,$ $f,$ $g$ are functions of $t$ only).
\end{lemma}

(This can be verified by a direct substitution.)

The substitution (\ref{ColeHopf}) turns the nonlinear Burgers-type equation (\ref{NABurgers}) into the diffusion-type equation (\ref{heat}). Then
solution of the corresponding Cauchy initial value problem can be
represented as\begin{equation}
v\left( x,t\right) =-2\frac{\partial }{\partial x}\ln \left[ \int_{-\infty
}^{\infty }K_{0}\left( x,y,t\right) \exp \left( -\frac{1}{2}\int_{0}^{y}v\left( z,0\right) \ dz\right) \ dy\right] ,  \label{CIVPBurgers}
\end{equation}where the heat kernel is given by (\ref{heatkernel}), for suitable initial
data $v\left( z,0\right) $ on $\mathbb{R}
.$

\section{Traveling Wave Solutions of Burgers-type Equation}

Looking for solutions of our equation (\ref{NABurgers}) in the form\begin{equation}
v=\beta \left( t\right) F\left( \beta \left( t\right) x+\gamma \left(
t\right) \right) =\beta F\left( z\right) ,\quad z=\beta x+\gamma 
\label{BurgersSubst}
\end{equation}($\beta $ and $\gamma $ are functions of $t$ only), one gets\begin{equation}
F^{\prime \prime }=\left( c_{0}+c_{1}\right) F^{\prime }+FF^{\prime
}+2c_{2}z+c_{3}  \label{BurgersPreRiccati}
\end{equation}provided that\begin{eqnarray}
&&\beta ^{\prime }=c\beta ,\qquad \quad \gamma ^{\prime }=c_{0}a\beta ^{2},
\label{BSysA} \\
&&g=c_{1}a\beta ,\qquad b=-\frac{1}{2}c_{2}a\beta ^{4},  \label{BSysB} \\
&&f=\frac{1}{2}a\beta ^{3}\left( 2c_{2}\gamma +c_{3}\right)   \label{BSysC}
\end{eqnarray}($c_{0},$ $c_{1},$ $c_{2},$ $c_{3}$ are constants). From (\ref{BurgersPreRiccati}):\begin{equation}
F^{\prime }=\left( c_{0}+c_{1}\right) F+\frac{1}{2}F^{2}+c_{2}z^{2}+c_{3}z+c_{4},  \label{BurgersRiccati}
\end{equation}where $c_{4}$ is a constant of integration. The substitution\begin{equation}
F=-2\frac{\mu ^{\prime }}{\mu }  \label{RiccatiSubstitution}
\end{equation}transforms the Riccati equation (\ref{BurgersRiccati}) into a special case
of generalized equation of hypergeometric type:\begin{equation}
\mu ^{\prime \prime }-\left( c_{0}+c_{1}\right) \mu ^{\prime }+\frac{1}{2}\left( c_{2}z^{2}+c_{3}z+c_{4}\right) \mu =0,  \label{NUEquation}
\end{equation}which can be solved in general by methods of Ref.~\cite{Ni:Uv}. Elementary
solutions are discussed, for example, in \cite{KudryashovBook10} and \cite{Kudryash:Sine09}.

\section{Some Examples}

Now we consider from a united viewpoint several elementary diffusion and
Burgers equations that are important in applications.

\bigskip

\textbf{Example~1\quad }For the standard \textit{diffusion equation} on $\mathbb{R}
:$\begin{equation}
\frac{\partial u}{\partial t}=a\frac{\partial ^{2}u}{\partial x^{2}},\qquad
a=\text{constant}>0  \label{sp1}
\end{equation}the heat kernel is given by\begin{equation}
K\left( x,y,t\right) =\frac{1}{\sqrt{4\pi at}}\exp \left[ -\frac{\left(
x-y\right) ^{2}}{4at}\right] ,\qquad t>0.  \label{sp2}
\end{equation}(See \cite{Cann}, \cite{NiPDE} and references therein for a detailed
investigation of the classical one-dimensional heat equation.)

\bigskip

\textbf{Example 2\quad }In mathematical description of the nerve cell a
dendritic branch is typically modeled by using cylindrical \textit{cable
equation} \cite{JackNobleTsien83}:\begin{equation}
\tau \frac{\partial u}{\partial t}=\lambda ^{2}\frac{\partial ^{2}u}{\partial x^{2}}+u,\quad \tau =\text{constant}>0.  \label{sp3}
\end{equation}The fundamental solution on $\mathbb{R}
$ is given by\begin{equation}
K_{0}\left( x,y,t\right) =\frac{\sqrt{\tau }e^{t/\tau }}{\sqrt{4\pi \lambda
^{2}t}}\exp \left[ -\frac{\tau \left( x-y\right) ^{2}}{4\lambda ^{2}t}\right]
,\quad t>0.  \label{sp4}
\end{equation}(See also \cite{Herr-Val:Sus} and references therein.)

\bigskip

\textbf{Example 3\quad }The fundamental solution of the \textit{Fokker-Planck equation }\cite{Risken89}, \cite{Yau04}: 
\begin{equation}
\frac{\partial u}{\partial t}=\frac{\partial ^{2}u}{\partial x^{2}}+x\frac{\partial u}{\partial x}+u  \label{sp5}
\end{equation}on $\mathbb{R}
$ is given by \cite{SuazoSusVega10}:\begin{equation}
K_{0}\left( x,y,t\right) =\frac{1}{\sqrt{2\pi \left( 1-e^{-2t}\right) }}\exp \left[ -\frac{\left( x-e^{-t}y\right) ^{2}}{2\left( 1-e^{-2t}\right) }\right]
,\quad t>0.  \label{sp6}
\end{equation}Here, 
\begin{equation}
\lim_{t\rightarrow \infty }K_{0}\left( x,y,t\right) =\frac{e^{-x^{2}/2}}{\sqrt{2\pi }},\qquad y=\text{constant}.  \label{FPLimit}
\end{equation}

\bigskip

\textbf{Example 4\quad }Equation\begin{equation}
\frac{\partial u}{\partial t}=a\frac{\partial ^{2}u}{\partial x^{2}}+\left(
g-kx\right) \frac{\partial u}{\partial x},\qquad a,k>0,\quad g\geq 0
\label{sp7}
\end{equation}corresponds to the heat equation with linear drift when $g=0$ \cite{Miller77}. In stochastic differential equations this equation corresponds the
Kolmogorov forward equation for the regular Ornstein--Uhlenbech process \cite{Craddock09}. The fundamental solution is given by\begin{align}
& K_{0}\left( x,y,t\right) =\frac{\sqrt{k}e^{kt/2}}{\sqrt{4\pi a\sinh \left(
kt\right) }}  \label{sp8} \\
& \qquad \times \exp \left[ -\frac{\left( k\left(
xe^{-kt/2}-ye^{kt/2}\right) +2g\sinh \left( kt/2\right) \right) ^{2}}{4ak\sinh \left( kt\right) }\right] ,\quad t>0.  \notag
\end{align}(See Refs.~\cite{Craddock09} and \cite{SuazoSusVega10} for more details.)

\bigskip

\textbf{Example 5\quad }The \textit{viscous Burgers equation }\cite{Bateman15}, \cite{Burgers48}, \cite{Kadom:Karp71}, \cite{Kudryash:Sine09}, 
\cite{Whitham}:\begin{equation}
\frac{\partial v}{\partial t}+v\frac{\partial v}{\partial x}=a\frac{\partial
^{2}v}{\partial x^{2}},\qquad a=\text{constant}>0  \label{vBurgers}
\end{equation}can be linearized by the \textit{Cole--Hopf substitution} \cite{Cole50}, 
\cite{Hopf50}:\begin{equation}
v=-\frac{2a}{u}\frac{\partial u}{\partial x},  \label{vColeHopf}
\end{equation}which turns it into the diffusion equation (\ref{sp1}). Solution of the
initial value problem has the form:\begin{equation}
v\left( x,t\right) =-\frac{a}{\sqrt{\pi at}}\frac{\partial }{\partial x}\ln \left[ \int_{-\infty }^{\infty }\exp \left( -\frac{\left( x-y\right) ^{2}}{4at}-\frac{1}{2a}\int_{0}^{y}v\left( z,0\right) dz\right) dy\right]
\label{vBurgersSolution}
\end{equation}for $t>0$ and suitable initial data on $\mathbb{R}
.\medskip $

\bigskip

\textbf{Example 6\quad }Equation (\ref{vBurgers}) possesses a solution of
the form:\begin{equation}
v=F\left( x+Vt\right) ,\qquad V=\text{constant}
\end{equation}(we follow the original Bateman paper \cite{Bateman15} with slightly
different notations), if\begin{equation}
VF^{\prime }+FF^{\prime }=aF^{\prime \prime },
\end{equation}or\begin{equation}
\left( F+V\right) ^{2}\pm A^{2}=2aF^{\prime },
\end{equation}where $A$ is a positive constant. The solution is thus either\begin{equation}
v+V=A\tan \left[ \frac{A\left( x+Vt-c\right) }{2a}\right]
\end{equation}or\begin{equation}
\frac{A-v-V}{A+v+V}=\exp \left[ \frac{A}{a}\left( x+Vt-c\right) \right] ,
\end{equation}according as the $+$ or $-$ sign is taken. In the first case there is no
definite value of $v$ when $a$ tends to zero, while in the second case the
limiting value of $v$ is either $A-V$ or $A+V$ according as $x+Vt$ is less
or greater than $c.$ The limiting form of the solution is thus discontinuous 
\cite{Bateman15}.\medskip

Further examples can be found in Refs.~\cite{Craddock09}, \cite{Kudryash:Sine09}, \cite{Lop:Sus}, \cite{Miller77} and \cite{SuazoSusVega10}.

\section{Conclusion}

In this note, we have discussed connections of certain nonautonomous and
inhomogeneous diffusion-type equation and Burgers equation with solutions of
the Riccati and Ermakov-type systems that seem are missing in the available
literature. Traveling wave solutions of the Burgers-type equations are also
discussed.

\noindent \textbf{Acknowledgments.\/} We thank Professor Carlos Castillo-Ch\'{a}vez and Professor Carl Gardner for support, valuable discussions and
encouragement.

\begin{thebibliography}{99}
\bibitem{Albev:Roz10} S.~Albeverio and O.~Rozanova, \emph{Suppression of
unbounded gradients in an SDE associated with Burgers equation\/},
Trans.~Amer. Math.~Soc. \textbf{138} (2010)~\#1, 241--251.

\bibitem{Bateman15} H.~Bateman, \emph{Some recent researches on the motion
of fluids\/}, Monthly Weather Review \textbf{43} (1915)~\#4, 163--170.

\bibitem{Burgers48} J.~M.~Burgers, \emph{A mathematical model illustrating
the theory of turbulence\/}, Adv. Appl. Mech. \textbf{1} (1948), 171--199.

\bibitem{Cann} J.~R.~Cannon, \textsl{The One-Dimensional Heat Equation\/},
Encyclopedia of Mathematics and Its Applications, Vol.~32, Addison--Wesley
Publishing Company, Reading etc, 1984.

\bibitem{Cole50} J.~D.~Cole, \emph{On a quasi-linear parabolic equation
occuring in aerodynamics\/}, Quart. Appl. Math. \textbf{9} (1951)~\#3,
225--236.

\bibitem{Cor-Sot:Lop:Sua:Sus} R.~Cordero-Soto, R.~M.~Lopez, E.~Suazo and
S.~K.~Suslov, \emph{Propagator of a charged particle with a spin in uniform
magnetic and perpendicular electric fields\/}, Lett.~Math.~Phys. \textbf{84}
(2008)~\#2--3, 159--178.

\bibitem{Cor-Sot:Sua:Sus} R.~Cordero-Soto, E.~Suazo and S.~K.~Suslov, \emph{Models of damped oscillators in quantum mechanics\/}, Journal of Physical
Mathematics, \textbf{1} (2009), S090603 (16 pages).

\bibitem{Cor-Sot:Sua:SusInv} R.~Cordero-Soto, E.~Suazo and S.~K.~Suslov, 
\emph{Quantum integrals of motion for variable quadratic Hamiltonians\/},
Ann. Phys. \textbf{325} (2010)~\#9, 1884--1912; see also arXiv:0912.4900v9
[math-ph] 19 Mar 2010.

\bibitem{Cor-Sot:Sus} R.~Cordero-Soto and S.~K.~Suslov, \emph{Time reversal
for modified oscillators\/}, Theoretical and Mathematical Physics \textbf{162} (2010)~\#3, 286--316; see also arXiv:0808.3149v9 [math-ph] 8~Mar 2009.

\bibitem{Craddock09} M.~Craddock, \emph{Fundamental solutions, transition
densities and the integration of Lie symmetries\/}, J.~Diff.~Eqs. \textbf{207} (2009)~\#6, 2538--2560.

\bibitem{Cor-Sot:SusDPO} R.~Cordero-Soto and S.~K.~Suslov, \emph{The
degenerate parametric oscillator and Ince's equation\/}, J.~Phys. A: Math.
Theor. \textbf{44} (2011)~\#1, 015101 (9 pages); see also\emph{\ }arXiv:1006.3362v3 [math-ph] 2 Jul 2010.

\bibitem{Erd} A.~Erd\'{e}lyi, \textsl{Higher Transcendental Functions\/},
Vols. I--III, A.~Erd\'{e}lyi, ed., McGraw--Hill, 1953.

\bibitem{ErdInt} A.~Erd\'{e}lyi, \textsl{Tables of Integral Transforms\/},
Vols. I--II, A.~Erd\'{e}lyi, ed., McGraw--Hill, 1954.

\bibitem{Friedman64} A.~Friedman, \textsl{Partial Differential Equations of
Parabolic Type\/}, Prentice Hall, Inc., Englewood Cliffs, 1964.

\bibitem{GagWint93} L.~Gagnon and P.~Winternitz, \emph{Symmetry classes of
variable coefficient nonlinear Schr\"{o}dinger equations\/}, J. Phys.~A:
Math. Gen. \textbf{26} (1993), 7061--7076.

\bibitem{Herr-Val:Sus} M.~Herrera-Vald\'{e}z and S.~K.~Suslov, \emph{A
Graphical approach to a model of neuronal tree with variable diameter\/},
arXiv:1101.0296v1 [q-bio.NC] 31 Dec 2010.

\bibitem{Hopf50} E.~Hopf, \emph{Partial differential equation }$u_{t}+uu_{x}=u_{xx},$ Communs. Pure Appl. Math. \textbf{3} (1950)~\#3,
201--230.

\bibitem{JackNobleTsien83} J.~J.~B.~Jack, D.~Noble and R.~W.~Tsien, \textsl{Electric Current Flow in Excitable Cells\/}, Oxford, UK, 1983.

\bibitem{Kadom:Karp71} B.~B.~Kadomtsev and V.~I.~Karpman, \emph{Nonlinear
Waves\/}, Soviet Physics Uspekhi \textbf{14} (1971) \#1, 40--60.

\bibitem{Kamb10} G.~S.~Kambarbaeva, \emph{Some explicit formulas for
calculation of conditional mathematical expectations of random variables and
their applications\/}, Moscow University Math. Bull. \textbf{65} (2010)~\#5,
186--190.

\bibitem{KaratzShreve} I.~Karatzas and S.~Shreve, \textsl{Brownian Motion
and Stochastic Calculus\/}, Second Edition, Grad. Texts in Math., Vol.~113,
Springer-Verlag, 1991.

\bibitem{KudryashovBook10} N.~A.~Kudryashov, \textsl{Methods of Nonlinear
Mathematical Physics\/}, Intellect, Dolgoprudny, 2010 [in Russian].

\bibitem{Kudryash:Sine09} N.~A.~Kudryashov and D.~I.~Sinelshchikov, \emph{A
note on \textquotedblleft New abandant solutions for the Burgers
equation\textquotedblright \/}, arXiv:0912.1542v1 [nlin.SI] 8 Dec 2009.

\bibitem{Lan:Lop:Sus} N.~Lanfear, R.~M.~Lopez and S.~K.~Suslov, \emph{Exact
wave functions for generalized harmonic oscillators\/}, arXiv:11002.5119v1
[math-ph] 24 Feb 2011.

\bibitem{Lan:Sus} N.~Lanfear and S.~K.~Suslov, \emph{The time-dependent Schr\"{o}dinger equation, Riccati equation and Airy functions\/},
arXiv:0903.3608v5 [math-ph] 22 Apr 2009.

\bibitem{Lop:Sus} R.~M.~Lopez and S.~K.~Suslov, \emph{The Cauchy problem for
a forced harmonic oscillator\/}, Revista Mexicana de F\'{\i}sica, \textbf{55}
(2009)~\#2, 195--215; see also arXiv:0707.1902v8 [math-ph] 27 Dec 2007.

\bibitem{Me:Co:Su} M.~Meiler, R.~Cordero-Soto, and S.~K.~Suslov, \emph{Solution of the Cauchy problem for a time-dependent Schr\"{o}dinger
equation\/}, J. Math. Phys. \textbf{49} (2008) \#7, 072102: 1--27; see also
arXiv: 0711.0559v4 [math-ph] 5 Dec 2007.

\bibitem{Miller77} W~Miller, Jr., \textsl{Symmetry and Separation of
Variables\/}, Encyclopedia of Mathematics and Its Applications, Vol.~4,
Addison--Wesley Publishing Company, Reading etc, 1977.

\bibitem{NiPDE} A.~F.~Nikiforov, \textsl{Lectures on Equations and Methods
of Mathematical Physics\/}, Intellect, Dolgoprudnii, 2009 [in Russian].

\bibitem{Ni:Uv} A.~F.~Nikiforov and V.~B.~Uvarov, \textsl{Special Functions
of Mathematical Physics\/}, Birkh\"{a}user, Basel, Boston, 1988.

\bibitem{Ni:Su:Uv} A.~F.~Nikiforov, S.~K.~Suslov, and V.~B.~Uvarov, \textsl{Classical Orthogonal Polynomials of a Discrete Variable\/},
Springer--Verlag, Berlin, New York, 1991.

\bibitem{Reid72} W.~T.~Raid, \textsl{Riccati Differential Equations\/},
Academic Press, New York, 1972.

\bibitem{Risken89} H.~Risken, \textsl{The Fokker--Planck Equation. Methods
of Solution and Applications\/}, Second Edition, Springer--Verlag, New York,
1989.

\bibitem{Rosen76} S.~Rosencrans, \emph{Perturbation algebra of an elliptic
operator\/}, J.~Math. Anal. Appl. \textbf{56} (1976)~\#2, 317--329.

\bibitem{Sach87} P.~L.~Sachdev, \textsl{Nonlinear Diffusive Waves\/},
Cambridge University Press, Cambridge, 1987.

\bibitem{Suaz:Sus} E.~Suazo and S.~K.~Suslov, \emph{Cauchy problem for Schr\"{o}dinger equation with variable quadratic Hamiltonians\/}, under
preparation.

\bibitem{SuazoSusVega10} E.~Suazo, S.~K.~Suslov and J.~M.~Vega-Guzm\'{a}n, 
\emph{The Riccati equation and a diffusion-type equation\/}, New York J.
Math. \textbf{17a} (2011), 225--244.

\bibitem{Suslov10} S.~K.~Suslov, \emph{Dynamical invariants for variable
quadratic Hamiltonians\/}, Physica Scripta \textbf{81} (2010)~\#5, 055006
(11~pp); see also arXiv:1002.0144v6 [math-ph] 11 Mar 2010.

\bibitem{Suslov11} S.~K.~Suslov, \emph{On integrability of nonautonomous
nonlinear Schr\"{o}dinger equations\/}, arXiv:1012.3661v1 [math-ph] 16 Dec
2010.

\bibitem{Wa} G.~N.~Watson, \textsl{A Treatise on the Theory of Bessel
Functions\/}, Second Edition, Cambridge University Press, Cambridge, 1944.

\bibitem{Whitham} G.~B.~Whitham, \textsl{Linear and Nonlinear Waves\/},
Wiley, John \& Sons, New York, 1999.

\bibitem{Wh:Wa} E.~T.~Whittaker and G.~N.~Watson, \textsl{A Course of Modern
Analysis\/}, Fourth Edition, Cambridge University Press, Cambridge, 1927.

\bibitem{Yau04} S.~Yau, \emph{Computation of Fokker--Planck equation\/},
Quart. Appl. Math. \textbf{62} (2004)~\#4, 643--650.
\end{thebibliography}

\end{document}

