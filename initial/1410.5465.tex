
\documentclass[reqno,12pt]{amsart}

\usepackage{latexsym}
\usepackage{amsmath,amsfonts,amssymb,epsfig}
\usepackage{amsthm,amscd}
\usepackage{mathrsfs} 
\usepackage{hyperref}
\usepackage{cancel}
\usepackage{wrapfig}
\usepackage[mathcal]{eucal}
\usepackage{mathtools}
\usepackage{setspace}
\usepackage{accents}
\usepackage{color}

\makeatletter
 
 \let\@texttop\relax
\makeatother

\usepackage[heads=LaTeX]{diagrams}
\diagramstyle[labelstyle=\scriptstyle]

\makeatletter

\makeatother

\setlength{\oddsidemargin}{0in}
\setlength{\evensidemargin}{0in}
\setlength{\topmargin}{-0.25in}

\setlength{\textheight}{8.5in}
\setlength{\textwidth}{6.5in}

\parskip = 0.01in
\linespread{1.1}

\newcounter{mtheorem} 
\newcounter{mcorollary} 

\newtheorem{theorem}{Theorem}
\newtheorem*{theorem*}{Theorem}
\newtheorem{mtheorem}[mtheorem]{Theorem} 
\newtheorem{corollary}[theorem]{Corollary}
\newtheorem{mcorollary}[mcorollary]{Corollary}
\newtheorem{proposition}[theorem]{Proposition}
\newtheorem{lemma}[theorem]{Lemma}
\newtheorem*{klemma*}{Key Lemma}
\newtheorem{conjecture}[theorem]{Conjecture}
\newtheorem{question}[theorem]{Question}

\newtheoremstyle{ourremark}
  {3pt}
  {3pt}
  {}
  {}
  {\bfseries}
  {}
  {.5em}
        
  {}
\theoremstyle{ourremark}
\newtheorem{remark}[theorem]{Remark}
\newtheorem{definition}[theorem]{Definition}
\newtheorem{example}[theorem]{Example}

\newlength{\dhatheight} 

 
 

\graphicspath{{./pict/},{./../pict/}}

\numberwithin{equation}{section} 
\numberwithin{theorem}{section}
 

\title[Borsuk conjecture and homotopy domination]{On the Borsuk conjecture concerning\\
homotopy domination.} 

\date{\today}

\author{R. Komendarczyk}\thanks{The first author acknowledges the support of  DARPA YFA N66001-11-1-4132 and NSF DMS 1043009.}

\author{S. Kwasik}\thanks{The second author acknowledges the support of the Simons Foundation Grant No. 281810}

\address{Department of Mathematics,
Tulane University,
New Orleans, United States } 
\email{rako@tulane.edu, kwasik@tulane.edu}

\author{W. Rosicki}
\address{Faculty of Mathematics,
University of Gdansk, 
Gdansk, Poland} 
\email{wrosicki@mat.ug.edu.pl}

\subjclass[2010]{Primary: 55P55; Secondary: 55P15, 54C56}	 	
\keywords{ANR spaces, homotopy domination, homotopy type.}
 
\begin{document}
\setcounter{section}{0}

\begin{abstract}
 In the seminal monograph {\em Theory of retracts}, Borsuk raised the following question: suppose  two compact ANR's are $h$--equal, i.e. mutually homotopy dominate each other, are they homotopy equivalent? The current paper approaches this question in two ways. On one end, we provide conditions on the fundamental group which guarantee a positive answer to the Borsuk question.  On the other end, we construct various examples of compact $h$--equal, not homotopy equivalent continua, with distinct properties. The first class of these examples has trivial all known algebraic invariants (such as homology, homotopy groups etc.) The second class is given by  $n$--connected  continua, for any $n$, which are infinite $CW$--complexes, and hence ANR's, on a complement of a point.  
\end{abstract}

\maketitle

\section{Introduction}\label{sec:intro}
Given two topological spaces $X$ and $Y$, $X$ is {\em homotopy dominated} by $Y$; denoted by $X\leq_h Y$, if and only if there exist maps $f:X\longrightarrow Y$ and $g:Y\longrightarrow X$,
such that $g\circ f\simeq \text{id}_X$. If $X\leq_h Y$ and $Y\leq_h X$, the spaces $X$ and $Y$ are called $h$--{\em equal}, the latter denoted by $X=_h Y$. In particular if $X$ is homotopy equivalent to $Y$, i.e. $X\simeq Y$, then they are $h$--equal. In the homotopy theory of Borsuk's ANR spaces, c.f. \cite{Borsuk67}, two basic problems are raised. Paraphrasing  Borsuk \cite{Borsuk67}, the first one can be stated as follows:
\begin{quote}
{\em  1) Is every compact ANR space homotopy equivalent to a finite CW-complex?}
\end{quote}
and the second one:
\begin{quote}
{\em 2) Are two $h$--equal compact ANR's homotopy equivalent? In other words, given ANR's $X$ and $Y$, does $X=_h Y$ imply $X\simeq Y$?}
\end{quote}
{\noindent} Both questions become less challenging if the compactness condition is relaxed, then the answer to the first question is positive \cite{Milnor59}, and negative for the second one, \cite{Stewart58}.  {\em Problem 1}, became known as the {\em Borsuk conjecture} and attracted a considerable interest (c.f. \cite{Milnor59, Kirby-Siebenmann77, Quinn79, Bryant-Ferry-Mio-Weinberger96}) which culminated in the positive solution by West in \cite{West77}. In contrast, for the second question surprisingly little progress has been made over the years: \cite{Kwasik84, Kolodziejczyk05, Stewart58}. One of the goals of the current paper is to renew interest in  {\em Problem 2}. 

The paper consists of essentially two parts. In the first part, we make some comments on the role of the fundamental group in {\em Problem 2}. By analogy to Hopfian groups, we define a notion of a {\em Hopfian pair} for $h$--equal spaces and make the following observation
\begin{theorem}\label{thm:hopfian}
A pair of ANRs $X$, $Y$ is a Hopfian pair, if and only if, $X$ and $Y$ are homotopy equivalent.
\end{theorem}
{\noindent} As a consequence, we obtain the following, previously observed by Ko{\l}odziejczyk in \cite{Kolodziejczyk05}. 
\begin{corollary}[\cite{Kolodziejczyk05}]\label{cor:polycyclic}
 Suppose $X$ and $Y$  are $h$--equal ANR's, and such that their fundamental groups are polycyclic-by-finite, then $X$ and $Y$ are  homotopy equivalent.
\end{corollary}
We also make several related observations in the context of Hopfian pairs, Poincar\'e complexes and $H$--spaces.

In the second part, inspired by constructions of \cite{Karimov-Repovs-Rosicki-Zastrow05} and \cite{Stewart58}, we construct  $2$--dimensional continua which are $h$--equal but not homotopy equivalent, see Theorem \ref{thm:WH-WBH}. A basic building block of these examples is a well known ``topological broom'' pictured on Figure \ref{fig:brooms}.
An interesting feature of these constructions is that these spaces have trivial all basic known algebraic invariants, such as singular or \v{C}ech homology groups, homotopy groups etc. Consequently, to prove that the spaces are not homotopy equivalent  requires a more direct, approach via techniques of set theoretic topology. 
Further, in Theorem \ref{thm:WH-WBH},  we provide examples of pairs $\mathcal{S}_0$, $\mathcal{S}_1$ of $2n$--dimensional continua (for $n\geq 2$), modeled on the {\em Hawaiian earrings}, c.f \cite{Eda-Kawamura00},  and  satisfying:    
\begin{itemize}
\item[(a)] $\mathcal{S}_0$, $\mathcal{S}_1$ are {\em singular ANR's}, i.e. for specific points $s_0\in \mathcal{S}_0$, and $s_1\in \mathcal{S}_1$, complements $\mathcal{S}^\circ_0=\mathcal{S}_0-\{s_0\}$, $\mathcal{S}^\circ_1=\mathcal{S}_1-\{s_1\}$ is a countable disjoint sum of connected ANR's.
\item[(b)] $\mathcal{S}_0$, $\mathcal{S}_1$ are $(n-1)$--connected, and each connected component of $\mathcal{S}^\circ_0$ and $\mathcal{S}^\circ_1$ is locally contractible.
\item[(c)] $\mathcal{S}_0=_h\mathcal{S}_1$ but $\mathcal{S}_0\not\simeq\mathcal{S}_1$.
\end{itemize}

\section{On a role of the fundamental group in Borsuk's problem.}\label{sec:pi_1-role}

\subsection{Hopfian pairs} We recall that a finitely presented group $G$ is called {\em Hopfian}, if every epimorphism $h:G\longrightarrow G$ is an isomorphism.
Analogously, given a ring $R$, a finitely generated $R$--module $M$ is {\em Hopfian}, if any module epimorphism $h:M\longrightarrow M$ is an 
isomorphism. Let $X$ and $Y$ be a pair of $h$--equal spaces then, from definition, there are maps 
\begin{equation}\label{eq:XY-dominated}
\begin{split}
 & f:X\longmapsto Y,\quad i:Y\longmapsto X,\quad f\circ i\simeq \text{id}_Y,\\
 & g:Y\longmapsto X, \quad j:X\longmapsto Y, \quad g\circ j\simeq \text{id}_X.
\end{split} 
\end{equation}
In particular it implies that induced homomorphisms $f_\ast$, $g_\ast$ on the fundamental group and homology groups, are epimorphisms.
\begin{definition}\label{def:hopfian}
A pair of spaces $X$, $Y$ is called {\em Hopfian pair}, if and only if $X=_h Y$ and one of the epimorphisms $(g\circ f)_*$ or $(f\circ g)_*$ induced on the fundamental groups and homology modules from maps in \eqref{eq:XY-dominated} is an isomorphism.
\end{definition}
{\noindent} Note that, in the above definition, if one of the epimorphisms is an isomorphism the second one is an isomorphism as well. For convenience, let us restate Theorem \ref{thm:hopfian}:
\begin{theorem*}
A pair of ANRs $X$, $Y$ is a Hopfian pair, if and only if, $X$ and $Y$ are homotopy equivalent.
\end{theorem*}
\begin{proof}
 From the above definition, both compositions
 \[
 \begin{split}
  & g_* f_*=(g\circ f)_*:\pi_1(X)\longrightarrow \pi_1(X),\\
  & f_* g_*=(f\circ g)_*:\pi_1(Y)\longrightarrow \pi_1(Y),
 \end{split}
\]
are isomorphisms. Thus,  $f_*$ and $g_*$ are monomorphisms and consequently they have to be isomorphisms as well. The same reasoning applies to the module homomorphisms $f_*:H_k(X;\Lambda)\longmapsto H_k(Y;\Lambda)$, $g_*:H_k(Y;\Lambda)\longmapsto H_k(X;\Lambda)$, $\Lambda={\mathbb{Z}}[\pi]$. As a consequence, maps $f$ and $g$ induce isomorhisms on $\pi_1$, and all homology with local coefficients, and the Whitehead Theorem implies that $f$ and $g$ are homotopy equivalences. As for the converse, let $f:X\longmapsto Y$ be a homotopy equivalence with the inverse $g:Y\longmapsto X$. Then, $g\circ f\simeq \text{id}_X$ and $f\circ g\simeq \text{id}_Y$, obviously the pair $X$, $Y$ is a Hopfian pair.
\end{proof}
\begin{remark}\label{rem:only-1}
Observe that in the above theorem, it suffices to only have one of the spaces $X$ or $Y$ assumed to be an ANR, then the result of Milnor \cite{Milnor59}, implies that the other space (homotopy dominated by the former) is also an ANR, up to homotopy. 
\end{remark}
Suppose $X$, and $Y$ are ANR's, such that $\pi_1(X)$ and $\pi_1(Y)$ are polycyclic-by-finite, where $X$ is compact, or more generally has finitely generated homology groups $H_k(X)$ for all $k$, allowing nontrivial $H_k(X)$ for arbitrarily large $k$. Polycyclic-by-finite groups are Hopfian, and their group rings are Notherian rings, which implies that modules $H_\ast(X;{\mathbb{Z}}[\pi_1(X)])$ and $H_\ast(Y;{\mathbb{Z}}[\pi_1(Y)])$ are Hopfian. In turn,  if $X$ and $Y$ are $h$--equal they must be a Hopfian pair, implying Corollary \ref{cor:polycyclic}. The very recent preprint of Ko{\l}odziejczyk \cite{Kolodziejczyk14-preprint} contains an extensive list of (fundamental) groups for which the corresponding $h$--equal ANR spaces form Hopfian paris. It substantially expands a class of ANR spaces, for which the Borsuk homotopy domination conjecture has a positive solution. 

The class of Hopfian groups is considerably larger than the polycyclic-by-finite groups. In particular, the following question is a weaker form of {\em Problem 2}.
\begin{question}\label{q:hopfian-pair}
Let $X$, $Y$ be finite CW--complexes (compact ANR's) such that  $X=_h Y$, suppose further $\pi_1(X)$ (and hence $\pi_1(Y)$) is Hopfian. Is $X$, $Y$ a Hopfian pair? 
\end{question}
{\noindent} The following example, guided by the results of \cite{Berridge-Dunwoody79, Harlander-Jensen06}, illustrates a delicate nature of the above question. It is reflected in the fact that if $G=\pi_1(X)\cong \pi_1(Y)$ and $G$ is Hopfian, it is not necessarily the case that $X=_h Y$, even for $2$--dimensional $CW$--complexes $X$ and $Y$.
\begin{example}
  Let $G=\langle x,y\,|\,x^2=y^3\rangle$ be the standard presentation for the fundamental group of the trefoil knot, and 
  let
 \[
  G_i=\langle x,y,\bar{x},\bar{y}\,|\,x^2=y^3,\, \bar{x}^2=\bar{y}^3,\, x^{2i+1}=\bar{x}^{2i+1},\, y^{3i+1}=\bar{y}^{3i+1}\rangle,\qquad i\in \mathbb{N},
  \]
be different presentations of $G$ (c.f. \cite{Berridge-Dunwoody79, Harlander-Jensen06}). For infinitely many $i$, there are $2$--dimensional $CW$--complexes $K_i$ of distinct homotopy type, with $\pi_1(K_i)\cong G_i\cong G$, c.f. \cite{Harlander-Jensen06}. Note that the commutator subgroup $[G,G]$ of $G$ is isomorphic to $F_2$, i.e. free group on two generators, and $G\bigl/[G,G]\cong H_1(G;{\mathbb{Z}})\cong {\mathbb{Z}}$.

 Note that $G$ is Hopfian, since both $[G,G]\cong F_2$ and $G\bigl/ [G,G]\cong {\mathbb{Z}}$ are Hopfian, it is well know that $G$ is  {\em not} polycyclic--by--finite. We claim that there are infinitely many pairs $i$ and $j$, $i\neq j$, such that $K_i\neq_h K_j$. 

 Recall  that it is shown in \cite{Berridge-Dunwoody79, Harlander-Jensen06} that there are infinitely many pairs $i$ and $j$, $i\neq j$, such that $K_i$ is not homotopy equivalent to $K_j$, because $H_2(\widetilde{K}_i;{\mathbb{Z}})$ and $H_2(\widetilde{K}_j;{\mathbb{Z}})$ are not isomorphic as ${\mathbb{Z}}[G]$--modules. More precisely, for some prime number $p$, there are infinitely many distinct $i$ and $j$ such that ${\mathbb{Z}}_p\otimes_{\mathbb{Z}} H_2(\widetilde{K}_i;{\mathbb{Z}})$ has just one generator and  ${\mathbb{Z}}_p\otimes_{\mathbb{Z}} H_2(\widetilde{K}_j;{\mathbb{Z}})$ has at least two generators. 
Suppose $K_i=_h K_j$, for $i\neq j$, where the above holds, then one obtains an epimorphism  $f_\ast:H_2(K_i;{\mathbb{Z}}[G])\longmapsto H_2(K_j;{\mathbb{Z}}[G])$ (see Definition \ref{def:hopfian}), and therefore an obvious epimorphism
\[
 \text{id}\otimes f_\ast:{\mathbb{Z}}_p\otimes_{\mathbb{Z}} H_\ast(K_i;{\mathbb{Z}}[G])\longmapsto {\mathbb{Z}}_p\otimes_{\mathbb{Z}} H_\ast(K_j;{\mathbb{Z}}[G]),
\]
this however contradicts that  $H_\ast(K_i;{\mathbb{Z}}[G])$ has just one generator versus $H_\ast(K_j;{\mathbb{Z}}[G])$ having two generators. Thus by contradiction, we conclude that $K_i\neq_h K_j$ and hence the pair $K_i$ and $K_j$ is not a Hopfian pair. 

 The above considerations lead one to a surprising outcome when one considers spaces $K_i\vee S^2$ and $K_j\vee S^2$ in place of $K_i$ and $K_j$. By work in 
\cite{Berridge-Dunwoody79, Harlander-Jensen06} we know that 
\[
 K_i\vee S^2\simeq K_j\vee S^2,\qquad\text{thus}\qquad K_i\vee S^2=_h K_j\vee S^2.
\]
Note that $G\cong \pi_1(K_i\vee S^2)\cong \pi_1(K_j\vee S^2)$, and the pair $K_i\vee S^2$, $K_j\vee S^2$ is  Hopfian. Indeed the modules $H_2(K_i\vee S^2;{\mathbb{Z}}[G])$, $H_2(K_j\vee S^2;{\mathbb{Z}}[G])$ are Hopfian as both are isomorphic to the free ${\mathbb{Z}}[G]$--module ${\mathbb{Z}}[G]\oplus{\mathbb{Z}}[G]$ (c.f. \cite{Berridge-Dunwoody79}). This shows a difficulty in dealing with modules  $H_\ast(X;{\mathbb{Z}}[G])$ and  $H_\ast(Y;{\mathbb{Z}}[G])$, in the context of Question \ref{q:hopfian-pair}, even if $\pi_1(X)\cong\pi_1(Y)$ is a ``nice'' group.
\end{example}
\subsection{Poincar\'e complexes}  Now, let $M^n$ be a closed $n$--dimensional manifold and $Y$ any space (see Remark \ref{rem:only-1}), such that $M^n=_h Y$, then $M^n$, $Y$ is a Hopfian pair \cite{Berstein-Ganea59, Kwasik84}. More generally, let $X$ be a finite Poincar\'e complex of formal dimension $n$, c.f. \cite{Wall-book99}. To be specific, $X$ has a homotopy type of a finite CW--complex and there exists a class $[X]\in H_n(X;{\mathbb{Z}})$, such that for all $r$ the cap product with $[X]$ induces an isomorphism
\[
 [X]\cap\cdot :H^r(X;\Lambda)\longrightarrow H_{n-r} (X;\Lambda),\qquad \Lambda={\mathbb{Z}}[\pi_1(X)].
\]
If $Y$ is any space, such that $X=_h Y$ then $X$, $Y$ is a Hopfian pair, \cite{Kwasik84}. 
\begin{theorem}
 Suppose $X$ is a homology manifold of formal dimension $n$, i.e. $X$ is a finite dimensional ANR space such that
\[
H_\ast(X,X-\{\text{pt}\})\cong H_\ast({\mathbb{R}}^n,{\mathbb{R}}^n-\{\text{pt}\})=\begin{cases}
{\mathbb{Z}},\quad \ast=n,\\
0,\quad \ast\neq 0.
\end{cases}
\]
 Then $X$ is a finite Poincar\'e complex of formal dimension $n$.
\end{theorem}
 The above theorem is stated without a proof in \cite[p. 5099]{Johnston-Ranicki-00}. It is a well known fact that $X$ satisifies the Poincar\'e duality with integer coefficients, \cite{Borel57}. The only argument we are aware of, that shows $X$ is a Poincar\'e complex, is based on the existence of a spectral sequence for the indentity map $\text{id}_X: X\longmapsto X$ in sheaf homology giving a very general version of Poincar\'e duality in Theorem 9.2 of \cite{Bredon-book97}. It should be noted that if $X$ is {\em polyhedral homology manifold} then a much simpler argument shows that $X$ is a Poincar\'e complex (see Theorem 2.1 in \cite{Wall-book99}).
\begin{corollary}
Let $X$ be a homology manifold of formal dimension $n$ and $Y$ any space with $X=_h Y$, then $X$, $Y$ is a Hopfian pair.
\end{corollary}
Recall, that the well known conjecture asserts that finite dimensional homogeneous ANRs are homology manifolds, \cite{Bryant-Ferry-Mio-Weinberger96}.

Following, \cite{Bredon70}, recall that  $X$ is {\em locally isotopic} if for each path $\lambda:[0,1]\longmapsto X$, there is a neighborhood $N$ of $\lambda(0)$ in $X$ and a map $H:I\times N\longmapsto X$, such that $H(t,\lambda(0))=\lambda(t)$ and such that each $H(t,\,\cdot\,)$ is a homeomorphism of $N$ onto a neighborhood of $\lambda(t)$. Clearly, manifolds are locally isotopic. Suppose $X$ is a compact finite dimensional ANR space which is locally isotopic. By Theorem 4.6 of \cite{Bredon70}, $X$ is a homology manifold of some formal dimension  $n$. Thereore, we obtain
\begin{corollary}
Let $X$ be a compact finite dimensional ANR space which is locally isotopic, and let $Y$ any space such that $X=_h Y$. Then $X$, $Y$ is a Hopfian pair.
\end{corollary}

{\noindent} In the case $X$ admits an $H$--space structure, $\pi_1(X)$ is abelian, in particular polycyclic-by-finite, thus if  $H_\ast(X)$ to be finitely generated in each degree (where we allow the degree to go to infinity), we obtain 
\begin{proposition}
 Let $X$ be an $H$--space, such that $H_k(X)$ is finitely generated for each $k$, and $Y$ any space such that $X=_h Y$. Then $X$, $Y$ is a Hopfian pair.
\end{proposition}
{\noindent} Clearly,  if $X$ is a compact $H$--space the above homological condition holds. Curiously enough, compact $H$--spaces are also Poincar\'e complexes, as can be deduced from the work in \cite{Bauer-et.al04}.

\section{About \texorpdfstring{$h$}{h}--equal but not homotopy equivalent spaces}\label{sec:continua}
Looking for a counterexample to {\em Problem 2}, one may consider the following problem in the combinatorial group theory;  suppose $G$ and $H$, $G\not\cong H$ are two finitely presented groups and retracts of each other, which would make such pair of groups ``strongly'' non--Hopfian.
If both $G$ and $H$ are finite dimensional, i.e. $K(G;1)$ and $K(H;1)$ are finite complexes, then the functoriality of the construction of $K(\pi;1)$--spaces would imply the existence of a counterexample to {\em Problem 2}, namely
\[
 K(G;1)=_h K(H;1),\quad \text{and}\quad K(G;1)\not\simeq K(H;1).
\]
{\noindent} Dropping the requirement for $K(\,\cdot\,;1)$ to be finite complexes, one reduces the problem to the following algebraic question. 
\begin{question}
Find two finitely presented groups  $G$ and $H$, such that $G\not\cong H$ which are retracts of each other.
\end{question}
{\noindent} A positive answer to this question would give counterexamples to {\em Problem 2}, namely $K(G;1)$ and $K(H;1)$. In particular, $K(\,\cdot\,;1)$ have finite dimensional $n$--skeleta for each $n\in \mathbb{N}$. In this sense, such examples would be ``closer'' to a finite dimensional counterexample to {\em Problem 2}.
If one considers a more general class of spaces,  then the answer TO {\em Problem 2} is negative, as first observed by Stewart in \cite{Stewart58}, who provided examples of noncompact ANR's. The remainder of this paper is devoted to a construction of compact examples with particular properties as described in the introduction, Section \ref{sec:intro}.
\subsection{Infinite wedges of ``hairy disks''}
Our example is inspired by constructions of both \cite{Karimov-Repovs-Rosicki-Zastrow05} and \cite{Stewart58}, and based on the ``hairy disk'' as pictured on Figure \ref{fig:hairy-disk}.
First, consider a double broom $\mathcal{B}$ as pictured on Figure \ref{fig:brooms}.  $\mathcal{B}$ is a well known  space which is not contractible but has all trivial known algebraic invariants, such as homology and homotopy groups etc. \cite[p. 295]{Hilton-Wylie-book}. 
Denote the {\em center point} of the broom $\mathcal{B}$ by $v$ and the left (right) sequence of broom's endpoints converging to $v$ by $\{a_n\}$ and $\{b_n\}$ respectively.  Generally, $J_x$, provided it is uniquely determined, will refer to a segment of $\mathcal{B}$ containing $x\in\mathcal{B}$. An exception to this are the following cases:  for $x=v, a_0, b_0$, when we set
\begin{equation}\label{eq:J-arms}
 J_v=[v,a_0]\cup [v,b_0],\quad  J_{a_0}=[v,a_0],\quad  J_{b_0}=[v,b_0].
\end{equation}
In particular
\begin{equation}\label{eq:J_a-J_b-arms}
 J_{a_n}=[a_n, a_0],\quad J_{b_n}=[b_n, b_0].
\end{equation}
{\noindent} Clearly, we may view $\mathcal{B}$ as a wedge product of two pieces $A$ and $B$, containing sequence $\{a_n\}$ and $\{b_n\}$, specifically
\begin{equation}\label{eq:B-union}
 \mathcal{B}=A\vee B,\qquad A=J_{a_0}\cup\bigcup_{n} J_{a_n},\quad  B=J_{b_0}\cup\bigcup_{n} J_{b_n}.
\end{equation}
{\noindent} Later, it will be needed to order points in  $\mathcal{B}$, we always assume the increasing order from bottom to the top, e.g. any $x\in J_{a_n}$ satisfies $a_n\leq x\leq a_0$. In particular if  $x, y\in J_{w}$, and $x\leq y$ with respect to this order, then $[x,y]$ will denote a portion of the segment $J_{w}$ containing all $z$ such that $x\leq z\leq y$.

\begin{figure}[!ht] 
  \centering  
 \includegraphics[width=0.6\textwidth]{broomB-box}
  \caption{Topological broom denoted by $\mathcal{B}$, with the center point $v$. Location of the point $u$ and its neighborhhood $U$, as in the proof of Lemma \ref{lem:g-f-hf}.} \label{fig:brooms} 
\end{figure} 

\begin{definition}\label{def:h.f.}
Given a space $X$, we say $x\in X$ is {\em homotopically fixed in $X$}, if $x$ is homotopically fixed under any homotopy $f_t$, s.t. $f_0=\text{id}_X$, denote
\begin{equation}\label{eq:hf(X)}
 hf(X)=\{x\in X\ |\ x\ \text{is  homotopically fixed in $X$}\}.
\end{equation}
\end{definition}
{\noindent} The set $hf(X)$ is a closed subset of $X$, in particular we have the following fact about $\mathcal{B}$:

\begin{lemma}\label{lem:v-fixed-B}
 Suppose $f:\mathcal{B}\longrightarrow \mathcal{B}$ fixes $v$, i.e. $f(v)=v$, and $v$ is a limit point for both sets: $Z \cap A$ and $Z \cap B$, $Z=f(\mathcal{B})$. Then, any homotopy $f_t:\mathcal{B}\longrightarrow \mathcal{B}$, $f_0=f$ keeps $v$ fixed, i.e. $f_t(v)=v$.
\end{lemma}
\begin{proof}[Sketch of Proof]
 Choose sequences $\{u_n\}$, $u_n\in J_{a_n}$ in $Z\cap A$  and $\{w_n\}$, $w_n\in J_{b_n}$ in $Z\cap B$ respectively, such that
 \[
  u_n\longrightarrow v,\qquad\text{and}\qquad w_n\longrightarrow v\qquad \text{in}\quad \mathcal{B}.
 \]
{\noindent} Let $v_t=f_t(v)$, by continuity, for each $t$:
\[
  f_t(u_n)\longrightarrow v_t,\qquad\text{and}\qquad f_t(w_n)\longrightarrow v_t.
 \]
	{\noindent} But points in $u_n$ can only move up along the arm $J_{u_n}\subset A$ of $\mathcal{B}$ and $v_n$ move up along $J_{w_n}\subset B$, which implies $v_t=v_0$ for all $t$, since $A\cap B=\{v\}$.
\end{proof}

{\noindent} The above lemma is completely analogous to \cite[Lemma 2.3]{Karimov-Repovs-Rosicki-Zastrow05}, where a similar topological broom  is considered\footnote{We choose the broom $\mathcal{B}$ over the one constructed in \cite{Karimov-Repovs-Rosicki-Zastrow05} to simplify some arguments of this section.}.

\begin{corollary}\label{lem:hf(B)}
 $hf(\mathcal{B})=\{v\}$.
\end{corollary}
\begin{proof}
 We already know that $v\in hf(\mathcal{B})$ by Lemma \ref{lem:v-fixed-B}. It is easy to rule out other points in $\mathcal{B}$ as homotopically fixed, with an exception of possibly $a_0$ and $b_0$. Observe however, that $a_0$ and $b_0$ cannot be homotopically fixed as we may construct a homotopy which lets $a_0$ or $b_0$ to ``flow out'' along one of the arms of $\mathcal{B}$, e.g $J_{a_1}$ and  $J_{b_1}$ respectively.
\end{proof}
 Before introducing relevant spaces we make a convenient definition of a {\em wedge product} $\curlyvee$ of spaces $X$ and $Y$ disjointly embedded in ${\mathbb{R}}^N$ (for some $N$). Namely, 
\[
 X\curlyvee_{x,y} Y:= X\cup [x,y]\cup Y,\qquad x\in X,\ y\in Y,
\]
where $[x,y]$ is an arc in ${\mathbb{R}}^N$ connecting points $x$ and $y$ with interior $(x,y)$ disjoint from $X$ and $Y$. 
\begin{figure}[!ht] 
  \centering
   \includegraphics[width=0.33\textwidth]{hdisk} \qquad
  \caption{The ``hairy disk'' $\mathcal{H}$ from \cite[p. 286]{Karimov-Repovs-Rosicki-Zastrow05}, with countably many copies of $\mathcal{B}$ densely attached along the boundary of the unit disk in ${\mathbb{R}}^2$.} \label{fig:hairy-disk} 
\end{figure} 
The disk $\mathcal{H}$ is constructed by attaching brooms $\mathcal{B}$ (Figure \ref{fig:brooms}) along the boundary of the unit disk $D^2$ in ${\mathbb{R}}^2$, as follows (c.f. \cite{Karimov-Repovs-Rosicki-Zastrow05}).  Define 
\begin{equation}\label{eq:M-def}
M =\{m_i\}^{\infty}_{i=1}
\end{equation}
to be a countable dense subset of the boundary of $D^2$. Attach to each
point $m_i$ a copy of $\mathcal{B}$ at the homotopically fixed point $v$, such that the spaces do not intersect each
other and the diameters of these spaces tend to zero as $i\to\infty$. Next, denote by $c$ the center of the disk $D^2$ in $\mathcal{D}$. Following ideas of \cite{Stewart58} we construct infinite wedges of copies $\mathcal{H}(i)$ of $\mathcal{H}$, with center points $c(i)=-\frac{1}{i}$ along the $x$--axis of ${\mathbb{R}}^3$, where each $\mathcal{H}(i)$ is contained in the translated $yz$--plane to $c(i)$, and  and the connecting arcs are segments $[c(i),c(i+1)]$ along the $x$--axis. The countable dense subset $M$ described above is denoted by $M(k)$ for each copy of $\mathcal{H}(k)$ in $\mathcal{WH}^\circ$, and by $M'(k)$ for each copy of $\mathcal{H}'(k)$ in $\mathcal{WBH}^\circ$.
\begin{figure}[!ht] 
  \centering
   \includegraphics[width=0.4\textwidth]{wbh}\quad \includegraphics[width=0.4\textwidth]{wh}
  \caption{Infinite wedge products of hairy disks: $\mathcal{WBH}^\circ$ and $\mathcal{WH}^\circ$ embedded in ${\mathbb{R}}^3$.} \label{fig:WH} 
\end{figure} 
\begin{equation}\label{eq:WH-WBH-o}
\begin{split}
  \mathcal{WH}^\circ & =\mathcal{H}(1)\curlyvee_{c(1),c(2)}\mathcal{H}(2)\curlyvee_{c(2),c(3)}\cdots\curlyvee_{c(k-1),c(k)}\mathcal{H}(k)\curlyvee_{c(k),c(k+1)}\cdots,\\
  \mathcal{WBH}^\circ & =\mathcal{B}\curlyvee_{b,c'(1)} \mathcal{H}'(1)\curlyvee_{c'(1),c'(2)}\cdots\curlyvee_{c'(k-1),c'(k)}\mathcal{H}'(k)\curlyvee_{c'(k),c'(k+1)}\cdots,
\end{split}
\end{equation}
{\noindent} where $b$ is the center point of the $\mathcal{B}$ factor. Further, denote
\begin{equation}\label{eq:WH-WBH}
\mathcal{WH}=\mathcal{WH}^\circ\cup\{(0,0)\},\qquad \mathcal{WBH}=\mathcal{WBH}^\circ\cup\{(0,0)\}.
\end{equation}
{\noindent} Observe that $\mathcal{WH}$ and $\mathcal{WBH}$ are homeomorphic with one--point compactifications of  $\mathcal{WH}^\circ$ and $\mathcal{WBH}^\circ$, respectively. 
 
 First, consider  $\mathcal{WH}^\circ$ and $\mathcal{WBH}^\circ$, we claim that $\mathcal{WH}^\circ=_h\mathcal{WBH}^\circ$. Indeed there exist retractions 
\begin{equation}\label{eq:retractions}
 r_{\mathcal{WH}}:\mathcal{WBH}^\circ\longmapsto \mathcal{WH}^\circ,\qquad r_{\mathcal{WBH}}:\mathcal{WH}^\circ\longmapsto \mathcal{WBH}^\circ,
\end{equation}
 where  $r_{\mathcal{WH}}$ simply projects the $\mathcal{B}$ factor, together this the connecting segment of $v\in\mathcal{B}$ to the center point $c(1)$ of $\mathcal{H}(1)$ in $\mathcal{WH}^\circ$, and equals the identity everywhere else. 
 The retraction $r_{\mathcal{WBH}}$ of $\mathcal{WH}^\circ$ onto $\mathcal{WBH}^\circ$ can be built from the quotient projection $r_{\mathcal{B}}:\mathcal{H}\longmapsto \mathcal{B}_{m_k}$ of the hairy disk $\mathcal{H}$ onto one of the broom factors: $\mathcal{B}_{m_k}\subset \mathcal{H}$ along the boundary.  The projection $r_{\mathcal{B}}$  collapses the complement of that factor to the center point $v=m_k\in\mathcal{B}_{m_k}$. Continuity of this projection is a direct consequence of the construction of $\mathcal{H}$, indeed for any convergent sequence of points $\{h_n\}$ in the complement of $\mathcal{B}_{m_k}$, i.e. $\{h_n\}\subset \mathcal{H}-\mathcal{B}_{m_k}$, if the limit of $\{h_n\}$ is in $\mathcal{B}_{m_k}$ then it equals $m_k$. It follows that, the map
 \[
  r_{\mathcal{B}}:\mathcal{H}\longmapsto \mathcal{B},\qquad r_{\mathcal{B}}(x)=\begin{cases}
   x, & x\in \mathcal{B}_{m_k},\\
   m_k, & x\not\in \mathcal{B}_{m_k}
  \end{cases}
 \] 
 is continuous. The retraction $r_{\mathcal{WBH}}$ can be now defined as equal to $r_{\mathcal{B}}$ on the first factor of $\mathcal{WH}$ and identity on the remaining factors. 
 
The main result of this subsection can be now stated as follows
\begin{theorem}\label{thm:WH-WBH}
 Both pairs: $\mathcal{WH}^\circ$, $\mathcal{WBH}^\circ$ and  $\mathcal{WH}$, $\mathcal{WBH}$ are $h$--equal but not homotopy equivalent.
\end{theorem}
{\noindent} The proof is based on the several lemmas.  

\begin{lemma}\label{lem:hf(WH)-hf(WBH)}
 We have the following, homeomorphisms 
\begin{equation}\label{eq:hf(H_0)-hf(H_1)}
\begin{split}
   hf(\mathcal{WH}^\circ) & \cong \bigsqcup^\infty_{i=1} S^1,\quad
   hf(\mathcal{WBH}^\circ) \cong \{v\}\sqcup \bigsqcup^\infty_{i=1} S^1,\\
   hf(\mathcal{WH}) & \cong \bigsqcup^\infty_{i=1} S^1 \sqcup \{(0,0)\},\quad
   hf(\mathcal{WBH}) \cong \{v\}\sqcup \bigsqcup^\infty_{i=1} S^1 \sqcup \{(0,0)\},
   \end{split}
\end{equation}
where each copy of $S^1$ represents a boundary of the unit disk in each factor $\mathcal{H}$ of $\mathcal{WH}^\circ$ (or $\mathcal{WH}$) and $\mathcal{WBH}^\circ$ (or $\mathcal{WBH}$), and the topology is the subspace topology induced from ${\mathbb{R}}^3$ via the embeddings constructed in \eqref{eq:WH-WBH-o}. In particular, $hf(\mathcal{WH}^\circ)$ ($hf(\mathcal{WH})$) is not homeomorphic to $hf(\mathcal{WBH}^\circ)$ ($hf(\mathcal{WBH})$).
\end{lemma}
\begin{proof}
 By Lemma \ref{lem:hf(B)}, and the construction of the hairy disk $\mathcal{H}$ we have 
 $M\subset hf(\mathcal{H})$, (it is easy to observe that broom centers along $\mathcal{H}$ factors cannot be moved to the interior of the disk $D^2\subset\mathcal{H}$, see also \cite[$(v)$ on p. 288]{Karimov-Repovs-Rosicki-Zastrow05}). Since $M=\{m_k\}$ is dense in the boundary $S^1=\partial D^2$ of $D^2\subset\mathcal{H}$, we conclude that $S^1=\overline{M}\subset hf(\mathcal{H})$. Since, none of the interior points in $D^2$ is homotopically fixed, and for each $\mathcal{B}_{m_k}$--factor of $\mathcal{H}$, $m_k$ is the only homotopically fixed point (by Lemma  \ref{lem:hf(B)}), we have 
 $hf(\mathcal{H})=S^1$.  This immediately implies identities in \eqref{eq:hf(H_0)-hf(H_1)}, note that $\{(0,0)\}$ is a limit point  of  homotopically fixed points from $\mathcal{H}$--factors of 
$\mathcal{WH}$ or $\mathcal{WBH}$.
\end{proof}
{\noindent} Further, we obtain the following key lemma,
\begin{lemma}\label{lem:g-f-hf}
Let $f$ be the homotopy equivalence between $\mathcal{WH}^\circ$,  and $\mathcal{WBH}^\circ$, and $g$ its inverse. Then, 
\begin{equation}\label{eq:g-f-hf}
 f(hf(\mathcal{WH}^\circ))\subset hf(\mathcal{WBH}^\circ),\qquad g(hf(\mathcal{WBH}^\circ))\subset hf(\mathcal{WH}^\circ).
\end{equation}
The same property holds for the compactifications: $\mathcal{WH}$ and $\mathcal{WBH}$.
\end{lemma}
\begin{proof}
We will prove the first inclusion in \eqref{eq:g-f-hf}, as the proof of the second one is fully analogous. Observe that it suffices to prove, for each $k$: 
\begin{equation}\label{eq:f(M(k))-in-hf}
 f(M(k))\subset hf(\mathcal{WBH}^\circ),
\end{equation}
see \eqref{eq:M-def}, then the claim follows from continuity of $f$, and the fact that the closure of $\bigcup_k M(k)$ in $\mathcal{WH}^\circ$ is equal to $hf(\mathcal{WH}^\circ)$ (see Lemma \ref{lem:hf(WH)-hf(WBH)}). (Note that for the second inclusion in \eqref{eq:g-f-hf}, the only difference is that we must add the point $b$ (the center of the first broom factor of 
$\mathcal{WBH}^\circ$) to the union $\bigcup_k M(k)$). 

To prove \eqref{eq:f(M(k))-in-hf}, consider a point $v$ in $M(k)$. By definition it has to be the center point of one of a broom factors of $\mathcal{H}(k)\subset \mathcal{WH}^\circ$, see \eqref{eq:WH-WBH-o}. We  further denote this factor by $\mathcal{B}$, i.e. $v\in \mathcal{B}\subset \mathcal{H}(k)$. Let $u=f(v)$ we  consider two cases: 1$^\circ$, $\mathcal{WBH}^\circ$ is locally path connected at $u$, and 2$^\circ$, $\mathcal{WBH}^\circ$ is not locally path connected at $u$, and $u\not\in hf(\mathcal{WBH}^\circ)$. 
 
{\noindent} {\em Observation \eqref{eq:in-J}:} Since $v\in hf(\mathcal{WH})$, we must have $g\circ f(v)=v$ (as $g\circ f\simeq \text{id}_{\mathcal{WH}}$), consider sequences $a_n\to v$, $b_n\to v$ of points in $\mathcal{B}$ (see Figure \ref{fig:brooms}). Denote by $\tilde{a}_n=g\circ f(a_n)$, $\tilde{b}_n=g\circ f(b_n)$, clearly $\tilde{a}_n\to v$ and $\tilde{b}_n\to v$, we claim that for large $n$, we have 
\begin{equation}\label{eq:in-J}
 \tilde{a}_n\in J_{a_n},\qquad \tilde{b}_n\in J_{b_n}.
\end{equation}
{\noindent} Indeed, denoting the homotopy $g\circ f\simeq \text{id}_{\mathcal{WH}}$ by $h_t=h(t,\,\cdot\,)$, $h:I\times \mathcal{WH}\longrightarrow \mathcal{WH}$ we observe that for every $n$: 
$\gamma_{a_n}(t)=h_t(a_n)$ defines a path in $\mathcal{WH}$ connecting $a_n=\gamma_{a_n}(1)$ and $\tilde{a}_n=\gamma_{a_n}(0)=g\circ f(a_n)$ (analogously for the sequence $\{b_n\}$). Since in the limit $v$, $\gamma_v$ is a constant path, for a small $\varepsilon$--ball $B_v(\varepsilon)$ around $v$, the inverse image $h^{-1}(B_v(\varepsilon))\subset I\times \mathcal{WH}$ contains $I\times \{v\}$ and therefore some small neighborhood $I\times B_v(\delta)$ is also in $h^{-1}(B_v(\varepsilon))$.
For large enough $n$, $a_n$'s are in $B_v(\delta)$ and hence the paths $\gamma_{a_n}$ have image in $B_v(\varepsilon)$. It follows that each  $\gamma_{a_n}$ is contained in the connected component $J_{a_n}\cap B_v(\varepsilon)\subset \mathcal{B}$ of $B_v(\varepsilon)$, and since $J_{a_k}\cap J_{a_j}\cap B_v(\varepsilon)=\emptyset$ for small $\varepsilon$, we obtain the first part of \eqref{eq:in-J}, the second part can be obtained analogously.

{\em Case 1$^\circ$}: suppose $\mathcal{WBH}^\circ$ is locally path connected at $u=f(v)$. Choose a small path connected ball $B_u(\tilde{\varepsilon})$ around $u$, such that $f(B_v(\delta))\subset B_u(\tilde{\varepsilon})$, with $\delta$ as above, and therefore $g(B_u(\tilde{\varepsilon}))\subset B_v(\varepsilon)$, with $\varepsilon$ as above. Since $g(B_u(\tilde{\varepsilon}))$ is connected, and all $\{\tilde{a}_n\}$ for large $n$ are contained in  $g(B_u(\tilde{\varepsilon}))$, $\{\tilde{a}_n\}$ would have to belong entirely to one of the arms $J_{a_k}\cap B_v(\varepsilon)$ of $\mathcal{B}$. But, this leads to a contradiction with \eqref{eq:in-J}.

{\em Case 2$^\circ$:} Suppose $\mathcal{WBH}^\circ$ is not locally connected at $u=f(v)$ and $u\not\in hf(\mathcal{WBH}^\circ)$. 
Then $u$ belongs to one of the broom factors of $\mathcal{WBH}^\circ$, we denote by $\mathcal{B}'$ (note that $\mathcal{B}'$ is either the first factor of $\mathcal{WBH}^\circ$ or belongs to one of the $\mathcal{H}'(k)$ factors). 
We also endow $\mathcal{B}'$ with decorations of Figure \ref{fig:brooms}, i.e. $v'$ will stand for the center of $\mathcal{B}'$, $a'_n$, $b'_n$ will correspond to $a_n$ and $b_n$, etc. 
Note that the set of points where $\mathcal{B}'$ is not locally path connected equals to $V'=J_{v'}-(\{a'_0\}\cup \{b_0\})$. 
By assumption $u\in V'$, since $u\not\in hf(\mathcal{WBH}^\circ)$ we have $u\neq v'$ and without loss of generality, we may assume $u\in J_{a'_0}-\{a'_0\}$. 
Continuity of $f$ implies, $f(a_n)\to u$ and $f(b_n)\to u$, thus, for large $n$, both sequences $\{f(a_n)\}$ and $\{f(b_n)\}$ belong to a small neighborhood $U$ of $u$ consisting of infinitely many disjoint segments accumulating on  $J_{v'}\cap U$ (see Figure \ref{fig:brooms} for the illustration). 
Consider the shortest piece-wise linear paths $\alpha_n:I\longmapsto \mathcal{B}'$, joining $\alpha_n(0)=f(a_n)\in U$ and $\alpha_n(1)=u$; $\beta_n:I\longmapsto \mathcal{B}'$, joining $f(b_n)\in U$ and $u$. 
Clearly, both $\alpha_n$ and $\beta_n$ trace segments respectively:
\[
\alpha_n=[f(a_n),a'_0]\cup [u, a'_0]\subset \mathcal{B}',\qquad \beta_n=[f(b_n),a'_0]\cup [u, a'_0]\subset \mathcal{B}',
\] 
(which we identify with $\alpha_n$, $\beta_n$ abusing the notation slightly). 
In turn, the paths $g\circ \alpha_n$ and $g\circ \beta_n$, join points $\tilde{a}_n=g(f(a_n))$ and $v=g(f(u))$, see Equation \eqref{eq:in-J}. 
Since points $\tilde{a}_n$ (resp. $\tilde{b}_n$) are close to $a_n$ (resp. $b_n$), and belong to $J_{a_n}$ (resp. $J_{b_n}$).
The image of $g\circ \alpha_n$ contains $[\tilde{a}_n,a_0]\cup [v,a_0]$ and the image of  $g\circ \beta_n$ contains segments $[\tilde{b}_n,b_0]\cup [v,b_0]$. Therefore, for  $n$ large enough, we can find $s_n\in \alpha_n$, and $t_n\in \beta_n$, such that 
\begin{equation}\label{eq:s_n-t_n-to}
g(s_n)=a_0,\qquad g(t_n)=b_0.
\end{equation}
Moreover, for each $n$ we can choose minimal such $s_n$ and $t_n$ (i.e. closest to the initial point of the path $\alpha_n$, (resp. $\beta_n$)). 
These ``minimal'' points exist by compactness of $g^{-1}(a_0)$ (resp. $g^{-1}(b_0)$). 
Passing to subsequences, if necessary, we have $s_n\to s$, $t_n\to t$, and both limits belong to $J_{a'_0}$. Clearly,
\begin{equation}\label{eq:g(s)-g(t)}
g(s)=a_0,\qquad g(t)=b_0.
\end{equation}
By \eqref{eq:s_n-t_n-to}, we have  $s\neq t$, and both $s$ and $t$ are above $u$, i.e. $s>u$ and $t>u$, according to the order defined after Equation \eqref{eq:B-union}. 

Suppose $s > t > u$: the initial points $f(a_n)$ of $\alpha_n$  converge to $u$, and  $s_n\in \alpha_n$ (or a subsequence) converges to $s$ as $n\to\infty$.
Since, $s> t$ we can find points $\{e_n\}$, $e_n\in \alpha_n$,
\[
 f(a_n)\leq e_n\leq s_n,
\]
(as points ordered along $\alpha_n$) and such that $e_n\to t$, as a consequence $g(e_n)\to g(t)=b_0$. However, $s_n$, is the first point on $\alpha_n$ mapped to $a_0$ under $g$, by minimality. Further, the initial point of $\alpha_n$, $f(a_n)$ is mapped to $\tilde{a}_n$. By \eqref{eq:in-J}, we conclude that $g(e_n)\in J_{a_n}$ for large enough $n$, and therefore the limit of $\{g(e_n)\}$ has to belong to $J_{a_0}$, contradicting the fact that $b_0\not\in J_{a_0}$. 

In the case $t> s> u$, analogously considering paths $\beta_n$, we may find a sequence of points $\{h_n\}$, converging to $s$, and such that for large $n$:
\[
 f(b_n)\leq h_n\leq t_n.
\]
Then, again points $g(h_n)$ can only accumulate on $J_{b_0}$, contradicting 
\[
g(h_n)\to g(s)=a_0\not\in  J_{b_0}.
\]
For compactifications: $\mathcal{WH}$ and $\mathcal{WBH}$, the claim also follows, because the point at $\infty$, i.e $\{(0,0)\}$, is in the closure of $\bigcup_k M(k)$.  
\end{proof}

\begin{proof}[Proof of Theorem \ref{thm:WH-WBH}]
Given a homotopy equivalence $f:\mathcal{WH}^\circ\longmapsto \mathcal{WBH}^\circ$, and its inverse $g:\mathcal{WBH}^\circ\longmapsto \mathcal{WH}^\circ$ in the notation of Lemma \ref{lem:g-f-hf}, denote their restrictions to the sets of homotopically fixed points, as follows
\[
 \tilde{f}=f\bigl|_{hf(\mathcal{WH}^\circ)},\qquad \tilde{g}=g\bigl|_{hf(\mathcal{WBH}^\circ)}
\] 
By Lemma \ref{lem:g-f-hf}, compositions $\tilde{g}\circ\tilde{f}$ and $\tilde{f}\circ\tilde{g}$ are well defined and by Definition \ref{def:h.f.} they satisfy 
\[
 \tilde{g}\circ\tilde{f}=\text{id}_{hf(\mathcal{WH}^\circ)},\qquad  \tilde{f}\circ\tilde{g}=\text{id}_{hf(\mathcal{WBH}^\circ)}.
\]
Thus $\tilde{f}$, defines a homeomorphism between $\mathcal{WH}^\circ$ and $\mathcal{WBH}^\circ$, and $\tilde{g}$ its inverse, contradicting the statement of Lemma \ref{lem:hf(WH)-hf(WBH)}. For one-point compactifications:  $\mathcal{WH}$ and $\mathcal{WBH}$, points at infinity are homotopically fixed therefore the statement follows analogously.
\end{proof}

\subsection{Infinite wedges of products of \texorpdfstring{$n$}{n}--spheres}
For arbitrarly high connected examples, we follow a similar pattern as in the previous section. Let $\mathcal{S}$ be an $n$--sphere, and $\mathcal{S}^2=\mathcal{S}\times \mathcal{S}$, define the following countable 
wedge products at a common basepoint $s$:
\begin{equation}\label{eq:S_0-S_1}
  \mathcal{S}_0 =\mathcal{S}^2\vee\mathcal{S}^2\vee\cdots\vee\mathcal{S}^2\vee\cdots=\bigvee^\infty_{j=1} \mathcal{S}_0(j),\quad 
 \mathcal{S}_1 =\mathcal{S}\vee \mathcal{S}_0=\bigvee^\infty_{j=1} \mathcal{S}_1(j).
\end{equation}
Thus $\mathcal{S}_0$ and $\mathcal{S}_1$ differ just by the first factor, further we consider both $\mathcal{S}_0$ and $\mathcal{S}_1$ to be metrically embedded in ${\mathbb{R}}^{2 n+2}$, with the basepoint at the origin, and the diameters of factors $\mathcal{S}_\ast(j)$ tending to zero as $j\to\infty$, see Figure \ref{fig:S1-S0}. Both $\mathcal{S}_0$ and $\mathcal{S}_1$ are compact in the topology induced from the embedding. Clearly, $\mathcal{S}_0$ and $\mathcal{S}_1$ are $(n-1)$--connected in this topology since each factor is $(n-1)$--connected. Further, let $\mathcal{S}_\ast$ stand for either $\mathcal{S}_0$ or $\mathcal{S}_1$.

 The obvious retraction $r_{\mathcal{S}}:\mathcal{S}\times \mathcal{S}\longmapsto \mathcal{S}$, can be extended by the identity (and rescailing) to a retraction $r_{\mathcal{S}_1}:\mathcal{S}_0\longmapsto \mathcal{S}_1$. A retraction $r_{\mathcal{S}_0}:\mathcal{S}_1\longmapsto \mathcal{S}_0$ can be obtained by simply collapsing the $\mathcal{S}$ factor of $\mathcal{S}_1$ to the basepoint. Thus we obtain 
\[
\mathcal{S}_0=_h \mathcal{S}_1,\qquad \mathcal{S}_0=_h \mathcal{S}_1.
\]
\begin{figure}[!ht] 
  \centering
   \includegraphics[width=0.4\textwidth]{S0}\quad \includegraphics[width=0.4\textwidth]{S1}
  \caption{Hawaiian earrings $\mathcal{S}_0$(left) and $\mathcal{S}_1$(right) based on $\mathcal{S}\times \mathcal{S}$ and $\mathcal{S}$, for $n=1$.} \label{fig:S1-S0} 
\end{figure} 
The strategy for proving that $\mathcal{S}_0$ and $\mathcal{S}_1$ are not homotopy equivalent is a little different than before and based on 
some homological considerations. 

Both $\mathcal{S}_0$ and $\mathcal{S}_1$ are a special case of the generalized Hawaiian earrings construction considered in \cite{Eda-Kawamura00}.  Following \cite{Eda-Kawamura00} consider the following homomorphisms defined on $\pi_\ast(\mathcal{S}_\ast,s)$ in dimension $n$ (for $n>1$): 
\[
 h_0:\pi_n(\mathcal{S}_0,s)\longmapsto \prod^\infty_{j=1} \pi_n(\mathcal{S}_0(j),s),\qquad h_1:\pi_n(\mathcal{S}_1,s)\longmapsto \prod^\infty_{j=1} \pi_n(\mathcal{S}_1(j),s),
\]
and induced by the product of obvious coordinate retractions $r_{j,\ast}:\mathcal{S}_\ast\longmapsto \mathcal{S}_\ast(j)$ onto each factor of $\mathcal{S}_\ast$. The main theorem of \cite[p. 18]{Eda-Kawamura00} implies that both $h_0$ and $h_1$ are isomorphisms. 
By the Hurewicz Theorem 
\begin{equation}\label{eq:H_n-S_ast}
\begin{split}
 H_n(\mathcal{S}_0) & \cong \pi_n(\mathcal{S}_0,s)\cong \prod^\infty_{j=1} H_n(\mathcal{S}_0(j);{\mathbb{Z}})\cong \prod^\infty_{j=1} ({\mathbb{Z}}\times {\mathbb{Z}})=({\mathbb{Z}}\times {\mathbb{Z}})^\omega\cong {\mathbb{Z}}^\omega,\\
 H_n(\mathcal{S}_1) & \cong \pi_n(\mathcal{S}_1,s)\cong H_n(\mathcal{S};{\mathbb{Z}})\times \prod^\infty_{j=2} H_n(\mathcal{S}_1(j);{\mathbb{Z}})\cong {\mathbb{Z}}\times ({\mathbb{Z}}\times {\mathbb{Z}})^\omega\cong {\mathbb{Z}}^\omega.
\end{split}
\end{equation}
{\noindent} Since both $S_0$ and $S_1$ are $(n-1)$--connected, for $n>1$, the Universal Coefficients Theorem for cohomology and \eqref{eq:H_n-S_ast} implies additively
\begin{equation}\label{eq:H^n-S_ast}
 H^n(\mathcal{S}_\ast)\cong \text{Hom}(H_n(\mathcal{S}_\ast;{\mathbb{Z}});{\mathbb{Z}})\cong \bigoplus^\infty_{j=1} H^n(\mathcal{S}_\ast(j);{\mathbb{Z}})\cong  \bigoplus^\infty_{k=1}{\mathbb{Z}},
\end{equation}
(c.f. \cite[p. 67]{Nunke62} for the second isomorphism). 
\begin{theorem}\label{thm:S_0-S_1}
$\mathcal{S}_0$ and $\mathcal{S}_1$ are $h$--equal but not homotopy equivalent.
\end{theorem} 
\begin{proof}
The $h$--equality has been already argued at the beginning of this section. For the second claim, first we note that the graded ring structures of each factor $H^\ast(\mathcal{S}_\ast(j);{\mathbb{Z}})$ are well known, i.e.
\begin{equation}\label{eq:H^ast-factors}
\begin{split}
 H^\ast(\mathcal{S}_0(j)) & \cong {\mathbb{Z}}[x_j,y_j]\Bigl/\langle x^2_j=0, y^2_j=0\rangle,\\
 H^\ast(\mathcal{S}_1(1)) & \cong {\mathbb{Z}}[w]\Bigl/\langle w^2=0\rangle,\ H^\ast(\mathcal{S}_1(k))  \cong {\mathbb{Z}}[x_k,y_k]\Bigl/\langle x^2_k=0, y^2_k=0\rangle,\ k>1,
\end{split}
\end{equation}
where $x_i$, $y_i$ and $w$ are of degree $n$. Observe that the graded ring $H^\ast(\mathcal{S}_0)$ has the following property: 
\begin{quote}
$(\ast)$ {\em For any nontrival  $p$ in $H^n(\mathcal{S}_0)$  there exists $q$  in $H^n(\mathcal{S}_0)$ such that $p\cdot q\neq 0$.}
\end{quote}
{\noindent} Indeed, from \eqref{eq:H^n-S_ast} any $p\in H^n(\mathcal{S}_0)$ is given as 
\[
 p=\sum^\infty_{i=1} (a_i x_i+b_i y_i),\qquad a_i,b_i\in {\mathbb{Z}},
\]
where only finitely many $a_i$'s and $b_i$'s are nonzero. Let $r:\mathcal{S}_0\longrightarrow \mathcal{S}_0(1)\vee\cdots \vee \mathcal{S}_0(k)$ be a retraction on $k$ first factors of $\mathcal{S}_0$, and $r^\ast:H^\ast(\bigvee^k_{l=1} \mathcal{S}_0(l))\longmapsto H^\ast(\mathcal{S}_0)$ the induced monomorphism. Choosing $k$ large enough, and using the same symbols for the generators of $H^n(\bigvee^k_{l=1} \mathcal{S}_0(l))$ as in \eqref{eq:H^ast-factors}, we have
\[
 p=r^\ast(p')=r^\ast(\sum^k_{i=1} (a_i x_i+b_i y_i)), 
\]
for some $p'\in H^n(\bigvee^k_{l=1} \mathcal{S}_0(l))$. Note that $H^\ast(\bigvee^k_{l=1} \mathcal{S}_0(l))\cong \bigoplus^k_{l=1} H^\ast(\mathcal{S}_0(l))$ as graded rings.
Let $q'=y_i$, for $i$ such that $a_i\neq 0$, then $p'\cdot q'=a_i x_i y_i\neq 0$  and we obtain
\[
p\cdot q=r^\ast(p'\cdot q')\neq 0,
\]
by the injectivity of $r^\ast$. Clearly, the property $(\ast)$ is preserved under the graded ring isomorphisms.
 Note that for $H^\ast(\mathcal{S}_1)$ the property $(\ast)$ does not hold.
Indeed, let $p=w\in H^n(\mathcal{S}_1)$, which generates the cohomology of the first factor of $\mathcal{S}_1$, \eqref{eq:H^ast-factors}. By \eqref{eq:H^n-S_ast}, any $q\in H^n(\mathcal{S}_1)$ is represented by
\[
 q=c_0 w+\sum^\infty_{i=1} (a_i x_i+b_i y_i),\qquad c_0,a_i,b_i\in {\mathbb{Z}},
\]
where only finitely many $a_i$'s and $b_i$'s are nonzero. Again, choosing an appropriate retraction $r$ of $\mathcal{S}_1$ onto finitely many factors,  we have $p=r^\ast(p')$, $q=r^\ast(q')$ for some $p'$ and $q'$ in $H^n(\bigvee^k_{l=1} \mathcal{S}_1(l))$, and therefore 
\[
 p\cdot q=r^\ast(p'\cdot q')=r^\ast(w\cdot q')=0.
\]
We conclude,
\[
 H^\ast(\mathcal{S}_0)\not\cong H^\ast(\mathcal{S}_1),
\]
and consequently $\mathcal{S}_0$ and $\mathcal{S}_1$ cannot be homotopy equivalent. 
\end{proof} 
\begin{remark}
 Modifying slightly the construction of $\mathcal{S}_0$ and $\mathcal{S}_1$, one may consider an infinite compact bouquets $\mathcal{WS}_0$ and $\mathcal{WS}_1$ embedded in ${\mathbb{R}}^{2n+2}$ as pictured on Figure \ref{fig:S1-S0-long}, having the same factors as $\mathcal{S}_0$ and $\mathcal{S}_1$ above.
 \medskip
\begin{figure}[!ht] 
  \centering
   \includegraphics[width=0.45\textwidth]{S0-long-wedge}\qquad \includegraphics[width=0.45\textwidth]{S1-long-wedge}
  \caption{Boquets $\mathcal{WS}_0$(left) and $\mathcal{WS}_1$(right), for $n=1$.} \label{fig:S1-S0-long} 
\end{figure}  
We claim that $\mathcal{WS}_0$ and $\mathcal{WS}_1$ are homotopy equivalent to their respective counterparts $\mathcal{S}_0$ and $\mathcal{S}_1$. Intuitively, we may ``slide'' finitely many consecutive factors of $\mathcal{WS}_\ast$ along paths in each factor to have the common wedge point. Since the diameters of factors of $\mathcal{WS}_\ast$ tend to 
zero at infinity, this process can be continuously extended to the entire $\mathcal{WS}_\ast$, resulting in a copy of $\mathcal{S}_\ast$  (a precise argument is left to the reader). Observe that, contrary to $\mathcal{S}_\ast$, the complement of the infinity point in $\mathcal{WS}_\ast$, denoted by  $\mathcal{WS}^\circ_\ast$ is an $(n-1)$--connected, locally contractible, ANR. Note that if $\mathcal{WS}_\ast$ were locally $2n$--connected, it would imply it was an ANR. In this sense, $\mathcal{WS}_0$ and $\mathcal{WS}_1$ are ``close'' to being ANR spaces. In \cite{Stewart58}, Stewart shows  $\mathcal{WS}^\circ_0$ and $\mathcal{WS}^\circ_1$ for $n=1$ are examples of noncompact  $h$--equal ANRs, which are not homotopy equivalent. The proof of \cite{Stewart58}, relies heavily on the non-triviality of the fundamental group of constructed spaces and on the structure of isomorphisms between free products of groups, \cite{Stewart58}. We wish to point out that the computation of cohomology rings $H^\ast(\mathcal{WS}^\circ_0)$, $H^\ast(\mathcal{WS}^\circ_1)$ is basic and the argument of Theorem \ref{thm:S_0-S_1} can be easily adapted to show $H^\ast(\mathcal{WS}^\circ_0)\not\cong H^\ast(\mathcal{WS}^\circ_1)$, in this case. Implying that $\mathcal{WS}^\circ_0\not\simeq \mathcal{WS}^\circ_1$, moreover, the argument is valid for all $n\geq 1$, giving $2n$--dimensional examples which are $(n-1)$--connected for any $n\geq 1$. 
\end{remark}
\begin{remark}
 In examples, from the last two sections, homotopy dominations are given by retractions. Consequently, these are examples of $r$--equal continua, which are not homotopy equivalent (c.f. \cite{Borsuk67}). 
\end{remark}
\bibliography{domination}
\bibliographystyle{plain}  

\end{document}

