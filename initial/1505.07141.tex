\documentclass{amsart}
\pdfoutput=1
\usepackage{amssymb}
\usepackage{amsmath}
\usepackage{amsthm} 
\usepackage[colorlinks]{hyperref}

\usepackage{booktabs}
\usepackage[all,cmtip]{xy} 

\newtheorem{theorem}{Theorem}
\newtheorem*{question}{Question}

\theoremstyle{remark}
\newtheorem{remark}[theorem]{Remark}

\begin{document}

\title[Surjections of Mordell--Weil groups]
{Optimal quotients and surjections of Mordell--Weil groups}

\author{Everett W. Howe}
\address{Center for Communications Research,
        4320 Westerra Court,
        San Diego, CA 92121-1967, USA.}
\email{however@alumni.caltech.edu}
\urladdr{\href{http://www.alumni.caltech.edu/~however/}
             {http://www.alumni.caltech.edu/{\hbox{\lower0.7ex\hbox{\char`\~}}}{}however/}}

\date{21 May 2015}
\keywords{Jacobian, Mordell--Weil group, automorphism}

\subjclass[2010]{Primary 11G10; Secondary 11G05, 11G30, 11G35}

\begin{abstract}
Answering a question of Ed Schaefer, we show that if $J$ is the Jacobian of a
curve $C$ over a number field, if $s$ is an automorphism of $J$ coming from an 
automorphism of $C$, and if $u$ lies in ${{\mathbf{Z}}}[s]\subseteq\operatorname{End} J$ and has 
connected kernel, then it is not necessarily the case that $u$ gives a 
surjective map from the Mordell--Weil group of $J$ to the Mordell--Weil group
of its image.
\end{abstract}

\maketitle

\section{Introduction}
\label{S:intro}

Let $J$ be the Jacobian of a curve $C$ over a number field. If the automorphism
group $G$ of $J$ is nontrivial, one can use idempotents of the group algebra 
${{\mathbf{Q}}}[G]$ to decompose $J$ (up to isogeny) as a direct sum of abelian 
subvarieties.  This decomposition can be useful, for example, if one would like
to compute the rational points on $C$, because one of the subvarieties may 
satisfy the conditions necessary for Chabauty's method even when $J$ itself
does not.

In this context, Ed Schaefer asked the following question:

\begin{question}
\label{Q:Ed}
Let $C$ be a curve over a number field $k$, let $\sigma$ be a nontrivial
automorphism of $C$, let $s$ be the associated automorphism of the Jacobian $J$
of $C$, and let $u$ be an element of ${{\mathbf{Z}}}[s]\subseteq \operatorname{End} J$. Let 
$A\subseteq J$ be the image of $u$, and suppose the kernel of $u$ is connected.
Is it always true that map of Mordell--Weil groups $J(k)\to A(k)$ induced by
$u$ is surjective\textup{?}
\end{question}

A homomorphism $A\to A'$ of abelian varieties is called \emph{optimal} if its
kernel is connected, so Schaefer's questions asks whether an optimal map coming
from an automorphism necessarily induces a surjection on Mordell--Weil groups.

The purpose of this paper is to show by explicit example that the answer to 
Schaefer's question is `no'. In Section~\ref{S:double} we show that if 
$\varphi\colon C\to E$ is a degree-$2$ map from a genus-$2$ curve to an 
elliptic curve, and if $\sigma$ is the involution of $C$ that fixes $E$, then
the endomorphism $1+s$ of $J$ is optimal and its image is isomorphic to~$E$.
In fact, $1+s$ is isomorphic to the morphism $\varphi_*\colon J\to E$.  To show
that the answer to Question~\ref{Q:Ed} is `no', it therefore suffices to find a 
double cover $\varphi\colon C\to E$ of an elliptic curve by a genus-$2$ curve
such that $\varphi_*$ is not surjective on Mordell--Weil groups.  We provide 
one such an example in Section~\ref{S:example}, and show in 
Section~\ref{S:examples} that there are infinitely many such examples.

\section{Genus-2 double covers of elliptic curves}
\label{S:double}

In this section we review some facts about genus-$2$ double covers of elliptic
curves over an arbitrary field of characteristic not~$2$.  In 
Section~\ref{S:example} we will return to the case where the base field is a 
number field.
 
The general theory of degree-$n$ maps from genus-$2$ curves to elliptic curves
is explained in~\cite{FreyKani1991}. Over the complex numbers, the complete 
two-parameter family of genus-$2$ double covers of elliptic curves was given
in 1832 by Jacobi~(\cite[pp.~416--417]{Jacobi1832}, 
\cite[Volume~I, pp.~380--382]{Jacobi}) as a postscript to his review of 
Legendre's \emph{Trait\'e des fonctions elliptiques}~\cite{Legendre1828};
Legendre had himself given a one-parameter family of genus-$2$ double covers 
of elliptic curves (see Remark~\ref{R:Legendre}, below).
In~\cite[\S{}3.2]{HoweLeprevostEtAl2000}, Jacobi's construction is modified so
that it works rationally over any base field of characteristic not~$2$.

Let $k$ be an arbitrary field of characteristic not~$2$ and let $K$ be a
separable closure of $k$. Suppose we are given equations $y^2 = f$ and
$y^2 = g$ for two elliptic curves $E$ and $F$ over $k$, where $f$ and $g$ are
separable cubics in $k[x]$, and suppose further that we are given an
isomorphism $\psi\colon E[2]\to F[2]$ of group-schemes\footnote{
  This simply means that we are given a Galois-stable
  bijection between the roots of $f$ in $K$ and the 
  roots of $g$ in $K$.}
such that $\psi$ is not the restriction to $E[2]$ of a geometric isomorphism 
$E_K\to F_K$.  Then~\cite[Proposition~4, p.~324]{HoweLeprevostEtAl2000} gives 
an explicit equation for a genus-$2$ curve $C/k$ such that the Jacobian
$J = \operatorname{Jac} C$ is isomorphic to the quotient of $E\times F$ by the graph $G$
of~$\psi$. (We say that $C$ is the curve obtained by \emph{gluing} $E$ to $F$
along their $2$-torsion using $\psi$.) Let $\omega$ be the quotient map from 
$E\times F$ to $J$.  The construction from~\cite{HoweLeprevostEtAl2000} also 
shows that if $\lambda\colon J\to{\widehat}{J}$ is the canonical principal 
polarization on $J$, then there is a diagram
\begin{equation}
\label{eq-polarization}
\xymatrix{
E\times F\ar[rr]^{(2,2)}\ar[d]_{\omega} && E\times F\\ 
J\ar[rr]^\lambda                        && {\widehat}{J}\ar[u]_{{\widehat}{\omega}}.
}
\end{equation}
The automorphism $(1,-1)$ of $E\times F$ fixes $G$ and respects the product
polarization on $E\times F$, so it descends to give an automorphism $s$ of the 
polarized variety $(J,\lambda)$.  By Torelli's theorem, $s$ comes from an
automorphism $\sigma$ of $C$.  Clearly $\sigma$ has order~$2$, and the quotient
of $C$ by the order-$2$ group $\langle \sigma\rangle$ is isomorphic to $E$.  
Let $\varphi\colon C\to E$ be the associated double cover.

Let $u = 1 + s\in \operatorname{End} J$.  Then we have a diagram
\begin{equation}
\label{eq-diagram}
\xymatrix{
E\times F\ar[rr]^{(2,0)}\ar[d]_{\omega} && E\times F\ar[d]^\omega\\ 
J\ar[rr]^{u}                            && J.
}
\end{equation}
We claim that $u$ is optimal.  To see this, note that the kernel of
$\omega\circ (2,0)$ is simply $E[2]\times F$.  The image of $E[2]\times F$ in
$J$ (under the map $\omega$) is equal to the image of $0\times F$ in $J$
because every element of $E[2]\times 0$ is congruent modulo $G$ to an element
of $0\times F[2]$. Also, since $G$ intersects $0\times F$ only in the identity,
the image of $F$ in $J$ is isomorphic to $F$, so the kernel of $u$ is 
isomorphic to $F$.

On the other hand, we see from diagram~\eqref{eq-diagram} that the \emph{image}
of $u$ is equal to the image of $E\times 0$ in $J$.  Arguing analogously to the
preceding paragraph, we see that the image of $u$ is isomorphic to $E$.

Likewise, if we set $v = 1 - s$, then the image of $v$ is isomorphic to $F$.
In fact, from~\cite[\S{}1]{FreyKani1991} we see that the map 
$(u,v)\colon J\to E \times F$ is the same as the map obtained from 
diagram~\eqref{eq-polarization} by starting in the lower left and going to the
upper right. Also, we find that the map $u\colon J\to E$ is isomorphic to 
$\varphi_*\colon J\to E$.   

\begin{remark}
Frey and Kani prove a more general result: Given two elliptic curves $E$ and
$F$ over an algebraically closed field~$k$, an integer $n>1$, and an 
isomorphism $\psi\colon E[n]\to F[n]$ that is an anti-isometry with respect to 
the Weil pairings on $E[n]$ and $F[n]$, there is a possibly-singular curve $C$
over $k$ of arithmetic genus $2$ whose polarized Jacobian $(J,\lambda)$ fits 
into a diagram analogous to \eqref{eq-polarization}, but with $2$ replaced 
with~$n$. The Jacobian $J$ has natural degree-$n$ maps to both $E$ and $F$, and
arguments like the one we gave above show that both of these maps are optimal.
\end{remark}

\begin{remark}
\label{R:Legendre}
Legendre's family of genus-$2$ curve with split Jacobian
\cite[Troisi\`eme Suppl\'ement, \S{}XII, pp.~333--359]{Legendre1828} is the 
family obtained from the construction above by taking $F = E$ and by taking 
$\psi\colon E[2]\to E[2]$ so that it fixes one point of order $2$ and swaps 
the other two.
\end{remark}

\section{A small example}
\label{S:example}

Let $E$ and $F$ be the elliptic curves over ${{\mathbf{Q}}}$ defined by $y^2 = f$ and 
$y^2 = g$, respectively, where
\[
f = x^3 + 5 x^2 + 6 x + 1 
\text{\qquad and\qquad}
g = x^3 - 6 x^2 + 5 x - 1.
\]
Let $K$ be the number field defined by the irreducible polynomial~$f$. If $r$
is one of the roots of $f$ in $K$, then $-r^2 - 4 r - 4$ and $r^2 + 3 r - 1$ 
are also roots of~$f$. Set 
\[
\alpha_1 = r, \qquad 
\alpha_2 = -r^2 - 4 r - 4, \qquad 
\alpha_3 = r^2 + 3 r - 1,
\]
and note that if we set $\beta_i = -1/\alpha_i$ then the $\beta$'s are the 
three roots of~$g$.

Let $\psi\colon E[2]\to F[2]$ be the isomorphism that sends $(\alpha_i,0)$ to 
$(\beta_i,0)$, for $i=1,2,3$. Using the formulas 
from~\cite[Proposition~4, p.~324]{HoweLeprevostEtAl2000}, we see that the
curve $C$ over ${{\mathbf{Q}}}$ defined by $y^2 = 7^8 g(x^2)$ has Jacobian $J$ isomorphic 
to the quotient of $E\times F$ by the graph of~$\psi$. Rescaling $y$, we find
that $C$ has a model 
\[
y^2 = x^6 - 6 x^4 + 5 x^2 - 1.
\]
The double cover $C\to E$ is given by $(x,y)\mapsto (-1/x, y/x^3)$, and the
double cover $C\to F$ by $(x,y) \mapsto (x^2,y)$.

The curve $E$ is isomorphic to the curve 196A1 from Cremona's database; its 
Mordell--Weil group is generated by the point $P = (-2,1)$ of infinite order.
The curve $F$ is isomorphic to the curve 784F1 from Cremona's database, and its
Mordell--Weil group is trivial.

Let $\sigma$ be the involution $(x,y)\mapsto(-x,-y)$ of $C$, so that $\sigma$ 
generates the Galois group of the cover $C\to E$, and let $s$ be the 
corresponding involution of $J$.  We know from Section~\ref{S:double} that the
endomorphism $u = 1 + s$ of $J$ is optimal and has image isomorphic to $E$. We
claim that the point $P$ is not in the image under $u$ of the Mordell--Weil 
group of $J$.

We prove this claim by contradiction.  Suppose there were a point $R$ of 
$J({{\mathbf{Q}}})$ with $u(R) = P$. The only possible image for $R$ in $F({{\mathbf{Q}}})$ is the 
identity element~$O$, so we must have $(u,v)(R) = (P,O)$. 
But~\cite[Proposition~12, p.~338]{HoweLeprevostEtAl2000} gives a necessary and 
sufficient condition for a rational point of $E\times F$ to be the image of a
rational point of $J$. In our situation, this condition boils down to the 
following:

Let $S$ be the subgroup of $K^*$ consisting of elements whose norms to ${{\mathbf{Q}}}$
are nonzero squares, and let $X$ be the quotient group $S/K^{*2}$.  There is a
homomorphism $\iota\colon E({{\mathbf{Q}}})\to X$ defined by sending a nonzero point 
$(x,y)$ to $x - r\in N$; this makes sense, because the norm of $x-r$ is nothing
other than $y^2$. 
Applying~\cite[Proposition~12, p.~338]{HoweLeprevostEtAl2000}, we see that 
$(P,O)$ lies in the image of $J({{\mathbf{Q}}})$ if and only of $\iota(P)$ is the trivial
element of $X$. Since we are assuming that $(P,O)$ lies in the image 
of~$J({{\mathbf{Q}}})$, it must be the case that $\iota(P)$ is trivial; that is, $-2-r$
must be a square in $K$.  But $-2-r$ is \emph{not} a square in $K$; this can be
seen, for example, by looking modulo~$13$. Therefore $P$ is not in the image of
under $u$ of the Mordell--Weil group of $J$.

\section{Infinitely many examples}
\label{S:examples}

The specific example given in Section~\ref{S:example} was chosen because the
equations for the curves and the maps worked out to have small integer 
coefficients.  In this section we present a method for producing infinitely 
many examples, without concerning ourselves about the simplicity of the
equations we obtain.

Let $E$ and $F$ be two elliptic curves over ${{\mathbf{Q}}}$ defined by equations 
$y^2 = f$ and $y^2 = g$, respectively, where $f$ and $g$ are monic cubic
polynomials in ${{\mathbf{Q}}}[x]$ that split completely over~${{\mathbf{Q}}}$. Let $P_1$, $P_2$, 
$P_3$ be the points of order $2$ in $E({{\mathbf{Q}}})$, and let $Q_1$, $Q_2$, $Q_3$ be 
the points of order $2$ in $F({{\mathbf{Q}}})$.  Let $C$ be the genus-$2$ curve over ${{\mathbf{Q}}}$ 
produced by gluing $E$ and $F$ together along their $2$-torsion subgroups using
the isomorphism $\psi\colon E[2]\to F[2]$ that takes $P_i$ to $Q_i$, for 
$i=1,2,3$. Let $\varphi\colon C\to E$ be the degree-$2$ map from $C$ to $E$ 
associated to this data and let $J$ be the Jacobian of~$C$.  Suppose $P$ is a
rational point on $E$.  We know that $P$ is in the image of $J({{\mathbf{Q}}})$ under 
$\varphi_*$ if and only if there is a point $Q$ of $F({{\mathbf{Q}}})$ such that $(P,Q)$ 
is in the image of $J({{\mathbf{Q}}})$ under the map $(u,v)\colon J\to E\times F$ from 
Section~\ref{S:double}. How do we tell whether such a $Q$ exists?

Again~\cite[Proposition~12, p.~338]{HoweLeprevostEtAl2000} provides an answer. 
In this case, because $E$ and $F$ have all of their $2$-torsion points 
rational, the answer takes a slightly different shape. Let $X$ be the subgroup 
of $({{\mathbf{Q}}}^*/{{\mathbf{Q}}}^{*2})^3$ consisting of those triples $(r,s,t)$ whose product is
equal to the trivial element of~${{\mathbf{Q}}}^*/{{\mathbf{Q}}}^{*2}$. As is explained 
in~\cite[\S{}3.7]{HoweLeprevostEtAl2000}, there is a homomorphism $\iota$ from 
$E({{\mathbf{Q}}})/2E({{\mathbf{Q}}})$ to $X$ that takes a non-$2$-torsion point $P$ to the class of 
the triple
\[
(x(P) - x(P_1), x(P) - x(P_2), x(P) - x(P_3)).
\]
To define $\iota(P)$ for a point $P$ of order $2$, we note that for such a $P$ 
two of the three components of the triple above are nonzero; we take $\iota(P)$
to be the unique element of $X$ with the given values in those two components. 
Likewise, there is a homomorphism $\iota'\colon F({{\mathbf{Q}}})/2F({{\mathbf{Q}}})\to X$ defined by
sending a non-$2$-torsion point $Q$ to the class of the triple
\[
(x(P) - x(Q_1), x(P) - x(Q_2), x(P) - x(Q_3)).
\]
Then~\cite[Proposition~12, p.~338]{HoweLeprevostEtAl2000} says that a point 
$(P,Q)$ in $(E\times F)({{\mathbf{Q}}})$ is in the image of $(u,v)$ if and only if
$\iota(P) = \iota'(Q)$.

Suppose we are given an arbitrary elliptic curve $F/{{\mathbf{Q}}}$ with rational points
$Q_1$, $Q_2$, $Q_3$ of order~$2$. We will show that there are infinitely many
geometrically distinct choices for $E/{{\mathbf{Q}}}$ with rational points $P_1$, $P_2$, 
$P_3$ of order~$2$ such that if $\varphi\colon C\to E$ is constructed as 
above, then there is a point of infinite order in $E({{\mathbf{Q}}})$ that is not 
contained in the subgroup of $E({{\mathbf{Q}}})$ generated by the torsion elements and the
image of $J({{\mathbf{Q}}})$ under~$\varphi_*$.

If $x$ is an element of $({{\mathbf{Q}}}^*/{{\mathbf{Q}}}^{*2})^3$, we say that a prime $p$ 
\emph{occurs in $x$} if one of the components of $x$ has odd valuation at~$p$.
If $E$ is an elliptic curve over ${{\mathbf{Q}}}$ with all of its $2$-torsion rational 
over~${{\mathbf{Q}}}$, we say that a prime $p$ \emph{occurs in $E({{\mathbf{Q}}})$} if it occurs in
some element of~$\iota(E({{\mathbf{Q}}}))$. Let $\ell_1$ and $\ell_2$ be two distinct odd 
primes that do not occur in $F({{\mathbf{Q}}})$. Let $p$ be one of the infinitely many odd
primes that do not occur in $F({{\mathbf{Q}}})$ and that are congruent to $\ell_1+1$
modulo $\ell_1^2$ and to $\ell_2-1$ modulo~$\ell_2^2$, and let $E_p$ be the 
elliptic curve
\[
y^2 = x (x + p+1) (x - p + 1).
\]
Let $P_1$, $P_2$, and $P_3$ be the $2$-torsion points on $E_p$ with 
$x$-coordinates $0$, $-p-1$, and $p-1$, respectively, and let 
$P = (-1,p)\in E_p({{\mathbf{Q}}})$.  We compute that the images of these points in 
$X \subset ({{\mathbf{Q}}}^*/{{\mathbf{Q}}}^{*2})^3$ are as follows:
\begin{align*}
\iota(P)   &= (-1, p, -p) \\
\iota(P_1) &= (-p^2+1, p+1, -p+1) \\
\iota(P_2) &= (-p-1, 2p(p+1),-2p) \\
\iota(P_3) &= (p-1,2p,2p(p-1)).
\end{align*}
We see that $p$ occurs in $\iota(P)$, that $p$ occurs in $\iota(P+P_1)$, that
$\ell_2$ occurs in $P+P_2$, and that $\ell_1$ occurs in $P+P_3$.

Note that $\iota(P_1)$, $\iota(P_2)$, and $\iota(P_3)$ are nontrivial, because
either $\ell_1$ or $\ell_2$ occurs in each of them. This shows that none of the
points $P_1$, $P_2$, and $P_3$ is the double of a rational point.  Since we
know the possible torsion structures of elliptic curves 
over~${{\mathbf{Q}}}$~\cite[Theorem~8, p.~35]{Mazur1977}, we see that $E_p$ has torsion
subgroup isomorphic to either $({{\mathbf{Z}}}/2{{\mathbf{Z}}})\times ({{\mathbf{Z}}}/2{{\mathbf{Z}}})$ or
$({{\mathbf{Z}}}/2{{\mathbf{Z}}})\times ({{\mathbf{Z}}}/6{{\mathbf{Z}}})$.  If there is a rational $3$-torsion point $T$
on $E$, then $\iota(T) = (1,1,1)$, because $T$ is twice $-T$. Combined with 
what we have already shown, we find that $\iota(P)$ is not contained in the 
group generated by $\iota'(F({{\mathbf{Q}}}))$ and the image under $\iota$ of the torsion
subgroup of $E({{\mathbf{Q}}})$.  From this, we see both that $P$ does not lie in the 
torsion subgroup of $E({{\mathbf{Q}}})$, and that $P$ is not contained in the subgroup of
$E({{\mathbf{Q}}})$ generated by the torsion elements and the image of $J({{\mathbf{Q}}})$ 
under~$\varphi_*$.

Finally, we note that the $j$-invariant of $E_p$ is given by
\[
j(E_p) = \frac{64 (3 p + 1)^3}{p^2 (p-1)^2 (p+1)^2},
\]
so that, since $p$ is odd, it is the largest prime for which $j(E_p)$ has
negative valuation.  Therefore distinct odd primes $p$ and $q$ give 
geometrically nonisomorphic curves $E_p$ and $E_q$, so there are infinitely 
many curves $E_p$ that we can glue to $F$ as above to get examples showing that
the answer to Schaefer's question is `no'.

\bibliographystyle{hplaindoi} 
\bibliography{optimalMW}

\end{document}

R<x>:=PolynomialRing(Rationals());
f := x^3 + 5*x^2 + 6*x + 1;
g := x^3 - 6*x^2 + 5*x - 1;
K<r>:= NumberField(f);
S<y>:=PolynomialRing(K);
rootsf := [a[1] : a in Roots(f,K)];
alpha1 := r;
alpha2 := -r^2 - 4*r - 4;
alpha3 :=  r^2 + 3*r - 1;
assert Set(rootsf) eq {alpha1,alpha2,alpha3};
beta1 := -1/alpha1;
beta2 := -1/alpha2;
beta3 := -1/alpha3;
rootsg := [a[1] : a in Roots(g,K)];
assert Set(rootsg) eq {beta1,beta2,beta3};

a1 := (alpha3-alpha2)^2/(beta3-beta2) + 
      (alpha2-alpha1)^2/(beta2-beta1) +
      (alpha1-alpha3)^2/(beta1-beta3);
b1 := (beta3-beta2)^2/(alpha3-alpha2) +
      (beta2-beta1)^2/(alpha2-alpha1) +
      (beta1-beta3)^2/(alpha1-alpha3);
a2 := alpha1*(beta3-beta2) +
      alpha2*(beta1-beta3) +
      alpha3*(beta2-beta1);
b2 := beta1*(alpha3-alpha2) + 
      beta2*(alpha1-alpha3) + 
      beta3*(alpha2-alpha1);
A := K!(Discriminant(g)*a1/a2);
B := K!(Discriminant(f)*b1/b2);
h := -(  A*(alpha2-alpha1)*(alpha1-alpha3)*y^2 
       + B*(beta2-beta1)*(beta1-beta3)) * 
      (  A*(alpha3-alpha2)*(alpha2-alpha1)*y^2 
       + B*(beta3-beta2)*(beta2-beta1)) * 
      (  A*(alpha1-alpha3)*(alpha3-alpha2)*y^2 
       + B*(beta1-beta3)*(beta3-beta2));
h := R!h;

E := EllipticCurve(f);
F := EllipticCurve(g);

