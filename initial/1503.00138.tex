
\documentclass{amsart}
\usepackage{amssymb,amsfonts, color,epsf}
\usepackage [cmtip,arrow]{xy}
\xyoption{all}
\usepackage {pb-diagram,pb-xy}
\usepackage{enumerate}
\usepackage{accents}
\usepackage[noautoscale]{youngtab}

\ifx\pdfpageheight\undefined
   \usepackage[dvips,colorlinks=true,linkcolor=blue,citecolor=red,      urlcolor=green]{hyperref}
   \usepackage[dvips]{graphicx}
   \makeatletter
   
   \DeclareGraphicsRule{.mps}{eps}{*}{}
   \makeatother
\else
 
 \usepackage[pdftex]{graphicx}
   \usepackage[bookmarksopen=false,pdftex=true,breaklinks=true,      backref=page,pagebackref=true,plainpages=false,      hyperindex=true,pdfstartview=FitH,colorlinks=true,      pdfpagelabels=true,colorlinks=true,linkcolor=blue,      citecolor=red,urlcolor=green,hypertexnames=false      ]   {hyperref}
\fi
\usepackage{bm}

 

\usepackage{amsmath,amssymb,amsfonts}
\newtheorem{theorem}{Theorem}
\newtheorem{lemma}{Lemma}
\newtheorem{corollary}{Corollary}
\newtheorem{proposition}{Proposition}
\newtheorem{conjecture}{Conjecture}
\newtheorem{hypothesis}{Hypothesis}
\newtheorem{question}{Question}
\newtheorem{notationdefinition}[theorem]{Notation and definition}

\theoremstyle{definition}
\newtheorem{definition}{Definition}
\newtheorem{example}{Example}
\newtheorem{xca}{Exercise}
\newtheorem{notation}{Notation}
\newtheorem{warning}{Warning}

\theoremstyle{remark}
\newtheorem{remark}{Remark}

\definecolor{DarkBlue}{rgb}{0,0.1,0.55}

\numberwithin{equation}{section}

 
 

 \newtheorem{algorithm}{\sc Algorithm}

  
     
                  
                 
                   
     
     
                  
 
 

 

 

\begin{document}
\title[On the isotypic decomposition of cohomology modules]
{
On the isotypic decomposition of cohomology modules of symmetric semi-algebraic sets: polynomial bounds on multiplicities
}
\author{Saugata Basu}
\address{Department of Mathematics,
Purdue University, West Lafayette, IN 47906, U.S.A.}
\email{sbasu@math.purdue.edu}

\author{Cordian Riener}
\address{Aalto Science Institute\\
Aalto University, Espoo\\
Finland}
\email{cordian.riener@aalto.fi}

\subjclass{Primary 14P10, 14P25; Secondary 68W30}
\date{\textbf{\today}}

\keywords{Symmetric group, isotypic decomposition, semi-algebraic sets, Specht modules}
\thanks{Part of this research was performed while the authors were visiting the Institute for Pure and Applied Mathematics (IPAM), which is supported by the National Science Foundation.
Basu was also  partially supported by NSF grants
CCF-1319080 and DMS-1161629. 
 }

\begin{abstract}
We consider symmetric (as well as multi-symmetric) 
real algebraic varieties and semi-algebraic sets, as well
as symmetric complex varieties in affine and projective spaces,
defined by polynomials of fixed degrees. 
We give polynomial (in the dimension of the ambient space) bounds on
the number of irreducible representations of the symmetric group which acts on these sets,
as well as their multiplicities, 
appearing in the isotypic decomposition of their cohomology modules with coefficients in a field
of characteristic $0$.
We also give some applications of our methods in proving lower bounds on the degrees
of defining polynomials of certain symmetric semi-algebraic sets, as well as better bounds on the
Betti numbers of the images under projections of (not necessarily symmetric) bounded
real algebraic sets.

We conjecture that the multiplicities 
of the irreducible representations of the symmetric group  
in the cohomology modules of symmetric semi-algebraic sets defined by polynomials having fixed degrees
are computable with polynomial complexity, which would imply
that the Betti numbers of such sets are also computable with polynomial complexity. This is in contrast with general semi-algebraic sets, for which this problem is provably hard 
($\#\mathbf{P}$-hard). 
We also formulate a question asking whether these multiplicities can be expressed
as a polynomial in the number $k$ of variables for all large enough $k$.
 \end{abstract}

\maketitle
\tableofcontents

\section{Introduction}
\label{sec:intro}
Throughout this paper ${\mathrm{R}}$ will denote a fixed real closed field and ${\mathrm{C}}$ the algebraic closure of ${\mathrm{R}}$. 
We also fix a field ${\mathbb{F}}$ of characteristic $0$.
For any 
closed
semi-algebraic set $S$ we will denote by $b^i(S,{\mathbb{F}})$ the dimension of the $i$-th cohomology group,
${\mbox{\rm H}}^i(S,{\mathbb{F}})$, and by $b(S,{\mathbb{F}}) = \sum_{i \geq 0} b^i(S,{\mathbb{F}})$.
 (We refer the reader to 
\cite[Chapter 6]{BPRbook2} for the definition of homology/cohomology  groups of semi-algebraic
sets defined over arbitrary real closed fields, noting that they are isomorphic to the singular 
homology/cohomology groups in the special case of ${\mathrm{R}} = \mathbb{R}$.)

\subsection{History and motivation}
\label{subsec:history}
The problem of obtaining quantitative bounds on the topology measured by the the Betti numbers 
of real semi-algebraic as well as
complex constructible sets in terms of the degrees and the number of defining polynomials is
very well studied (see for example, \cite{BPR10} for a survey). For semi-algebraic (respectively, constructible) subsets of ${\mathrm{R}}^k$ (respectively, ${\mathrm{C}}^k$) defined by $s$ polynomials of degrees bounded by $d$, these bounds are typically exponential in $k$, and polynomial (for fixed $k$) in $s$ and $d$.

More precisely, suppose that $S$  is a semi-algebraic 
(resp. constructible) subset of ${\mathrm{R}}^k$  (resp. ${\mathrm{C}}^k$) defined by a quantifier-free formula involving
$s$ polynomials in ${\mathrm{R}}[X_1,\ldots,X_k]$ (resp. ${\mathrm{C}}[X_1,\ldots,X_k]$) of degrees bounded by $d$.

\begin{theorem}[Ole{\u\i}nik and Petrovski{\u\i} \cite{OP}, Thom \cite{T}, Milnor \cite{Milnor2}]
\label{thm:classical}
\[
b(S) \leq (s d)^{O(k)}.
\]
\end{theorem}

The single exponential dependence on $k$ of the bound in Theorem \ref{thm:classical} is unavoidable. In the real case it suffices to consider the real variety
\begin{equation}
\label{eg:basic}
V_k = \{-1,1\}^k \subset {\mathrm{R}}^k
\end{equation}
defined by the polynomial
\[
F_k = \sum_{i=1}^k \prod_{i=1}^d (X_i-i)^2.
\]
It is easy to see that $\deg(F_k)  =2d$, and  $b_0(V_k) = d^k$.

In the complex case, it follows from a classical formula of algebraic geometry \cite{Hirzebruch-book}
that the sum of the Betti numbers of a non-singular hypersurface  $V_k \subset {\mathbb{P}}_{\mathrm{C}}^k$
of degree $d$ is asymptotically $\Theta(d)^k$. 
Using a standard excision argument and induction on
dimension, the same
asymptotic estimate on the Betti numbers hold for the affine part of such a variety as well.

The problem of obtaining tighter estimates on the Betti numbers of semi-algebraic sets 
(motivated partly by applications in other areas of mathematics and theoretical computer science)
has been considered by several authors \cite{Basu1,GV07,BPR8}. 
The algorithmic problem of designing efficient
algorithms for computing these invariants has attracted attention as well \cite{BPRbettione,Bas05-first}. Most of this
work has concentrated on the real semi-algebraic case (since using the real structure a complex
constructible subset $S \subset {\mathrm{C}}^K$ can be considered as a real semi-algebraic subset of ${\mathrm{R}}^{2 K}$
defined by twice as many polynomials of the same degrees as those defining $S$), but the complex
case has also being considered separately as well \cite{Scheiblechner07, Walther1}. From the point of
view of algorithmic complexity, the problem of computing the Betti numbers is provably a hard problem -- and so in its full generality a polynomial time algorithm for solving this problem is not to be expected, except in special situations (see \cite{BP'R07joa, Bas05-top} for some of these exceptional
cases). However, even an algorithm with a singly exponential complexity is not known for computing all
the Betti numbers. 

It is a (unproven)  \emph{meta-theorem}  in algorithmic semi-algebraic geometry -- that the worst-case topological complexity of a class of semi-algebraic sets
(measured by the Betti numbers for example)
serve as a rough lower bound for the complexity of algorithms for computing topological
invariants or deciding topological properties of this class of sets.
So the best complexity known for algorithms for determining whether a general semi-algebraic set is empty or connected  is singly exponential, reflecting the singly exponential behavior of the topological
complexity of such sets as exhibited by the example given in Example \ref{eg:basic}. This is true even if the degrees of the
polynomials describing the given set
is bounded by some constant $> 2$). 
On the other hand there are certain classes of semi-algebraic sets where the situation is better.
For example, for semi-algebraic sets defined by few (i.e. any constant number of) quadratic inequalities, we have polynomial upper bounds on the Betti numbers \cite{Bar97}, as well as algorithms 
with polynomial complexities for computing them \cite{BP'R07joa}.

 The problem of proving bounds on Betti numbers of semi-algebraic sets defined by
polynomials of degrees bounded by some constant and admitting an action of 
a product of symmetric groups is considered in \cite{BC2013}. 
(Note that
 the topological structure of varieties (also symmetric spaces) 
admitting actions of Lie groups is a very well-studied
topic (see for example \cite{Mimura-Toda-book}). Here we concentrate on the action of finite reflection groups,
which seems to be a less developed field of study.)

It is intuitively clear that the symmetry imposes strong restrictions on the topology of such sets.
Nevertheless, as shown in Example \ref{eg:basic} below,  the Betti numbers of such sets can be exponentially large.  
However, when the degrees of the defining polynomials are fixed, a polynomial bound is proved on the \emph{equivariant} Betti numbers of such sets in \cite{BC2013}. Moreover, an algorithm with polynomially bounded complexity is
given in \cite{BC2013} for computing the (generalized) Euler-Poincar\'e characteristics of such sets.
Thus, from the point of view of the \emph{meta-theorem} of the previous paragraph, symmetric
semi-algebraic sets pose a dilemma. On one hand their Betti numbers can be exponentially large in the worst case, on the other hand there are reasons to believe that their topological invariants
(when the degree is fixed) has some structure allowing for efficient computation. The polynomial
bound on the equivariant Betti numbers proved in \cite{BC2013} is the first indication of such a
structure.

In this paper, we generalize (as well as sharpen) the results in \cite{BC2013} in several directions. 
The action of the symmetric group $\mathfrak{S}_k$ on a symmetric semi-algebraic subset 
$S \subset {\mathrm{R}}^k$,  induces an action on the cohomology groups ${\mbox{\rm H}}^*(S,{\mathbb{F}})$, making 
${\mbox{\rm H}}^*(S,{\mathbb{F}})$ into a (finite dimensional) $\mathfrak{S}_k$-representation. 

\begin{remark}[Homology versus cohomology]
\label{rem:homology}
Note that since ${\mathbb{F}}$ is a field, ${\mbox{\rm H}}^*(S,{\mathbb{F}}) \cong {\textrm{hom}}({\mbox{\rm H}}_*(S,{\mathbb{F}}),{\mathbb{F}})$ as vector spaces.
Moreover, from the basic property of $\mathfrak{S}_k$ that the conjugacy class of an element equals that of its inverse it follows that for any finite-dimensional representation $W$ of $\mathfrak{S}_k$, ${\textrm{hom}}(W,{\mathbb{F}}) \cong W$ as $\mathfrak{S}_k$-modules.
{}
\end{remark}

The equivariant
cohomology group, ${\mbox{\rm H}}^*_{\mathfrak{S}_k}(S,{\mathbb{F}})$, is isomorphic to the subspace of
${\mbox{\rm H}}^*(S,{\mathbb{F}})^{\mathfrak{S}_k}$ that is fixed by $\mathfrak{S}_k$. Another description of   ${\mbox{\rm H}}^*(S,{\mathbb{F}})^{\mathfrak{S}_k}$ is that it  is the isotypic component of ${\mbox{\rm H}}^*
(S,{\mathbb{F}})$ corresponding to the trivial one-dimensional representation of $\mathfrak{S}_k$, and the
dimension of ${\mbox{\rm H}}^*(S,{\mathbb{F}})^{\mathfrak{S}_k}$ is equal to the multiplicity of the trivial representation
in ${\mbox{\rm H}}^*(S,{\mathbb{F}})$. Thus, the main result of \cite{BC2013} can be expressed as saying that the multiplicity of the trivial representation in   ${\mbox{\rm H}}^*(S,{\mathbb{F}})$ is bounded polynomially. It is well known
(see Section \ref{subsec:representation-of-Sn} below) that the irreducible representations of $\mathfrak{S}_k$ (the so called 
\emph{Specht-modules}) are in correspondence with the finite set of \emph{partitions} of $k$. The number of partitions of $k$ is exponentially large. 

In this paper, we extend the results in 
\cite{BC2013} by proving a polynomial bound on both the number of irreducible representations
appearing with positive multiplicities in ${\mbox{\rm H}}^*(S,{\mathbb{F}})$ as well as on their multiplicities. Some of these
representations can have dimensions which are exponentially large (as is unavoidable since the 
dimension of ${\mbox{\rm H}}^*(S,{\mathbb{F}})$ can be exponentially large as in Example \ref{eg:basic}). 
Thus, we prove that
while the Betti numbers of symmetric semi-algebraic sets can be exponentially large, they can be expressed as a sum of polynomially many numbers (the dimensions of the isotypic components),
and each of these numbers is a product of a multiplicity (which is polynomially bounded) and the
dimension of a Specht module (which can be exponentially large, but efficiently computable due to the hook formula (see Theorem \ref{thm:hook}). 

We also conjecture that 
all these (polynomially bounded) multiplicities of all representations that appear in the
isotypic decomposition of ${\mbox{\rm H}}^*(S,{\mathbb{F}})$,
and hence all the Betti numbers, $b_i(S,{\mathbb{F}})$,
of any given symmetric semi-algebraic set $S$ defined by polynomials having degrees bounded by
some fixed constant, are computable with polynomially bounded complexity. 
See Conjecture \ref{conj:poly} for a more precise statement.

\subsection{Basic notation and definition}
\label{subsec:basic-notation-definition}
In this section we introduce notation and definitions that we will use for the rest of the paper.

\begin{notation}
\label{not:zeros}
  For $P \in {\mathrm{R}} [X_{1} , \ldots ,X_{k}]$ (respectively $P \in {\mathrm{C}} [ X_{1} ,
  \ldots ,X_{k} ]$) we denote by ${{\rm Z}}(P, {\mathrm{R}}^{k})$ (respectively
  ${{\rm Z}} (P, {\mathrm{C}}^{k})$) the set of zeros of $P$ in
  ${\mathrm{R}}^{k}$(respectively ${\mathrm{C}}^{k}$). More generally, for any finite set
  $\mathcal{P} \subset {\mathrm{R}} [ X_{1} , \ldots ,X_{k} ]$ (respectively
  $\mathcal{P} \subset {\mathrm{C}} [ X_{1} , \ldots ,X_{k} ]$), we denote by ${{\rm Z}}
  (\mathcal{P}, {\mathrm{R}}^{k})$ (respectively ${{\rm Z}} (\mathcal{P},
  {\mathrm{C}}^{k})$) the set of common zeros of $\mathcal{P}$ in
  ${\mathrm{R}}^{k}$(respectively ${\mathrm{C}}^{k}$).  For a homogeneous polynomial $P \in {\mathrm{R}} [X_{0} , \ldots ,X_{k-1}]$ (respectively $P \in {\mathrm{C}} [ X_{0} ,
  \ldots ,X_{k-1} ]$)   we denote by ${{\rm Z}}(P, {\mathbb{P}}_{\mathrm{R}}^{k-1})$ (respectively
  ${{\rm Z}} (P, {\mathbb{P}}_{\mathrm{C}}^{k-1})$) the set of zeros of $P$ in
  ${\mathbb{P}}_{\mathrm{R}}^{k-1}$(respectively ${\mathbb{P}}_{\mathrm{C}}^{k-1}$). And, more generally, for any finite set of 
  homogeneous polynomials
  $\mathcal{P} \subset {\mathrm{R}} [ X_{0} , \ldots ,X_{k-1} ]$ (respectively
  $\mathcal{P} \subset {\mathrm{C}} [ X_{0} , \ldots ,X_{k-1} ]$), we denote by ${{\rm Z}}
  (\mathcal{P}, {\mathbb{P}}_{\mathrm{R}}^{k-1})$ (respectively ${{\rm Z}} (\mathcal{P},
  {\mathbb{P}}_{\mathrm{C}}^{k-1})$) the set of common zeros of $\mathcal{P}$ in
  ${\mathbb{P}}_{\mathrm{R}}^{k-1}$ (respectively ${\mathbb{P}}_{\mathrm{C}}^{k-1}$). 
\end{notation}

\begin{notation}
  \label{not:sign-condition} For any finite family of polynomials $\mathcal{P}
  \subset {\mathrm{R}} [ X_{1} , \ldots ,X_{k} ]$, we call an element $\sigma \in \{
  0,1,-1 \}^{\mathcal{P}}$, a \emph{sign condition} on $\mathcal{P}$. For
  any semi-algebraic set $Z \subset {\mathrm{R}}^{k}$, and a sign condition $\sigma \in
  \{ 0,1,-1 \}^{\mathcal{P}}$, we denote by ${{\mathcal R}} (\sigma ,Z)$ the
  semi-algebraic set defined by $$\{ \mathbf{x} \in Z \mid {\mbox{\bf sign}} (P (
  \mathbf{x})) = \sigma (P)  ,P \in \mathcal{P} \},$$ and call it the
  \emph{realization} of $\sigma$ on $Z$. More generally, we call any
  Boolean formula $\Phi$ with atoms, $P \{ =,>,< \} 0, P \in \mathcal{P}$, to
  be a \emph{$\mathcal{P}$-formula}. We call the realization of $\Phi$,
  namely the semi-algebraic set
  \begin{eqnarray*}
    {{\mathcal R}} \left(\Phi , {\mathrm{R}}^{k} \right) & = & \left\{ \mathbf{x} \in {\mathrm{R}}^{k} \mid
    \Phi (\mathbf{x}) \right\}
  \end{eqnarray*}
  a \emph{$\mathcal{P}$-semi-algebraic set}. Finally, we call a Boolean
  formula without negations, and with atoms $P \{\geq, \leq \} 0$, $P\in \mathcal{P}$, to be a 
  \emph{$\mathcal{P}$-closed formula}, and we call
  the realization, ${{\mathcal R}} \left(\Phi , {\mathrm{R}}^{k} \right)$, a \emph{$\mathcal{P}$-closed
  semi-algebraic set}.
\end{notation}

The notion of \emph{partitions} of a given integer will play an important role in what follows,
which necessitates the following notation that we fix for the remainder of the paper.

\begin{notation}[Partitions]
  \label{not:Partition1} 
  We denote by ${\mathrm{Par}}(k)$ the set of \emph{partitions} of $k$, where each partition $\pi \in {\mathrm{Par}}(k)$
  (also denoted $\lambda \vdash k$) 
  is a tuple  $(\pi_{1} , \pi_{2} , \ldots
  , \pi_{\ell})$, with $\pi_{1} \geq \pi_{2} \geq \cdots \geq
  \pi_{\ell} \geq 1$, and $\pi_{1} + \pi_{2} + \cdots + \pi_{\ell} =k$. We
  call $\ell$ the length of the partition $\pi$, and denote
  ${\mathrm{length}}(\pi) = \ell$. 
  {}
  
  
  More generally, for any tuple $\mathbf{k}= (k_{1} , \ldots ,k_{\ell})
  \in {\mathbb{N}}^\ell$, we will denote by
  ${\mathrm{Par}}(\mathbf{k}) = {\mathrm{Par}}(k_{1}) \times \cdots
  \times {\mathrm{Par}}(k_{\ell})$, and for each $\pmb{\pi}= (
  \pi^{(1)} , \ldots , \pi^{(\ell)}) \in
  {\mathrm{Par}}(\mathbf{k})$, we denote by
  ${\mathrm{length}}(\pmb{\pi}) = \sum_{i=1}^{\ell} {\mathrm{length}}(\pi^{(i)})$. 
  We also denote for each $\mathbf{p}= (p_{1} , \ldots , p_{\ell}) \in
  {\mathbb{N}}^{\ell}$,
  \begin{eqnarray*}
    | \mathbf{p} | & = & p_{1} + \cdots + p_{\ell},\\
  
   
   
    F (\mathbf{k},\mathbf{p}) & = & {\mathrm{card}} (\{
    \pmb{\pi}= (\pi^{(1)} , \ldots , \pi^{(\ell)})
    \mid {\mathrm{length}} (\pi^{(i)}) = p_{i} ,  1
    \leq i \leq \ell \}) .
  \end{eqnarray*}
\end{notation}

\begin{notation}[Transpose of a partition and partitions of bounded lengths]
\label{not:Partition2}
For a partition $\lambda =(\lambda_1,\ldots,\lambda_\ell) \vdash k$, we will denote by $\tilde{\lambda}$ the \emph{transpose} of $\lambda$.
More precisely,
$\tilde{\lambda} = (\tilde{\lambda}_1, \ldots,\tilde{\lambda}_{\tilde{\ell}})$, where $\tilde{\lambda}_j = {\mathrm{card}}(\{i \mid \lambda_i \geq j \})$.
For $k,d\geq 0$, we denote 
\[
{\mathrm{Par}}(k,d) := \{ \lambda \in {\mathrm{Par}}(k)  \mid {\mathrm{length}}(\lambda) \leq d \}.
\]
More generally,  for ${\mathbf{k}} = (k_1,\ldots,k_\ell), {\mathbf{d}} = (d_1,\ldots,d_\ell)$ we denote 
\[
{\mathrm{Par}}({\mathbf{k}},{\mathbf{d}}) := \{ \pmb{\lambda} = (\lambda^{(1)},\ldots,\lambda^{(\ell)}) \mid \lambda^{(i)} \in {\mathrm{Par}}(k_i),  {\mathrm{length}}(\lambda^{(i)}) \leq d_i, 1 \leq i \leq \ell \}.
\]
When ${\mathbf{d}} = (d,\ldots,d)$, we will also use ${\mathrm{Par}}({\mathbf{k}},d)$ to denote ${\mathrm{Par}}({\mathbf{k}},{\mathbf{d}})$.
\end{notation}

\begin{notation}[Products of symmetric groups]
\label{not:symmetric-group}
For each $k \in {\mathbb{N}}$, we denote by $\mathfrak{S}_k$ the symmetric group on $k$ letters (or equivalently the Coxeter group $A_{k-1}$).
For ${\mathbf{k}}=(k_1,\ldots,k_\ell) \in {\mathbb{N}}^\ell$ we denote by $\mathfrak{S}_{\mathbf{k}}$ the product group 
$\mathfrak{S}_{k_1} \times \cdots \times \mathfrak{S}_{k_\ell}$, and we will
usually denote $k = |{\mathbf{k}}| = \sum_{i=1}^{\ell} k_i$.
\end{notation}

\begin{definition}[$\mathfrak{S}_{\mathbf{k}}$-symmetric polynomials]
\label{def:action}
Let ${\mathbb{K}}$ be the field ${\mathrm{R}}$ or ${\mathrm{C}}$.
Suppose that ${\mathbf{k}}=(k_1,\ldots,k_\ell), {\mathbf{m}} =(m_1,\ldots,m_\ell) \in {\mathbb{N}}^\ell$, and let 
$P \in {\mathbb{K}}[{\mathbf{X}}^{(1)},\ldots,{\mathbf{X}}^{(\ell)}]$
where for $1 \leq h \leq \ell$, 
${\mathbf{X}}^{(h)} = \left(X^{(h)}_{i,j}\right)_{1\leq i \leq k_h, 1\leq j \leq m_h}$.

The group $\mathfrak{S}_{\mathbf{k}}$, acts on ${\mathbb{K}}[{\mathbf{X}}^{(1)},\ldots,{\mathbf{X}}^{(\ell)}]$
by permuting for each $i,1\leq i \leq \ell$, 
the rows of ${\mathbf{X}}^{(h)}$ by the group $\mathfrak{S}_{k_h}$. For $\pmb{\pi} \in \mathfrak{S}_{\mathbf{k}}$,
and $P \in {\mathbb{K}}[{\mathbf{X}}^{(1)},\ldots,{\mathbf{X}}^{(\ell)}]$, we denote the by $\pmb{\pi}\cdot P$ the image of
$P$ under $\pmb{\pi}$. 
We say that \emph{$P$ is $\mathfrak{S}_{\mathbf{k}}$-symmetric} if it is invariant under the action of 
$\mathfrak{S}_{\mathbf{k}}$, i.e. if $\pmb{\pi}\cdot P = P$ for every $\pmb{\pi} \in \mathfrak{S}_{\mathbf{k}}$.
{}
Similarly, we say that a 
subset $S \subset {\mathbb{K}}^K, K = \sum_{1\leq i \leq \ell} k_i m_i$,
is $\mathfrak{S}_{\mathbf{k}}$-symmetric if it is stable under the above action of $\mathfrak{S}_{\mathbf{k}}$. 

When $\ell=1,m_1=1$, and $K= k_1m_1 = k$, the action defined above is the usual action of $\mathfrak{S}_k$ on ${\mathbb{K}}^k$
permuting coordinates. 
\end{definition}

\begin{remark}
\label{rem:complex-action}
Note in case ${\mathbb{K}} = {\mathrm{C}}$, the action of $\mathfrak{S}_{\mathbf{k}}$ on ${\mathrm{C}}^K$ defined above in Definition
\ref{def:action}  can also be seen as the action of $\mathfrak{S}_{\mathbf{k}}$ on ${\mathrm{R}}^{2 K}$ (considering ${\mathrm{C}} = {\mathrm{R}} \oplus  i{\mathrm{R}}$), replacing ${\mathbf{m}}$ by $2{\mathbf{m}}$.
{}
\end{remark}

\subsection{Basic example}
\label{subsec:basic-example}
Before proceeding further, we discuss an example which is our guiding example for the rest of the paper. While explaining the example we will need to
refer to certain classical notions from the representation theory of symmetric groups, which we recall later in the paper 
(and which the reader can refer to if needed).

\begin{example}[Real affine case]
\label{eg:basic}
Let 
\[
F_k = \sum_{i=1}^k X_i^2(X_i -1)^2 - {{\varepsilon}},
\]
and 
\begin{equation}
\label{eqn:eg:basic}
V_k = {{\rm Z}}(F_k,{\mathrm{R}}^k).
\end{equation}

Then, for all  ${{\varepsilon}}, 0 < {{\varepsilon}} \ll 1$,
$V_k$ is a compact non-singular hypersurface in ${\mathrm{R}}^k$, (in fact also in ${\mathbb{P}}_{\mathrm{R}}^k$), 
the semi-algebraic set $S_k$ defined by $F_k \leq 0$ is homotopy equivalent to the finite set 
of points $\{0,1\}^k$, and is bounded by $V_k$.

Clearly, $b_0(V_k,{\mathbb{F}}) = 2^k$, and 
it follows from Poincar\'e duality applied to $V_k$  that $b_{k-1}(V_k,{\mathbb{F}}) = 2^k$ as well.
It also follows from Alexander-Lefshetz duality that ${\mbox{\rm H}}^i(V_k,{\mathbb{F}})=0$ for $0< i < k-1$.
 
The real algebraic variety $V_k$ is symmetric under the standard action of the symmetric group $\mathfrak{S}_k$ on ${\mathrm{R}}^k$ 
permuting the coordinates. This action induces an $\mathfrak{S}_k$-module structure on  ${\mbox{\rm H}}^*(V_k,{\mathbb{F}})$,
and it is interesting to study the isotypic decomposition of this representation into its isotypic components corresponding to the
various irreducible representations of $\mathfrak{S}_k$, namely the Specht modules $\mathbb{S}^\lambda$ indexed by
different partitions $\lambda\vdash k$ (see for example \cite{Procesi-book} for the definition of Specht modules).

We now describe this decomposition.

Clearly, 
\[
{\mbox{\rm H}}^0(V_k,{\mathbb{F}})  \cong \bigoplus_{0 \leq i \leq k} {\mbox{\rm H}}^0(V_{k,i},{\mathbb{F}}), 
\] 
where for $0 \leq i \leq k$, $V_{k,i}$ is the $\mathfrak{S}_k$-orbit of the 
connected component of $V_k$ infinitesimally close (as a function of ${{\varepsilon}}$)  to the
point ${\mathbf{x}}^i = (\underbrace{0,\ldots,0}_i,\underbrace{1,\ldots,1}_{k-i} )$,
and ${\mbox{\rm H}}^0(V_{k,i},{\mathbb{F}})$ is a sub-representation of ${\mbox{\rm H}}^0(V_k,F)$.
  
It is also clear that the isotropy subgroup of 
the class in ${\mbox{\rm H}}^0(V_k,{\mathbb{F}})$ corresponding to $V_{k,i}$
is isomorphic to $\mathfrak{S}_i \times \mathfrak{S}_{k-i}$, and
hence,
\begin{eqnarray*}
{\mbox{\rm H}}^0(V_{k,i},{\mathbb{F}}) &\cong& {\mathrm{Ind}}_{\mathfrak{S}_i \times \mathfrak{S}_{k-i}}^{\mathfrak{S}_k}( \mathbb{S}^{(i)} \boxtimes \mathbb{S}^{(k-i)}) \\
&\cong& M^{(i,k-i)} \mbox{ if }  i \geq k-i,\\
&\cong & M^{(k-i,i)} \mbox{ otherwise}.  
\end{eqnarray*}

where for any $\lambda \vdash k$, we denote by $M^\lambda$ the Young module corresponding to
$\lambda$ (see Definition \ref{def:Young}).

 

Also, observe that ${\mbox{\rm H}}^0(V_{k,i},{\mathbb{F}})$ and ${\mbox{\rm H}}^0(V_{k,k-i},{\mathbb{F}})$ are isomorphic  as $\mathfrak{S}_k$-modules.  
In the following, for partitions $\mu,\lambda \vdash k$, we will denote by $K(\mu,\lambda)$ the corresponding
\emph{Kostka number} (see Definition \ref{def:Kostka} below). For this example, it is sufficient to observe that
if $\mu {{\;\underline{\triangleright}\;}} \lambda$ (see Definition \ref{def:dominance} for the definition of the \emph{dominance order} ${{\;\underline{\triangleright}\;}}$ on the set of
partitions), and if $\mu$ has at most $2$ rows, then $K(\mu,\lambda) = 1$.
It now follows from 
Proposition \ref{prop:Young}
that for $k$ odd,
\begin{eqnarray*}
{\mbox{\rm H}}^0(V_k,{\mathbb{F}}) &\cong & \bigoplus_{\substack{\lambda \vdash k\\ \ell(\lambda) \leq 2}}  (M^\lambda \oplus M^\lambda) \\
& \cong & \bigoplus_{\substack{\lambda \vdash k\\ \ell(\lambda) \leq 2}} \bigoplus_{\mu {{\;\underline{\triangleright}\;}} \lambda} 2 K(\mu,\lambda)  \mathbb{S}^\mu\\
& \cong & \bigoplus_{\substack{\lambda \vdash k\\ \ell(\lambda) \leq 2}} \bigoplus_{\mu {{\;\underline{\triangleright}\;}} \lambda} 2 \mathbb{S}^\mu\\
& \cong & \bigoplus_{\substack{\mu \vdash k\\ \ell(\mu) \leq 2}} m_\mu \mathbb{S}^\mu,
\end{eqnarray*}
where  for each $\mu = (\mu_1,\mu_2)  \vdash k$,   
\begin{eqnarray}
m_\mu &=& 2(\mu_1 - \lfloor k/2 \rfloor)  \nonumber\\
            &=&  2\mu_1 - k +1. \label{eqn:odd}
\end{eqnarray}

For $k$ even we have,
\begin{eqnarray*}
{\mbox{\rm H}}^0(V_k,{\mathbb{F}}) &\cong & \left(\bigoplus_{\substack{\lambda \vdash k\\ \ell(\lambda) \leq 2 \\ \lambda \neq (k/2,k/2)}}  (M^\lambda \oplus M^\lambda) \right)  \bigoplus M^{(k/2,k/2)}  \\
& \cong & \left(
\bigoplus_{\substack{\lambda \vdash k \\ \ell(\lambda) \leq 2 \\ \lambda \neq (k/2,k/2)}} \bigoplus_{\mu {{\;\underline{\triangleright}\;}} \lambda} 2 K(\mu,\lambda)  \mathbb{S}^\mu 
\right) \oplus
 \left(
 \bigoplus_{\mu {{\;\underline{\triangleright}\;}} (k/2,k/2)} K(\mu,(k/2,k/2))\mathbb{S}^\mu
 \right)\\
& \cong & \bigoplus_{\substack{\mu \vdash k \\ \ell(\mu) \leq 2}} m_\mu \mathbb{S}^\mu,
\end{eqnarray*}
where for each $\mu = (\mu_1,\mu_2)  \vdash k$,  
\begin{eqnarray}
m_\mu &=& 2(\mu_1 - {k/2})+1  \nonumber\\
	     &=& 2\mu_1 -k +1. \label{eqn:odd}
\end{eqnarray}

We deduce for all $k$, 
\begin{eqnarray}
m_\mu &=& 2\mu_1 -k +1 \label{eqn:even-and-odd} \\
            &\leq & k+1. \nonumber
\end{eqnarray}

For $\mu = (\mu_1,\mu_2) \vdash k$, by the hook-length formula (Theorem \ref{thm:hook}) 
we have,
\begin{eqnarray}
\label{eqn:hook-length}
\dim \; \mathbb{S}^\mu &=& \frac{k! \; (\mu_1 - \mu_2+1)}{(\mu_1+1)!\mu_2!}.
\end{eqnarray}

This completes the description of the isotypic decomposition of ${\mbox{\rm H}}^0(V_k,{\mathbb{F}})$. 

In particular for $k=2,3$ we have:
\begin{eqnarray*}
{\mbox{\rm H}}^0(V_2,{\mathbb{F}}) &\cong &  3\mathbb{S}^{(2)} \oplus \mathbb{S}^{(1,1)}, \\
{\mbox{\rm H}}^0(V_3,{\mathbb{F}}) &\cong &  4\mathbb{S}^{(3)} \oplus 2\mathbb{S}^{(2,1)}.
\end{eqnarray*}

The isotypic decomposition of ${\mbox{\rm H}}^{k-1}(V_k,{\mathbb{F}})$ requires one further ingredient -- namely, an
$\mathfrak{S}_k$-equivariant version of the classical Poincar\'e duality theorem for oriented manifolds. We include 
a proof of this result (Theorem \ref{thm:poincare-duality}) in  Section \ref{subsec:Poincare-duality}.

We note that $V_k$ is a compact real orientable manifold, 
by Poincar\'e duality theorem there exists an isomorphism between ${\mbox{\rm H}}^0(V,{\mathbb{F}})$ and ${\mbox{\rm H}}^{k-1}(V,{\mathbb{F}})$. This
isomorphism is not necessarily a $\mathfrak{S}_k$-module isomorphism. However, it follows from  
Theorem \ref{thm:poincare-duality} (which is a stronger form of Poincar\'e duality for orientable symmetric manifolds) 
that the isotypic representation
of ${\mbox{\rm H}}^{k-1}(V_k,{\mathbb{F}})$ is isomorphic (as an $\mathfrak{S}_k$-module) to 
${\mbox{\rm H}}^0(V_k,{\mathbb{F}}) \otimes
\textbf{sign}_{k}
$.

Thus, denoting for each $\lambda \vdash k$, the \emph{transpose} of the partition $\lambda$ by $\tilde{\lambda}$, 

\begin{eqnarray*}
{\mbox{\rm H}}^{k-1}(V_k,{\mathbb{F}}) &\cong &  \bigoplus_{\substack{\mu \vdash k\\ \ell(\mu) \leq 2}} m_\mu \mathbb{S}^{\tilde{\mu}},
\end{eqnarray*}
where  for  each $\mu = (\mu_1,\mu_2)  \vdash k$,  
$m_\mu$ is defined above in \eqref{eqn:even-and-odd}.
In particular for $k=2,3$ we have:
\begin{eqnarray*}
{\mbox{\rm H}}^1(V_2,{\mathbb{F}}) &\cong &  3\mathbb{S}^{(1,1)} \oplus \mathbb{S}^{(2)}, \\
{\mbox{\rm H}}^2(V_3,{\mathbb{F}}) &\cong &  4\mathbb{S}^{(1,1,1)} \oplus 2\mathbb{S}^{(2,1)}.
\end{eqnarray*}

Notice that the multiplicity $m_{1^k}$ of the Specht module 
$\mathbb{S}^{1^k} = \textbf{sign}_{k}$ in
${\mbox{\rm H}}^0(V_k,{\mathbb{F}})$ is equal to $0$ for $k>2$. This implies that the multiplicity of the trivial
representation $\mathbb{S}^{(k)}$ is equal to $0$ in ${\mbox{\rm H}}^{k-1}(V_k,{\mathbb{F}})$, and thus
${\mbox{\rm H}}^{k-1}_{\mathfrak{S}_k}(V_k,{\mathbb{F}})=0$ as well (for $k >2$).

 Also, notice that  the multiplicity of each Specht-module, $\mathbb{S}^\mu, \mu \vdash k$, 
in the isotypic decomposition of ${\mbox{\rm H}}^*(V_k,{\mathbb{F}})$ is bounded
polynomially (in fact, linearly) in $k$, but the dimension of ${\mbox{\rm H}}^*(V_k,{\mathbb{F}})$ itself  is exponentially large in $k$.  

As an aside note that  since $\dim {\mbox{\rm H}}^0(V_k,{\mathbb{F}}) =  2^k$, we obtain as a consequence  (from \eqref{eqn:even-and-odd} and 
 \eqref{eqn:hook-length}) the somewhat interesting identity
 \begin{eqnarray*}
  k!\;\left(\sum_{\substack{
  \mu_1 \geq \mu_2\geq 0\\ \mu_1+\mu_2 =k} }  \frac{(\mu_1 - \mu_2 +1)^2}{(\mu_1+1)!\mu_2!}\right)
   &=& 2^k.
  \end{eqnarray*}
\end{example}

\begin{example}[Projective case]
\label{eg:real-projective}
Let 
\[
P = \sum_{0\leq i< j \leq k-1} (X_i^2 - X_j^2)^2, 
\]
and let $W_k = {{\rm Z}}(P,{\mathbb{P}}_{\mathrm{R}}^{k-1})$.
Then,
\[
W_k = \{(x_0:\cdots:x_{k-1}) \mid x_i = \pm 1, 0 \leq i \leq k-1\},
\]
and is symmetric under the action of $\mathfrak{S}_{k}$ on ${\mathbb{P}}_{\mathrm{R}}^{k-1}$ permuting the homogeneous coordinates.

It is clear that 
\[
{\mbox{\rm H}}^0(W_k,{\mathbb{F}}) \cong{\mbox{\rm H}}^0(V_k,{\mathbb{F}}),
\]
where $V_k$ is the real affine variety defined in \eqref{eqn:eg:basic}, and the stated isomorphism is
an isomorphism of $\mathfrak{S}_k$-modules. 
\end{example}

\subsection{Equivariant cohomology}
We recall also the definition of \emph{equivariant cohomology groups} of a
$G$-space for an arbitrary compact Lie group $G$. For $G$ any compact Lie
group, there exists a \emph{universal principal $G$-space}, denoted $E G$,
which is contractible, and on which the group $G$ acts freely on the right. 
The \emph{classifying space} $B G$, is the orbit space of this action, i.e.
$B G= E G/G$.

\begin{definition}
  \label{def:equivariant-cohomology} (Borel construction) Let $X$ be a space
  on which the group $G$ acts on the left. Then, $G$ acts diagonally on the
  space $E G \times X$ by $g (z,x) = (z \cdot g^{-1} ,g \cdot x)$. For any
  field of coefficients ${\mathbb{F}}$, the \emph{$G$-equivariant cohomology
  groups of $X$} with coefficients in ${\mathbb{F}}$, denoted by
  ${\mbox{\rm H}}^{\ast}_{G} (X,{\mathbb{F}})$, is defined by
${\mbox{\rm H}}^{\ast}_{G} (X,{\mathbb{F}})  =  {\mbox{\rm H}}^{\ast} (E G \times X/G,{\mathbb{F}})$.
\end{definition}

It is well known (see for example {\cite{Quillen71}}) that when a 
finite  group $G$ acts on a topological space $X$, 
and
${\mathrm{card}}(G)$ is invertible in ${\mathbb{F}}$ (and so in particular, if ${\mathbb{F}}$ is a field of characteristic $0$)
that there is an isomorphism
\begin{equation}
\label{eqn:iso1}
{\mbox{\rm H}}^{\ast} (X/G,{\mathbb{F}}) \xrightarrow{\sim} {\mbox{\rm H}}_{G}^{\ast} (X,{\mathbb{F}}).
\end{equation}

The action of $G$ on $X$ induces an action of  $G$ on the 
cohomology ring ${\mbox{\rm H}}^*(X,{\mathbb{F}})$, and we denote the subspace of  ${\mbox{\rm H}}^*(X,{\mathbb{F}})$ fixed by this action
${\mbox{\rm H}}^*(X,{\mathbb{F}})^G$. 
When $G$ is finite and ${\mathrm{card}}(G)$ is invertible in ${\mathbb{F}}$, 
the Serre spectral sequence associated to the map $X\rightarrow X/G$, degenerates at its $E_2$-term, 
where 
\[
E_2^{p,q} = {\mbox{\rm H}}^p(G,{\mbox{\rm H}}^q(X,{\mathbb{F}}))
\]
(see for example \cite[Section VII.7]{Brown-book}).
This is due to the fact that  ${\mbox{\rm H}}^p(G,{\mbox{\rm H}}^q(X,{\mathbb{F}}))= 0$ for all $p > 0$, 
which implies that 
\begin{equation}
\label{eqn:iso2}
{\mbox{\rm H}}^n(X/G,{\mathbb{F}}) \cong  {\mbox{\rm H}}^0(G,{\mbox{\rm H}}^q(X,{\mathbb{F}})), 
\end{equation}
for each $q \geq 0$.
Finally, 
\begin{equation}
\label{eqn:iso3}
{\mbox{\rm H}}^0(G,{\mbox{\rm H}}^\ast(X,{\mathbb{F}})) \cong {\mbox{\rm H}}^\ast(X,{\mathbb{F}})^G.
\end{equation}

Thus, combining \eqref{eqn:iso1}, \eqref{eqn:iso2} and \eqref{eqn:iso3} we have the isomorphisms
\begin{equation}
\label{eqn:iso}
{\mbox{\rm H}}^{\ast} (X/G,{\mathbb{F}}) \xrightarrow{\sim} {\mbox{\rm H}}_{G}^{\ast} (X,{\mathbb{F}}
) \xrightarrow{\sim} {\mbox{\rm H}}^\ast(X,{\mathbb{F}})^G.
\end{equation}

\subsection{Prior work}
\label{subsec:prior-work}
The problem of bounding the equivariant Betti numbers of symmetric semi-algebraic subsets of 
${\mathrm{R}}^k$ was investigated in \cite{BC2013}. We recall in this section a few results from
\cite{BC2013} that are generalized in the current paper.

We recall some definitions and notation from \cite{BC2013}.

\begin{notation}
  \label{not:equivariant-betti} 
  For any $\mathfrak{S}_{\mathbf{k}}$ symmetric
  semi-algebraic subset $S \subset {\mathrm{R}}^{k}$ with $\mathbf{k}= (k_{1} , \ldots
  ,k_{\ell}) \in {\mathbb{N}}^{\ell}$, with $k= \sum_{i=1}^{\ell} k_{i}$,
  and any field ${\mathbb{F}}$,  we denote
  \begin{eqnarray*}
    b_{\mathfrak{S}_{\mathbf{k}}}^{i} (S,{\mathbb{F}}) & =  & b_{i} (S/\mathfrak{S}_{\mathbf{k}} ,{\mathbb{F}}) ,\\
    b_{\mathfrak{S}_{\mathbf{k}}} (S,{\mathbb{F}}) & = & \sum_{i \geq 0}
    b_{\mathfrak{S}_{\mathbf{k}}}^{i} (S,{\mathbb{F}}).
  \end{eqnarray*}
\end{notation}

The following theorem is proved in \cite{BC2013}.

\begin{theorem} \cite{BC2013}
  \label{thm:main} Let $\mathbf{k}= (k_{1} , \ldots ,k_{\ell}) \in
  {\mathbb{N}}^{\ell}$,with  $k= \sum_{i=1}^{\ell} k_{i}$. Suppose that 
  $P \in {\mathrm{R}} [{\mathbf{X}}^{(1)} , \ldots,{\mathbf{X}}^{(\ell)} ]$, where each
  ${\mathbf{X}}^{(i)}$ is a block of $k_{i}$ variables, is a 
  non-negative polynomial, such that $V= {{\rm Z}}(P, {\mathrm{R}}^{k})$ is
  invariant under the action of $\mathfrak{S}_{\mathbf{k}}$ permuting each
  block ${\mathbf{X}}^{(i)}$ of $k_{i}$ coordinates. Let
  $\deg_{{\mathbf{X}}^{(i)}} (P)   \leq  d$ for $1 \leq i \leq \ell$. Then, for any
  field of coefficients ${\mathbb{F}}$,
  \begin{eqnarray*}
    b (V/\mathfrak{S}_{\mathbf{k}} ,{\mathbb{F}}) & \leq &
    \sum_{\mathbf{p}= (p_{1} , \ldots , p_{\ell}) ,1 \leq p_{i}
    \leq \min (2d,k_{i})  } F (\mathbf{k},\mathbf{p})  d (2d-1)^{|
    \mathbf{p} | +1}
  \end{eqnarray*}
  (where $F(\mathbf{k},\mathbf{p})$ is defined in Notation \ref{not:Partition1}).
  If for each $i,1 \leq i \leq \ell$, $2d  \leq k_{i}$, then
  \begin{eqnarray*}
    b (V/\mathfrak{S}_{\mathbf{k}} ,{\mathbb{F}}) & \leq & (k_{1}
    \cdots k_{\ell})^{2d} (O (d))^{2 \ell d+1} .
  \end{eqnarray*}
\end{theorem}

More generally, the following bound holds for symmetric semi-algebraic sets.

\begin{theorem}\cite{BC2013}
  \label{thm:main-sa-closed}
  Let $\mathbf{k}= (k_{1} , \ldots ,k_{\ell})
  \in {\mathbb{N}}^{\ell}$, with $k= \sum_{i=1}^{\ell} k_{i}$, and let
  $\mathcal{P} \subset {\mathrm{R}} [ {\mathbf{X}}^{(1)} , \ldots,{\mathbf{X}}^{(\ell)} ]$ be a finite set of polynomials,
  where each ${\mathbf{X}}^{(i)}$ is a block of $k^{(i)}$
  variables, and such that each $P \in \mathcal{P}$ is symmetric in each block
  of variables ${\mathbf{X}}^{(i)}$. Let $S \subset {\mathrm{R}}^{k}$
  be a $\mathcal{P}$-closed-semi-algebraic set. Suppose that $\deg (P)  
  \leq  d$ for each $P  \in  \mathcal{P}$, ${\mathrm{card}} (
  \mathcal{P}) =s$, and let $D=D (\mathbf{k},d) = \sum_{i=1}^{\ell} \min
  (k_{i} ,5d)$. Then, for any field of coefficients ${\mathbb{F}}$,
  \begin{eqnarray*}
    b (S/\mathfrak{S}_{\mathbf{k}} ,{\mathbb{F}}) & \leq & \sum_{i=0}^{D-1}
    \sum_{j=1}^{D-i} \binom{2 s}{j} 6^{j} G (\mathbf{k},2d)
  \end{eqnarray*}
  where 
    \begin{eqnarray*}
    G (\mathbf{k},d) & = & \sum_{\mathbf{p}= (p_{1} , \ldots ,
    p_{\ell}) ,1 \leq p_{i} \leq \min (2d,k_{i})} F (
    \mathbf{k},\mathbf{p})  d (2d-1)^{| \mathbf{p} | +1}
  \end{eqnarray*}
  (and $F(\mathbf{k},\mathbf{p})$ is defined in Notation \ref{not:Partition1}).
\end{theorem}

\begin{remark}
\label{rem:main-sa-closed}
  In the particular case, when $\ell =1$, $d=O (1)$, the bound in Theorem
  \ref{thm:main-sa-closed} takes the following asymptotic (for $k \gg 1$)
  form.
  \begin{eqnarray*}
    b (S/\mathfrak{S}_{k} ,{\mathbb{F}}) & \leq & O  (s^{5d} k^{4d-1}) .
  \end{eqnarray*}
\end{remark}

The rest of the paper is organized as follows. In Section \ref{sec:results} we state the new results proved in this paper.
In Section \ref{sec:preliminaries} we prove or recall certain preliminary facts that will be needed in the proofs of the
main theorems.  In Section \ref{sec:proofs-of-main} we prove the main theorems, and finally in Section
\ref{sec:conclusion} we end with some open problems.

\section{Main Results}
\label{sec:results}
In view of the isomorphism \eqref{eqn:iso},  Theorem \ref{thm:main} (respectively, 
Theorem \ref{thm:main-sa-closed}) gives  a 
bound (which is polynomial for fixed $d$)  on the multiplicity of the trivial representation in the $\mathfrak{S}_{\mathbf{k}}$-module ${\mbox{\rm H}}^\ast(V,{\mathbb{F}})$ (respectively, ${\mbox{\rm H}}^\ast(S,{\mathbb{F}})$).
In the current paper we generalize  both Theorems \ref{thm:main} and \ref{thm:main-sa-closed}  by proving a polynomial bound on the multiplicities of every irreducible
representation  appearing in the isotypic decomposition of ${\mbox{\rm H}}^\ast(V,{\mathbb{F}})$ and ${\mbox{\rm H}}^\ast(S,{\mathbb{F}})$ .
Note that as Example \ref{eg:basic} shows,
the dimensions of ${\mbox{\rm H}}^\ast(V,{\mathbb{F}})$, where $V$ is a 
symmetric real variety in ${\mathrm{R}}^k$ 
defined by polynomials of degree bounded by $d$ could be
exponentially large in $k$.
We also extend these basic results in several directions -- including more general actions of the symmetric group, and as a particular case symmetric varieties in ${\mathrm{C}}^k$,  as well as symmetric projective varieties.

\subsection{Affine algebraic case}
We first state our results for symmetric real algebraic subvarieties of real affine space. The main structural
result giving restrictions on the irreducible representations of $\mathfrak{S}_{\mathbf{k}}$,
and their multiplicities,  in the cohomology modules of such varieties is given in Theorem
\ref{thm:main-product-of-symmetric}. The quantitative estimates that follow are stated in
Theorem \ref{thm:main-product-of-symmetric-quantitative}.
The case of symmetric complex affine varieties is dealt with in Theorem \ref{thm:main-product-of-symmetric-quantitative-complex}. 
\begin{notation}
  \label{not:multiplicity-betti} 
  Let ${\mathbf{k}} =(k_1,\ldots,k_\ell),{\mathbf{m}}=(m_1,\ldots,m_\ell) \in {\mathbb{N}}^\ell$, and 
  $K = \sum_{i=1}^\ell k_i m_i$.
  For any $\mathfrak{S}_{\mathbf{k}}$-symmetric
  semi-algebraic subset $S \subset {\mathrm{R}}^{K}$, any field ${\mathbb{F}}$, 
 and $\pmb{\lambda} \in {\mathrm{Par}}({\mathbf{k}})$, we denote
  \begin{eqnarray*}
    m_{i,\pmb{\lambda}} (S,{\mathbb{F}}) & =  & \dim_{\mathbb{F}} {\textrm{hom}}_{\mathfrak{S}_{\mathbf{k}}} (\mathbb{S}^{\pmb{\lambda}},{\mbox{\rm H}}^i(S,{\mathbb{F}}))\\
    &=& {\mathrm{mult}}(\mathbb{S}^{\pmb{\lambda}}, {\mbox{\rm H}}^i(S,{\mathbb{F}})), \\
    m_{\pmb{\lambda}} (S,{\mathbb{F}}) & =  & \sum_{i}  m_{i,\pmb{\lambda}} (S,{\mathbb{F}}).
  \end{eqnarray*}
  Note that in the particular case when $\pmb{\lambda} = ((k_1),\ldots,(k_\ell))$ (i.e. when $\mathbb{S}^{\pmb{\lambda}}$ is the trivial representation of $\mathfrak{S}_{\mathbf{k}}$), 
 \begin{eqnarray*}
    m_{i,\pmb{\lambda}} (S,{\mathbb{F}}) & =  &b_i(S/\mathfrak{S}_{\mathbf{k}},{\mathbb{F}})\\
    m_{\pmb{\lambda}} (S,{\mathbb{F}}) & =  & b(S/\mathfrak{S}_{\mathbf{k}},{\mathbb{F}}).
  \end{eqnarray*}
  
  
\end{notation}

\begin{notation}
For ${\mathbf{d}} = (d_1,\ldots,d_\ell),{\mathbf{m}} = (m_1,\ldots,m_\ell) \in {\mathbb{N}}^\ell$, we denote by 
${\mathbf{d}}^{\mathbf{m}} = (d_1^{m_1},\ldots,d_\ell^{m_\ell})$.
\end{notation}

\begin{notation}
\label{def:set-of-irred}

For ${\mathbf{k}} \in {\mathbb{N}}^\ell, \pmb{\lambda} \in {\mathrm{Par}}({\mathbf{k}})$, we denote 
(see also Notation \ref{not:induced-rep-and-mult})

\begin{equation}
\label{eqn:set-of-irred-lambda}
\mathcal{I}(\pmb{\lambda}) = 
\bigcup_{\substack{\lambda^{(i)} = {\lambda^{(i)}}' \coprod {\lambda^{(i)}}''\\ 1 \leq i \leq \ell}}
\{
 {\pmb{\mu}} \mid \pmb{\mu} =(\mu^{(1)},\ldots,\mu^{(\ell)})  \in {\mathrm{Par}}({\mathbf{k}}),  
 m^{\mu^{(i)}}_{{\lambda^{(i)}}',{\lambda^{(i)}}''} >0 
\}.
\end{equation}
For ${\mathbf{k}},{\mathbf{d}} ,{\mathbf{m}}\in {\mathbb{N}}^\ell$, we denote
\begin{equation}
\label{eqn:set-of-irred}
\mathcal{I}({\mathbf{k}},{\mathbf{d}},{\mathbf{m}}) :=
\bigcup_{\pmb{\lambda}  \in {\mathrm{Par}}({\mathbf{k}}, (2{\mathbf{d}})^{\mathbf{m}})} \mathcal{I}(\pmb{\lambda}).
\end{equation}
If $\ell=1$, ${\mathbf{k}}=(k),{\mathbf{d}}= (d),{\mathbf{m}}=(m)$, we will denote $\mathcal{I}({\mathbf{k}},{\mathbf{d}},{\mathbf{m}})$ by
$\mathcal{I}(k,d,m)$.
\end{notation}

The following theorem gives a restriction on the partitions in  $\mathcal{I}(k,d,m)$.

\begin{theorem}
\label{thm:restriction}
Let $k,d,m > 0$, and $\mu=(\mu_1,\mu_2,\ldots) \in  \mathcal{I}(k,d,m)$.
Then,
\[
{\mathrm{card}}(\{ i \mid \mu_i \geq (2d)^m \}) \leq (2d)^m, \\
{\mathrm{card}}(\{j \mid \tilde{\mu}_j \geq (2d)^m\}) \leq (2d)^m.
\]
\end{theorem}

\begin{remark}
\label{rem:restriction}
Note that Theorem \ref{thm:restriction} implies that the Young diagram for each  $\mu \in \mathcal{I}(k,d,m)$ is contained in the union of $(2d)^m$ rows and$(2d)^m$ columns. This is shown in Figure
\ref{fig:young} for fixed $d,m$ and large $k$. 
The shaded area inside the $k \times k$ sized box contains all possible Young diagrams of
partitions of $k$. The darker part contains the partitions belonging to $\mathcal{I}(k,d,m)$.

\begin{figure}

\caption{The shaded area contains all Young diagrams of partitions in ${\mathrm{Par}}(k)$, while the darker area contains the Young diagrams of the partitions in the subset $\mathcal{I}(k,d,m)\subset {\mathrm{Par}}(k)$ for fixed $d,m$ and large $k$.}
\label{fig:young}
\end{figure}
\end{remark}

\begin{theorem}
\label{thm:main-product-of-symmetric}
Let ${\mathbf{k}}=(k_1,\ldots,k_\ell),
{\mathbf{m}} =(m_1,\ldots,m_\ell), 
{\mathbf{d}}=(d,\ldots,d) \in {\mathbb{N}}^\ell$, and $K = \sum_{i=1}^{\ell} m_i k_i$.
Let 
$P \in {\mathrm{R}}[{\mathbf{X}}^{(1)},\ldots,{\mathbf{X}}^{(\ell)}]$
be a $\mathfrak{S}_{\mathbf{k}}$-symmetric polynomial, 
with $\deg(P) \leq d$. Let  $V = {{\rm Z}}(P,{\mathrm{R}}^K)$.
Then, for all $\pmb{\mu} \in {\mathrm{Par}}({\mathbf{k}})$,  $m_{\pmb{\mu}}(V,{\mathbb{F}}) > 0$ implies that
\begin{equation}
\label{eqn:restriction-on-specht}
\pmb{\mu} \in 
\mathcal{I}({\mathbf{k}},{\mathbf{d}},{\mathbf{m}}).
\end{equation}
Moreover, for each  $\pmb{\mu} = (\mu^{(1)},\ldots,\mu^{(\ell)}) \in \mathcal{I}({\mathbf{k}},{\mathbf{d}},{\mathbf{m}})$,  
\begin{eqnarray*}
  m_{\pmb{\mu}}(V,{\mathbb{F}}) &\leq& 
\sum_{\pmb{\lambda}=(\lambda^{(1)},\ldots,\lambda^{(\ell)})\in {\mathrm{Par}}({\mathbf{k}}, (2{\mathbf{d}})^{\mathbf{m}})}
G(\pmb{\mu},\pmb{\lambda},{\mathbf{d}},{\mathbf{m}}),
\end{eqnarray*}
where 
\[
G(\pmb{\mu},\pmb{\lambda},{\mathbf{d}},{\mathbf{m}}) =
\prod_{1 \leq i \leq \ell}
\left(
 (2 d)^{(m_i{\mathrm{length}}(\lambda^{(i)}))}
 \max_{\lambda^{(i)} = {\lambda^{(i)}}' \coprod {\lambda^{(i)}}''}
 m^{\mu^{(i)}}_{{\lambda^{(i)}}',{\lambda^{(i)}}''}
 \right)
 \]
(the maximum on the right hand side is taken over all decompositions
$\lambda^{(i)} = {\lambda^{(i)}}' \coprod {\lambda^{(i)}}''$). 
\end{theorem}

\begin{remark}
\label{re:restriction-is-non-trivial}
Note that 
the restriction on the Specht modules that are allowed to appear in the cohomology
module ${\mbox{\rm H}}^*(V,{\mathbb{F}})$ \emph{does not} follow only from dimension considerations,
and the Ole{\u\i}nik-Petrovski{\u\i}-Thom-Milnor bound (Theorem \ref{thm:classical}) 
on $b(V,{\mathbb{F}})$.

For example, let $\ell=1,m_1=1,k_1=k = 2^p-1$, and let $\lambda \vdash k$ be the
partition $(2^{p-1},\ldots,1)$. In this case:

\begin{eqnarray*}
\dim_{\mathbb{F}} \mathbb{S}^{\lambda} &\leq & \dim_{\mathbb{F}} {\mathrm{Ind}}_{\mathfrak{S}_\lambda}^{\mathfrak{S}_k} 
(\mbox{ since }  K(\lambda,\lambda)=1 \mbox{ (Definition \ref{def:Kostka} and Proposition \ref{prop:Young}})
  \\
&=&  \binom{k}{2^{p-1},\ldots,1} \\
&\leq & O(1)^k \mbox{ using Stirling's approximation}.
\end{eqnarray*}

Thus, if $V$ is defined by a polynomial of degree bounded by $d$, and $k$ is large enough,
$\mathbb{S}^\lambda$ is not ruled out of appearing with positive multiplicity in 
${\mbox{\rm H}}^*(V,{\mathbb{F}})$ just on the basis of the upper bound in Theorem \ref{thm:classical}.
On the other hand, it follows from \eqref{eqn:set-of-irred} that for all $k$ large enough, 
and fixed $d$, 
\[
\mathbb{S}^\lambda \not\in \mathcal{I}(k,d,1),
\] 
and hence by Theorem \ref{thm:main-product-of-symmetric} 
cannot appear with positive multiplicity in ${\mbox{\rm H}}^*(V,{\mathbb{F}})$. 
\end{remark}

With the same hypothesis as in Theorem \ref{thm:main-product-of-symmetric} we have:

\begin{theorem}[Symmetric real affine varieties]
\label{thm:main-product-of-symmetric-quantitative}
For each $\pmb{\lambda} \in{\mathrm{Par}}({\mathbf{k}}, (2{\mathbf{d}})^{\mathbf{m}})$,
\[
{\mathrm{card}}(\mathcal{I}(\pmb{\lambda})) \leq \prod_{1 \leq i \leq \ell} k_i^{O(d^{2 m_i})},
\]
and for each $\pmb{\mu} \in \mathcal{I}({\mathbf{k}},{\mathbf{d}},{\mathbf{m}})$
\[
m_{\pmb{\mu}}(V,{\mathbb{F}}) \leq
\prod_{1 \leq i \leq \ell} k_i^{O(d^{2 m_i})} d^{m_id}.
\]

In the particular case, when $\ell=1$, and $d_1=d$ and $m_1=m$  are fixed, both bounds are polynomial in $k_1=k$.
\end{theorem}

\begin{remark}
\label{rem:b0}
We also remark here (without proof)  that if we fix some $i \geq 0$,
it is possible to  obtain 
using techniques from \cite{BC2013} (proof of \cite[Proposition 8]{BC2013})
slightly better bound on  $m_{i,\pmb{\mu}}$ compared
to the bound in Theorem \ref{thm:main-product-of-symmetric}. 
For example taking $i=0$ and with the same notation as in Theorem \ref{thm:main-product-of-symmetric}, 
it is possible to prove (using techniques from \cite{BC2013})  that
for all $\pmb{\mu} =(\mu^{(1)},\ldots,\mu^{(\ell)}) \in {\mathrm{Par}}({\mathbf{k}})$,  $m_{0,\pmb{\mu}}(V,{\mathbb{F}}) > 0$ implies that
${\mathrm{length}}(\mu^{(j)}) \leq 2d$ for $1 \leq j \leq \ell$.

Moreover, for each  $\pmb{\mu} =(\mu^{(1)},\ldots,\mu^{(\ell)}) \in {\mathrm{Par}}({\mathbf{k}})  $ with $m_{0,\pmb{\mu}}(V,{\mathbb{F}}) > 0$,
since ${\mathrm{length}}(\mu^{(j)})\leq 2d, 1 \leq j \leq \ell$, 
it follows with the same arguments  presented in the proof of 
Theorem \ref{thm:main-product-of-symmetric-quantitative} that
\begin{eqnarray*}
  m_{0,\pmb{\mu}}(V,{\mathbb{F}}) &\leq& 
\sum_{\pmb{\lambda}=(\lambda^{(1)},\ldots,\lambda^{(\ell)})\in {\mathrm{Par}}({\mathbf{k}}, (2{\mathbf{d}})^{\mathbf{m}})}
\prod_{1 \leq i \leq \ell}
\left(
 (2 d)^{(m_i{\mathrm{length}}(\lambda^{(i)}))}
 K(\mu^{(i)},\lambda^{(i)})
 \right).
\end{eqnarray*}

The quantitative estimates obtained on $m_{0,\pmb{\mu}}(V,{\mathbb{F}})$ using the above result will
be slightly better than that obtained directly from Theorem \ref{thm:main-product-of-symmetric}, but since it does not affect the polynomial dependence on the ${\mathbf{k}}$ we prefer 
not to expand on this further.
\end{remark}

A particular case of Theorem \ref{thm:main-product-of-symmetric} deserves mention, and
we state it as a corollary.
\begin{corollary}
\label{cor:equivariant} 
Suppose that
$\mathbf{k}= (\underbrace{1, \ldots 1}_{\ell-1} ,k)$, 
$\mathbf{m}= (\underbrace{1, \ldots 1}_{\ell-1} ,m)$, 
and $(2d)^m \leq k$.
If $\pmb{\mu} = ((1),\ldots,(1),(k))$ (i.e.  $\mathbb{S}^{\pmb{\mu}}$ is the trivial representation),
then
\begin{eqnarray*}
 m_{\pmb{\mu}}(V,{\mathbb{F}}) &=&  b(V/\mathfrak{S}_{\mathbf{k}},{\mathbb{F}})  \\
 &\leq& \sum_{\pmb{\lambda}=(\lambda^{(1)},\ldots,\lambda^{(\ell)})\in {\mathrm{Par}}({\mathbf{k}},(2{\mathbf{d}})^{\mathbf{m}})} 
 \left(\prod_{1 \leq i \leq \ell}   
 (2 d)^{(m_i{\mathrm{length}}(\lambda^{(i)}))}
 \right) \\
 &\leq & k^{(2d)^{m}} O(d)^{m (2d)^m + \ell}.
 \end{eqnarray*}
\end{corollary}

Notice that Corollary \ref{cor:equivariant} generalizes to the case $m > 1$, Corollary 3 in \cite{BC2013}.

We have the following theorem for symmetric complex affine varieties.

\begin{theorem}
[Symmetric complex affine varieties]
\label{thm:main-product-of-symmetric-quantitative-complex}
Let ${\mathbf{k}}=(k_1,\ldots,k_\ell),
{\mathbf{m}} =(m_1,\ldots,m_\ell), 
{\mathbf{d}}=(d,\ldots,d) \in {\mathbb{N}}^\ell$,
and 
$K = \sum_{i=1}^{\ell} k_i m_i$.
Let 
$\mathcal{P} \subset {\mathrm{C}}[{\mathbf{X}}^{(1)},\ldots,{\mathbf{X}}^{(\ell)}]$
be a finite set  $\mathfrak{S}_{\mathbf{k}}$-symmetric polynomials, 
with $\deg(P) \leq d$ for each $P \in \mathcal{P}$. Let  $V = {{\rm Z}}(\mathcal{P},{\mathrm{C}}^K)$.

Then, for all $\pmb{\mu} \in {\mathrm{Par}}({\mathbf{k}})$,  $m_{\pmb{\mu}}(V,{\mathbb{F}}) > 0$ implies that
\[
\pmb{\mu} \in 
\mathcal{I}({\mathbf{k}}, 2{\mathbf{d}},2{\mathbf{m}}).
\]
Moreover,
for each $\pmb{\mu} \in \mathcal{I}({\mathbf{k}},2{\mathbf{d}},2{\mathbf{m}})$
\[
m_{\pmb{\mu}}(V,{\mathbb{F}}) \leq
\prod_{1 \leq i \leq \ell} k_i^{O(d^{4 m_i})} d^{2 m_i d}.
\]
\end{theorem}

\subsection{Affine semi-algebraic case}
We now state our results in the semi-algebraic case. As before we state first a structural result
(Theorem \ref{thm:main-product-of-symmetric-sa} below), and deduce quantitative estimates
from it (Theorem \ref{thm:main-product-of-symmetric-sa-quantitative}). 
\begin{theorem}
\label{thm:main-product-of-symmetric-sa}
Let ${\mathbf{k}}=(k_1,\ldots,k_\ell),
{\mathbf{m}} =(m_1,\ldots,m_\ell), 
{\mathbf{d}}=(d,\ldots,d) \in {\mathbb{N}}^\ell$, and $K = \sum_{i=1}^{\ell} k_i m_i$.
Let 
$\mathcal{P} \subset {\mathrm{R}}[{\mathbf{X}}^{(1)},\ldots,{\mathbf{X}}^{(\ell)}]$
be a finite set of $\mathfrak{S}_{\mathbf{k}}$-symmetric polynomials, 
with $\deg(P) \leq d$ for all  $P \in \mathcal{P}$, and let ${\mathrm{card}}(\mathcal{P}) = s$. Let  
$S \subset {\mathrm{R}}^K$  be a $\mathcal{P}$-closed semi-algebraic set.

Then, for all $\pmb{\mu} \in {\mathrm{Par}}({\mathbf{k}})$,  $m_{\pmb{\mu}}(S,{\mathbb{F}}) > 0$ implies that
\[
\pmb{\mu} \in 
\mathcal{I}({\mathbf{k}},{\mathbf{d}},{\mathbf{m}}).
\]
Moreover, 
let $D=D (\mathbf{k},\mathbf{m},d) = \sum_{i=1}^{\ell} \min
  (m_i k_i ,d^{m_i})$.
Then,
for each  
\[
\pmb{\mu} = (\mu^{(1)},\ldots,\mu^{(\ell)}) \in \mathcal{I}({\mathbf{k}},{\mathbf{d}},{\mathbf{m}}),
\]  
\begin{eqnarray*}
  m_{\pmb{\mu}}(S,{\mathbb{F}})  &\leq& 
    \sum_{i=0}^{D-1}
    \sum_{j=1}^{D-i} \binom{2 s+1}{j} 6^{j} \cdot \left(\sum_{\pmb{\lambda}=(\lambda^{(1)},\ldots,\lambda^{(\ell)})\in {\mathrm{Par}}({\mathbf{k}}, (2{\mathbf{d}})^{\mathbf{m}})}
G(\pmb{\mu},\pmb{\lambda},{\mathbf{d}},{\mathbf{m}})\right),
\end{eqnarray*}
where 
\[
G(\pmb{\mu},\pmb{\lambda},{\mathbf{d}},{\mathbf{m}}) =
\prod_{1 \leq i \leq \ell}
 \left(
 (2 d)^{(m_i{\mathrm{length}}(\lambda^{(i)}))}
 \max_{\lambda^{(i)} = {\lambda^{(i)}}' \coprod {\lambda^{(i)}}''}
 m^{\mu^{(i)}}_{{\lambda^{(i)}}',{\lambda^{(i)}}''}
 \right)
 \]
(the maximum on the right hand side is taken over all decompositions
$\lambda^{(i)} = {\lambda^{(i)}}' \coprod {\lambda^{(i)}}''$). 
\end{theorem}

Using the same notation as in Theorem \ref{thm:main-product-of-symmetric-sa}:
\begin{theorem}[Symmetric affine semi-algebraic sets]
\label{thm:main-product-of-symmetric-sa-quantitative}
For each  $\pmb{\mu} \in \mathcal{I}({\mathbf{k}},{\mathbf{d}},{\mathbf{m}})$
\[
m_{\pmb{\mu}}(S,{\mathbb{F}}) \leq
O(s)^{D} \cdot \prod_{1 \leq i \leq \ell} k_i^{O(d^{2 m_i})} d^{m_i(2d)^{m_i}}.
\]
In the particular case, when $\ell=1$, and $d_1=d$ and $m_1=m$  are fixed, both bounds are polynomial in $s$ and
$k_1=k$.
\end{theorem}

\subsection{Projective case}
We can apply our results obtained in the previous section to study the topology of symmetric projective
varieties as well. 
We state one such result below.

\begin{theorem}[Symmetric complex projective varieties]
\label{thm:symmetric-complex-projective}
Let $V \subset {\mathbb{P}}_{\mathrm{C}}^k$ be defined by symmetric homogeneous polynomials in ${\mathrm{C}}[X_0,\ldots,X_k]$
of degrees bounded by $d$. 
Then, the irreducible representations, $\mathbb{S}^\mu, \mu \vdash k+1$,  that occur in
${\mbox{\rm H}}^i(V,{\mathbb{F}})$ with positive multiplicities belong to the set 
$\mathcal{I}(k,2d,2)$,
and the multiplicity 
of each such representation is bounded by 
\[
k^{O(d^{4})} d^{2d}.
\]
\end{theorem}

\begin{remark}
\label{rem:symmetric-complex-projective}
Suppose $V \subset {\mathbb{P}}_{\mathrm{C}}^k$ be defined by symmetric homogeneous polynomials 
in ${\mathrm{C}}[X_0,\ldots,X_k]$ of degrees bounded by $d$.
Unlike in the affine case, it is not true that dimensions of equivariant cohomology,
$\dim_{\mathbb{F}} {\mbox{\rm H}}_{\mathfrak{S}_{k+1}}^*(V,{\mathbb{F}})$,
are bounded by a function of $d$ independent of $k$.
For example,
\[
{\mbox{\rm H}}^*({\mathbb{P}}_{\mathrm{C}}^k,{\mathbb{F}}) \cong {\mbox{\rm H}}^*({\mathbb{P}}_{\mathrm{C}}^k/\mathfrak{S}_{k+1},{\mathbb{F}}),
\]
 and thus
 \[\dim_{\mathbb{F}}  {\mbox{\rm H}}^*({\mathbb{P}}_{\mathrm{C}}^k/\mathfrak{S}_{k+1},{\mathbb{F}}) = k+1,
 \]
 which clearly grows linearly with $k$.
\end{remark}

\subsection{Application to bounding topological complexity of images of polynomial maps}
\label{subsec:non-equivariant} 
In this section we discuss an application of 
Corollary  \ref{cor:equivariant}  
to bounding the Betti
numbers of images of real algebraic varieties under linear projections.
In \cite{BC2013}, similar results were proved in the very special case of projections of the form 
$\pi:{\mathrm{R}}^{k+1} \rightarrow {\mathrm{R}}^k$. In this paper, since we consider more general actions of the symmetric group, we are able to handle projections along more than one variables, and so are
able to strengthen as well as generalize the results in \cite{BC2013}.

In order to state our results more precisely we first introduce some notation.
Let 
$P \in {\mathrm{R}} [ Y_{1} , \ldots ,Y_{k} ,X_{1} , \ldots ,X_{m}]$ be a non-negative polynomial with 
$\deg ( P ) \leq d$.
Let $\pi : {\mathrm{R}}^{m+k}
\longrightarrow {\mathrm{R}}^{k}$ be the projection map to the first $k$ co-ordinates,
and let 
$V = {{\rm Z}}(P,{\mathrm{R}}^{m+k})$.
We
consider the problem of bounding the Betti numbers of the image 
$\pi(V)$.
Bounding the complexity of the image under projection of semi-algebraic sets is a very
important and well-studied problem related to quantifier elimination in the the first order
theory of the reals, and has many ramifications -- including in computational
complexity theory (see for example \cite{Basu-sheaf}).

There are two different approaches. One can first obtain a semi-algebraic
description of the image 
$\pi(V)$
with bounds on the degrees and the number
of polynomials appearing in this description (via results in effective quantifier elimination in the
the first order theory of the reals), and then apply known bounds on
the Betti numbers of semi-algebraic sets in terms of these parameters. Another
approach (due to Gabrielov, Vorobjov and Zell \cite{GVZ04}) 
is to use the ``descent spectral sequence'' of the map 
$\pi |_{V}$
which abuts to the cohomology of 
$\pi ( V )$
and bound the Betti numbers of
$\pi ( V )$
by bounding the dimensions of the $E_{1}$-terms of this spectral
sequence. 
For this approach it is essential that 
the map $\pi$ is proper (which is ensured by requiring that 
$V$
is 
bounded)
since in the general case the spectral sequence might not converge to ${\mbox{\rm H}}^*(S,{\mathbb{F}})$. The second approach produces a slightly better bound. The following
theorem (in the special case of algebraic sets) whose proof uses the second approach appears in {\cite{GVZ04}}.

\begin{theorem} \cite{GVZ04}
  \label{thm:descent-quantitative} 
With the same notation as above,
  \begin{eqnarray}
  \label{eqn:descent-quantitative}
    b ( \pi ( V ) ,{\mathbb{F}} ) & = & ( O ( d ) )^{( m+1 ) k} .
  \end{eqnarray}
\end{theorem}

Notice that in the exponent of the bound in  \eqref{eqn:descent-quantitative}, there is a 
factor of $(m+1)$ which is linear in the dimension of the fibers of the projection $\pi$.
This factor is also present if one uses effective quantifier elimination method to bound the Betti numbers of 
$\pi(V)$.
  Using Theorem \ref{thm:main-product-of-symmetric-sa-quantitative} we are able to remove this \emph{multiplicative} factor of $(m+1)$ in the exponent of the bound in 
\eqref{eqn:descent-quantitative}
at the expense of an extra additive term that depends just  on $d$ and $m$.

We now state the result more precisely.
In \cite{BC2013},
the following
bound on the Betti numbers of the image under projection to a subspace of 
dimension one less than that of the ambient space of real algebraic varieties (i.e. with $m=1$), as well as of semi-algebraic sets (not
necessarily symmetric).

\begin{theorem} \cite{BC2013}
  \label{thm:descent2-quantitative}Let $P \in {\mathrm{R}} [ Y_{1} , \ldots ,Y_{k} ,X
  ]$ be a non-negative polynomial and with $\deg ( P ) \leq d$. Let $V= {{\rm Z}}
  \left( P, {\mathrm{R}}^{k+1} \right)$ be bounded, and $\pi : {\mathrm{R}}^{k} \times {\mathrm{R}}
  \longrightarrow{\mathrm{R}}^{k}$ be the projection map to the first $k$ coordinates.
  Then,
  \begin{eqnarray*}
    b ( \pi ( V ) ,{\mathbb{F}} ) & \leq &  \left( \frac{k}{d} \right)^{2d} ( O ( d ) )^{k+2d+1}.
  \end{eqnarray*}
\end{theorem}

In this paper we generalize the above results to the case $m>1$. We prove the following theorem.
\begin{theorem}
  \label{thm:descent2-quantitative-new}
  Let $P \in {\mathrm{R}} [ Y_{1} , \ldots ,Y_{k} ,X_1,\ldots,X_m]$ be a non-negative polynomial and with $\deg ( P ) \leq d$. Let $V= {{\rm Z}}(P, {\mathrm{R}}^{k+m})$ be bounded, and $\pi : {\mathrm{R}}^{k} \times {\mathrm{R}}^{m}
  \longrightarrow{\mathrm{R}}^{k}$ be the projection map to the first $k$ coordinates.
  Then,
  \begin{eqnarray*}
    b ( \pi ( V ) ,{\mathbb{F}} ) & \leq & k^{(2d)^m} (O(d))^{k+ m (2d)^m +1} .
  \end{eqnarray*}
\end{theorem}

\subsection{Application to proving lower bounds on degrees}
The upper bounds in the theorems stated above can be potentially applied to prove lower bounds on the degrees of polynomials needed to define symmetric varieties having certain prescribed geometry.
We describe one such example.

 \begin{example}
 \label{eg:lower-bound}
 Let $k = 2^p-1$, and let 
 $\tilde{V}_k$ be any non-empty compact semi-algebraic set contained in the subset of ${\mathrm{R}}^k$ defined by
 \begin{eqnarray*}
  X_1=&\cdots &=X_{2^{p-1}}  \\
  &\neq& \\
   X_{2^{p-1}+1} = &\cdots& = X_{2^{p-1}+2^{p-2}} \\
    &\neq&\\
  X_{2^{p-1}+2^{p-2}+1} = &\cdots& =  X_{2^{p-1}+\cdots+2^2+1} X_{2^{p-1}+\cdots+2^1} \\
   &\neq& \\
   &X_{2^{p-1}+\cdots+2^1+1}&.
  \end{eqnarray*}
  
Then, the stabilizer of $\tilde{V}_k$ under the action of $\mathfrak{S}_k$ on ${\mathrm{R}}^k$, is the
Young subgroup $\mathfrak{S}_{\lambda^{(k)}}$, where $\lambda^{(k)} = (2^{p-1},2^{p-2},\ldots,1)$.
Let $V_k$ be the orbit of $\tilde{V}_k$ under the action of $\mathfrak{S}_k$. In other words,
\[
V_k  = \mathfrak{S}_k \cdot \tilde{V}_k.
\]
 
 
Then, 
\begin{eqnarray*}
b_0(V_k,{\mathbb{F}}) & = &  b_0(\tilde{V}_k,{\mathbb{F}}) \cdot \binom{k}{2^{p-1}, 2^{p-2},\ldots,2^0}  \\
               &=&  b_0(\tilde{V}_k,{\mathbb{F}}) \cdot (\Theta(1))^k \mbox{ using Stirling's approximation}.
\end{eqnarray*}

We claim that that for any constant $d_0$, for  all $k$ large enough, $V_k$ cannot be described as the
set of real zeros of a polynomial $P \in {\mathrm{R}}[X_1,\ldots,X_k]$ with $\deg(P) \leq d_0$.
To see this observe that 
\[
{\mbox{\rm H}}^0(V_k,{\mathbb{F}}) \cong_{\mathfrak{S}_k} b_0(\tilde{V}_k,{\mathbb{F}}) \cdot M^{\lambda^{(k)}},
\]
with  $\lambda^{(k)}= (2^{p-1},2^{p-2},\ldots,1)$, and it follows that
$m_{0,\lambda^{(k)}}(V_k,{\mathbb{F}}) > 0$. However, it follows from  the definition of 
$\mathcal{I}(k,d,m)$  (see \eqref{eqn:set-of-irred})
that for any fixed $d_0$,
\[
\lambda^{(k)} \not\in \mathcal{I}(k,d_0,1),
\]
for all $k$ large enough, and hence by Theorem \ref{thm:main-product-of-symmetric}
for all large enough $k$, $V_k$ cannot be defined by polynomials with degrees bounded by $d_0$. 

Note that
in the case, when $\tilde{V}_k$ is a finite set of points, the same result can also be deduced from 
Proposition \ref{prop:half-degree}.
\end{example} 

\section{Preliminaries}
\label{sec:preliminaries}
Before proving the new theorems stated in the previous section we need some preliminary results.
These are described in the following subsections.

\subsection{Real closed extensions and Puiseux series}
\label{subsec:Puiseux}
In this section we recall some basic facts about real closed fields and real
closed extensions.

We will need some
properties of Puiseux series with coefficients in a real closed field. We
refer the reader to {\cite{BPRbook2}} for further details.

\begin{notation}
\label{not:Puiseux}
  For ${\mathrm{R}}$ a real closed field we denote by ${\mathrm{R}} \left\langle {{\varepsilon}}
  \right\rangle$ the real closed field of algebraic Puiseux series in ${{\varepsilon}}$
  with coefficients in ${\mathrm{R}}$. We use the notation ${\mathrm{R}} \left\langle {{\varepsilon}}_{1} ,
  \ldots , {{\varepsilon}}_{m} \right\rangle$ to denote the real closed field ${\mathrm{R}}
  \left\langle {{\varepsilon}}_{1} \right\rangle \left\langle {{\varepsilon}}_{2} \right\rangle
  \cdots \left\langle {{\varepsilon}}_{m} \right\rangle$. Note that in the unique
  ordering of the field ${\mathrm{R}} \left\langle {{\varepsilon}}_{1} , \ldots , {{\varepsilon}}_{m}
  \right\rangle$, $0< {{\varepsilon}}_{m} \ll {{\varepsilon}}_{m-1} \ll \cdots \ll {{\varepsilon}}_{1} \ll 1$.
\end{notation}

\begin{notation}
\label{not:Ext}
  If ${\mathrm{R}}'$ is a real closed extension of a real closed field ${\mathrm{R}}$, and $S
  \subset {\mathrm{R}}^{k}$ is a semi-algebraic set defined by a first-order formula
  with coefficients in ${\mathrm{R}}$, then we will denote by ${\mathrm{Ext}} \left(S, {\mathrm{R}}'
  \right) \subset {\mathrm{R}}'^{k}$ the semi-algebraic subset of ${\mathrm{R}}'^{k}$ defined by
  the same formula. It is well-known that ${\mathrm{Ext}} \left(S, {\mathrm{R}}' \right)$ does
  not depend on the choice of the formula defining $S$ {\cite{BPRbook2}}.
\end{notation}

\begin{notation}
  For $x \in {\mathrm{R}}^{k}$ and $r \in {\mathrm{R}}$, $r>0$, we will denote by $B_{k} (x,r)$
  the open Euclidean ball centered at $x$ of radius $r$. If ${\mathrm{R}}'$ is a real
  closed extension of the real closed field ${\mathrm{R}}$ and when the context is
  clear, we will continue to denote by $B_{k} (x,r)$ the extension ${\mathrm{Ext}}
  \left(B_{k} (x,r) , {\mathrm{R}}' \right)$. This should not cause any confusion.
\end{notation}

\subsection{Tarski-Seidenberg transfer principle}
In some proofs involving
Morse theory (see for example the proof of Lemma \ref{lem:equivariant_morseB}), where integration of gradient flows is used in an essential way, we
first restrict to the case ${\mathrm{R}} =\mathbb{R}$. After having proved the result
over $\mathbb{R}$, we use the Tarski-Seidenberg transfer theorem to extend
the result to all real closed fields. We refer the reader to 
{\cite[Chapter 2]{BPRbook2}} for an exposition of the Tarski-Seidenberg transfer
principle.

\subsection{Equivariant Morse theory}
\label{subsec:equivariant-morse-theory}

In this section we develop some basic results in equivariant Morse theory that we will need for the proof of 
Theorem \ref{thm:main-product-of-symmetric}. Since the results of this section applies to more general (finite) groups
acting on a manifold, we state and prove our results in a more general setting than what we need in this paper.

Let $G$ be a finite group acting on a compact semi-algebraic $S \subset {\mathrm{R}}^k$, 
defined by $Q \leq 0$, and $W = {{\rm Z}}(Q,{\mathrm{R}}^k) = \partial S$ a
bounded non-singular real algebraic hypersurface.  
Let $e:W \rightarrow {\mathrm{R}}$ be a $G$-equivariant regular 
function with isolated  non-degenerate critical points on $W$.
For each such critical point ${\mathbf{x}}$, we will denote by ${\mathrm{ind}}^-({\mathbf{x}})$ the dimension of the negative eigenspace
of the Hessian of $e$ at ${\mathbf{x}}$. More precisely, the Hessian $\mathrm{Hess}(e)({\mathbf{x}})$ is a symmetric, non-degenerate quadratic form
on the tangent space $T_p W$, and ${\mathrm{ind}}^-({\mathbf{x}})$ is the number of negative eigenvalues of $\mathrm{Hess}(e)({\mathbf{x}})$. 

Consider the set of critical points, $\mathcal{C}$, of the function $e$ restricted to $V$. 
For any subset $I \subset {\mathrm{R}}$, we will denote by $S_I = S \cap e^{-1}(I)$. 
If $I = [\infty,c]$ we will denote $S_I = S_{\leq c}$.

In the next two lemmas we will  ${\mathrm{R}} = \mathbb{R}$ since we will use properties of gradient flows.
\begin{lemma}
  \label{lem:equivariant_morseA} 
  Let $v_1<\cdots < v_N$ be the critical values of $e$ restricted to $W$.
  Then, for $1 \leq i<N$, and for each $v \in
  [ v_{i} ,v_{i+1})$, $S_{\leq v_{i}}$ is a deformation retract of $S_{\leq v}$, and the retraction
  can be chosen to be $G$-equivariant.
\end{lemma}
\begin{proof}
See for example the proof of Theorem 7.5 (Morse Lemma A) in \cite{BPRbook2} with 
$W = {{\rm Z}}(Q,{\mathrm{R}}^k)$, $a=v_{i}$ and $b=v$, noting since $W$ is symmetric and $e$ is symmetric,  the retraction of $W_{\leq v}$ to $W_{\leq v_i}$ that is constructed in the proof of
Theorem 7.5 in \cite{BPRbook2} is symmetric as well. 
\end{proof}
We also need the following equivariant version of Morse Lemma B.

\begin{lemma}
 \label{lem:equivariant_morseB}
Let $v \in e(\mathcal{C})$ be a critical value of $e$.
Let $\mathcal{C}_v^+,\mathcal{C}_v^- , \mathcal{C}_v \subset \mathcal{C}$ be defined by
\begin{eqnarray*}
\mathcal{C}_v^+ &=& \{{\mathbf{x}} \in \mathcal{C}\;\mid \; e({\mathbf{x}}) =v, \;\langle {\mathrm{grad}}(e), {\mathrm{grad}}(Q)\rangle({\mathbf{x}})  >0 \}, \\
\mathcal{C}_v^- &=& \{{\mathbf{x}} \in \mathcal{C}\;\mid \; e({\mathbf{x}}) =v, \;\langle {\mathrm{grad}}(e), {\mathrm{grad}}(Q)\rangle({\mathbf{x}})  <0 \}, \\
\mathcal{C}_v &=& \mathcal{C}_v^+ \;\accentset{\circ}{\cup}\;  \mathcal{C}_v^-
\end{eqnarray*}
($\accentset{\circ}{\cup}$ denotes disjoint union).
  Then for all $0 < {{\varepsilon}} \ll 1$, $S_{\leq v+{{\varepsilon}}}$ retracts
  $G$-equivariantly to a space 
  \[
  S_{\leq v-{{\varepsilon}}} \cup_{B} A,
  \]
  where  
  \[(A,B) = \coprod_{{\mathbf{y}} \in \mathcal{C}_v} ( A_{\mathbf{y}} ,B_{\mathbf{y}}),
  \]
  and for each ${\mathbf{y}} \in \mathcal{C}_v$,  $(A_{\mathbf{y}} ,B_{\mathbf{y}})$ is $G$-equivariantly homotopy equivalent to the pair 
   \[(
  \mathbf{D}^{{\mathrm{ind}}^-({\mathbf{y}})} \times [ 0,1 ] , \partial \mathbf{D}^{{\mathrm{ind}}^-({\mathbf{y}})} \times [ 0,1 ] \cup
  \mathbf{D}^{{\mathrm{ind}}^-({\mathbf{y}})} \times \{ 1 \})
  \]
  if ${\mathbf{y}} \in \mathcal{C}_v^+$,
  or to the pair 
  \[
  (\mathbf{D}^{{\mathrm{ind}}^{-} ({\mathbf{y}})} , \partial \mathbf{D}^{{\mathrm{ind}}^{-} ({\mathbf{y}})}),
  \]
  if ${\mathbf{y}} \in \mathcal{C}_v^-$.
\end{lemma}
\begin{proof}
See proof of Proposition 7.19 in \cite{BPRbook2}, noting again that the retraction constructed in that proof is symmetric in case $Q$ is a symmetric polynomial and the Morse function $e$ is symmetric as well. 
\end{proof}

Now, let $\overline{\mathcal{C}}$ be a set containing a unique representative from 
each $G$-orbit of $\mathcal{C}$.
 
Let $\overline{\mathcal{C}}_i\subset \overline{\mathcal{C}}$ be the set of representatives of the different orbits corresponding to the critical value $v_i$ -- in other words, 
$e(\overline{\mathcal{C}}_i) =\{v_i\}$. 
Note that the cardinality of $\overline{\mathcal{C}}_i$ can be greater than one.
For each ${\mathbf{x}} \in \overline{\mathcal{C}}_i$, let $G_{\mathbf{x}} \subset G$ 
denote the stabilizer subgroup of ${\mathbf{x}}$.

Let also for each $i,0 \leq i \leq N$, and $0< {{\varepsilon}} \ll 1$, $j_{i,{{\varepsilon}}}$ denote inclusion $S_{\leq v_i -{{\varepsilon}}} \hookrightarrow S_{\leq v_i +{{\varepsilon}}}$, and $j_{i,{{\varepsilon}}}^\ast: {\mbox{\rm H}}^*(S_{\leq v_i +{{\varepsilon}}},{\mathbb{F}})  \rightarrow
{\mbox{\rm H}}^*(S_{< v_i -{{\varepsilon}}},{\mathbb{F}})$ the induced homomorphism (which is in fact a homomorphism
of the corresponding $G$-modules).

Let $\overline{\mathcal{C}}_i = \{{\mathbf{x}}^{i,1},\ldots,{\mathbf{x}}^{i,N_i} \}$ (choosing an arbitrary order).

\begin{proposition}
\label{prop:equivariant-morseb}
The homomorphism $j_{i,{{\varepsilon}}}^\ast$ factors through $N_i$ homomorphisms
as follows:
\[
{\mbox{\rm H}}^*(S_{\leq v_i +{{\varepsilon}}},{\mathbb{F}}) =M_0 \xrightarrow{j_{i,{{\varepsilon}},1}^\ast} M_1 \xrightarrow{j_{i,{{\varepsilon}},2,\ast}} \cdots
\xrightarrow{j_{i,{{\varepsilon}},N_i,\ast}} M_{N_i+1} ={\mbox{\rm H}}^*(S_{\leq v_i -{{\varepsilon}}},{\mathbb{F}}) ,
\]
where each $M_h$ is a finite dimensional $G$-module, and
for each $h, 1\leq h \leq N_i$,
 either
\begin{enumerate}[(a)]
\item
\label{item:theorem-main:a}
$j_{i,{{\varepsilon}},h}^\ast$ is injective, and 
\[
M_{h+1} \cong M_h\oplus {\mathrm{Ind}}_{G_{{{\mathbf{x}}^{i,h}}}}^{G}(W_{{\mathbf{x}}^{i,h}}),
\]
for some one-dimensional representation $W_{{\mathbf{x}}^{i,h}}$ of $G_{{\mathbf{x}}^{i,h}}$, or
\item
\label{item:theorem-main:b}
the homomorphism
$j_{i,{{\varepsilon}},h}^\ast$ is surjective, and 
\[
M_{h} \cong M_{h+1} \oplus {\mathrm{Ind}}_{G_{{\mathbf{x}}^{i,h}}}^{G}(W_{{\mathbf{x}}^{i,h}}),
\]
for some one-dimensional representation $W_{{\mathbf{x}}^{i,h}}$ of $G_{{\mathbf{x}}^{i,h}}$.
\end{enumerate}
\end{proposition}
 
\begin{proof}
We first assume that ${\mathrm{R}} = \mathbb{R}$.
Using Lemma \ref{lem:equivariant_morseB} (equivariant Morse Lemma B), we have that 
$S_{\leq v_i +{{\varepsilon}}}$ can be retracted $G$-equivariantly to a 
semi-algebraic set
\[
\tilde{S}_i = S_{\leq v_i -{{\varepsilon}}} \coprod_{\substack{1 \leq j \leq N_i,\\
 \langle{\mathrm{grad}} (e), {\mathrm{grad}}(Q)\rangle({\mathbf{x}}^{i,j}) < 0}}
\coprod_{{\mathbf{y}} \in G \cdot {\mathbf{x}}^{i,j}} \mathbf{D}^{{\mathrm{ind}}^-({\mathbf{x}}^{i,j})}/\sim,
\]
where the identification $\sim$ identifies the boundaries of the disks
$\mathbf{D}^{{\mathrm{ind}}^-({\mathbf{x}}^{i,j})}$ with spheres of the same dimension in  
$S_{\leq v_i -{{\varepsilon}}}$.

Since the different balls $\mathbf{D}^{{\mathrm{ind}}^-({\mathbf{x}}^{i,j})}$ are disjoint, we can decompose the 
glueing process by glueing disks belonging to each orbit successively,  
choosing the order arbitrarily, and thus obtain a filtration,
\[
 S|_{e \leq v_i -{{\varepsilon}}}=S_{i,0} \subset S_{i,1} \subset \cdots  \subset S_{i,N'_i} = \tilde{S}_i,
\]
where $S_{i,j} = S_{i,j-1} \coprod \left(\coprod_{{\mathbf{y}} \in {\mathrm{orbit}}( {\mathbf{x}}^{i,j'})}
 \mathbf{D}^{{\mathrm{ind}}^-({\mathbf{y}})}/\sim\right)$ for some $j', 1 \leq j' \leq N_i$.
 
 Let $D_{i,j'}$ denote the disjoint union of the balls $\mathbf{D}^{{\mathrm{ind}}^-({\mathbf{x}}^{i,j'})}$, and 
 $C_{i,j'} \subset D_{i,j'}$ the disjoint union of their boundaries.
 
 We have the Mayer-Vietoris exact sequence
 \[
 \cdots \leftarrow {\mbox{\rm H}}^p(C_{i,j'},{\mathbb{F}}) \leftarrow {\mbox{\rm H}}^p(D_{i,j'},{\mathbb{F}}) \oplus {\mbox{\rm H}}^p(S_{i,j-1},{\mathbb{F}}) \leftarrow {\mbox{\rm H}}^{p}(S_{i,j},{\mathbb{F}}) \leftarrow {\mbox{\rm H}}^{p-1}(C_{i,j'},{\mathbb{F}}) \leftarrow \cdots 
 \]
 which is also equivariant.
 Let $n = {\mathrm{ind}}^-({\mathbf{x}}^{i,j'})$, and assume $n \neq 0$.
 In this case, ${\mbox{\rm H}}^p(C_{i,j'},{\mathbb{F}})=0$ unless $p=0,n-1$. 
 Now, ${\mbox{\rm H}}^{n-1}(C_{i,j'},{\mathbb{F}})$ is a direct sum of ${\mathrm{card}}({\mathrm{orbit}}({\mathbf{x}}^{i,j'}))$, each summand
 is stable under the action of a 
 subgroup of $G$ each isomorphic to $G_{{\mathbf{x}}^{i,j'}}$, and is thus
 a one-dimensional representation  of $G_{{\mathbf{x}}^{(i,j')}}$ which we denote by
 by $W_{{\mathbf{x}}^{i,j'}}$. It follows that the representation ${\mbox{\rm H}}^{n-1}(C_{i,j'},{\mathbb{F}})$ is the induced representation ${\mathrm{Ind}}_{G_{{\mathbf{x}}^{i,j'}}}^{G}(W_{{\mathbf{x}}^{i,j'}})$.
 From the Mayer-Vietoris sequence it is evident that either
 \begin{enumerate}[(i)]
 \item
 \[
 {\mbox{\rm H}}^{n}(S_{i,j},{\mathbb{F}}) = {\mbox{\rm H}}^{n}(S_{i,j-1},{\mathbb{F}}) \oplus  {\mbox{\rm H}}^{n-1}(C_{i,j'},{\mathbb{F}}),
 \]
 or
 \item
 \[
 {\mbox{\rm H}}^{n-1}(S_{i,j},{\mathbb{F}}) \oplus {\mbox{\rm H}}^{n-1}(C_{i,j'},{\mathbb{F}}) \cong {\mbox{\rm H}}^{n}(S_{i,j-1},{\mathbb{F}}).
 \]
\end{enumerate}
 These two cases corresponds to \eqref{item:theorem-main:a} and \eqref{item:theorem-main:b}
 respectively.
 Finally, we extend the proof to general ${\mathrm{R}}$ using the Tarski-Seidenberg transfer principle in 
 the usual way (see \cite[Chapter 7]{BPRbook2} for example).
\end{proof}

\subsection{Degree principle for the action of symmetric group on ${\mathrm{R}}^k$}
\label{subsec:degree-principle}
In this section we prove an important proposition that forms the basis of all our quantitative results.
It generalizes to the multi-symmetric case (i.e. for ${\mathbf{m}}$ not necessarily equal to $(1,\ldots,1)$) 
a similar result proved earlier (see \cite{Riener,Timofte03,BC2013}).

\begin{notation}
 Let ${\mathbf{k}} = (k_1,\ldots,k_\ell),{\mathbf{m}} = (m_1,\ldots,m_\ell),
 \mathbf{p} =(p_1,\ldots,p_\ell)  \in {\mathbb{N}}^\ell$, and let
 \[
 k = \sum_{1 \leq i \leq \ell} k_i m_i.
 \]
 
 We denote by $A_{{\mathbf{k}},{\mathbf{m}}}^{\mathbf{p}}$ the
  subset of ${\mathrm{R}}^{k}$ defined by
  \begin{eqnarray*}
    A^{\mathbf{p}}_{{\mathbf{k}},{\mathbf{m}}} & = & \left\{ {\mathbf{x}}= ({\mathbf{x}}^{(1)} , \ldots, {\mathbf{x}}^{(
    \ell)}) \mid  \ensuremath{\operatorname{card}} 
    \left(\bigcup_{j=1}^{k_i} \{ {\mathbf{x}}_{j}^{(i)} \} \right) = p_{i} \right\} .
  \end{eqnarray*}
  
  In the special case when $\ell=1,m=1$, we define for $p \leq k$,
  \begin{eqnarray*}
    A^{p}_{k} & = & \left\{ {\mathbf{x}}=(x_1,\ldots,x_k)  \mid  \ensuremath{\operatorname{card}} \left(
    \bigcup_{j=1}^{k} \{ x_j \} \right) = p \right\} .
  \end{eqnarray*}
\end{notation}

Let ${\mathbf{k}} = (k_1,\ldots,k_\ell),{\mathbf{m}} = (m_1,\ldots,m_\ell),
 {\mathbf{d}} =(d_1,\ldots,d_\ell)  \in {\mathbb{N}}^\ell$,
  Let $K= \sum_{i=1}^{\ell} k_{i}$, and $P
  \in {\mathrm{R}} [ {\mathbf{X}}^{(1)} , \ldots, {\mathbf{X}}^{(\ell)} ]$, where each
  for $1\leq h \leq \ell$, ${\mathbf{X}}^{(h)}$ is a block of $m_h\times k_{h}$ variables, 
  $\left( X^{(h)}_{i,j}\right)_{1\leq i \leq m_h,1\leq j\leq k_h}$, such that
  that $P$ is non-negative and $\mathfrak{S}_{\mathbf{k}}$-symmetric.
Also, suppose that $\deg_{{\mathbf{X}}^{(h)}} (P) \leq d_h, 1\leq h \leq \ell$.

The following proposition generalizes (to the case ${\mathbf{m}} \neq (1,\ldots,1)$) 
Proposition 6 in \cite{BC2013}. 
\begin{proposition}[Degree Principle]
  \label{prop:half-degree}
  Let 
  \[
  e =\sum_{1 \leq h \leq \ell} \sum_{1 \leq i \leq m_h}\sum_{1 \leq j \leq k_h}
X^{(h)}_{i,j}. 
\]

Let $\mathcal{C}$ denote the set of critical points of $e$ restricted to 
$W= {{\rm Z}} \left(P, {\mathrm{R}}^{K}\right)$, and suppose that $\mathcal{C}$ is a finite set. Then, 
\[
\mathcal{C} \subset \bigcup_{\mathbf{p} \leq {\mathbf{d}}^{\mathbf{m}}} A^{\mathbf{p}}_{{\mathbf{k}},{\mathbf{m}}}.
\] 
\end{proposition}
\begin{proof}
For ${\mathbf{m}}= \mathbf{1} := (1,\ldots,1) $, the proposition follows immediately from \cite[Proposition 6]{BC2013}.
Suppose that 
${\mathbf{x}} = ({\mathbf{x}}^{(1)},\ldots,{\mathbf{x}}^{(\ell)}) \in \mathcal{C}$. 
For ${\mathbf{x}} = ({\mathbf{x}}^{(1)},\ldots,{\mathbf{x}}^{(\ell)}) \in {\mathrm{R}}^k$,
and $\mathbf{i} = (i_1,\ldots,i_\ell) \in [1,m_1] \times \cdots \times [1,m_\ell] $ 
denote by $\bar{\mathbf{x}}_{\mathbf{i}} =({\mathbf{x}}^{(1)}_{i_1},\ldots,{\mathbf{x}}^{(\ell)}_{i_\ell}) \in {\mathrm{R}}^{k'}$.
It follows from the case $\mathbf{1}= (1,\ldots,1) $ (\cite[Proposition 6]{BC2013}), that 
for each $\mathbf{i} \in [1,m_1] \times \cdots \times [1,m_\ell]$,
\[
{\mathbf{x}}_{\mathbf{i}} \in \bigcup_{\mathbf{p} \leq {\mathbf{d}}} A^{\mathbf{p}}_{{\mathbf{k}},\mathbf{1}}.
\]

This proves that for each ${\mathbf{x}}  = ({\mathbf{x}}^{(1)},\ldots,{\mathbf{x}}^{(\ell)}) \in \mathcal{C}$,
each row of each $(m_h \times k_h)$-matrix ${\mathbf{x}}^{(h)}$ has at most $d$ distinct entries,
and this implies that the matrix ${\mathbf{x}}^{(h)}$ has at most $d^{m_h}$ distinct columns.
This implies that  
\[
{\mathbf{x}} \in \bigcup_{\mathbf{p} \leq {\mathbf{d}}^{\mathbf{m}}} A^{\mathbf{p}}_{{\mathbf{k}},{\mathbf{m}}},
\] 
which proves the proposition.
\end{proof}

\subsection{Deformation}
\label{subsec:deformation}
Let ${\mathbf{k}} =(k_1,\ldots,k_\ell),{\mathbf{m}}=(m_1,\ldots,m_\ell) \in {\mathbb{N}}^\ell, K = \sum_{i=1}^\ell k_i m_i$, and $d \geq 0$. Following the notation introduced previously,
\begin{notation}
  \label{not:def}For any $P \in {\mathrm{R}} [ {\mathbf{X}}^{(1)} , \ldots, {\mathbf{X}}^{(\ell)} ]$ we denote
  \[ {\mathrm{Def}} (P, \zeta ,d) = P -  \zeta   \left(1+ \sum_{1\leq h\leq \ell}\sum_{1\leq i \leq m_h} \sum_{1\leq j\leq k_h}  (X^{(h)}_{i,j})^{d} \right), 
  \]
  where $\zeta$ is a new variable.
\end{notation}

Notice that if $P$ is $\mathfrak{S}_{\mathbf{k}}$-symmetric,
then so is ${\mathrm{Def}} (P,\zeta ,d)$.

\begin{proposition}\cite[Proposition 4]{BC2013}
  \label{prop:alg-to-semialg}
 Let $d$ be even, and suppose that $V = {{\rm Z}} \left(P, {\mathrm{R}}^{K} \right)$ is bounded. 
 The variety
  ${\mathrm{Ext}} \left(V, {\mathrm{R}} \langle \zeta \rangle^{K} \right)$ is a semi-algebraic
  deformation retract of the (symmetric) semi-algebraic subset $S$ of ${\mathrm{R}}
  \langle \zeta \rangle^{K}$ defined by the inequality 
  \[
  {\mathrm{Def}} (P, \zeta ,d) \leq 0,
  \] 
  and hence is semi-algebraically homotopy equivalent to $S$.
 \end{proposition}

\begin{proposition}\cite[Proposition 5]{BC2013}.
  \label{prop:non-degenerate}
  Let 
  $P \in {\mathrm{R}} [ {\mathbf{X}}^{(1)} , \ldots, {\mathbf{X}}^{(\ell)} ]$,
  and
  $d$ be an even number with $\deg (P) \leq d=p+1$, with $p$ a prime. Let
  \[
  e = \sum_{1\leq h\leq \ell}\sum_{1 \leq i\leq m_h} \sum_{1\leq j\leq k_h}  (X^{(h)}_{i,j})^{d},
  \] and
  \[
  V_{\zeta} = {{\rm Z}} \left({\mathrm{Def}} (P,\zeta ,d) , {\mathrm{R}} \langle \zeta \rangle^{K} \right).
  \] 
  Suppose also that $\gcd(p,K) =1$. Then, the critical points of $e$ restricted to $V_{\zeta}$ are
  finite in number, and each critical point is non-degenerate.
\end{proposition}

    
\subsection{Representation theory of products of symmetric groups}
\label{subsec:representation-of-Sn}
We begin with some notation.

\begin{notation}[Product of symmetric groups and certain subgroups]
For $\lambda =(\lambda_1,\ldots, \lambda_d) \in {\mathrm{Par}}(k)$, we will denote by 
$\mathfrak{S}_{\pmb{\lambda}} \cong \mathfrak{S}_{\lambda_1} \times \cdots \times \mathfrak{S}_{\lambda_d}$ the subgroup of 
$\mathfrak{S}_k$ which is the direct product of the subgroups $G_i \cong \mathfrak{S}_{\lambda_i}$, where $G_i$ is the subgroup
of permutations of $[1,k]$ fixing $[1,k] \setminus [\lambda_1+\cdots+\lambda_{i-1}+1, \lambda_1+\cdots+\lambda_i]$.

More generally, for ${\mathbf{k}} = (k_1,\ldots,k_\ell) \in {\mathbb{N}}^\ell$, $\pmb{\lambda} = (\lambda^{(1)},\ldots,\lambda^{(\ell)}) \in {\mathrm{Par}}({\mathbf{k}})$, we denote by 
$\mathfrak{S}_{\pmb{\lambda}}$ the subgroup $\mathfrak{S}_{\lambda^{(1)}} \times \cdots \times \mathfrak{S}_{\lambda^{(\ell)}}$ of
$\mathfrak{S}_{\mathbf{k}}$, where for $1 \leq j \leq \ell$, $\mathfrak{S}_{\lambda^{(j)}}$ is the subgroup of $\mathfrak{S}_{k_j}$ defined
above. 
\end{notation}

\begin{notation}[Irreducible representations of symmetric groups]
For $\lambda \in {\mathrm{Par}}(k)$, we will denote by $\mathbb{S}^{\lambda}$ the irreducible representation of 
$\mathfrak{S}_k$ corresponding to
$\lambda$ (see \cite{Procesi-book} for definition). 
Note that $\mathbb{S}^{(k)}$ is the trivial representation (corresponding to the  partition $(k) \in {\mathrm{Par}}(k)$
(which we also denote by $\mathbf{1}_{\mathfrak{S}_k}$), and 
$\mathbb{S}^{(1^k)}$ is the sign representation, which we will also denote by $\mathbf{sign}_k$.
It is  well known fact that for any $\lambda \in {\mathrm{Par}}(k)$, 
\[
\mathbb{S}^{(\tilde{\lambda})} \cong \mathbb{S}^{(\lambda)} \otimes \mathbf{sign}_k.
\]

For ${\mathbf{k}} = (k_1,\ldots,k_\ell) \in {\mathbb{N}}^\ell$, $\pmb{\lambda} = (\lambda^{(1)},\ldots,\lambda^{(\ell)}) \in {\mathrm{Par}}({\mathbf{k}})$, we denote by $\mathbb{S}^{\pmb{\lambda}}$ the irreducible representation 
$\mathbb{S}^{\lambda^{(1)}} \boxtimes \cdots \boxtimes \mathbb{S}^{\lambda^{(\ell)}}$ of $\mathfrak{S}_{\mathbf{k}}$.
\end{notation}

The following classical formula (due to Frobenius) gives the dimensions of the representations $\mathbb{S}^{\lambda}$ in terms of the \emph{hook lengths} of the partition $\lambda$ defined below.

\begin{definition}[Hook lengths]
Let $B(\lambda)$ denote the set of boxes in the  Young diagram corresponding to a partition 
$\lambda \vdash k$. For a box $b \in B(\lambda)$, the length of the hook of $b$, denoted 
$h_b$ is the number of boxes strictly to the right and below $b$ plus $1$.
\end{definition}

\begin{theorem}[Hook length formula]
\label{thm:hook}
Let $\lambda \vdash k$. Then,
\[
\dim_{\mathbb{F}} \mathbb{S}^{\lambda} =  \frac{k!}{\prod_{b \in B(\lambda)} h_b}.
\]
\end{theorem}

\begin{definition}[Young module]
\label{def:Young}
For $\lambda \vdash k$, we will denote 
\[
M^\lambda = {\mathrm{Ind}}_{\mathfrak{S}_\lambda}^{\mathfrak{S}_k} (\mathbf{1}_{\mathfrak{S}_\lambda})
\] 
(where $\mathbf{1}_{\mathfrak{S}_\lambda}$ denotes the trivial one-dimensional representation
of $\mathfrak{S}_\lambda$).
\end{definition}

\begin{notation}[Induced representations and multiplicities]
\label{not:induced-rep-and-mult}
For each $\lambda \vdash k$,
we denote by $\overline{\mathrm{Par}}(\lambda)$ the set of partitions $\mu \vdash k$ such that,
there exists a decomposition $\lambda = \lambda' \coprod \lambda''$, 
$\lambda' = (\lambda'_1,\ldots,\lambda'_{\ell'}), \lambda''=(\lambda''_1,\ldots,\lambda''_{\ell''}), \ell'+\ell''=\ell = {\mathrm{length}}(\lambda)$,such that $\mathbb{S}^\mu$ occurs with positive multiplicity
in the representation 
\[
\mathbb{S}_{\lambda',\lambda''} := 
{\mathrm{Ind}}_{\mathfrak{S}_{\lambda'} \times \mathfrak{S}_{\lambda''}}^{\mathfrak{S}_k}\left(\left(\boxtimes_{i=1}^{\ell'} \mathbb{S}^{(\lambda'_i)}\right) \boxtimes \left(\boxtimes_{j=1}^{\ell''} \mathbb{S}^{(1^{\lambda''_j})}\right) \right),
\] 
and we denote the multiplicity of $\mathbb{S}^\mu$ in $\mathbb{S}_{\lambda',\lambda''}$  by $m^\mu_{\lambda',\lambda''}$.

More generally, for ${\mathbf{k}} \in {\mathbb{N}}^\ell$, and $\pmb{\lambda}= (\lambda^{(1)},\ldots,\lambda^{(\ell)}) \in {\mathrm{Par}}({\mathbf{k}})$, we denote by 
$\overline{\mathrm{Par}}(\pmb{\lambda}) = \overline{\mathrm{Par}}(\lambda^{(1)}) \times \cdots \times \overline{\mathrm{Par}}(\lambda^{(\ell)})$.
\end{notation}

\begin{definition}[Dominance order]
\label{def:dominance}
For any two partitions $\mu=(\mu_1,\mu_2,\ldots),\lambda=(\lambda_1,\lambda_2,\ldots) \in {\mathrm{Par}}(k)$,
we say that $\mu {{\;\underline{\triangleright}\;}} \lambda$, if for each $i\geq 0$, $\mu_1+\cdots+\mu_i \geq \lambda_1+\cdots+\lambda_i$.
This is a partial order on ${\mathrm{Par}}(k)$.
More generally, for ${\mathbf{k}} = (k_1,\ldots,k_\ell) \in {\mathbb{N}}^\ell$,
and $\pmb{\mu} = (\mu^{(1)},\ldots,\mu^{(\ell)}), \pmb{\lambda} = (\lambda^{(1)},\ldots,\lambda^{(\ell)}) \in {\mathrm{Par}}({\mathbf{k}})$, we denote
$\pmb{\mu} {{\;\underline{\triangleright}\;}} \pmb{\lambda}  $ if and only if $ \mu^{(i)} {{\;\underline{\triangleright}\;}} \lambda^{(i)} $ for each $i, 1\leq i \leq \ell$.
\end{definition}

We also need the definitions of Kostka numbers and the Littlewood-Richardson coefficients.

\begin{definition}[Kostka numbers]
\label{def:Kostka}
For $\lambda,\mu \vdash k$, $K(\mu,\lambda)$ denotes the number of semi-standard Young tableux of
shape $\mu$ and weight $\lambda$ (see \cite{Procesi-book} for definitions of semi-standard Young tableaux, and 
also their shape and weight). 
\end{definition} 

The following fact is very basic  (see for example \cite[Theorem 3.6.11]{Ceccherini-book} or \cite[page 541, Section 7.3]{Procesi-book}).

\begin{proposition}[Young's rule]
\label{prop:Young}
Let $k \in {\mathbb{N}}$, and $\lambda\in {\mathrm{Par}}(k)$. Then,
\[
{\mathrm{Ind}}_{\mathfrak{S}_\lambda}^{\mathfrak{S}_k} \big(\mathbb{S}^{(\lambda_1)}\boxtimes \cdots \boxtimes \mathbb{S}^{(\lambda_{{\mathrm{length}}(\lambda)})} \big) \cong \bigoplus_{\mu {{\;\underline{\triangleright}\;}}  \lambda} K(\mu,\lambda) \mathbb{S}^{\mu}.
\]
\end{proposition}

\begin{definition}[Littlewood-Richardson coefficients]
\label{def:LR}
For $\lambda \vdash m, \mu \vdash n, \nu \vdash m+n$, $c^\nu_{\lambda,\mu}$ is the multiplicity
of the irreducible representation $\mathbb{S}^\nu$ in 
${\mathrm{Ind}}_{\mathfrak{S}_m \times \mathfrak{S}_n}^{\mathfrak{S}_{m+n}}(\mathbb{S}^\lambda \boxtimes \mathbb{S}^\mu)$.
\end{definition}

\begin{proposition}
\label{prop:multiplicity}
Let $k,d > 0$, $\lambda \in {\mathrm{Par}}(k,d)$
such that $\lambda = \lambda' \coprod \lambda''$,
and $\mu  \in \overline{\mathrm{Par}}(\lambda)$.
Then,
\begin{enumerate}
\item
\label{item:prop:multiplicity1}
\[
{\mathrm{card}}(\{ i \mid \mu_i \geq d \}) \leq d, \\
{\mathrm{card}}(\{j \mid \tilde{\mu}_j \geq d\}) \leq d,
\]
\item
\label{item:prop:multiplicity2}
\begin{eqnarray}
\label{eqn:prop:multiplicity2}
m^\mu_{\lambda',\lambda''} &=& 
\sum_{ \substack{\nu' \vdash |\lambda'|,\nu' {{\;\underline{\triangleright}\;}} \lambda' \\ \nu'' \vdash |\lambda''|,\nu'' {{\;\underline{\triangleright}\;}} \widetilde{\lambda''}}} K(\nu',\lambda')\cdot K(\nu'',\widetilde{\lambda''})\cdot c^{\mu}_{\nu',\nu''},
\end{eqnarray}
\item
\label{item:prop:multiplicity3}
\begin{eqnarray*}
\label{eqn:prop:multiplicity3}
\sum_{\mu \vdash k} m^\mu_{\lambda',\lambda''}&\leq& 
k^{O(d^2)}.
\end{eqnarray*}
\end{enumerate}
\end{proposition}

\begin{remark}
It is well known that
$K(\mu,\mu) = 1$ for all $\mu \in {\mathrm{Par}}(k)$, $K(\mu,\lambda)= 0$ unless $\mu {{\;\underline{\triangleright}\;}} \lambda$. Finally, if $\mu$ is the maximal element in the dominance ordering ${{\;\underline{\triangleright}\;}}$ on ${\mathrm{Par}}(k)$, that is $\mu = (k)$, then $K(\mu,\lambda) = 1$ for all $\lambda \in {\mathrm{Par}}(k)$.
In particular, in conjunction with Schur's lemma the above fact implies, that the trivial representation, $\mathbb{S}^{(k)}$ occurs with multiplicity equal to $1 (= K((k),\lambda))$ in ${\mathrm{Ind}}_{\mathfrak{S}_\lambda}^{\mathfrak{S}_k}\big(\boxtimes_{j=1}^{{\mathrm{length}}(\lambda)}\mathbb{S}^{(\lambda_j)}\big)$.
\end{remark}

\begin{remark}
Note also that the representation
${\mathrm{Ind}}_{\mathfrak{S}_\lambda}^{\mathfrak{S}_k}(\boxtimes_{j=1}^{{\mathrm{length}}(\lambda)} \mathbb{S}^{(\lambda_j)})$
is isomorphic to the permutation representation
of $\mathfrak{S}_k$ on the set of cosets $\mathfrak{S}_k/\mathfrak{S}_\lambda$, and in particular
\[
\dim_{\mathbb{F}} {\mathrm{Ind}}_{\mathfrak{S}_\lambda}^{\mathfrak{S}_k}(\boxtimes_{j=1}^{{\mathrm{length}}(\lambda)} \mathbb{S}^{(\lambda_j)})
=
\frac{k!}{\prod_{1\leq j \leq {\mathrm{length}}(\lambda)} \lambda_j!}.
\]
\end{remark}

\begin{definition}[Skew partitions, horizontal and vertical strips]
\label{def:skew-partition-strips}
For any two partitions, $\lambda=(\lambda_1,\lambda_2,\ldots) \vdash m$, $\mu=(\mu_1,\mu_2,\ldots) \vdash n$, $m \leq n$,  we say that $\lambda \subset \mu$,
if $\lambda_i \leq \mu_i$ for all $i$.

Identifying $\lambda,\mu$ with their respective Young diagrams,
we say that the \emph{skew partition} $\mu/\lambda$ is a \emph{horizontal strip} if no two cells
of $\mu/\lambda$ belong to the same \emph{column}. 
We say that $\mu/\lambda$ is a \emph{vertical strip}
if no two cells of $\mu/\lambda$ belong to the same \emph{row}.
\end{definition}

\begin{proposition}[Pieri's rule]
\label{prop:Pieri}
For $\lambda \vdash m$, and $n \geq 0$, we have the two following relations.
\begin{eqnarray*}
{\mathrm{Ind}}_{\mathfrak{S}_m \times \mathfrak{S}_n}^{\mathfrak{S}_{m+n}} (\mathbb{S}^\lambda \boxtimes \mathbb{S}^{(n)}) 
&\cong& \bigoplus_{\substack{\mu \vdash m+n \\ \mu/\lambda \mbox{ is a horizontal strip}}} \mathbb{S}^\mu, \\
{\mathrm{Ind}}_{\mathfrak{S}_m \times \mathfrak{S}_n}^{\mathfrak{S}_{m+n}} (\mathbb{S}^\lambda \boxtimes \mathbb{S}^{1^n}) 
&\cong& \bigoplus_{\substack{\mu \vdash m+n \\ \mu/\lambda \mbox{ is a vertical strip}}} \mathbb{S}^\mu.
\end{eqnarray*}
\end{proposition}

We also have the following associativity relationship that allows us to apply Pieri's rule (Proposition \ref{prop:Pieri})
iteratively.

\begin{proposition}
\label{prop:associative}
Let $n = m_1+\cdots+m_\ell$, where for each $i, 1\leq i \leq \ell$. Then,
\[
{\mathrm{Ind}}_{\mathfrak{S}_{m_1} \times\cdots \times \mathfrak{S}_{m_\ell}}^{\mathfrak{S}_{n}} 
(V_1 \boxtimes \cdots \boxtimes V_\ell)
\]
is isomorphic to 
\[
{\mathrm{Ind}}_{\mathfrak{S}_{m_1+\ldots+m_{\ell-1}} \times \mathfrak{S}_{m_\ell}}^{\mathfrak{S}_n}
({\mathrm{Ind}}_{\mathfrak{S}_{m_1} \times \cdots \times \mathfrak{S}_{m_{\ell-1}}}^{\mathfrak{S}_{m_1+\cdots+m_{\ell-1}}}
(V_1 \boxtimes \cdots \boxtimes V_{\ell-1}) \boxtimes V_\ell),
\]
where for each $i, 1\leq i \leq \ell$, $V_i$ is an $\mathfrak{S}_{m_i}$-module.
\end{proposition}
 

\begin{proof}[Proof of Proposition \ref{prop:multiplicity}]
We first prove \eqref{item:prop:multiplicity2}.
Let $k'=|\lambda'|$ and $k'' = |\lambda''|$. Then, using 
Young's rule (Proposition \ref{prop:Young})
\begin{eqnarray*}
{\mathrm{Ind}}_{\mathfrak{S}_{\lambda'}}^{\mathfrak{S}_{k'}}\left(\boxtimes_{i=1}^{\ell'} \mathbb{S}^{(\lambda'_i)}\right)
&\cong&
\bigoplus_{\nu' \vdash k',\nu' {{\;\underline{\triangleright}\;}} \lambda'} K(\nu',\lambda')\mathbb{S}^{\nu'}, \\
{\mathrm{Ind}}_{\mathfrak{S}_{\lambda''}}^{\mathfrak{S}_{k''}}\left(\boxtimes_{i=1}^{\ell''} \mathbb{S}^{1^{\lambda''_i}}\right)
&\cong&
\bigoplus_{\nu'' \vdash k',\nu'' {{\;\underline{\triangleright}\;}} \widetilde{\lambda''}} K(\nu'',\widetilde{\lambda''})\mathbb{S}^{\nu''}.
\end{eqnarray*}
It follows that
\begin{eqnarray*}
 {\mathrm{Ind}}_{\mathfrak{S}_{\lambda'} \times \mathfrak{S}_{\lambda''}}^{\mathfrak{S}_{k'} \times \mathfrak{S}_{k''}}\left(\left(\boxtimes_{i=1}^{\ell'} \mathbb{S}^{(\lambda'_i)}\right) \boxtimes \left(\boxtimes_{j=1}^{\ell''} \mathbb{S}^{(1^{\lambda''_j})}\right) \right)
 \end{eqnarray*}
 is isomorphic to 
 \begin{eqnarray*}
 \bigoplus_{ \substack{\nu' \vdash k', \nu' {{\;\underline{\triangleright}\;}} \lambda' \\ \nu'' \vdash k'' , \nu'' {{\;\underline{\triangleright}\;}} \widetilde{\lambda''}}} K(\nu',\lambda')K(\nu'',\widetilde{\lambda''}) \mathbb{S}^{\nu'} \boxtimes \mathbb{S}^{\nu''}.
 \end{eqnarray*}
Eqn. \eqref{eqn:prop:multiplicity2} then follows from the isomorphism
\begin{eqnarray*}
\mathbb{S}_{\lambda',\lambda''} &\cong&
{\mathrm{Ind}}_{\mathfrak{S}_{k'} \times \mathfrak{S}_{k''}}^{\mathfrak{S}_k} {\mathrm{Ind}}_{\mathfrak{S}_{\lambda'} \times \mathfrak{S}_{\lambda''}}^{\mathfrak{S}_{k'} \times \mathfrak{S}_{k''}}\left(\left(\boxtimes_{i=1}^{\ell'} \mathbb{S}^{(\lambda'_i)}\right) \boxtimes \left(\boxtimes_{j=1}^{\ell''} \mathbb{S}^{(1^{\lambda''_j})}\right) \right)
\end{eqnarray*}
and the definition of the Littlewood-Richardson's coefficients, $c^{\mu}_{\nu',\nu''}$ (Definition \ref{def:LR}).

An alternative way of obtaining the multiplicities $m^{\mu}_{\lambda',\lambda''}$ is by applying
Pieri's rule iteratively at most  $d$ times using Propositions \ref{prop:associative} and \ref{prop:Pieri}.
Let ${\mathrm{length}}(\lambda') = \ell',{\mathrm{length}}(\lambda'') = \ell''$, so that 
$\ell' + \ell'' = {\mathrm{length}}(\lambda) \leq d$. 

Let for $1 \leq i \leq  \ell'$,
\[
M_i = {\mathrm{Ind}}_{\mathfrak{S}_{\lambda'_1 + \ldots +\lambda'_{i-1}} \times \mathfrak{S}_{\lambda'_i}}^{\mathfrak{S}_{\lambda'_1 + \ldots +\lambda'_{i}}}(M_{i-1}\boxtimes \mathbb{S}^{(\lambda'_i)}),
\]
with the convention that $M_0 = \mathbf{1}$.
For $\nu \vdash \lambda'_1 + \ldots +\lambda'_{i}$, let $m^\nu_i$ denote the multiplicity of 
$\mathbb{S}^\nu$ in $M_i$, and $m_i = \sum_{\nu \vdash \lambda'_1 + \ldots +\lambda'_{i}}  m^\nu_i$.
We prove by induction on $i$ the following two statements.
\begin{enumerate}[(a)]
\item
\label{item:multiplicity:a}
\[
m_i \leq m_{i-1} \cdot \binom{\lambda'_i+i-1}{i-1}.
\]
\item
\label{item:multiplicity:b}
For each $\nu \vdash \lambda'_1 + \ldots +\lambda'_{i}$, such that
$m^\nu_i > 0$, ${\mathrm{length}}(\nu) \leq i$.
\end{enumerate}

Assuming statements \eqref{item:multiplicity:a} and \eqref{item:multiplicity:b} hold for $i-1$ we prove them for $i$.
By induction for each $\nu' \vdash \lambda'_1 + \ldots +\lambda'_{i-1}$ with $m^{\nu'}_{i-1} > 0$,
${\mathrm{length}}(\nu') \leq i-1$. Applying Pieri's rule (Proposition \ref{prop:Pieri}) we obtain 
\begin{eqnarray}
\label{eqn:multiplicity:1}
{\mathrm{Ind}}_{\mathfrak{S}_{\lambda'_1+\cdots+\lambda'_{i-1}}\times \mathfrak{S}_{\lambda'_{i}}}^{\mathfrak{S}_{\lambda'_1+\cdots+\lambda'_i}} (\mathbb{S}^{\nu'} \boxtimes \mathbb{S}^{(\lambda'_i)}) 
&\cong& \bigoplus_{\substack{\nu \vdash \lambda'_1+\cdots+\lambda'_i \\ \nu/\nu' \mbox{ is a horizontal strip}}} \mathbb{S}^\nu.
\end{eqnarray}
Observe that each choice of $\nu \vdash \lambda'_1+\cdots+\lambda'_i$ such that  
$\nu/\nu'$is a horizontal strip, corresponds uniquely to a composition of $\lambda'_i$ into at most
${\mathrm{length}}(\nu')$ parts, and the number of such compositions is clearly bounded by 
$\binom{\lambda'_i+i-1}{i-1}$. This proves part \eqref{item:multiplicity:a}. 
Part \eqref{item:multiplicity:b} also follows from \eqref{eqn:multiplicity:1} noting that the length of each $\mu$ that occurs on the 
right is at most ${\mathrm{length}}(\mu')+1$ which is $\leq i$ using the induction hypothesis.
This complete the proof of parts \eqref{item:multiplicity:a}  and \eqref{item:multiplicity:b}.
 
Now let for $1 \leq j \leq  \ell''$,
\[
N_j = {\mathrm{Ind}}_{\mathfrak{S}_{\ell' + \lambda''_1 + \ldots +\lambda''_{i-1}} \times \mathfrak{S}_{\lambda''_j}}^{\mathfrak{S}_{\ell' + \lambda''_1 + \cdots +\lambda''_{j}}}(N_{j-1}\boxtimes \mathbb{S}^{(\lambda''_j)}),
\]
with the convention that $N_0 = M_{\ell'}$.
For $\nu \vdash \lambda''_1 + \ldots +\lambda''_{j}$, let $n^\nu_j$ denote the multiplicity of 
$\mathbb{S}^\nu$ in $N_j$, and $n_j = \sum_{\nu \vdash \lambda''_1 + \cdots +\lambda''_{j}}  n^\nu_j$.

The following two statements from an induction on $j$. The proofs are very similar to the proofs
of \eqref{item:multiplicity:a} and \eqref{item:multiplicity:b} above and are omitted.

\begin{enumerate}[(a)]
\addtocounter{enumi}{2}
\item
\label{item:multiplicity:c}
\[
n_j \leq n_{j-1} \cdot \binom{\lambda''_j+\ell'+j-1}{\ell'+j-1}.
\]
\item
\label{item:multiplicity:d}
For each $\nu \vdash \lambda''_1 + \cdots +\lambda''_{j}$, such that
$n^\nu_j > 0$, ${\mathrm{length}}(\widetilde{\nu}) \leq \ell'+j$.
\end{enumerate}

It follows from \eqref{item:multiplicity:a}, \eqref{item:multiplicity:b}, \eqref{item:multiplicity:c}, and \eqref{item:multiplicity:d}, that
\[
\sum_{\mu\vdash k} m^\mu_{\lambda',\lambda''} \leq k^{O(d^2)},
\]
which proves \eqref{item:prop:multiplicity3}.
Finally, it is easy to check that for each $\mu$ with $m^{\mu}_{\lambda',\lambda''} > 0$
that arises in the above process satisfies
\[
{\mathrm{card}}(\{ i \mid \mu_i \geq d \}) \leq d, \\
{\mathrm{card}}(\{j \mid \tilde{\mu}_j \geq d\}) \leq d,
\]
which proves \eqref{item:prop:multiplicity1}.
\end{proof} 

\begin{remark}
\label{rem:multiplicity}
The following particular cases of Proposition \ref{prop:multiplicity} will be of interest.
\begin{enumerate}
\item
If $\lambda' = \lambda$ (and hence, $\lambda''$ is the empty partition),
\[m^\mu_{\lambda',\lambda''} = K(\mu,\lambda).
\]
\item
If $\mu=(k)$, 
\[m^\mu_{\lambda',\lambda''} = 1.
\]
\end{enumerate}
\end{remark}

\begin{proof}[Proof of Theorem \ref{thm:restriction}]
The theorem follows from Part (\ref{item:prop:multiplicity1}) of Proposition \ref{prop:multiplicity}.
\end{proof}

\subsection{Equivariant Poincar\'e duality}
\label{subsec:Poincare-duality}
\begin{theorem}
\label{thm:poincare-duality}
Let $V \subset {\mathrm{R}}^k$ be a bounded smooth compact semi-algebraic oriented hypersurface, which is 
stable under the standard action of $\mathfrak{S}_k$ on ${\mathrm{R}}^k$. Then, for each $p,0 \leq p \leq k$,
there is an $\mathfrak{S}_k$-module isomorphism
\[
{\mbox{\rm H}}^p(V,{\mathbb{F}}) \xrightarrow{\sim} {\mbox{\rm H}}^{k-p-1}(V,{\mathbb{F}}) \otimes 
\mathbf{sign}_k.
\] 
\end{theorem}

\begin{proof}
If $M$ is a $C^0$-manifold of dimension $\ell$, then the following sheaf-theoretic statement
of Poincar\'e duality is well known (see for example \cite[Corollary 5.5.6]{Schapira-notes}).

\begin{equation}
\label{eqn:iso1}
{\textrm{hom}}_{\mathbb{F}}({\mbox{\rm H}}^*_c(M;{\mathbb{F}}_M),{\mathbb{F}})  \cong {\mbox{\rm H}}^*(M;\mathrm{or}_M)[\ell].
\end{equation}

In our case, with $M=V$.  The $\mathfrak{S}_k$-action on the ambient space ${\mathrm{R}}^k$, induces
an $\mathfrak{S}_k$-module structure on ${\mbox{\rm H}}^*(V;{\mathbb{F}}_V)$ by the induced isomorphisms
$\pi^*: {\mbox{\rm H}}^*(V;{\mathbb{F}}_V) \xrightarrow{\sim} {\mbox{\rm H}}^*(V;{\mathbb{F}}_V), \pi \in \mathfrak{S}_k$.

Now for $\pi \in \mathfrak{S}_k$ (and also denoting by $\pi$ the induced map $\pi:V \rightarrow V$), we have that $\pi$ induces the sign representation 
on the  
one dimensional vector space, $\Gamma(V; \mathrm{or}_V)$,  of global sections of the orientation sheaf on $V$.  
This implies the following $\mathfrak{S}_k$-isomorphism for each $p \geq 0$,
\begin{equation}
\label{eqn:iso2}
{\mbox{\rm H}}^p(V;\mathrm{or}_V) \cong {\mbox{\rm H}}^p(V;{\mathbb{F}}_V) \otimes 
\textbf{sign}_k.
\end{equation}
The theorem follows from \eqref{eqn:iso1} and \eqref{eqn:iso2}, after noting that since $V$ is assumed to be compact
\[
{\textrm{hom}}_{\mathbb{F}}({\mbox{\rm H}}^*(V,{\mathrm{C}}),{\mathbb{F}}) \cong {\mbox{\rm H}}^*_c(V,{\mathbb{F}}) \cong  {\mbox{\rm H}}^*(V,{\mathbb{F}}) 
\]
where all isomorphisms are $\mathfrak{S}_k$-module isomorphisms.
\end{proof}

\subsection{Equivariant Mayer-Vietoris inequalities}

Suppose that $S_1,S_2 \subset {\mathrm{R}}^K$ are $\mathfrak{S}_{\mathbf{k}}$-symmetric closed semi-algebraic sets.
Then $S_1 \cup S_2$, and $S_1 \cap S_2$ are also $\mathfrak{S}_{\mathbf{k}}$-symmetric closed semi-algebraic sets, and  there is the classical Mayer-Vietoris exact sequence,
\[
\cdots \rightarrow {\mbox{\rm H}}^{i}(S_1\cup S_2,{\mathbb{F}}) \rightarrow {\mbox{\rm H}}^{i}(S_1,{\mathbb{F}}) \oplus {\mbox{\rm H}}^{i}(S_2,{\mathbb{F}}) \rightarrow {\mbox{\rm H}}^i(S_1\cap S_2,{\mathbb{F}}) \rightarrow {\mbox{\rm H}}^{i+1}(S_1 \cup S_2,{\mathbb{F}}) \rightarrow \cdots 
\]  
where all the homomorphisms are $\mathfrak{S}_{\mathbf{k}}$-equivariant. Denoting by ${\mbox{\rm H}}^*(S,{\mathbb{F}})_{\pmb{\mu}}$ the isotypic component of ${\mbox{\rm H}}^*(S,{\mathbb{F}})$ corresponding to $\pmb{\mu} \in {\mathrm{Par}}({\mathbf{k}})$ for any $\mathfrak{S}_{\mathbf{k}}$-symmetric closed semi-algebraic set $S \subset{\mathrm{R}}^k$, we
obtain using Schur's lemma for each $\pmb{\mu} \in {\mathrm{Par}}({\mathbf{k}})$, an exact sequence,
\[
\cdots \rightarrow {\mbox{\rm H}}^{i}(S_1\cup S_2,{\mathbb{F}})_{\pmb{\mu}} \rightarrow {\mbox{\rm H}}^{i}(S_1,{\mathbb{F}})_{\pmb{\mu}} \oplus {\mbox{\rm H}}^{i}(S_2,{\mathbb{F}})_{\pmb{\mu}} \rightarrow {\mbox{\rm H}}^i(S_1\cap S_2,{\mathbb{F}})_{\pmb{\mu}} \rightarrow {\mbox{\rm H}}^{i+1}(S_1 \cup S_2,{\mathbb{F}})_{\pmb{\mu}} \rightarrow \cdots 
\]  

The following inequalities follow from the above exact sequence (the proofs are similar to the 
non-equivariant case and can be found in \cite{BPRbook2}).

Let $S_{1} , \ldots ,S_{s} \subset {\mathrm{R}}^{K}$, $s \ge 1$, be $\mathfrak{S}_{\mathbf{k}}$-symmetric closed
semi-algebraic sets of ${\mathrm{R}}^{K}$, contained in a $\mathfrak{S}_{\mathbf{k}}$-symmetric 
closed semi-algebraic set $T$.

For $1 \leq t \leq s$, let $S_{\le t} = \bigcap_{1 \leq j \leq t} S_{j}$, and
$S^{\le t} = \bigcup_{1 \leq j \leq t} S_{j}$.
Also, for $J \subset \{1, \ldots ,s\}$, $J \neq \emptyset$, let $S_{J} =
\bigcap_{j \in J} S_{j}$, and
$S^{J} = \bigcup_{j \in J} S_{j}$. Finally, let $S^{\emptyset} =T$.

\begin{proposition}
  \label{7:prop:prop1}{$\mbox{}$}
   \begin{enumerate}[(a)]
    \item 
    \label{7:prop:prop1:itema}
    For $\pmb{\mu} \in {\mathrm{Par}}({\mathbf{k}})$ and $i \geq 0$,
    
     \begin{equation*}
      m_{i,\pmb{\mu}} (S^{\le s} ,{\mathbb{F}}) \leq \sum_{j=1}^{i+1}
      \sum_{\substack{
        J \subset \{ 1, \ldots ,s \}\\
        {\mathrm{card}} (J) =j}} 
       m_{i-j+1,\pmb{\mu}} (S_{J} ,{\mathbb{F}}) .
    \end{equation*}
    
    \item 
    \label{7:prop:prop1:itemb}
    For $\pmb{\mu} \in {\mathrm{Par}}({\mathbf{k}})$ and $0 \le i \le K$,
    
     \begin{equation*}
      \label{7:eqn:prop1} m_{i,\pmb{\mu}} (S_{\le s} ,{\mathbb{F}}) \leq \sum_{j=1}^{K-i}
      \sum_{\substack{
        J \subset \{ 1, \ldots ,s \}\\
        {\mathrm{card}} (J) =j}}
      m_{i+j-1,\pmb{\mu}} (S^{J} ,{\mathbb{F}}) + \binom{s}{K-i} m_{K,\pmb{\mu}}
      (S^{\emptyset} ,{\mathbb{F}}) .
    \end{equation*}
  \end{enumerate}
\end{proposition}

\begin{proof} Follows from the proof of 
\cite[Proposition 7.33]{BPRbook2} and Schur's lemma.
\end{proof}

\begin{proposition}
  \label{prop:MV}
  If $S_{1} ,S_{2}$ are $\mathfrak{S}_{\mathbf{k}}$-symmetric  closed semi-algebraic sets, then for
  $\pmb{\mu} \in {\mathrm{Par}}({\mathbf{k}})$, any field ${\mathbb{F}}$ and every $i \geq 0$
  \begin{eqnarray*}
    m_{i,\pmb{\mu}} (S_{1} ,{\mathbb{F}}) +m_{i,\pmb{\mu}} (S_{2} ,{\mathbb{F}}) & \leq & m_{i,\pmb{\mu}}
    (S_{1} \cup S_{2} ,{\mathbb{F}}) + m_{i,\pmb{\mu}} (S_{1} \cap S_{2} ,{\mathbb{F}}).
    \end{eqnarray*}
\end{proposition}
\begin{proof}
It follows from the proof of 
\cite[Proposition 6.44]{BPRbook2} and Schur's lemma.
\end{proof}

\subsection{Descent spectral sequence}
\label{subsec:descent}
We recall here a result proved in \cite{BC2013} that will be needed in the proof of 
Theorem \ref{thm:descent2-quantitative-new}.

We first introduce a notation.

 \begin{notation}[Symmetric product]
  We denote for each $p \geq 0$, ${\mathrm{Sym}}^{(p)} (X
 )$ the $(p+1)$-fold symmetric product of $X$ i.e.
  \begin{eqnarray*}
    {\mathrm{Sym}}^{(p)} (X) & = & \underbrace{X \times\cdots \times X}_{p+1} /\mathfrak{S}_{p+1} .
  \end{eqnarray*}
More generally, given a semi-algebraic map $f:X \rightarrow Y$,
we denote for each $p \geq 0$, by  ${\mathrm{Sym}}^{(p)}_{f} (X)$) the
  quotient
  $\underbrace{X \times_{f} \cdots \times_{f} X}_{p+1} /\mathfrak{S}_{p+1}$m
  where $\underbrace{X \times_{f} \cdots \times_{f} X}_{p+1}$ denote the $(p+1)$-fold fiber
  product of $f$.
\end{notation}

Now suppose that $X,Y$ are compact semi-algebraic sets and $f:X\twoheadrightarrow Y$ a continuous surjection,
and let $S \subset {\mathrm{R}}^{m} \times {\mathrm{R}}^{k}$ be a closed and bounded semi-algebraic set, and 
$\pi:{\mathrm{R}}^{m} \times {\mathrm{R}}^{k} \rightarrow {\mathrm{R}}^{k}$ the projection to the second factor. 

\begin{theorem}\cite{BC2013}
  \label{thm:descent2} With the above notation and for any field of
  coefficients ${\mathbb{F}}$
  \begin{eqnarray*}
    b (\pi (S) ,{\mathbb{F}}) & \leq & \sum_{0 \leq p<k} b ({\mathrm{Sym}}^{(p)}_{\pi} (S) ,{\mathbb{F}}) .
  \end{eqnarray*}
\end{theorem}

\section{Proofs of the main theorems}
\label{sec:proofs-of-main}

\subsection{Proofs of Theorems \ref{thm:main-product-of-symmetric}, \ref{thm:main-product-of-symmetric-quantitative} and \ref{thm:main-product-of-symmetric-quantitative-complex}}

\begin{proof}[Proof of Theorem \ref{thm:main-product-of-symmetric}]
First replace $V$ by the set $S$ defined by ${\mathrm{Def}}(P,d',\zeta) \leq 0$,
where $d'$ is the least even number such that $d' > d$ and
where $d'-1$ is prime. It follows from Bertrand's postulate that $d' \leq 2d$.
The Theorem follows from Proposition \ref{prop:non-degenerate}, 
Lemma   \ref{lem:equivariant_morseA},
Proposition \ref{prop:equivariant-morseb}, 
Proposition \ref{prop:half-degree}, and 
Proposition \ref{prop:Young}.
\end{proof}

\begin{proof}[Proof of Theorem \ref{thm:main-product-of-symmetric-quantitative}]
Theorem \ref{thm:main-product-of-symmetric-quantitative} follows from
Theorem \ref{thm:main-product-of-symmetric} and Proposition \ref{prop:multiplicity}.
\end{proof}

\begin{proof}[Proof of Theorem \ref{thm:main-product-of-symmetric-quantitative-complex}]
Substituting  ${\mathbf{X}}^{(j)} = {\mathbf{Y}}^{(j)}+ i {\mathbf{Z}}^{(j)}, 1 \leq j \leq \ell$ in $\mathcal{P}$
and separating the real and imaginary parts, obtain another family of polynomials,
$\mathcal{Q} \subset {\mathrm{R}}[{\mathbf{Y}}^{(1)},{\mathbf{Z}}^{(1)},\ldots, {\mathbf{Y}}^{(\ell)},{\mathbf{Z}}^{(\ell)}]$ with
$\deg_{{\mathbf{Y}}^{(j)}}(Q),\deg_{{\mathbf{Z}}^{(j)}}(Q) \leq d, 1 \leq j \leq \ell$, such that the polynomials in
$\mathcal{Q}$ are $\mathfrak{S}_{\mathbf{k}}$-symmetric.

Now apply Theorem \ref{thm:main-product-of-symmetric-quantitative} with
${\mathbf{k}}=(k_1,\ldots,k_\ell)$,
${\mathbf{m}} =(2m_1,\ldots,2m_\ell)$, and  
${\mathbf{d}}=(d,\ldots,d)$.
\end{proof}

\subsection{Proofs of Theorems \ref{thm:main-product-of-symmetric-sa} and \ref{thm:main-product-of-symmetric-sa-quantitative}}

We first need a few preliminary definitions and results.

\begin{definition}
  For any finite family $\mathcal{P} \subset {\mathrm{R}} [ X_{1} , \ldots ,X_{k} ]$ and
  $\ell \geq 0$, we say that $\mathcal{P}$ is in $\ell$-general position with
  respect to a semi-algebraic set $V \subset {\mathrm{R}}^{k}$ if for any subset
  $\mathcal{P}' \subset \mathcal{P}$, with ${\mathrm{card}} (\mathcal{P}') > \ell$, ${{\rm Z}} (\mathcal{P}' ,V) = \emptyset$. 
\end{definition}

Let ${\mathbf{k}}=(k_1,\ldots,k_\ell),
{\mathbf{m}} =(m_1,\ldots,m_\ell)\in  {\mathbb{N}}^\ell$, and $K = \sum_{i=1}^{\ell} k_i m_i$.
Let 
$\mathcal{P} = \{P_1,\ldots, P_s\} \subset {\mathrm{R}}[{\mathbf{X}}^{(1)},\ldots,{\mathbf{X}}^{(\ell)}]$
be a finite set of $\mathfrak{S}_{\mathbf{k}}$-symmetric polynomials, with
$\deg_{{\mathbf{X}}^{(i)}}(P_j) \leq d$ for $1\leq i \leq \ell, 1\leq j \leq s$.
Let  
$S \subset {\mathrm{R}}^K$  be a $\mathcal{P}$-closed semi-algebraic set.
Let $\overline{{\varepsilon}} = \left({{\varepsilon}}_{1} , \ldots , {{\varepsilon}}_{s} \right)$
be a tuple of new variables, and let $\mathcal{P}_{\overline{{\varepsilon}}} =
\bigcup_{1 \leq i \leq s} \left\{ P_{i}   \pm {{\varepsilon}}_{i} \right\}$. We have the
following two lemmas.

\begin{lemma}
  \label{lem:gen-pos1-with-parameters}Let
  \begin{eqnarray*}
    D (\mathbf{k},\mathbf{m},d) & = & \sum_{i=1}^{\ell} \min (k_{i}m_i ,d^{m_i}) .
  \end{eqnarray*}
  The family $\mathcal{P}_{\overline{{\varepsilon}}} \subset {\mathrm{R}}'[{\mathbf{X}}^{(1)} , \ldots ,{\mathbf{X}}^{(\ell)}]$ is in $D$-general position with respect to any semi-algebraic subset $Z' \subset {\mathrm{R}}'^K$, 
where ${\mathrm{R}}' = {\mathrm{R}} \langle \overline{{\varepsilon}} \rangle$ (cf. Notation \ref{not:Puiseux}), and
where $Z' = {\mathrm{Ext}}(Z,{\mathrm{R}}'^K)$ (cf. Notation \ref{not:Ext}), and $Z\subset {\mathrm{R}}^{K}$ is a semi-algebraic set stable under the action of $\mathfrak{S}_{\mathbf{k}}$.
  \end{lemma}

\begin{proof}
The lemma follows from the fact that the ring of multi-symmetric polynomials is generated by the
multi-symmetric power sum polynomials \cite[Theorem 1.2]{Dalbec}, and the cardinality of the
set of multi-symmetric power sum polynomials in the variables $X^{(i)}$ of degree bounded by $d$
is bounded by $d^{m_i}$.
\end{proof}

Let $\Phi$ be a $\mathcal{P}$-closed formula, and let $S= {{\mathcal R}} (\Phi ,V)$ be
bounded over ${\mathrm{R}}$.
\begin{notation}
For $\pmb{\mu} \in {\mathrm{Par}}({\mathbf{k}})$ and $i \geq 0$, we will denote
\begin{eqnarray*}
m_{i,\pmb{\mu}}(\Phi,{\mathbb{F}}) &=&  m_{i,\pmb{\mu}}(S,{\mathbb{F}}), \\
m_{\pmb{\mu}}(\Phi,{\mathbb{F}}) &=&  m_{\pmb{\mu}}(S,{\mathbb{F}}).
\end{eqnarray*}
\end{notation}
  
Let $\Phi_{\overline{{\varepsilon}}}$ be the
$\mathcal{P}_{\overline{{\varepsilon}}}$-closed formula obtained from $\Phi$ be
replacing for each $i,1 \leq i \leq s$,
\begin{enumerate}[i.]
  \item each occurrence of $P_{i} \leq 0$ by $P_{i} - {{\varepsilon}}_{i} \leq 0
  $, and
  
  \item each occurrence of $P_{i} \geq 0$ by $P_{i} + {{\varepsilon}}_{i} \geq 0
  $.
\end{enumerate}
Let ${\mathrm{R}}' = {\mathrm{R}} \left\langle {{\varepsilon}}_{1} , \ldots , {{\varepsilon}}_{s} \right\rangle$, and 
$S_{\overline{{\varepsilon}}} = {{\mathcal R}}(\Phi_{\overline{{\varepsilon}}} , {\mathrm{R}}'^{K})$.

\begin{lemma}
  \label{lem:gen-pos2-with-parameters} For any $r>0$, $r \in {\mathrm{R}}$, the
  semi-algebraic set set ${\mathrm{Ext}} (S \cap \overline{B_{K} (0,r)} , {\mathrm{R}}')$ is contained in 
  $S_{\overline{{\varepsilon}}} \cap \overline{B_{K} (0,r)}$, and the inclusion 
  ${\mathrm{Ext}}(S \cap \overline{B_{K} (0,r)} , {\mathrm{R}}') \hookrightarrow S_{\overline{{\varepsilon}}} \cap
  \overline{B_{K} (0,r)}$ is a semi-algebraic homotopy equivalence.
  The induced isomorphism,
  \[
  {\mbox{\rm H}}(S_{\overline{{\varepsilon}}} \cap \overline{B_{K} (0,r)},{\mathbb{F}}) \overset{\sim}{\rightarrow} {\mbox{\rm H}}^*({\mathrm{Ext}}(S \cap \overline{B_{K} (0,r)} , {\mathrm{R}}'),{\mathbb{F}}) 
  \]
  is an isomorphism of $\mathfrak{S}_{\mathbf{k}}$-modules.
\end{lemma}

\begin{proof} The proof is similar to the one of Lemma 16.17 in
{\cite{BPRbook2}}.
\end{proof}

\begin{remark}
  \label{rem:gen-pos3} In view of Lemmas \ref{lem:gen-pos1-with-parameters} and
  \ref{lem:gen-pos2-with-parameters} we can assume (at the cost of doubling the number of
  polynomials) after possibly replacing $\mathcal{P}$ by
  $\mathcal{P}_{\overline{{\varepsilon}}}$, and ${\mathrm{R}}$ by ${\mathrm{R}} \left\langle {{\varepsilon}}_{1} ,
  \ldots , {{\varepsilon}}_{s} \right\rangle$, that the family $\mathcal{P}$ is in
  $D(\mathbf{k},\mathbf{m},d)$-general position.
\end{remark}

Now, let $\delta_{1} , \cdots , \delta_{s}$ be new infinitesimals, and let
${\mathrm{R}}' = {\mathrm{R}} \langle \delta_{1} , \ldots , \delta_{s} \rangle$.

\begin{notation}
  We define $\mathcal{P}_{>i} = \{P_{i+1} , \ldots ,P_{s} \}$ and
  \begin{eqnarray*}
    \Sigma_{i} & = & \{P_{i} =0,P_{i} = \delta_{i} ,P_{i} = - \delta_{i}
    ,P_{i} \geq 2 \delta_{i} ,P_{i} \leq -2 \delta_{i} \} ,\\
    \Sigma_{\le i} & = & \{\Psi \mid \Psi = \bigwedge_{j=1, \ldots ,i}
    \Psi_{i} , \Psi_{i} \in \Sigma_{i} \} .
  \end{eqnarray*}
  Note that for each $\Psi \in \Sigma_{i}$, ${{\mathcal R}}(\Psi , {\mathrm{R}} \langle
  \delta_{1} , \ldots , \delta_{i} \rangle^{K})$ is symmetric with respect
  to the action of $\mathfrak{S}_{\mathbf{k}}$,  and for
  $\Psi \neq \Psi'$, $\Psi , \Psi' \in \Sigma_{\leq i}$,
  \begin{eqnarray}
    {{\mathcal R}} \left(\Psi , {\mathrm{R}} \langle \delta_{1} , \ldots , \delta_{i} {{\rangle}}^{K} \right)
    \cap {{\mathcal R}} \left(\Psi' , {\mathrm{R}} {{\langle}} \delta_{1} , \ldots , \delta_{i} {{\rangle}}^{K}
    \right) & = & \emptyset .  \label{eqn:disjoint}
  \end{eqnarray}
  
  
  If $\Phi$ is a $\mathcal{P}$-closed formula, we denote
  \begin{eqnarray*}
    {{\mathcal R}}_{i} (\Phi) & = & {{\mathcal R}}(\Phi , {\mathrm{R}} {{\langle}} \delta_{1} , \ldots ,\delta_{i} {{\rangle}}^{K}) ,
  \end{eqnarray*}
  and
  \begin{eqnarray*}
    {{\mathcal R}}_{i} (\Phi \wedge \Psi) & = & {{\mathcal R}}(\Psi , {\mathrm{R}} {{\langle}} \delta_{1} ,
    \ldots , \delta_{i} {{\rangle}}^{K}) \cap {{\mathcal R}}_{i} (\Phi) .
  \end{eqnarray*}
\end{notation}

The proof of the following proposition is very similar to Proposition 7.39 in
{\cite{BPRbook2}} where it is proved in the non-symmetric case.

\begin{proposition}
  \label{7:prop:closed-with-parameters}
  For every $\mathcal{P}$-closed formula
  $\Phi$, and $\pmb{\mu} \in {\mathrm{Par}}({\mathbf{k}})$,
  \[  m_{\pmb{\mu}}(\Phi,{\mathbb{F}}) \leq
     \sum_{\substack{
       \Psi \in \Sigma_{\le s}\\
       {{\mathcal R}}_{s} (\Psi , {\mathrm{R}}'^{K}) \subset {{\mathcal R}}_{s} (\Phi , {\mathrm{R}}'^{K})
       }}
      m_{\pmb{\mu}}(\Psi,{\mathbb{F}}) . 
     \]
\end{proposition}

\begin{proof}
 The symmetric spaces ${{\mathcal R}} \left(\Psi , {\mathrm{Ext}} \left(V, {\mathrm{R}}' \right) \right) ,
\Psi \in \Sigma_{\leq s}$ are disjoint by (\ref{eqn:disjoint}). The 
proposition now follows from Schur's lemma,
and the proof of Proposition 7.39 in {\cite{BPRbook2}}.
\end{proof}

\begin{proposition}
  \label{7:prop:betti closed}
  Suppose for $\pmb{\mu} \in {\mathrm{Par}}({\mathbf{k}})$ and $i \geq 0$,
  $m_{i, \pmb{\mu}}(S,{\mathbb{F}}) >0$. 
 Then, 
 \begin{equation}
\label{eqn:restriction-on-specht-sa}
\pmb{\mu}  \in 
\mathcal{I}({\mathbf{k}},{\mathbf{d}},{\mathbf{m}}),
\end{equation}
where ${\mathbf{d}} = (d,\ldots,d)$.
For $i \geq 0$, and $\pmb{\mu}  \in 
\mathcal{I}({\mathbf{k}},{\mathbf{d}},{\mathbf{m}})$,
  \[ 
  \sum_{\Psi \in \Sigma_{\le s}} m_{i,\pmb{\mu}}(\Psi,{\mathbb{F}}) \leq 
  \sum_{j=0}^{D({\mathbf{k}},{\mathbf{m}},d)} \binom{s}{j} 6^{j} 
  F(\pmb{\mu}, {\mathbf{k}},{\mathbf{m}},d),
  \]
  where 
  \[
  F(\pmb{\mu},{\mathbf{k}},{\mathbf{m}},d) = 
  \sum_{\pmb{\lambda}=(\lambda^{(1)},\ldots,\lambda^{(\ell)})\in {\mathrm{Par}}({\mathbf{k}}, (2{\mathbf{d}})^{\mathbf{m}})}
G(\pmb{\mu},\pmb{\lambda},{\mathbf{d}},{\mathbf{m}}),
 \]
 and 
\[
G(\pmb{\mu},\pmb{\lambda},{\mathbf{d}},{\mathbf{m}}) =
\prod_{1 \leq i \leq \ell}
 \left(
 (2 d)^{(m_i{\mathrm{length}}(\lambda^{(i)}))}
 \max_{\lambda^{(i)} = {\lambda^{(i)}}' \coprod {\lambda^{(i)}}''}
 m^{\mu^{(i)}}_{{\lambda^{(i)}}',{\lambda^{(i)}}''}
 \right)
 \]
(the maximum on the right hand side is taken over all decompositions
$\lambda^{(i)} = {\lambda^{(i)}}' \coprod {\lambda^{(i)}}''$). 
 \end{proposition}

In order to prove Proposition \ref{7:prop:betti closed} we first need the following lemmas.

Let for $1 \leq i \leq s$, 
$Q_{i} =P_{i}^{2} (P_{i}^{2} - \delta_{i}^{2})^{2} (P_{i}^{2} -4
\delta_{i}^{2})$.

For $j \ge 1$ let,
\begin{eqnarray*}
  V'_{j} & = & {{\mathcal R}}(\bigvee_{1 \leq i \leq j} Q_{i} =0, {\mathrm{R}} {{\langle}}\delta_{1} , \ldots , \delta_{j} {{\rangle}}^{K}) ,\\
  W'_{j} & = & {{\mathcal R}}(\bigvee_{1 \leq i \leq j} Q_{i} \geq 0, {\mathrm{R}} {{\langle}} \delta_{1} , \ldots , \delta_{j} {{\rangle}}^{K}) .
\end{eqnarray*}
\begin{lemma}
  \label{lem:gen-position} Let $I \subset [ 1,s ]$, $\sigma  =  (\sigma_{1} ,
  \ldots , \sigma_{s}) \in \{ 0, \pm 1, \pm 2 \}^{s}$ and let
  $\mathcal{P}_{I, \sigma} =  \bigcup_{i \in I} \{ P_{i} + \sigma_{i}
  \delta_{i} \}$. Then, ${{\rm Z}} \left(P_{I, \sigma} , {\mathrm{R}}'^K \right) = \emptyset$,
  whenever ${\mathrm{card}}(I) >D$.
\end{lemma}

\begin{proof}
This follows from the fact that $\mathcal{P}$ is in
$D$-general position by Remark
\ref{rem:gen-pos3}.
\end{proof}

\begin{lemma}
  \label{7:lem:union2} For each $\pmb{\mu}  \in 
\mathcal{I}({\mathbf{k}},{\mathbf{d}},{\mathbf{m}})$, and $i \geq 0$,
\[ m_{i,\pmb{\mu}} (V'_{j},{\mathbb{F}}) \leq (6^{j} -1) F (
     \mathbf{k},{\mathbf{m}},2d) . 
\]
\end{lemma}

\begin{proof}

The set ${{\mathcal R}} ((P_{j}^{2} (P_{j}^{2} - \delta_{j}^{2}
)^{2} (P_{j}^{2} -4 \delta_{j}^{2})=0), {\mathrm{R}} {{\langle}} \delta_{1} , \ldots ,
\delta_{j} {{\rangle}}^{K})$ is the disjoint union of

\begin{equation}
 \begin{array}{c}
     {{\mathcal R}} (P_{i} =0, {\mathrm{R}} {{\langle}} \delta_{1} , \ldots , \delta_{j} {{\rangle}}^{K}) ,\\
     {{\mathcal R}} (P_{i} = \delta_{i} , {\mathrm{R}} {{\langle}} \delta_{1} , \ldots , \delta_{j} {{\rangle}}^{K}
    ) ,\\
     {{\mathcal R}} (P_{i} =  -\delta_{i} , {\mathrm{R}} {{\langle}} \delta_{1} , \ldots , \delta_{j}
     {{\rangle}}^{K}) ,\\
     {{\mathcal R}} (P_{i} =2 \delta_{i} , {\mathrm{R}} {{\langle}} \delta_{1} , \ldots , \delta_{j}
     {{\rangle}}^{K}) ,\\
     {{\mathcal R}} (P_{i} =  -2 \delta_{i} , {\mathrm{R}} {{\langle}} \delta_{1} , \ldots , \delta_{j}
     {{\rangle}}^{K}) .
   \end{array} \label{eqn:list}
\end{equation}

It follows from part (\ref{7:prop:prop1:itema}) of Proposition \ref{7:prop:prop1}  
that $m_{i,\pmb{\mu}} (V'_{j} ,{\mathbb{F}})$ 
is bounded by the sum for 
$1 \leq p \leq i+1$, of the multiplicities of  $\mathbb{S}^{\pmb{\mu}}$
in the $(i- p +1)$-th cohomology module of all possible $p$-ary
intersections  amongst the sets listed in \eqref{eqn:list}. It is clear that
the total number of such non-empty $p$-ary intersections is at most
$\binom{j}{\ell} 5^{p}$. It now follows from Theorem \ref{thm:main} applied
to the non-negative symmetric polynomials $P_{i}^{2} , (P_{i} \pm
\delta_{i})^{2} , (P_{i} \pm 2 \delta_{i})^{2}$, and noting that the
degrees of these polynomials are bounded by $2d$, that
\begin{eqnarray*}
  m_{i, \pmb{\mu}} (V'_{j},{\mathbb{F}}) 
  & \leq & 
  \sum_{p=1}^{\min (j,D)} \binom{j}{p} 5^{p} F (\pmb{\mu}, \mathbf{k},{\mathbf{m}},2d) .
\end{eqnarray*}
\end{proof}

\begin{lemma}
  \label{7:lem:union1}
  For each $\pmb{\mu}  \in 
\mathcal{I}({\mathbf{k}},{\mathbf{d}},{\mathbf{m}})$, and $i \geq 0$,
  \[ m_{i,\pmb{\mu}} (W'_{j},{\mathbb{F}}) \leq \sum_{p=1}^{\min (
     j,D)} \binom{j}{p} 5^{p}  (F (\pmb{\mu}, \mathbf{k},{\mathbf{m}},2 d)) +m_{i,\pmb{\mu}} ({\mathrm{R}} {{\langle}}
     \delta_{1} , \ldots , \delta_{j} {{\rangle}}^{K},{\mathbb{F}}) . \]
 \end{lemma}

\begin{proof} Let
\[ F= {{\mathcal R}} \left(\bigwedge_{1 \leq i \leq j} Q_{i} \leq 0 \vee \bigvee_{1 \leq
   i \leq j} Q_{i} =0, {\mathrm{Ext}} (Z, {\mathrm{R}} {{\langle}} \delta_{1} , \ldots , \delta_{i} \rangle)
   \right) . \]

Now, from the fact that \[ W'_{j} \cup F= {\mathrm{R}} {{\langle}} \delta_{1} , \ldots ,
   \delta_{j} {{\rangle}}^{k} ,W'_{j} \cap F=V'_{j},\] it follows
immediately that
\[ (W'_{j} \cup F)  =W'_{j}
    \cup F = {\mathrm{R}} {{\langle}}
   \delta_{1} , \ldots , \delta_{j} {{\rangle}}^{K}  , \]
and
\[ W'_{j}  \cap F = (W'_{j} \cap F)  =V_{j}'. 
   \]
Using Proposition \ref{prop:MV} we get that
\begin{eqnarray*}
  m_{i,\pmb{\mu}} (W'_{j} ,{\mathbb{F}}) & \leq & m_{i,\pmb{\mu}} ((
  W'_{j} \cap F),{\mathbb{F}}) +m_{i,\pmb{\mu}} ((W'_{j}
  \cup F),{\mathbb{F}})\\
  & = & m_{i,\pmb{\mu}} (V'_{j},{\mathbb{F}}) + m_{i,\pmb{\mu}} (
  {\mathrm{R}} {{\langle}} \delta_{1} , \ldots , \delta_{j} {{\rangle}}^{K},{\mathbb{F}}).
\end{eqnarray*}
We conclude using Lemma \ref{7:lem:union2}.
\end{proof}

Now, let
\[ T_{i} = {{\mathcal R}} \left(P_{i}^{2} (P_{i}^{2} - \delta_{i}^{2})^{2} (P_{i}^{2}
   -4 \delta_{i}^{2}) \geq 0, {\mathrm{Ext}} (Z, {\mathrm{R}} {{\langle}} \delta_{1} , \ldots ,
   \delta_{s} \rangle) \right), 1 \leq i \leq s, 
   \]
and let $T$ be the intersection of the $T_{i}$ with the closed ball in ${\mathrm{R}} {{\langle}}
\delta_{1} , \ldots , \delta_{s} , \delta {{\rangle}}^{K}$ defined by $\delta^{2}
\left(\sum_{1 \leq i \leq k} X_{i}^{2} \right) \leq 1$. Then, it is clear
from Lemma \ref{lem:gen-pos2-with-parameters} that
\begin{equation}
  \sum_{\Psi \in \Sigma_{\le s}} m_{i,\pmb{\mu}} (\Psi,{\mathbb{F}}) =
  m_{i,\pmb{\mu}} (T,{\mathbb{F}}) .
  \label{eqn:Psi-S}
\end{equation}

\begin{proof}[Proof of Proposition \ref{7:prop:betti closed}]
Using part (\ref{7:prop:prop1:itemb}) of Proposition \ref{7:prop:prop1} we get that
\begin{eqnarray*}
  \sum_{\Psi \in \Sigma_{\le s}} m_{i,\pmb{\mu}} (\Psi,{\mathbb{F}}) & \leq & \sum_{j=1}^{\min(D,K-i)}
  \sum_{\begin{array}{c}
    J \subset \{ 1, \ldots ,s \}\\
    {\mathrm{card}}(J) =j
  \end{array}} m_{i+j-1,\pmb{\mu}} (S^{J},{\mathbb{F}}) + \binom{s}{K-i}
  m_{K,\pmb{\mu}} (S^{\emptyset},{\mathbb{F}}).
\end{eqnarray*}
It follows from Lemma \ref{7:lem:union1} that,
\begin{eqnarray*}
  m_{i+j-1,\pmb{\mu}} (S^{J}) & \leq & \sum_{p =1}^{\min (j,D)} \binom{j}{p}
  5^{p} F (\pmb{\mu},\mathbf{k},{\mathbf{m}},2 d) +m_{K,\pmb{\mu}}({\mathrm{R}}^{K},{\mathbb{F}}) .
\end{eqnarray*}
Hence,
\begin{eqnarray*}
  \sum_{\Psi \in \Sigma_{\le s}} m_{i,\pmb{\mu}} (\Psi,{\mathbb{F}})
  & \leq & \sum_{j=1}^{D}
  \sum_{\substack{
    J \subset \{ 1, \ldots ,s \}\\
    {\mathrm{card}} (J) =j}}
 m_{i+j-1,\pmb{\mu}} (S^{J}, {\mathbb{F}}) + \binom{s}{K-i}
  m_{K,\pmb{\mu}} (S^{\emptyset},{\mathbb{F}})\\
  & \leq & \sum_{j=1}^{D} \binom{s}{j} \left(\sum_{p=1}^{\min (
  j,D)} \binom{j}{p} 5^{p}  F (\pmb{\mu},\mathbf{k},{\mathbf{m}},2d) \right)\\
  & \leq & \sum_{j=1}^{D} \binom{s}{j} 6^{j} F (\pmb{\mu},\mathbf{k},{\mathbf{m}},2d) .
\end{eqnarray*}
\end{proof}

\begin{proof}[Proof of Theorem \ref{thm:main-product-of-symmetric-sa}]
Follows from Propositions \ref{7:prop:closed-with-parameters} and \ref{7:prop:betti closed}.
\end{proof}

\begin{proof}[Proof of Theorem \ref{thm:main-product-of-symmetric-sa-quantitative}]
Follows immediately from Theorem \ref{thm:main-product-of-symmetric-sa} and 
Proposition \ref{prop:multiplicity}. 
\end{proof}

\subsection{Proof of Theorem \ref{thm:symmetric-complex-projective}}
\begin{proof}[Proof of Theorem \ref{thm:symmetric-complex-projective}]
Let ${\mbox{${\bf S}$}}^{2k+1} \subset {\mathrm{C}}^{k+1}$ denote the unite sphere defined by 
$|Z_0|^2 +\cdots + |Z_k|^2 =1$. 
Consider the Hopf fibration $\phi: {\mbox{${\bf S}$}}^{2k+1} \rightarrow {\mathbb{P}}_{\mathrm{C}}^k$, defined by
$(z_0,\ldots,z_k) \mapsto (z_0:\cdots:z_k)$. 
We denote by $\tilde{V} = \phi^{-1}(V)$.
We have the following commutative diagram:
\[
\xymatrix{
\tilde{V} \ar[r]^{i} \ar[d]^{\phi|_{\tilde{V}}} & {\mbox{${\bf S}$}}^{2k+1} \ar[d]^\phi \\
V \ar[r]^i & {\mathbb{P}}_{\mathrm{C}}^k
}
\]
Note that $\tilde{V}$ is a ${\mbox{${\bf S}$}}^1$-bundle over $V$, and using the fact that ${\mathbb{P}}_{\mathrm{C}}^k$ is simply connected, 
there is a $\mathfrak{S}_{k+1}$-equivariant spectral sequence degenerating at its $E_3$ term converging to the cohomology of  $\tilde{V}$. 

The $E_3$-term of the spectral sequence is given by
\begin{eqnarray*}
E_2^{p,q} & \cong & {\mbox{\rm H}}_p(V,{\mathbb{F}}), \mbox{ if } q=0,1,\\
E_2^{p,q} &=& 0, \mbox{ else },
\end{eqnarray*} 
and the differentials $d_2^{p,q}: E_2^{p,q} \rightarrow E_2^{p+2,q-1}$ shown below. 

{\tiny
\[
\xymatrix{
 \vdots & \vdots & \vdots & \vdots & \vdots & \vdots & \vdots  \\
 0 \ar[rrd]^{d_2^{-1,2}} &0 &0 &\cdots &0\ar[rrd]^{d_2^{i,2}}  &0 & 0\\
 0 \ar[rrd]^{d_2^{-1,1}} &{\mbox{\rm H}}^0(V,{\mathbb{F}}) & {\mbox{\rm H}}^1(V,{\mathbb{F}})   &\cdots& {\mbox{\rm H}}^i(V,{\mathbb{F}})\ar[rrd]^{d_2^{i,1}}  & {\mbox{\rm H}}^{i+1}(V,{\mathbb{F}}) & {\mbox{\rm H}}^{i+2}(V,{\mathbb{F}})\\
0  & {\mbox{\rm H}}^0(V,{\mathbb{F}}) & {\mbox{\rm H}}^1(V,{\mathbb{F}})  &\cdots
& {\mbox{\rm H}}^i(V,{\mathbb{F}})  & {\mbox{\rm H}}^{i+1}(V,{\mathbb{F}}) &  {\mbox{\rm H}}^{i+2}(V,{\mathbb{F}}) \\
}
\]
}
Fix $\lambda \vdash k+1$, and recall that we denote for each $i \geq 0$, $m_{i,\lambda}(V,{\mathbb{F}})$ (resp. ${m}_{i,\lambda}(\tilde{V},{\mathbb{F}})$) the multiplicity of
$\mathbb{S}^\lambda$ in ${\mbox{\rm H}}^i(V,{\mathbb{F}})$ (resp. ${\mbox{\rm H}}^i(\tilde{V},{\mathbb{F}})$).

We observe that since ${\mbox{\rm H}}^0(V,{\mathbb{F}}) \cong_{\mathfrak{S}_{k+1}} {\mbox{\rm H}}^0(\tilde{V},F)$, 
we have for all $\lambda \vdash k+1$,
\begin{equation}
\label{eqn:spectral-multiplicity-0}
m_{0,\lambda}(V,{\mathbb{F}}) = {m}_{\lambda,0}(\tilde{V},{\mathbb{F}}).
\end{equation}

Also, note that it follows from the fact that the spectral sequence $E_r^{p,q}$  degenerates at its
$E_3$ term that, 
\[
{\mbox{\rm H}}^1(V,{\mathbb{F}}) \oplus {\mbox{\rm ker}}(d_2^{0,1}) \cong_{\mathfrak{S}_{k+1}}  {\mbox{\rm H}}^1(\tilde{V},{\mathbb{F}}),
\]
and we obtain from the fact that the spectral sequence $E^r_{p,q}$ is $\mathfrak{S}_{k+1}$-equivariant that
\begin{equation}
\label{eqn:spectral-multiplicity-1}
m_{1,\lambda}(V,{\mathbb{F}}) \leq {m}_{1,\lambda}(\tilde{V},{\mathbb{F}}).
\end{equation}

More generally, we have from the $E^2$-term of the spectral sequence that
\[
{\mbox{\rm H}}^i(\tilde{V},{\mathbb{F}})  \cong_{\mathfrak{S}_{k+1}}  \mathrm{coker}(d_2^{i-2,1}) \oplus {\mbox{\rm ker}}(d_2^{i-1,1}).
\]

For $\lambda \vdash k+1, i\geq 0$, 
and any finite dimensional ${\mathbb{F}}$-representation $W$ of $\mathfrak{S}_{k+1}$, we denote 
by ${\mathrm{mult}}_{\lambda}(W,{\mathbb{F}})$ the multiplicity of $\mathbb{S}^\lambda$ in $W$.

Since, 
\[
{\mbox{\rm H}}^i(V,{\mathbb{F}}) \cong_{\mathfrak{S}_{k+1}}  \mathrm{Im}(d_2^{i-2,1}) \oplus \mathrm{coker}(d_2^{i-2,1}),
\]
we have for all $\lambda \vdash k+1, i \geq 0$,
\[
{\mathrm{mult}}_{\lambda}( \mathrm{coker}(d_2^{i-2,1}),{\mathbb{F}}) = m_{i,\lambda}(V,{\mathbb{F}})  -  {\mathrm{mult}}_{\lambda}(\mathrm{Im}(d_2^{i-2,1}),{\mathbb{F}}),
\]
and we also have for $i \geq 2$,

\begin{equation}
\label{eqn:Im}
{\mathrm{mult}}_\lambda(\mathrm{Im}(d_2^{i-2,1}),{\mathbb{F}}) \leq m_{i-2,\lambda}(V,{\mathbb{F}}).
\end{equation}
This implies that for all $\lambda \vdash k+1, i \geq 0$
\[
m_{i,\lambda}(\tilde{V},{\mathbb{F}}) =  (m_{i,\lambda}(V,{\mathbb{F}}) - {\mathrm{mult}}_{\lambda}(\mathrm{Im}(d_2^{i-2,1}),{\mathbb{F}})) + 
{\mathrm{mult}}_\lambda(\mathrm{coker}(d_2^{i-1,1}),{\mathbb{F}}).
\]
It follows that
\begin{eqnarray*}
m_{i,\lambda}(V,{\mathbb{F}}) &=& 
m_{i,\lambda}(\tilde{V},{\mathbb{F}}) + {\mathrm{mult}}_{\lambda}(\mathrm{Im}(d_2^{i-2,1}),{\mathbb{F}})) - 
{\mathrm{mult}}_\lambda(\mathrm{coker}(d_2^{i-1,1}),{\mathbb{F}}) \\
                     &\leq& m_{i,\lambda,i}(\tilde{V},{\mathbb{F}}) + {\mathrm{mult}}_\lambda(\mathrm{Im}(d_2^{i-2,1}),{\mathbb{F}}) \\
                     &\leq& m_{i,\lambda}(\tilde{V},{\mathbb{F}}) + m_{i-2,\lambda}(V,{\mathbb{F}}) \mbox{ using \eqref{eqn:Im}}.
                     \end{eqnarray*}

Finally we have shown that for each $\lambda \vdash k+1$ and $i \geq 2$,
 
\begin{eqnarray}
\label{eqn:spectral-multiplicity-general}
 m_{i,\lambda}(V,{\mathbb{F}}) &\leq& m_{i,\lambda}(\tilde{V},{\mathbb{F}}) + m_{i-2,\lambda}(V,{\mathbb{F}})  \nonumber\\
 				&\leq& \sum_{0\leq j \leq \lfloor \frac{i}{2}\rfloor } m_{i - 2j,\lambda}(\tilde{V},{\mathbb{F}}) \mbox{ using induction}. 
\end{eqnarray}
 
The Theorem follows from applying Theorem \ref{thm:main-product-of-symmetric-quantitative} to the set $\tilde{V}$, and 
inequalities \eqref{eqn:spectral-multiplicity-0}, \eqref{eqn:spectral-multiplicity-1}, and
\eqref{eqn:spectral-multiplicity-general}.
\end{proof}

\begin{remark}
Note that in the proof of Theorem \ref{thm:symmetric-complex-projective},
since
\[
\tilde{V}/\mathfrak{S}_{k+1} \sim V/\mathfrak{S}_{k+1},
\]
and hence
\[
{\mbox{\rm H}}^*_{\mathfrak{S}_{k+1}}(\tilde{V},{\mathbb{F}}) \cong {\mbox{\rm H}}^*_{\mathfrak{S}_{k+1}}(V,{\mathbb{F}}),
\]
 we can avoid 
the argument involving the spectral sequence if we are only interested in the equivariant cohomology of $V$.
Also note that it is possible to replace the spectral sequence argument altogether
by an argument using the equivariant version of the Gysin exact sequence. 
\end{remark}

\subsection{Proof of Theorem \ref{thm:descent2-quantitative-new}}
\begin{proof}[Proof of Theorem \ref{thm:descent2-quantitative-new}]
First notice that 
\[
{\mathrm{Sym}}^{(p)}_{\pi} (V) =  V^{(p)}/\mathfrak{S}_{{\mathbf{k}}(p)},
\]
where 
\[
V^{(p)} = {{\rm Z}}(P^{(p)},{\mathrm{R}}^{k+(p+1)m}),
\]
$P^{(p)} \in {\mathrm{R}}[{\mathbf{X}},{\mathbf{Y}}_0,\ldots,{\mathbf{Y}}_{p}]$ is defined by
\[
 P^{(p)} = P({\mathbf{X}},{\mathbf{Y}}_0)+ \cdots+ P({\mathbf{X}},{\mathbf{Y}}_{p}),
 \]
 and 
 ${\mathbf{k}}(p) = (\underbrace{1,\ldots,1}_{k},p+1)$.

  Notice that since $V$ is bounded, so is $V^{(p)}= {{\rm Z}}(P,{\mathrm{R}}^{k+(p+1)m})$.
 Moreover, 
 $\deg(P^{(p)}) = \deg(P)$, and $P^{(p)}$ is symmetric in $({\mathbf{Y}}_0,\ldots,{\mathbf{Y}}_{p})$, and is thus
 $\mathfrak{S}_{{\mathbf{k}}(p)}$-symmetric.
 
By Theorem \ref{thm:descent2},
\begin{eqnarray*}
    b (\pi (V) ,{\mathbb{F}}) & \leq & \sum_{0 \leq p<k} b ({\mathrm{Sym}}^{(p)}_{\pi} (V) ,{\mathbb{F}}).
\end{eqnarray*}

Now using Corollary  \ref{cor:equivariant},
\[
b ({\mathrm{Sym}}^{(p)}_{\pi} (V) ,{\mathbb{F}}) \leq (p+1)^{(2d)^{m}} (O(d))^{k+m (2d)^m + 1},
\]
and hence,
\begin{eqnarray*}
  b (\pi (V) ,{\mathbb{F}}) & \leq & \sum_{0 \leq p<k} b ({\mathrm{Sym}}^{(p)}_{\pi} (V) ,{\mathbb{F}})\\
  &\leq & \sum_{0 \leq p<k} (p+1)^{(2d)^{m}} (O(d))^{k+ m (2d)^m +1} \\
  &\leq & k^{(2d)^{m}} (O(d))^{k+m (2d)^m +1}.
\end{eqnarray*}
 
This completes the proof of the theorem.
\end{proof}

 

\section{Conclusion and open problems}
\label{sec:conclusion}

In this paper we have proved polynomial bounds on the number and the multiplicities of the
irreducible representations of the symmetric group (or more generally product of symmetric groups) that appear in the cohomology modules of symmetric real algebraic and more generally real semi-algebraic sets. We have given several applications of the main results, including to improve existing
bounds on the topological complexity of sets defined as images of semi-algebraic maps, 
and proving lower bounds on the degrees etc. We end with some open problems and future research
directions.
 
\subsection{Representational Stability Question}
The bounds on the multiplicities that we prove in this paper are all polynomial in the number of variables
(for fixed  degrees).  Motivated by the recently developed theory of FI-modules \cite{Church-et-al} it makes sense  to ask whether it is possible to prove some stability result as $k \rightarrow \infty$.
We formulate one such question below.

Let $F \in {\mathrm{R}}[X_1,\ldots,X_d]^{\mathfrak{S}_d}_{\leq d}$ be a symmetric polynomial of degree at most $d$, 
and let for $k \geq d$ 
$F_k = \phi_{d,k}(F) \in {\mathrm{R}}[X_1,\ldots,X_k]^{\mathfrak{S}_k}$ where $\phi_{d,k}: {\mathrm{R}}[X_1,\ldots,X_d]^{\mathfrak{S}_d}_{\leq d} \hookrightarrow {\mathrm{R}}[X_1,\ldots,X_k]^{\mathfrak{S}_k}$ is the canonical
injection, defined by 
\[
F_k = G(p_1^{(k)}(X_1,\ldots,X_k) ,\ldots,p_d^{(k)}(X_1,\ldots,X_k)),
\]
where $F = G(p_1^{(d)},\ldots, p_d^{(d)})$ denoting
for each $n>0$, $p_i^{(n)} = \sum_{j=1}^n X_j^i$ (the $i$-th power sum polynomial). 
Let $V_k = {{\rm Z}}(F_k,{\mathrm{R}}^k)$.
Also, let $\mu = (\mu_1,\ldots,\mu_\ell) \vdash k_0$ be any fixed partition, and for all $k \geq k_0 + \mu_1$, let
\begin{eqnarray}
\label{eqn:def-of-mu-k}
\{\mu\}_k &=& (k-k_0, \mu_1,\mu_2,\ldots,\mu_\ell) \vdash k.
\end{eqnarray} 

It is a consequence of the hook-length formula (Theorem \ref{thm:hook}) that 
\begin{eqnarray}
\label{eqn:polynomial}
\dim_{\mathbb{F}}(\mathbb{S}^{\{\mu\}_k})=  \frac{\dim_{\mathbb{F}}(\mathbb{S}_\mu)}{|\mu|!} P_\mu(k),
\end{eqnarray}
where $P_\mu(T)$ is a  monic polynomial having distinct integer roots,  and $\deg(P_\mu) = |\mu|$ (see \cite[7.2.2]{Deligne2004}).

Finally, for a fixed number $p \geq 0$ we pose the following question.

\begin{question}
\label{question:stability}
Does there exist a polynomial $P_{F,p,\mu}(k)$ such that
for all sufficiently large $k$, 
$m_{p,\{\mu\}_k}(V_k,{\mathbb{F}}) = P_{F,p,\mu}(k)$ ? 
In conjunction with
\eqref{eqn:polynomial}, a positive answer would imply that
\[
\dim_{\mathbb{F}}({\mbox{\rm H}}^p(V_k,{\mathbb{F}}))_{\{\mu\}_k} =   \frac{\dim_{\mathbb{F}}(\mathbb{S}_\mu)}{|\mu|!} P_{F,p,\mu}(k) P_\mu(k)
\]
is also given by a polynomial for all large enough $k$.

In particular, taking $\mu=()$ to be the empty partition, is it true that
$m_{p,\{\mu\}_k}(V_k,{\mathbb{F}}) = b^p_{\mathfrak{S}_k}(V_k,{\mathbb{F}})$ (that is the $p$-th equivariant Betti number of $V_k$ cf. Notation \ref{not:equivariant-betti} ) is given by a polynomial in $k$ ?

A stronger question is to ask for a bound on the degree of $P_{F,p,\mu}(k)$ as a function of $d,\mu$ and $p$. 
\end{question}

\begin{remark}
Note that it follows from the results of this paper (Theorem \ref{thm:main-product-of-symmetric-quantitative}) that there exists a polynomial  
 $P_{F,p,\mu}(k)$ of degree $O(d^2)$,  with the property that 
 \[
 m_{p,\{\mu\}_k}(V_k,{\mathbb{F}}) \leq  P_{F,p,\mu}(k)
 \]
 for all $k \geq 0$. In the particular case when $\mu$ is the empty partition, it follows from Corollary \ref{cor:equivariant}
 that the degree of $P_{F,p,\mu}(k)$ can be bounded  by $2d$.
 \end{remark}
 
\begin{remark}
Question \ref{question:stability} has a positive answer for the sequence of polynomials $F_k$ introduced
in Example \ref{eg:basic}. In that example we can take $F$ to be the following polynomial:
\[
F = \sum_{i=1}^4 X_i^2(X_i-1)^2 - {{\varepsilon}} \in {\mathrm{R}}{{\langle}}{{\varepsilon}}{{\rangle}}[X_1,X_2,X_3,X_4]^{\mathfrak{S}4}_{\leq 4}.
\] 
From the discussion in Example \ref{eg:basic} we deduce that for each $p>0,\mu \vdash k_0$, and for all large enough $k$,
\[
m_{p,\{\mu\}_k}(V_k,{\mathbb{F}}) = 0.
\]
For $p=0$, and partitions $\mu$ with ${\mathrm{length}}(\mu) >1$, we again have for all large enough $k$,
\[
m_{p,\{\mu\}_k}(V_k,{\mathbb{F}}) = 0.
\] 
Finally, for  $p=0$, and any partition $(k_0)$ of length $\leq 1$, and  for all $k \geq  2 k_0$,
\begin{eqnarray*}
m_{0,\{\mu\}_k}(V_k,{\mathbb{F}}) &=& 2(k-k_0) -k+1  \mbox{ (using \eqref{eqn:def-of-mu-k} and \eqref{eqn:even-and-odd})}\\
				      &=& k  - 2k_0 +1.
\end{eqnarray*}
Thus, $m_{p,\{\mu\}_k}(V_k,{\mathbb{F}})$ is given by a polynomial for all large $k$, for any fixed $p$ and $\mu$.
Notice also that the degree of this polynomial is bounded by $1$.
\end{remark}

\subsection{Algorithmic Conjecture}
As mentioned in the Introduction, a polynomial bound on any topological invariant
of a class of semi-algebraic sets usually implies also that there exists an algorithm with polynomially bounded
complexity for computing it. 
Since we have we proved that the multiplicities of the irreducible representations of 
$\mathfrak{S}_k$ appearing in the cohomology group  of a symmetric $\mathcal{P}$-semi-algebraic set $S \subset {\mathrm{R}}^k$, where $\deg(P), P \in \mathcal{P}$ is bounded by a constant, is bounded 
by a polynomial function of ${\mathrm{card}}(\mathcal{P})$ and $k$, the mentioned  principle implies that these multiplicities 
should be computable by an algorithm with polynomially bounded complexity (for fixed $d$).
If this holds, then since the number of irreducibles
that are allowed to appear with positive multiplicity is also polynomially bounded, and
their respective dimensions are polynomially computable using the hook length formula (Theorem 
\ref{thm:hook}),
we deduce that once these multiplicities are computed, the dimensions of the cohomology groups of $S$ (with coefficients in ${\mathbb{Q}}$)
can be computed with polynomially bounded complexity. 
 
This leads us to make the following algorithmic conjecture.

\begin{conjecture}
\label{conj:poly}
For any fixed $d >0$, there is an algorithm that takes as input
the description of a symmetric semi-algebraic set $S \subset {\mathrm{R}}^k$,
defined by a $\mathcal{P}$-closed formula, where 
$\mathcal{P}$ is a set symmetric polynomials of degrees bounded by $d$, and computes
$m_{i,\lambda}(S,{\mathbb{Q}})$, for each $\lambda \vdash k$, and  
$m_{i,\lambda}(S,{\mathbb{Q}})>0$, 
as well as all the Betti numbers $b_i(S,{\mathbb{Q}})$, with complexity which is polynomial
in  ${\mathrm{card}}(\mathcal{P})$ and $k$.
\end{conjecture}

\begin{remark}
\label{rem:poly}
We note that Conjecture \ref{conj:poly} is not completely unreasonable, since
an analogous result for computing the generalized Euler-Poincar\'e characteristic
of symmetric semi-algebraic sets has been proved in \cite{BC2013}. However, computing the
Betti numbers of a semi-algebraic set is usually a much harder task than computing the
Euler-Poincar\'e characteristic.
\end{remark}

\bibliographystyle{amsplain}
\bibliography{master}
\end{document}

