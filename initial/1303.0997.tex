

\documentclass[12pt,a4paper]{amsart}
\textwidth 6.5in 
\oddsidemargin 0in 
\evensidemargin 0in 
\setlength{\textheight}{8.5in} 

\addtolength{\headheight}{3.2pt} 
 
\allowdisplaybreaks 

\usepackage{color}
\usepackage{amsmath}
\usepackage{amssymb}
\usepackage{amsthm}
\usepackage{amsfonts}
\usepackage{latexsym}
\usepackage{hyperref}

\theoremstyle{plain}
\newtheorem{theorem}{Theorem}[section]
\newtheorem{corollary}[theorem]{Corollary}
\newtheorem{lemma}[theorem]{Lemma}
\newtheorem{proposition}[theorem]{Proposition}
\theoremstyle{definition}
\newtheorem{definition}[theorem]{Definition}

\newtheorem{remark}[theorem]{Remark}
\numberwithin{equation}{section}

  

  
   

 
 

 
 
 

 

  

\begin{document}

\title[Paley--Wiener Theorems on the Heisenberg group]
{Paley--Wiener Theorems \\for the ${{\text{U}({n})}}$--spherical transform \\on the Heisenberg group}

\author[F. Astengo, B. Di Blasio, F. Ricci]
{Francesca Astengo, Bianca Di Blasio, Fulvio Ricci}

\address
{Dipartimento di Matematica\\
Via Dodecaneso 35\\
16146 Genova\\ Italy} \email{astengo@dima.unige.it}

\address
{Dipartimento di Matematica e Applicazioni\\
Via Cozzi 53\\
  20125 Milano\\ Italy}
\email{bianca.diblasio@unimib.it}

\address
{Scuola Normale Superiore\\
Piazza dei Cavalieri 7\\
56126 Pisa\\ Italy}
\email{fricci@sns.it}

\thanks{Work partially supported
by  MIUR (project ``Analisi armonica").
}

\subjclass[2000]{Primary: 43A80  
; Secondary:  22E25              
}                         

\keywords{Fourier trasform, Schwartz space, Paley--Wiener Theorems, Heisenberg group}

\begin{abstract}
 We prove several Paley--Wiener-type theorems related to the spherical transform
on the Gelfand pair $\big({{H_{n}}}\rtimes{{\text{U}({n})}},{{\text{U}({n})}}\big)$, where  ${{H_{n}}}$ is the 
$2{n}+1$-dimensional Heisenberg group. 

Adopting the standard realization of the Gelfand spectrum as the Heisenberg fan in ${\mathbb R}^2$, 
we prove that spherical transforms of ${{\text{U}({n})}}$--invariant functions and distributions 
with compact support in ${{H_{n}}}$ admit unique entire extensions to ${\mathbb C}^2$, 
and we find  real-variable characterizations of such transforms.
Next, we characterize the inverse spherical transforms of compactly supported functions 
and distributions on the fan, giving analogous characterizations. 
\end{abstract}

\maketitle

 
 

\section{Introduction}

The spherical transform for the Gelfand pair $\big({{H_{n}}}\rtimes{{\text{U}({n})}},{{\text{U}({n})}}\big)$ maps ${{\text{U}({n})}}$--invariant
functions, i.e. radial functions,
on the Heisenberg group ${{H_{n}}}$ to functions on the Heisenberg fan $\Sigma$, 
which is naturally realized as a closed subset of~${\mathbb R}^2$, the {\it Heisenberg fan} defined in \eqref{cfan}. 
In \cite{ADR,ADR1} we have studied the image of the space ${{\mathcal S}_{\text{\rm rad}}({{H_{n}}})}$ of radial Schwartz functions, 
showing that it consists of the restrictions to $\Sigma$ of Schwartz functions on ${\mathbb R}^2$.

In this paper we first use this result to extend the notion of spherical transform 
to tempered radial distributions, identifying such transforms 
as the distributions on ${\mathbb R}^2$ which are ``synthetizable''   on $\Sigma$, i.e., 
vanish on functions which are  identically zero on the fan.
Then we prove Paley--Wiener type theorems
for the spherical transform ${\mathcal G} $ and its inverse.

The natural starting point   for establishing Paley-Wiener theorems for ${\mathcal G}$ is the fact that, when $f$ has compact support, its spherical transform ${\mathcal G} f$ can be extended
from the set of bounded spherical functions (the Gelfand spectrum) to the set of all spherical 
functions. Spherical functions are parametrized by the pairs $(\xi,\lambda)\in{\mathbb C}^2$ of their 
eigenvalues with respect to the two fundamental differential operators, $L$ (the sublaplacian) and 
$i{^{-1}} T$ (the central derivative).    Moreover, the
spherical function $\Phi_{\xi,\lambda}$ with eigenvalues $(\xi,\lambda)\in{\mathbb C}^2$ depends holomorphically on $(\xi,\lambda)$.
This allows to extend the  spherical transform of a function or distribution
with compact support to an entire function on ${\mathbb C}^2$.

Symmetrically, each spherical function $\Phi_{\xi,\lambda}$ extends to an entire function 
on the complexification  $ H_n^{\mathbb C}$ of ${{H_{n}}}$, and  the inversion formula shows that 
if   ${\mathcal G} f$  has compact support in the Gelfand spectrum, then 
the function itself extends to an entire function   on $H_n^{\mathbb C}$.

It does not look plausible to have a simple ``complex variable'' description of the entire functions which are in the range of the spherical, or inverse spherical, transform of the space of $C^\infty$-functions, or of distributions, with compact support, see also the comments in Fuhr~\cite{MN}, in a context that is  closely related to ours.

We rather look for analogues of the ``real variable'' characterization of the classical Paley-Wiener spaces in ${\mathbb R}^n$, in the spirit of the works of Bang~\cite{Bang} 
and Tuan~\cite{Tuan}, later expanded and refined by Andersen and deJeu~\cite{Nils}. 
We take the following as our model statement~\cite{Nils}:
 a function $f$ on ${\mathbb R}^n$ is the Fourier transform of a $C^\infty$ function with compact support if and only if it is a Schwartz function and, for some $p\in[1,\infty]$,
\begin{equation}\label{Delta}
 \limsup_{k\to\infty}\|\Delta^kf\|_p^\frac1k<\infty\ .
 \end{equation}
 
 In this case, the left-hand side is finite for every $p$, the ``$\limsup$'' is a limit and, for every $p\in[1,\infty]$,
 $$
 \lim_{k\to\infty}\|\Delta^kf\|_p^\frac1k=\max_{x\in{\rm supp}{\mathcal F}^{-1} f}|x|^2\ .
 $$ 

When restricted to radial functions, this theorem can be reformulated in terms of the spherical transform ${\mathcal G}$ for the Gelfand pair $({\mathbb R}^n\rtimes{\rm SO}_n,{\rm SO}_n)$, given by ${\mathcal G} f({\lambda})=\hat f(\xi)$ with $|\xi|^2={\lambda}\ge0$. Then we have    two different statements, depending on the side of the Fourier transform it is applied on:
\begin{enumerate}
\item[(i)] a function $g$ on the Gelfand spectrum $[0,+\infty)$ is the spherical transform of a radial $C^\infty$ function on ${\mathbb R}^n$ with compact support if and only if it is a Schwartz function and, for some $p\in[1,\infty]$,
$$
 \limsup_{k\to\infty}\|g^{(k)}\|_p^\frac1k<\infty\ ;
 $$
  in this case, for every $p\in[1,\infty]$, 
$$
 \lim_{k\to\infty}\|g^{(k)}\|_p^\frac1k=\max_{x\in{\text{\rm supp\,}} {\mathcal G}{^{-1}} g}|x|^2\ .
 $$
\item[(ii)] a radial function $f$ on ${\mathbb R}^n$ is the inverse spherical transform of a $C^\infty$ function with compact support in $[0,+\infty)$ if and only if it is a Schwartz function and \eqref{Delta} holds for some $p\in[1,\infty]$; in this case, for every $p\in[1,\infty]$,
 $$
 \lim_{k\to\infty}\|\Delta^kf\|_p^\frac1k=\max_{{\lambda}\in{\rm supp}{\mathcal G} f}{\lambda}\ .
 $$
 \end{enumerate}
 
 We regard (i) as a Paley-Wiener theorem for the (direct) spherical transform, and (ii) as a Paley-Wiener theorem for the inverse spherical transform.
 
Possible  analogues of (i) and (ii) for the pair $\big({{H_{n}}}\rtimes{{\text{U}({n})}},{{\text{U}({n})}}\big)$ rely on the identification, for each direction, of  a differential operator on one side of the spherical transforms and the corresponding ``norm'' on the other side. 

Our results are related to the following choices:
\begin{enumerate}
\item[(i')]  the difference/differential  operators $M_\pm$ of Benson, Jenkins and Ratcliff \cite{BJR} on $\Sigma$ and the Kor\'anyi norm   \eqref{norm} on $H_n$;
\item[(ii')] the sublaplacian on $H_n$ and its eigenvalue $\xi$ on~$\Sigma$.
\end{enumerate}

We first prove real Paley-Wiener theorems for the direct spherical transform, i.e. analogues of (i) with the ingredients in (i'). We
 treat the cases of $C^\infty$ and $L^2$ functions and of  tempered distributions. These characterizations are summarized in Therorem~\ref{unoequiv},  Corollary~\ref{L2} and Theorem~\ref{sch}.   
 
 We also remark that 
the (unique) entire extension of the transform of  a function in ${\mathcal D}_{\text{\rm rad}}({{H_{n}}})$
needs not be Schwartz on ${\mathbb R}^2$. This shows that, in general, the Schwartz extensions 
to ${\mathbb R}^2$ constructed in \cite{ADR1} are  different from  the entire extension discussed here.

 
In the second part of the paper, we show that, given a  distribution~$U$ on ${\mathbb R}^2$ with compact support , the inversion formula for the spherical transform produces a function on the Heisenberg group ${{H_{n}}}\simeq {\mathbb R}^{2{n}+1}$ which can be analytically extended
 to ${\mathbb C}^{2{n}+1}$. 
 If the distribution $U$ is synthetizable on $\Sigma$, the function so obtained on $H_n$ coincides with its inverse spherical transform. For such distributions $U$, we obtain a real Paley-Wiener analogue of (ii) with the ingredients in~(ii').  A similar theorem is also proved for functions on $\Sigma$ which are either restrictions of $C^\infty$ functions or are square integrable with respect to the Plancherel measure.
 Our characterizations are summarized in Therorem~\ref{maininv}, Theorem~\ref{L2inv} and Theorem~\ref{schinv}.
 These results can be interpreted as a ``real''  spectral Paley-Wiener theorems 
 for the spectral measure of the sublaplacian, a point of view which coincides  with that of \cite{MN}. 

There is a wide literature on Paley--Wiener theorems on the Heisenberg group. 
The earliest result is due to Ando~\cite{PJapAc},
followed by
 Thangavelu~\cite{Revista,JFA,HirMJ}, 
Arnal and Ludwig \cite{PAMS},
 Narayanan and Thangavelu \cite{AnnIFour}.
  Results are mostly  related to the group (operator-valued) Fourier transform 
 and its inverse, but there are also ``spectral'' Paley--Wiener theorems,
  as in the already mentioned paper \cite{MN}, where the condition of compact support 
  on the transform of a given function is replaced 
 by the condition that the function itself belongs to the image of the 
 spectral measure of a compact set in ${\mathbb R}^+$ 
 associated to the sublaplacian (see also Strichartz \cite{JFA-Laplacians},
  Bray \cite{Monats.M},
 and Dann and \'Olafsson~\cite{Olaf} in other contexts).
  
Our paper is organized as follows. In Section~2 we introduce the basic notation. In Section~3 we treat spherical functions
noting that they can be extended to holomorphic functions in each variable
and providing some easy estimates. Section~4 and Section~5 deal 
with the spherical transform of radial functions and radial tempered distributions, respectively. 
In Section 6 we prove some properties of  the operators $M_\pm$
first introduced in \cite{BJR}.
These are exploited in Sections 7 and 8 to obtain real Paley--Wiener theorems
for the spherical transform and its inverse, respectively.

\section{Notation}
We denote by ${{H_{n}}}$ the Heisenberg group, i.e.,
the real manifold ${\mathbb C}^{n}\times{\mathbb R}$ equipped with the group law
$$
(z,t)(w,u)=\bigl(z+w,t+u+\tfrac12\,{\text{\rm Im}\,} \langle w | z \rangle \bigr)
\qquad\forall z,w\in {\mathbb C}^{n},\quad
t,u\in {\mathbb R},
$$
where $\langle\cdot|\cdot\rangle$ denotes the Hermitian innner product in ${\mathbb C}^{n}$.

It is easy to check that the Lebesgue measure $dm=dz\,dt$ is a Haar measure on ${{H_{n}}}$.
\medskip

We denote by $T$, $Z_j$ and $\bar{ Z_j}$, where $j=1,\ldots,{n}$,
the left-invariant vector fields
$$
Z_j=\partial_{z_j}-\tfrac{i}{4}\, \bar{z}_j\,\partial_t
\qquad
\bar{Z}_j=\partial_{\bar{z}_j}+\tfrac{i}{4}\, z_j\,\partial_t ,
\qquad T=\partial_t\ .
$$
The only nontrivial brackets are $T=-2i\,[Z_j,\bar Z_j]$.

The operators $Z_j$ and $\bar Z_j$ are homogeneous of degree $1$
while $T$ is homogeneous of degree~$2$
with respect to the anisotropic dilations
$r\cdot (z,t)=(rz,r^2t)$, where $r>0$ and $(z,t)\in {{H_{n}}}$.
Let $I=(i_1,\ldots,i_{2{n}+1})$ be in ${\mathbb N}^{2{n}+1}$;
we denote by ${{D}}^I$ a
differential operator of homogeneous degree 
$\text{deg}\,I=i_1+\cdots+i_{2{n}}+2i_{2{n}+1}$ of the form
\begin{equation}
\label{monomi}
{{D}}^I=Z_{1}^{i_1} \bar Z_1^{i_2}
\cdots Z_{n}^{i_{2{n}-1}} \bar Z_n^{i_{2{n}}} T^{i_{2{n}+1}} .
\end{equation}
The monomials ${{D}}^I$ with  $\text{deg}\,I=j$
form a basis of the space of all left-invariant differential operators on ${{H_{n}}}$ 
which are homogeneous of degree
$j$.

We write ${\mathcal S} ({{H_{n}}})$ for the Schwartz space of
functions on ${{H_{n}}}$, i.e., the space of infinitely differentiable
functions~$f$ on~${{H_{n}}}$ such that
all  partial derivatives~${{D}}^I f$ of~$f$
are rapidly decreasing.
The Schwartz space is equipped with the following
family of norms, parametrized by a nonnegative integer $p$:
$$
\|f\|_{(p)}=\sup_{(z,t)\in {{H_{n}}}}\{(1+| (z,t)|)^{p}\, |{{D}}^I f(z,t)|
\,:\, \text{deg}\, I\leq p\}\ ,
$$
where 
\begin{equation}\label{norm}
| (z,t)|=\left( \frac{|z|^4}{16}+t^2\right)^{1/4} .
\end{equation}
We also define ${\mathcal A}$ as    
$$
\mathcal A(z,t)= \frac{|z|^2}{4}+it 
\qquad \forall (z,t)\in {{H_{n}}},
$$ 
so that 
$
|A(z,t)|=| (z,t)|^2.
$

\section{Spherical functions}
The unitary group ${{\text{U}({n})}}$  
acts on ${{H_{n}}}$ via
$$
k\cdot (z,t)=(kz, t) \qquad \forall (z,t)\in {{H_{n}}},\quad
k\in {{\text{U}({n})}}.
$$ 
This action induces an action on functions~$f$ on ${{H_{n}}}$ by the formula
$$
k\cdot f(z,t)=f(k^{-1}z,t)\qquad \forall k\in {{\text{U}({n})}},\quad (z,t)\in {{H_{n}}}.
$$
We note that a function $f$ on ${{H_{n}}}$ is ${{\text{U}({n})}}$--invariant if and only if
it depends
only on $|z|$ and~$t$
, therefore we shall call it radial.
We denote by ${{\mathcal S}_{\text{\rm rad}}({{H_{n}}})}$ the space
 of  radial
 Schwartz  functions.
  
  Denote by $G$ the   semidirect product ${{H_{n}}}\rtimes {{\text{U}({n})}}$.
    We may identify the space of smooth bi-${{\text{U}({n})}}$--invariant functions  ${\mathcal D}(G/\!/{{\text{U}({n})}})$ 
with the   algebra  ${{\mathcal D}_{\text{\rm rad}}}({{H_{n}}})$ 
of smooth radial  functions on ${{H_{n}}}$ with compact support. It is known~\cite{HR,DR} that 
$(G,{{\text{U}({n})}})$ is a Gelfand pair, i.e.,
${{\mathcal D}_{\text{\rm rad}}}({{H_{n}}})$  is a commutative algebra.  
We may also identify the commutative 
algebra $\mathbb D(G/{{\text{U}({n})}})$ of $G$--invariant differential operators on 
$G/{{\text{U}({n})}}$   with the algebra
${{\mathbb D}_{\text{\rm rad}}}$ of all left-invariant and
${{\text{U}({n})}}$--invariant differential operators on ${{H_{n}}}$, which    has
two essentially self-adjoint 
generators, namely $i^{-1}T$
and the 
  sublaplacian
$$
L=-2\sum_{j=1}^{n}\bigl( Z_j\bar Z_j+\bar Z_j Z_j\bigr) .
$$

The   spherical 
functions are characterized as the  
joint eigenfunctions
of all 
$G$--invariant differential operators on 
$G/{{\text{U}({n})}}$, i.e., as the radial eigenfunctions of $i^{-1}T$
and $L$,  normalized to take value $1$ at identity.
Spherical functions are analytic and are uniquely determined by the pair  of their eigenvalues relative to 
$L$ and $i^{-1}T$ respectively. 

The next subsection  shows that the spherical function 
$\Phi_{\xi,{\lambda}}$ exists for every pair $(\xi,{\lambda})$ of 
eigenvalues  and that  depends  holomorphically
on variables and parameters.

\subsection{Holomorphy of spherical functions}

We initially consider real eigenvalues $\xi$ and ${\lambda}$ and look for a radial solution of the system
\begin{equation}\label{sistema}
\left\{
\begin{array}{l}
Lu=\xi\, u
\\
Tu=i{\lambda}\, u
\\
u(0,0)=1\ .
\end{array}
\right.
\end{equation}
Following~\cite{Kora}, for ${\lambda}\neq 0$  we write the solution in the form 
$$
u(z,t)=e^{i{\lambda} t}\,e^{-{\lambda} |z|^2/4}\,v({\lambda} |z|^2/2),
$$
obtaining that $v$ satisfies
the confluent hypergeometric  differential equation
\begin{equation}\label{eqdiffconfl}
s\,v''(s)+(c-s)v'(s)-a\,v(s)=0
\end{equation}
with parameters $a=\frac{n}{2}-\frac{\xi}{2{\lambda}}$, $c={n}$.
The normalized solution of {~(\ref{{eqdiffconfl}})} is the confluent hypergeometric function
$$
{ {_{1} F}_{\!1}}(a,c;s)=1+\frac{a}{c}\, s+\frac{a(a+1)}{c(c+1)}\, \frac{s^2}{2!}+\cdots
=\sum_{k=0}^\infty \frac{(a)_k}{(c)_k}\, \frac{s^k}{k!}\ ,
$$
where $(a)_0=1$, $(a)_k=\Gamma(a+k)/\Gamma(a)$,
so that for real ${\lambda}\neq 0$
$$
u(z,t)=e^{i{\lambda} t}\,e^{-{\lambda} |z|^2/4}\,
{ {_{1} F}_{\!1}}(\tfrac{n}{2}-\tfrac{\xi}{2{\lambda}},{n};{\lambda} |z|^2/2)\ .
$$
When ${\lambda}=0$  and $\xi$ is real,
a similar procedure shows that 
$$
u 
(z,t)=\sum_{k=0}^\infty \frac{(-1)^k}{({n})_k}\,
\frac{(\xi |z|^2/4)^k}{k!}={\mathcal J}_{{n}-1}(\xi |z|^2/4)
\qquad\forall (z,t)\in {{H_{n}}},
$$
where 
$$
{\mathcal J}_{\beta}(s) 
= 
\sum_{k=0}^{+\infty}
\frac{(-s)^k }{k!\,(\beta+1)_k}
\qquad \forall s\in {\mathbb C}.
$$
Note that $J_{\beta}(u)=\displaystyle\frac{(u /2)^{\beta }}{\beta!\,\,}  
{\mathcal J}_{\beta}\left(u^2 /4\right)$
  is the   Bessel function of the first kind of order $\beta$.
  
  Therefore for every pair of  real 
numbers $\xi$ and ${\lambda}$ we have the spherical function
\begin{equation*} 
  \Phi_{\xi,{\lambda}}(z,t)=
\begin{cases}
 e^{i{\lambda} t}\, e^{-{\lambda} |z|^2/4}\, 
{ {_{1} F}_{\!1}} \left(\frac{n}{2} -\frac{\xi}{2 {\lambda}},{n};\frac{{\lambda} |z|^2}{2}\right) 
&  
{\lambda}\not= 0
\\
{\mathcal J}_{{n}-1}(\xi |z|^2/4)&    
{\lambda} = 0
\end{cases}
\qquad \forall (z,t)\in {{H_{n}}}.
\end{equation*}
 We now verify
  that  ${\lambda}\longmapsto  \Phi_{\xi,{\lambda}}(z,t)$ is regular in ${\lambda}=0$.
Indeed, something more holds.
 
\begin{lemma}\label{olosferiche}
The function  $(x,y,t,\xi,{\lambda})\longmapsto   
\Phi_{\xi,{\lambda}}(x+iy,t)$ extends to a holomorphic function on 
${\mathbb C}^{2{n}+3}$.
\end{lemma}

\begin{proof}
Note that when ${\lambda}\not=0$, $z=x+iy$,
$$
\begin{aligned}
{ {_{1} F}_{\!1}} \left(\frac{{n} {\lambda}-\xi}{2 {\lambda}},{n};\frac{{\lambda} |z|^2}{2}\right)
&= 
\sum_{k=0}^{\infty}
\frac{\left( 
\frac{{n} {\lambda}-\xi}{2 {\lambda}}
\right)_k}
{(n)_k\,k! }
   \left(
 \frac{{\lambda} |z|^2}{2}
 \right)^k 
 \\
  &=1+
\sum_{k=1}^{\infty}
\frac{(x^2+y^2)^{k}}{(n)_k\,k!\,4^k}\,\,
\prod_{d=0}^{k-1}\left(
  {\lambda}(2d+{n})-\xi  \right)
,
  \end{aligned}
$$ 
so that for all $(\xi,{\lambda},x,y,t)$ the function
\begin{equation}\label{Phi}
\Phi_{\xi,{\lambda}}(x+iy,t)
=e^{i{\lambda} t}\, e^{-{\lambda} (x^2+y^2)/4}\, \left(1+
\sum_{k=1}^{\infty}
\frac{(x^2+y^2)^{k}}{(n)_k\,k!\,4^k}\,\,
\prod_{d=0}^{k-1}\left(
  {\lambda}(2d+{n})-\xi  \right)\right)
\end{equation}
is a series of entire functions converging  uniformly
on compact sets.
\end{proof}

By analytic continuation, $\Phi_{\xi,{\lambda}}$ 
is the spherical function for every $(\xi,{\lambda})\in{\mathbb C}^2$.

\subsection{Bounded spherical functions} 
 The Gelfand spectrum of the Banach algebra ${L^1_{\text{\rm rad}}({{H_{n}}})}$ of radial integrable functions  
is given by the set 
  of normalized bounded spherical functions, equipped with the compact open topology.   
 We recall
  \cite{Kora}  that  $\Phi_{\xi,{\lambda}}$ is bounded on ${{H_{n}}}$ 
  if and only if $(\xi,{\lambda})$ 
belongs to the so called  Heisenberg fan  given by
\begin{equation}\label{cfan}
{\Sigma}={{\Sigma^*}}\cup {\left\{{(\xi,0)\in {\mathbb R}^2\,\,:\,\, \xi\geq 0}\right\}},
\end{equation}
where
$$
{{\Sigma^*}}={\left\{{(\xi,{\lambda})\in{\mathbb R}^2:{\lambda}\ne0\,,\, \xi=|{\lambda}|(2j+ {n})\,,\, \,  j\in {\mathbb N} }\right\}}\ .
$$
It is known that ${\Sigma}$ is homeomorphic to  the Gelfand spectrum~\cite{FR, BJRW}. 
{{}}
 
When $(\xi,{\lambda})$ is in ${{\Sigma^*}}$, the spherical function
$\Phi_{\xi,{\lambda}}$ can be written 
in terms of Laguerre polynomials, which is the form that we usually find in literature.
Indeed,  
the relation (see \cite[p.~253, formula~(7)]{E}
\begin{equation}\label{lebedev}
{ {_{1} F}_{\!1}}(a,{n};x)=e^x\, { {_{1} F}_{\!1}}({n}-a, {n};-x),
 \end{equation} 
implies that 
$$
\Phi_{\xi,{\lambda}}(z,0)
=\Phi_{\xi,|{\lambda}|}(z,0)=
e^{-|{\lambda}| |z|^2/4}\,
{ {_{1} F}_{\!1}} \left(\frac{n}{2}-\frac{\xi}{2 |{\lambda}|},{n};\frac{|{\lambda}| |z|^2}{2}\right)
$$
and when $\xi=|{\lambda}|(2j+{n})$ 
the hypergeometric function in the previous formula coincides with 
the normalized
   $j^{\text{th}}$ Laguerre polynomial
 of order ${n} -1$, i.e., 
 $$
 { {_{1} F}_{\!1}} \left(-j,{n};\frac{|{\lambda}| |z|^2}{2}\right) 
 =\frac1{\binom{j+ {n}-1}{j}}\sum_{k=0}^j \binom{j+\beta }{ j-k}
   \frac{(-\frac{|{\lambda}| |z|^2}{2})^k}{ k!}.
 $$
\medskip

\subsection{Estimates of derivatives of spherical functions}
In this subsection we exploit the fact that  the bounded  spherical functions 
$\left\{\Phi_{\xi,{\lambda}}\right\}_{(\xi,{\lambda})\in {\Sigma}}$  are averages of coefficients of irreducible 
unitary representations of ${{H_{n}}}$   
to give some estimates that we shall need in the sequel.
Referring to the Bargmann-Fock model 
of the irreducible representations $\pi^\lambda$ of $H_n$ 
associated to the character $e^{i\lambda t}$ on the center, 
we represent the operators $\pi^\lambda(z,t)$ as matrices 
$\big(\pi^\lambda_{\bf j,\bf k}(z,t)\big)_{\bf j, \bf k\in {\mathbb N}^{n}}$ 
in the basis of normalized monomials (for more details see the monographs~\cite{Folland}
or~\cite{Th}). Then the bounded spherical functions 
can be written as averages of
diagonal entries of this matrix   according to the rule
\begin{equation}\label{defPhi}
\Phi_{\xi,{\lambda}}=\frac1{\binom{j+ {n}-1}{j}}\sum_{|{\bf j}|=j}\pi^\lambda_{\bf j,\bf j},
\end{equation}
 where 
$\xi=|{\lambda}|(2j+{n})$ and $|{\bf j}|={\bf j}_1+\ldots +{\bf j}_n$ for $ {\bf j}  =\left({\bf j}_1, \ldots ,{\bf j}_n\right)\in {\mathbb N}^{n}$.

\begin{lemma}\label{stimasferiche} Let ${{D}}^I$ be a
differential operator of homogeneous degree $\deg I=  {\alpha}$ as in{~(\ref{{monomi}})}. Then  
$$
|{{D}}^I \Phi_{\xi,{\lambda}}(z,t)|\leq C_{\alpha}\, (1+\xi)^{{\alpha} /2}
\qquad\forall (\xi,{\lambda})\in {\Sigma},\,\, (z,t)\in {{H_{n}}}.
$$
 \end{lemma}

\begin{proof} Let ${\lambda}\neq 0$.
Here and afterwards, if any component of the multiindeces ${\bf j}$ or ${\bf k}$
is negative, then $\pi^\lambda_{\bf j,\bf k}=0$.
Since the representations are unitary, $|\pi^\lambda_{\bf j,\bf k}|\leq 1$ and
it is easy to check that
$$
Z_i \pi^\lambda_{\bf j,\bf k}=
\begin{cases}
-\sqrt{\frac{k_i{\lambda}}{2}}\, \pi^{\lambda}_{{\bf j},{\bf k}-{\bf e}_i}
&{\lambda}>0
\\
\sqrt{\frac{(k_i+1)|{\lambda}|}{ 2}}\, \pi^{\lambda}_{{\bf j},{\bf k}+{\bf e}_i}
&{\lambda}<0
\end{cases}
\qquad
\bar Z_i \pi^{\lambda}_{{\bf j},{\bf k}}=
\begin{cases}\sqrt{\frac{(k_i+1){\lambda}}{2}}\, \pi^{\lambda}_{{\bf j},{\bf k}+{\bf e}_i}
&{\lambda}>0
\\
-\sqrt{\frac{k_i|{\lambda}|}{ 2}}\,\pi^{\lambda}_{{\bf j},{\bf k}-{\bf e}_i}
&{\lambda}<0\ 
\end{cases}
$$
and $T\pi^{\lambda}_{{\bf j},{\bf j}}=i\,{\lambda}\, \pi^{\lambda}_{{\bf j},{\bf j}}$.
Here ${\bf e}_i$ is the multiindex with just the $i^{th}$ component equal to $1$.

Suppose that $(\xi,{\lambda})$ is in ${{\Sigma^*}}$, with $\xi=|{\lambda}|(2|{\bf j}| +{n})$.
Then if ${{D}}^I =Z^{\bf k}\bar Z^{\bf h} T^s$ with 
$\deg I={\alpha} =|{\bf k}|+|{\bf h}|+2s$ we have 
\begin{align*}
|
{{D}}^I\pi^{\lambda}_{{\bf j},{\bf j}}
|
&
=|{\lambda}|^s\,
|Z^{\bf k}\bar Z^{\bf h}\pi^{\lambda}_{{\bf j},{\bf j}} |
\\
&=|{\lambda}|^s\,
\begin{cases}
\sqrt{\prod_{i=1}^n\prod_{\ell=1}^{h_i}\tfrac{\lambda}{2}(j_i+\ell)}\, 
\sqrt{\prod_{i=1}^n\prod_{\ell=1}^{k_i}\tfrac{\lambda}{2}(j_i+h_i+1-\ell)}\,
|\pi^{\lambda}_{{\bf j},{\bf j+h-k}}|
&{\lambda}>0
\\
\sqrt{\prod_{i=1}^n\prod_{\ell=1}^{h_i}\tfrac{|{\lambda}|}{2}(j_i+1-\ell)}\,
\sqrt{\prod_{i=1}^n\prod_{\ell=1}^{k_i}\tfrac{|{\lambda}|}{2}(j_i-h_i+\ell)}\,
|\pi^{\lambda}_{{\bf j},{\bf j-h+k}}|
&{\lambda}<0
\end{cases}
\\
&\leq C\,
|{\lambda}|^s\,\sqrt{
 |{\lambda}|^{|{\bf h}|+|{\bf k}|}(|{\bf j}|+|{\bf h}|)^{|{\bf h}|}
\,\,\,
(|{\bf j}|+|{\bf h}|+|{\bf k}|)^{|{\bf k}|}}
\\
&\leq C_{\alpha}\, (1+\xi)^{{\alpha}/2}.
\end{align*}
Here we have used the fact that in ${{\Sigma^*}}$ we have $\xi=|{\lambda}|(2|{\bf j}|+{n})\geq |{\lambda}|$.
By~{~(\ref{{defPhi}})} the thesis follows on ${{\Sigma^*}}$, and 
by continuity the same estimates hold on ${\Sigma}$, thus proving the lemma. 
\end{proof}

\section{Spherical transform}
As usual, we denote by  ${\langle {\cdot},{\cdot} \rangle_{{H_{n}}}}$ the dual  pairing on 
 ${{H_{n}}}$ and we shall also write 
$$
{\langle {f},{g} \rangle_{{H_{n}}}}=\int_{{H_{n}}} f(z,t)\, g(z,t)\, dz\, dt,
  \qquad \forall f,g\in {\mathcal S}({{H_{n}}}).
$$
Given a measurable  function $f$ on ${{H_{n}}}$  we  
denote by $\check f$   the function defined by 
 $\check f(x)=f(x^{-1})$ for every $x$ in ${{H_{n}}}$.

\subsection{Definitions and main facts}
\label{defgel}
Let $f$ be in ${L^1_{\text{\rm rad}}({{H_{n}}})}$. We  define   its   spherical 
transform~${\mathcal G} f$     by  
$$
{\mathcal G} f(\xi,\lambda) =\int_{{H_{n}}} f(x)\, \Phi_{\xi,\lambda}(x^{-1})\, dx=
{\langle {f},{\check\Phi_{\xi,\lambda}} \rangle_{{H_{n}}}}
\qquad \forall (\xi,{\lambda})\in {\Sigma}.
$$
Then ${\mathcal G} f$ is a continuous function on ${\Sigma}$. 

 The inversion formula for a function~$f$ in ${{\mathcal S}_{\text{\rm rad}}({{H_{n}}})}$ is 
$$
f(x)=\int_{\Sigma} {\mathcal G} f(\xi,{\lambda})\, \Phi_{\xi,{\lambda}}(x)\, d\mu(\xi,{\lambda})
\qquad \forall x\in {{H_{n}}},
$$
where $\mu$ is the Plancherel measure defined by  
$$
\int_{\Sigma} \psi \, d\mu
=\frac{1}{(2\pi)^{{n}+1}}\,\int_{\mathbb R} 
  \sum_{j=0}^{\infty}\binom{j+ {n}-1}j\,
  \psi(|{\lambda}|(2j+ {n}),{\lambda})\, |{\lambda}|^{n}\, d{\lambda}
\qquad\forall \psi\in  
C_c({\Sigma}).
$$

It is easy to check that the function $(\xi,{\lambda})\mapsto (1+\xi)^{-({n}+2)}$
is in $L^1({\Sigma})$, so  
\begin{equation}
\label{integrabilita}
\| \psi \|_{L^1({\Sigma})}\leq C\,
\|(1+|\xi|)^{{n}+2}\,\psi \|_{L^\infty({\mathbb R}^2)}
\qquad \forall \psi\in {\mathcal S}({\mathbb R}^2).
\end{equation}

As in~\cite{ADR}, we denote by 
${\mathcal S}({\Sigma})$  the space of 
restrictions to ${\Sigma}$ of Schwartz functions on ${\mathbb R}^2$,
endowed with the quotient topology ${\mathcal S}({\mathbb R}^2)/{\left\{{\phi\, :\, \phi|_{\Sigma}=0}\right\}}$.
For  radial Schwartz functions  on ${{H_{n}}}$, we have proved
the following result.
 
\begin{theorem}\textnormal{\cite[Corollary 1.2]{ADR}} \label{nostro}
The spherical transform is a topological isomorphism between
the spaces ${{\mathcal S}_{\text{\rm rad}}({{H_{n}}})}$
 and ${\mathcal S}({\Sigma})$.
\end{theorem}

On the other hand, Lemma~\ref{olosferiche}
implies that when $f$ is compactly supported
we can regard ${\mathcal G} f$ as a function on ${\mathbb C}^2$.

\begin{proposition}  \label{olomorfia}
If  $f$ is  in ${{\mathcal D}_{\text{\rm rad}}}({{H_{n}}})$ then  ${\mathcal G} f$ extends 
to the  holomorphic function $F$ 
on ${\mathbb C}^2$ given by the rule
$$
F(\xi,{\lambda})={\langle {f},{\check\Phi_{\xi,{\lambda}}} \rangle_{{H_{n}}}}
\qquad\forall (\xi,{\lambda})\in {\mathbb C}^2.
$$
\end{proposition}

\subsection{Holomorphic versus Schwartz extensions}\label{sec:olo}
Given $f$ in ${{\mathcal D}_{\text{\rm rad}}}({{H_{n}}})$, we have found two
ways of extending its spherical transform ${\mathcal G} f$ to a smooth 
function
on ${\mathbb R}^2$. Namely, by Theorem~\ref{nostro}, there exists a Schwartz function
$G$ on ${\mathbb R}^2$ such that $G|_{\Sigma}={\mathcal G} f$, and by Proposition~\ref{olomorfia}
the  function $F$ is the holomorphic extension of ${\mathcal G} f$ to ${\mathbb C}^2$.
So $G|_{\Sigma}=F|_{\Sigma}={\mathcal G} f$.

We observe that any two entire functions on ${\mathbb C}^2$,
which coincide on~${\Sigma}$, 
are everywhere equal, so $F$ is the unique
continuation of ${\mathcal G} f$
to an entire function on ${\mathbb C}^2$. 

A question arises naturally: 
if $f$ is  in ${{\mathcal D}_{\text{\rm rad}}}({{H_{n}}})$, is it true that $F$, 
when restricted to real values of $(\xi,{\lambda})$,  
is a Schwartz function on ${\mathbb R}^2$?

In the rest of this subsection we show that the answer   can be negative.

Let $f$ be a function of the form 
$$
f(z,t)=g(z)\otimes h(t) \qquad \forall (z,t)\in {{H_{n}}}
$$ 
where $h$ is even and
compactly supported
and $g$ is nonpositive and supported in  $1<|z|<4$,  equal to $-1$
when $2<|z|<3$.

Let $\mathcal{F}h$ be the Euclidean Fourier transform of $h$.
We now show that the function ${\lambda}\ge 0\mapsto |\mathcal{F}h({\lambda})|\,e^{{\lambda}/2}$
is not bounded. Indeed, since $h$ is even, if it were bounded, 
then  
for every $b$, $0\leq b<1/2$, the function 
${\lambda}\mapsto e^{b|{\lambda}|}\mathcal{F}h({\lambda})$ would be in $L^2({\mathbb R})$. By 
the Paley--Wiener Theorem for the Euclidean transform 
$h=\mathcal{F}^2h$ would continue analytically to
 ${\left\{{w\,:\, |{\text{\rm Im}\,}(w)|<1/2}\right\}}$, but this cannot
 be true since $h$ has compact support.
 Therefore the function ${\lambda}\ge 0\mapsto |\mathcal{F}h({\lambda})|\,e^{{\lambda}/2}$
is not bounded. 

Now, if $(\xi,{\lambda})\in {\mathbb R}^2\mapsto F(\xi,{\lambda})$ were rapidly decreasing, then the same
would hold true for the function
${\lambda}\mapsto F(({n}+1){\lambda},{\lambda})$. Note that when ${\lambda}>0$
$$
F(({n}+1){\lambda},{\lambda})=\mathcal{F}h({\lambda}) 
\int_{1<|z|<4}g(z)\, e^{-{\lambda} |z|^2/4}\, { {_{1} F}_{\!1}}(-\tfrac{1}{2}, {n},{\lambda} |z|^2/2)\, dz.
$$

 
Moreover ${ {_{1} F}_{\!1}}(-\tfrac{1}{2},{n},x)\leq 0$ when $x\geq 2{n}$
 and by the estimate 
(see~\cite[p.~27, formula~(3)]{E})
$${ {_{1} F}_{\!1}}(a,{n};x)=\frac{\Gamma({n})}{\Gamma(a)}e^x\,x^{a-{n}}(1+O(x^{-1})),
\qquad {\text{\rm Re}\,}(x)\to\infty,\quad a\not=0,-1,-2,\ldots,
$$
when ${\lambda}\to+\infty$ we obtain 
\begin{align*}
|F(({n}+1){\lambda},{\lambda})|
&=|\mathcal{F}h({\lambda})|\, 
\int_{1<|z|<4}  g(z)\,e^{-{\lambda} |z|^2/4}\, { {_{1} F}_{\!1}}(-\tfrac{1}{2}, {n},{\lambda} |z|^2/2)\, dz
\\
&
\geq C\,
|\mathcal{F}h({\lambda})|\, \int_{2<|z|<3}  e^{-{\lambda} |z|^2/4}\,
e^{{\lambda} |z|^2/2}\, ({\lambda} |z|^2/2)^{-1/2-{n}}\,\, dz
\\
&\geq C\, |\mathcal{F}h({\lambda})|\,e^{{\lambda}/2},
\end{align*}
so the function ${\lambda}\mapsto F(({n}+1){\lambda},{\lambda})$ is not bounded.
 

 
\section{The spherical transform of radial tempered distributions}
   
As usual, we denote by ${\langle {\cdot},{\cdot} \rangle_{{\mathbb R}^2}}$   the dual  pairing on 
${\mathbb R}^2$   and we also write 
$$
{\langle {\varphi},{\psi} \rangle_{{\mathbb R}^2}}=\int_{{\mathbb R}^2} \varphi(x)\,\psi(x)\, dx\qquad \qquad
\forall \varphi,\psi\in {\mathcal S}({{\mathbb R}^2}).
$$  
Let $\Pi\, :\, {\mathcal S}({{H_{n}}}) \longrightarrow {\mathcal S}({{H_{n}}}) $ be the averaging projector  
defined by  
$$
\Pi f 
=\int _\Unf\circ k\,\,dk\qquad \forall f\in {\mathcal S}({{H_{n}}}).
$$
Then the Schwartz space on ${{H_{n}}}$ decomposes into the direct sum 
\hbox{${\mathcal S}({{H_{n}}})= {{\mathcal S}_{\text{\rm rad}}({{H_{n}}})} \oplus \ker \Pi$} so that  
$ {{\mathcal S}_{\text{\rm rad}}({{H_{n}}})} $  is isomorphic to the quotient space 
 ${\mathcal S}({{H_{n}}})/ \ker \Pi$. It follows that  
 the dual space  $ \big({{\mathcal S}_{\text{\rm rad}}({{H_{n}}})}\big)' $ is isomorphic    
 to the subspace ${{\mathcal S}'_{\text{\rm rad}}({{H_{n}}})}$ of ${\mathcal S}'({{H_{n}}})$
 consisting of all 
tempered distributions $\Lambda$  on ${{H_{n}}}$ such that
$$ 
{\langle { \Lambda},{ f} \rangle_{{H_{n}}}}=0\ \qquad\forall\,f\in\ker \Pi.
$$
 On the other hand the dual space of ${\mathcal S}({\Sigma})$ 
 is naturally isomorphic   to the subspace ${\mathcal S}_0'({\Sigma})$ of  ${\mathcal S}'({\mathbb R}^2)$ 
 consisting of all 
tempered distributions $U$  on ${\mathbb R}^2$ such that
\begin{equation*} 
{\langle { U},{ g} \rangle_{{\mathbb R}^2}}=0\ ,\qquad\forall\,g\in{\mathcal S}({\mathbb R}^2)\text{ such that }g=0\text{ on }{\Sigma}\ .
\end{equation*}
 We note that the Plancherel formula can be written as
$$
{\langle {f},{\overline g} \rangle_{{H_{n}}}}= {\langle {{\mathcal G} f\, \mu},{\overline{{\mathcal G} g}} \rangle_{{\mathbb R}^2}}
= {\langle {{\mathcal G} f\, \mu},{{{\mathcal G} \overline{\check g}}} \rangle_{{\mathbb R}^2}}
\qquad \forall f,g\in {{\mathcal S}_{\text{\rm rad}}({{H_{n}}})}.
$$  
Therefore we are led to define
the spherical transform of a radial  tempered distribution $\Lambda$ on ${{H_{n}}}$ as the 
distribution ${\mathcal G} \Lambda$ in ${\mathcal S}'({\mathbb R}^2)$ given  by 
\begin{equation}\label{tildeT}
{\langle {{\mathcal G} \Lambda},{\varphi} \rangle_{{\mathbb R}^2}}={\langle {\Lambda},{({\mathcal G}{^{-1}} {\varphi}_{|_\Sigma})^{\check{\phantom a}}} \rangle_{{H_{n}}}}
\qquad \forall {\varphi}\in {\mathcal S}({\mathbb R}^2). 
\end{equation}
Clearly, if $\Lambda$ is in ${{\mathcal S}'_{\text{\rm rad}}({{H_{n}}})}$, then for every 
function ${\varphi}$ in ${\mathcal S}({\mathbb R}^2)$ such that ${\varphi}=0$  on ${\Sigma}$ we have
$
{\langle {{\mathcal G} \Lambda},{\varphi} \rangle_{{\mathbb R}^2}}
={\langle {\Lambda},{({\mathcal G}{^{-1}} {\varphi}_{|_\Sigma})^{\check{\phantom a}}} \rangle_{{H_{n}}}}
=0,
$
i.e., ${\mathcal G}\Lambda$ is in ${\mathcal S}_0'({\Sigma})$.  
We recall that we have denoted by ${m}$ the Lebesgue measure on ${{H_{n}}}$.
If  $f$ is a radial function in $L^1\cap L^2({{H_{n}}})$,
then $f{m}$ is in ${{\mathcal S}'_{\text{\rm rad}}({{H_{n}}})}$
and ${\mathcal G} (f{m})=({\mathcal G} f)\,\mu$ where ${\mathcal G} f$ has been defined in Subsection~\ref{defgel},
so formula~{~(\ref{{tildeT}})} provides an extension of the usual spherical transform.
 

Moreover it is easy to verify 
that ${\mathcal G}{(L^j \Lambda)} 
=\xi^{j}\, {\mathcal G} \Lambda 
$
for every $\Lambda $ in ${{\mathcal S}'_{\text{\rm rad}}({{H_{n}}})}$.
{{}}

With this notation Theorem~\ref{nostro} extends to radial tempered distributions 
in the following form.

 
\begin{corollary}\label{nostrodist} The spherical transform 
${\mathcal G}$  is a topological isomorphism between
the spaces ${{\mathcal S}'_{\text{\rm rad}}({{H_{n}}})}$
 and ${\mathcal S}_0'({\Sigma})$.
\end{corollary}
 
 We now study the behavior of the spherical transform of 
  radial compactly supported distributions.    
\begin{proposition}\label{distolo} 
Let $\Lambda$ be a radial compactly supported distribution on ${{H_{n}}}$. Then   
\begin{equation}
\widehat \Lambda\; :\; (\xi,\lambda)  \longmapsto {\langle {\Lambda},{\check\Phi_{\xi,\lambda}} \rangle_{{H_{n}}}}
\end{equation}
   is a holomorphic function on ${\mathbb C}^2$. Moreover
 $\widehat \Lambda\,\mu$ is in ${\mathcal S}_0'({\Sigma})$ and
$$
{\mathcal G} \Lambda= \widehat \Lambda\,\mu\ ,
$$
i.e., ${\mathcal G} \Lambda$ coincides with the function $\widehat \Lambda$.
\end{proposition}
 

\begin{proof} Using Lemma~\ref{olosferiche}
it is easy to prove that $\widehat \Lambda$ is entire.
For the second part, we first check that 
for every $\psi$ in ${\mathcal S}({\mathbb R}^2)$, the integral $\int_{\Sigma} \widehat \Lambda\, \psi\,d\mu$ 
is absolutely convergent, and therefore $\widehat \Lambda\,\mu$ is in ${\mathcal S}_0'({\Sigma})$.
Indeed, if $(\xi,{\lambda})=\big(|{\lambda}|(2j+{n}),{\lambda}\big)$ is in~${{\Sigma^*}}$, for some $m$ in ${\mathbb N}$
$$
|\widehat \Lambda(\xi,{\lambda})|=
|{\langle { \Lambda},{\check{\Phi}_{\xi,{\lambda}}} \rangle_{{H_{n}}}}| \le C\,\big\|
{\check\Phi_{\xi,{\lambda}}}\,\big\|_{C^m(K)}\ ,
$$
where $K={\text{\rm supp\,}} \Lambda \subset {\Sigma}$.
By Lemma~\ref{stimasferiche}, the function
 $\widehat \Lambda$ is slowly growing on $\Sigma$
and so for every $\psi $ in ${\mathcal S}({\mathbb R}^2)$, the integral $\int_{\Sigma} \widehat \Lambda\, \psi\,d\mu$ 
is absolutely convergent.

When $g$ is in ${\mathcal S}({\mathbb R}^2)$,
\begin{align*}
{\langle { \widehat \Lambda\,\mu},{\psi} \rangle_{{\mathbb R}^2}}
&=\int_{\Sigma} \widehat \Lambda(\xi,{\lambda})\, \psi(\xi,{\lambda})\,d\mu\\
&=\frac{1}{(2\pi)^{{n}+1}}\,
\int_{\mathbb R}\sum_{j=0}^\infty\binom{j+ {n}-1}j\,\widehat \Lambda\big(|{\lambda}|(2j+{n}),{\lambda}\big)
\, \psi\big(|{\lambda}|(2j+{n}),{\lambda}\big)\,|{\lambda}|^{n}\,d{\lambda}\\
&=\lim_{N\to\infty} \frac{1}{(2\pi)^{{n}+1}}\,
\int_{-N}^N
\sum_{j=0}^N \binom{j+ {n}-1}j\, \widehat \Lambda\big(|{\lambda}|(2j+{n}),{\lambda}\big)\,
\psi\big(|{\lambda}|(2j+{n}),{\lambda}\big)\,|{\lambda}|^{n}\,d{\lambda}\\
&=\lim_{N\to\infty}\frac{1}{(2\pi)^{{n}+1}}\,
\int_{-N}^N\sum_{j=0}^N  \binom{j+ {n}-1}j\,
{\langle { \Lambda},{
\check\Phi_{|{\lambda}|(2j+{n}),{\lambda}}} \rangle_{{H_{n}}}}\, \psi\big(|{\lambda}|(2j+{n}),{\lambda}\big)\,|{\lambda}|^{n}\,d{\lambda}\\
&=\lim_{N\to\infty}\frac{1}{(2\pi)^{{n}+1}}\,
\, {\left\langle { \Lambda},{\int_{-N}^N\sum_{j=0}^N \, \binom{j+ {n}-1}j\,
            \check\Phi_{|{\lambda}|(2j+{n}),{\lambda}}\,\, 
            \psi\big(|{\lambda}|(2j+{n}),{\lambda}\big)\,|{\lambda}|^{n}\,d{\lambda}} \right\rangle_{{H_{n}}}}
\ .
\end{align*}
 

Since
$$
\lim_{N\to\infty}\frac{1}{(2\pi)^{{n}+1}}\,
\int_{-N}^N\sum_{j=0}^N \, \check\Phi_{|{\lambda}|(2j+{n}),{\lambda}}\, 
\psi\big(|{\lambda}|(2j+{n}),{\lambda}\big)\,
|{\lambda}|^{n}\,d{\lambda}=({\mathcal G}{^{-1}}  \psi_{|_\Sigma})\check{\phantom a}
$$
uniformly on compacta and the same holds for all derivatives,
\[
{\langle { \widehat \Lambda\,\mu},{\psi} \rangle_{{\mathbb R}^2}}
={\langle { \Lambda},{({\mathcal G}{^{-1}} \psi_{|_\Sigma})\check{\phantom a}} \rangle_{{H_{n}}}}
={\langle {{\mathcal G} \Lambda},{\psi} \rangle_{{\mathbb R}^2}}\ .\qedhere
\]
\end{proof}

 
\section{The operators ${M}_\pm$}

Denote by   ${M}_\pm$   the  operators acting on a
smooth function   $\psi$  on ${\mathbb R}^2$ by the rule~\cite{BJR}
 
$$
\begin{aligned}
  M_\pm \psi(\xi,{\lambda})&= \partial_{\lambda} \psi(\xi,{\lambda})\mp{n}
   \partial_\xi \psi(\xi,{\lambda})-2({n}{\lambda}\pm\xi)
\int_0^1\partial^2_\xi \psi(\xi\pm2{\lambda} t,{\lambda} )(1-t)\,dt. 
 \\
 &=\frac{1}{\lambda}\left({\lambda} \partial_{\lambda}+\xi\partial_\xi\right)\psi(\xi,{\lambda})-
 \frac{{n}{\lambda}\pm\xi}{2{\lambda}^2}\big(\psi(\xi\pm 2{\lambda},{\lambda})-\psi(\xi,{\lambda}) \big).
 \end{aligned}
$$
Since ${\lambda} \partial_{\lambda}+\xi\partial_\xi$ is  the derivative in the radial 
direction,
the operators ${M}_\pm$ depend only on the restriction to  the Heisenberg fan.   

The operators $M_\pm$ have the following relevant property. If $f$  is radial  and $(1+\mathcal A )f$ is
integrable   on ${{H_{n}}}$   then~(see \cite{BJR})
 \begin{equation}\label{Mpm}
 {\mathcal G}({{{\mathcal A}f)}}=M_+ ( {\mathcal G}{ f })
 \qquad{\text{ and}}\qquad  {\mathcal G}({{\bar{\mathcal A}f}})=-M_- ( {\mathcal G}{ f})
 .
\end{equation}
 One can verify that 
$$
{\langle {M_+({\mathcal G} f)\mu},{{\mathcal G} h} \rangle_{{\mathbb R}^2}}=-{\langle {({\mathcal G} f)\mu},{M_-({\mathcal G} h)} \rangle_{{\mathbb R}^2}}
\qquad \forall f, h\in {{\mathcal S}_{\text{\rm rad}}({{H_{n}}})}.
$$
Hence by Theorem~\ref{nostro}
 
 \begin{equation}\label{Mtrasposto}
\int_{\Sigma} (M_+{\varphi})\,\psi\, d\mu=
-\int_{\Sigma} {\varphi}\, (M_-\psi)\, d\mu
\qquad \forall {\varphi},\psi\in {\mathcal S}({\mathbb R}^2).
\end{equation}
 
According to~{~(\ref{{Mtrasposto}})},
when $U$ is in ${\mathcal S}_0'({\Sigma})$ we define the distribution $M_+U$ by
$$
{\langle {M_+U},{\psi} \rangle_{{\mathbb R}^2}}
=-{\langle {U},{M_-\psi} \rangle_{{\mathbb R}^2}}
\qquad\forall \psi\in {\mathcal S}({\mathbb R}^2)
$$
and similarly 
we define $M_-U$
with $M_+$ and $M_-$ interchanged.

Clearly if $\psi_{|_{\Sigma}}=0$ then $(M_-\psi)_{|_{\Sigma}}=0$, so that $M_+U$ is in ${\mathcal S}_0'({\Sigma})$.

Moreover it is easy to verify that {~(\ref{{Mpm}})} extends to distributions, i.e.
\begin{equation}\label{emmepiu}
 {\mathcal G}({{{\mathcal A}\Lambda)}}=M_+ ( {\mathcal G}{ \Lambda })
 \qquad{\text{ and}}\qquad  {\mathcal G}({{\bar{\mathcal A} \Lambda}})=- M_- ( {\mathcal G}{ \Lambda})
 \qquad \forall \Lambda \in {\mathcal S}'({{H_{n}}})
 .
\end{equation}
  Finally, given a distribution in ${\mathcal S}_0'({\Sigma})$
of the form $F\mu$, where $F$ is smooth and slowly growing
on ${\mathbb R}^2$,
we note that for all $\psi$ in ${\mathcal S}({\mathbb R}^2)$,
\begin{align*}
{\langle {M_+(F\mu)},{\psi} \rangle_{{\mathbb R}^2}}
&=
-{\langle {F\mu},{M_- \psi} \rangle_{{\mathbb R}^2}}
\\
&=-\int_{\Sigma} F\, M_- \psi\, d\mu
\\
&=\int_{\Sigma} M_+ F\, \psi\, d\mu
={\langle {(M_+F)\,\mu},{\psi} \rangle_{{\mathbb R}^2}},
\end{align*}
therefore
\begin{equation}
\label{emmepiuF}
M_+(F\mu)=(M_+ F)\,\mu\ .
\end{equation}
 
For later use, we prove the following estimate.

\begin{lemma}\label{potenzeM+}
Let  $a$ be a positive integer, then
for every $\psi$ in 
 ${\mathcal D}({\mathbb R}^2)$
  with support in the set 
$ {\left\{{(\xi,{\lambda})\in {\mathbb R}^2\,\,:\,\, |\xi|\leq {\rho}}\right\}}$
$$
\|M_\pm^a \psi \|_{L^1({\Sigma})}\leq C_a\, (1+{\rho})^{a+{n}+2}
\sum_{s,r=0}^{2a}\|\partial_{\lambda}^s\partial_\xi^r \psi \|_{L^\infty({\mathbb R}^2)}. 
$$
\end{lemma}
\begin{proof} 
It is enough to prove the statement for $M_+$, since   
$M_-\check\psi = -\left[M_+\psi \right]\!\!\check{\phantom f}$, where
$\check\psi(\xi,{\lambda})=\psi(\xi,-{\lambda})$. 
 Let 
 $W$ denote the operator acting on  a smooth function $\psi $ on ${\mathbb R}^2$ by
 $$
 W \psi(\xi,{\lambda}) = 2\int_0^1\partial^2_\xi \psi(\xi+2{\lambda} t,{\lambda} )(1-t)\,dt.
 $$
For every $ j\geq 0$ let $\eta_j$ be the function 
and let  $V_j$  be the operator defined by
$$
\eta_j(\xi,{\lambda})=\xi+(2j+{n}){\lambda}
\qquad V_j= \partial_{\lambda}-(2j+{n}) \partial_\xi.
$$  
With this notation $M_+=V_0-\eta_0W$. 
Moreover, as proved in~\cite[{Lemma 4.5}]{ADR2}, for every positive integer $a$,
\begin{equation}
M_+^a=V_0^a+\sum_{k=1}^a \eta_0\cdots \eta_{k-1}\, D_{k,a},
\end{equation}
where  $D_{k,a}$ is a polynomial  in $V_0,\ldots,V_{k},W$ of degree $a$ 
such that in each monomial the operator $W$ appears  $k$ times.

Let $\psi $ be in   ${\mathcal D}({\mathbb R}^2)$
 with support in the set 
$ {\left\{{(\xi,{\lambda})\in {\mathbb R}^2\,\,:\,\, |\xi|\leq {\rho}}\right\}}$.
Then it is easy to see that 
 ${\text{\rm supp\,}} D_{k,a} \psi\subseteq  {\left\{{(\xi,{\lambda})\in {\mathbb R}^2\,\,:\,\, |\xi|\leq c\,{\rho}}\right\}}$, 
 with $c$ depending on $a$. Therefore, using~{~(\ref{{integrabilita}})},
  \begin{align*}
\|  M_+^a \psi\|_{L^1({\Sigma})}&
 \leq
 \|V_0^a \psi\|_{L^1({\Sigma})}+\sum_{k=1}^a \| \eta_0\cdots \eta_{k-1}\, D_{k,a} \psi\|_{L^1({\Sigma})}
  \\
 &\leq
 C_a\, (1+{\rho})^{a+n+2}\left(
\sum_{r+s\leq a} \|\partial_{\lambda}^s\partial_\xi^r \psi \|_{L^\infty({\mathbb R}^2)}
+
\,\sum_{k=1}^a \| D_{k,a} \psi\|_{L^\infty({\mathbb R}^2)}
\right).
\end{align*}

We complete the proof by showing that 
$$
\|D_{k,a} \psi \|_{L^\infty({\mathbb R}^2)}\leq C_a\, 
\sum_{s+r\leq 2a}\|\partial_{\lambda}^s\partial_\xi^r \psi \|_{L^\infty({\mathbb R}^2)}
\qquad k=1,2,\ldots,a,
$$
by induction on $a$. Indeed, the case $a=1$ is trivial since
$$
\|W \psi \|_{L^\infty({\mathbb R}^2)}\leq 
2\int_0^1\|\partial^2_\xi \psi \|_{L^\infty({\mathbb R}^2)}(1-t)\,dt
\leq C\, 
\|\partial_\xi^2 \psi \|_{L^\infty({\mathbb R}^2)} .
$$
If $a>1$ then either $D_{k,a}=D_{k-1,a-1}\,W$ or $D_{k,a}=D_{k,a-1}\,V_j$, 
for some $j$ and $k\leq a-1$. The second case is trivial.
If $D_{k,a}=D_{k-1,a-1}\,W$, we note that  by induction on $s$ it is easy to verify that 
$$
\partial_{\lambda}^s\partial_\xi^r W \psi =
 2\sum_{k=0}^{s}\binom{s}{k}
 \int_0^1 \partial_{\lambda}^{s-k}\partial_\xi^{r+2+k}\, \psi(\xi+2{\lambda} t,{\lambda} )\,
 (2t)^{k}\,(1-t)\,dt.
$$
Therefore
  \begin{align*}
\|D_{k-1,a-1}\,W \psi \|_{L^\infty({\mathbb R}^2)}&\leq C_a\, 
\sum_{s+r\leq 2a-2}\|\partial_{\lambda}^s\partial_\xi^r W \psi \|_{L^\infty({\mathbb R}^2)}
\\ &\leq C_a\, 
\sum_{s+r\leq 2a-2}\sum_{k=0}^{s}\|\partial_{\lambda}^{s-k}\partial_\xi^{r+2+k} \psi \|_{L^\infty({\mathbb R}^2)}
\\ &\leq C_a\, 
\sum_{s+r\leq 2a}\|\partial_{\lambda}^s\partial_\xi^r   \psi \|_{L^\infty({\mathbb R}^2)}
.\qedhere
\end{align*}
\end{proof}

\bigskip

\section{Real Paley--Wiener results for the spherical transform}

Suppose that  $\Lambda$ is a radial tempered distribution on ${{H_{n}}}$.
Motivated by~\cite{Nils}, we define 
$R(\Lambda)$  in $[0,\infty]$ by
$$
R(\Lambda)=
{\rm max}{\left\{{|x|\, :\, x\in {\rm supp}\, \Lambda}\right\}}
$$
and we call $R(\Lambda)$ the radius of the support of the distribution~$\Lambda$.

The purpose of this section is to prove real Paley--Wiener Theorems for the spherical transform;
we start with a characterization of compactly supported radial
distributions and then we specialize these results to square integrable
radial functions and Schwartz radial functions. 
When a distribution $U$ on ${\mathbb R}^2$ is of the form $U=F_U\,\mu$ with $F_U$
a (smooth)  
 function on ${\mathbb R}^2$, by abuse of notation 
we shall also denote by $U$ the associated function $F_U$.

Our first characterization reads as follows.

\begin{theorem}\label{unoequiv}
Let $\Lambda$ be in ${{\mathcal S}'_{\text{\rm rad}}({{H_{n}}})}$. The following conditions are equivalent.
\begin{enumerate} 
\item
$R(\Lambda)$ is finite; 

\item  ${\mathcal G} \Lambda $ is the restriction to ${\Sigma}$ of
  a smooth function on ${\mathbb R}^2$ and for every $p$ in $[1,\infty]$ there exists $\beta>0$ such that
$$
\limsup_{j\to\infty}
\|(1+\xi)^{-\beta}\, M_+^j {\mathcal G} \Lambda\|_{L^p({\Sigma})}^{1/j}<\infty;
$$

\item 
 for every large $j$  the distribution
$  M_+^j {\mathcal G} \Lambda$ 
is the restriction to ${\Sigma}$ of
  a smooth function on ${\mathbb R}^2$
     and
       there exist $\beta>0$ and $p$ in $[1,\infty]$
such that
$$
\liminf_{j\to\infty} 
\|(1+\xi)^{-\beta}\, M_+^j {\mathcal G} \Lambda\|_{L^p({\Sigma})}^{1/j}
<\infty .
$$ 

\end{enumerate}
Moreover, if any of these conditions is satisfied, then   for every $p$ in $[1,\infty]$ there exists $\beta>0$ such that
\begin{equation}\label{Rspettr}
\lim_{j\to\infty} 
\|(1+\xi)^{-\beta}\, M_+^j{\mathcal G} \Lambda\|_{L^p({\Sigma})}^{1/j}= R(\Lambda)^2.
\end{equation}
\end{theorem}
  
  
Since $M_-\psi = -\left[M_+\check\psi \right]\check{\phantom f}$, where
$\check\psi(\xi,{\lambda})=\psi(\xi,-{\lambda})$,
 we have also a corresponding analogue
 with $M_-$ in place of $M_+$.  

 The proof of Theorem~\ref{unoequiv} is given after some preliminary results,
 the first of which is the following technical lemma.

 \begin{lemma}\label{lemmafj}
Let $R>0$ and $j$ be a positive integer.
 Suppose that $f$ is a smooth function on ${{H_{n}}}$
with compact support in the set $\{x\in {{H_{n}}}\,:\, |x|>R\}$
and let $f_j=\bar{\mathcal A}^{-j}\, f$.\\
Then for every $N$ in ${\mathbb N}$  
\begin{align*}
\|(1+\xi)^N\, {\mathcal G} f_j\|_{L^\infty({\Sigma})}
&\leq
 C_{N}\,j^{2N}\,R^{-2j}\, 
\max_{h\leq N}
\sum_{{k+\deg J=2h}}
\|{\bar{\mathcal A}}^{-k}\,{{D}}^{J} f\|_{L^1({{H_{n}}})}
\end{align*}
\end{lemma}
 \begin{proof} Note that since $f$ is supported away from the origin,
the function $f_j= \bar{\mathcal A}^{-j}\,f$ is again smooth and compactly supported.
Moreover,
\begin{align*}
\|(1+\xi)^{N}\,{\mathcal G}{f_j}\|_{L^{\infty}({\Sigma})}
&= 
\|{\mathcal G}\bigl({(I+L)^{N}f_j}\bigr)\|_{{L^{\infty}({\Sigma})}}
\\
&\leq 
\|(I+L)^Nf_j \|_{L^1({{H_{n}}})}
.
\end{align*}
Clearly $(I+L)^Nf_j=\sum\binom{M}{h}L^hf_j$ and
by the Leibniz rule
  
\begin{align*}
\|(I+L)^Nf_j \|_{L^1({{H_{n}}})}
&
\leq C_N \,\max_{h\leq N}\|L^hf_j\|_{L^1({{H_{n}}})}
\\
&\leq C_N\,
\max_{h\leq N}
\sum_{{\deg I+\deg J=2h}}
\|({{D}}^I {\bar{\mathcal A}}^{-j})\,({{D}}^{J} f) \|_{L^1({{H_{n}}})}
\\
&\leq C_N\,
\max_{h\leq N}
\sum_{{\deg I+\deg J=2h}} j^{|I|}
\|({\bar{\mathcal A}}^{-j-\deg I})\,({{D}}^{J} f) \|_{L^1({{H_{n}}})}
\\
&\leq C_N\,j^{2N}\,R^{-2j}\, 
\max_{h\leq N}
\sum_{{\deg I+\deg J=2h}}
\|{\bar{\mathcal A}}^{-\deg I}\,{{D}}^{J} f\|_{L^1({{H_{n}}})}.
\qedhere
\end{align*}
\end{proof}

Now we note that the spherical transform of radial compactly supported distributions
satisfies a pointwise estimate on the Heisenberg fan ${\Sigma}$.

\begin{proposition}
\label{puntuale}
Let $\Lambda$ be a radial compactly supported distribution of order $N$ on ${{H_{n}}}$. Then for every    $R>R(\Lambda)$
there exists a constant $C=C_R>0$ 
 such that for every $j$ in ${\mathbb N}$
  \begin{equation}\label{claim}
|M_+^j \widehat \Lambda(\xi,{\lambda})|\leq C\,  
 R^{2j}\, (1+\xi)^{N/2}
\qquad
\forall (\xi,{\lambda})\in{\Sigma} .
\end{equation}
\end{proposition}

\begin{proof} We have already proved in Proposition~\ref{distolo}
that ${\mathcal G}\Lambda=\widehat\Lambda\, \mu$
and that $\widehat\Lambda$ extends to an entire function, so 
$\widehat\Lambda$ is in  $C^\infty({\Sigma})$.
Moreover by equations~{~(\ref{{emmepiu}})}
and~\eqref{emmepiuF}
\begin{align*}
(M_+^j\widehat\Lambda)\, \mu
&=M_+^j(\widehat\Lambda\, \mu) 
=M_+^j{\mathcal G}\Lambda 
={\mathcal G}({{\mathcal A}^j \Lambda}) 
=\widehat{{\mathcal A}^j \Lambda}\,\mu,
\end{align*}
therefore $M_+^j\widehat\Lambda=\widehat{{\mathcal A}^j \Lambda}$.

Let
 $R>R(\Lambda)$ and choose $R_1$ such that $R>R_1>R(\Lambda)$. 
Suppose 
that $g$ is a radial test function on ${{H_{n}}}$ such that $g(x)=1$
when $x$ is in the support of $\Lambda$ and $g(x)=0$ if $|x|> R_1$.
Then for all $(\xi,{\lambda})$ in ${{\Sigma^*}}$,
\begin{align*}
|M_+^j \widehat \Lambda(\xi,{\lambda})|
&=| \widehat {{\mathcal A}^j \Lambda}(\xi,{\lambda})|
=| \widehat {g\,{\mathcal A}^j\Lambda}(\xi,{\lambda})|
\\
&=|{\langle {g\,{\mathcal A}^j\Lambda},{\check{\Phi}_{\xi,{\lambda}}} \rangle_{{H_{n}}}} |
\\
&=| {\langle {\Lambda},{g\,{\mathcal A}^j\Phi_{\xi,-{\lambda}}} \rangle_{{H_{n}}}} |
\\
&\leq C\, \sum_{\deg I\leq N} 
\| {{D}}^I (g\,{\mathcal A}^j\Phi_{\xi,-{\lambda}})\|_{L^\infty({{H_{n}}})}.
\end{align*}
We conclude 
by the Leibniz rule
and Lemma~\ref{stimasferiche} that
$$
|M_+^j \widehat \Lambda(\xi,{\lambda})|\leq C\, (1+j)^N\, R_1^{2j}\, (1+\xi)^{N/2}
\qquad
\forall (\xi,{\lambda})\in{{\Sigma^*}},
$$
which, since $R>R_1$ and by the smoothness of $\widehat\Lambda$, implies~{~(\ref{{claim}})}.
\end{proof}

\bigskip
Conversely, it is easy to deduce that a radial tempered distribution is compactly
supported
when a certain limit is finite.
\begin{proposition}
\label{due}
Let $\Lambda$ be in ${{\mathcal S}'_{\text{\rm rad}}({{H_{n}}})}$.   
Suppose that there exists  $J$ in ${\mathbb N}$ such that
  for every 
 $j\geq J$ the distribution $ M_+^j{\mathcal G} \Lambda$ is of the form $G_j\,\mu $, where $G_j$ is 
 a locally integrable function with respect to $\mu$. Then for every $N$ in ${\mathbb N}$
and every $p$ in $[1,\infty]$
$$
\liminf_{j\to\infty} 
\|\,(1+\xi)^{-N
}\, G_j\,\,\|_{L^p({\Sigma})}^{1/j}
\geq R(\Lambda)^2.
$$ 
\end{proposition}
 

 
\begin{proof}
Suppose that $R(\Lambda)>0$ and let $0<{\varepsilon}<R(\Lambda)/2$.
Then  
we may find a smooth function
$f$ with compact support in the set
$$
\{x\in {{H_{n}}}\, :\, R(\Lambda)-{\varepsilon}<|x|< R(\Lambda)+{\varepsilon} \}
$$
 such that
${\langle {\Lambda},{\check f} \rangle_{{H_{n}}}}\neq 0$.
As in the previous lemma, 
the function~$f$ is supported away from the origin and we let
$f_j= \bar{\mathcal A}^{-j}\,f$.
By{~(\ref{{tildeT}})}  and{~(\ref{{emmepiu}})}
\begin{align*}
|{\langle {\Lambda},{\check f} \rangle_{{H_{n}}}}|
&=|{\langle {\Lambda},{ {\mathcal A}^{j}\, {\mathcal A}^{-j}\,\check f} \rangle_{{H_{n}}}}|
=|{\langle {\Lambda},{ {\mathcal A}^{j}\,\check f_j} \rangle_{{H_{n}}}}|
\\
&=|{\langle {{\mathcal A}^{j}\,\Lambda},{\check{f_j}} \rangle_{{H_{n}}}}|
=|{\langle {{\mathcal G}{({\mathcal A}^{j}\,\Lambda)}},{{\mathcal G}{ {f_j}}} \rangle_{{\mathbb R}^2}}|
\\
&=|{\langle {M_+^{j} {\mathcal G}{\Lambda}},{{\mathcal G}{ {f_j}}} \rangle_{{\mathbb R}^2}}|
=|{\langle {G_j\, \mu},{{\mathcal G}{ {f_j}}} \rangle_{{\mathbb R}^2}}|
\\
&\leq 
\| (1+\xi)^{-N}\,G_j\|_{L^p({\Sigma})}\,
\|(1+\xi)^{N}\,{\mathcal G}{ f_j}\|_{L^{p'}({\Sigma})}.
\end{align*}
 In the case where $\| (1+\xi)^{-N}\,G_j\|_{L^p({\Sigma})}=\infty$ for all $j$,
 there is nothing to prove. Otherwise, since $|{\langle {\Lambda},{\check{f}} \rangle_{{H_{n}}}}|\neq0$,
$$
\liminf_{j\to\infty}
\| (1+\xi)^{-N}\,G_j \|_{L^p({\Sigma})}^{1/j}
\geq \liminf_{j\to\infty}
\left(\frac{|{\langle {\Lambda},{\check{f}} \rangle_{{H_{n}}}}|}{\|(1+\xi)^{N}\,{\mathcal G}{{f_j}}\|_{L^{p'}({\Sigma})}}
\right)^{1/j}
\!= \liminf_{j\to\infty}\|(1+\xi)^{N}\,{\mathcal G}{f_j}\|_{L^{p'}({\Sigma})}^{-1/j}.
$$
Since there exists $M$ in ${\mathbb N}$ such that
$\xi\mapsto (1+\xi)^{N-M}$ is in $L^{p'}({\Sigma})$, by Lemma~\ref{lemmafj}
we conclude that
\begin{align*}
\|(1+\xi)^{N}\,{\mathcal G}{f_j}\|_{L^{p'}({\Sigma})}
&\leq 
\|(1+\xi)^{N-M}\|_{L^{p'}({\Sigma})}\,
\|
(1+\xi)^{M}\,
{\mathcal G}{f_j}
\|_{{L^{\infty}({\Sigma})}}
\\
&
\leq C\,\,j^{2M}\,(R(\Lambda)-{\varepsilon})^{-2j}\ ,
\end{align*}
and the thesis follows easily.

When $R(\Lambda)=\infty$ we use the same arguments to show that \hbox{$\liminf\limits_{j\to\infty}
\| (1+\xi)^{-N}\,G_j \|_{L^p({\Sigma})}^{1/j}\geq R$} for every $R>0$.
\end{proof}

 

Putting together Proposition~\ref{puntuale} and Proposition~\ref{due},
we obtain the following criterion, by which we can measure
the size of the support of a radial compactly supported distribution.

\begin{corollary}
\label{prop68}
Let $\Lambda$ be a radial compactly supported distribution of order $N$. Then 
$$
\lim_{j\to\infty} 
\|(1+\xi)^{-N/2}\, M_+^j\widehat\Lambda\|_{L^\infty({\Sigma})}^{1/j}= R(\Lambda)^2.
$$
\end{corollary}

\begin{proof}
From the pointwise estimate~{~(\ref{{claim}})}, we deduce that for every $R>R(\Lambda)$
$$
\limsup_{j\to\infty}
\|(1+\xi)^{-N/2}\, M_+^j\widehat\Lambda\|_{L^\infty({\Sigma})}^{1/j}\leq R^2,
$$
therefore $\displaystyle{\limsup_{j\to\infty}
\|(1+\xi)^{-N/2}\, M_+^j\widehat\Lambda\|_{L^\infty({\Sigma})}^{1/j}\leq R(\Lambda)^2}$.
The thesis follows by Proposition~\ref{due}.
\end{proof}

 
 \begin{proof}[Proof of Theorem~\ref{unoequiv}]
If $\Lambda$ is compactly supported  and 
of order $N$ then  by Proposition~\ref{distolo} it coincides with 
  a smooth slowly growing function $G$ on ${\mathbb R}^2$. If 
$\beta>0$ is such that
$(1+\xi)^{N/2-\beta}$ is in ${L^p({\Sigma})}$, we have
\begin{equation}
\label{casop}
\|(1+\xi)^{-\beta}\, M_+^jG\|_{L^p({\Sigma})} 
\leq 
\|(1+\xi)^{N/2-\beta} \|_{L^p({\Sigma})}
\|(1+\xi)^{-N/2}\, M_+^jG\|_{L^\infty ({\Sigma})}.
\end{equation}
Hence by  Corollary~\ref{prop68} we have that    (1) $\Rightarrow$ (2).
The implication (2) $\Rightarrow$ (3) is trivial and the implication   
   (3) $\Rightarrow$ (1) is a consequence of Proposition~\ref{due}. 
   Finally \eqref{Rspettr} follows by \eqref{casop},  Proposition~\ref{due} and Corollary~\ref{prop68}.
 \end{proof}

   \subsection{Square-integrable functions}

\begin{theorem}\label{teoL2}\label{tre}
Suppose that for every $j\geq 0$ the function
$M_+^j\psi$ is in $L^2({\Sigma})$. Then the function~$f$ such that ${\mathcal G} f=\psi$
is in $L^2({{H_{n}}})$and
$$
\lim_{j\to\infty} \|\,M_+^j\psi\,\,\|_{L^{2}({\Sigma})}^{1/j}= R(f)^2.
$$ 
\end{theorem}

\begin{proof} By Proposition~\ref{due}
it is enough to check that 
$\limsup_{j\to\infty} \|M_+^j\psi\|_{L^{2}({\Sigma})}^{1/ j}\leq R(f)^2$,
and this is easily established by using the Plancherel formula. Indeed,
when $R(f)$ is finite,
\begin{align*}
\|M_+^j\psi \,\|_{L^{2}({\Sigma})}
&= \|{\mathcal A}^j\,f\|_{L^{2}({{H_{n}}})}
\leq   R(f)^{2j} \, \|f\|_{L^{2}({{H_{n}}})}.
\qedhere
\end{align*}
\end{proof}

\begin{corollary}\label{L2}
Let $R\geq 0$. Then  
${\mathcal G}$
is a bijection from the space $L^2_{\text{\rm rad},R}({{H_{n}}})$ 
of square integrable radial functions~$f$ such that $R(f)\leq R$
onto $\{\psi\in L^2({\Sigma})
\,\,: 
\lim_{j\to\infty} 
\|\,\,M_+^j \psi\,\,\|_{L^2({\Sigma})}^{1/j}
\leq R^2
\}$
.
\end{corollary}

\subsection{Schwartz functions}
 
The purpose of this subsection is to prove the following characterization.

\begin{theorem}\label{sch}
Let $f$ be in ${{\mathcal S}_{\text{\rm rad}}({{H_{n}}})}$. The following conditions are equivalent.
\begin{enumerate}

\item  $R(f)$ is finite;

\item  for every $h\geq 0$ and every $p$ in $[1,\infty]$,
 $\limsup_{j\to\infty} 
\|\,\xi^h\,M_+^j{\mathcal G} f\,\,\|_{L^p({\Sigma})}^{1/j}
$ is finite;

\item  there exists  $p$ in $[1,\infty]$ such that
 $\liminf_{j\to\infty} 
\|\,M_+^j{\mathcal G} f\,\,\|_{L^p({\Sigma})}^{1/j}
$ is finite.

\end{enumerate}
Moreover, if any of these conditions is satisfied, then for every $h\geq 0$ and every $p$ in $[1,\infty]$,
$$
\lim_{j\to\infty} 
\|\,(1+\xi)^h\,M_+^j{\mathcal G} f\,\,\|_{L^p({\Sigma})}^{1/j}
=R(f)^2.
$$
\end{theorem} 

Note that the implication $(2)\Rightarrow (3)$ is trivial,
and that $(3)\Rightarrow (1)$ follows from Proposition~\ref{due}.
In the next proposition we prove the implication $(1)\Rightarrow (2)$.
\begin{proposition} 
\label{uno}
Suppose that $f$ is a radial Schwartz function on ${{H_{n}}}$.
 Then
for every $h\geq 0$ and every $p$ in $[1,\infty]$
$$
\limsup_{j\to\infty} 
\|\,(1+\xi)^h\,M_+^j{\mathcal G} f\,\,\|_{L^p({\Sigma})}^{1/j}
\leq R(f)^2.
$$ 
\end{proposition}

\begin{proof}  
If $R(f)=\infty$ there is nothing to prove. If $R(f)=0$, then $f=0$
and the conclusion is again trivial.
We therefore suppose that $R(f)$ is positive. Note that
$$
\xi^h\,M_+^j{\mathcal G} f({\lambda},\xi)
=
{\mathcal G}\bigl({L^h\mathcal{A}^{j} f}\bigr) 
$$
and when $j\geq 2h$,  by the Leibniz rule
\begin{align*}
|L^h \mathcal{A}^{j}\, f|
&=\left| 
\sum_{{\deg I+\deg J=2h}}
c_{h,I,J}\,({{D}}^I \mathcal{A}^{j})\,({{D}}^{J} f)
\right|
\\ 
&\leq 
\sum_{{\deg I+\deg J=2h}}|c_{h,I,J}|
\, j^{|I|} \,|\mathcal{A}^{j-\deg I}|\,|{{D}}^{J} f|.
\end{align*}
 Therefore 
\begin{align*}
\|\, \xi^h\,M_+^j{\mathcal G} f\,\,\|_{L^\infty({\Sigma})}
&\leq \|L^h\mathcal{A}^{j}\, f\|_{L^1({{H_{n}}})}
\\
&
\leq C_h\, j^{2h}\,\sum_{q\leq 2h}
\max_{\deg J=2h-q} \|\mathcal{A}^{j-q}\,{{D}}^{J} f\|_{L^1({{H_{n}}})}
\\
&
\leq C_h\, j^{2h}\,\sum_{q\leq 2h}\, R(f)^{2j-2q}\, 
\max_{\deg J=2h-q}\|{{D}}^{J} f\|_{L^1({{H_{n}}})}
\\
&
= C_{f,h}\,j^{2h}\, R(f)^{2j}.
\end{align*}
 
We note that
for a sufficiently big integer $M$ the function $({\lambda},\xi)\mapsto
(1+\xi)^{-M}$ is in $L^{p}({\Sigma})$, so that
\begin{align*}
\|\,(1+\xi)^h\,M_+^j{\mathcal G} f\,\|_{L^{p}({\Sigma})}
&\leq C\,\,
\|(1+\xi)^{M+h}\,M_+^j{\mathcal G} f\,\|_{L^{\infty}({\Sigma})}
\\
&\leq C_{f,M,h}\, \left( 1 +j^{2M+2h}\right)\,R(f)^{2j},
\end{align*}
and  
taking the $j$-th root, the desired inequality follows.
\end{proof}

\begin{corollary}
Suppose that $f$ is a radial Schwartz function on ${{H_{n}}}$ and let $1\leq p\leq \infty$.  Then
for every $h$ in ${\mathbb N}$
$$
\lim_{j\to\infty} \|\,(1+\xi)^h\,M_+^j{\mathcal G} f\,\,\|_{L^{p}({\Sigma})}^{1/j}= R(f)^2.
$$ 
\end{corollary}

\begin{proof} Since $\|(1+\xi)^h\,M_+^j{\mathcal G} f\|_{L^{p}({\Sigma})}
\geq \|\,M_+^j{\mathcal G} f\,\|_{L^{p}({\Sigma})}$, by Proposition~\ref{due} we obtain
$$
\liminf_{j\to\infty} \|\,(1+\xi)^h\,M_+^j{\mathcal G} f\,\,\|_{L^{p}({\Sigma})}^{1/j}
\geq \liminf_{j\to\infty}\|\,M_+^j{\mathcal G} f\,\,\|_{L^{p}({\Sigma})}^{1/j}\geq R(f)^2.
$$
The thesis follows from Proposition~\ref{uno}.
\end{proof}

\section{Paley--Wiener theorems for the inverse spherical transform }

In this section we describe the inverse spherical transform
of compactly supported distributions in ${\mathcal S}_0'({\Sigma})$.

Given a compactly supported distribution in ${\mathcal S}'({\mathbb R}^2)$,
we define the function $f_U$ on the Heisenberg group by
$$
f_U(z,t)={\langle {U},{\Phi_{(\cdot)}(z,t)} \rangle_{{\mathbb R}^2}}\qquad \forall  (z,t)\in{{H_{n}}}.
$$ 
An easy consequence of 
 Lemma~\ref{olosferiche} is the following.

\begin{lemma}\label{oloinv}
Let $U$ be a compactly supported distribution in ${\mathcal S}'({\mathbb R}^2)$.
 Then the function
$$
(x,y,t)\mapsto f_U(x+iy,t)={\langle {U},{\Phi_{(\cdot)}(x+iy,t)} \rangle_{{\mathbb R}^2}}
$$ 
extends to a holomorphic function   on ${\mathbb C}^{2{n}+1}$.
\end{lemma}
 

If $U$ is in ${\mathcal S}'({\mathbb R}^2)$, define
 $$
{\rho} (U)=
{\rm max}{\left\{{|\xi|\, :\, (\xi,{\lambda})\in {\rm supp}\, U}\right\}},
$$
 so that
  a distribution $U$ in ${\mathcal S}_0'({\Sigma})$  is compactly supported 
 if and only if $\rho(U)$ is finite.

In the next proposition we prove that if $U$ is 
a compactly supported distribution in ${\mathcal S}_0'({\Sigma})$ then the function $f_U$
is a slowly growing function on ${{H_{n}}}$ and it coincides with 
the inverse spherical transform of $U$.   

\begin{proposition}\label{invtemp}
Let $U$ be a compactly supported distribution in ${\mathcal S}_0'({\Sigma})$ and let, as before,
$$
f_U(z,t)={\langle {U},{\Phi_{(\cdot)}(z,t)} \rangle_{{\mathbb R}^2}}\qquad \forall  (z,t)\in{{H_{n}}}.
$$ 
Then $  U={\mathcal G}(f_U\,m)$ and $f_U$ is a  slowly growing function on ${{H_{n}}}$ together  with all its derivatives. Moreover, for every $\rho>{\rho}(U)$
there exist $C=C_\rho$ and $M$ such that for all $j\geq 0$ 
\begin{equation}
\label{e:Lj}
 \left|
     L^j f_U(z,t)
     \right|
     \leq C \,(1+j)^k\, \rho^j\, \big(1+|(z,t)|\big)^M
     \qquad \forall (z,t)\in {{H_{n}}},
\end{equation}
where $k$ is the order of $U$.
\end{proposition}

\begin{remark}
Observe that if $U$ is 
a  distribution in ${\mathcal S}'({\mathbb R}^2)$ with compact support in ${\Sigma}$,
then the function $f_U$
may not be slowly growing.
Indeed let $U=\partial_\xi\delta_{({n},1)}$ where $\delta_{({n},1)}$ is the Dirac measure
at the point $({n},1)$ in ${\Sigma}$. Then, reasoning as in Lemma~\ref{olosferiche},
when $(z,t)$ is in ${{H_{n}}}$
$$
f_U(z,t)=-\partial_\xi\Phi_{\xi,{\lambda}}{\,}_{|_{({n},1)}}(z,t)
=\frac{e^{it}\, e^{-|z|^2/4}}{2}\sum_{k=1}^\infty \frac{(|z|^2/2)^k}{k\, ({n})_k}.
$$
Since $k<k+ {n}$ and $({n})_k\leq ({n}+k-1)!$ we obtain
when $|z|$ is large
$$
|f_U(z,t)|>\frac{e^{-|z|^2/4}}{2}\sum_{k=1}^\infty \frac{(|z|^2/2)^k}{({n}+k)!}
\sim \frac{e^{|z|^2/4}}{2(|z|^2/2)^{n}}.
$$
This is due, much as in Subsection~\ref{sec:olo}, to the fact that
the holomorphic extension of spherical functions does not satisfy good estimates
away from the Heisenberg fan.
The main point in the proof of Proposition~\ref{invtemp} is that,
according to formula~{~(\ref{{numero}})},
if $U$ is in  ${\mathcal S}_0'({\Sigma})$ one is allowed to choose a different extension.
 \end{remark}

\begin{proof}  By Theorem~\ref{nostrodist} there exists $\Lambda$ in ${{\mathcal S}'_{\text{\rm rad}}({{H_{n}}})}$ such that ${\mathcal G}  \Lambda = U $. 
Let $g$ be in ${\mathcal D}({{H_{n}}})$,  then 
$$
\begin{aligned}
\langle f_U,g\rangle_{H_n}&=
\int_{H_n}f_U(z,t)\,g(z,t)\,dz\,dt 
\\
&=\int_{H_n}{\langle {U},{\Phi_{(\cdot)}(z,t)} \rangle_{{\mathbb R}^2}}\,\,g(z,t)\,dz\,dt\\
&=\Big\langle U,\int_{H_n}\Phi_{(\cdot)}(z,t)\,g(z,t)\,dz\,dt\Big\rangle_{{\mathbb R}^2}\\
&={\langle {U},{{\mathcal G} \check g} \rangle_{{\mathbb R}^2}}\\
&=\langle \Lambda,g\rangle_{H_n}\ .
\end{aligned}
$$
Hence the distribution  $\Lambda$ coincides with the function $f_U$, which is smooth.

We first prove the estimate~{~(\ref{{e:Lj}})}.
Fix $(z,t)$ in ${{H_{n}}}$. 
Let $k$ be the order of $U$, let  $\rho>\rho(U)$ and   
denote by  $B_\rho$ the ball of radius $\rho$ in ${\mathbb R}^2$. Then for every $j\geq 0$
 \begin{align}
   \nonumber  \left|
     L^j f_U(z,t)
     \right|&=
     \left|{\langle { 
    U },{ {L}^j \Phi_{(\cdot)}(z,t)} \rangle_{{\mathbb R}^2}}
     \right|
     =
      \left|{\langle {\,U},{ \xi^j \Phi_{(\cdot)}(z,t)} \rangle_{{\mathbb R}^2}}
     \right|
     \\ \label{numero} &
           \leq
      C \inf{\left\{{ 
     \|
     \xi^j \varphi^{z,t}
      \|_{C^k(B_\rho)}\,:\,\varphi^{z,t}\in C^k({\mathbb R}^2),\quad
 {\varphi^{z,t}}_{|_{{\Sigma}\cap B_\rho}}=\Phi_{(\cdot)}(z,t)
}\right\}}
     \\ &
     \nonumber\leq
     C_{\rho}\, (1+j)^k\, \rho^j\inf{\left\{{ 
      \|\varphi^{z,t}
      \|_{C^k(B_\rho)}\,:\,\varphi^{z,t}\in C^k({\mathbb R}^2),\quad
 {\varphi^{z,t}}_{|_{{\Sigma}\cap B_\rho}}=\Phi_{(\cdot)}(z,t)
}\right\}}
\end{align}

In order to obtain the desired estimate we shall choose a suitable extension  $\varphi^{z,t}$ of $\Phi_{(\cdot)}(z,t)$.

Let $\psi$ be a smooth function on ${\mathbb R}^2$ with compact support such that $\psi_{|{B_\rho} }=1$. By Theorem~\ref{nostro} there exists $u$ in ${{\mathcal S}_{\text{\rm rad}}({{H_{n}}})}$ such that ${\mathcal G} u=\psi_{|{\Sigma}}$. If  
$\nu_{(z,t)}$ denotes the measure defined by 
$$
\int_{{H_{n}}} g(w,s)\, d\nu_{(z,t)}(w,s)=
\int_{U(n)} g(kz,t)\, dk \qquad\forall g\in C_c({{H_{n}}}),
$$
then for every $(\xi,{\lambda})$ in $B_\rho\cap {\Sigma}$
$$
\Phi_{\xi,{\lambda}}(z,t)
={\mathcal G}{\check\nu_{(z,t)}}(\xi,{\lambda})=
{\mathcal G}{\check\nu_{(z,t)}}(\xi,{\lambda})\, \psi(\xi,{\lambda})=
{\mathcal G}\left(\check\nu_{(z,t)}\ast u
\right)(\xi,{\lambda})
.
$$
Since $\check\nu_{(z,t)}\ast u$ belongs to $ {{\mathcal S}_{\text{\rm rad}}({{H_{n}}})}$, then by Theorem~\ref{nostro} there exist
$\varphi^{z,t}$ in $ {\mathcal S}({\mathbb R}^2)$ and  $M\geq 0$ such that 
$$
\varphi^{z,t}(\xi,{\lambda})={\mathcal G}\left(\check\nu_{(z,t)}\ast u
\right)(\xi,{\lambda})
\qquad \forall(\xi,{\lambda})\in   {\Sigma}
$$
and 
$$
 \|\varphi^{z,t}
      \|_{C^k(B_\rho)}\leq C \, \|\check\nu_{(z,t)}\ast u\|_{(M)}.
$$
Moreover
$$
\varphi^{z,t}(\xi,{\lambda})= \Phi_{(\xi,{\lambda})}(z,t)
\qquad \forall(\xi,{\lambda})\in  B_\rho\cap{\Sigma}.
$$
If 
$ \tau_{(w,s)}u(w',s')=u\big((w,s)^{-1}(w',s')\big)$ denotes the left translation, then
 \begin{align*}
\|\varphi^{z,t}
      \|_{C^k(B_\rho)}
      &\leq {  C \, \|\nu_{(z,t)}\ast u\|_{(M)}
}
\\
&=C \, 
{ \left\|\int_{{H_{n}}} \tau_{(w,s)}u\, d\nu_{(z,t)}(w,s)\right\|_{(M)}}
 \\
&\leq C \, 
  \int_{{H_{n}}} \|\tau_{(w,s)}u\|_{(M)}\, d\nu_{(z,t)}(w,s)
   \\
&\leq C \, 
{   \int_{{H_{n}}} (1+|w|^4+s^2)^{M/4}\, d\nu_{(z,t)}(w,s)
}
  \\
  &=    C \, \big(1+|(z,t)|\big)^{M}.
\end{align*}
Therefore there exists $M$ such that for all $j\geq 0$ 
$$
 \left|
     L^j f_U(z,t)
     \right|
     \leq C \,(1+j)^k\, \rho^j\, \big(1+|(z,t)|\big)^{M}
\qquad\forall (z,t)\in {{H_{n}}}.
$$

The proof above can be adapted to prove that for every differential operator ${{D}}^I  $ 
of the form~{~(\ref{{monomi}})}
there exists $M>0$ such that   
$$
|{{D}}^I f_U(z,t)|\leq C\, (1+|(z,t)|)^M
\qquad \forall (z,t)\in {{H_{n}}}.
$$
Indeed, note that
$$
 {{D}}^I f_U(z,t)={\langle {U},{{{D}}^I\Phi_{(\cdot)}(z,t)} \rangle_{{\mathbb R}^2}}\ ,
$$
therefore
\begin{align*}
| {{D}}^I f_U(z,t)|
&\leq C \inf{\left\{{\|\varphi^{z,t,I}\|_{C^k(B_\rho)}\,:\,\varphi^{z,t,I}\in C^k({\mathbb R}^2)\quad
{\varphi^{z,t,I}}_{|_{{\Sigma}\cap B_\rho}}= {{D}}^I \Phi_{(\cdot)}(z,t)}\right\}}.
\end{align*} 
Fix $(z,t)$ in ${{H_{n}}}$ and consider the distribution
 ${{D}}^I_{(z,t)}\nu_{(z,t)}$ defined by the rule
$$
{\langle {{{D}}^I_{(z,t)}\nu_{(z,t)}},{\varphi} \rangle_{{H_{n}}}}
={{D}}^I\left(\int_K \varphi(kz,t)\, dk \right)
$$
Then ${{D}}^I_{(z,t)}\nu_{(z,t)}$ is a radial distribution
 supported in the orbit of $(z,t)$,
hence it has compact support. So, for $\psi$ and $u$ as above,
${{D}}^I_{(z,t)}\check\nu_{(z,t)}*u$ is in ${{\mathcal S}_{\text{\rm rad}}({{H_{n}}})}$ and
by~\cite[Proposition~3.2]{ADR} there exists
$\varphi^{z,t,I}$ in $C^k({\mathbb R}^2)$ and $M$ such that
$$
\varphi^{z,t,I}_{|_{\Sigma}}={\mathcal G}({{D}}^I_{(z,t)}\check\nu_{(z,t)}*u)
\qquad\qquad
\|\varphi^{z,t,I}\|_{C^k(B_\rho)}\leq C\,\|{{D}}^I_{(z,t)}\check\nu_{(z,t)}*u\|_{(M)}
$$
Since ${\mathcal G} u_{|_{{\Sigma}\cap B_\rho}}=\psi_{|_{{\Sigma}\cap B_\rho}}=1$,
$$
\varphi^{z,t,I}(\xi,{\lambda})={\mathcal G}({{D}}^I_{(z,t)}\check\nu_{(z,t)}*u)(\xi,{\lambda})
={\mathcal G}({{D}}^I_{(z,t)}\check\nu_{(z,t)})(\xi,{\lambda})
\qquad \forall (\xi,{\lambda})\in {\Sigma}\cap B_\rho ,
$$
and by Proposition~\ref{distolo}
\begin{align*}
{\mathcal G}({{D}}^I_{(z,t)}\check\nu_{(z,t)})(\xi,{\lambda})
&=
{\langle {{{D}}^I_{(z,t)}\check\nu_{(z,t)}},{\check\Phi_{\xi,{\lambda}}} \rangle_{{\mathbb R}^2}}
={{D}}^I\left((w,s)\mapsto\int_K \Phi_{\xi,{\lambda}}(kw,s)\, dk \right)(z,t)
\\
&={{D}}^I\Phi_{\xi,{\lambda}}(z,t).
\end{align*} 
Finally, reasoning as before,
 \begin{align*}
\| \varphi^{z,t,I}\|_{C^k( B_\rho)}
&\leq 
C\,\|{{D}}^I_{(z,t)}\check\nu_{(z,t)}*u\|_{(M)}
\leq    C \, \big(1+|(z,t)|\big)^{M}
\qquad
\forall (z,t)\in {{H_{n}}}.
\qedhere
\end{align*}
\end{proof}

Our characterization of the inverse spherical transform of compactly 
supported distributions is the following.

 \begin{theorem}\label{maininv}
Let $U$ be in ${\mathcal S}'_0(\Sigma)$. The following conditions are equivalent.
\begin{enumerate}
\item
${\rho}(U)$ is finite; 

\item  
${\mathcal G}^{-1}U$ coincides with a smooth slowly growing function function
on ${{H_{n}}}$ and for every $p$ in $[1,\infty]$ there exists $\beta>0$ such that
$$
\limsup_{j\to\infty}
\|\,(1+{\mathcal A})^{-\beta}\,L^j{\mathcal G}^{-1}U\,\,\|_{L^p({{H_{n}}})}^{1/j}
 <\infty;
$$
 
\item 
 for every large $j$  the distribution
$ L^j {\mathcal G}^{-1}U$ coincides with a measurable function
on ${{H_{n}}}$   and there exist $\beta>0$ and $p$ in $[1,\infty]$
such that
$$
\liminf_{j\to\infty} 
\|\,(1+{\mathcal A})^{-\beta}\,L^j{\mathcal G}^{-1}U\,\,\|_{L^p({{H_{n}}})}^{1/j}
<\infty .
$$ 

\end{enumerate}
Moreover, if any of these conditions is satisfied, then ${\mathcal G}^{-1}U$
 is a smooth slowly growing function on ${{H_{n}}}$
and for every $p$ in $[1,\infty]$ there exists $\beta>0$ such that
\begin{equation}\label{rhospettr}
\lim_{j\to\infty} \|\,(1+{\mathcal A})^{-\beta}\,L^j{\mathcal G}^{-1}U\,\,\|_{L^p({{H_{n}}})}^{1/j}= {\rho}(U).
\end{equation}
\end{theorem}

As in the previous section, we split the proof of our characterization into several parts.

 \begin{proposition}\label{inf}
 Let $U$ be in ${\mathcal S}'_0({\Sigma})$.   
Suppose that there exists  $J$ in ${\mathbb N}$ such that
  for every 
 $j\geq J$ the distribution $ L^j {\mathcal G}^{-1} U$ is of the form $f_j\,{m} $, where $f_j$ is 
 a locally integrable function on~${{H_{n}}}$. Then for every $N$ in ${\mathbb N}$
and every $p$ in $[1,\infty]$
  $$
 \liminf_{j\to\infty} 
\|\,(1+{\mathcal A})^{-N}\,f_j\,\,\|_{L^p({{H_{n}}})}^{1/j}
\geq {\rho}(U).
 $$
 \end{proposition}
 
 \begin{proof} For the 
 reader's convenience we write the proof although it  
  follows the lines of that of Proposition~\ref{due}.
We may suppose that  ${\rho}(U)$ is positive, because
in the case where ${\rho}(U)=0$, there is nothing to prove.
 
   Let  $\|\,(1+{\mathcal A})^{-N}\, f_j\,\,\|_{L^p({{H_{n}}})}<\infty$.
 Suppose that $0<{\varepsilon} < {\rho}(U)/2$ and let  $\psi$ be smooth function on ${\mathbb R}^2$
  with compact support in the set 
$$
\{(\xi,{\lambda})\in {\mathbb R}^2\, :\, {\rho}(U)-{\varepsilon}<\xi< {\rho}(U)+{\varepsilon} \}
$$
 such that
${\langle {U},{\psi} \rangle_{{\mathbb R}^2}}\neq 0$. 
For every nonnegative integer $j$, 
define a smooth function on ${\mathbb R}^2$
  with compact support by 
  $\psi_j(\xi,{\lambda})=\xi^{-j}\, \psi(\xi,{\lambda})$ for every $(\xi,{\lambda})$ in ${\mathbb R}^2$.
  Then for $1\leq p\leq \infty$,
  \begin{align*}
|{\langle {U},{\psi} \rangle_{{\mathbb R}^2}}|
&=
|{\langle {\xi^j\,U},{ \psi_j} \rangle_{{\mathbb R}^2}}|
=
|{\langle { {L^j {\mathcal G}^{-1}U}},{({\mathcal G}{^{-1}} {\psi_j}_{|_\Sigma})\check{\phantom a}} \rangle_{{H_{n}}}}|
\\
&\leq \|(1+{\mathcal A})^{-N}f_j\|_{L^p({{H_{n}}})}\, \|(1+{\mathcal A})^{N}{\mathcal G}{^{-1}} {\psi_j}_{|_\Sigma}
\|_{L^{p'}({{H_{n}}})}
\end{align*}

Let $a$ be a positive integer such that 
$\|(1+{\mathcal A})^{N-a} 
\|_{L^{p'}({{H_{n}}})}<\infty$, then by Lemma~\ref{potenzeM+}

  \begin{align*}
  \|(1+{\mathcal A})^{N}\,{\mathcal G}{^{-1}} {\psi_j}_{|_\Sigma}
\|_{L^{p'}({{H_{n}}})}
 & \leq
\|(1+{\mathcal A})^{N-a} 
\|_{L^{p'}({{H_{n}}})} 
\| (1+M_+ )^a\,  {\psi_j} 
\|_{L^1({\Sigma})}
\\ 
& \leq C_a
\,\big({\rho}(U)+{\varepsilon}\big)^{a}\, 
\sum_{s,r=1}^{2a}\|\partial_{\lambda}^s\partial_\xi^r \psi_j \|_{L^\infty({\mathbb R}^2)} 
\\ & \leq 
\,C_a
\,\big({\rho}(U)+{\varepsilon}\big)^{a}\, \, j^{2a}\,\big({\rho}(U)-{\varepsilon}\big)^{-j}.
\end{align*}
  Therefore
  $$
  \|(1+{\mathcal A})^{-N}\,f_j\|_{L^p({{H_{n}}})}
\geq 
 {|{\langle {U},{\psi} \rangle_{{\mathbb R}^2}}|}  \, C_{a,{\varepsilon}}\,j^{-2a}\,\big({\rho}(U)-{\varepsilon}\big)^{j}
   $$
   and the thesis follows.
  Similar considerations can be used in the case where $\rho(U)=\infty$. 
 \end{proof}
 
 
  
 
 \begin{proposition}\label{limiteinv}
Let $U$ be in  ${\mathcal S}'_0({\Sigma})$ with ${\rho}(U)<\infty$. 
  Then    ${\mathcal G}{^{-1}} U$
  coincides with a smooth slowly growing  function 
  $f$ on ${{H_{n}}}$ and for every $p$ in $[1,\infty]$ there exists $h>0$ such that
$$
\limsup_{j\to\infty}
\|\,(1+{\mathcal A})^{-h}\,L^j f\,\,\|_{L^p({{H_{n}}})}^{1/j}
 \leq {\rho}(U) .
$$
\end{proposition}

  \begin{proof}
  Since ${\rho}(U)<\infty$, the distribution~$U$ is compactly supported
  and therefore  ${\mathcal G}{^{-1}} U$
  coincides with the smooth function $f_U$ on ${{H_{n}}}$ 
  by Lemma~\ref{oloinv} and $f_U$ is slowly growing by Proposition~\ref{invtemp}. 
 Moreover, the estimate {~(\ref{{e:Lj}})} holds: if $\rho>{\rho}(U)$ 
and $k$ is the degree of $U$, 
there exists $M$ such that for all $j\geq 0$ 
$$
 \left|
     L^j f(z,t)
     \right|
     \leq C \,(1+j)^k\, \rho^j\, \big(1+|(z,t)|\big)^{M}.
$$
Let $p$ in $[1,\infty]$ be fixed and choose $h$ such that $(1+{\mathcal A})^{-h+M/2}$ is in $L^p({{H_{n}}})$.
Then for every $\rho>\rho(U)$ 
\begin{align*}
\|\,(1+{\mathcal A})^{-h}\,L^j f\,\,\|_{L^p({{H_{n}}})}
&\leq \|\,(1+{\mathcal A})^{-h+M/2}\,\|_{L^p({{H_{n}}})} \, \|\,(1+{\mathcal A})^{-M/2}\,L^j f\,\,\|_{L^\infty({{H_{n}}})}
\\
&\leq C\,(1+j)^k\, \rho^j
\end{align*}
so that 
$
\limsup_{j\to\infty}
\|\,(1+{\mathcal A})^{-h}\,L^j f\,\,\|_{L^p({{H_{n}}})}^{1/j}
 \leq \rho 
$, for every  $\rho>{\rho}(U)$.
\end{proof}
 
\subsection{Square-integrable functions and Schwartz functions }

Reasoning as in the proof of Theorem~\ref{teoL2} and Corollary~\ref{L2} it easy to prove the following characterization for square-integrable functions.

\begin{theorem}\label{L2inv}
Let $\rho\geq 0$. Then  
${\mathcal G}$
is a bijection from the space 
 $$\{f\in{L^2_{\text{\rm rad}}({{H_{n}}})} 
\,\,: 
\lim_{j\to\infty} 
\|\,\,L^j f\,\,\|_{L^2({{H_{n}}})}^{1/j}
\leq \rho
\}$$
onto the space $$\{F\in L^2({\Sigma})
\,\,: \rho(F)
\leq \rho
\}\ .$$ 
\end{theorem}

In the case of Schwartz functions, we obtain the following results. 
For $F$ in ${\mathcal S}({\Sigma})$ we denote ${\rho}(F)={\rho}(F\mu)$, so that
$$
{\rho}(F)=\sup\{\xi\, :\,F(\xi,{\lambda})\not= 0\,\quad\text{and}\quad (\xi,{\lambda})\in {\Sigma}\}.
$$
 
 \begin{proposition}\label{liminvsch}
  Let $1\leq p\leq \infty$ and let $F$ be in ${\mathcal S}({\Sigma})$. 
  Then for every $h\geq 0$,
 $$
 \lim_{j\to\infty} 
\|\, \left(1+{\mathcal A} \right)^{h} L^j {\mathcal G}{^{-1}} F\,\,\|_{L^p({{H_{n}}})}^{1/j}
= {\rho}(F).
 $$
\end{proposition}

 \begin{proof} Suppose $0<{\rho}(F)<\infty$.
 If $\gamma>0$ is big enough so that 
 $ \left(1+{\mathcal A} \right)^{-\gamma}$ is in $L^p({{H_{n}}})$ by Lemma~\ref{potenzeM+}  we obtain
   \begin{align*}
  \|\,\left(1+{\mathcal A} \right)^{h} L^j {\mathcal G}{^{-1}} F\,\,\|_{L^p({{H_{n}}})}
  & \leq 
   \|\, \left(1+{\mathcal A} \right)^{-\gamma}\,\|_{L^p({{H_{n}}})}
   \|\, \left(1+{\mathcal A} \right)^{h+ \gamma}\,L^j {\mathcal G}{^{-1}} F\,\|_{L^\infty({{H_{n}}})}
    \\
   & \leq  C\,
    \|\,  \left(1+M_+ \right)^{h+ \gamma}\big( \xi^j\, F\big)\|_{L^1({\Sigma})}
\\
   & \leq
      C\, j^{2h+2 \gamma}\left({\rho}(F)\right)^j.
\end{align*}
Hence 
$$
 \limsup_{j\to\infty} 
\|\, \left(1+{\mathcal A} \right)^{h}\, L^j {\mathcal G}{^{-1}} F\,\,\|_{L^p({{H_{n}}})}^{1/j}
\leq {\rho}(F).
 $$
 and the thesis follows from Propostition~\ref{inf}.
  The cases ${\rho}(F)=0,\infty$ are trivial. 
 \end{proof}
  
  
  
\begin{theorem}\label{schinv}
Let $F$ be in ${\mathcal S}({\Sigma})$. The following conditions are equivalent.
\begin{enumerate}

\item  $\rho(F)$ is finite;

\item  for every $h\geq 0$ and every $p$ in $[1,\infty]$,
 $\limsup_{j\to\infty} 
\|\,{\mathcal A}^h\,L^j{\mathcal G}{^{-1}} F\,\,\|_{L^p({{H_{n}}})}^{1/j}
$ is finite;

\item  there exists  $p$ in $[1,\infty]$ such that
 $\liminf_{j\to\infty} 
\|\,L^j{\mathcal G}{^{-1}} F\,\,\|_{L^p({{H_{n}}})}^{1/j}
$ is finite.

\end{enumerate}
Moreover, if any of these conditions is satisfied, then for every $h\geq 0$ and every $p$ in $[1,\infty]$,
$$
\lim_{j\to\infty} 
\|\, \left(1+{\mathcal A} \right)^{h}\, L^j {\mathcal G}{^{-1}} F\,\,\|_{L^p({{H_{n}}})}^{1/j}
= {\rho}(F).
$$
\end{theorem}

\begin{proof} The implication $(1)\Rightarrow (2)$ follows by Propostion~\ref{liminvsch}. The implication
  $(2)\Rightarrow (3)$ is trivial. The implication $(3)\Rightarrow (1)$ follows by Propostion~\ref{inf}.   
  \end{proof}

\begin{thebibliography}{9999}

\bibitem{Nils}
	{{\textrm{{N.~B. Andersen, M.~de Jeu,}}}}
	{{\textrm {{Real Paley--Wiener theorems and local spectral radius formulas}}}},
	{{\textit{\frenchspacing{Trans.\ Amer.\ Math.\ Soc.}}}}
    {{\textbf{{362}}}}
     (2010), 3613--3640.

\bibitem{PJapAc}
	{{\textrm{{S.~Ando,}}}}
	{{\textrm {{Paley--Wiener type theorem for the Heisenberg groups}}}},
	{{\textit{\frenchspacing{Proc.\ Japan Acad.}}}}
	 {{\textbf{{52}}}}
	   (1976),  331--333.

\bibitem{PAMS}
	{{\textrm{{D.~Arnal, J.~Ludwig,}}}}
	{{\textrm {{Q.{U}.{P}.\ and {P}aley-{W}iener properties of unimodular,
              especially nilpotent, {L}ie groups}}}},
   {{\textit{\frenchspacing{Proc.\ Amer.\ Math.\ Soc.}}}}
   {{\textbf{{125}}}}
      (1997),  1071--1080.

\bibitem{ADR}
	{{\textrm{{F.~Astengo, B.~Di Blasio, F.~Ricci,}}}}
	{{\textrm {{Gelfand transforms of polyradial Schwartz functions 
					on the Heisenberg group}}}},
	{{\textit{\frenchspacing{J.\ Funct.\ Anal.}}}}
    {{\textbf{{251}}}}
     (2007), 772--791.

\bibitem{ADR1}
	{{\textrm{{F.~Astengo, B.~Di Blasio, F.~Ricci,}}}}
	{{\textrm {{Gelfand pairs  on the Heisenberg group of and Schwartz functions}}}},
	{{\textit{\frenchspacing{J.\ Funct.\ Anal.}}}}
    {{\textbf{{256}}}}
     (2009), 1565--1587.

\bibitem{ADR2}
	{{\textrm{{F.~Astengo, B.~Di Blasio, F.~Ricci,}}}}
	{{\textrm {{Fourier transform of Schwartz functions on the Heisenberg group}}}},
	to appear in {{\textit{\frenchspacing{Studia Math.}}}} 

\bibitem{Bang}
	{{\textrm{{H.~H.~Bang,}}}}
	{{\textrm {{A property of infinitely differentiable functions}}}},
	{{\textit{\frenchspacing{Proc.\ Amer.\ Math.\ Soc.}}}}
	{{\textbf{{108}}}}
	(1990), 73--76.

\bibitem{BJR}
    {{\textrm{{C.~Benson, J.~Jenkins, G.~Ratcliff,}}}}
    {{\textrm {{The spherical transform of a Schwartz function on the
Heisenberg
       group}}}},
    {{\textit{\frenchspacing{J.\ Funct.\ Anal.}}}}
    {{\textbf{{154}}}}
     (1998), 379--423.

     
     
\bibitem{BJRW}
    {{\textrm{{C.~Benson, J.~Jenkins, G.~Ratcliff, T. Worku}}}}
    {{\textrm {{Spectra for Gelfand pairs associated with the
Heisenberg
       group}}}},
    {{\textit{\frenchspacing{Colloq.\ Math.}}}}
    {{\textbf{{71}}}}
     (1996), 305--328.

\bibitem{Monats.M}
	{{\textrm{{W.~O.~Bray,}}}}
	{{\textrm {{A spectral Paley--Wiener theorem}}}},
   {{\textit{\frenchspacing{Monatsh.\ Math.}}}}
   {{\textbf{{116}}}}
   (1993), {1--11}.

\bibitem{DR}
    {{\textrm{{E.~Damek, F.~Ricci,}}}}
    {{\textrm {{Harmonic analysis on solvable extensions of $H$--type groups}}}},
    {{\textit{\frenchspacing{J.\ Geom.\ Anal.}}}}
    {{\textbf{{2}}}}
     (1992), 213--248.
     
     
     
\bibitem{Olaf}
	{{\textrm{{S.~Dann, G.~\'Olafsson,}}}}
     {{\textrm {{Paley--Wiener theorems with respect to the spectral parameter}}}},
     in: {{\textit{{New developments in Lie theory and its applications}}}}, 
	{{\textit{\frenchspacing{Contemp.\ Math.}}}} {{\textbf{{544}}}}, Amer. Math. Soc., Providence, RI, 2011
	, p.~55--83. 

\bibitem{E}
	{{\textrm{{A.~Erdelyi, W.~Magnus, F.~Oberhettinger, G.~Tricomi,}}}}
	{{\textit{{Higher Transcendental Functions}}}},
	 {{\textbf{{\uppercase{i}}}}}
	McGraw-Hill, New York, 1953.

\bibitem{FR}
    {{\textrm{{F.~Ferrari Ruffino,}}}}
    {{\textrm {{The topology of the spectrum for Gelfand pairs on Lie groups}}}},
    {{\textit{\frenchspacing{Boll. Unione Mat. Ital. Sez. B}}}}
 {{\textbf{{10}}}} (2007),  569--579. 

\bibitem{Folland}
{{\textrm{{G.~B.~Folland,}}}}
	{{\textit{{Harmonic analysis in phase space}}}},
	Annals of Mathematics Studies
	{{\textbf{{122}}}}, 
	Princeton University Press, Princeton, NJ,  1989.

\bibitem{MN}
	{{\textrm{{H.~F\"uhr,}}}}
	{{\textrm {{Paley--Wiener estimates for the Heisenberg group}}}},
	{{\textit{\frenchspacing{Math.\ Nachr.}}}}
	{{\textbf{{283}}}}  (2010),  200--214.

\bibitem{HR}
    {{\textrm{{A.~Hulanicki, F.~Ricci,}}}}
    {{\textrm {{A {T}auberian theorem and tangential convergence for bounded
              harmonic functions on balls in {${\bf C}^{n}$}}}}},
    {{\textit{\frenchspacing{Invent.\ Math.}}}}
    {{\textbf{{2}}}}
     (1980/81), 325--331.

\bibitem{Kora}
{{\textrm{{A.~Kor\'anyi,}}}}
{{\textrm {{Some applications of Gelfand pairs in classical analysis}}}},
in :{{\textit{{Harmonic Analysis and Group Representations}}}},
C.I.M.E., Liguori, Napoli, 1980, p.~333--348.

\bibitem{AnnIFour}
	{{\textrm{{E.~K.~ Narayanan, S. Thangavelu}}}},
	{{\textrm {{A spectral Paley--Wiener theorem for the Heisenberg group 
	and a support theorem for the twisted spherical means on ${\mathbb C}^n$}}}},
 	{{\textit{\frenchspacing{Ann.\ Inst.\ Fourier (Grenoble)}}}}  
	{{\textbf{{56}}}}  (2006),   459--473.

\bibitem{JFA-Laplacians}
	{{\textrm{{R.~S.~Strichartz,}}}}
	{{\textrm {{Harmonic analysis as spectral theory of Laplacians}}}}, 
	{{\textit{\frenchspacing{J.\ Funct.\ Anal.}}}}
	{{\textbf{{87}}}} (1989), 51--148.
	

\bibitem{Revista}
	{{\textrm{{S. Thangavelu}}}},
 	{{\textrm {{A Paley--Wiener theorem for step two nilpotent Lie groups}}}},
 	{{\textit{\frenchspacing{Rev.\ Mat.\ Iberoamericana}}}}  
	{{\textbf{{10}}}}  (1994),   177--187.

\bibitem{JFA}
	{{\textrm{{S. Thangavelu}}}},
	{{\textrm {{On Paley--Wiener theorems for the Heisenberg group}}}},
 	{{\textit{\frenchspacing{J.\ Funct.\ Anal.}}}}  
	{{\textbf{{115}}}}  (1993),   24--44.

\bibitem{HirMJ}
	{{\textrm{{S. Thangavelu}}}},
	{{\textrm {{A Paley--Wiener theorem for the inverse Fourier transform on some
               homogeneous spaces}}}},
	{{\textit{\frenchspacing{Hiroshima Math.\ J.}}}}
	{{\textbf{{37}}}}  (2007),   145--159.
		
		

\bibitem{Th}
{{\textrm{{S. Thangavelu}}}},
   {{\textit{{Harmonic Analysis on the Heisenberg group}}}},
   Progress in Mathematics
   {{\textbf{{159}}}},
Birkh\"auser, Boston, Mass., 1998.

\bibitem{Tuan}
	{{\textrm{{V.~K.~Tuan,}}}}
	{{\textrm {{Paley--Wiener-type theorems}}}},
	{{\textit{\frenchspacing{Fract.\ Calc.\ Appl.\ Anal.}}}}
	{{\textbf{{2}}}}
	 (1999), 135--143. 

\end{thebibliography}

\end{document}

