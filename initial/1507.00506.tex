\documentclass{amsart}
\usepackage[english]{babel}

\usepackage{graphicx} 
\usepackage{longtable} 
\usepackage[toc,page]{appendix}
\usepackage[all]{xy}
\usepackage{hyperref}
\usepackage{fancyhdr}
\usepackage{amscd}
\usepackage{amssymb}
\usepackage{pb-diagram}
\usepackage{mathrsfs}

\newtheorem{thm}{Theorem}[section]
\newtheorem{prop}[thm]{Proposition}
\newtheorem{lm}[thm]{Lemma}
\newtheorem{condition}[thm]{Condition}
\newtheorem{cor}[thm]{Corollary}
\newtheorem{conj}{Conjecture}
\theoremstyle{definition}
\newtheorem{ex}[thm]{Example}
\newtheorem{definition}[thm]{Definition}
\newtheorem{rmk}[thm]{Remark}

\numberwithin{equation}{section}

\begin{document}
\title{On the Log-Picard functor for aligned degenerations of curves}
\author{Alberto Bellardini}
\address{KU Leuven Department of Mathematics\\
Celestijnenlaan 200B - box 2400\\
3001 Leuven\\
Belgium}
\email{albertobellardini@yahoo.it}
\subjclass[2010]{14H10; 14H40; 14K30}
\keywords{Logarithmic Picard functor; aligned deformations of curves; N\'eron models of jacobians; logarithmic geometry}
\begin{abstract}
We give a representability result as algebraic space for the relative logarithmic Picard functor of families of log-semistable curves satisfying a certain combinatorial condition over a regular base scheme of any dimension. We show that a more general version of the maximal separated quotient of the relative Picard functor constructed by Raynaud is a subgroup of it.
\end{abstract}
 \maketitle

\tableofcontents

\section*{Introduction}

\noindent In this work we show that the relative logarithmic Picard functor attached to a family of log-semistable curves satisfying a combinatorial condition called \emph{alignment} (definition \ref{defaligned}) over a regular base of any dimension and smooth over an open dense subscheme of the base is representable by an algebraic space (theorem \ref{teorema}).\\
As far as we know the logarithmic Picard functor was firstly introduced by \cite{kaj} in the case of log-curves without self-intersection over a field. \\
A relative version of this functor for families with fibers of any dimension was studied by Olsson in \cite{olpic} where a representability result is proved under a condition which we call in definition \ref{logcohfl} \emph{log-cohomological flatness} (in order to distinguish it from the classical cohomological flatness).\\
It is not clear to us under which general assumptions the log-cohomological flatness condition is satisfied.\\ 
Anyway in theorem \ref{teorema} we show that in the case of families of aligned curves the classical condition of cohomological flatness in dimension zero suffices to show the representability of the relative logarithmic Picard functor and it also gives the log-cohomological flatness.\\
\noindent It turns out that we obtain an algebraic space in groups, which naturally contains the generalization given in \cite{hol} of the maximal separated model of the relative Picard functor constructed by Raynaud in \cite{ray}. We suspect that this maximal separated quotient is indeed the relative logarithmic Picard functor but we are still not able to proof this. The obstruction to the equality is described by the short exact sequence \ref{beautifulextension}.\\

\noindent There is a naive way to understand the fact that the maximal separated quotient of the relative Picard functor is a subgroup of the relative logarithmic Picard functor.
It is known that for the relative Picard functor over discrete valuation rings the obstruction to being separated comes from line bundles arising from divisors concentrated in the special fiber. If one puts a log-structure coming from the divisor of the special fiber, one sees that the transition functions describing the 1-cocycle of these particular line bundles are naturally sections of the sheaf in groups which is the groupification of the sheaf of the log-structure. \\
In particular they can be interpreted as different trivializations of the trivial torsor under this group.
If we consider isomorphism classes of torsors under this group, these line bundles define the same isomorphism class, namely the one of the trivial torsor.\\
In other words these line bundles become trivial as logarithmic torsors. \\
This naive idea can be made more precise and natural by using cohomological methods as we show in lemma \ref{imagedelta}.\\

\noindent Beside the fact that this functor is closely related to the N\'eron model of the jacobians of these curves, we want to give another motivation about the reason for which we care about it. \\
In \cite{b14} we studied compactifications for jacobians of singular curves having at worst nodal singularities by relating the construction given by Oda and Seshadri in \cite{os} with the Relatively Complete Models of Mumford, Faltings and Chai as presented in \cite{fc}.\\
\noindent In particular we were able to recover and uniformize, in a modular way, some coarse moduli spaces of Oda and Seshadri, without using geometric invariant theory, and we have a functor for the uniformizing object.\\
Here the word ``some'' means that we can do this only for particular choices of the polarization one uses to construct the compactified jacobians of Oda and Seshadri.\\
The sheaves we used to uniformize the compactified jacobians naturally correspond to certain logarithmic torsors and we used the formalism of log-geometry to give functoriality to our construction.\\
In particular we showed that the sheaves we obtained have a natural interpretation in terms of the logarithmic Picard functor of a certain analytic covering of the curves. \\ In this paper we generalize the representability result 5.0.5 given in \cite{b14} for curves over discrete valuation rings.
\section*{Acknowledgements}
\noindent The author is supported by the ERC Starting Grant MOTZETA of J. Nicaise.\\

\section{Preliminaries in log-geometry}
\subsection{Log-semistable curves}\label{lageometrialogaritmica}

Standard facts about log-geometry can be found in \cite{ka}, we recall here some definition of \cite{olpic} that we need in this work.
\begin{definition}
    Given a separably closed field $k$ then a scheme $X$ over $k$ is called \emph{semistable variety} if for any closed point $\bar{x}\in X$ there exists an \'etale neighborhood $(U,u)$ and positive integers $m\leq n$ such that 
$U$ is \'etale over 
$$
Spec(k[X_1,\dots,X_n]/(X_1\cdot \dots \cdot X_m))
$$
\noindent and the point $u$ is sent to the point corresponding to the ideal $(X_1,\dots,X_n)$.
\end{definition}

\noindent Given a log-structure $M_X$ on a scheme $X$ we always have an exact sequence of sheaves in monoids
$$
0{\rightarrow} {\mathcal{O}}_X^{\times}{\rightarrow} M_X{\rightarrow} \overline{M}_X{\rightarrow} 0
$$
\noindent The sheaf $\overline{M}_X$ is usually called the \emph{characteristic of the log structure} $M_X$.\\

\begin{definition}\label{semistablemorphism}
    A log-smooth morphism $f:(X,M_X){\rightarrow} (S,M_S)$ is called \emph{ essentially semistable} if for each geometric point $\bar{x}{\rightarrow} X$ the monoids $(f^{-1}\overline{M}_S)_{\bar{x}}$ and $\overline{M}_{X,\bar{x}}$ are free and there exist isomorphisms $(f^{-1}\overline{M}_S)_{\bar{x}}\cong{\mathbb{N}}^r$ and $\overline{M}_{X,\bar{x}}\cong{\mathbb{N}}^{r+s}$ such that the induced map
$$
(f^{-1}\overline{M}_S)_{\bar{x}}{\rightarrow}\overline{M}_{X,\bar{x}}
$$
\noindent is ``multidiagonal'' namely on the generators is given by
$$
1_i{\rightarrow} \left\{
\begin{array}{cc}
1_i & \mbox{ if }i\neq r\\
1_r+1_{r+1}+\dots +1_s & \mbox{ if }i= r\\
\end{array}
\right.
$$
\end{definition}

\noindent Essentially semistable morphism are automatically flat and vertical (\cite{olun}Lemma2.3).\\ Vertical means that the cokernel of the map
$$
f^{*}M_S{\rightarrow} M_X
$$
\noindent is a sheaf of groups.\\
Let $f:(X,M_X){\rightarrow} (S,M_S)$ be a morphism of log-schemes. Let $I(\overline{M_S})$ be the set of irreducible elements in $\overline{M}_S$. Define 
$$
C(X):=\left\{\mbox{ connected components of the singular points of }X\right\}
$$
\noindent If $f$ is essentially semistable then there is a morphism to the set of irreducible elments
$$
s_X:C(X){\rightarrow} I(\overline{M_S})
$$
\noindent given by sending a component to the unique irreducible element whose image in $\overline{M}_{X,x}$ is not irreducible.\\

\begin{definition}\label{specialmorphism}
An essentially semistable morphism of log-schemes $f:(X,M_X){\rightarrow} (S,M_S)$ is called \emph{special} at a geometric point $\bar{s}\in S$ if the map
$$
s_{X_{\bar{s}}}:C(X_{\bar{s}}){\rightarrow} I(\overline{M}_{S,\bar{s}})
$$
induces a bijection between the set of connected components of the singular locus of $X_{\bar{s}}$ and $I(\overline{M}_{S,\bar{s}})$. A morphism is called special if it is special at every closed point.
\end{definition}
\noindent General facts about special morphisms are given in \cite{olun}.\\
\noindent Assume that the base scheme is the spectrum of a field and that 
$$
f:(X,M_X){\rightarrow} (S,M_S)
$$
\noindent is special.\\

\noindent There is an isomorphism, induced by $s_X$,
$$
\overline{M}_S\cong{\mathbb{N}}^{C(X)}
$$
\noindent Given $c\in C(X)$ one defines the subsheaves of ``the branches at $c$''
$$
\overline{M}_X\supset\overline{M}_c:=\left\{
\begin{array}{c}
x\in \overline{M}_X \mbox{ s.t. \'etale locally }\exists \;y\in \overline{M}_X \mbox{ with }\\
x+y \mbox{ is a multiple of }c
\end{array}
\right\}
$$
\noindent whose preimage in $M_X$ gives subsheaves $M_c$. One recovers the log-structure using the previous sheaves by push-out w.r.t. ${\mathcal{O}}_{X}^{\times}$, i.e. there is an isomorphism
$$
M_X\cong \bigoplus_{c\in C(X), {\mathcal{O}}_X^{\times}} M_c
$$
\noindent There is also another way to see these sheaves. A connected component corresponding to a $c\in C(X)$ is defined \'etale locally around a point $x$ by an ideal of the form
$$
J_c=(x_1\cdots \hat{x}_j \cdots x_r)_{j=1}^r
$$
\noindent where $\hat{x}_j$ means that we put $1$ in the place of that function. One considers the blowup of $X$ along $J_c$ 
$$
\nu_c:\tilde{X}_c{\rightarrow} X
$$
\noindent and shows (\cite{olpic} 2.15) that there is an isomorphism 
$$
\overline{M}_{c}\cong \nu_{c,*}{\mathbb{N}}
$$
Locally if $\bar{x}$ is a closed point with branches $x_1,\dots,x_r$ in $\tilde{X}_c$ then the isomorphism is given by
$$
\overline{M}_{c,\bar{x}}{\rightarrow} \bigoplus_{x_i}{\mathbb{N}}_{x_i}
$$
\begin{definition}
We call a log-smooth, proper, flat morphism 
$$
f:(X,M_X){\rightarrow} (S,M_S)
$$
\noindent whose geometric fibers are one dimensional and essentially semi-stable a \emph{log-semistable curve}.
\end{definition}
\begin{rmk}\label{makespecial}
Observe that given a log-smooth, proper, integral morphism 
$$
f:(X,M_X){\rightarrow} (S,M_S)
$$
\noindent such that the geometric fibers are semistable is always possible to change the log-structure, i.e. find $M_X^{\sharp}$ on $X$ and $M_S^{\sharp}$ on $S$, in an unique way (up to isomorphism) such that the induced morphism 
$$
f:(X,M_X^{\sharp}){\rightarrow} (S,M_S^{\sharp})
$$
\noindent is special (\cite{olun} 2.6). Hence it is harmless to assume this condition if we are only interested in the geometry of the underlying scheme. 
\end{rmk}
\subsection{Log-aligned curves}
In the paper \cite{hol} Holmes introduced a class of curves for which the N\'eron model of the jacobian is representable.\\
He called these curves \emph{aligned}. Let us recall this construction and let us translate it in the log-geometry world.\\
Consider $S$ a locally noetherian scheme and $C{\rightarrow} S$ be a semistable curve on $S$. For any point $\bar{s}\in S$ with separably closed residue field, the fiber $C_{\bar{s}}$ is a semistable curve. Let $\Gamma_{\bar{s}}$ be the dual intersection graph of $C_{\bar{s}}$. \\
For any edge $e$ of $\Gamma_{\bar{s}}$, corresponding to a node $c\in C_{\bar{s}}$ we can find an element $\alpha\in\mathfrak{m}_{{\mathcal{O}}_{S,\bar{s}}^{et}}$ such that we have an isomorphism
$$
\widehat{{\mathcal{O}}_{C,c}}\cong \widehat{{\mathcal{O}}_{S,\bar{s}}}[[x,y]]/(xy-\alpha)
$$
\noindent The element $\alpha$ is not unique but the ideal
$$
\alpha{\mathcal{O}}_{S,s}^{et}
$$
\noindent is unique (\cite{hol} 2.6). \\
In particular for any edge $e\in \Gamma_{\bar{s}}$ we get a well defined element in the monoid 
$$
L={\mathcal{O}}_{S,\bar{s}}^{et}/{\mathcal{O}}_{S,\bar{s}}^{et,\times}
$$

\noindent This gives us a so called $L$-edge labelling, namely a map
$$
l:E{\rightarrow} L
$$
\noindent where $E$ is the set of edges of $\Gamma_{\bar{s}}$.\\
Let us assume now that the curve is smooth over a non empty open $U\subset S$ and that $S\setminus U$ is a divisor. We put the \'etale log-structure $M_S$ on $S$ induced by this divisor. 
With this definition the element $\alpha$ is in the image of the map
$$
M_{S,\bar{s}}{\rightarrow} {\mathcal{O}}_{S,\bar{s}}^{et}
$$
\noindent We have the following diagram
$$
\xymatrix{
    {0}\ar[r] & {{\mathcal{O}}_{S,\bar{s}}^{et,\times}}\ar[r]\ar[d]^{\cong} & {M_{S,\bar{s}}}\ar[d]\ar[r] & {\overline{M}_{S,\bar{s}}}\ar[r] \ar[d]^{g}&{0}\\
    {0}\ar[r] & {{\mathcal{O}}_{S,\bar{s}}^{et,\times}}\ar[r] & {{\mathcal{O}}_{S,\bar{s}}^{et}}\ar[r] & {{\mathcal{O}}_{S,\bar{s}}^{et}/{\mathcal{O}}_{S,\bar{s}}^{et,\times}=L}\ar[r] & {0}\\
}
$$
\noindent By uniqueness up to invertible functions we see that for any $l(e)$ there exists a unique element in $g^{-1}(l(e))\in \overline{M}_{S,\bar{s}}$.\\
We get a well defined map
$$
l^{log}: E{\rightarrow} \overline{M}_{S,\bar{s}}
$$
\noindent with $l^{log}=g^{-1}(l(e))$.\\
\begin{definition}[\cite{hol}]\label{defaligned}
Let $M$ be a commutative monoid and $(\Gamma,l,M)$ be a graph with an $M$ edge labelling. We say that $(\Gamma,l,M)$ is \emph{aligned} if for any circuit $H$ in $\Gamma$ and any pairs of edges $e_1,e_2$ in $H$ there exist positive integers $n_1,n_2$ such that
$$
n_1l(e_1)=n_2l(e_2)
$$
\noindent Given $S$ a locally noetherian scheme, $C{\rightarrow} S$ a family of semistable curves and geometric point $s\in S$ we say that $C$ is aligned at $s$ if the triple $(\Gamma_{s},l,{\mathcal{O}}_{S,s}^{et}/{\mathcal{O}}_{S,s}^{et,\times})$ defined before is aligned. \\
We say that a semistable curve $C/S$ is aligned if it is aligned at any geometric point $s\in S$.\\
Given $(C,M_C){\rightarrow} (S,M_S)$ a log-semistable curve with $M_S$ divisorial
we say that it is log-aligned if for any geometric point $s\in S$ the triple
$$
(\Gamma_{s},l^{log},\overline{M}_{S,s})
$$
\noindent is aligned.
\end{definition}
It is immediate from the definitions now the following lemma.
\begin{lm}
Let $S$ be a locally noetherian scheme with a divisorial log-structure $M_S$. Let $(C,M_C){\rightarrow} (S,M_S)$ be a log-semistable curve with $M_C$ induced by the branches of the singularities of $C$.\\ 
The curve $(C,M_C){\rightarrow} (S,M_S)$ is log-aligned iff it the underlying scheme map $C{\rightarrow} S$ of curves is aligned.
\end{lm}

Observe that when $S$ is the spectrum of a discrete valuation ring or when the graphs $\Gamma_{\bar{s}}$ have no circuits then the alignment (resp. log-alignment) condition is automatically satisfied.\\

\begin{rmk}\label{alignmentcod2}
For families of semistable curves over a regular in codimension one scheme which are smooth over a dense open, regular subscheme, one can achieve alignment over an open whose complementary has codimension at least 2.
This follows from the fact that for semistable curves, generically smooth over discrete valuation rings, the alignment condition is automatic.\\
\end{rmk}

The notion of log-alignement is important in the study of the N\'eron model for families of curves over excellent regular base with dimension bigger than one.\\
We only recall the facts we are going to use later.
\begin{thm}[\cite{hol} Theorem 1.2, Theorem 6.13]\label{hol:main}
Let $S$ be an excellent, separated, regular scheme and $C{\rightarrow} S$ be a semistable and aligned curve over $S$ with $C$ regular. Let $U\subset S$ be a schematically dense open subset and assume that $C|_U$ is smooth. Then
\begin{enumerate}
    \item The jacobian of $C|_U$ admits a N\'eron model over $S$.
    \item Let ${\mbox{Pic}}_{C/S}^{[0]}$ be the closure of ${\mbox{Pic}}_{C_U/U}^0$ in $Pic_{C/S}$ and $clo(e_U)$ be the closure of the identity section $e_U\in Pic_{C_U/U}^{0}$ in $Pic_{C/S}^{[0]}$. Then $clo(e_U)$ is \'etale over $S$.
\end{enumerate}
\end{thm}
\begin{proof}
We only need to explain point 2) because in \cite{hol} Theorem 6.13 the closure of the identity section is claimed to be flat and not \'etale. However looking at the proof to get \'etaleness one only needs to check that in Lemma 6.14 we can replace ``flat'' by ``\'etale''. Looking at the proof of 6.14 Holmes shows a local isomorphism statement to get flatness. Hence with the same argument we get \'etaleness.
\end{proof}
\begin{definition}
  Define $E$ to be the schematic closure in ${\mbox{Pic}}_{C/S}$ of the unity section in ${\mbox{Pic}}_{C_U/U}^{0}$.
\end{definition}
\noindent Since in our case $Pic_{C/S}$ is representable by an algebraic space in groups locally of finite type we have that $E$ is also representable by an algebraic space in groups locally of finite type.\\
From the inclusions $E\subset {\mbox{Pic}}_{C/S}^{[0]}\subset {\mbox{Pic}}_{C/S}$ we have that
$$
E=clo(e_U)
$$
\begin{cor}\label{Eetale}
The algebraic group space $E$ is \'etale over $S$.
\end{cor}
\section{The Log Picard functor}\label{chlogpic}

\noindent In this section we want to study the representability problem for the log-Picard functor associated with families of log-aligned curves.\\
We first recall some well known facts that can be found for example in \cite{blr} or \cite{ray}. \\

\begin{definition}\label{cohomofla}
    Let $f:C{\rightarrow} S$ be a proper morphism of finite presentation. We call the morphism $f$ \textbf{cohomologically flat in dimension zero} over $S$ if the formation of $f_{*}{\mathcal{O}}_X$ commutes with every base change over $S$.
\end{definition}
\begin{rmk}
In the case of flat families of proper curves this happens if for example $S$ is reduced and the dimension of the vector spaces $H^1(X_s,{\mathcal{O}}_{X_s})$ is constant.
\end{rmk}
\noindent In order to distinguish from the classical cohomological flatness we recall the following notion which is also called cohomological flatness in dimension zero in \cite{olpic}.
\begin{definition}[\cite{olpic} 4.3]\label{logcohfl}
A morphism $(C,M_C){\rightarrow} (S,M_S)$ of log-schemes is called \textbf{log-cohomologically flat in dimension zero} if for any nilpotent closed immersion $Spec(A_0){\rightarrow} Spec(A)$ over $S$ defined by square zero ideal, the natural map
$$
H^0(C_A,M_{C_A}^{gp}){\rightarrow} H^0(C_{A_0},M_{C_{A_0}}^{gp})
$$
is surjective.
\end{definition}
We consider the category $(Sch/S)$ inside the category of log-schemes as follows: let $g:T{\rightarrow} S$ then we put on $T$ the log-structure $g^{*}M_S$.\\
\begin{rmk}\label{univsat}
    Observe that since the morphism $(T,g^{*}M_S){\rightarrow} (S,M_S)$ is strict then the fiber product $(C_T,M_{C_T})$ in the category of fine and saturated log schemes coincides with the fiber product in the category of log-schemes. In particular the log-scheme $(C_T,M_{C_T})$ is fine and saturated (see \cite{fka} lemma 3.4). 
\end{rmk}
\begin{definition}

\begin{enumerate}

\item Given a morphism of log-schemes 
    $$
    f:(C,M_C){\rightarrow} (S,M_S)
    $$
    \noindent the \textbf{log Picard stack} on $Sch/S$ is the stack in the \'etale topology corresponding to the groupoid whose fiber over a scheme $g: T{\rightarrow} S$ is defined by
$$
\mathcal{P}ic_{C/S}^{log}(T):=\{M_{C_T}^{gp} \mbox{-torsors on } (C_{T})_{et}\}
$$
\item the \textbf{log Picard functor}, denoted with ${\underline{\mbox{Pic}}^{log}}_{C/S}$ is the \'etale sheafification on $(Sch/S)_{{et}}$ of the functor of isomorphism classes for log Picard stack. As sheaf it can be written as the sheafification of
$$
T{\rightarrow} \mathcal{P}ic_{C/S}^{log}(T)/\cong
$$
\item Assume that the fibers of ${\underline{\mbox{Pic}}^{log}}_{C/S}$ over geometric points are representable by group schemes then we define  
$$
({\underline{\mbox{Pic}}^{log}}_{C/S})^{0}(T):= \{u\in {\underline{\mbox{Pic}}^{log}}_{C/S}(T) \; |\;  \forall \; t\in T \; u(|t|)\in |({\underline{\mbox{Pic}}^{log}}_{C_t/t})^0(t)|\}
$$
\end{enumerate}
\end{definition}

\noindent The only representablity result we know about this functor is the following.
\begin{thm}[\cite{olpic} 4.6]\label{olssonrepres}
    Let $f:(X,M_X){\rightarrow} (T,M_T)$ be a proper, special (\ref{specialmorphism}) morphism of schemes which is log-cohomologically flat in dimension zero then ${\underline{\mbox{Pic}}^{log}}_{X/S}$ is representable by an algebraic space.
\end{thm}
\noindent The proof is a careful application of Artin representability criterion.\\ 
Having fixed $(C,M_C){\rightarrow} (S,M_S)$ we use the notation
$$
{\underline{\mbox{Pic}}^{log}}={\mbox{Pic}}_{C/S}^{log}
$$
\noindent\\
Our main result is the following.
\begin{thm}\label{teorema}
Let $(S,M_S)$ be a log-scheme such that $S$ is separated, regular and excellent. Assume that the log-structure $M_S$ is induced by a normal crossing divisor whose complement we denote with $U$. Let 
$$
f:(C,M_C){\rightarrow} (S,M_S)
$$
\noindent be a special, log-aligned curve, such that the underlying map of schemes is cohomologically flat in dimension zero over $S$ and smooth over $U\subset S$. \\ 
Assume that the log-structure on $C$ is trivial on $f^{-1}U$ and that $C$ is regular. Then 
\begin{enumerate}
    \item\label{repr} The sheaf ${\underline{\mbox{Pic}}^{log}}$ is representable by an algebraic space locally of finite type over $S$.
\item The algebraic space $({\underline{\mbox{Pic}}^{log}})^0$ is separated and it isomorphic to the connected component of the identity of the N\'eron model of the relative jacobian of $C$ (which exists by \ref{hol:main}).
\item The morphism $f:(C,M_C){\rightarrow} (S,M_S)$ is log-cohomologically flat in dimension zero.
\end{enumerate}
\end{thm}
\begin{proof}
The proof we use here is a generalization of the proof given in \cite{b14}.\\
We consider the exact sequence of sheaves on $C$ in the \'etale topology
$$
0{\rightarrow}{\mathbb{G}m}{\rightarrow} M_C^{gp}{\rightarrow} \overline{M}_C^{gp}{\rightarrow} 0
$$

\noindent This gives us a long exact sequence
\begin{equation}\label{longexactseq}
f_{*}\overline{M}_{C}^{gp}\stackrel{\delta}{\rightarrow} R^1f_{*}{\mathbb{G}m}{\rightarrow} R^1f_{*}M_{C}^{gp}{\rightarrow} R^{1}f_{*}\overline{M}_{C}^{gp}{\rightarrow} R^2f_{*}{\mathbb{G}m}
\end{equation}
\noindent of sheaves in group. \\ We want to study the representability of the terms in this sequence.\\
Namely the functors we are interested in are obtained by taking $H^0$ for the \'etale topology of the sheaves in the previous sequence (\cite{blr} 8.1). Using a little abuse of notation we will omit to write the symbol $H^0$ because it will be clear from the contest whether we are using it or not.\\
We recall now a definition.
\begin{definition}
  Let $A$ be noetherian local ring with maximal ideal ${\mathfrak{m}}$ and $\hat{A}$ be its ${\mathfrak{m}}$-adic completion. We say that $A$ satisfies the \emph{approximation property} if for any $A$-scheme $X$ of finite type and for any point $\hat{a}\in X(\hat{A})$ there exists a point $a\in X(A)$ such that if we denote with $\overline{\hat{a}}$ (resp. $\overline{a}$) the compostion 
  $$
  Spec(\hat{A}/{\mathfrak{m}}\hat{A}){\rightarrow} Spec(\hat{A})\stackrel{\hat{a}}{\rightarrow} X
  $$
\noindent (resp. 

$$
Spec(\hat{A}/{\mathfrak{m}}\hat{A}){\rightarrow} Spec(A)\stackrel{a}{\rightarrow} X
$$
\noindent ) then $\overline{\hat{a}}=\overline{a}$.\\
\end{definition}
We need the following remark in the proof of the next lemma.
\begin{rmk}\label{pope} By a theorem of Popescu (\cite{popescu} Theorem 1.3) any excellent, henselian local ring satisfies the approximation property.
\end{rmk}

\begin{lm}\label{stredarg}
    Let $C{\rightarrow} S$ be a proper flat morphism with 1-dimensional fibers. Assume that there is a line bundle $\mathcal{L}$ on $C$ which is ample (defined in \cite{ega} II 4.5.3), that $C$ is regular and that $S$ is excellent. Then
  $$
  R^2f_{*}\mathbb{G}_m=0
  $$
\end{lm}
\begin{proof}
The proof is essentially the same as in \cite{br3} with small modifications. Let us recall the argument.\\
We need to show that the geometric fibers of the sheaf are zero. \\
By standard reduction arguments (\cite{sga} 4, Exp. VIII, Th\'eor\`eme 5.2) we can assume that $(S,s)$ is strictly local and after base changing to this situation we need to show that $H^2(C,{\mathbb{G}m})$ is zero.\\
Since $C$ is regular then $H^2(C,{\mathbb{G}m})$ coincides with its torsion subgroup $Br(C)^{\prime}$ by \cite{br2} Proposition 1.4.\\ 
Under our hypothesis we can apply Gabber's theorem (see \cite{gabber} ) to conclude that
$$
H^2(C,{\mathbb{G}m})=Br(C)^{\prime}\cong Br(C)
$$
\noindent
Let $C_s$ be the special fiber. This is a scheme of dimension one hence 
$$
H^2(C_s,{\mathbb{G}m})=H^2(C_{\bar{s}},{\mathbb{G}m})=Br(C_{\bar{s}})=0
$$
\noindent where $\bar{s}$ is a geometric point over $s$ (\cite{br3} 5.8). \\
Hence it is enough to show that the canonical map
$$
Br(C){\rightarrow} Br(C_s)
$$
\noindent is injective.\\
By using infinitesimal thickenings of $C_s$ inside $C$ one can reduce to lemma 3.3 in \cite{br3}. \\
This lemma still works in our situation because we can replace Greenberg's theorem with the approximation property (see \cite{br3} Remarques 3.4), since we use the fact that we reduced to the case in which $S$ is strictly henselian (use remark \ref{pope}).  \\
The Mittag-Leffler property follows in our case because the fibers are curves and there are no obstructions to lift line bundles.\\ 
\end{proof}
\noindent We want to use this result in our case.\\
\begin{prop}\label{r2zero}
Let $f:C{\rightarrow} S$ be the underlying morphism of schemes as in the theorem then
$$
R^2f_{*}{\mathbb{G}m} =0
$$
\end{prop}
\begin{proof}
Using the previous lemma we are almost done.\\
Working locally we may assume that $S$ is affine and replacing $S$ with the strict henselianization of $S$ at a point $s\in S$ we may assume that $S$ is the spectrum of a local, henselian and excellent ring. \\
In particular we may assume that we have enough sections of $C{\rightarrow} S$ through the smooth locus, so that $C/S$ is projective over $S$.\\ 
Hence there is an invertible sheaf $\mathcal{L}$ on $C$ ample relatively to $S$.\\
Since $S$ is affine, then by EGA II 4.5.10 i) the line bundle $\mathcal{L}$ is also ample on $C$. Using the previous lemma we can conclude.
\end{proof}

\noindent We see in lemma \ref{imagedelta} that the obstruction to the separateness property of $Pic_{C/S}$ is represented by the sheaf $f_{*}\overline{M}_C^{gp}$. \\
In particular we cannot expect that this sheaf is a separated object. Anyway we have local separateness as next result establishes.
\begin{prop}\label{mbaralgsp}
  Let $(C,M_C){\rightarrow} (S,M_S)$ be a family of curves as in the theorem then the \'etale sheaves 
$
R^{1}f_{*}\overline{M}_{C}^{gp}
$
 and  
 $f_{*}\overline{M}_{C}^{gp}$ (viewed as sheaves on $(Sch/S)_{et}$) are representable by algebraic spaces in groups over $S$ which are locally separated and \'etale.
\end{prop}

\begin{proof}
    We are going to verify the conditions of Artin's representability theorem in the form of theorem 5.3 in \cite{aalg}. \\
Observe that in that paper Artin restricts to the case in which the scheme $S$ is of finite type over a field or over an excellent Dedekind domain in order to obtain the algebraization property.\\
It has been shown in \cite{cdej} that this condition can be omitted. We can apply Artin theorem also in the case that $S$ is assumed to be only excellent.\\

\noindent Condition $[0^{\prime}]$ follows because there is no difference between \'etale and flat log-structure (\cite{ollog} Theorem A.1) hence we have that $f_{*}\overline{M}_{C}^{gp}$ and $R^1f_{*}\overline{M}_{C}^{gp}$ are sheaves in the fppf topology.\\

\noindent Let us see that the functors are limit preserving.\\
Take an inductive family $\{A_i\}$ of noetherian rings over $S$.\\
Take a system $T_i:=Spec(A_i)$, $T=\varprojlim T_i$, $f_i:C_i{\rightarrow} T_i$, $u_i:T{\rightarrow} T_i$ the projection morphism, $C=\varprojlim C_i$ and analogously for the log structures.\\
Using \cite{sga} 4, Exp. VII, Corollaire 5.11 we see that the morphisms
$$
\varinjlim u_i^{*}R^nf_{i,*}\overline{M}_{C_i}^{gp}(S_i){\rightarrow} R^nf_{\infty,*}\overline{M}_{C_{\infty}}^{gp}(S)
$$
\noindent are bijective for all $n\geq 0$ in particular condition $[1^{\prime}]$ is also satisfied.\\

\noindent The sheaf $\overline{M}_{C}^{gp}$ is a constructible sheaf of ${\mathbb{Z}}$-modules by \cite{ollog} Lemma 3.5 ii.\\ 

\noindent Since $f$ is proper, special and log-semistable, if we take the base change to an affine scheme $T$, which is the spectrum of a complete local ring with separably closed residue field, we have a decomposition
$$
H^0(T,\overline{M}_{T})\cong \bigoplus_{c_i\in C(C_0)}{\mathbb{N}} n_{c_i}
$$
\noindent and 
$$
\overline{M}_{C_{T}}^{gp}\cong \bigoplus_{c\in C(C_0)}\overline{M}_{c}^{gp} 
$$
\noindent where $C(C_0)$ denotes the set of connected components of the singular locus of the special fiber $C_0$, and $\overline{M}_{c}$ are ``the branches at $c$'' defined as follows
$$
\overline{M}_{c}:=\left\{
\begin{aligned}
x\in \overline{M}_{C_0} \mbox{ such that \'etale locally}\\
\mbox{exists }y\in \overline{M}_{C_0} \mbox{ with } x+y\in (n_c)
\end{aligned}
\right\}
$$

\noindent Write $T=Spec(R)$ and let $t_c\in R$ the function corresponding to the component $c\in C(C_R)$.\\
The sheaf $\overline{M}_{c}^{gp}$ is supported on $t_c=0$ so we can assume that in $R$ all $t_c$ are zero.\\
Let $Z_c$ be the corresponding connected component of the locus where $f$ is not smooth.\\
If $z\in Z_c$ then, \'etale locally, around $z$ the scheme $C_R$ is isomorphic to 
$$
Spec(R[x_1,x_2]/(x_1x_2))
$$
\noindent Let $J_c$ be the ideal corresponding to $Z_c$, i.e. locally given by 
$$(x_1,x_2)$$
\noindent and we define 
$$
\nu_c:\tilde{C}_{R,c}{\rightarrow} C_R
$$
\noindent as the proper transform of the blow-up of $C_R$ at $J_c$.\\
On $C_0$ the sheaf $\overline{M}_c^{gp}$ is isomorphic to $\nu_{c,*}{\mathbb{Z}}$.\\
The same argument repeats whenever we base change with an affine artinian thickening over $S$ 
$$
Spec(A){\rightarrow} S
$$

\noindent In particular, given a surjective morphism $A{\rightarrow} A_0$ of artinian $S$-algebras, with square zero kernel, we have that the map
\begin{equation}\label{fdefzero}
H^0(C_{A},\overline{M}_{C_A}^{gp}){\rightarrow} H^0(C_{A_0},\overline{M}_{C_{A_0}}^{gp})
\end{equation}
\noindent is an isomorphism. Indeed by the previous decomposition over complete local rings we have that the group
$$
H^0(C_A,\overline{M}_{C_{A}}^{gp})
$$
\noindent is isomorphic to direct sum of copies of ${\mathbb{Z}}$ indexed by $C(C_0)$. \\
Since the set of connected components does not changes under nilpotent thickenings (\cite{aas} 3.1) we obtain the claim.\\

\noindent This also shows that condition $[2^{\prime}]$ is satisfied by
$$
f_{*}\overline{M}_C^{gp}
$$

\noindent Let $A$ be a complete, local noetherian ring with maximal ideal $\mathfrak{m}$ over $S$. Let $A_n:=A/\mathfrak{m}^n$. \\
Using \cite{olpic} Lemma 4.18 we have  
$$
H^0(C_A,\overline{M}_{C_A}^{gp})\cong \varprojlim H^0(C_{A_n},\overline{M}_{C_{A_n}}^{gp})
$$
\noindent and 
\begin{equation}\label{r1fdef}
H^1(C_A,\overline{M}_{C_A}^{gp})=0=H^1(C_{A_n},\overline{M}_{C_{A_n}}^{gp})
\end{equation}
\noindent In particular the condition $[2^{\prime}]$ is also satisfied by
$$
R^1f_{*}\overline{M}_C^{gp}
$$
\noindent We are now going to check $[3^{\prime}]$.
First let us remark the following:\\
Let $A_0$ be local henselian over $S$. Assume we have two sections $\xi_1$ and $\xi_2$ of $f_{*}\overline{M}_C^{gp}(Spec(A_0))$ coinciding on a point $x\in Spec(A_0)$.\\
Since we are in a group functor we can consider the difference
$$
s=\xi_1-\xi_2\in H^0(C_{A_0},\overline{M}_{C_{A_0}}^{gp})
$$
\noindent Now the set of points $c\in C_{A_0}$ where $s$ is the trivial element in the sheaf $\overline{M}_{C_{A_0},c}^{gp}$ is a non empty open $W\subset C_{A_0}$ because by \cite{ollog} 3.5 it is constructible and stable under generalization. \\
Let $Z=C_{A_0}\setminus W$.\\
By properness of $f$ we have that $f(Z)$ is closed.\\ Its complementary $V$ is then open and non empty because it contains $x$. Over the open $V$ the element $s$ is zero and the two sections coincide.\\
Since we are looking for open subschemes we can always base change to the henselianization at points and apply the previous argument there to check that the locus where two sections coincide is open.\\
Consider now $[3]^{\prime}(a)$. For the sheaf $f_{*}\overline{M}_C^{gp}$ the previous argument applies.\\
Let us consider $R^1f_{*}\overline{M}_C^{gp}$. Since the statement is true in the closed point of $Spec(A_0)$ and $A_0$ is a dvr it is enough to find an open for which the statement is true. \\
It is enough to check this over the strict henselianization $A^h$ of $A_0$ at the closed point $s$ of $Spec(A_0)$. \\
Using the proper base change theorem in the form of \cite{sga} 4, Exp. XII Cor. 5.5 we have an inclusion
$$
R^1f_{*}\overline{M}_C^{gp}(A^h)=H^1(C_{A^h}, \overline{M}_{C_{A^h}}^{gp})\hookrightarrow H^1(C_{s}, \overline{M}_{C_s}^{gp})
$$
\noindent hence the two torsors coincide globally.\\

\noindent Let us check $[3]^{\prime}(b)$. 
For the sheaf $f_{*}\overline{M}_C^{gp}$ there are no problems by the same argument as before.\\
Let us consider $R^1f_{*}\overline{M}_C^{gp}$.\\
Again it is enough to check the statement after base change to the henselianization. If $x$ denotes one of the points in the dense set and $A_x$ denotes the strict henselianization of $A_0$ at $x$ it is enough to find an open $x\in V\subset Spec(A_x)$ where the statement is true.\\
We apply as before the proper base change theorem and we get $V=Spec(A_x)$.\\

\noindent From conditions \ref{fdefzero} and \ref{r1fdef} we get that the module of the deformation theory is the zero module for both functors. Furthermore obstruction theory is also trivial. In particular conditions $[4^{\prime}]$ and $[5^{\prime}]$ are also satisfied by both functors.\\
The absence of obstructions to lift over infinitesimal thickenings and trivial deformation theory tells us that the corresponding algebraic spaces are \'etale over $S$.
\end{proof}

\begin{cor}\label{r1zero}
Let $(C,M_C){\rightarrow} (S,M_S)$ as in the previous proposition. The subfunctors of connected component of the identity $(f_{*}\overline{M}_C^{gp})^0 $ and $(R^1f_{*}\overline{M}_C^{gp})^0$ (which exist because the fibers are representable by schemes) are isomorphic to the trivial group.
\end{cor}
\begin{proof}
The algebraic spaces in the statement have the same properties we need in order to prove the corollary. We denote both of them with $F{\rightarrow} S$.\\
By \'etaleness, connectedeness, finite type property and representability over fields, the fiber of $F{\rightarrow} S$ over a point $s\in S$ is the spectrum of a finite separable extension of the residue field at $s$.\\ 
In particular the morphism $F{\rightarrow} S$ is quasi-finite. 
Furthermore it is also separated by the following lemma.
\begin{lm}
Let $S$ be a scheme, $G{\rightarrow} S$ be an algebraic space in groups over $S$ \'etale over $S$ and with connected fibers then $G$ is separated.
\end{lm}
\begin{proof}
Since $G{\rightarrow} S$ is \'etale then it is universally open. Using the algebraic space version of \cite{sga} 3, Exp. $VI_B$ 5.5 we can conclude. 
\end{proof}
The lemma gives that $F$ is representable by a scheme (\cite{kn} Chapter II, Corollary 6.17).
We want to show that $F{\rightarrow} S$ is an isomorphism.\\
Since being an isomorphism descends under \'etale base change we may assume that $S$ is henselian and local.\\ By the structure theorem for quasi-finite, separated schemes over an henselian base we have
$$
F=F_{\eta}\coprod F^f
$$
\noindent with $F^f$ finite over $S$ and $F_{\eta}$ having empty closed fiber. \\
Since $F$ has non empty closed fiber and the fibers are connected we must have $F=F^f$.\\
In particular $F{\rightarrow} S$ is finite.\\ Using the Stein factorization
applied to the structure morphism $f:F{\rightarrow} S$ we have $F=Spec_{S}(f_{*}{\mathcal{O}}_F)$ and flatness of $f:F{\rightarrow} S$ tells us that $f_{*}{\mathcal{O}}_F$ is a locally free sheaf of ${\mathcal{O}}_S$-algebras of a certain fixed rank $n$. 
Over the open $U\subset S$ we have $F|_U\cong U$ because the log-structure is trivial on $U$ and by locally freeness we have that $n=1$. But then
$$
f_{*}{\mathcal{O}}_F\cong {\mathcal{O}}_S
$$
\noindent and we are done.\\
\end{proof}
We want now to relate the image of the morphism $\delta$ in \ref{longexactseq} with $E$.
\begin{lm}\label{imagedelta}
\noindent The morphism $\delta$ in \ref{longexactseq} is surjective onto $E$. \\
\end{lm}
\begin{proof}
    Using the same reduction arguments as in the proof of \ref{stredarg} we can assume that $S$ is strictly henselian and local with special point $s$.\\
Hence we can replace $ R^1f_{*}\mathbb{G}m$ with $H^1(C,\mathbb{G}_m)$ and $f_{*}\overline{M}_C^{gp}$ with $ H^0(C,\overline{M}_{C}^{gp})$. \\
Furhtermore without groupification this also holds for the derived functors in degree zero, i.e. we can replace $(f_{*}\overline{M}_C)$ with $H^0(C,\overline{M}_{C})$ even if $\overline{M}_C$ is not a sheaf in abelian groups.\\
Using the the proper base change theorem in degree zero and one we have an injective map (\cite{sga} 4, Exp. XII Corollaire 5.5)
\begin{equation}\label{h1special}
H^1(C,\mathbb{G}_m){\rightarrow} H^1(C_{s},\mathbb{G}_m)
\end{equation}
\noindent and a bijection
\begin{equation}\label{h0special}
H^0(C,\overline{M}_{C}^{gp})= H^0(C_{s},\overline{M}_{C_{s}}^{gp})
\end{equation}
\noindent where $s$ is the closed point in $S$. \\
Under this locality assumption we can assume that $C$ has a sections through the smooth locus in every connected component of every fiber.
As seen in the proof of \ref{r2zero} we can then assume that $C$ is projective. With these reductions we can also assume that that 
$$
Pic_{C/S}
$$
\noindent is representable by a scheme (\cite{blr} theorem 8.2.2) and that points in $Pic_{C/S}$ are representable by line bundles on $C$.\\
Over the open $U\subset S$ the log structure is trivial hence there is no difference between $M_{C_U}^{gp}$ and ${\mathbb{G}m}_{C_U}$. \\
In this way we get an isomorphism
 $$
 f_{*}\overline{M}_{C}^{gp}|_U\stackrel{\cong}{\rightarrow} E|_U
 $$
\noindent Using \ref{h1special} and \ref{h0special} it is enough to show that for the closed point $s\in S$ and any section $\sigma\in H^0(C_s,\overline{M}_{C_s}^{gp})$ we have $\delta(\sigma)\in E_s$.\\

\noindent Let $W\subset f_{*}\overline{M}_C^{gp}$ be the preimage of the open $U\subset S$ and consider a trait 
$$
\Sigma:T{\rightarrow} f_{*}\overline{M}_C^{gp}
$$
\noindent through $\sigma$ mapping the generic point of $T=Spec(R)$ to $W$.\\
Composing with the structure map
$$
f_{*}\overline{M}_C^{gp}{\rightarrow} S
$$
\noindent we get a trait 
$$
\varphi:T{\rightarrow} S
$$
\noindent mapping the generic point to $U$ and the special point to $s$.\\
We put on $T$ the pull back of the log-structure on $S$.\\ In particular the morphism
$(T,M_T){\rightarrow} (S,M_S)$ is strict.\\
Let $\eta$ be the generic point of $T$.
The pullback curve 
$$
(C_T,M_{C_T}){\rightarrow} (T,M_T)
$$
\noindent is log-semistable and generically smooth and the underlying scheme is the pull-back of the underlying scheme of $C$ (see remark \ref{univsat}).\\

Let now $\mathcal{D}$ be the group of divisors on $C_T$ whose support is concentrated in the special fiber of $C_T$ and $\mathcal{D}_0$ be the subgroup of principal divisors.\\

\noindent By \cite{ray} 6.1.3 in our situation over discrete valuation rings there is a bijection
$$
E(T)= \mathcal{D}/\mathcal{D}_0
$$
\noindent 
Furthermore since $f_T:C_T{\rightarrow} T$ is cohomologically flat then the map
$$
E(T){\rightarrow} E(s)
$$
\noindent is bijective (\cite{ray} 6.4.1).\\
It is enough now to find a lift of $\sigma$ to $E(s)$ showing a factorization via the suggested dotted arrow in the next diagram
$$
\xymatrix{
   & & & {E(s)}\ar[dl]\\
   {\sigma}\ar[d]  & { \in H^0(C_s,\overline{M}_{C_s}^{gp})}\ar[r] &{ H^1(C_s,\mathbb{G}_m)} &\\
   {\Sigma}  & {\in H^0(C_T,\overline{M}_{C_T}^{gp})}\ar@{.>}[dr]\ar[u] \ar[r] & {H^1(C_T,\mathbb{G}_m)}\ar[u] & \ar[l]{E(T)}\ar[uu]_{\sim}\\
   &  &  {\mathcal{D}/\mathcal{D}_0} \ar[u]\ar[ur]_{\sim}     & 
}
$$
\noindent Since $C_T$ is quasi-projective then we can represent any cohomology class using \v{C}ech cocycles and there is no need to use hypercoverings by \cite{acech} Corollary 4.1.\\
Given an \'etale cover
$$
\alpha:V=\{V_i\}{\rightarrow} C_T
$$
\noindent let
$$
\check{C}^i(V,F)
$$
\noindent be the sheaves of \v{C}ech chains of a sheaf $F$ associated with the covering. \\
Using the fact that the curve is semistable over $S$, we can always refine with an affine \'etale covering
$$
V_1{\rightarrow} V
$$
\noindent made up of pieces of the form $Spec(A)$ where either
$$
\hat{A}=\widehat{R^{sh}}[[t]]
$$
\noindent or
$$
\hat{A}=\widehat{R^{sh}}[[x,y]]/(xy-v)
$$
\noindent with $T=Spec(R)$ and $v\in \mathfrak{m}_R$ according to the singularities of $C_T$.
\\ 

Observe that the elements $x,y$ in the completed rings can be descended to give local sections for the sheaf $\overline{M}_{V}^{gp}$ over $V_1$ (w.r.t. the \'etale topology).\\

\noindent Since we can use Zariski coverings to compute the class of a line bundle we have that
$$
\delta(\sigma)\in H^1(C_T,\mathbb{G}_m)
$$
\noindent can be trivialized over an affine Zariski covering
$$
\alpha:V{\rightarrow} C_T
$$
\noindent and it can be written as the \v{C}ech class of an element
$$
\gamma\in \Gamma(V,\check{C}^1(V,\mathbb{G}_{m,V}))
$$
\noindent This class is a boundary then after taking an \'etale covering $V^{\prime}{\rightarrow} V$ we can write 
$$
\gamma_{V^{\prime}}= (p_1^{*}\tilde{\sigma})(p_2^{*}\tilde{\sigma})^{-1}
$$
\noindent where 
 
$$
p_i:V^{\prime}\times_V V^{\prime}{\rightarrow} V^{\prime}
$$
\noindent are the canonical projections and
$$
\tilde{\sigma}\in \Gamma(V^{\prime},\check{C}^0(V^{\prime},M_{V^{\prime}}^{gp}))
$$
\noindent is a lift using the surjective morphism
$$
\check{C}^0(V^{\prime},M_{V^{\prime}}^{gp}) {\rightarrow}  \check{C}^0(V^{\prime},\overline{M}_{V^{\prime}}^{gp}){\rightarrow} 0
$$
\noindent induced by the morphism 
\begin{equation}\label{surj:zar}
M_{V}^{gp}{\rightarrow} \overline{M}_{V}^{gp}{\rightarrow} 0
\end{equation}
\noindent which is surjective as \'etale sheaves, and it gives a lift of $\sigma$.\\
By replacing we assume from now on that $V^{\prime}=V$.\\
By construction and the hypothesis that over $\eta$ (the generic point of $T$) we map to the unit section, then the sections $\tilde{\sigma}$ have the property that 
$$
\tilde{\sigma}_{\eta}
$$
\noindent are never zero functions on $V_{\eta}$.\\ 
In this way we may interpret $\tilde{\sigma}$ as rational functions on $V$ whose divisor of zero and poles is concentrated on the special fiber. 
Let 
$$
div( \tilde{\sigma})
$$
\noindent be the associated divisor. \\ 
We claim that precomposing with the morphism
$$
\xymatrix{
    {H^0(C_T,\overline{M}_C^{gp})}\ar[r] & {\Gamma(V,\check{C}^0(V,\overline{M}_V^{gp}))}\\
    {\sigma} \ar[r] & {\alpha^{*}\sigma}
}
$$
\noindent we get that
$$
    [div\big( (\widetilde{\alpha^{*}\sigma})\big)]\in\mathcal{D}/\mathcal{D}_0
$$
\noindent First we want to ``erase the poles''. Let us multiply $\sigma$ by and element 
$$
w\in H^0(C_T,\overline{M}_{C_T})
$$
\noindent such that
$$
0\neq\sigma w\in H^0(C_T,\overline{M}_{C_T})
$$
\noindent By the definition of the log structure we can choose $w$ such that when pulled back to $V$ then locally around a node with complete local ring
$$
\hat{A}=\hat{R}[[x,y ]]/(xy-v) 
$$
\noindent it becomes either of the form $w|_{Spec(\hat{A})}=rx^my^n$ for some $n,m\geq 0$ and $r\in \mathfrak{m}_{\widehat{R}}$ (recall $T=Spec(R)$) or a unit.\\
Outside the nodes the log-structure is induced by pulling back the log-structure on the base hence at smooth points of the special fiber the divisor of $w$ is concentrted on the special fiber unless it is zero.\\
Since being Cartier descends under faithfully flat morphisms (\cite{hol} 4.1) we see that 
$$
div\big(\widetilde{(\alpha^*w)}\big)\in\mathcal{D}/\mathcal{D}_0
$$
\noindent In particular we see that if the claim is true for $\sigma w$ then this is also true for $\sigma$.
Now since we have a map of sheaves in the \'etale topology
$$
M_V{\rightarrow} {\mathcal{O}}_V
$$
\noindent then \'etale locally around a node as before we can think about the function 
$$
\widetilde{\big(\alpha^{*}(\sigma w)\big)}
$$
\noindent as an element of the ring $A$ (where $V=Spec(A)$) such that it becomes a unit in $A\otimes K$ where $K=Frac(R)$ and $T=Spec(R)$, i.e. locally
$$
\widetilde{(\alpha^{*}\sigma w)}_{\eta}\in (A\otimes K)^{\times}
$$
\noindent 
At points where the morphism $V{\rightarrow} S$ is smooth it is not difficult to see that the associated divisor is a Cartier divisor concentrated in the special fiber and gives an element in the analogous of the group $\mathcal{D}/\mathcal{D}_0$ that can be defined for $V$ outside the singular locus. \\
Points where the morphism is not smooth correspond to closed points of the special fiber whose completed local ring is of the form
$$
\hat{A}=\hat{R}[[x,y ]]/(xy -b)
$$
\noindent with $b$ a non zero divisor. \\
In this case we have
$$
\widetilde{(\alpha^{*}\sigma w)}|_{Spec(\hat{A})} = rux^sy^t
$$
\noindent where $u\in \hat{A}^{\times}$, $r\in R$, $s,t\in{\mathbb{N}}$, $st=0$. This can be deduced for example by \cite{th} 2.2.1 or by \cite{hol}.\\ The corresponding divisor is clearly an element of $\mathcal{D}/\mathcal{D}_0$ for $V$.\\

\noindent The function giving this divisor has a descent datum w.r.t. the map $V{\rightarrow} C_T$ and since the structure sheaf is a sheaf in the \'etale topology we get a function on $C_T$ with divisor concentrated in the special fiber.\\
By faithfully flat descent of the Cartier property (\cite{hol} 4.1) we get that the descended divisor of the function
$$
\widetilde{(\alpha^{*}\sigma w)}
$$
\noindent is Cartier and supported on the special fiber.\\
\noindent Hence
$$
\widetilde{(\alpha^{*}\sigma w)}$$
\noindent descends to an element of
$$
\mathcal{D}/\mathcal{D}_0
$$
\noindent The claim now follows and the proof is complete.
\end{proof}

\noindent Using lemma \ref{imagedelta} we finally get (with abuse of notation) a sequence of functors 
\begin{equation}\label{beautifulextension}
    0{\rightarrow} Pic_{C/S}/E\stackrel{i}{\rightarrow} {\underline{\mbox{Pic}}^{log}}{\rightarrow} R^{1}f_{*}\overline{M}_{C}^{gp}{\rightarrow} 0 
\end{equation}
\noindent The previous exact sequence does not change if we compute the derived functors in the fppf topology (\cite{br3} Appendice 11 and \cite{ollog} A.1).
In particular the fppf sheaf ${\underline{\mbox{Pic}}^{log}}$ is an extension of fppf sheaves which are representable by algebraic spaces in groups. 
\\ 

\noindent As consequence of Artin's representability theorem of flat quotients (\cite{aim} 7.3) we have that
$$
{\underline{\mbox{Pic}}^{log}}
$$
\noindent is representable by an algebraic space in groups.\\
This proves  part 1 in theorem \ref{teorema}.\\

\noindent We want now to see that this is enough to have the log-cohomological flatness and to get the separateness property.

\begin{lm}\label{iopen}
  Under the hypothesis of theorem \ref{teorema} the morphism $i$ in \ref{beautifulextension} is an open immersion and $f:(C,M_C){\rightarrow} (S,M_S)$ is log-cohomologically flat in dimension zero.
\end{lm}
\begin{proof}
    Since a smooth map is open and $R^1f_{*}\overline{M}_C^{gp}$ is \'etale, it is enough to check that $i$ is smooth. Observe that both $Q:=Pic_{C/S}/E$ and ${\underline{\mbox{Pic}}^{log}}$ are smooth over $S$.\\
This follows from the (log-)exponential sequence (\ref{expseq}) and the fact that 
$$H^2(C_s,{\mathcal{O}}_{C_s})=0$$
\noindent over fibers $s\in S$.\\
Since the fibers over fields are representable by group schemes we can consider the map induced between connected components of the identity:
$$
i^{0}:Q^0{\rightarrow} ({\underline{\mbox{Pic}}^{log}})^0
$$
\noindent We have seen in corollary \ref{r1zero} that
$$
(R^1f_{*}\overline{M}_{C}^{gp})^0=0
$$
In particular we get an isomorphism
$$
i^{0}:Q^0\cong ({\underline{\mbox{Pic}}^{log}})^0
$$
\noindent Hence the induced map $Q^0{\rightarrow} ({\underline{\mbox{Pic}}^{log}})^0$ is smooth.
By \cite{hol} 7.5 we have that $i$ is also smooth.\\

Since $i$ is an open immersion then for any affine $Spec(A){\rightarrow} S$ the morphism
$$
Lie(Pic_{C/S})(Spec(A))\cong Lie(Q)(Spec(A)){\rightarrow} Lie({\underline{\mbox{Pic}}^{log}})(Spec(A)) 
$$

\noindent is bijective (\cite{llr} Proposition 1.1 c ). \\
Let us describe this map in a special case.

Given a nilpotent surjection $A{\rightarrow} A_0$ of rings over $S$ with kernel $I$, we have the exponential sequence (\cite{olpic} 4.12.1)
\begin{equation}\label{expseq}
0{\rightarrow} {\mathcal{O}}_{C_{A_0}}\otimes I{\rightarrow} M_{C_A}^{gp}{\rightarrow} M_{C_{A_0}}^{gp}{\rightarrow} 0
\end{equation}
Using the exponential sequence for a thickening 
 we can read the morphism on Lie algebras from the diagram
\begin{equation}\label{tangentdiagram}
\xymatrix{
{H^0(C_{A_0},{\mathbb{G}m}_{C_{A_0}})}\ar[d]_{a} \ar[r]&  {H^0(C_{A_0},M_{C_{A_0}}^{gp})}\ar[d]^{b}\\
{H^1(C_{A_0},{\mathcal{O}}_{C_{A_0}}\otimes I)} \ar[d]\ar[r]_{di} &{H^1(C_{A_0},{\mathcal{O}}_{C_{A_0}}\otimes I)}\ar[d]\\
 {H^1(C_{A},{\mathbb{G}m}_{C_{A}})}\ar[d]\ar[r] & {H^1(C_{A},M_{C_{A}}^{gp})}\ar[d] \\
 {H^1(C_{A_0},{\mathbb{G}m}_{C_{A_0}})} \ar[r]&  {H^1(C_{A_0},M_{C_{A_0}}^{gp})}\\
}
\end{equation}
\noindent In particular we have that the module 
$$
H^1(C_{A_0},{\mathcal{O}}_{C_{A_0}}\otimes I)/im(a)
$$
\noindent has to be isomorphic to the module 
$$
H^1(C_{A_0},{\mathcal{O}}_{C_{A_0}}\otimes I)/im(b)
$$
\noindent Remember now that by \cite{ega} III.7.8.6 or \cite{blr} 8.1.8 the functor 
$$
T{\rightarrow} \Gamma(C_T,{\mathcal{O}}_{C_T})
$$
\noindent is represented by a vector bundle $V$ over $S$ if and only if $f:C{\rightarrow} S$ is cohomologically flat in dimension zero. In this case the subfunctor 
$$
\Gamma({\mathcal{O}}_C^{*}):T{\rightarrow} \Gamma(C_T,{\mathcal{O}}_{C_T}^{*})
$$
\noindent is represented by an open subgroup scheme (\cite{blr}8.2, Lemma 10). In particular under our hypothesis on cohomological flatness of the family, the functor $\Gamma({\mathcal{O}}_{C}^{*})$ is smooth and as consequence the morphism $a$ in diagram \ref{tangentdiagram} has to be zero because smoothness implies the absence of obstructions to lift over infinitesimal thikenings. In this way we get an isomorphism 
$$
di:H^1(C_{A_0},{\mathcal{O}}_{C_{A_0}}\otimes I){\rightarrow} H^1(C_{A_0},{\mathcal{O}}_{C_{A_0}}\otimes I)/im(b)
$$
\noindent But this implies that $b$ is the zero map which implies the log-cohomological flatness.\\
\end{proof}
\noindent Since $(R^1f_{*}\overline{M}_C^{gp})^0=0$ ( Corollary \ref{r1zero}), then we have
$$
Q^0\cong ({\underline{\mbox{Pic}}^{log}})^0
$$
\noindent 
Furthermore we have
$$
({\underline{\mbox{Pic}}^{log}})^0\cong Q^0=({\mbox{Pic}}_{C/S}/E)^0\cong ({\mbox{Pic}}_{C/S}^{[0]}/E)^0\cong \mathcal{N}^0
$$
\noindent where the right one is the connected component of the identity of the N\'eron model of ${\mbox{Pic}}_{C_U/U}$ (which exists by \ref{hol:main}). This follows from the proof of 7.2 part 1 in \cite{hol}, the fact that 
$$
E^0\subset ({\mbox{Pic}}_{C/S}^{[0]})^0=({\mbox{Pic}}_{C/S})^0
$$
\noindent and the diagram
$$
\xymatrix{
    {0}\ar[r] & {E^0}\ar[r]\ar[d]^{\cong}& {({\mbox{Pic}}_{C/S}^{[0]})^0}\ar[r]\ar[d]^{\cong} & {(Pic_{C/S}^{[0]}/E)^0}\ar[r]\ar[d] & {0}\\
    {0}\ar[r] & {E^0}\ar[r]& {({\mbox{Pic}}_{C/S})^0}\ar[r] & {(Pic_{C/S}/E)^0}\ar[r] & {0}\\
}
$$
This completes the proof of theorem \ref{teorema}.
\end{proof}
\begin{rmk}
When the dimension of the base scheme $S$ is one then a separated algebraic space in groups is always a scheme by \cite{anan} Th\'eor\`eme 4.B.
In particular in this case $({\underline{\mbox{Pic}}^{log}})^0$ is a separated group scheme over $S$.\\ This was showed with similar techniques in \cite{b14} 5.0.5.
\end{rmk}

\section{Open questions}

\noindent It is natural to ask whether theorem \ref{teorema} can be proved without assuming that the scheme $C$ is regular. Using the explicit theory in \cite{alter} for resolution of singularities for families of curves one could expect to proceed has follows. Start with a family $C{\rightarrow} S$ as in \ref{teorema} without assuming that $C$ is regular. Construct a resolution of singularities $\tilde{C}{\rightarrow} C$ over $S$ obtained by blowing up certain subschemes of $C$ in order to obtain that $\tilde{C}$ is regular.\\
We apply theorem \ref{teorema} to obtain an algebraic space ${\underline{\mbox{Pic}}^{log}}_{\tilde{C}/S}$. By functoriality we get a morphism
$$
{\underline{\mbox{Pic}}^{log}}_{C/S}{\rightarrow} {\underline{\mbox{Pic}}^{log}}_{\tilde{C}/S}
$$
\noindent If this morphism is surjective and we have a good understanding of the fibers we can hope to give a representability result for ${\underline{\mbox{Pic}}^{log}}_{C/S}$.\\
We state this generalization with the following conjecture.
\begin{conj}
Let $(S,M_S)$ be a log-scheme such that $S$ is separated, excellent and regular in codimension 1 and that the log-structure is induced by a reduced divisor obtained as complementary of a schematically dense open, regular subscheme $U\subset S$.\\
Let $f:(C,M_C){\rightarrow} (S,M_S)$ be a special log-curve, whose underlying scheme map is proper and cohomologically flat in dimension zero over $S$, smooth over $U$, and such that the log-structure on $C$ has support the complementary of $f^{-1}U$.\\
There exists an open $U\subset V\subset S$, whose complementary is of codimension at least 2 in $S$, such that
$$
{\underline{\mbox{Pic}}^{log}}_{C_V/V}
$$
\noindent is representable by an algebraic space in groups.
\end{conj}
We give now some motivations about the hypothesis in the conjecture.\\
Since $S$ is excellent the regular locus is open and this open has complementary of codimension at least 2 (by regularity in codimension one).
In particular after shrinking we can assume that $S$ is regular.\\ By the regularity hypothesis on $U$ we still have $U\subset S$.\\
Furthermore after removing closed subsets of codimension 2 we can assume that the complementary of $U$ is strictly normal crossing.\\ 
Using \cite{alter} 3.2 there exists a modification of semistable curves, $h:\tilde{C}{\rightarrow} C$ such that $\tilde{C}$ is regular outside a set of codimension at least 3, $\tilde{C}|_U\cong C|_U$ and $\tilde{C}{\rightarrow} S$ is a split semistable curve.\\
The image of the singular locus of $\tilde{C}$ in $S$ is a closed (by properness) and it has codimension at least $2$.\\
After removing this closed from $S$ we can assume that $\tilde{C}$ is regular.\\
Furthermore outside codimension at least 2 the alignment condition is always satisfied in our situation. Hence we can assume $\tilde{C}$ and $C$ aligned.\\

\noindent Put on $\tilde{C}$ the pullback log-structure from $C$.\\
In this way we get log-alignment. \\
Assume we can prove that on $\tilde{C}$ the log-structure we obtain is also special.
Observe that the locus where a morphism is special is open by \cite{olun} 2.16. 
Hence after shrinking we could assume that the morphisms 
$$
(\tilde{C},M_{\tilde{C}}){\rightarrow} (S,M_S)
$$
\noindent 
is special. The problem with this operation is that we may loose the property that the complementary has codimension at least 2. \\
Anyway let us assume we can show that this does not happen.\\
Using theorem \ref{teorema} we have that
${\underline{\mbox{Pic}}^{log}}_{\tilde{C}/S}$ is representable by a separated algebraic space in groups. We have a pullback morphism
$$
{\underline{\mbox{Pic}}^{log}}_{C/S}\stackrel{h^{*}}{\rightarrow} {\underline{\mbox{Pic}}^{log}}_{\tilde{C}/S}
$$
\noindent At this point we need to understand the fibers of $h^{*}$.\\ In \cite{alter} the morphism $h$ is described by explicit blow-ups. \\
We will investigate this aspect in a future work.\\

\noindent Another interesting question is to investigate when we have an isomorphism
$$
{\underline{\mbox{Pic}}^{log}}_{C/S}\cong {\mbox{Pic}}_{C/S}/E
$$
\noindent and of course what happens in higher dimension or when the family of curves is no more nodal.
\begin{thebibliography}{99}
\bibitem[EGA]{ega} J. Dieudonn\'e, A. Grothendieck -\emph{\'El\'ements de g\'eom\'etrie alg\'ebrique}, Publ. Math. I.H.\'E.S., Nos. \textbf{4},\textbf{8}, \textbf{11}, \textbf{17}, \textbf{20}, \textbf{24}, \textbf{28}, \textbf{32}, (1960-1967)

\bibitem[SGA]{sga} A. Grothendieck et al. - \emph{S\'eminaire de G\'eom\'etrie Alg\'ebrique du Bois Marie}, Lecture Notes in Mathematics, Nos. \textbf{224}, \textbf{151}, \textbf{152}, \textbf{153}, \textbf{269}, \textbf{270}, \textbf{305}, \textbf{569}, \textbf{589}, \textbf{225}, \textbf{288}, \textbf{340}, and Advanced Studies in Pure Mathematics \textbf{2}, (1960-1977)

\bibitem[Anan]{anan} S. Anantharaman - \emph{Sch\'emas en groupes, espaces homog\`enes et espaces alg\'ebriques sur une base de dimension 1}, M\'emoires de la S.M.F., tome \textbf{33}, (1973), 5-79

\bibitem[A1]{aas} M. Artin - \emph{Algebraic approximation of structures over complete local rings}, Publications Math\'ematiques de l'IH\'ES \textbf{36}, (1969), 23-58

\bibitem[A2]{aim} M. Artin - \emph{The implicit function theorem in algebraic geometry}, Algebraic geometry, Papaers presented at the Bobmay Colloquium, pp.13-34. Bombay-Oxford, 1969
\bibitem[A3]{aalg} M. Artin - \emph{Algebraization of formal moduli: I}, in Global Analysis (Papers in honor of K. Kodaira), Univ. of Tokyo Press, Tokyo, (1969), 21-71
\bibitem[A4]{acech} M. Artin - \emph{On the Joins of Hensel Rings}, Adv. in Math. \textbf{7}, (1971), 282-296
\bibitem[A5]{avd} M. Artin - \emph{Versal deformations and Algebraic Stacks}, Inventiones math. \textbf{27}, (1974), 165-189

\bibitem[B14]{b14} A. Bellardini - \emph{On GIT compactified jacobians via Relatively Complete Models and Logarithmic Geometry}, Bonn, Univ., Diss., 2014 URN: urn:nbn:de:hbz:5n-36691
\bibitem[B-L-R]{blr} S. Bosch, W. L\"utkebohmert, Raynaud - \emph{N\'eron Models}, Ergebnisse der Mathematik und ihrer Grenzgebiete \textbf{21}, Springer-Verlag, Berlin, 1990
\bibitem[C-deJ]{cdej} B. Conrad, A.J. de Jong - \emph{Approximation of versal deformations}, Journal of Algebra \textbf{225}, (2002), 489-515

\bibitem[deJ]{gabber} A.J. de Jong - \emph{A result of Gabber}, preprint
\bibitem[deJ96]{alter} A.J. de Jong - \emph{Smoothness, semi-stability and alterations}, I.H.E.S. Publ. Math., \textbf{83} (1996), p. 51-93
\bibitem[F-C]{fc} G. Faltings, C.-L. Chai - \emph{Degeneration of Abelian Varieties},  Ergebnisse der Mathematik und ihrer Grenzgebiete \textbf{22}, Springer-Verlag 1990

\bibitem[Br2]{br2} A. Grothendieck - \emph{Le groupe de Brauer II}, in Dix Exposes sur la Cohomologie des Schemas, Masson \& Cie, North-Holland Publishing Company - Amsterdam (1968)
\bibitem[Br3]{br3} A. Grothendieck - \emph{Le groupe de Brauer III}, in Dix Exposes sur la Cohomologie des Schemas, Masson \& Cie, North-Holland Publishing Company - Amsterdam (1968)

\bibitem[Ho14]{hol} D. Holmes - \emph{N\'eron models of jacobians over base schemes of dimension greater than 1}, arXiv:1402.0647v3 [math.AG]

\bibitem[Kaj]{kaj} T. Kajiwara - \emph{Logarithmic compactifications of the generalized Jacobian variety}, Jour. of the Faculty of Science Tokyo Math. \textbf{40}, 1993, 473-502 

\bibitem[FK]{fka} F. Kato - \emph{Log Smooth Deformation Theory}, T\^{o}hoku Math. J.  \textbf{48}, 1996, 317-354
\bibitem[K]{ka} K. Kato - \emph{Logarithmic structures of Fontaine-Illusie}, in Algebraic analysis, geometry and number theory (J.-I. Igusa, ed.), Johns Hopkins University Press, Baltimore, 1989, 191-224
\bibitem[Kn]{kn} D. Knutson - \emph{Algebraic Spaces}, Lecture Notes in Mathematics, \textbf{203}, Springer, Berlin (1971)

\bibitem[L-L-R]{llr} Q. Liu, D. Lorenzini, M. Raynaud - \emph{N\'eron models, Lie algebras, and reduction of curves of genus one}, Invent. math. \textbf{157}, 2004, 455-518

\bibitem[O-S]{os} T. Oda, C.S. Seshadri - \emph{Compactification of the generalized Jacobian variety}, Trans. Amer. Math. Soc. \textbf{253}, (1979), 1-90

\bibitem[Ol04]{olpic} M.C. Olsson - \emph{Semi-stable degenerations and period spaces for polarized K3 surfaces}, Duke Math. J. \textbf{125} (2004), 121-203

\bibitem[Ol03]{ollog} M.C. Olsson - \emph{Logarithmic geometry and algebraic stacks}, Ann. Sci. d'ENS \textbf{36}, (2003), 747-791

\bibitem[OlU]{olun} M.C. Olsson - \emph{Universal log structures on semi-stable varieties}, T\^{o}hoku Math. Journal \textbf{55}, Number 3, (2003), 397-438

\bibitem[P]{popescu} D. Popescu - \emph{General N\'eron desingularization and approximation}, Nagoya Math. J. \textbf{104}, (1986), 85-115

\bibitem[Ra]{ray} M. Raynaud - \emph{Sp\'ecialization du foncteur de Picard}, Publications Math\'ematiques de l'I.H.\'E.S. \textbf{38}, (1970), 27-76

\bibitem[Th]{th} A. Thuillier - \emph{Potential theory on curves in non-Archimedean geometry. Applications to Arakelov theory}, PhD Thesis, Mathematics. Universit\'e Rennes 1, (2005)
\end{thebibliography}
\end{document}

