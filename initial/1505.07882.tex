\documentclass[10pt, reqno]{amsart}
\usepackage{amsmath, amsthm, amscd, amsfonts, amssymb, graphicx, color, stmaryrd}
\usepackage[all,cmtip]{xy}	
\usepackage[bookmarksnumbered, colorlinks, plainpages]{hyperref}
\usepackage[french, english]{babel}

\usepackage[onehalfspacing]{setspace}

\textheight 24truecm \textwidth 18truecm
\setlength{\oddsidemargin}{-0.2in}\setlength{\evensidemargin}{-0.2in}

\setlength{\topmargin}{-0.5in}
\setlength{\parindent}{0in} 

\newtheorem{Thm}{Theorem}[section]
\newtheorem{Lem}[Thm]{Lemma}
\newtheorem{Prop}[Thm]{Proposition}
\newtheorem{Cor}[Thm]{Corollary}
\theoremstyle{definition}
\newtheorem{Def}[Thm]{Definition}
\newtheorem{Exa}[Thm]{Example}
\newtheorem{Exc}[Thm]{Exercise}
\newtheorem{Conclusion}[Thm]{Conclusion}
\newtheorem{Conjecture}[Thm]{Conjecture}
\newtheorem{Criterion}[Thm]{Criterion}
\newtheorem{Summary}[Thm]{Summary}
\newtheorem{Axiom}[Thm]{Axiom}
\newtheorem{problem}[Thm]{Problem}
\newtheorem*{acknowledgement}{Acknowledgement}
\newtheorem{Rem}[Thm]{Remark}
\newtheorem{Ass}[Thm]{Assumption}
\newtheorem{Not}[Thm]{Notation}

\makeatletter
\newenvironment{abstracts}{  \ifx\maketitle\relax
    \ClassWarning{\@classname}{Abstract should precede
      \protect\maketitle\space in AMS document classes; reported}  \fi
  \global\setbox\abstractbox=\vtop \bgroup
    \normalfont\Small
    \list{}{\labelwidth\z@
      \leftmargin3pc \rightmargin\leftmargin
      \listparindent\normalparindent \itemindent\z@
      \parsep\z@ \@plus\p@
      \let\fullwidthdisplay\relax
      \itemsep\medskipamount
    }}{  \endlist\egroup
  \ifx\@setabstract\relax \@setabstracta \fi
}

\makeatother

\begin{document}
\setcounter{page}{1}

\title{Boutet de Monvel operators on singular manifolds}

\author[Karsten Bohlen]{Karsten Bohlen}

\address{$^{1}$ Leibniz University Hannover, Germany}
\email{\textcolor[rgb]{0.00,0.00,0.84}{bohlen.karsten@math.uni-hannover.de}}

\subjclass[2000]{Primary 58J32; Secondary 58B34.}

\keywords{Boutet de Monvel's calculus, groupoids, Lie manifolds.}

\begin{abstracts}
{  \otherlanguage{{english}}  \item[\hskip\labelsep\scshape\abstractname.]}
We construct a Boutet de Monvel calculus for general pseudodifferential boundary value problems defined on a broad class of singular manifolds, so-called 
Lie manifolds with boundary.

{  \otherlanguage{{french}}  \item[\hskip\labelsep\scshape\abstractname.]}
Nous construisons une calcul des type Boutet de Monvel pour des probl\`emes de valeurs au bord pseudodiff\'erentiels defin\'es sur une large classe de vari\'et\'es singuli\'eres, des
vari\'et\'es de Lie \`a bord.
\end{abstracts}

\maketitle

\section{Introduction}

The analysis on singular manifolds has a long history, and the subject is to a large degree motivated by the study of partial differential
equations (with or without boundary conditions) and generalizations of index theory to the singular setting, e.g. Atiyah-Singer type index theorems.
One particular approach is based on the observation first made by A. Connes (cf. \cite{C}) that groupoids are good models for singular spaces.
The outlines for a pseudodifferential calculus on singular foliations were made precise for the general case of (longitudinally smooth) groupoids later by B. Monthubert, V. Nistor, A. Weinstein and P. Xu; see e.g. \cite{NWX}. 
Such a calculus can be defined on manifolds with singularities of various types in a unified and general setting.
A notion of global ellipticity with Fredholm conditions is also possible. 
On the other hand it is important for applications in the study of PDE's to pose boundary conditions and a parametrix for general boundary value problems.
It is a hard problem to incorporate boundary conditions in the singular context. The construction of a refined parametrix, with
appropriate Fredholm conditions, in such a setting is particularly difficult for the technical reason that the small calculus is not sufficient.
In this work we will develop a theory of general pseudodifferential boundary value problems for the groupoid calculus.

The pseudodifferential calculus on a Lie manifold was constructed in \cite{ALN} via representations of pseudodifferential operators
on a Lie groupoid. This representation also yields closedness under composition.

In our case we consider the following data: a Lie manifold $(X, \V)$ with boundary $Y$ which is an embedded, transversal hypersurface $Y \subset X$ 
and which is a Lie submanifold of $X$ (cf. \cite{ALN}, \cite{AIN}). 

We will describe a general calculus with pseudodifferential boundary conditions on the Lie manifold with boundary $(X, Y, \V)$. 
The future goal is a generalization of the index theorem of Boutet de Monvel (\cite{BM}) to the singular setting.

\subsection*{Acknowledgements}

For helpful discussions and remarks I thank Magnus Goffeng, Victor Nistor, Julie Rowlett, Elmar Schrohe and Georges Skandalis. 
I thank the Deutsche Forschungsgemeinschaft (DFG) for their financial support.

\section{Boutet de Monvel's calculus}

Boutet de Monvel's calculus (e.g. \cite{BM}) was established in 1971. For a detailed account we refer the reader to the book \cite{G}. 
This calculus provides a convenient and general tool to study the classical boundary value problems.
Let $X$ be a smooth compact manifold with boundary. 
An operator of order $m \leq 0$ and type $0$ in Boutet de Monvel's calculus is a matrix

\[
A = \begin{pmatrix} P_{+} + G & K \\
T & S \end{pmatrix} \colon \begin{matrix} C^{\infty}(X, E_1) \\ \oplus \\ C^{\infty}(\partial X, F_1) \end{matrix} \to \begin{matrix} C^{\infty}(X, E_2) \\ \oplus \\ C^{\infty}(\partial X, F_2) \end{matrix} \in \B^{m,0}(X, \partial X).
\]

Here $P \in \Psi_{tr}^m(M)$ is pseudodifferential operator (\cite{G}, p. 20 (1.2.4)) with transmission property defined
on a suitable smooth neighborhood $M$ of $X$ where $P_{+}$ means $P = r^{+} P e^{+}$ the truncation. The transmission property (\cite{G}, p. 23, (1.2.6)) ensures that $P_{+}$ maps functions smooth
up to the boundary to functions which are smooth up to the boundary.
Additionally, $G \colon C^{\infty}(X) \to C^{\infty}(X)$ is a singular Green operator (\cite{G}, p. 30), $K \colon C^{\infty}(\partial X) \to C^{\infty}(X)$ is a potential operator (\cite{G}, p. 29) and $T \colon C^{\infty}(X) \to C^{\infty}(\partial X)$ is a trace operator (\cite{G}, p. 27) 
We also have a pseudodifferential operator on the boundary $S \in \Psi^m(\partial X)$. 

\textbf{The calculus has the following \emph{features:}} 
\begin{itemize}
\item If the bundles match, i.e. if $E_1 = E_2 = E, \ F_1 = F_2 = F$ the calculus is \textbf{\emph{closed under composition}}. 

\item If $F_1 = 0, \ G = 0$ and $K, \ S$ are not present, we obtain a \textbf{\emph{classical BVP}}, e.g. the Dirichlet problem.

\item If $F_2 = 0$ and $T, \ S$ are not present, the calculus \textbf{\emph{contains inverses of classical BVP's}} whenever they exist. 
\end{itemize}

The proof that two Boutet de Monvel operators composed are again of this type is technical, see e.g. \cite{G}, chapter 2. 
Additionally, the symbolic structure of the operators involves more complicated behaviour than that of merely pseudodifferential operators.

\section{Lie manifolds with boundary}

In this section we will consider the general setup for the analysis on singular and non-compact manifolds.

\begin{Exa}
On a compact manifold with boundary $M$ we will introduce an alternative metric which is no longer smooth everywhere.
This metric models the singular structure where the manifold with boundary is viewed as a compactification
of a non-compact manifold with cylindrical end.
Precisely, the metric is a product metric $g = g_{\partial M} + dt^2$ in a tubular neighborhood of the boundary (or the far end of
the cylinder). 
The cylindrical end is mapped to a tubular neighborhood of the boundary via the Kondratiev transform $r = e^t$ based on \cite{Kon}. 
Assume that we are given a tubular neighborhood of the form $[0, \epsilon) \times \partial M$ and let $(r, x') \in [0,\epsilon) \times \partial M$
be local coordinates.
The $b$-\emph{differential operators} take the form for $n = \dim(M)$
\begin{align}
P &= \sum_{|\alpha| \leq m} a_{\alpha}(r, x') (r \partial_r)^{\alpha_1} \partial_{x_2'}^{\alpha_2} \cdots \partial_{x_n'}^{\alpha_n} = \sum_{|\alpha| \leq m} a_{\alpha} (r \partial_r)^{\alpha_1} \partial^{\alpha'}. \tag{$*$} \label{1}
\end{align}

We observe that the vector fields, which are \emph{local generators}, in this example are 
\[
\{r \partial_r, \partial_{x_2}, \cdots, \partial_{x_n}\}.
\] 

We can define a locally finitely generated module of vector fields $\V_b$ that has local generators as defined in our example.
These can alternatively be characterized as all those vector fields that are tangent
to the boundary $\partial M$. 
An operator of order $m$ in the universal enveloping algebra $P \in \Diff_{\V_b}^m(M)$ is locally written as in \eqref{1}. 
Since $\V_b$ is a locally finitely generated and projective $C^{\infty}(M)$-module we obtain a vector bundle $\A_b \to M$ such that the smooth sections identify $\Gamma(\A_b) \cong \V_b$ by the Serre-Swan theorem. On $\A_b$ we have a structure of a \emph{Lie algebroid} with anchor $\varrho \colon \A_b \to TM$ (see \cite{ALN} for further details).
\label{Exa:Kontradiev}
\end{Exa}

There are sub Lie algebras of $\V_b$ constituting so-called \emph{Lie structures} which model different types of singular structures on a manifold, see also \cite{AIN}, \cite{ALN}, \cite{ALNV}.
In this general setup $M$ is a compact manifold \emph{with corners} (generalizing Example \ref{Exa:Kontradiev}) viewed as the compactification endowed with a Riemannian metric. 
The Riemannian metric is of product type in a tubular neighborhood of the singular hyperfaces. The topological structure of $M$ is such that $M$ has a finite number of embedded (intersecting) codimension one hypersurfaces. 
Such a manifold with corners can be locally modelled on sets of the type $(0,1)^k \times \Rr^{n-k}$, where $k$ is the codimension.
We can require the transition maps to be smooth and obtain a smooth structure on $M$. In our setup we will consider such a Lie manifold $X$ with an additional hypersurface (denoted $Y$ below) which is \emph{transversal}.
Hence $Y$ is allowed to intersect the singular strata (at infinity) of $X$ as long as this intersection does not
occur in a corner (where two singular strata meet).

\begin{Def}[\cite{ALN}, Def. 1.1]
A \emph{Lie manifold} $(X, \A^{\pm})$ consists of the following data.

\emph{i)} A compact manifold with corners $X$. 

\emph{ii)} A Lie algebroid $(\A^{\pm}, \varrho_{\pm})$ with projection map $\pi_{\pm} \colon \A^{\pm} \to X$. 

\emph{iii)} The module of vector fields $\V_{\pm} = \Gamma(\A^{\pm})$ is a locally finitely generated, projective
$C^{\infty}(X)$-module. 
\end{Def}

\begin{Def}[\cite{AIN}, Def. 2.1, Def. 2.5]
A \emph{Lie manifold with boundary} $(X, Y, \A^{\pm})$ consists of the following data.

\emph{i)} A Lie manifold $(X, \A^{\pm})$. 

\emph{ii)} An embedded codimension one submanifold with corners $Y \hookrightarrow X$. 

\emph{iii)} There is a Lie algebroid $(\A_{\partial}, \varrho_{\partial})$ on $Y$ with projection map
$\pi_{\partial} \colon \A_{\partial} \to Y$ such that $\A_{\partial}$ is a Lie subalgebroid of $\A^{\pm}$. 

\emph{iv)} The submanifold $Y$ is \emph{transversal}, i.e. 
\[
\varrho_{\pm}(\A_y) + T_y Y = T_y X, \ y \in \partial Y. 
\]

\emph{v)} The interior $(X_0, Y_0)$ is diffeomorphic to a smooth manifold with boundary. 
\end{Def}

\begin{Rem}
\emph{i)} In \cite{AIN} the authors define a Lie manifold with boundary $(X, Y, \V_{\pm})$ and also the \emph{double} of a given Lie manifold with boundary.
We denote this double by $M = 2X$ which is a Lie manifold $(M, \V)$.

The \emph{Lie structure} $\V$ is defined such that
\[
\V_{\pm} = \{V_{|X_{\pm}} : V \in \V\}.
\]

We obtain a Lie manifold $(M, \A)$ with the Lie algebroid $(\A, \varrho)$. 

\emph{ii)} We set $\W = \Gamma(\A_{\partial})$ for the \emph{Lie structure} of $Y$. Using \emph{iii)} and \emph{iv)}
of the definition we obtain
\begin{align*}
\W &= \{V \in \Gamma(Y, \A_{|Y}) : \varrho \circ V \in \Gamma(Y, TY)\} \\
&= \{V_{|Y} : V \in \V, \ V_{|Y} \ \text{tangent to} \ Y\}. 
\end{align*}

\label{Rem:double}
\end{Rem}

\section{Quantization}

In this section we will describe the quantization of H\"ormander symbols defined on the conormal bundles.
We will restrict ourselves to the case of trace operators. The other cases are defined analogously.

\textbf{\emph{We fix the following data:}} 
\begin{itemize}
\item A Lie manifold with boundary $(X, Y, \V)$ and the double $M = 2X$ of $X$, endowed with Lie structure $2 \V$. Fix a Lie algebroid $(\pi \colon \A \to M, \ \varrho_M)$ such that $\Gamma(\A) = 2\V$. The hypersurface $Y$ is endowed with the Lie structure $\W$ as defined in \ref{Rem:double}.
Furthermore, fix the vector bundle $(\pi_{\partial} \colon \A_{\partial} \to Y, \ \varrho_{\partial})$ with $\Gamma(\A_{\partial}) = \W$.  

\item We fix the \emph{normal bundles} $\A_{|Y} / \A_{\partial} =: \N \to Y$ as well as\footnote{We denote by $\Delta_Y$ the diagonal in $Y \times Y$ being understood as a submanifold of $Y \times M, \ M \times Y$ and $M \times M$, groupoids $\G$, $\G_{\partial}$ and spaces $\X$, $\Xop$ as defined in \cite{B2}.}
\[
\N^{\X} \Delta_Y \to Y, \ \N^{\X^t} \Delta_Y \to Y, \ \N^{\G} \Delta_Y \to Y
\]
which are used to quantize pseudodifferential, trace, potential and singular Green operators respectively. 
\end{itemize}

The notation is reminiscent of the underlying geometry which is described using groupoids and groupoid correspondences \cite{B2}.
We will keep using this notation, though we remark that there are (non-canonical) isomorphisms
\begin{align*}
& \N^{\X} \Delta_Y \cong \A_{\partial} \times \N, \ \N^{\Xop} \Delta_{Y} \cong \N \times \A_{\partial} \ \text{and} \ \N^{\G} \Delta_Y \cong \A_{|Y} \times \N. 
\end{align*}

\begin{Rem}
\emph{i)} On the singular normal bundles we define the H\"ormander symbols spaces $S^m(\N^{\X} \Delta_Y^{\ast}) \subset C^{\infty}(\N^{\X} \Delta_Y^{\ast})$ as in \cite{HIII}, Thm 18.2.11.

\emph{ii)} Define the \emph{inverse fiberwise Fourier transform} 
\[
\Ff^{-1}(\varphi)(\zeta) = \int_{\overline{\pi}(\zeta) = \pi(\xi)} e^{i \scal{\xi}{\zeta}} \varphi(\xi) \,d\xi, \ \varphi \in S(\N^{\X} \Delta_Y^{\ast}).
\]

Here we use the notation $S(\N^{\X} \Delta_Y^{\ast})$ for the space of rapidly decreasing functions on the conormal bundle, see also \cite{S}, Chapter 1.5. 

The spaces of conormal distributions are defined as:
\[
I^{m}(\N^{\X} \Delta_Y, \Delta_Y) := \Ff^{-1} S^m(\N^{\X} \Delta_Y^{\ast})
\]

and $I^{m}(\N^{\Xop} \Delta_Y, Y), \ I^{m}(\N^{\G} \Delta_Y, Y)$ analogously.

\label{Rem:fwise}
\end{Rem}

On a Lie manifold the injectivity radius is positive, see \cite{ALN2}, Thm. 4.14. 
Let $r$ be smaller than the injectivity radius and write 
\[
(\N^{\X} \Delta_Y)_r = \{v \in \N^{\X} \Delta_Y : \|v\| < r\}
\]

as well as
\[
I_{(r)}^m(\N^{\X} \Delta_Y, \Delta_Y) = I^m((\N^{\X} \Delta_Y)_r, \Delta_Y).
\]

Fix the restriction 
\[
\R \colon I_{(r)}^m(\N^{\X} \Delta_Y, \Delta_Y) \to I_{(r)}^m(N^{Y_0 \times M_0} \Delta_{Y_0}, \Delta_{Y_0}).
\]

Additionally, denote by $\J_{tr}$ the action of a conormal distribution (its induced linear operator).

We denote by $\Psi$ the normal fibration of the inclusion $\Delta_{Y_0} \hookrightarrow Y_0 \times M_0$
such that $\Psi$ is the local diffeomorphism mapping an open neighborhood of the zero section $O_{Y_0} \subset V \subset N^{Y_0 \times M_0} \Delta_{Y_0}$
onto an open neighborhood $\Delta_{Y_0} \subset U \subset Y_0 \times M_0$ (cf. \cite{S}, Thm. 4.1.1). 
Then we have the induced map on conormal distributions
\[
\Psi_{\ast} \colon I_{(r)}^m(N^{Y_0 \times M_0} \Delta_{Y_0}, \Delta_{Y_0}) \to I^m(Y_0 \times M_0, \Delta_{Y_0}).
\]

Also let $\chi \in C_c^{\infty}(\N^{\X} \Delta_Y)$ be a cutoff function which acts by multiplication 
\[
I^m(\N^{\X} \Delta_Y, \Delta_Y) \to I_{(r)}^{m}(\N^{\X} \Delta_Y, \Delta_Y).
\]

\begin{Def}[Quantization]
Define
\[
q_{T, \chi} \colon S^m(\N^{\X} \Delta_Y^{\ast}) \to \Trace^{m,0}(M, Y) 
\]

such that for $t \in S^m(\N^{\X} \Delta_Y^{\ast})$ we have
\[
q_{T, \chi}(t) = \J_{tr} \circ q_{\Psi, \chi}(t)
\]

where
\[
q_{\Psi, \chi}(t) = \Psi_{\ast}(\R(\chi \Ff^{-1}(t))).
\]
\end{Def}

From the compactness of $M$ we can associate to each vector field in $2 \V$ a \emph{global flow}
\begin{align*}
& 2\V \ni V \mapsto \Phi_V \colon \Rr \times M \to M.
\end{align*}

Then consider the diffeomorphism evaluated at time $t = 1$
\[
\Phi(1, -) \colon M \to M
\]

and fix the corresponding group actions on functions which we denote by
\begin{align*}
& 2\V \ni V \mapsto \varphi_V \colon C^{\infty}(M) \to C^{\infty}(M).
\end{align*}

\begin{Def}
The class of $\V$-trace operators is defined as
\[
\Trace_{2\V}^{m,0}(M, Y) := \Trace^{m,0}(M, Y) + \Trace_{2\V}^{-\infty, 0}(M, Y).
\]

Here $\Trace^{m,0}(M, Y)$ consists of the extended operators from the previous definition.
The residual class is defined as follows
\begin{align*}
& \Trace_{2\V}^{-\infty, 0}(M, Y) := \mathrm{span}\{q_{\chi, T}(t) \varphi_{V_1} \cdots \varphi_{V_k} : V_j \in 2\V, \ \chi \in C_c^{\infty}(\N^{\X} \Delta_Y), \ t \in S^{-\infty}(\N^{\X} \Delta_Y^{\ast})\}.
\end{align*}
\end{Def}

We will henceforth denote by $\B_{2\V}^{m,0}(M, Y)$ the class of extended Boutet de Monvel operators which consist of
matrices of operators $\begin{pmatrix} P + G & K \\ T & S \end{pmatrix}$. The components are given via the fibrations on the 
appropriate normal bundles.

\section{Compositions and Parametrices}

To prove closedness under composition we require to assume that a groupoid $\G$ which integrates the Lie structure on $M$ and a
groupoid $\G_{\partial}$ which integrates the Lie structure on $Y$ are chosen in the following sense.
Precisely, we construct for the given Lie structures on $M$ and $Y$ respectively integrating groupoids $\G$ and $\G_{\partial}$ as well as a morphism from $\G \to \G_{\partial}$ and a morphism from $\G_{\partial} \to \G$ 
in the category of Lie groupoids.
These morphisms are described using \emph{correspondences} of groupoids, see \cite{MO}.
\begin{Exa}
Consider the example of the algebroids $\A = TM, \ \A_{\partial} = TY$, i.e. the Lie structures consisting of \emph{all vector fields}.  
The pair groupoids $M \times M \rightrightarrows M, \ Y \times Y \rightrightarrows Y$ and also the \emph{path groupoids} (see \cite{LN}, example 2.9) $\P_M \rightrightarrows M, \ \P_Y \rightrightarrows Y$ integrate these algebroids. 
\label{Exa:corr}
\end{Exa}
For several Lie structures, groupoids and correspondences with good geometry exist, e.g. the Lie structure of $b$-vector fields or the structure of fibered cusp vector fields \cite{B}, \cite{B2}.
The following results hold for Lie structures of this type.

\begin{Thm}
The class of extended Boutet de Monvel operators $\B_{2\V}^{0,0}(M, Y)$ is closed under composition and adjoint, hence
$\B_{2\V}^{0,0}(M, Y)$ forms an associative $\ast$-algebra. 
\label{Thm:closed1} 
\end{Thm}

We define the class of \emph{truncated} Boutet de Monvel operators as follows.
The restriction $r^+$ to the interior $\mathring{X}_0 := X_0 \setminus Y_0$ and the extension by zero operator $e^+$ are given on the manifold level by
\[
\xymatrix{
L^2(M_0) \ar@/^1pc/[r]^{r^{+}} & \ar@/-0pc/[l]^-{e^{+}} L^2(\mathring{X}_0) 
} 
\]

with $r^{+} e^{+} = \id_{L^2(\mathring{X}_0)}$ and $e^{+} r^{+}$ being a projection onto a subspace of $L^2(M_0)$.
We define
\[
\End\begin{pmatrix} C_c^{\infty}(M_0) \\ \oplus \\ C_c^{\infty}(Y_0) \end{pmatrix} \supset \B_{2\V}^{m,0}(M, Y) \ni A = \begin{pmatrix} P + G & K \\ T & S \end{pmatrix} \mapsto \C(A) = \begin{pmatrix} r^{+} (P + G) e^{+} & r^{+} K \\ T e^{+} & S \end{pmatrix} \in \End\begin{pmatrix} C_c^{\infty}(X_0) \\ \oplus \\ C_c^{\infty}(Y_0) \end{pmatrix}.
\]

\begin{Def}
The class of \emph{truncated operators} is for $m \leq 0$ defined as
\[
\B_{\V}^{m,0}(X, Y) := \C \circ \B_{2\V}^{m,0}(M, Y).
\]
\label{Def:truncated}
\end{Def}

To show closedness under composition we use the longitudinally smooth structure of the integrating groupoids
as well as the previous Theorem.
This enables us to state the second main result.

\begin{Thm}
The calculus $\B_{\V}^{0,0}(X, Y)$ is closed under composition and adjoint. 
\end{Thm}

A priori, the inverse of an invertible Boutet de Monvel operator will not be contained in our calculus due to the definition via compactly supported distributional kernels. 
We define a completition $\overline{\B}_{\V}^{-\infty,0}(X, Y)$ of the residual Boutet de Monvel operators with regard to the family of norms
of operators $\L\left(\begin{matrix} H_{\V}^t(X) \\ \oplus \\ H_{\W}^t(Y) \end{matrix}, \ \begin{matrix} H_{\V}^r(X) \\ \oplus \\ H_{\W}^r(Y) \end{matrix}\right)$ on Sobolev spaces, cf. \cite{ALNV}.
Define the completed algebra of Boutet de Monvel operators as
\[
\overline{\B}_{\V}^{0,0}(X, Y) = \B_{\V}^{0,0}(X, Y) + \overline{\B}_{\V}^{-\infty, 0}(X, Y).
\]
The resulting algebra contains inverses and has favorable algebraic properties, e.g. it is spectrally invariant, \cite{B2}. 
We obtain a parametrix construction after defining a notion of \emph{Shapiro-Lopatinski ellipticity}.
The indicial symbol $\R_{F}$ of an operator $A$ on $X$ is an operator $\R_{F}(A)$ defined as the restriction to a singular hyperface $F \subset X$ (see \cite{ALN}). 
Note that if $F$ intersects the boundary $Y$ non-trivially we obtain in this way a non-trivial Boutet de Monvel
operator $\R_F(A)$ defined on the Lie manifold $F$ with boundary $F \cap Y$. 

\begin{Def}
\emph{i)} We say that $A \in \overline{\B}_{\V}^{0,0}(X, Y)$ is \emph{$\V$-elliptic} if the principal symbol $\sigma(A)$ and the principal boundary symbol $\sigma_{\partial}(A)$ are 
both pointwise invertible.

\emph{ii)} A $\V$-elliptic operator $A$ is \emph{elliptic} if $\R_F(A)$ is pointwise invertible for each hyperface $F \subset X$. 
\label{Def:elliptic}
\end{Def}

\begin{Thm}
\emph{i)} Let $A \in \overline{\B}_{\V}^{0,0}(X, Y)$ be $\V$-elliptic. There is a parametrix $B \in \overline{\B}_{\V}^{0,0}(X, Y)$ of $A$, in the sense
\[
I - AB \in \overline{\B}_{\V}^{-\infty, 0}(X, Y), \ I - BA \in \overline{\B}_{\V}^{-\infty, 0}(X, Y). 
\]

\emph{ii)} Let $A \in \overline{\B}_{\V}^{0,0}(X, Y)$ be elliptic. There is a parametrix $B \in \overline{\B}_{\V}^{0,0}(X, Y)$ of $A$ up to 
compact operators
\[
I - AB \in \K\begin{pmatrix} L_{\V}^2(X) \\ \oplus \\ L_{\W}^2(Y) \end{pmatrix}, \ I - BA \in \K\begin{pmatrix} L_{\V}^2(X) \\ \oplus \\ L_{\W}^2(Y) \end{pmatrix}. 
\]

\label{Thm:parametrix}
\end{Thm}

{ \small {
\begin{thebibliography}{99}
\bibitem[1]{AIN} B. Ammann, A. Ionescu, V. Nistor, \emph{Sobolev spaces on Lie manifolds and regularity for polyhedral domains}, Doc. Math. 11, 161-206 (2006) 
\bibitem[2]{ALN} B. Ammann, R. Lauter, V. Nistor, \emph{Pseudodifferential operators on manifolds with a Lie structure at infinity}, Ann. of Math. 165, 717-747 (2007)
\bibitem[3]{ALN2} B. Ammann, R. Lauter, V. Nistor, \emph{On the geometry of Riemannian manifolds with a Lie structure at infinity}, Int. J. Math. and Math. Sciences 4: 161-193
\bibitem[3]{ALNV} B. Ammann, R. Lauter, V. Nistor, A. Vasy, \emph{Complex powers and non-compact manifolds}, Comm. Part. Diff. Eq. 29, no. 5/6 671-705 (2004)
\bibitem[4]{B} K. Bohlen, \emph{Boutet de Monvel operators on Lie manifolds with boundary}, in preparation
\bibitem[5]{B2} K. Bohlen, \emph{Boutet de Monvel's calculus via groupoid actions}, PhD thesis, Hannover 2015
\bibitem[6]{BM} L. Boutet de Monvel, \emph{Boundary problems for pseudo-differential operators}, Acta Math., 126(1-2):11-51, 1971
\bibitem[7]{C} A. Connes, \emph{Noncommutative Geometry}, Academic Press, 1994
\bibitem[8]{G} G. Grubb, \emph{Functional Calculus of Pseudodifferential Boundary Problems}, Birkhäuser; 2nd edition, 1986
\bibitem[9]{HIII} L. H\"ormander, \emph{The Analysis of Linear Partial Differential Operators III}, Springer-Verlag, Berlin Heidelberg, 1985
\bibitem[10]{Kon} V. A. Kondratiev, \emph{Boundary value problems for elliptic equations in domains with conical or angular points}, Transl. Moscow Math. Soc., 16:227-313, 1967
\bibitem[11]{LN} R. Lauter, V. Nistor \emph{Analysis of geometric operators on open manifolds: A groupoid approach}, Progress in Mathematics, Vol. 198, 2001, pp 181-229
\bibitem[12]{MO} M. M. Stadler, M. O'Ouchi, \emph{Correspondence of groupoid $C^{\ast}$-algebras}, J. Operator Theory, 42(1999) 103-119
\bibitem[13]{NWX} V. Nistor, A. Weinstein, P. Xu, \emph{Pseudodifferential Operators on Differential Groupoids}, Pacific J. Math. 189 (1999), 117-152
\bibitem[14]{S} S. R. Simanca, \emph{Pseudo-differential Operators}, Pitman Research Notes in Mathematics 236, 1990
\end{thebibliography}
}}

\end{document}

