\begin{document}

\maketitle

\begin{abstract}
A powerful approach to solve the Coulombic quantum three-body problem
is proposed. The approach is exponentially convergent and more efficient
than the Hyperspherical Coordinate(HC) method and the Correlation Function
Hyperspherical Harmonic(CFHH) method. This approach is numerically competitive
with the variational methods, such as that using the Hylleraas-type basis
functions. Numerical comparisons are made to demonstrate them, by calculating
the non-relativistic \& infinite-nuclear-mass
limit of the ground state energy of
the helium atom. The exponentially convergency of this approach is due to the
full matching between the analytical structure of the basis functions that
I use and the true wave function. This full matching was not reached by almost
any other methods. For example, the variational method using the Hylleraas-type
basis does not reflects the logarithmic singularity of the true wave function
at the origin as predicted by Bartlett and Fock. Two important approaches
are proposed in this work to reach this full matching: the coordinate
transformation method and the asymptotic series method. Besides these, this
work makes use of the least square method to substitute complicated numerical
integrations in solving the Schr\"{o}dinger equation, without much loss of
accuracy; this method is routinely used by people to fit a theoretical curve
with discrete experimental data, but I use it here to simplify the computation.

\vspace{3mm}
{\noindent
PACS number(s):}

\end{abstract}

\vspace{5mm}

\section{INTRODUCTION}

Most approximate methods to solve a linear partial differential
equation, such as the stationary state Schr\"{o}dinger equation,
are actually to choose an
EQIX23Q-dimensional subspace of the infinite-dimensional Hilbert space and
then to reduce the partial differential equation to EQIX23Q linear algebraic
equations defined in this subspace. The efficiency of this kind of methods
is mainly determined by whether one can use sufficient small EQIX23Q to reach
sufficient high accuracy, i.e., make the vector most close
to the true solution in this subspace sufficiently close to the true
solution while keeping the dimension EQIX23Q not too large to handle.

Most methods to solve the Coulombic quantum three-body problem belong to
this class, except for some variational methods that make use of some
non-linear
variational parameters. The differences between different methods of this kind
mainly lie in different choices of the subspaces of the Hilbert space,
i.e., different choices of
the basis functions to expand the wave function.

Theoretically, any discrete
and complete set of basis functions may be used to expand the wave function,
and the convergency is easy to fulfilled. But actually, the convergency is
often slow and makes sufficient accuracy difficult to achieve. The naive
hyperspherical harmonic function method[1-3] in solving the Coulombic quantum
three-body problem is such an example--this slow convergency can be illustrated
by an analogous and simple example: to expand the function EQIX325Q
(EQIX326Q) as a series of the Legendre polynomials of EQIX327Q. This series
is convergent like EQIX328Q, where EQIX329Q is a positive constant not large and EQIX23Q is
the number of Legendre polynomials involved. The reason for this slow
convergency is that EQIX330Q is singular at EQIX331Q but the Legendre
polynomials of EQIX327Q are not. I call this the mismatching between the analytical
structures of the basis functions (the polynomials of EQIX327Q) and EQIX330Q.

The correlation function hyperspherical harmonic(CFHH) method[4] were proposed to
overcome this difficulty. The spirit of this method can be simply illustrated,
still using the above example: to divide EQIX330Q by an appropriately selected
function(called the correlation function) to cancel the low order singularities
of EQIX330Q at EQIX331Q, then to expand the remaining function by the
Legendre polynomials of EQIX327Q. This time, the series is still convergent as
EQIX328Q, but EQIX329Q is increased by an amount depending on how many orders'
singularities have been canceled.

From this simple discussion one can see that the singularities of the function
EQIX330Q are not completely canceled by the correlation function, although more
sophisticated correlation function can cancel more orders' singularities.

A very simple approach to \emph{totally} eliminate the singularity is
to make an appropriate coordinate transformation,
and in the same time thoroughly give up the original hyperspherical
harmonic function method, not just repair it.
For example, for EQIX325Q, one may write EQIX332Q, where
EQIX333Q, then EQIX334Q and one can expand EQIX330Q
as the series about the Legendre polynomials of EQIX335Q.
This time the series is \emph{factorially} convergent. The reason is that
the analytical structures of EQIX330Q and EQIX336Q match--they are
both analytical functions on the whole complex plane of EQIX337Q.

Another useful approach to solve this problem is to use the asymptotic series.
Still considering the example EQIX325Q, one may write the Taylor
series
 EQDS95Q 
Of course, this series is slowly convergent near EQIX331Q. But one can use the
following asymptotic series to calculate EQIX338Q when EQIX187Q is large:
 EQDS96Q 
or, equivalently,
 EQDS97Q 
where EQIX339Q,
and EQIX340Q. For a given EQIX341Q, the error of
this formula is minimized
when EQIX342Q, and the minimum error is about 
EQIX343Q, which exponentially decreases with EQIX187Q
increasing. Using such kind of asymptotic formulae to calculate the high order
coefficients of the Taylor series, one can expand the singular function EQIX330Q
at high precision, with only finite linear parameters, EQIX344Q and
EQIX345Q.

Now I introduce an alternative approach to reduce a differential equation to
a given finite dimensional subspace EQIX346Q of tbe Hilbert space.
Here EQIX23Q is the dimension of the subspace. The central problem is how
to reduce an operator EQIX347Q in the Hilbert
space, e.g., the kinetic energy operator or the potential energy operator,
to an EQIX348Q matrix in the given subspace. For a state
EQIX349Q, the state EQIX350Q usually
EQIX351Q. To reduce EQIX347Q into an EQIX348Q matrix means to
find a state EQIX352Q to approximate EQIX353Q. The usual approach
to select EQIX138Q is to minimize
 EQDS98Q 
where EQIX354Q is the
innerproduct of the Hilbert space. This approach will reduce EQIX347Q to a matrix
with elements
 EQDS99Q 
where EQIX355Q is a set of orthonormal basis in EQIX346Q,
satisfying EQIX356Q, EQIX357Q. In numerical
calculation, the innerproduct is usually computed by numerical integration,
which needs sufficient accuracy and might be complicated. An alternative approach
that does not need these integrations is to write the states as
wavefunctions under a particular representation(e.g., the space-coordinate
representation), and then select EQIX138Q to minimize
 EQDS100Q 
where EQIX358Q is some sample points in the defining area of the wavefunctions.
In order to ensure EQIX138Q to be a good approximation of EQIX353Q, the sample
points should be appropriately chosen. Usually the number of the sample points
is greater than and approximately proportional to EQIX23Q, and the separation between
two neighboring sample points should be less than the least quasi-semiwavelength
of a wavefunction in EQIX346Q.

This alternative approach (I call it the least square method)
leads to a reduction of the operator EQIX347Q:
 EQDS101Q 
where EQIX359Q is a pseudo-innerproduct defined as
EQIX360Q for arbitrary EQIX361Q
and EQIX362Q, and EQIX363Q is a set of pseudo-orthonormal basis in
EQIX346Q satisfying EQIX364Q.
We find that this approach is very similar to the usual one, except that
a discrete sum over sample points takes the place of the usual innerproduct
integration. And there is a great degree of freedom in the selection of the sample
points. In fact, as soon as the sample points are selected according to the
spirit mentioned above, the accuracy of the solution of the differential
equation usually will not decrease significantly. The major factor that determines
the accuracy of the solution is the choice of the subspace EQIX346Q, which
has been discussed to some extent in previous pages.

In this work, solving the simpliest quantum three-body problem, the three methods
discussed above are all used: the coordinate transformation method,
the asymptotic series method, and the least square method. A high
precision is reached for the ground state energy of the ideal helium atom, and
the solution has also some merit in comparison with the Hyleraas-type variational
solution[5,6]. In section 2 the Bartlett-Fock expansion[7,8,9]
is studied, in order to
reflect the analytical structure of the wavefunction near the origin.
In this study, the asymptotic series are used to represent the hyper-angular
dependence of the wavefunction. In section 3 the EQIX365Q coordinate system is
used to study the hyper-angular dependence of the wavefunction. This coordinate
system cancels the singularity of the hyper-angular functions totally. The
relationship between this coordinate system and the Hyleraas-type variational
method is also discussed. The least square method is used to 
reduce the hyper-angular parts of the kinetic
energy operator and the potential energy operator to finite-dimensional
matrices. In section 4 the connection
of the outer region solution and the inner region
Bartlett-Fock expansion is studied,
using the least square method. In section 5 the numerical result is presented
and compared with those of other methods. Some explanations are made. In section
6 some discussions are presented and some future developments are pointed out.

\section{BARTLETT-FOCK EXPANSION}

Considering an S state of an ideal helium atom, that is, assuming an infinite
massive nucleus and infinite light speed, one may write the Schr\"{o}dinger
equation
 EQDS102Q 
where EQIX366Q, EQIX367Q,
EQIX368Q, and
EQIX369Q. EQIX370Q and EQIX371Q
are the distances of the electrons from the nucleus, and EQIX372Q
is the angle formed by the two electronic position vectors measured from the
nucleus. In this equation, an S state is assumed, so the wavefunction
EQIX362Q is only dependent on EQIX370Q, EQIX371Q and EQIX372Q, or, equivalently,
EQIX327Q, EQIX373Q, and EQIX308Q. The atomic unit, i.e.,
EQIX374Q, is assumed throughout this paper.
The potential energy is
 EQDS103Q 
where EQIX375Q is the distance between the two electrons.
 EQDS104Q 

\vspace{3mm}
The Bartlett-Fock expansion is
 EQDS105Q 
where EQIX376Q, and EQIX377Q.
EQIX378Q only depends on the two hyper-angles, say,
EQIX379Q and EQIX380Q, and does not depend
on the hyper-radius, EQIX381Q. When EQIX382Q, 
EQIX383Q.

Using the coordinates EQIX117Q, EQIX384Q, and EQIX385Q, one may rewrite
the Schr\"{o}dinger equation (1) as
 EQDS106Q 
where EQIX386Q, and
 EQDS107Q 
 EQDS108Q 

Substituting eq.(4) into eq.(5), and comparing the corresponding coefficients
before EQIX387Q, one will obtain
 EQDS109Q 
where EQIX388Q.

The functions EQIX378Q are solved out in the order
with n increasing; and for each n, with k decreasing. The physical
area of EQIX389Q is the unit circle: EQIX390Q.
And the function EQIX391Q may has singularities at
EQIX392Q and at EQIX393Q. The singularities are of these kinds:
EQIX394Q, EQIX395Q, and EQIX396Q, with
EQIX397Q. So one may write
the Taylor series in the EQIX398Q unit circle:
 EQDS110Q 
The singularities make the usual cutoff, EQIX399Q, inappropriate, because
the error decreases slowly when EQIX400Q increases.
But since we have known the forms of the singularities, we can write the
asymptotic formulae to calculate those high order Taylor coefficients that
have important contributions:
 EQDS111Q 
 EQDS112Q 
Eq.(10-1) is appropriate when EQIX401Q and EQIX402Q,
while eq.(10-2) is appropriate when EQIX403Q and EQIX404Q.
EQIX405Q, and EQIX340Q.
Here I have assumed the state is a spin-singlet, and thus
EQIX406Q. For a spin-triplet,
the factor EQIX407Q in eq.(10-1) should be substituted by
EQIX408Q.

In my actual calculation, the EQIX409Q plane is divided into four areas:\\
the finite area: EQIX410Q and EQIX411Q (EQIX412Q),\\
the EQIX20Q-asymptotic area: EQIX413Q and EQIX414Q,\\
the EQIX21Q-asymptotic area: EQIX415Q and EQIX416Q,\\
and the cutoff area: the remain area.

Eq.(10-1) is used in the EQIX20Q-asymptotic area, and eq.(10-2) is used in the
EQIX21Q-asymptotic area, while the contribution from the cutoff area is neglected for
it is extremely tiny when EQIX412Q.

In a word, a relevant hyper-angular function is described by a finite set of
parameters up to a high precision. These parameters are some Taylor
coefficients and some asymptotic coefficients.
To operate with some functions of this kind means
to operate with the corresponding sets of parameters. The relevant operations
are: addition of two functions--adding the corresponding parameters of the
two sets; multiplying a function by a constant--multiplying each parameter in the
set by the constant; multiplying a function by EQIX417Q(eq.(7))--
an appropriate linear transformation of the set of parameters of the multiplied
function; solving an equation EQIX418Q with g known and f unknown--solving
a set of linear equations about the parameters corresponding to EQIX419Q. Here, I write
the relevant linear equations corresponding to the equation EQIX418Q:
 EQDS113Q 
 EQDS114Q 
 EQDS115Q 

The detailed order to solve EQIX378Q is:

Case 1: EQIX420Q, where EQIX421Q is an integer. In this case,
solve EQIX422Q from eq.(EQIX423Q);
and then solve EQIX424Q from eq.(EQIX425Q);
EQIX426Q; at last solve EQIX427Q from eq.(EQIX428Q). For each
EQIX378Q, the order is: first solve the asymptotic coefficients,
from EQIX429Q to EQIX430Q; then solve the
Taylor coefficients, from EQIX431Q to EQIX432Q(i.e.,EQIX433Q).

Case 2: EQIX187Q is an integer. In this case, the order is more complicated,
because the operator EQIX434Q has zero eigenvalue(s) in this case. The order
is as following:

\vspace{0.5mm}
\noindent Step 1: \parbox[t]{146mm}
{set the asymptotic coefficients and the EQIX435Q Taylor coefficients
of EQIX436Q to zero;}

\vspace{1.5mm}
\noindent step 2: \parbox[t]{146mm}
{EQIX437Q;}

\vspace{1.5mm}
\noindent step 3: \parbox[t]{146mm}
{if EQIX438Q, goto step 8;}

\vspace{1.5mm}
\noindent step 4: \parbox[t]{146mm}
{solve the asymptotic coefficients and EQIX435Q Taylor coefficients of
EQIX378Q, from eq.(EQIX439Q), in the order analogous to that of case 1.}

\vspace{4mm}
\noindent step 5: \parbox[t]{146mm}
{solve the EQIX440Q Taylor coefficients of EQIX441Q, from
eq.(EQIX439Q).}

\vspace{1.5mm}
\noindent step 6: \parbox[t]{146mm}
{solve the EQIX442Q Taylor coefficients of EQIX441Q, from
eq.(EQIX443Q), with EQIX444Q decreasing(analogous to case 1) to EQIX445Q.}

\vspace{4mm}
\noindent step 7: \parbox[t]{146mm}
{EQIX446Q, and goto step 3;}

\vspace{1.5mm}
\noindent step 8: \parbox[t]{146mm}
{set the EQIX440Q Taylor coefficients of EQIX427Q with some free
parameters;}

\vspace{1.7mm}
\noindent step 9: \parbox[t]{146mm}
{solve the EQIX442Q Taylor coefficients of EQIX427Q,
from eq.(EQIX428Q), with EQIX444Q decreasing(analogous to case 1) to EQIX445Q.}

\vspace{4mm}
The free parameters in solving eq.(8)(see step 8 of case 2) are
finally determined by the boundary condition: EQIX447Q,
when EQIX448Q. In principle, we can use the Bartlett-Fock expansion
(eq.(4)) for arbitrary EQIX117Q, because it is always convergent. But actually,
when EQIX117Q is large, the convergency is slow and there is canceling of large
numbers before this convergency is reached, both of which make
the Bartlett-Fock expansion impractical. So I only use this expansion when
EQIX117Q is relatively small(see ref.[15] for similarity):
EQIX449Q.

In atual calculation,
I chose EQIX450Q, EQIX451Q, EQIX452Q, EQIX453Q (the largest n value
of the terms in eq.(4) that are not neglected),
and EQIX454Q, and found that
the numerical error for the calculation
of the inner region (EQIX455Q)
wavefunction is no more than a few parts in EQIX456Q.
I use this method to test the accuracy of the calculation: set EQIX457Q in
eq.(8) (note that EQIX458Q) equal to an initial value
(for example, set EQIX459Q, or set EQIX460Q),
and use the approximate wavefunction EQIX461Q
thus obtained to calculate the value
EQIX462Q, where EQIX463Q is the exact Hamiltonian operator,
and I find it to be almost equal to the initial value EQIX464Q,
with a relative error no more than a few parts in EQIX456Q.

When EQIX117Q is larger, another approach is used:

\section{THE HYPER-ANGULAR DEPENDENCE OF THE WAVEFUNCTION}

We have seen that the hyper-angular dependence of the wavefunction,
described as a function of EQIX398Q for each fixed
EQIX465Q, has
singularities at EQIX392Q and at EQIX393Q. Physically,
this corresponds to the case that the distance between two of the three
particles equals zero. It can be proved that,
for a spin-singlet, the following coordinate
transformation will eliminate these singularities \emph{thoroughly}:
 EQDS116Q 
Equivalently,
 EQDS117Q 
If the energy-eigenstate EQIX362Q is symmetric under the exchange
of EQIX370Q and EQIX371Q(spin-singlet),
I believe that, for each fixed EQIX466Q, EQIX362Q is a \emph{entire}
function of EQIX365Q.
If the energy-eigenstate EQIX362Q is antisymmetric under the interchange
of EQIX370Q and EQIX371Q(spin-triplet), I believe that, for each fixed EQIX466Q,
EQIX467Q, where EQIX361Q is a \emph{entire}
function of EQIX365Q.

This beautiful characteristic makes it especially appropriate to approximate
EQIX362Q, for each fixed EQIX466Q, by an EQIX187Q-order polynomial of EQIX365Q, not by
an EQIX187Q-order polynomial of EQIX398Q. The former expansion, a polynomial
of EQIX365Q, matches the analytical structure of EQIX362Q; while the latter one,
a polynomial of EQIX398Q, does not. The hyper-spherical harmonic
function method belongs to the latter expansion, a polynomial of
EQIX398Q. So the hyper-spherical harmonic function expansion does not
correctly reflect the analytical structure of EQIX362Q. The slow convergency
of the hyper-spherical harmonic function expansion is only a consequence of
this analytical structure mismatching.

We expect that the EQIX365Q polynomial expansion converges \emph{factorially}
to the true wavefunction. It is worthful to demonstrate a similar example
to illustrate this. Consider a function EQIX468Q;
expand EQIX330Q by Legendre polynomials:
EQIX469Q; it can be proved that the error
of this formula is of the order EQIX470Q, which factorially approach zero
as EQIX187Q increases.

Using the EQIX471Q coordinates, one can write the Schr\"{o}dinger
equation as:
 EQDS118Q 
where EQIX472Q and EQIX473Q are the hyper-angular parts of the kinetic energy and
the potential energy, respectively.
 EQDS119Q 
 EQDS120Q 

The physical area EQIX474Q of EQIX365Q is:
\vspace{5mm}

\begin{center}\begin{picture}(210,175)(0,0)

\LongArrow(0,35)(210,35)
\Text(213,32)[t]{EQIX475Q}
\LongArrow(105,0)(105,175)
\Text(108,177)[l]{EQIX476Q}
\Text(99,28)[]{EQIX347Q}
\Line(134,35)(134,134)
\Line(134,134)(105,105)
\DashLine(105,105)(35,35){3}
\DashLine(134,134)(105,134){3}
\CArc(35,35)(99,0,45)
\Text(35,28)[]{EQIX231Q}
\Text(134,28)[]{EQIX477Q}
\Text(102,99)[t]{EQIX230Q}
\Text(95,134)[]{EQIX478Q}
\Text(123,104)[]{EQIX474Q}
\Text(138,40)[]{EQIX479Q}
\Text(140,134)[]{EQIX473Q}
\Text(97,111)[]{EQIX182Q}

\end{picture}\end{center}

\vspace{1mm}
\noindent In this figure, point EQIX479Q corresponds to the coincidence of the
two electrons, and point EQIX182Q corresponds to the coincidence of the nucleus
and one electron.

For a spin-singlet,
we can use an n-order polynomial of EQIX365Q to approximate EQIX362Q.
The coefficients of this polynomial are functions of EQIX466Q.
Denote by EQIX346Q the set of all the polynomials of EQIX365Q
with order no more than EQIX187Q. Here, EQIX480Q is the dimension.
In the physical area EQIX474Q, I choose a set of points as sample points:
 EQDS121Q 
 EQDS122Q 
where EQIX481Q is the minimum physical EQIX475Q value for a EQIX476Q value.
EQIX482Q, if EQIX483Q; and EQIX484Q, if EQIX485Q.
EQIX486Q, and EQIX487Q, EQIX488Q.
I chose EQIX489Q, so there are altogether EQIX490Q sample points.
These sample points define a pseudo-innerproduct.
I constructed a set of
pseudo-orthonormal basis in EQIX346Q,
by using the Schmidt orthogonalization method, and then reduce the operators
EQIX472Q and EQIX473Q to EQIX348Q matrices under this basis,
using the method introduced in section 1.

\section{CONNECTION OF THE INNER REGION AND THE OUTER REGION WAVEFUNCTIONS}

In the area EQIX491Q(inner region), the Bartlett-Fock expansion is used.
In the area EQIX492Q(outer region), EQIX362Q is approximated
by a vector in EQIX346Q for each given EQIX466Q, and the partial
derivatives with respect to EQIX466Q are substituted by optimized
variable-order and variable-step differences, which requires the selection
of a discrete set of EQIX466Q values.
The overlap region of the inner region and the
outer region ensures the natural connection of the derivative of EQIX362Q,
as well as the connection of EQIX362Q itself. The connection is performed
by using the least square method:
for a polynomial of EQIX365Q at EQIX493Q, appropriately
choose the values of the free parameters of the solution of eq.(8)
(see section 2) so that the sum of the squares of the differences of the
the inner region solution and the outer region polynomial at the sample points
is minimized. This defines a linear transformation to calculate the values
of those free parameters from the given polynomial. When the values
of these free parameters are determined, one can calculate the values
of EQIX362Q in the region EQIX494Q, using the Bartlett-Fock
solution, and further use these EQIX362Q values to construct polynomials
ofEQIX365Q at EQIX494Q (according to the law of
least square), and then use these polynomials
in the difference calculation of the partial derivative of EQIX362Q with
respect to EQIX466Q at EQIX495Q. At a sufficient large value
EQIX496Q, the first-class boundary condition is exerted; of course,
future development may substitute this by a connection with the long range
asymptotic solution of EQIX362Q.

At last, the whole Schr\"{o}dinger equation is reduced to an eigen-problem
of a finite-dimensional matrix. The dimension of the matrix is
EQIX497Q, where EQIX498Q is the number of free 
EQIX466Q nodes
used in discretizing the partial derivatives with respect to EQIX466Q, and EQIX23Q
is the number of independent hyper-angular polynomials used. Note that
the energy value should be used in solving eq.(8), but it is unknown. The
actual calculation is thus an iteration process: choose an initial value
of EQIX499Q to solve eq.(8) and form the EQIX497Q dimensional matrix,
and calculate the eigenvalue of this matrix to get a new value EQIX500Q, etc..
The final result is the fixed point of this iteration process. In actual
calculation, I found that the convergency of this iteration process
is very rapid if EQIX501Q is relatively small. Choosing EQIX454Q,
I found that each step of iteration cause the difference between the eigenvalue
of the matrix and the fixed point decrease by about EQIX502Q times, when
calculating the ground state.

\section{NUMERICAL RESULT AND COMPARISONS}

Using 20 independent Bartlett-Fock series(up to the EQIX503Q term in eq.(4),
neglecting higher order terms),
choosing EQIX504Q (so that EQIX505Q), choosing EQIX506Q, with
EQIX454Q and EQIX507Q, and
with the discrete values of EQIX466Q
equal to , and
EQIX508Q (the first three
points are for the natural connection of the derivative of EQIX362Q,
the last point is for the first-class
boundary condition, and the remained 40 points are free nodes),
and discretizing the partial derivatives with respect to EQIX466Q according
to the complex-plane-division rule(that is: when calculating the partial
derivatives with respect to EQIX466Q at EQIX509Q, use and only use
those node points satisfying EQIX510Q in the difference format, because the
point EQIX511Q is the singular point), I obtained the
result for the ground state
energy of the ideal helium atom:
 EQDS123Q 
compared with the accurate value:
 EQDS124Q 
So the relative error of the result (19) is about EQIX512Q.
Since my method is not a variational method,
the error of the approximate wavefunction
that I obtained should be of a similar order of magnitude, so if one
calculate the expectation value of the Hamiltonian under this approximate
wavefunction, the accuracy of the energy will be further raised by
several orders of magnitude.

The result (19) is much more accurate than the result
of ref.[10]:EQIX513Q, which
used the hyper-spherical coordinate method.
In ref.[10], the quantum numbers EQIX514Q
(angular momenta of the two electrons) are used and a cutoff for them
is made; this cutoff does not correctly reflect the analytical
structure of EQIX362Q at EQIX515Q (equivalently EQIX393Q). This is the major
reason causing the inaccuracy of the result of ref.[10].

It is also worthful to compare my result with that of ref.[4], in which
the correlation function hyper-spherical harmonic method is used. Note that
the result (19) is obtained by using a set of EQIX505Q
hyper-radius-dependent coefficients to expand the wavefunction. For a similar
size in ref.[4], N=64, the result is EQIX516Q, with relative error
about EQIX517Q. When N=169, the result of ref.[4] is
EQIX518Q, with relative error about EQIX519Q. Apparently
my method converges more rapidly than that of ref.[4]. The major reason
is that the correlation function hyper-spherical harmonic method does not
cancel the singularities totally---there is still some discontinuity
for the higher order derivatives, although the low order singularities,
which trouble the naive hyperspherical harmonic method, are canceled by
the correlation function.

\section{CONCLUSIONS, DISCUSSIONS AND FUTURE DEVELOPMENTS}

In conclusion, there are several important ideas in my work that
should be emphasized: first, I use the asymptotic series to compute
the Bartlett-Fock series up to a high precision, with error no more than,
for example, a few parts in EQIX456Q. Second, I propose an alternative
coordinate system, the EQIX365Q system, in which the hyper-angular
singularities are thoroughly eliminated, which renders a factorial
convergency for the expansion of the hyper-angular function. Third,
I make use of the least square method to reduce an operator(infinite
dimensional matrix) to a finite dimensional matrix in a finite dimensional
subspace of the Hilbert space and to connect the solutions in different
regions, avoiding complicated numerical integrations, without much loss
of the accuracy for the solution. Fourth, the optimized difference format
---the complex plane division rule---is used to discretize the partial
derivatives of the wavefunction with respect to EQIX466Q. I calculated the
ground state energy of an ideal helium atom concretely and obtained a very
high precision, demonstrating that my method is superior to many other
methods and competitive with any sophisticated methods.

About the analytical structure of the stationary wavefunction:
1. there are logarithmic singularities at EQIX511Q, in the forms
of EQIX520Q; 2. for a given EQIX466Q, EQIX362Q (for a spin-singlet)
or EQIX521Q(for a spin-triplet) has no singularity,
as a function of EQIX365Q.

Here, I must mention the well known variational method based on
the Hyleraas-type functions, because it also
satisfies the second characteristic of
the wavefunction mentioned in the above paragraph.
One can see this by
a simple derivation. The Hyleraas-type function is a entire function
of EQIX370Q, EQIX371Q and EQIX375Q, or equivalently, a entire function
of EQIX522Q, EQIX523Q, and EQIX375Q. For a fixed EQIX466Q,
one can substitute EQIX524Q in this function by
EQIX525Q, so that, for fixed EQIX466Q,
the function is a entire function
of EQIX522Q and EQIX375Q for spin-singlet, or such kind of entire
function times a common factor EQIX523Q for spin-triplet. Equivalently,
for fixed EQIX466Q, the Hyleraas-type function is a entire function of
EQIX365Q(spin-singlet) or such kind of entire function times
EQIX526Q(spin-triplet).
This characteristic is one of the most important reasons that
account for the high accuracy of the Hyleraas-type variational method.

But this variational method also has its shortcoming: the Hyleraas-type
function does not reflect the logarithmic singularities with respect to
EQIX466Q. So, although this method has high precision for the energy levels,
the approximate wavefunctions that it renders may deviate significantly
from the true wavefunctions near the origin. See ref.[4,13,14] for detailed
discussions.

A central idea of this paper is: devising the calculation method
according to the analytical structure of the true solution. The EQIX365Q
coordinates, the Bartlett-Fock expansion and the asymptotic series approach
to compute this expansion, and the complex-plane-division rule in calculating
the partial derivatives with respect to EQIX466Q, all reflect this central
idea. The basic principle that ensures high numerical precision is just
this idea.

This preliminary work is incomplete in the following aspects:

First, how to \emph{prove} that EQIX362Q(for spin-singlet, or 
EQIX521Q for spin-triplet) has no singularity for fixed EQIX466Q,
as a function of EQIX365Q? Note that if this function still has singularities outside
of the physical area EQIX474Q(see previous figure), the convergency of
the expansion of the hyper-angular function
will be only exponential, not factorial. Of course, even if
such kind of singularities do exist, my method will still converge more rapidly
than the correlation function hyperspherical harmonic method, because the latter
method only converges like EQIX527Q, slower than EQIX528Q.
The rapid convergency of my method make me guess that such kind of singularities
do not exist.

Second, the asymptotic behavior of the wavefunction, when one electron is
far away from the nucleus, is not studied in this work. This problem
will be important when the highly excited states and the scattering states are
studied, a topic that will become my next object. 

Third, how to use the ideas proposed in this work to study a helium atom
with finite nuclear mass? Besides this, the relativistic and QED corrections
must be calculated, if one want to obtain a result comparable with
high-precision experiments.

Fourth, I have focused on the S states till now. When the total angular momentum
is not zero, there might be more than one distance-dependent functions (see, for
example, ref.[11]). I believe that some important analytical structures of the
S states studied in this work are also valid for those functions.

Surely, some important aspects of this work will also play an important
role in the highly excited states and the scattering states: the logarithmic
singularities about EQIX466Q and the method to compute the Bartlett-Fock expansion,
the non-singularity with respect to the coordinates EQIX365Q, and the technique
to connect solutions of different regions, etc.. They can be applied
to the study of the highly excited states and the scattering states.

\vspace{5mm}
\noindent {\large\bf ACKNOWLEDGEMENTS}

\noindent The encouraging discussions with Prof. SUN Chang-Pu and with
Prof. Zhong-Qi MA are gratefully
acknowledged. I thank Prof. HOU Boyuan for providing me some useful references.
I am grateful to Prof.~C.M.~Lee~(Jia-Ming~Li) and Dr.~Jun~Yan for
their attention to this work and their advices.

\begin{thebibliography}{99}

\bibitem{1} V.B.Mandelzweig, Phys.Lett.{\bf A.78}, 25 (1980)

\bibitem{2} M.I.Haftel, V.B.Mandelzweig, Ann.Phys.{\bf 150}, 48 (1983)

\bibitem{3} R. Krivec, Few-Body Systems, {\bf 25}, 199 (1998)

\bibitem{4} M.I.Haftel, V.B.Mandelzweig, Ann.Phys.{\bf 189}, 29-52 (1989)

\bibitem{5} E.A.Hylleraas and J.Midtdal, Phys.Rev.{\bf 103}, 829 (1956)

\bibitem{6} K.Frankowski and C.L.Pekeris, Phys.Rev.{\bf 146}, 46 (1966)

\bibitem{7} J.H.Bartlett, Phys.Rev.{\bf 51}, 661 (1937)

\bibitem{8} V.A.Fock, Izv.Akad.Nauk SSSR, Ser.Fiz.{\bf 18}, 161 (1954)

\bibitem{9} J.D.Morgan, Theor.Chem.Acta{\bf 69}, 81 (1986)

\bibitem{10} Jian-zhi~Tang, Shinichi~Watanabe, and Michio~Matsuzawa,
Phys.Rev.{\bf A.46}, 2437 (1992)

\bibitem{11} W.T.Hsiang and W.Y.Hsiang, {\it On the reduction of the
Schr\"{o}dinger's equation of three-body problem to a system of linear
algebraic equations}, preprint (1998)

\bibitem{12} Zhong-Qi Ma and An-Ying Dai, {\it Quantum three-body problem},
preprint, physics /9905051 (1999); Zhong-Qi Ma, {\it Exact solution to the
Schr\"{o}dinger equation for the quantum rigid body},
preprint, physics /9911070 (1999).

\bibitem{13} J.H.Bartlett {\it et al.}, Phys.Rev.{\bf 47}, 679 (1935)

\bibitem{14} M.I.Haftel and V.B.Mandelzweig, Phys.Rev.{\bf A.38}, 5995 (1988)

\bibitem{15} James~M.Feagin, Joseph~Macek and Anthony~F.Starace,
Phys.Rev.{\bf A.32}, 3219 (1985)
\end{thebibliography}

\end{document}