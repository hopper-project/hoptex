
\documentclass[journal]{IEEEtran}
\IEEEoverridecommandlockouts
\usepackage{cite}
\usepackage{amsmath,amsfonts,amssymb}
\usepackage{xcolor}
\usepackage{url}

\ifCLASSINFOpdf
	\usepackage[pdftex]{graphicx}
 	\DeclareGraphicsExtensions{.pdf,.jpeg,.png,.eps}
\else
\fi
\usepackage{epstopdf}
\usepackage{ulem}
\usepackage{gensymb} 
\usepackage{eso-pic}

 
\hyphenation{op-tical net-works semi-conduc-tor}
\usepackage{tikz}
\usepackage{textcomp}
\usepackage{hyperref}
\usepackage{lipsum}

\begin{document}

\title{A study of  periodograms standardized using training data sets  and application to exoplanet detection}
\author{Sophia~Sulis, David~Mary and~Lionel~Bigot
\thanks{This work was supported by Thales Alenia Space, PACA region, CNRS project DETECTION/Imag'In and Programme National de Physique Stellaire (PNPS) of CNRS/INSU, France.}
\thanks{ The authors are with the  Universit\'e C\^ote d'Azur, OCA, CNRS, Laboratoire Lagrange, France
  (e-mail: Sulis.Sophia@oca.eu; David.Mary@unice.fr; Lionel.Bigot@oca.eu).}}

 

\maketitle
{\begin{tikzpicture}[remember picture,overlay]
\node[anchor=south,yshift=1pt] at (current page.south) {\fbox{\parbox{\dimexpr\textwidth-\fboxsep-\fboxrule\relax}{
  \footnotesize \textcopyright 2017 IEEE. Personal use of this material is permitted.
  Permission from IEEE must be obtained for all other uses, in any current or future
  media, including reprinting/republishing this material for advertising or promotional
  purposes, creating new collective works, for resale or redistribution to servers or
  lists, or reuse of any copyrighted component of this work in other works.
  DOI: \href{http://ieeexplore.ieee.org/document/7815441/}{10.1109/TSP.2017.2652391}}}};
\end{tikzpicture}}

\begin{abstract}
	When the noise affecting time series is colored with unknown statistics, a difficulty for sinusoid detection  is to control the true significance level of the test outcome.
	This paper investigates the possibility of using training data sets of  the noise to improve this control. 
	Specifically, we analyze the performances of various detectors {applied to} periodograms standardized using training  data sets. Emphasis is put on sparse detection in the Fourier domain and on the limitation posed by  the necessarily finite size of the training  sets available in practice. We study  the resulting false alarm and detection rates and show that  standardization leads in some cases to powerful constant false alarm rate tests.
	The study is both analytical and numerical.   Although analytical results are derived in an asymptotic regime, numerical  results show that theory accurately describes the tests'  behaviour for moderately large sample sizes. 
	{Throughout the paper, an application of the considered periodogram standardization  is presented  for exoplanet detection in radial velocity data.}
\end{abstract}

\begin{IEEEkeywords}
	 Multiple sinusoids' detection, colored noise, \\periodogram standardization, sparse detection, asymptotic.
\end{IEEEkeywords}

\IEEEpeerreviewmaketitle

\section{Introduction}
\subsection{Considered problem and  application}
\label{secIA}
\IEEEPARstart{D}{etecting} sinusoids in noise is one of the most studied problems in signal processing. 
 Sinusoid detection is classically based on Fourier analysis, which benefits  from a 
 considerable set of statistical results. Often, the  assumptions required for these results to hold are :\\
{ (i)} Under the null hypothesis ($\mathcal{H}_0$ : noise only), 
the  random process $X$ is stationary Gaussian with known statistics;\\
{ (ii)} The time series $\{X(t_j)\}_{j=1,\cdots,N}$ is regulary sampled,  in which case Fourier analysis is performed through the 
 periodogram {\cite{Schuster_1898, Brillinger_1981}}
\begin{equation}
\label{eq1}
	P(\nu)   :=  \frac{1}{N\Delta t } \Big| \sum_{j=1}^{N} X(t_j)\mathrm{e}^{-{\rm{i}}2\pi\nu j\Delta t} \Big|  ^2,
\end{equation}
where $\nu$ is the considered frequency, $N$ the number of samples and $\Delta t$ the time sampling step;  \\
(iii) The number of samples $N$ is large (asymptotic regime);\\ 
(iv) The periodogram is evaluated at Fourier frequencies. \\
\indent Assumptions  (i)$-$(iv)  allow to characterize  the statistical properties of the periodogram as an  estimate of the power spectral density (PSD){{\cite{Bartlett_1950, Grenander_1957, Brillinger_1981, Priestley_1981, Brockwell_1991, Bloomfield_2000,Stoica_2005, Quinn_2001}}.
 In particular, any finite set of periodogram ordinates 
 near some fixed frequency 
 are asymptotically independent and exponentially distributed with parameters dependent on the noise PSD at the relevant frequencies (this assumption actually extends to linear processes other than Gaussian, see Th. 4 of \cite{Quinn_2001}).
In detection, such properties can be exploited
to obtain the  probabilities of false alarm and detection of tests based on periodogram ordinates.}

Despite a large literature on the subject, the detection of periodic signals remains an active field of research because in practical situations some (or all) assumptions (i)-(iv) above may not be met. Before reviewing some such situations, it is worth introducing the  application that motivated the present study 
{ (the proposed detection approach may cover other applications, however).}

With almost { $3\,000$} confirmed exoplanets and nearly { $2\,500$} candidates (in { September} 2016,  \url{www.exoplanets.org}), the detection of extrasolar planets is an extremely active research field in Astrophysics since two decades. This field  benefits from constant technological improvements allowing extremely low noise detectors \cite{Pepe_2014b} and  long (months to years) spaceborne observations with high (a few tens of seconds) sampling rates{ \cite{Batalha_2014, Auvergne_2009, Rauer_2014}}. 
The Radial Velocity (RV)  technique is one method for exoplanet detection. When planets orbit a  host star, the resulting { gravitational} force creates a periodic displacement of the star. 
This induces a modulation of the RV of the star with respect to (w.r.t.) Earth, which translates into a periodic Doppler shift of the stellar light.  The RV  technique consists in detecting such variations in stellar RV time series {\cite{Fischer_2014,Perryman_2011}.}\\
Modern instrumental performances have reached signal to noise ratios allowing, in principle, to detect exoplanets comparable to the Earth {by} the RV method
. However, at such low  levels of instrumental noise, a new critical and limiting issue appears. The stellar surface can be seen as a boiling fluid, with millions of convection cells generating  upward and downward plasma flows visible under the form of granules having typical lifetime of a few minutes. These motions generate random fluctuations in the measured RV of the star that can mimic, or hide, exoplanetary signatures.
The  case of  $\alpha$ Centauri Bb planet,  a detection claimed in 2012 \cite{Dumusque_2012} (with an evaluated $P$-value of $0.02 \%$) and since then subject to controversy \cite{Hatzes_2013, Rajpaul_2016}, is one example showing how our incomplete knowledge of the noise affects the reliability assigned to  exoplanet detection claims. This specific issue motivated the present study.\\
A key point is that in parallel to  technological advances,   astrophysicists have continuously improved stellar models and elaborated numerical simulation codes able to account for the complex interplay of various astrophysical processes in the star's interior and surface.  Recent works  demonstrate that 
{ granulation noise }  can be reproduced by large scale numerical simulations in a reliable way \cite{Bigot_2011}. This suggests the possibility of using such simulations to \textit{calibrate} the detection process, as RV time series are strongly affected by  this stellar noise. The present study shows that { one such calibration} indeed leads to improved  control of the statistical significance and to detection tests with increased power. 
\subsection{Unknown  noise statistics: related works}
\label{rel}
Turning back to the deviations encountered in practice w.r.t. assumptions (i)$-$(iv) above, this work will be primarily interested  in the case of
an incomplete knowledge of the noise statistics under the null hypothesis (\textit{i.e.}, relaxing condition { (i)}; we mention in the last section perspectives to relax { the} other conditions). 
In this situation, the distribution of $P$ under the null is not known and  consequently the significance level (the size) of the test is not known either. 
{  Constant false alarm rate (CFAR) detectors have been devised when the noise is white \cite{Fisher_1929,Chiu_1989,Shimshoni_1971}.}
When the noise is colored, the detection problem is  more complicated. In practice, two approaches can be followed. The first approach is simply to ignore possible noise correlations and to apply { tests designed for white noise}. However, as will be illustrated in this study, the statistical behavior of the resulting testing procedure may be hazardous, with unpredictable significance level and poor power (see 
also \cite{Kay_1999} on this point). \\
 A more sophisticated approach consists { of} estimating  the noise PSD (called $S_E$ below) so that condition { (i)} above is considered to hold approximately.  
This estimate  can then be used to calibrate the periodogram of the data $P(\nu)$, leading to a frequency-wise standardized periodogram of the form  
\begin{equation}
\widetilde{P}(\nu\;|\; \widehat{S}_E):=\frac{P(\nu)}{\widehat{S}_E(\nu)}.
\label{ptilde}
\end{equation}
Note that the classical Fisher's test \cite{Fisher_1929} {  standardizes the periodogram ordinates
by the estimated  PSD of a white noise. Standardization \eqref{ptilde} can be seen as a generalization of this approach  (see \cite{Li_2014} for  a  recent review).} \\
  The estimate ${\widehat{S}_E(\nu)}$ can be  parametric 
  
{  or non-parametric}.
  
Non parametric approches  originate from  seminal works  {  of  Whittle  \cite{Whittle_1952} and  Bartlett \cite{Bartlett_1955}}.
Parametric methods  often proceed by fitting  AutoRegressive (AR) or ARMA (AR Moving Average)  processes to the time series. 

{  A further complication arises  when multiple sinusoids are present under the alternative, as they perturb the estimation of the noise PSD
\cite{Siegel_1980,Priestley_1981}. Standardized tests  for this case  can be found in   \cite{Shimshoni_1971, Siegel_1980,Bolviken_1983a, Bolviken_1983b, Chiu_1989,Li_2014}.  
These tests are however non adaptive in the  number of sinusoids (which must be set {\textit{a priori}}) and designed for white noise. 
  For adaptive procedures for colored noise
see  Chap. 8 of \cite{Priestley_1981}, and 
\cite{Sachs_1993,Sachs_1994,Bhansali_1979,Truong_1990,Quinn_1991,Quinn_1999,Kavalieris_1994,Hannan_1961,Nicholls_1967}.
 
Techniques reducing the influence of signal peaks under the alternative are proposed in  \cite{Chiu_1990,Gryca_1998}. 
Different approaches,  related to standardization \eqref{ptilde}, can be found in  \cite{White_1999,Lu_2005,Liavas_1998,Zheng_2012}.
\\ When following Generalized Likelihood Ratio (GLR) approaches { for detecting  multiple sinusoids in unknown number,  the GLR  must} be combined with model selection procedures. While  sharp model selection criteria and CFAR detectors exist under white noise, the correlated case remains an open problem \cite{Nadler_2011}.}
  
  {  \cite{Koen_2015a,Koen_2015b,Hatzes_2013,Tuomi_2012}}.  
 
  
  
   
   

In conclusion, regarding the problem of assessing tests' significance levels for multiple sinusoids detection in noise, an inspection of the literature shows that:\\
$\bullet$	For white noise of unknown variance, several studies provide accurate results for standardized test statistics of the form \eqref{ptilde}, \textit{e.g.},    \cite{Shimshoni_1971, Siegel_1980,Bolviken_1983a, Bolviken_1983b, Chiu_1989}.\\
	$\bullet$ For colored noise with unknown { PSD}, we are not aware of works studying the  false alarm rate when AR/ARMA or other models are used for test standardization as in \eqref{ptilde}. The difficulty in this case is the dependence of the distribution of $\widehat{S}_E$ on  estimated   {
	parameters, which complicates} the analytical characterization of the  distribution of $\widetilde{P}$.\\
The procedures described in \cite{Hannan_1961, Bhansali_1979,Priestley_1981}  provide asymptotic control of the false alarm rate. These procedures do not operate explicitly on test statistics of the form \eqref{ptilde} but rather on windowed periodograms that depend on several parameters. We found that {  these tests are  in practice sensitive to  parameter}  setting
{ and that
estimating these parameters  impacts the significance level at which the tests are conducted}. 
This  level  can of course be approximated by simply neglecting  { the influence of 
such} a `preprocessing stage' { (dealing, \textit{e.g.}, with model order selection, filtering, adaptive window design, or standardization)}. For instance,  { we might pretend that $\widehat{S}_E={S}_E$ in \eqref{ptilde}.
As  will be highlighted in  Sec. \ref{sec6} and \ref{sec7}, the actual significance level obtained when doing so may  however be} far from the assumed one. This leaves open the question of designing both powerful and CFAR tests for unknown colored noise and we propose  such tests in this paper. 

Before closing this literature survey, we mention a few tests designed for a particularly interesting configuration of the detection problem,   
which is the so-called \textit {rare and weak} setting. In this setting, the sinusoids 
are both of small amplitudes { w.r.t. the noise level} and in small (and unknown) number w.r.t. the number of samples $N$. 
{   When viewed in the Fourier domain, sinusoid detection can } be casted as a sparse heterogeneous mixtures problem, which has attracted much attention in the last decade {\cite{Donoho_2004, Ingster_2010, Walther_2011, Moscovich_2016, Gontscharuk_2016}}. { We will see that while  fixed and adaptive  (in the number of sinusoids) procedures  lead to inconsistencies  when  noise correlations are ignored (because then the statistics under the null hypothesis are wrongly specified), such tests keep their nominal properties, with the CFAR property added, when applied to periodograms  suitably standardized with training data sets. In the particular case of the Higher Criticism \cite{Donoho_2004}, standardization \eqref{ptilde} is  an alternative approach
 to that of \cite{Hall_2010}.}   
\subsection{This study}
The present study  (an extended version of {\cite{Sulis_2016a}}) focuses  on the statistical characterization of test
statistics when both the PSD of the colored  noise and  the parameters of the sinusoids are unknown. We propose a detailed analysis of the effects of periodogram standardization by means   of, say, $L$ training time series{, which are}
 independent realizations of the noise process alone,
and of the gain  that can be expected by using such training signals in a detection framework. \\
We, of course, make the important assumption that such a training data set is  available. Beyond the case of exoplanet detection  considered here,
one may imagine various situations where training signals can be obtained. In astronomical instruments for instances, secondary optical paths are often devoted to monitor `empty'  regions of the sky or calibration stars
{
\cite{Gupta_2001}. \\}
Note that training noise vectors are routinely used for detection in radar systems, with however, an important difference.  
Adaptive test statistics   in radar typically use estimates of the covariance matrix of the training  vectors 
and therefore require $L>N$ for this matrix to be nonsingular. 
This is a very different regime from that considered here, where $L\ll N$.\\
In the present study, we assume that the training data set is unbiased, in the sense that an averaged periodogram obtained from an infinitely large batch would converge uniformly to the true noise PSD. In practice, finite (possibly small) batch sizes can be encountered. For this reason, we say below that the noise is partially unknown and we address the effect of {  small values of $L$} on the detection performances. \\
Because one important objective of this study is to obtain analytical characterization of the test performances, we consider here a regular sampling. Comments on how to relax this assumption are discussed at the end of the paper.
Also,  our results are asymptotic in the number of samples $N$, which is characteristic of time series analysis. However, we will also pay attention to whether asymptotic theory accurately describes reality for finite sample sizes { through simulations}.\\
{ In the considered application framework of exoplanet detection  in RV data the working hypotheses are justified because accurate   simulations of stellar noise can be produced to form training data sets}. These  simulations are however computationally demanding. Obtaining a simulation of 100 days, for a star similar to the one shown  in \cite{Bigot_2011}, takes about 3 months of computing time on 120 cores
on modern clusters. 
Consequently, realistic values of $L$ are in the range of one to, say,  a hundred at most. This motivates the study of the impact of estimation noise in the proposed standardization approach. \\
We proceed as follows. { Sec.} \ref{sec2} presents the model and the detection approach. Sec. \ref{sec3} recalls classical results regarding periodogram's distribution.
Sec. \ref{sec4} and  \ref{sec5}  derive false alarm and detection rates for several tests. Sec. \ref{sec7}
is a numerical study. Table \ref{tab1} summarizes the main notations used in the paper.
  {
\begin{table}[ht!] 
 \caption{Table of notations}
   \label{tab1} 
\begin{tabular}{|c|c|} 
\hline
	$  N$ & Number of data points \\%[0.1cm] 
\hline
	$  X(t_j)$ &  Regularly sampled time series  \\[0.1cm] 
\hline
	$ E(t_j)$ &  Zero-mean stationary Gaussian colored noise  \\[0.1cm] 
\hline
	$\nu,  \nu_k$ &  Continuous frequency, Fourier frequency\\[0.1cm] 
	\hline
	$ S_E(\nu)$ &  Noise PSD  \\[0.1cm] 
\hline
	$ r_E$ &  Noise autocorrelation function \\[0.1cm] 
\hline
	 $  N_s, \alpha_q, f_q, \varphi_q $ &  Parameters of model \eqref{hyp}: Number of sines,\\
	 & sines' amplitude, frequency and phase \\[0.1cm] 
\hline
	$ N_p $ & Number of exoplanets orbiting target star \\[0.1cm] 
\hline
	$\textsf{T}_p, K_p,M_p$ & Planet period and its six Keplerian parameters\\
	$e_p,\omega_p,t_0,\gamma_0$ &  \\%[0.1cm] 
\hline
	$ N_C $ & Proxy for $N_s$ \\[0.1cm] 
\hline
	$ P(\nu)$ &  Classical periodogram \\[0.1cm] 
\hline
	 $  L$ &  Number of available training data sets  \\[0.1cm] 
\hline
	$ \overline{P}_L(\nu)$ &  Periodogram averaged with $L$ training data sets \\[0.1cm] 
\hline
	$ \widetilde{P}(\nu|\overline{P}_L)$ &  Periodogram standardized by $ \overline{P}_L$ \\[0.1cm] 
\hline
	$ \Omega $ & Indices set of considered Fourier frequencies \\[0.1cm] 
\hline
	$ \lambda_k := \lambda(\nu_k)$ & Non centrality parameter \\[0.1cm] 
\hline
	$ F_{\lambda_k}(d_1,d_2) $ & Non central Fisher-Snedecor distribution \\
	& with $d_1$ and $d_2$ degrees of freedom \\[0.1cm] 
\hline
	$Z,z $ & Scalar random variable, one realization of $Z$ \\[0.1cm] 
\hline
	$ v_Z(z) $ & Pr$(Z>z)$ (observed p-value)  \\[0.1cm] 
\hline
	$ V_{Z} $ & P-value as a random variable ($ V_{Z}\sim {{\cal{U}}_{[0\; 1]}}) $ \\
\hline
	${\bf{Z}}=[Z_1,\hdots,Z_N]^\top $ & Vector of random variables \\[0.1cm] 
\hline
	$ V_{{\bf{Z}},k} $ & P-value associated to component $Z_k$ of  ${\bf{Z}}$\\[0.1cm] 
\hline
	$ V_{{\bf{Z}},(k)} $ & k-th ordered p-value of  {\bf{Z}} \\[0.1cm] 
\hline
\end{tabular}
\label{default}
\end{table}}
  
 
\section{Statistical model and detection approach}
 \label{sec2}
We consider the two hypotheses: 
\begin{equation} 
  \left\{         
      \begin{aligned}
	 \text{ ${\cal{H}}_0$ : } {X}(t_j) &= {\displaystyle{{{E}}}}(t_j) \\
	 \text{ ${\cal{H}}_1$ : } X(t_j) &=  \sum_{q = 1}^{N_s} \alpha_q \sin(2\pi f_q t_j+\varphi_q)+ E(t_j) \\
      \end{aligned}
    \right.
    \label{hyp}
\end{equation}
where $X(t_j=j\Delta t),\; j=1,\hdots,N$, is an evenly sampled data time series, $E(t_j)$ is a zero-mean second-order stationary Gaussian noise, 
{ with $\inf (S_E(\nu) ) > 0$ and ${ \sum_{u \in \mathbb{R}}} | r_E(u) | < \infty$}.   This is the noise of which we assume a set of training time series is available. { To simplify the presentation, we consider for the rest of the paper a unit  sampling step $\Delta t=1$ in \eqref{eq1}.}

 Under the alternative, the $N_s$  amplitudes $\alpha_q\in \mathbb{R}^{*+}$, frequencies  $f_q\in \mathbb{R}^{*+}$ and phases $\varphi_q \in [0,2\pi[$ of the deterministic part are unknown.  In RV exoplanet detection, this deterministic part represents the planetary signature(s). 
 
{
Note that model \eqref{hyp} can actually be useful for the detection of periodic signals more general than only pure sinusoids.
In such cases, the Fourier spectrum may contain many harmonics. Because the fundamental frequency has zero probability to coincide with a Fourier frequency, the corresponding number of nonzero Fourier coefficients (\textit{i.e.}, of deviations under the alternative) always equals $N$.
However, most of the energy of periodic signals is captured by a small fraction of Fourier coefficients, so that model \eqref{hyp}  would often be accurate for such signals
 with some $N_s\ll N$. \\
 In the case of RV signals for instance, a study of the influence of the Keplerian parameters (planets' { orbital} parameters) shows that this  is indeed the case \cite{Sulis_2016c}.
In all  cases except perhaps very rare and 
exotic systems, the RV spectral signatures exhibit only a small fraction { of} significant harmonic{s} because the planets 
 tend to have low eccentricities and are in small number ($N_p$). In short, the spectrum is sparse (though not strictly sparse) and RV signals can be modeled by a sum of a
small number of  pure sinusoids.
We call this number $N_s$, and we say that $N_s \ll N$. When there is one planet with frequency close to the Fourier grid, $N_s$
will be essentially $1$. In our simulations { for} Sec. VII, we found that  multiplanetary systems with $5$ eccentric planets behave
like model (3) with $N_s$ not exceeding, say,  $20$ at most. 
}

\section{Periodograms' statistics: asymptotics}
 \label{sec3}
  
\subsection{Classical (Schuster's) periodogram}
 
The  frequencies considered { in 
 \eqref{eq1} will be}  Fourier frequencies $\{\nu_k:=\frac{k}{N}\}_{k=0,\hdots,N-1}$ and  $N$ is considered even. For simplicity but without loss of generality we will often consider the subset of $(\frac{N}{2}-1)$  Fourier frequencies corresponding to $k\in \Omega := \{ 1,\hdots,  {\frac{N}{2}-1}\}$.
Asymptotically, the  periodogram $P$ in 
{ \eqref{eq1}} is an unbiased but inconsistent estimate of the PSD \cite{Brillinger_1981}. 
   Under the above assumptions on $E$, the periodogram ordinates at different frequencies $\nu_k$ and $\nu_{k'}$ are asymptotically independent  \cite{Li_2014}. \\
Under  ${\cal{H}}_0$, the asymptotic distribution of $P$ is (Th. 5.2.6,\cite{Brillinger_1981}):
\begin{equation}  P(\nu_k | {\cal{H}}_0) \sim
  \left\{         
      \begin{aligned}
	& \frac{S_E(\nu_k)}{2}  \chi^2_2 , ~~~~ \forall k ~ \in ~ \Omega,   \\
	&  S_E(\nu_k) \chi^2_1, ~ \text{ for } ~ k = 0, \frac{N}{2}.
      \end{aligned}
    \right.
    \label{dist}
\end{equation}
 Under ${\cal{H}}_1$, the distribution of $P$ is known when cisoids are present  in (\ref{hyp}) (\cite{Li_2014}, Corollary 6.2(b)).
The real case of  model  (\ref{hyp}) can be treated similarly (see  Appendix \ref{app1}). This leads to
the asymptotic distribution:
\begin{equation}  P(\nu_k | {\cal{H}}_1) \sim
  \left\{         
      \begin{aligned}
	 & \frac{S_E(\nu_k)}{2}  \chi_{2,  \lambda_k}^2 ,~ \forall k ~ \in ~ \Omega,  \\
	 &{S_E(\nu_k)}  \chi^2_{1,  \lambda_k}, ~ \text{ for } ~ k = 0, \frac{N}{2}.
      \end{aligned}
    \right.
    \label{dist2}
\end{equation}
The  $\{\lambda_k := \lambda(\nu_k)\}$ are  non centrality parameters given for $k ~ \in ~ \Omega$ by
\begin{equation} \hspace{-3mm}
  \label{lambda}  
	 \lambda_k \!= \!  \frac{N}{2 S_E(\nu_k)}   \sum_{q=1}^{N_s}  \Big[  \alpha_q^2 \kappa_q ^2 + 2 \alpha_q \kappa_q \sum_{\ell = q+1}^{N_s}  \alpha_\ell \kappa_\ell  \cos(\theta_q-\theta_\ell )\Big],
\end{equation}
and for $k = 0, \frac{N}{2}$ this expression is halved.
The terms $\kappa_q$ and $\theta_q$, given by \eqref{eq_keppa} and \eqref{eq_theta} in Appendix A, arise from  spectral leakage
{  through the spectral window $K_N$  \eqref{eq_Kn}.
Owing to the fast decay of $K_N(\nu)$,
the proportion of parameters  $\lambda_k$ that significantly differ from $0$ is small if $N_s\ll N$.}

\subsection{Averaged periodogram}
We assume that a training data set $\mathcal{T}$ of independent realisations of the { colored }noise is available. This set  is obtained by $L$ independent simulations corresponding to $L$ time series $X_\ell$ sampled on the same grid as the observations:  $\mathcal{T}={\big\{}\{X_\ell(t_j)\}_{j=1,\hdots,N}{\big\}}_{\ell=1,\hdots,L}$.
A straightforward estimate of the noise PSD {  is} the averaged periodogram \cite{Bartlett_1950}:

$$ 
	\overline{P}_L(\nu_k |  {\cal{H}}_0)  : =  \frac{1}{L} \sum_{\ell=1}^{L}  \frac{1}{N} \Big| \sum_{j=1}^{N} X_\ell(t_j)\mathrm{e}^{-{\rm{i}}2\pi\nu_k j} \Big| ^2.
$$

Using \eqref{dist}, the asymptotic distribution of $\overline{P}_L$ {  is:}
\begin{equation}
 \overline{P}_L(\nu_k | {\cal{H}}_0)  \sim
  \left\{         
      \begin{aligned}
	  &\frac{S_E(\nu_k)}{2L}  \chi_{2L}^2,  ~ &\forall k ~ \in ~ \Omega, \\
	   &\frac{S_E(\nu_k)}{L}  \chi_{L}^2, ~ &\text{ for } ~ k = 0, \frac{N}{2}.
      \end{aligned}
    \right.
    \label{dPmoy}
\end{equation}
$\overline{P}_L$ is a consistent and unbiased estimate of $S_E(\nu)$ when both $L\rightarrow \infty$ and $N \rightarrow \infty$ (\cite{Proakis_1996}, Chap.14).

The effect of  stochastic estimation noise caused by the finiteness of $\mathcal{T}$ is encapsulated in $L$. This clearly impacts the distribution
of $\overline{P}_L$ in \eqref{dPmoy} and in turn the efficiency of the subsequent standardization by $\overline{P}_L$. 
\subsection{Periodograms standardized with $\overline{P}_L$}
\label{standard}
We now turn to  statistical properties of standardized periodograms of the form \eqref{ptilde}.
When the averaged periodogram $\overline{P}_L$ is used, this yields:
\begin{equation} 
\label{eq_pr}
	 \widetilde{P}(\nu_k \;|\; \overline{P}_L) := \frac{P(\nu_k)}{\overline{P}_L(\nu_k)}.
\end{equation}
\indent As the numerator and denominator are independent variables with known asymptotic distributions, assessing the distribution of their ratio is straightforward. 
The ratio of two  independent random variables (r.v.)  $V_1\!\sim\! \chi_{d_1}^2$ and $V_2\!\sim\! \chi_{d_2}^2$ follows a Fisher-Snedecor law  noted $F(d_1,d_2)$ with ($d_1,d_2$) degrees of freedom: $\frac{V_1/d_1}{V_2/d_2} \sim F(d_1,d_2)$ \cite{Abramowitz_1972}. \\
Consequently, from (\ref{dist}) and (\ref{dPmoy}), the asymptotic distribution of this standardized periodogram under $\mathcal{H}_0$ is:
\begin{equation} \hspace{-3mm}
\widetilde{P}(\nu_k   |  \overline{P}_L, {\cal{H}}_0)  \! \sim \!
  \left\{         
      \begin{aligned}
	   &  \! \frac{S_E(\nu_k)\chi_2^2/2}{S_E(\nu_k)\chi_{2L}^2/2L}  \!  \sim F(2,2L),  \forall k  \in  \Omega ,\\
	    & \!  \frac{S_E(\nu_k)\chi_1^2}{S_E(\nu_k)\chi_{L}^2/L}  \! \sim F(1,L),   \text{ for }  k  \!= \! 0, \frac{N}{2}.
      \end{aligned}
    \right.
    \label{dist3}
\end{equation}
Similarly, from (\ref{dist2}) and  (\ref{dPmoy}), { we have  under $\mathcal{H}_1$}:
\begin{equation}
 \! \widetilde{P}(\nu_k |  \overline{P}_L, {\cal{H}}_1) \! \sim \!
  \left\{         
      \begin{aligned}
	   &\frac{\chi_{2, \lambda_k}^2/2}{\chi_{2L}^2/2L} \sim F_{\lambda_k}(2,2L),   ~ \forall k  \in  \Omega, \\
	    &\frac{\chi_{1, \lambda_k}^2}{\chi_{L}^2/L} \sim F_{\lambda_k}(1,L),  \text{ for }  k = 0, \frac{N}{2},
      \end{aligned}
    \right.
	\label{dist_H1}
\end{equation}
where $F_{\lambda_k}$ denotes a non-central $F$ distribution with non centrality parameter  $\lambda_k$ given by (\ref{lambda}).

Under the null hypothesis,  \eqref{dist3} shows that the distribution of the standardized periodogram is independent of the nuisance signal, \textit{i.e.}, of the partially unknown noise PSD. This  property is  important as it will be inherited by some of the test statistics investigated in Sec. \ref{sec6}, leading to  CFAR tests.
\section{Considered tests}
 \label{sec4}
\subsection{Preliminary notations}
{ These tests are better presented using $P$-values\footnote{We will denote the $P$-values 
by $V$ because $P$ denotes the periodogram in this work.} and order statistics. When necessary
the notation will distinguish between a r.v.  $Z$ and its realization $z$. We recall that the observed 
$P$-value $v_Z$ 
 { is defined as:}
$$
	v_Z(z)  := \textrm{Pr\;} (Z>z)
$$
and $v_Z$ is one realization of the r.v. $V_Z$, which is uniformly distributed.
Similarly,  for a vector of r.v.  ${\bf{Z}}=[Z_1,Z_2,\hdots,Z_N]^\top$ of which ${\bf{z}}=[z_1,z_2,\hdots,z_N]^\top$
 is one realization we  will denote by
 $$ \displaystyle{\min_k\;} {{z_k}}:=z_{(1)}<z_{(2)}< \hdots<{ z_{(N)}} := \max_k \; {{z_k}}$$
 the ordered values of ${\bf{z}}$ and by $Z_{(1)},\hdots,Z_{(N)}$ the  order statistics of  ${\bf{Z}}$.
The observed $P$-values corresponding to ${\bf{z}}$ will be denoted by $v_{{\bf{Z}},k}$ (with $v_{{\bf{Z}},k} := \textrm{Pr\;} (Z_k>z_k)$)
and the observed ordered  $P$-values by $v_{{\bf{Z}},(k)}$. The corresponding r.v. will be denoted  by
 $V_{{\bf{Z}},k}$ and $V_{{\bf{Z}},(k)}$. The ordered $P$-values $V_{{\bf{Z}},(k)}$ are not uniform, because they are order statistics 
 from a uniform distribution, and obviously dependent{\cite{David_2003}}.

 
 }

We now present  some  tests discussed in the Introduction and selected for reference as they  cover different { classical models}. 
Under { assumptions specified below} on the distributions of  the variates $\{Z_i\}$,  the properties of these tests are {
 known and we shall summarize them}. 
 
 
 \subsection{ { Test} statistics}
 \label{Sec_tests_cl}
 All tests below are of the form
  ${\rm T}({\bf{z}})  \mathop{\gtrless}_{\mathcal{H}_0}^{\mathcal{H}_1} \gamma$,  
  with  { ${\bf{z}}$ the data (which below will be a set of  ordinates of one of the periodograms discussed above),  T$(\cdot)$ the test statistic and $\gamma\in \mathbb{R^+}$  a threshold that determines the false alarm rate. \\
  
}
  
  
  
  \label{sec31}
   \subsubsection{ Test of the maximum}
   \label{Maxtest}

   \begin{equation}
	{\rm {T}_{M}}({\bf{Z}}) :=Z_{(N)}.
 	 \label{lemax}
\end{equation}
For independent variates $Z_i$ of Cumulative Distribution Function (CDF) $\Phi_{Z_i}$, the false alarm of the test is $ {\rm P_{FA}}(\gamma)=1-\prod_{i=1}^N \Phi_{Z_i}(\gamma)$. 
 For a model involving  under $\mathcal{H}_1$ a single sinusoid { with unknown  frequency (but on the Fourier grid) }and under  $\mathcal{H}_0$ a white Gaussian noise { (WGN)} of known variance $\sigma^2$, the ordinates of the periodogram ${\bf{P}}$  evaluated at successive Fourier frequencies are under  $\mathcal{H}_0$   independent and identically distributed (i.i.d.) with distributions given by \eqref{dist}, where $S_E=\sigma^2$.
  For this model  ${\rm T_M}({\bf{P}})$  corresponds to the GLR test  \cite{Kay_1998}. \\
  
 \subsubsection{Fisher's test}
\begin{equation} 
	\label{eq3b}
	{\rm {T}_{F}}({\bf{Z}}):= \frac{Z_{({N})}}{\displaystyle{\sum_{k ~ } Z_{(k)}}}.
\end{equation}

When applied to periodogram of { WGN} of unknown variance, Fisher's test is CFAR  (the distribution of the test statistics is established in \cite{Fisher_1929}). {This test is actually the GLR test under the model of a single sinusoid on the Fourier grid and WGN of unknown variance (see  \cite{Quinn_1986}, who also covers the case of more than one sinusoid).
}
 Examples of works using this test in Astronomy are\cite{Koen_2015a,Koen_2015b,Schwarzenberg_1998,Aittokallio_2001,Guitierrez_2009}. \\
 
  \subsubsection{A test inspired by the tests of Chiu and Shimshoni}
\begin{equation} 
\label{test_ch}  
      \begin{aligned}
{\rm T_C}({\bf{Z}}) := {Z_{({N}-N_C+1)}},
	\end{aligned}
\end{equation} 
 {with $N_C\geq 1$ a parameter related to the number of sinusoids. }
  
This test statistic is justified by the observation made in \cite{Shimshoni_1971,Chiu_1989} that for multiple sinusoids, order statistics
different from the maximum may be more discriminative {than $Z_{(N)}$} against the null. 
These tests are designed for periodograms of white noise of unknown variance and their asymptotic false alarm rates are given in  \cite{Shimshoni_1971,Chiu_1989}.
As Fisher's test, they involve denominators whose purpose is normalization by consistent estimates of the noise variance.
${\rm T_C}$ is a simplification of these tests: the normalization is ignored, because it will not be necessary to yield a CFAR detector
once applied to periodograms standardized by $\overline{P}_L$. The false alarm rate of test  ${\rm T_C}$ for white noise can be deduced from the expression obtained in Sec. \ref{TC}.

As for the tests \cite{Shimshoni_1971,Chiu_1989}, we expect ${\rm T_C}$ to have decreasing power in case of strong mismatch between the value of parameter $N_C$ and the  {  number of detectable deviations under   ${\mathcal{H}}_1$ (roughly speaking, $N_s$)}.
Not fixing $N_C$ in advance but estimating this parameter from the data may lead to more powerful tests, but at the cost of a more difficult control of the FA rate (as $N_C$ becomes random). 
This suggests to consider other approaches that are adaptive in the number of sinusoids, which is the case of the last two tests \eqref{HC_test} and \eqref{BJ}.\\

  \subsubsection{Higher Criticism}
  \label{hc}
{This   test statistic  is defined by  \cite{Donoho_2004}}:
{
 \begin{equation}  
		 { \rm HC^{\star}}({\bf{  Z}}):= \!\!\displaystyle{\max_{1 \le { k \le \alpha_0 N}}  \frac{\sqrt{N}(k /N - v_{{\bf{Z}},( k )})}{\sqrt{v_{{\bf{Z}},(k)}(1-v_{{\bf Z},({ k })}})}},
		\label{HC_test}
\end{equation}	
}
where $v_{{\bf Z},(k)}$ are ordered $P$-values and parameter $\alpha_0\in [\frac{1}{N},1]$.

HC is  designed under the assumption that under the null hypothesis the  ordinates $\{Z_k\}$ are  i.i.d. with known marginal distribution.
When ${\bf Z}=\frac{2{ \bf{P}}}{\sigma_E^2}$, with ${\bf P}$ the periodogram of a white noise of known  variance $\sigma_E^2$ under the null, this distribution is given by \eqref{dist} with  $S_E=\sigma_E^2$. 
 {
 The ordered $P$-values involved  in \eqref{HC_test} are thus ordered values of
\begin{equation}
v_{  {2{ \bf{P}}}/{\sigma_E^2}  ,\;k}
  \left\{         
      \begin{aligned}
	   &1-\Phi_{\chi^2_{2}}(P(\nu_k)),   ~ \forall k ~ \in ~ \Omega, \\
	    &1-\Phi_{\chi^2_{1}}(P(\nu_k)),   ~ \text{ for } ~ k = 0, \frac{N}{2}.
      \end{aligned}
    \right.
    \label{pval1}    
\end{equation}
}
Under the alternative, a fraction of the ordinates contain a deterministic part, and hence follow \eqref{dist2}, with  $S_E=\sigma_E^2$ ({see Sec. 1.7  of \cite{Donoho_2004}}). The frequencies at which these deviations occur are unknown. Their magnitudes ({ related to} $\lambda_k$) are also unknown and weak, in the sense that they are comparable to the expected magnitude of the periodogram maximum  under the null hypothesis. 

Optimality\footnote{A test is said optimal in \cite{Donoho_2004} if the sum of the probability of error and the probability of missed detection tends to zero.} results of HC are  asymptotic and established for a specific sparsity \textit{vs} amplitude model. 
When the deviation occurs at a single frequency or for extremely sparse signatures in the Fourier domain, 
\cite{Donoho_2004} showed that the test based on the maximum  \eqref{lemax} is asymptotically optimal in the sense that whenever the Neyman-Pearson test has full power, test \eqref{lemax} has full power as well. In such cases, the largest periodogram ordinate (or, equivalently,  the smallest $P$-value) is the most powerful test statistic to discriminate against the null.\\
As discussed about the tests \cite{Shimshoni_1971,Chiu_1989} and ${\rm T_C}$ above, better regions than the maximum ordinate/smallest $P$-value
may be useful for sparse but not extremely sparse signatures. 
\cite{Donoho_2004} showed that there exists sparse signals of weak amplitude that can be  optimally detected (in the sense of full power) by HC but not by ${\rm T_M}$. This makes HC  particularly interesting in the present study.\\
 \subsubsection{Berk-Jones }
 \label{bj}
A  test statistic related to $\rm HC$ is that of Berk-Jones \cite{Moscovich_2016,Aldor_2013,Mary_2014,Kaplan_2014,Gontscharuk_2014} defined by:
{
 \begin{equation}  
	{ \rm BJ({\bf{Z}})}:= \!\!\displaystyle{\max_{1 \le k \le \alpha_0 N}} I_{1-v_{{\bf Z},(k)}} (N-k+1,k), 
	\label{BJ}
\end{equation}	
}
where $I$ denotes the regularized incomplete beta function  \cite{Abramowitz_1972}.\\
The deviations in form of $Z-$scores in \eqref{HC_test} for HC are established using  the asymptotic convergence of a binomial distribution to a Gaussian distribution. In the tails, however, this convergence is very slow (see{\cite{Mary_2014, Li_2015,Moscovich_2016}} for illustrations). For this reason the test statistics \eqref{BJ}, which compares favorably to HC and other goodness-of-fit (GOF) tests in some cases, was recently   (and almost simultaneously) proposed by{ \cite{Moscovich_2016,Aldor_2013,Kaplan_2014,Gontscharuk_2014}}. As noted in{ \cite{Moscovich_2016, Gontscharuk_2016}},  this test was initially proposed by Berk and Jones (and called $M_n^+$) in \cite{Berk_1979}.\\
This test is based on the exact significance reflected by the $P$-values, that is, on the $P$-values of the ordered $P$-values. Since the ordered $P$-values are Beta distributed, with  { $V_{{  {2{ \bf{P}}}/{\sigma_E^2} },(k)}\sim \text{Beta}(k,N-k+1)$, their $P$-values involve the CDF of Beta variables, which is an incomplete Beta function: $\Pr(\text{Beta}(k,N-k+1)\leq x)=I_x(k,N-k+1)$ } and leads to test \eqref{BJ} with the $P$-values computed as in \eqref{pval1}.\\
 $\rm BJ$ presents the same adaptive optimality as $\rm HC$ for sparse mixture detection, and the asymptotic distribution of the $\rm BJ$ test statistic  can be found in Th. 4.1 of  \cite{Moscovich_2016}.  As for $\rm HC$, convergence to the asymptotic distribution may be slow. { Efficient algorithms for computing significance levels  (and hence the function $\gamma \mapsto {\rm P_{FA}}(\gamma)$)  of $\rm HC$ and $\rm BJ$ for finite (but possibly large) values of $N$ can be found in\cite{Moscovich_2016,Moscovich_2015b}. $\rm BJ$ and $\rm HC$  are studied from the viewpoint of  local levels in {\cite{Gontscharuk_2016}}.}
\section{ Tests { applied to $\bf{Z}=$}  $\widetilde{\bf P}\; |\; \overline{\bf P}_L$}
 \label{sec5}
 

To simplify the presentation of the results we restrict in the following to the frequency set  $\Omega$, \textit{i.e.},
to standardized vectors
$$
{\bf{\widetilde{P}\;|\;\overline{P}}}_L:=\left[\frac{P(\nu_1)}{\overline{P}_L(\nu_1)},\hdots, \frac{P(\nu_{\frac{N}{2}-1})}{\overline{P}_L(\nu_{\frac{N}{2}-1})}\right]^\top.
$$
Extension of the results
to { $\nu_0,\nu_{\frac{N}{2}}$} can be obtained using the distributions \eqref{dist3} and \eqref{dist_H1}.

{
In the following we evaluate false alarm and detection rates by postulating independence of the considered ordinates.
While the asymptotic independence of the periodograms ordinates at positive Fourier frequencies is well established
(\textit{e.g.} \cite{Fan_2003}, Th. 2.14), theoretical results regarding the joint distribution of periodogram ordinates under departures from whiteness (and Gaussian) assumptions are lacking, however. For some results on the largest order statistics, see \cite{Turkman_1984}, who consider MA Gaussian processes, and \cite{Davis_1999}, who consider non-Gaussian sequences. For finite values of $N$, the approaches of \cite{Rife_1974,Quinn_1994} 
might be followed to better characterize the performances of some tests considered below.

In practical situations, the marginal distributions of the considered ordinates are only approximated by their asymptotic distribution 
(this is visible in Eq. \eqref{lerho} for instance).
Besides, as $N$ grows the correlation between the periodogram ordinates at the signal frequencies approaches zero at a slower rate than the correlation between other ordinates ($\!\!$\cite{Li_2014}, Theorem 6.5), indicating that departure to the independence assumption may be more pronounced under ${\mathcal{H}}_1$ than ${\mathcal{H}}_0$. Consequently,  we are using the independence as an operational assumption to quantify  the tests' performances, and the validity of the resulting expressions  below should be checked against numerical simulations. Sec. \ref{sec7} provides several examples.   }
 
Under ${\cal{H}}_0$ and in the asymptotic regime, ${\rm T_M}({\bf \widetilde{\bf P}}\;|\;\overline{\bf P}_L)$ 
is the maximum of independent variables given by \eqref{dist3}. For $k \in \Omega$ these $\frac{N}{2}-1$ variables follow an $F(2,2L)$ distribution with general density given in \cite{Abramowitz_1972}. Using the beta function ${\cal{B}}(1,L):=\int_0^1(1-t)^{L-1}\textrm{d}t=\frac{1}{L}$, this density $\varphi_F(\gamma,2,2L)$ can be expressed as
$$
\varphi_F(\gamma,2,2L)=\frac{1}{{\cal{B}}(1,L)}\cdot\frac{1}{L}\cdot \Big(1+\frac{\gamma}{L}\Big)^{-L-1}=\Big(1+\frac{\gamma}{L}\Big)^{-L-1}.
$$
It can be checked that $\int_0^\infty\varphi_F(\gamma,2,2L)\mathrm{d}\gamma=1$. 
The corresponding CDF  $\Phi_F(\gamma,2,2L)$ is obtained  by  integration of $\varphi_F$:
\begin{equation} 
	\Phi_F(\gamma,2,2L) = \displaystyle \int_{0}^{\gamma} \varphi_F(\gamma,2,2L) \mathrm{d}\gamma = 1 - \Bigg(\frac{L}{L+\gamma} \Bigg)^L.
	\label{laphi}
 \end{equation}
The probability of false alarm ($\rm{P_{FA}}$) can be computed thanks to the  asymptotic independence of the ordinates of ${\widetilde{\bf P}\; |\; \overline{\bf P}_L}$:
\begin{equation} 
\begin{aligned} 
	&\!\rm{P_{FA}}({{\rm T_M}} ({\bf \widetilde{\bf P}}\;|\;\overline{\bf P}_L) ,\gamma) := \textrm{Pr}\; ({\rm T_M}({\bf \widetilde{\bf P}}\;|\;\;\overline{\bf P}_L)\! > \!\gamma | {\cal{H}}_0\!) \\
		&=1 -\displaystyle{ \prod_{k\in \Omega}} \textrm {Pr}\, \left(\widetilde{P}({ \nu_k})  \leq \gamma | {\cal{H}}_0,\overline{P}_L\right) = 1 - \Big(\Phi_F(\gamma,2,2L)\Big)^{\frac{N}{2}-1}\\
		&= 1-  \Big( 1-\Big(\frac{L}{\gamma+L}\Big)^L\Big)^{\frac{N}{2}-1}.
\end{aligned}
\label{pfa}
 \end{equation}
 
The probability of false alarm  is { (asymptotically)} independent of the  noise PSD, which makes   ${\rm T_M}({\bf \widetilde{\bf P}}\;|\;\overline{\bf P}_L)$ a CFAR detector. 

 Under ${\mathcal{H}_1}$, using (\ref{dist_H1}) and the approximate independence of periodogram ordinates (\cite{Li_2014}, Theorem 6.5), the probability of detection ($\rm{P_{DET}}$) of ${\rm T_M} ({\bf \widetilde{\bf P}}\;|\;\overline{\bf P}_L)$ can be  approximated as: 
   \begin{equation} 
  \begin{aligned} 
	\!\!\!\!\rm{P_{DET}}({{\rm T_M}} ({\bf \widetilde{\bf P}}\;|\;\overline{\bf P}_L) ,\gamma)&:=  \textrm{Pr} \left({\rm T_M}({\bf \widetilde{\bf P}}\;|\;\;\overline{\bf P}_L)\! > \!\gamma | {\cal{H}}_1\! \right) \\
	&\approx 1 - \displaystyle{ \prod_{k \in \Omega}} \Phi_{F_{\lambda_k}}(\gamma, 2,2L).
	\end{aligned}
		\label{Eq_Pdet}
\end{equation}	
With (\ref{pfa}),  the relationship $\gamma(\rm{P_{FA}})$ for ${{\rm T_M}}$ can be derived as:
 \begin{equation} 
\label{Eq_gam}
		\gamma({{\rm T_M} ({\bf \widetilde{\bf P}}\;|\;\overline{\bf P}_L}),\rm{P_{FA}}) =  L  \Big[ \Big( 1 - ( 1 - \rm{P_{FA}})^{\frac{1}{\eta}} \Big)^{-\frac{1}{L}} -1 \Big],   
\end{equation}	
{where $\eta := \frac{N}{2}-1$}. With (\ref{Eq_Pdet}) and (\ref{Eq_gam}), we  deduce $\rm{P_{DET}(P_{FA})}$ which can be used to compute ROC (Receiver Operating Characteristics) curves:
 \begin{equation} 
   \begin{aligned} 
\label{Eq20}
	&{\rm{P_{DET}}({{\rm T_M}} ({\bf \widetilde{\bf P}}\;|\;\overline{\bf P}_L)},{\rm{P_{FA}}})  \\
	&\approx 1 - \displaystyle{ \prod_{k\in \Omega }}  \Phi_{F_{\lambda_k}}(L  [ ( 1 - ( 1 - \rm{P_{FA}})^{\frac{1}{\eta}} )^{-\frac{1}{L}} -1 ],2,2L)
		\end{aligned}
\end{equation}	
 
 \subsection{${\rm T_F}$.}
 \label{fisher}
{ Under $\mathcal{H}_0$,} the Fisher's test ${\rm T_F}$ applied to  ${\bf \widetilde{\bf P}}\;|\;\overline{\bf P}_L$ looks for the maximum of identically distributed variables, which from  (\ref{dist3}) and (\ref{eq3b})
{ correspond to the} ratio  of a $F$ variable over a sum of $F$ variables. To our knowledge, there is no { analytical} characterization of the resulting distribution
{ for finite values of $L$. }
Hence, although this test is CFAR, { computing} the false alarm rate is problematic. Resorting to Monte Carlo simulations to evaluate the function $\gamma \mapsto \rm{{P}_{FA}}$ is not possible, owing to the limited number of available noise realizations.
 Similar remarks can be made about standardized versions of other tests like \cite{Shimshoni_1971, Siegel_1980,Chiu_1989}.
\subsection{${\rm T_C}$.}
\label{TC}
We turn to ${\rm T_C}({\bf \widetilde{\bf P}}\;|\;\overline{\bf P}_L,N_C) $, where parameter $N_C$ allows to focus on the regions of order statistics 
where  deviations under the alternative are likely to be  significant. The $\rm{P_{FA}}$ can be obtained by observing
 that, owing to \eqref{dist3}, the number $K$ of standardized ordinates larger than $\gamma$  under $\mathcal{H}_0$ follows a binomial distribution:
 {
$	 K\sim \rm{Bin}(N/2-1,1-\Phi_{F}(\gamma,2,2L)),$
  whose CDF is: 
 $	\textrm{Pr}\;(K\leq k)=I_{\Phi_{F}(\gamma,2,2L)}(N/2-N_C,N_C), $ see \cite{Abramowitz_1972}.
}
 Using \eqref{test_ch}, (\ref{laphi}) and noting
\begin{equation*}  
 	u :=  1-\Phi_{F}(\gamma,2,2L) = \Big( \frac{L}{\gamma+L} \Big)^L,
 \end{equation*}
the $\rm{P_{FA}}$ of this test applied to ${\bf \widetilde{\bf P}}\;|\;\overline{\bf P}_L$ can be expressed as:
\begin{equation}  
\begin{aligned} 
	&\!\!\rm{P_{FA}}({\rm T_C}( {\bf \widetilde{\bf P}}|\overline{\bf P}_L, N_C),\gamma) \!:=\! \textrm{Pr}\; ( {T_C  ({\bf \widetilde{\bf P}}|\overline{\bf P}_L,N_C)}\!\! > \!\gamma | {\cal{H}}_0\!)\! \\
	&= 1- { \sum_{k=0}^{N_C-1} } \textrm{Pr}\; ( K=k\;|\;{\cal{H}}_0) = 1 -  I_{1-u}( \frac{N}{2}-N_C, N_C)  \\
	
	&=   I_{u} (N_C, \frac{N}{2}-N_C),
\end{aligned}
\label{pfa_Tor}
 \end{equation}
 where the last equation uses   $I_x(a,b)=1-I_{1-x}(b,a)$  (\textit{cf} Prop. 6.6.3 in \cite{Abramowitz_1972}).
As ${\rm T_M}({\bf \widetilde{\bf P}}\;|\;\overline{\bf P}_L)$, this test is CFAR.

As a side remark, note that the false alarm rate of the test ${\rm T_C}$ applied to the periodogram of a white noise of known variance is obtained as above,
by replacing $\Phi_F$ with the CDF of $\chi^2$ variables according to \eqref{dist}.

 The function $\gamma \mapsto \rm{P_{DET}} (\gamma)$ of ${\rm T_C}( {\bf \widetilde{\bf P}}\;|\;\overline{\bf P}_L,N_C)$  can be deduced similarly to (\ref{pfa_Tor}), with the difference that $K$ is no longer binomial owing to the $\{\lambda_k\}$.  Appendix \ref{app6} shows that
  \begin{equation}  
 \begin{aligned} 
 		&\rm{P_{DET}}( T_C( {\bf \widetilde{\bf P}}\;|\;\overline{\bf P}_L,N_C),\!\gamma):= \textrm{Pr}\; (\!  T_C( {\bf \widetilde{\bf P}}\;|\;\overline{\bf P}_L,N_C)\!  > \gamma | {\cal{H}}_1) \\
		&\approx 1 -\!\! \displaystyle{ \sum_{i=0}^{N_C - 1} \sum_{\Omega^{(i)} \in \Omega^i} \prod_{k=1}^{ i  } } \Big(1 - \varphi_{F_{\lambda_{\Omega^{(i)}_k}}\!^{\!\!\!\!\!\!\!\!\!(\gamma, 2,2L)}} \Big) \displaystyle{ \!\!\!\! \prod_{k'=1}^{ \frac{N}{2}-1-i }}\!\!\varphi_{F_{\lambda_{ \overline{\Omega}^{(i)}\!_{\!\!\!\!\!\!k'} }}{(\gamma, 2,2L)} },
		\label{lagrosse}
\end{aligned} 
\end{equation}	
which can be used with \eqref{pfa_Tor} to compute ROC curves. The non centrality parameters $\{\lambda_{\Omega_k^{(i)}}:= \lambda(\nu_{\Omega_k^{(i)}})\}$ and $\{\lambda_{\overline{\Omega}_{k'}^{(i)}}:= \lambda(\nu_{\overline{\Omega}_{k'}^{(i)}})\}$ are given by (\ref{lambda}) with the notation defined in \eqref{pfac}.
{ Note that the tests $\rm T_M$ and $\rm T_C$  are both working on order statistics of  $P$-values, which are Beta random variables, so in both cases the $\rm{P_{FA}}$ is given by the tail of a Beta distribution. }

\subsection{${\rm HC^\star}$ and ${\rm BJ}$.}
 \label{sec_HC}
 When applied to ${\bf \widetilde{\bf P}}\;|\;\overline{\bf P}_L$, the $P$-values involved in the HC test \eqref{HC_test} and in the BJ test 
 \eqref{BJ} should be computed according to the distribution under the null given by \eqref{dist3}. Hence,  \eqref{pval1} is replaced by:
\vspace{-0.4cm}
{
$$
\vspace{-0.2cm}
{ v_{{\bf \widetilde{\bf P}}|\overline{\bf P}_L,\;k}}  :=
  \left\{         
      \begin{aligned}
	   &1-\Phi_F\left(\frac{P(\nu_k)}{\overline{P}_L(\nu_k)}, 2, 2L\right) ,   ~ \forall k ~ \in ~ \Omega, \\
	    &1-\Phi_F\left(\frac{P(\nu_k)}{\overline{P}_L(\nu_k)}, 1, L\right),   ~ \text{ for } ~ k = 0, \frac{N}{2}.
      \end{aligned}
    \right.
$$
}
The properties of the two tests are otherwise left unchanged, with the CFAR property added:  thanks to the standardization of ${{\bf P}}$ by $\overline{\bf P}_L$, the $P$-values are
independent of the noise PSD. 

\section{ Tests { applied to ${\bf{Z}}=$ }  $\widetilde{\bf P} \;|\;\widehat{\bf S}_E$}
\label{sec6}
Estimates of  $\widehat{S}_E$ different from the  averaged periodogram $\overline{P}_L$ can be used for standardization in \eqref{eq_pr}. 
Sec. \ref{rel} has reviewed some  methods to obtain such estimates.
For the purpose of comparing  $\widetilde{\bf P} |\widehat{\bf S}_E$ with  ${\bf \widetilde{\bf P}}|\overline{\bf P}_L$,   
we opt for parametric estimates  allowing { for} an automatic parameter setting, as this approach is  commonly used in practice.
For instance, when using an estimate based on an  AR process of order $o$, this order can be estimated using many criteria, {\textit{e.g.}} \cite{Akaike_1969, Akaike_1974,Parzen_1975, Hannan_1979, Rissanen_1984}. 

In our studies, we found that the selected order is often different from the true order (as expected, see \textit{e.g.} { p. 211 of \cite{Akaike_1969} and \cite{Boardman_2002}} for similar conclusions) but, as far as detection results are concerned,  these criteria  have very similar  behaviour for sufficiently large $N$.   

In any such method, let us denote respectively  by $\widehat{o}_{AR}$, $\{\widehat{c}_j\}_{j=,1,\cdots,\widehat{o}_{AR}}$ and 
$\widehat{\sigma}^2(\widehat{o}_{AR})$ the selected order,  AR coefficients and corresponding estimated prediction error variance.
{ The  PSD estimate and resulting standardized periodogram are
 \begin{equation}  \vspace{-0.2cm}
	\widehat{S}_{E,AR}(\nu)\!: =\! \frac{\widehat{\sigma}^2(\widehat{o}_{AR})}{\Big|1\!+\!\displaystyle{\sum_{j=1}^{\widehat{o}_{AR}}} \widehat{c}_j \mathrm{e}^{-2\pi i j \nu }\! \Big|^2},
	 \widetilde{P}(\nu_k | \widehat{S}_{E,AR}  )\! :=\! \frac{P(\nu_k)}{\widehat{S}_{E,AR} (\nu_k)}.
	 \label{eq_SAR}
\end{equation}
}

Even if such approaches are relatively straightforward to implement, characterizing the distribution of $ \widetilde{P}(\nu_k |\widehat{S}_{E,AR}  )$
 is  more difficult than in the case of  $ \widetilde{P}(\nu_k | \overline{P}_L  )$, owing essentially to the stochastic nature of $\widehat{o}_{AR}$ in \eqref{eq_SAR}. 
In practice, the `whitening' effect of  $\widehat{S}_{E,AR}$ is efficient because the selection procedures have good fitting properties (they are approximately consistent,  \textit{i.e.},  $\hat{S}_{E,AR} \overset{\approx}{\longrightarrow}S_E$ as $N \longrightarrow+\infty$).
 This leads to consider as reasonable  the assumptions that, in effect,  $ \frac{{ { P (\nu_k)}}}{\widehat{ S}_{E,AR} (\nu_k)}\approx  \frac{P(\nu_k)}{S_E(\nu_k)}$
 and, with \eqref{dist}, that $\frac{P(\nu_k)}{S_E(\nu_k)}$ is approximately a  $\chi_2^2/2$ r.v. for $  k \in \Omega$ and a ${\chi_1^2}$ r.v. for $  k= 0,\frac{N}{2}$.
An approximate false alarm rate can then be evaluated from these assumptions.
For example, for the ${\rm T_C}$ test applied to $\frac{\bf{P}}{\widehat{ \bf{S}}_{E,AR} }$, following the lines of \eqref{pfa_Tor} leads to:

\begin{equation}\vspace{-0.1cm}
	\begin{aligned} 
	\label{PFA_param}
 	 &{\rm{P_{FA}}}({\rm T_C}( {\bf \widetilde{\bf P}}|{\widehat{ \bf S}_{E,AR}},N_C ),\gamma)  \approx  {\rm{P_{FA}}}({\rm T_C}( {\bf \widetilde{\bf P}}|{\bf S}_E, N_C),\gamma) \\
 	 &~~~~~~~~~~~~~~~~~~~~~~~~~~~~~~~~ = I_{u} (N_C, \frac{N}{2}-N_C) .
 	 \end{aligned}
\end{equation}

with $ u := \Phi_{\chi_2^2}(2\gamma)= \mathrm{e}^{-\gamma}$.
 We will evaluate the reliability of this approximation by numerical simulations in Sec.\ref{SecD}.
\section{Numerical study}
 \label{sec7} 

 \subsection{Simulation setting}
\label{SecA}
{ 
We consider under $ {\cal{H}}_0$ two PSD models for the noise $E$.
The first model comes from real RV data of the Sun  
obtained from { the} GOLF spectrophotometer on board SoHO satellite \cite{Garcia_2005}. This instrument has been observing the Sun for 18 years with a sampling rate of 20 s. As several gaps are present in the resulting time series, we selected some ($158$) regularly sampled data blocks of $T \approx 23$ days, of which we averaged and smoothed the periodograms (see Fig. \ref{Fig0},  left panel).
	(Note that the data we used are filtered at low frequencies so that the resulting PSD estimate does probably not accurately reflect the solar { PSD} at low frequencies).\\
	 The second model (Fig. \ref{Fig0},  right panel) corresponds to a zero-mean second-order stationary Gaussian AR(6) process.
	 
	
   	
	
	
     	
	
The coefficients were chosen to yield a correlated process exhibiting higher energy at low frequencies  and local variations, as in some stars,
but the choice of this { PSD} is  not intended to reflect  the reality of a particular star. 
  \begin{figure}[htb!]     \centering
	\includegraphics[width= 9cm,height=4.5cm]{Fig1.eps}
	\caption{  Left: Estimated PSD of the solar noise with part of GOLF data (blue) and periodogram of one of the $158$ data blocks (grey). Parameters: $N = 1110$, $\Delta t = 30$ min. This PSD  is used to generate the noise for Figs. \ref{Fig2} and \ref{Fig5}. Right: Theoretical PSD of the AR(6) noise (blue) and one noise periodogram (grey). This PSD is used for Fig. \ref{Fig3}. Parameters:$N = 1024$, $\Delta t = 1$ min.}
	\label{Fig0}
		\vspace{-0.2cm}
\end{figure}

}

{  Under ${\cal{H}}_1$, several cases of exoplanetary RV signatures will be considered.
}
The tests' performances will be illustrated by ROC curves representing  $\rm P_{DET}$ as a function of $\rm P_{FA}$.
{   For { Monte Carlo (MC)} simulations,  $10^4$  realizations have been used. }
  \subsection{Analytical expressions for tests based on ${\bf \widetilde{\bf P}}\;|\;\overline{\bf P}_L$ }
  \label{SecB}
We first consider the tests ${\rm T_M}( {\bf \widetilde{\bf P}}\,|\,\overline{\bf P}_L)$ and ${\rm T_{C}}( {\bf \widetilde{\bf P}}\;|\;\overline{\bf P}_L)$.  
{ The first panel of Fig.\ref{Fig2} compares to empirical results obtained from MC simulations to the expressions obtained for the  $\rm P_{FA}(\gamma)$ for both tests (see (\ref{pfa}) and (\ref{pfa_Tor})). The second panel regards the corresponding expressions for $\rm P_{DET}(\gamma)$    (\eqref{Eq_Pdet} and  \eqref{lagrosse}) and the last panel the  expression   \eqref{Eq20} for $\rm P_{DET}(P_{FA})$. All the theoretical expressions are shown by color dots and the empirical results from MC simulations are plotted in full lines. }
Different values of $L$ are considered to illustrate the improvement brought by larger training data sets.
The figure shows a fair agreement between theoretical and empirical results, even for the { not} so large value of $N$ considered here { (N = 1110)}. The test performances logically increase with $L$, as the estimation noise decreases with the increasing size of the training data set. 
\begin{figure}[htb!]   
\centerline{\includegraphics[width= 9.5cm,height=9cm]{Fig2.eps}	}
	\caption{ Theoretical \textit{vs} empirical results for ${\rm T_M}( {\bf \widetilde{\bf P}}\,|\,\overline{\bf P}_L)$ and ${\rm T_C}( {\bf \widetilde{\bf P}}\;|\;\overline{\bf P}_L)$ by MC simulations. { Parameters: $\Delta t = 30$ min, $N_s =  3$, $\alpha_q = 0.2$ m.s$^{-1}$ for the three RV signatures of respective periods $11$ h, $2.45$ d and $6.61$ d. The curves in dashed lines show the inconsistency of ignoring noise correlations (Sec. \ref{secC}). }}
		\vspace{-0.4cm}
	\label{Fig2}
\end{figure}
\subsection{Effects of standardization by $\overline{\bf P}_L$}
 \label{secC}
We compare the detection performances of the tests ${\rm T_M}$ and ${\rm T_C}$ { applied to} the standardized periodogram $(\widetilde{\bf P}|\overline{\bf P}_L)$ with their unstandardized versions ${\rm T_M}(\bf P)$ and ${\rm T_{C}}(\bf P)$, as described in Sec. \ref{Sec_tests_cl}.
We evaluate first how accurate would be the false alarm rate obtained by neglecting noise correlation for tests ${\rm T_C}$ and ${\rm T_M}$.
For this we assume the detectors consider the noise is white and have knowledge of $\sigma^2_E$, the exact variance of $E$. When ${\bf{Z}}=2 {\bf{P}}/\sigma^2_E$, it is easy to show  that the  false alarm rates assumed by these two tests are  given by 
\begin{equation}
	\!\!\!\!
	\left\{
	\begin{aligned}
		&{\rm P_{FA}}\left({\rm T_M}(2 {\bf{P}}/\sigma^2_E),\gamma\right)=1- \Phi_{\chi^2_2}^{\frac{N}{2}-1}(\gamma),\\
		&{\rm P_{FA}}\left({\rm T_C}(2 {\bf{P}}/\sigma^2_E, N_C),\gamma\right)= I_{\Phi_{\chi^2_2(\gamma)}} (N_C, \frac{N}{2}-N_C).
	\end{aligned}
	\right.
	\label{vs}
\end{equation}
These expressions are compared to the true false alarm rates  in the top left panel of Fig. \ref{Fig2}.
The cyan { dashed} and blue  { dotted} curves show respectively approximation  \eqref{vs} and empirical $\rm P_{FA}$ for ${\rm T_M}$,
 while the yellow  { dashed} and red  { dotted} curves show respectively approximation  \eqref{vs} and empirical $\rm P_{FA}$ for ${\rm T_C}$.
 Clearly, the correspondence between the thresholds values and the target false alarm rates
is destroyed because of noise correlation in absence of standardization. 
 
\subsection{Effects of standardization by $\widehat{\bf S}_{E,AR}$}
 \label{SecD}
As discussed in Sec.\ref{sec6}, a way to deal with the frequency dependence of the noise is to estimate its PSD by parametric models. 
A difficulty with such methods is the injection of estimation noise in the { detection}  process. A standard approach discussed in Sec.\ref{sec6} is to consider that the estimates are sufficiently accurate for their {intrinsic error} to be negligible. We study the performances of this approach here.
For this we consider the  case of  five sinusoidal signals with frequencies falling into a `valley'  of the    noise PSD  { (Fig.\ref{Fig0}, right panel)}.
We assume the noise PSD follows an AR model (which is indeed the case here) and we  estimate $\widehat{S}_{E,AR}$ as described in Sec.\ref{sec6}.
 The question is the reliability of tests using  ${\bf \widetilde{\bf P}}\;|\;\widehat{\bf S}_{E,AR}$, and in particular how accurate is expression \eqref{PFA_param}
  { with this approach}.
  
 Fig.\ref{Fig3} compares, as a function of the test threshold,  the $\rm P_{FA}$ assumed by approximation \eqref{PFA_param} (blue curve) with the actual false alarm rates obtained for  $1000$ MC simulations.
 \begin{figure}[htb!]  	\centerline{
	 \hspace{-0.2cm} \includegraphics[width= 9.5cm,height=5cm]{Fig3.eps}}
	\caption{ Comparison of the approximated $\rm P_{FA}$  \eqref{PFA_param} (blue curve) with true $\rm P_{FA}$ for $L=1,20$ and $100$.
	The solid lines with dots represent the average actual $\rm P_{FA}$. The shaded regions are the corresponding  $3\sigma$ enveloppes. The  right panel is a zoom in log-scale on the violet square in left panel.  { Parameters: $N_s = 5$, $\alpha_q = 0.07$ {m.s$^{-1}$}, $f_q =[5.0, 5.5, 5.75, 6, 6.5]$ mHz.}}
	\label{Fig3}
\end{figure}

 In each such simulation, an estimate $\widehat{\bf S}_{E,AR}(L)$ was obtained  with Akaike's Final Prediction Error (FPE)   \cite{Akaike_1969}  from $L$ noise time series and used to calibrate the periodogram. 
      For each such estimate, 
  
  the true $\rm P_{FA}$ of test 
  
  ${\rm T_C}(  {\bf \widetilde{\bf P}}\;|\;{ \widehat{\bf S}_{E,AR}},N_C=N_s)$
   was evaluated using 100  MC simulations. The figure plots, respectively for $L=1, 20$ and $100$ the average  $\rm P_{FA}$,  respectively in black, green and red solid lines with dots. We see that \eqref{PFA_param} is accurate, \textit{on average}, only when $L$ is large.
This figure also indicates the variability of the true false alarm rate. For each value of $L$, the figure shows the $3\sigma$ dispersion of the true  $\rm P_{FA}$  w.r.t. its empirical average  (shaded regions in grey for $L=1$, green for $L=20$ and red for $L=100$). Even when $N$ is large, the true
significance levels at which such  tests are conducted can undergo wild (and in practice unknown) variations. The right panel is a zoom on the $3\sigma$ region for $L=100$.
{
For a threshold $\gamma =5.9$ for instance, the $\rm P_{FA}\approx 0.013$ from \eqref{PFA_param}. In the right panel, we see
that  the true false alarm rate for this threshold varies in reality  in the range $[0.001 \; 0.3]$.}
For smaller values of $L$, the excursions of the true false alarm rates are so large that \eqref{PFA_param} is simply useless. Conclusions drawn from tests based on  parametric estimation may thus be very hazardous, even
{ when the parametric model is true and }
 when large  data sets are available for PSD estimation. 
\subsection{${\rm  T_C}({\bf \widetilde{\bf P}}|\overline{\bf P}_L,N_C)$ and ${\rm T_M}({\bf \widetilde{\bf P}}|\overline{\bf P}_L)$ vs adaptative approaches} 
 \label{SecF}
  We do not attempt to apply  ${\rm HC^\star}$ and ${\rm BJ}$ to ${\bf{P}}$ in the correlated case as this leads to the same inconsistencies as those illustrated in Sec. \ref{secC} and Fig. \ref{Fig2},   {top left panel, dotted curves.}
 We compare here the  adaptative approaches ${\rm HC^\star}$ and ${\rm BJ}$  standardized by $\overline{\bf P}_L$  as in Sec.\ref{sec_HC} with the corresponding  tests ${\rm T_M}$ and ${\rm T_C}$  for $L=1$ and $L=50$. (Note that when $N_C=1$, ${\rm T_C}$ reduces to ${\rm T_M}$, \textit{cf} \eqref{test_ch}). 
 
 { For this comparison, we consider two different cases of Keplerian signals under $\mathcal{H}_1$, which are typical of `super-Earth' planets (Fig. \ref{Fig5}):  $N_p = 1$ planet with null eccentricity in Case 1 (left column), and $N_p = 5$ eccentric planets in Case 2 (right column). 
 \begin{figure}[htb!]  \centering
	\includegraphics[width=9cm,height=7cm]{Fig5_2.eps}
	\caption{ { Top panels: Empirical ROC curves comparing classical and adaptative tests for different signals.
	 Case 1:  $N_p = 1$, $M_p = 3.5 M_\oplus$ (for $L=1$), $T_p = 5.7813$ d (signal frequency $1/T_p$ on-grid), $e_p=0$, $\omega_p=0$, $T_0=0$, $\gamma_0= 0$. For $L = 50$, $M_p = 0.8 M_\oplus$. 
	Case 2: $N_p = 5$, $M_p =  [0.15,0.15,0.25,0.25,0.25]  M_\oplus$ (for $L=1$), $T_p = [11.21~ {\rm h}, 1.33 ~ {\rm d}, 2.45~ {\rm d}, 6.91 ~ {\rm d}, 9.25~ {\rm d}] $ (signal frequencies off-grid), $e_p=0.9$, $\omega_p=\pi$, $T_0=0$, $\gamma_0= 0$. For $L = 50$,  $M_p(T_p = 11.21~ {\rm h}) = 0.07 M_\oplus$. 
	 In  the adaptative tests,   $\alpha_0=1/2$.  Bottom panels: periodograms (logscale) of the signals under ${\mathcal{H}}_1$.}}
	\label{Fig5}
\end{figure} 

 
 
 The lower panels illustrate the periodogram of the (noiseless) Keplerian signatures for the two cases. We see the apparition of significant harmonic{s} in the case of off-grid signal frequencies and highly eccentric  orbits. For each case, the planet masses have been adapted depending on the considered value of $L$  for a better display of the ROC curves.
 
 From $L=1$ (dashed lines) to $L=50$ (solid lines),  the performances of all  tests increase in both cases.  When the signal is extremely sparse in the Fourier domain (Case 1), ${\rm T_M}$ is more powerful than the considered adaptive approaches. This situation  changes when the spectrum is less sparse (Case 2,  compare the bottom panels), which is expected. 
 
 Results of \cite{Donoho_2004} show that for a proportion of deviations in the range $ [(\frac{N}{2}-1)^{\frac{1}{4}}\; (\frac{N}{2}-1)^{\frac{1}{2}}]$ an adaptive procedure such as HC may have better asymptotic power  than $\rm T_M$. For the case $N=1110$ considered here,  this corresponds to the range $[5\;23]$ (considering $\Omega$). The situation should be opposite  for  very sparse signals (in the range  $ [1\; 4]$ here). It turns out that the superiority of adaptive procedures is confirmed in Fig. \ref{Fig5}, 
 where the spectrum is $1$-sparse in Case 1 and about $10$-sparse in Case 2.
Note, however, that while the theory of  \cite{Donoho_2004} may be used as a guideline for guessing the  sparsity range in which each test should work better,  we should not expect a too tight agreement with  this theory. Indeed, in the framework of \cite{Donoho_2004},
all deviations under the alternative have the same amplitude, while RV signals lead to  different amplitudes in general. Moreover, these theoretical results are asymptotical (while  $N$ is not so large here).\\
 An interesting point is the comparison of $\rm HC^\star$ and $\rm BJ$ with ${\rm T_C}({\bf \widetilde{\bf P}}|\overline{\bf P}_L,N_C)$, for which $N_C$ is a proxy for the number of significant deviations in the Fourier spectrum (here we considered $N_C=1$ in Case 1 and $N_C=10$ in Case 2). ${\rm T_C}$ represents a  kind of Oracle, which knows  in which region of the $P$-values to look at in order to `make the case' against $\mathcal{H}_0$. The right panel shows that adaptive procedures  have better power  than ${\rm T_C}$, yet without prior knowledge.}
 
{ \subsection{A detectability study}
\label{SecG}

A direct application of the previous results is detectability studies, which can be used for the design of observational strategies for instance. 
To illustrate this{,} we consider under  $\mathcal{H}_1$  a planet with the  parameters of $\alpha$ Centauri B's exoplanet candidate as estimated in \cite{Dumusque_2012} (see the legend of Fig.\ref{Fig6}).  As the eccentricity is supposed null, the signal can be considered very sparse in the Fourier domain. We consider the $\rm T_M$ test and a time sampling of $\Delta t = 4$ hours. It was allowed to slightly vary from one value of $N$ to another in order to guarantee
that the planet's period yields a frequency exactly on the Fourier grid, in which case the spectrum is $1$-sparse on $\Omega$ and  $\rm T_M$ is  the test that  yields the best performances.

As for the noise PSD, we used a model based on HD simulations of a star with similar spectral type as that of $\alpha$CenB. 
{ There is some mismatch between the simulated RV time series of the spectral type} (G2) and { the} true spectral type (K1) of $\alpha$CenB. Because  spectral type affects the noise properties \cite{Meunier_2016},  these results should not be considered to reflect perfectly the case of the candidate planet orbiting $\alpha$CenB. 

In Fig.\ref{Fig6}, we illustrate the feasible performance compromises $(\rm P_{DET}, \rm P_{FA})$ as a function of $N$, for three target $\rm P_{FA}$ ($0.5,\; 0.1$ and $0.01$, indicated  respectively by the dotted, solid and dashed lines) and for different sizes of available training data sets  $L$ ($ \infty, 100, 20, 5$, shown respectively in black, green, blue and red). These curves were built using the expressions   (\ref{Eq_gam}) and  (\ref{Eq20}) for the ${\rm T_M}$ test and we checked that they are accurate in separate MC simulations (not shown). In the case $L \to+\infty$, we use the fact that $ F(2,2L)  \underset{L \to+\infty}{\longrightarrow} \chi_2^2$ to calculate the theoretical $\rm P_{DET}(P_{FA})$.

The study presented in Fig.\ref{Fig6}  allows to quantify interesting facts. 
First, of course,  $\rm P_{DET}$ is larger if the allowed $\textrm{P}_{\textrm{FA}}$ is larger. Second, for a fixed   $\textrm{P}_{\textrm{FA}}$,   $\rm P_{DET}$ is larger for a larger value of $L$. Going to specific cases, we see that if a planet similar to the considered candidate was orbiting a star of the considered spectral type (G2), of which 100 training time series are available, it would require  250 days ($1500\times 4$h) of observations   with 1 point every 4 hours to guarantee a probability of detection of $0.9$ while ensuring a false alarm rate of $0.01$. This situation is indicated by the black square. With only $L=5$ training time series, the probability of detection would fall to about $0.1$, all other parameters equal (red square).
}
 \begin{figure}[htb!] \centering
	\includegraphics[width=9cm,height=5.5cm]{Fig6_3.eps}
	\caption{{Example of a detectability study for a  planet (relevant Keplerian parameters:  $K = 0.54$ m.s$^{-1}$, $T_p  = 3.23$ d and $e=0$; $M_p = 1.241 M_\oplus$, {orbit inclination of $90 \degree$}, semi-major axis $a     =   0.0425$ Astronomical Unit) orbiting a G2 type star. The plot shows the achievable $\rm P_{DET}$ for different configurations of $\textrm{P}_{\textrm{FA}}$ budgets and numbers of available training light curves.  
	}}
	\label{Fig6}
\end{figure}

\section{Summary and perspectives}
\label{conc}
This paper has  investigated the possibility of using training data sets to standardize periodograms in order to improve the control of the resulting false alarm rate. The paper first { provided}  an extended (though unavoidably selective)  overview of classical and recent techniques in sinusoid detection, with emphasis  on the problem of designing CFAR detectors for the composite hypotheses of multiple sinusoids and colored noise.
We proposed an asymptotic analysis of the periodograms statistics after standardization for a model involving an unknown number of sinusoids with unknown parameters in partially unknown colored noise. This analysis allowed
to  characterize the performances of some standardized tests in terms of false alarm and detection rates. We showed that when standardization is performed with a simple averaged periodogram as a noise PSD estimate  these tests are CFAR for all sizes of training data sets. In contrast, we pointed out that standardization based on parametric estimates of the noise PSD may present  actual false alarm rates that may be very far from the assumed ones, even for large data sets.\\
The tests we considered include classical approaches and also more recent adaptive tests designed for the rare and weak setting. For the latter tests, the standardization (by $\overline{\bf{P}}_L$) offers the same benefits as if the statistics of the noise were known {\textit{a priori}}, with the CFAR property added. 
{We also showed that such tests can present better power comparing to procedures for which the number of sinusoids would be known in advance.}

{ In practical  situations, some of  the assumptions (i-iv) in Sec. \ref{secIA} may not be met. This can be the case  in exoplanet detection  using RV time series, owing for instance to astrophysical effects linked to magnetic activity (like spots, affecting (i)) or instrumental defects / observational constraints
 (affecting (ii) and (iii)).  Comparing theory to pratice, the present study is useful to   make feasibility studies (as in Sec. \ref{SecG}), which then describe best achievable performances (\textit{i.e.}, around ``quiet'' stars and in absence of other unmodelled perturbations) for a regular sampling. }
 
   
 
 An important extension to the considered framework regards the case  when, for periodicity analysis, the time series is not correlated to orthogonal exponentials. This case encompasses situations where (a) the sampling is  irregular, (b)   $P$ is not evaluated on the Fourier grid (as in oversampled periodograms) (c) $P$ is modified in the form of  ``generalized periodograms'', which correlate the time series with highly redundant dictionaries of specific features 
{\cite{Bolviken_1983a,Baluev_2008,Suveges_2012, Scargle_1982,Bretthorst_2003,Thong_2004,Zechmeister_2009,Baluev_2015,Gregory_2016}}. {   In such cases,   the considered ordinates exhibit strong dependencies.
With the additional complication of partially unknown colored noise, analytical evaluation of the false alarm rate  for the considered tests seems out of reach. 
We conducted however preliminary studies suggesting that it might be possible to obtain accurate estimates of the false alarm rate when the noise is colored,
the sampling irregular, the considered frequencies not restricted to Fourier grid and $N$ small, by
combining  the standardization proposed in the present paper  with bootstrap procedures \cite{Zoubir_1993} and maximum likelihood estimation of Generalized Extreme Values distributions' parameters \cite{Suveges_2012,Suveges_2015}.  This numerical approach, still under study, would allow to address important questions, like that of the impact of the sampling distribution on the detection performances.}

	

\appendices

 \section{Derivation of expressions \eqref{dist2} and \eqref{lambda}}
 \label{app1}
  
 We prove  here that for model \eqref{hyp} the periodogram is asymptotically distributed as in \eqref{dist2} with non centrality parameters as in \eqref{lambda}.
 The proof is adapted from Theorem 6.2 of \cite{Li_2014}, which considers the complex case.
  We first prove \eqref{dist2} and then turn to  \eqref{lambda}. The time series of model \eqref{hyp} can also be written as
$$
X(j)  =  \sum_{q = 1}^{N_s} \alpha_q \sin(2\pi f_q j+\varphi_q)+ E(j)= R(j)+E(j),
$$
with $R(j):=   \sum_{q=1}^{N_s}  \alpha_q \sin(2\pi  f_q j + \varphi_q)$ a deterministic part, which using Euler formulae 
can  be written as
$$
      \begin{aligned}
	R(j)  	     &=  \sum_{q=1}^{N_s} \frac{ \alpha_q}{2}  \mathrm{e}^{{\rm{i}} (\varphi_q-\frac{\pi}{2})} \mathrm{e}^{2\pi {\rm{i}} f_q j } -  \frac{ \alpha_q}{2}  \mathrm{e}^{-{\rm{i}} (\varphi_q+\frac{\pi}{2})} \mathrm{e}^{-2\pi {\rm{i}} f_q j}.
     \end{aligned}
     \vspace{-0.3cm}
 $$
$$
\vspace{-0.2cm}
\!\!\!\!\!\! \!\! \text{By introducing:}~~~~~~
	\left\{
      \begin{aligned}
      		& {\bf{f}}(\nu):=[\mathrm{e}^{{\rm{i}}2\pi \nu},\hdots, \mathrm{e}^{{\rm{i}}2\pi N\nu}]^\top,\\
		&{\bf{f}}^{+}(f_q)
		:=\mathrm{e}^{{\rm{i}} (\varphi_q-\frac{\pi}{2})}  {\bf{f}}(f_q), \\
		&{\bf{f}}^{-}(f_q):=
		\mathrm{e}^{-{\rm{i}} (\varphi_q+\frac{\pi}{2})}  {\bf{f}}(f_q),
     \end{aligned}
     \right.
$$
the time series  writes in vector form :
\begin{equation}
      \begin{aligned}
	{\bf{X}} &=   \displaystyle{\sum_{q=1}^{N_s}} \frac{\alpha_q}{2} \Big( {{{\bf{f}^{+}}(f_q)} -  {{\bf{f}^{-}}(f_q)}}  \Big)+ {\bf{E}}={\bf{R}}+ {\bf{E}} \\
  \end{aligned}
  \label{RE}
\end{equation}
and its discrete Fourier transform (DFT) $y_k$ at frequency $\nu_k$ { is}
$$
	y_k=\frac{1}{N}  {\bf{f}}^H(\nu_k){\bf{X}} = \frac{1}{N}  {\bf{f}}^H(\nu_k){\bf{R}} + \frac{1}{N}  {\bf{f}}^H(\nu_k){\bf{E}}.
$$
The DFT is composed of a deterministic part, ${\mu_k}$, and a stochastic part, ${\epsilon}_k$, defined as
\begin{equation}
	{\mu_k}:= \frac{1}{N} {\bf{f}}^H(\nu_k){\bf{R}}\quad{\textrm{and}}\quad {\epsilon}_k:= \frac{1}{N} {\bf{f}}^H(\nu_k)\bf{E}.
	\label{mu}
\end{equation}
 Because $E$ is a zero mean Gaussian process, the random variable $y_k=\mu_k+\epsilon_k$ is Gaussian with mean $\mu_k$ and variance $\sigma^2_k$. This is a complex variable for all Fourier frequencies except for $\nu_0$ and $\nu_{\frac{N}{2}}$ because $ {\bf{f}}(\nu_0)$ and $ {\bf{f}}(\nu_{\frac{N}{2}})$ are real.

The distribution of the periodogram requires to investigate the variance of $\epsilon_k$ which, with \eqref{mu}, writes:
$$
      \begin{aligned}
      &N \textrm{var}{\;\epsilon_k}=\frac{1}{N}\mathbb{E}\; \left( {\bf{f}}^H(\nu_k){\bf{E}}{\bf{E}}^\top  {\bf{f}}(\nu_k)\right) \\
	&=\frac{1}{N}\displaystyle{\sum_{t,s=1}^N}r_E(t-s)\mathrm{e}^{-{\rm{i}}2\pi \nu_k(t-s)} \\
	&= \frac{1}{N}\displaystyle{\sum_{|u|<N}}(N-|u|)r_E(u)\mathrm{e}^{-{\rm{i}}2\pi \nu_ku}\\
	&= S_{E}(\nu_k)-\!\!\!\displaystyle{\sum_{|u|<N}}\frac{|u|}{N}r_E(u)\mathrm{e}^{-{\rm{i}}2\pi \nu_ku}
 -\!\!\!\displaystyle{\sum_{|u|\geq N}}r_E(u)\mathrm{e}^{-{\rm{i}}2\pi \nu_ku}\\
	 & =S_{E}(\nu_k) + {\mathcal{O}}(r_N),
   \end{aligned}
$$
since, using { the absolutely summable autocorrelation function $\sum_u |r_E(u)| < \infty$} and the dominated convergence theorem we have
\vspace{-0.4cm}
\begin{equation}
	r_N:=\sum_u \min(1,\frac{|u|}{N}) |r_E(u)| \to 0 \textrm{\;\;as\;\;} N \to \infty.
	\vspace{-0.2cm}
	\label{lerho} 
\end{equation}
Hence, for all Fourier frequencies,
 \begin{equation}
	 \sigma^2_k:= \textrm{var}\;\epsilon_k= N^{-1} S_{E}(\nu_k) + {\mathcal{O}}(N^{-1} r_N).
	 \label{lavar}
 \end{equation}
 
 By Lemma 12.2.1(b) of \cite{Li_2014}, we obtain 
$$
  |y_k|^2/\sigma^2_k \sim   \left\{
      \begin{aligned}
		&{  \frac{1}{2}} \chi^2_{2,2\frac{|\mu_k|^2}{\sigma^2_k}},  ~~~~ \forall k ~ \in ~ \Omega,\\
		&  \chi^2_{1,\frac{|\mu_k|^2}{\sigma^2_k}}, ~ \text{ for } ~ k = 0, \frac{N}{2}.
	 \end{aligned}
	  \right.
$$

Hence,  for the periodogram this implies
\begin{equation} 
\begin{aligned} 
P(\nu_k | H_1)& = N|y_k|^2 = N\sigma^2_k(|y_k|^2/\sigma^2_k)\\
& \sim
  \left\{         
      \begin{aligned}
	 &\frac{1}{2}\rho^{-1}_k S_E(\nu_k) \chi_{2, 2\rho_k \gamma_k}^2 ,~ \forall k ~ \in ~ \Omega,  \\
	 &\rho^{-1}_k S_E(\nu_k)  \chi^2_{1,\rho_k \gamma_k }, ~ \text{ for } ~ k = 0, \frac{N}{2},
      \end{aligned}
    \right.
      \end{aligned}
      \vspace{-0.2cm}
       \label{leP}
\end{equation}
where 
\vspace{-0.2cm}
\begin{equation}
\rho_k:=S_E(\nu_k)/(N\sigma^2_k) \quad{\textrm{and}}\quad\gamma_k:=N|\mu_k|^2/S_E(\nu_k).
\label{gamma}
\end{equation} 
With \eqref{lavar}, we see that
\begin{equation}
	\rho_k=\frac{S_E(\nu_k)}{S_E(\nu_k)+{\mathcal{O}(r_N)}}=1+{\mathcal{O}(r_N)}
	 \label{lerho2}
\end{equation}
 for all Fourier frequencies.
 Owing to \eqref{lerho}, an approximated distribution of  \eqref{leP} can be obtained by neglecting
  the $\mathcal{O}(r_N)$ in { \eqref{lerho2}}. The distribution \eqref{dist2} follows by noting
\begin{equation}
	\hspace{-3mm}
	\lambda_k:=2\gamma_k\;\; \text{for}\;\; k ~ \in ~ \Omega\quad \text{and}\quad \lambda_k:=\gamma_k \;\; \text{for}  \;\; k = 0, \frac{N}{2}.
\label{fact}
\end{equation} 
 We now turn to the computation of the non centrality parameters. We have from \eqref{RE}, \eqref{mu} and \eqref{gamma}
\begin{equation} \small 
	  \begin{aligned}
	\gamma_k &= \frac{N}{S_E(\nu_k)} |  \frac{1}{N} {\bf{f}}^H(\nu_k)\bf{R}|^2\\
	 &= \frac{1}{NS_E(\nu_k)}  \Big| \sum_{j =1}^{N}   \sum_{q=1}^{N_s}  \frac{\alpha_q}{2} \Big( \mathrm{e}^{{\rm{i}} (\varphi_q-\frac{\pi}{2})} \mathrm{e}^{2\pi {\rm{i}} (f_q-\nu_k) j } \! \\
	 &~~~~~~~~~~~~~~~~~~~~~~~~~~\hdots  - \!\mathrm{e}^{-{\rm{i}} (\varphi_q+\frac{\pi}{2})} \mathrm{e}^{-2\pi {\rm{i}} (f_q+\nu_k) j} \Big) \Big|^2\!. \\
	 \end{aligned}
 	\label{gamma3}
\end{equation}
{ Introducing the Dirichlet Kernel (\textit{cf} Lemma 12.1.3 of \cite{Li_2014}):\\ }
$$
	D_N(\nu) := \frac{1}{N}  \sum_{j=1}^{N}  \mathrm{e}^{{\rm{i}} 2\pi \nu j} = \frac{\sin(N\pi \nu)}{N\sin(\pi \nu)} \mathrm{e}^{{\rm{i}} (N+1)\pi \nu},
     \label{eq_Dn}
$$
	and the corresponding  Fej\'er kernel (or spectral window)
\begin{equation}
	K_N(\nu) := |D_N(\nu)|^2 =  \Bigg( \frac{\sin(N\pi \nu)}{N\sin(\pi \nu)} \Bigg)^2,
     \label{eq_Kn}
\end{equation}
expression \eqref{gamma3} becomes
{
$$ 
 \small
  \begin{aligned}
 \gamma_k 
	
				&=  \frac{N}{4 S_E(\nu_k)} \Big| \sum_{q=1}^{N_s}  \alpha_q \Big( D_N(f_q -\nu_k)  \mathrm{e}^{{\rm{i}} (\varphi_q-\frac{\pi}{2})} \\
				&~~~~~~~~~~~~~~~~~~~~~~~~~~~~~~~~~~~~\hdots  - D_N(f_q +\nu_k) \mathrm{e}^{-{\rm{i}} (\varphi_q+\frac{\pi}{2})}\! \Big)\! \Big|^2. \\
 \end{aligned}
 $$
 }
 This equation can also be written as :
\begin{equation} 
	 \gamma_k =  \frac{N}{4 S_E(\nu_k)} \Big| \sum_{q=1}^{N_s}  \alpha_q z_q(\nu_k) \Big|^2,
	 \label{gamma2}
 \end{equation} 
 with
  
 \begin{equation} 
  \begin{aligned}
	z_q(\nu_k) &:= D_N(f_q -\nu_k)  \mathrm{e}^{{\rm{i}} (\varphi_q-\frac{\pi}{2})} \!-\! D_N(f_q +\nu_k) \mathrm{e}^{-{\rm{i}}(\varphi_q+\frac{\pi}{2})} \\
	        &=  x_{+}  \mathrm{e}^{{\rm{i}} \theta_+ } -  x_{-}  \mathrm{e}^{ {\rm{i}} \theta_{-} }\\
		&=  ( x_{+}  \cos\theta_+ -   x_{-}  \cos\theta_{-} )+ {\rm{i}}  ( x_{+} \sin\theta_+ -   x_{-}\sin\theta_{-} ),
 \end{aligned}
 \label{zp}
\end{equation}
where
$$
\left\{
  \begin{aligned}
 	x_{+}&=  x_{+}\left({\nu_k},q\right)  :=  \frac{\sin(N\pi (f_q - \nu_k))}{N\sin(\pi (f_q - \nu_k))},  \\
 	x_{-}&=  x_{-}\left({\nu_k},q\right)    :=  \frac{\sin(N\pi (f_q + \nu_k))}{N\sin(\pi (f_q + \nu_k))} , \\
	\theta_+ &= \theta_+\left({\nu_k},q\right) :=  + [ (N+1)\pi (f_q - \nu_k)+ (\varphi_q-\frac{\pi}{2})], \\
	\theta_- &= \theta_-\left({\nu_k},q\right)  :=  -[ (N+1)\pi (f_q + \nu_k)+ (\varphi_q+\frac{\pi}{2})]. \\	
 \end{aligned}
 \right.
$$
The modulus $\kappa_q$ of $z_q$ may be written as  
  \begin{equation} 
 	\begin{aligned}
		 & \kappa_q  :=  | z_q | = \left( x_{+}^2 +  x_{-}^2 - 2 x_{+}  x_{-}  \cos{(\theta_+-\theta_{-}) }\right)^{\frac{1}{2}},
 	\end{aligned}
	\vspace{-0.4cm}
 	 \label{eq_keppa}	
 \end{equation} 

$$
\text{with}~~~~~~~~~
\left\{
 \begin{aligned}
 	x_{+}^2 &= K_N(f_q - \nu_k), \\
	x_{-}^2 &= K_N(f_q+ \nu_k) , \\
	\theta_+-\theta_{-} &=   2 \pi (N+1) f_q + 2 \varphi_q,
 \end{aligned}
 \right.
$$
\begin{equation} 
\text{and}~~~~~~~~~~~~~~~~~ \theta_q   := \angle\; z_q ,\; \theta_q\; \in\;]-\pi, \pi],
  \label{eq_theta}
  \end{equation} 
  the phase of $z_p$ obtained from the real and imaginary parts of \eqref{zp} \cite{Kasana_2005}.
With these notations, it  is easy to show that 
 $$
 \Big| \sum_{q=1}^{N_s} z_q \Big|^2 = \sum_{q=1}^{N_s} \Big[ \kappa_q^2 + 2  \kappa_q \sum_{\ell = q+1}^{N_s} \kappa_\ell \cos(\theta_q-\theta_\ell)  \Big] ~~ \text{for $N_s>1$}.
 $$
Consequently, the  expression of the $\{\gamma_k\}$ in \eqref{gamma2} becomes 
\begin{equation} 
	 \begin{aligned}
		&\gamma_k \!=\!  \frac{N}{4 S_E(\nu_k)} \! \! \sum_{q=1}^{N_s} \!\! \Big[  \alpha_q^2 \kappa_q ^2 \!\!+\! 2\alpha_q \kappa_q\!\! \sum_{\ell = q+1}^{N_s}\!\! \! \alpha_\ell \kappa_\ell  \!\cos(\theta_q-\theta_\ell )\Big]	\\
	\end{aligned}
 \label{eq_50}	
\end{equation} 
 and  the non centrality parameters $\{\lambda_k\}$ of \eqref{lambda} follow from \eqref{fact},with $\kappa_q$ and $\theta_q$  given by \eqref{eq_keppa} and \eqref{eq_theta}.  ${\hfill\ensuremath{\square}}$

Note that if all signal frequencies $\{f_p\}$ fall on the Fourier frequency grid, the crossed term in \eqref{eq_50} vanish owing to the orthogonality
of the Fej\'er kernels centered at different signal frequencies. In this case, expression \eqref{eq_50} precisely reduces to expression given in Remark 6.6 of \cite{Li_2014}.

We finally wish to mention that the expression of the non centrality parameters is erroneously reported in exp. (5) of  \cite{Sulis_2016a} (sign error and crossed terms missing).

 \section{Derivation of expression  (\ref{lagrosse}) }
   \label{app6} 
   Let $K$ denote the number of  ordinates of  ${\bf \widetilde{\bf P}}\;|\;\overline{\bf P}_L$  larger than $\gamma$  under $\mathcal{H}_1$,
 and $p_i := \textrm{Pr}\; ( K=i\;|\;{\cal{H}}_1)$.  From the definition \eqref{test_ch}, we have:
  \begin{equation}  
\begin{aligned} 
	&\!\!\rm{P_{DET}}({T}_{C}( {\bf \widetilde{\bf P}}|\overline{\bf P}_L),\gamma,N_C) \!:=\! \textrm{Pr}\; ( {T_C  ({\bf \widetilde{\bf P}}|\overline{\bf P}_L,N_C)}\!\! > \!\gamma | {\cal{H}}_1\!)\! \\
	&=  \textrm{Pr}\; ( K\geq N_C\;|\;{\cal{H}}_1) \\
	&= 1- { \sum_{i=0}^{N_C-1} } \textrm{Pr}\; ( K=i\;|\;{\cal{H}}_1) =  1- \displaystyle{ \sum_{i = 0}^{N_C-1}} p_i.
\end{aligned}
\label{pfab}
 \end{equation}
 
\noindent Owing to \eqref{dist_H1} each  ordinate $({\bf \widetilde{\bf P}}|\overline{\bf P}_L)_i:=\frac{\widetilde{P}(\nu_i)}{\overline{ P}_L(\nu_i)}$ has probability $1-\Phi_{F_{\lambda_i}}(\gamma)$ to be larger than $\gamma$. These variates can be considered approximately independent but not i.i.d. Hence, the variable
$K$ is not binomially distributed (as it is under ${\cal{H}}_0$) and the probabilities $\{p_i\}$ require further investigation.  
We proceed by induction.  In the following, all probabilities are under $\mathcal{H}_1$. The first probability  $p_0$ can simply be approximated as 
$$
 \begin{aligned} 
		&p_0=  \textrm{Pr} \left\{\bigcap_{ k=1}^{\frac{N}{2}-1}
	   ({\bf \widetilde{\bf P}}\;|\;\overline{\bf P}_L)_k \leq \gamma   \right\}  \approx \displaystyle{ \prod_{k =1}^{ \frac{N}{2}-1}} \Phi_{F_{\lambda_{k}}}. \\
\end{aligned} 
$$
The probability $p_1=\text{Pr}(K=1)$ is similarly
$$
 \begin{aligned} 
 	p_1 &= \textrm{Pr} {\displaystyle{\bigcup_{k=1}^{\frac{N}{2}-1} }}\left\{  ({\bf \widetilde{\bf P}}\;|\;\overline{\bf P}_L)_k> \gamma      \bigcap_{ j\neq k}
	   ({\bf \widetilde{\bf P}}\;|\;\overline{\bf P}_L)_j \leq \gamma   \right\}   \\
	&\approx \displaystyle{ \sum_{k = 1}^{\frac{N}{2}-1} \Big[  ( 1 -  \Phi_{F_{\lambda_{k}}} ) \displaystyle{ \prod_{\substack{j =1,\\ j \neq k}}^{ \frac{N}{2}-1}} \Phi_{F_{\lambda_{j}}}  \Big] }. \\
\end{aligned} 
$$	
To generalize further, denote by ${\Omega}^{(i)}$ one particular combination of $i$ indices taken in $\Omega$ and $\overline{\Omega}^{(i)}:=\Omega \backslash \Omega^{(i)}$ the set of remaining indices. Let $\{\Omega^{(i)}_1,\hdots , \Omega^{(i)}_i\}$ (resp. $\{\overline{\Omega}^{(i)}_1,\hdots, \overline{\Omega}^{(i)}_{\frac{N}{2}-1-i}\}$) denote the indices in  two such combinations, and let   $\Omega^i$  be the set of all the  $\{{\Omega}^{(i)}\}$. 
With these notations we obtain  for $i>1$ :
\begin{equation}
 \begin{aligned} 
 	p_i &= \textrm{Pr} \!\!\! {\displaystyle{\bigcup_{ {\Omega}^{(i)} \in {\Omega}^{i} } }}
	\left\{   \bigcap_{ k=1}^i	   ({\bf \widetilde{\bf P}}\;|\;\overline{\bf P}_L)_{\Omega^{(i)}_k } > \gamma
	   \bigcap_{ k'=1}^{\frac{N}{2}-1-i}	   ({\bf \widetilde{\bf P}}\;|\;\overline{\bf P}_L)_{\overline{\Omega}^{(i)}_{k'}}  \leq \gamma \right\}   \\	
	&\approx  \displaystyle{  \sum_{\Omega^{(i)} } \prod_{k=1}^{ i  } } \Big(\!1\! - \Phi_{F_{\lambda_{\Omega^{(i)}_k}}\!\!^{\!\!\!\!\!\!\!\!\!(\gamma, 2,2L)}} \!\!\Big) \displaystyle{  \prod_{k'=1}^{ \frac{N}{2}-1-i }}\Phi_{F_{\lambda_{ \overline{\Omega}^{(i)}\!_{\!\!\!\!\!\!k'} }}{(\gamma, 2,2L)} }.
\end{aligned} 
\label{pfac}
\end{equation}	
 Expression  (\ref{lagrosse}) follows by combining \eqref{pfab} and \eqref{pfac}.  ${\hfill\ensuremath{\square}}$

	
\section*{Acknowledgement}
The GOLF instrument onboard SoHO is a cooperative effort of scientists, engineers, and technicians, to whom we are indebted. SoHO is a project of international collaboration between ESA and NASA.
\bibliographystyle{IEEEbib}
\bibliography{maBiblio,Biblio_IEEE} 

\end{document}

