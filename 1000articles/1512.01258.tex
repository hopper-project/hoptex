
\documentclass[12pt]{amsart}
\usepackage{graphicx}
\usepackage{hyperref}
\usepackage{setspace}
\usepackage{enumerate}
\usepackage{amsmath,amsfonts,amssymb,amscd}

\newtheorem{thm}{Theorem}[section]
\newtheorem{cor}[thm]{Corollary}
\newtheorem{lem}[thm]{Lemma}
\newtheorem{prop}[thm]{Proposition}
\newtheorem{example}[thm]{Example}
\theoremstyle{definition}
\newtheorem{defn}[thm]{Definition}
\newtheorem{pro}[thm]{Properties}
\theoremstyle{remark}
\newtheorem{rem}[thm]{Remark}
\numberwithin{equation}{section}

\setlength{\oddsidemargin}{0cm} \setlength{\evensidemargin}{0cm}
\setlength{\marginparwidth}{0in}
\setlength{\marginparsep}{0in}
\setlength{\marginparpush}{0in}
\setlength{\topmargin}{0in}
\setlength{\headheight}{0pt}
\setlength{\headsep}{15pt}    
\setlength{\footskip}{.3in}
\setlength{\textheight}{9.2in}
\setlength{\textwidth}{16.5cm}
\setlength{\parskip}{4pt}

\begin{document}

\title[]{Zeroes of polynomials in many variables \\ with prime inputs}

\author{Stanley Yao Xiao}
\address{Department of Pure Mathematics \\
University of Waterloo \\
Waterloo, ON\\  N2L 3G1 \\
Canada}
\email{y28xiao@uwaterloo.ca}

\author{Shuntaro Yamagishi}
\address{Department of Mathematics \& Statistics \\
Queen's University \\
Kingston, ON\\  K7L 3N6 \\
Canada}
\email{sy46@queensu.ca}
\indent

\date{Revised on \today}

\begin{abstract}
In this paper we apply the Hardy-Littlewood circle method to show that a polynomial equation admits infinitely many prime-tuple solutions assuming only that the equation satisfies suitable local conditions and the polynomial is sufficiently non-degenerate algebraically.
Our notion of algebraic non-degeneracy is related to the $h$-invariant introduced by W. M. Schmidt and does not require geometric data. This extends the work of B. Cook and \'{A}. Magyar \cite{CM} for hypersurfaces of degree $2$ and $3$.
\end{abstract}

\subjclass[2010]
{11D72, 11P32}

\keywords{Hardy-Littlewood circle method, diophantine equations, prime numbers}

\maketitle

\section{Introduction}

Let $b(x_1, \cdots, x_n)$ be a polynomial of degree $d$ and integer coefficients. We are interested in finding prime solutions,
which are solutions with each coordinate a prime number,
to the equation
\begin{equation}
\label{main system}
b(\mathbf{x}) = 0.
\end{equation}
Solving diophantine equations in primes are among the most difficult problems in number theory. For example, the well known work of B. Green and
T. Tao \cite{GT}
on arithmetic progressions can be phrased as the statement that the system of linear equations
$$
x_{i+2} - {x_{i+1}} = x_{i+1} - x_{i} \ (1 \leq i \leq n)
$$
has a prime solution for any $n \in \mathbb{N}$. More recently,
Y. Zhang \cite{Z} and J. Maynard \cite{M} made significant progress on the problem of gaps between primes
by building on the work of D.A. Goldston, J.Pintz, and C.Y. Y{\i}ld{\i}r{\i}m \cite{GPY}.
In particular, it was shown in \cite{M} that one of the equations
$$
x_1 - x_2 = 2j \ (1 \leq j \leq 300)
$$
has infinitely many prime solutions. The Goldbach problem is another important example of this area.
It was proved by I. M. Vinogradov \cite{V} that the equation $x_1 + x_2 + x_3 = N$ has a prime solution for
all sufficiently large odd $N \in \mathbb{N}$.
The Ternary Goldbach Problem, which is the assertion that the former equation is solvable in primes for all odd $N \in \mathbb{N}$ greater than or equal to 7, was solved by H.A. Helfgott in \cite{H1, H2}.
The examples given thus far are all systems of linear equations.
The scenario for systems involving higher degree homogeneous polynomials is equally complex.
Indeed, even the problem of whether a system of non-linear polynomial equations has a solution over the rationals is
`one of considerable complexity' \cite{BDLW}.
\\ \\
The Hardy-Littlewood circle method, pioneered by Hardy and Littlewood to give an asymptotic formula for solutions to Waring's problem, has been shown to be quite effective at producing asymptotic formulas for the number of
integer points of bounded height on varieties
when the number of variables is sufficiently large.
The results of this type on the distribution of integral points on varieties
are provided by B. J. Birch \cite{B} and W.M. Schmidt \cite{S}.
We remark that recently in \cite{BP}, T.D. Browning and S.M. Prendiville improved on the mentioned result of B. J. Birch in case of hypersurfaces.
\\ \\
For prime solutions, there are results due to L. K. Hua \cite{H} for certain systems of homogeneous polynomials that are additive, for example
on the system $x_1^j + ... + x_n^j = N_j \ (1 \leq j \leq d)$, where $N_j \in \mathbb{N}$, and also on the Waring-Goldbach problem.
For the case of regular indefinite integral quadratic forms there is a result due to J. Liu \cite{L},
where the term `forms' refers to homogeneous polynomials. The first results regarding prime solutions of general systems of non-linear polynomials is contained in the recent work of B. Cook and \'{A}. Magyar \cite{CM}, which we state in Theorem \ref{CM theorem}.
Before we can state their result we need to introduce some notations.
We also note that there is a discussion in \cite{CM} on this topic from the point of view of some recent results in sieve theory,
which the list includes \cite{BGS, DRS, LS}. We refer the reader to \cite{CM} for the details on this discussion.

Let $\mathbf{f} = \{f_1, ..., f_{r_d}  \} \subseteq \mathbb{Q}[x_1, ..., x_n]$ be a system of forms of degree $d$.
We define the \emph{singular locus} $V_{\mathbf{f}}^*$ to be the set given by
$$
\text{rank } \left( \frac{ \partial f_r(\mathbf{x}) }{ \partial x_j }\right)_{ \substack{ 1 \leq r \leq r_d  \\ 1 \leq j \leq n } }  < r_d.
$$
Observe that this defines an affine variety over $\mathbb{C}$.
We define the \textit{Birch rank} $\mathcal{B} (\mathbf{f})$ to be the codimension of $V_{\mathbf{f}}^*$.
Let $\mathbf{b} = \{ b_1, ..., b_{r_d} \} \subseteq \mathbb{Q}[x_1,..., x_n]$ be a system of degree $d$ polynomials.
We let $f_{b_r}$ to be the degree $d$ portion of $b_r \ (1 \leq r \leq r_d)$, and extend the notion of the Birch rank to systems of degree $d$ polynomials by
defining the Birch rank of $\mathbf{b}$ to be
$$
\mathcal{B} (\mathbf{b}) :=  \mathcal{B} ( \{ f_{b_r} : 1 \leq r \leq r_d  \}).
$$

We define the following quantity
$$
\mathcal{M}_{\mathbf{b}}(N) := \sum_{\mathbf{x} \in [0, N]^n \cap \mathbb{Z}^n} \delta_{\mathbf{b} }(\mathbf{x} ),
$$
where
\begin{eqnarray}
\notag
\delta_{\mathbf{b}} (\mathbf{x} )
=
\left\{
    \begin{array}{ll}
         \prod_{1 \leq i \leq n} \log p_i,
         &\mbox{if } x_i = p_i^{t_i}, p_i \mbox{ is prime}, t_i \in \mathbb{N} \ (1 \leq i \leq n)\\
         & \mbox{ and } b_r(\mathbf{x}) = 0 \ (1 \leq r \leq r_d),
         \\
         & \\
         0,
         &\mbox{otherwise }.
    \end{array}
\right.
\notag
\end{eqnarray}
When the system only has one polynomial, in other words $\mathbf{b} = \{ b \}$, we denote
the quantities $\mathcal{M}_{\mathbf{b}}(N)$ and $\delta_{\mathbf{b}} (\mathbf{x} )$ as $\mathcal{M}_b(N)$ and $\delta_b(\mathbf{x})$, respectively. \\ \\
For a positive integer $q$, put $\mathbb{U}_q$ for the group of units in $\mathbb{Z}/q \mathbb{Z}$. We will use the notation $e(x)$ to denote $e^{2\pi i x}$. Let us define
\begin{equation} \label{BQ}
B( q ) = \sum_{ \mathbf{m} \in \mathbb{U}_q^{r_d}} \frac{1}{\phi(q)^n} \ \sum_{\mathbf{k} \in \mathbb{U}_q^n}\  e\left( \sum_{r=1}^{r_d} {b}_r(\mathbf{k}) \cdot {m_r}/q \right),
\end{equation}
where $\phi$ is Euler's totient function.
For each prime $p$, we define the following quantity
\begin{equation}
\label{def mu p}
\mu(p) =  1  + \sum_{t=1}^{\infty} B( p^t)
\end{equation}
when the sum converges. For $\boldsymbol{\eta} \in \mathbb{R}^{r_d}$, let us write
\begin{equation}
\label{I-eta} \mathcal{I}(\boldsymbol{\eta}) = \int_{[0,1]^n} e \left( \sum_{r=1}^{r_d} \eta_r  \cdot f_{b_r}(\boldsymbol \xi) \right) \mathbf{d} \boldsymbol \xi.
\end{equation}
We also define
$$
\mu(\infty) = \int_{\mathbb{R}^{r_d}} \mathcal{I}( \boldsymbol{\eta}) \ \mathbf{d} \boldsymbol{\eta}
$$
when $\mathcal{I}( \boldsymbol{\eta})$ is integrable.
We may now phrase the main result of B. Cook and \'{A}. Magyar in \cite{CM} as follows:
\begin{thm} \cite[Theorem 1]{CM}
\label{CM theorem}
Let $\mathbf{b} = \{ b_1, ..., b_{r_d} \} \subseteq \mathbb{Z}[x_1,..., x_n]$
be a system of degree $d$ polynomials. 
Then there exists a positive number $B_{d, r_d}$ dependent only on $d$ and $r_d$ such that the following holds. If $ \mathcal{B}(\mathbf{b}) > B_{d, r_d}$, then 
there exists $c>0$ such that
$$
\mathcal{M}_{\mathbf{b} }(N) = \ \prod_{p \ \text{prime}} \mu(p ) \cdot  \mu(\infty ) \cdot N^{n-d r_d} + O\left( \frac{N^{n-d r_d}}{(\log N)^c } \right).
$$
\end{thm}
The quantities $\prod_{p \ \text{prime}} \mu(p )$ and $\mu(\infty)$ are positive assuming certain local conditions, which the details can be found in \cite[pp. 704]{CM}. We will defer the discussion until the paragraph after Theorem \ref{the main theorem}. We also remark that the result of B. Cook and \'{A}. Magyar in \cite{CM} deals with the set of equations $b_r(\mathbf{x}) = L_r \ (1 \leq r \leq r_d)$ for a fixed $\mathbf{L} = (L_1, ..., L_{r_d}) \in \mathbb{Z}^n$
instead of only considering zeros of the polynomials.

Given a form $f \in \mathbb{Q}[x_1, ..., x_n]$ of degree at least $2$, we define the $h$-\textit{invariant},
also known as the \textit{rational Schmidt rank}, $h(f)$ to be the
least positive integer $h$ such that $f$ can be written identically as
\begin{equation}
\label{defn of hinv}
f = U_1 V_1 + ... + U_h V_h,
\end{equation}
where each $U_i$ and $V_i$ are forms in $\mathbb{Q}[x_1, ..., x_n]$ of degree at least $1 \ (1 \leq i \leq h)$.
We also let $h(0) = 0$. Note we have $h(f) = 0$ if and only if $f = 0$.
Let $\mathbf{f} = \{f_1, ..., f_{r_d}  \} \subseteq \mathbb{Q}[x_1, ..., x_n]$ be a system of forms of degree $d$. We generalize the
definition of $h$-invariant for a single form, and
define the $h$-invariant of $\mathbf{f}$ by
\begin{equation}
h(\mathbf{f}) = \min_{\boldsymbol{\mu} \in \mathbb{Q}^{r_d} \backslash \{ \boldsymbol{0} \}}  h( \mu_{1} f_1 + ... + \mu_{r_d} f_{r_d} ).
\end{equation}
Given an invertible linear transformation $T \in GL_n(\mathbb{Q})$, let $\mathbf{f} \circ T = \{f_1 \circ T, ..., f_{r_d} \circ T \}$.
It follows from the definition of the $h$-invariant that
$$
h(\mathbf{f}) = h(\mathbf{f} \circ T).
$$
Let $\mathbf{b} = ( b_1, ..., b_{r_d} ) \subseteq \mathbb{Q}[x_1,..., x_n]$ be a system of degree $d$ polynomials.
We let $f_{b_r}$ to be the degree $d$ portion of $b_r \ (1 \leq r \leq r_d)$, and define
\begin{equation}
h(\mathbf{b}) := h(\{ f_{b_r} : 1 \leq r \leq r_d  \}).
\end{equation}

Although the Birch rank, the codimension of the singular locus, seems to be a natural measure of singularity, it is not clear if it accurately determines whether one can apply the circle method or not. For example, in \cite{DamS} the dimension of the singular locus
is replaced by the maximal dimension of the singular loci of forms in the linear system of the given forms.
For the circle method to work, we need an effective bound on the exponential sum over the minor arcs (defined in Section \ref{section circle method}). Let $\mathbb{F}_p$ be the finite field of order $p$, where $p$ is prime, and let $f$ be a degree $d$ polynomial over $\mathbb{F}_p$. It was proved that if $f$ lacks equidistribution, then $f$ can be written as a function of a `small' number of lower degree polynomials in \cite{GT1} when $d < p$, and further the restriction $d< p$ was removed in \cite{KL}.
Though this statement is over finite fields of prime order,
considering the close connection between the equidistribution of a polynomial and its exponential sum presented by Weyl equidistribution criterion,
it is possible that the question of whether one can apply the circle method or not depends more on the structure of the polynomials rather than the singularities. The $h$-invariant is useful in this aspect.
From this perspective, it is a natural question to consider if we can exchange the hypotheses on Birch rank in Theorem \ref{CM theorem}
for suitable hypotheses on the $h$-invariant.
In fact, B. Cook and \'{A}. Magyar conjectured in \cite[pp. 736]{CM} that Theorem \ref{CM theorem} holds assuming only the largeness of the $h$-invariant.
It is known that large Birch rank implies large $h$-invariant, since we have
$h(\mathbf{f})\geq 2^{1-d} \mathcal{B}(\mathbf{f})$ by \cite[Lemma 16.1, (10.3), (10.5), (17.1)]{S}.

In this paper, we give a partial solution to the conjecture of B. Cook and \'{A}. Magyar. We establish the conjecture for polynomials with an additional assumption. However, our assumption is redundant for quadratic polynomials and cubic polynomials, therefore we can establish the conjecture of B. Cook and \'{A}. Magyar unconditionally for quadratic and cubic polynomials.
Given a form $f \in \mathbb{Q}[x_1, ..., x_n]$ with $h = h(f)$, let us define \textit{linear count of the $h$-invariant of $f$} to be
the following quantity
$$
h^{\star}(f) = \max ( |\{ U_i  : \deg U_i = 1 \}| ),
$$
where the maximum is over all representations of the shape
$$
f = U_1 V_1 + ... + U_h V_h,
$$
$U_i$ and $V_i$ are rational forms, and $1 \leq \deg U_i \leq \deg V_i \ (1 \leq i \leq h)$.
In other words, $h^{\star}(f)$ is the maximum number of linear forms involved
in the representation of $f$ as a sum of $h = h(f)$ products of rational forms.
Clearly, we have $h^{\star}(f) \leq h(f).$
For a degree $d$ polynomial $b(\mathbf{x}) \in \mathbb{Q}[x_1, ..., x_n]$, we define $h(b):=h(f_b)$
where $f_b(\mathbf{x})$ is the degree $d$ portion of $b(\mathbf{x})$.
We note that any polynomial $b(\mathbf{x})$ of degree $2$ or degree $3$ satisfies
$$
h(b) = h^{\star}(b).
$$

Let $b(\mathbf{x}) \in \mathbb{Z}[x_1,..., x_n]$. Let $\Lambda$ be the von Mangoldt function, where
$\Lambda(x)$ is $\log p$, if $x$ is a power of prime $p$, and $0$ otherwise.
We define
\begin{equation}
\label{def of exp sum T}
T(b; \alpha) := \sum_{\mathbf{x} \in [0, N]^n \cap \mathbb{Z}^n} \Lambda(\mathbf{x}) \ e( \alpha \cdot b(\mathbf{x})),
\end{equation}
where
$$
\Lambda(\mathbf{x}) = \Lambda(x_1) ... \Lambda(x_n)
$$
for $\mathbf{x} = (x_1, ... , x_n) \in (\mathbb{Z}_{\geq 0})^n$.
By the orthogonality relation, we have
\begin{equation}
\label{defn of MbN}
\mathcal{M}_{b}(N) = \sum_{\mathbf{x} \in [0, N]^n \cap \mathbb{Z}^n} \delta_b(\mathbf{x} ) = \int_0^1 T(b; \alpha) \ d\alpha. 
\end{equation}
\iffalse
where
\begin{eqnarray}
\notag
\delta_b(\mathbf{x} ) =
\left\{
    \begin{array}{ll}
         \prod_{1 \leq i \leq n} \log p_i,
         &\mbox{if } x_i = p_i^{t_i}, p_i \mbox{ is prime }(1 \leq i \leq n), \mbox{ and } b(\mathbf{x}) = 0,\\
         0,
         &\mbox{otherwise }.
    \end{array}
\right.
\end{eqnarray}
\fi
We obtain the following theorem by estimating the integral ~(\ref{defn of MbN}), which is the main result of this paper.
\begin{thm}
\label{the main theorem}
Let $b(\mathbf{x}) \in \mathbb{Z}[x_1, ..., x_n]$ be a polynomial of degree $d$.
Then there exists a positive number $A_{d}$ dependent only on $d$ such that the following holds.
If $h^{\star}(b) > A_d$, then there exists $c>0$ such that
$$
\mathcal{M}_{b}(N) =  \prod_{p \ \text{prime}} \mu(p) \cdot  \mu(\infty) \cdot N^{n-d} + O\left( \frac{N^{n-d}}{(\log N)^{c} } \right).
$$
\end{thm}

We prove in Lemma \ref{singular series lemma} that the condition $h^{}(b) \geq  h^{\star}({b}) > A_{d}$ in fact guarantees
$$
\prod_{p \ \text{prime}} \mu(p) \ll 1.
$$
If the affine variety defined by the equation ~(\ref{main system}), $b(\mathbf{x}) =0$,  has a non-singular solution in $\mathbb{Z}_p^{\times}$, the units of $p$-adic integers, then we also have
$$
\mu(p) > 0.
$$
Furthermore, if the equation ~(\ref{main system}) has a non-singular solution in $\mathbb{Z}_p^{\times}$ for every prime $p$, then we have
$$
\prod_{p \ \text{prime}} \mu(p) > 0.
$$
These claims are justified in Section \ref{section singular series}.
As stated in \cite[pp. 704]{CM}, we have $\mu(\infty) > 0$ if the affine variety defined by $f_b$ has a
non-singular real zero in the interior of $\mathfrak{B}_0 = [0,1]^n$.

The following result is an immediate consequence of Theorem \ref{the main theorem}, which replaces the assumption of large Birch rank
in Theorem \ref{CM theorem}
with large $h$-invariant for
quadratic and cubic polynomials.
\begin{cor}
\label{the main theorem 2}
Let $d = 2$ or $3$, and $b(\mathbf{x}) \in \mathbb{Z}[x_1, ..., x_n]$ be a polynomial of degree $d$.
Then there exists a positive number $A_{d}$ dependent only on $d$ such that the following holds.
If $h(b) > A_d$, then there exists $c>0$ such that
$$
\mathcal{M}_{b}(N) = \ \prod_{p \ \text{prime}} \mu(p) \cdot  \mu(\infty) \cdot N^{n-d} + O\left( \frac{N^{n-d}}{(\log N)^c } \right).
$$
\end{cor}

In \cite{CM}, Theorem \ref{CM theorem} is obtained via the Hardy-Littlewood circle method, where
roughly speaking the minor arc estimate is based on decomposing the forms with large Birch rank
into two components which both have sufficiently large Birch rank. We
establish something similar in terms of the $h$-invariant as well, exploiting the fact that the representation
~(\ref{defn of hinv}) has enough linear terms. We also modify the method in \cite{CM} to better suit our purposes, which
is in terms of the $h$-invariant instead of the Birch rank.

The organization of the rest of the paper is as follows. In Section \ref{section h-inv}, we prove some basic
properties of the $h$-invariant, and we also describe the decomposition of the forms.
A sufficiently large linear count of the $h$-invariant allows us to massage our system into something amenable to the circle method, through a process called regularization. We collect results related to the regularization process in Section \ref{Sec reg lem}.
In Section \ref{technical estimate}, we obtain results from \cite{S} based on Weyl differencing in terms of polynomials instead of forms as in \cite{S}. We chose to present the details in Section \ref{technical estimate} to make certain dependency
of the constants explicit, because it plays an important role in our estimates. We then obtain minor arc estimates in
Section \ref{section minor arc}, and major arc estimates in Section \ref{section major arcs}. Finally, we present our conclusion
in Section \ref{sec concln}.

Throughout the paper we do not distinguish between the two terms `homogeneous polynomial' and `form', and
we will be using these terms interchangeably. We use $\ll$ and $\gg$ to denote Vinogradov's well-known notation.
We would also like to thank Dr. Damaris Schindler for many helpful comments.

\section{Properties of the $h$-invariant}
\label{section h-inv}
We begin this section by proving two basic lemmas regrading the properties of the $h$-invariant.
Let $f \in \mathbb{Q}[x_1, ..., x_n]$ be a form of degree $d$. For $1 \leq i \leq n$, let $f |_{x_i=0} = f(x_1, ..., x_{i-1}, 0, x_{i+1},..., x_n ) \in \mathbb{Q}[x_1, ..., x_n]$, which is either $0$ or a form of degree $d$.
We prove the following simple lemma.
\begin{lem}
\label{h ineq 1}
Let ${f} \in \mathbb{Q}[x_1, ..., x_n]$ be a form of degree $d>1$.
Then for any  $1 \leq i \leq n$, we have
$$
h( {f} ) -1   \leq h( {f} |_{x_i=0} ) \leq h( {f} ).
$$
\end{lem}

\begin{proof}
Without loss of generality, we consider the case $i=1$. Let us write
\begin{equation}
\label{lem 1 eqn 1}
f(x_1, ..., x_n) = x_1 g(x_1, ..., x_n) + f(0, x_2, ..., x_n).
\end{equation}
Clearly, $g(\mathbf{x})$ is either identically $0$ or a form of degree $d-1$.
Let $h = h( {f} )$ and $h' = h( {f} |_{x_1=0} )$.
By the definition of $h$-invariant, we can find rational forms $U_{j'}, V_{j'} \ (1 \leq j' \leq h')$ of positive degree
that satisfy
$$
f(0, x_2, ..., x_n)= U_1 V_1 + ... + U_{h'} V_{h'}.
$$
Note if $h'=0$, we assume the right hand side to be identically $0$.
By substituting the above equation into ~(\ref{lem 1 eqn 1}), we obtain
$$
f  = x_1 g+ U_1 V_1 + ... + U_{h'} V_{h'}.
$$
Because $g(\mathbf{x})$ is either $0$ or a form of degree $d-1$, it follows that
$$
h \leq 1 + h'.
$$

For the other inequality, let $u_{j}, v_{j} \ (1 \leq j \leq h)$ be rational forms of positive degree that
satisfy
\begin{equation}
\label{f=u1v1...}
f  = u_1 v_1 + ... + u_{h} v_{h}.
\end{equation}
By substituting $x_1 = 0$ into each form on both sides of the equation, it is clear that
we obtain $h' \leq h$. This completes the proof of the lemma.
We add a remark that in the special case when $f$ satisfies
$$
f  = x_1 v_1 + u_2 v_2 + ... + u_{h} v_{h},
$$
in other words when we have $u_1 = x_1$ in ~(\ref{f=u1v1...}),
we easily obtain $h' = h-1$.
\end{proof}

We also have the following lemma, which is a generalization of Lemma \ref{h ineq 1}.
We omit the proof as it is similar to that of Lemma \ref{h ineq 1}.
\begin{lem}
\label{h ineq 1'}
Let $\mathbf{f} = \{f_1, ..., f_{r_d} \} \subseteq \mathbb{Q}[x_1, ..., x_n]$ be a system of forms of degree $d>1$.
Suppose $h(\mathbf{f})>1$. Then for any  $1 \leq i \leq n$, we have
$$
h( \mathbf{f} ) -1   \leq h( \mathbf{f} |_{x_i=0} ) \leq h( \mathbf{f} ),
$$
where $\mathbf{f} |_{x_i=0} = \{ f_1|_{x_i=0}, ... , f_{r_d}|_{x_i = 0} \}$.
\end{lem}

Let $f(\mathbf{x}) \in \mathbb{Q}[x_1, ..., x_n]$ be a homogeneous polynomial, and let $h = h(f)$ and $0 < M  \leq h$. Suppose we have
$$
f = u_1 V_1 + ... + u_{M} V_{M} + U_{M+1} V_{M+1} + ... + U_h V_h,
$$
where each $u_i$ is a linear rational form $(1 \leq i \leq M)$, and each $U_{i'}$ and $V_j$
are rational forms of positive degree $(M+1 \leq i' \leq h, 1 \leq j \leq h)$. It can be easily
verified that the linear forms $u_1, ..., u_M$ are linearly independent over $\mathbb{Q}$.
Then by considering the reduced row echelon form of the matrix formed by the coefficients of $u_1, ..., u_M$, and relabeling the
variables if necessary, we may suppose without loss of generality that
\begin{equation}
\label{h-inv decomp after linear transfn}
f = (x_1 + \ell_1) v_1 + ... + (x_M + \ell_M) v_M + u_{M+1} v_{M+1} + ... + u_h v_h,
\end{equation}
where each $\ell_i$ is a linear form in $\mathbb{Q}[x_{M+1}, ..., x_n] \ (1 \leq i \leq M)$, and each $u_{i'}$ and $v_j$
are rational forms of positive degree $(M+1 \leq i' \leq h, 1 \leq j \leq h)$.
We then define $g_M \in \mathbb{Q}[x_1, ..., x_n]$ in the following manner,
\begin{eqnarray}
\label{def gM}
f ( x_1, x_2, ..., x_n ) 
= g_M (x_1, ..., x_n) + f(-\ell_1, ...,  - \ell_M, x_{M+1}, ..., x_n).
\end{eqnarray}
We note that there is no ambiguity for defining the polynomial
$$
f(-\ell_1, ..., -\ell_M, x_{M+1}, ..., x_n) \in \mathbb{Q}[x_{M+1}, ..., x_n]
$$
obtained by substitution, because each $\ell_i \in  \mathbb{Q}[x_{M+1}, ..., x_n] \ (1 \leq i \leq M)$.
It is also clear that
\begin{equation}
\label{gM is 0}
g(-\ell_1, ...,  - \ell_M, x_{M+1}, ..., x_n) = 0.
\end{equation}

\begin{lem}
\label{h ineq2}
Let $1 \leq M \leq h$. Suppose a degree $d$ form $f(\mathbf{x}) \in \mathbb{Q}[x_1, ..., x_n]$ satisfies
~(\ref{h-inv decomp after linear transfn}). Define $g_M(\mathbf{x})$ as in ~(\ref{def gM}). Then we have
$$
h(g_M) \geq M   \ \ \text{  and } \ \  h( f( -\ell_1, - \ell_2, ... , - \ell_M, x_{M+1}, ..., x_n ) ) = h-M.
$$
\end{lem}

\begin{proof}
Since the linear forms $(x_1 - \ell_1), ... , (x_M - \ell_M)$ are linearly independent over $\mathbb{Q}$,
we can find $A \in GL_n(\mathbb{Q})$ such that
$$
\left( \begin{array}{c}
x_1 - \ell_1 \\
\vdots \\
x_M - \ell_M \\
x_{M+1} \\
\vdots \\
x_n \end{array} \right)
=
A \circ \left( \begin{array}{c}
x_1 \\
\vdots \\
x_n \end{array} \right).
$$
Let
$$
\widetilde{f} (\mathbf{x} )= f(A^{} \circ \mathbf{x} ).
$$
We then have
$$
\widetilde{f} (A^{-1} \circ \mathbf{x} )= f( \mathbf{x} ),
$$
and also that $h(\widetilde{f} ) = h(\widetilde{f} \circ A^{-1}  ) =  h(f) = h$.
Because $f(\mathbf{x})$ satisfies ~(\ref{h-inv decomp after linear transfn}),
it follows that $\widetilde{f} (\mathbf{x} )$ satisfies
$$
\widetilde{f} =  x_1 V_1 + ... + x_M V_M + U_{M+1} V_{M+1} + ... +  U_h V_h,
$$
where each $U_i$ and $V_j$ are rational forms of positive degree $(M+1 \leq i \leq h, 1 \leq j \leq h)$.

Recall each $\ell_i$ is a linear form in $\mathbb{Q}[x_{M+1}, ..., x_n] \ (1 \leq i \leq M)$.
Clearly, we have
\begin{eqnarray}
\widetilde{f} (0, ...,0, x_{M+1}, ...,  x_n ) &=&
f(A^{} \circ (0, ...,0, x_{M+1}, ...,  x_n ) )
\notag
\\
&=& f( -\ell_1, - \ell_2, ... , - \ell_M, x_{M+1}, ..., x_n ).
\notag
\end{eqnarray}
Then we can deduce from Lemma \ref{h ineq 1} (see the remark at the end of the proof of Lemma \ref{h ineq 1}) that
$$
h( f( -\ell_1, - \ell_2, ... , - \ell_M, x_{M+1}, ..., x_n ) )
= h(\widetilde{f} (0, ...,0, x_{M+1}, ...,  x_n )) = h-M.
$$
It then follows easily from the fact that $h(f) = h$, the definition of $h$-invariant, and ~(\ref{def gM}), that
$$
h(g_M) \geq M,
$$
for otherwise we obtain a contradiction.
\end{proof}

\section{Regularization lemmas}
\label{Sec reg lem}

In this section, we collect results from \cite{CM} and \cite{S} related to regular systems (see Definition \ref{def regular}) and the
regularization process. Given a system of rational forms $\textbf{f}$, via the regularization process we obtain
another system which has at most the expected number of integer solutions, cardinality bounded by a constant, and partitions the level sets
of $\textbf{f}$. This was an important component of the method in \cite{CM} used to split the
exponential sum in a controlled manner during the minor arc estimate. 
Throughout this section we use the following notation. Let $d, n >1$, and let $\textbf{f}$ be a system of forms in $\mathbb{Q}[x_1, ..., x_n]$ of degree less than or equal to $d$.
We denote $\textbf{f} = ({\textbf{f}}^{(d)}, ..., {\textbf{f}}^{(1)})$, where ${\textbf{f}}^{(i)}$
is the subsystem of all forms of degree $i$ in $\textbf{f} \  (1 \leq i \leq d)$. We label the elements of ${\textbf{f}}^{(i)}$ by
$$
{\mathbf{f}}^{(i)} = \{ f^{(i)}_1, ..., f^{(i)}_{r_i} \},
$$
where $r_i = | {\mathbf{f}}^{(i)} |$, the number of elements in ${\mathbf{f}}^{(i)}$.

\begin{defn}
\label{def regular}
Let $d>1$. Let $\boldsymbol{\psi} = (\boldsymbol{\psi}^{(d)}, ..., \boldsymbol{\psi}^{(1)})$ be a system of polynomials in $\mathbb{Q}[x_1, ..., x_n]$,
where $\boldsymbol{\psi}^{(i)}$ is the subsystem of all polynomials of degree $i$ in
$\boldsymbol{\psi} \  (1 \leq i \leq d)$.
We denote $V_{\boldsymbol{\psi}, \mathbf{0}} (\mathbb{Z})$ to be the set of solutions in $\mathbb{Z}^n$ of the equations
$$
\psi^{(i)}_j (\mathbf{x}) = 0 \ (1 \leq i \leq d, 1 \leq j \leq | \boldsymbol{\psi}^{(i)} |),
$$
which we denote by
$\boldsymbol{\psi}(\mathbf{x}) = \mathbf{0}$.
Let $r_i = | \boldsymbol{\psi}^{(i)} | \ (1 \leq i \leq d)$,
and let $D_{\boldsymbol{\psi}} = \sum_{i=1}^d i r_i$. We say the system $\boldsymbol{\psi}$ is \textit{regular} if
$$
| V_{\boldsymbol{\psi}, \mathbf{0}} (\mathbb{Z}) \cap [-N,N]^n | \ll N^{n-D_{\boldsymbol{\psi} }  }.
$$
\end{defn}
Similarly as above we also define $V_{\boldsymbol{\psi}, \mathbf{0}} (\mathbb{R})$ to be the set of solutions in $\mathbb{R}^n$
of the equations $\boldsymbol{\psi}(\mathbf{x}) = \mathbf{0}$.
\begin{thm}[Schmidt, \cite{S}]
\label{Schmidt main}
Let $d>1$. Let $\boldsymbol{\psi} = (\boldsymbol{\psi}^{(d)}, ..., \boldsymbol{\psi}^{(2)})$ be a system of rational polynomials
with notation as in Definition \ref{def regular},
and also let $\mathbf{f}^{(i)}$ be the system of degree $i$ portion of the polynomials $\boldsymbol{\psi}^{(i)} \ (2 \leq i \leq d)$.
We denote $r_i = | \boldsymbol{\psi}^{(i)} | = | \mathbf{f}^{(i)} | \ (2 \leq i \leq d)$, and $R_{\boldsymbol{\psi} }  = \sum_{i=2}^d r_i$.
If we have
$$
h(  \mathbf{f}^{(i)} )  
\geq d \ 2^{4 i}  (i!)  r_i R_{\boldsymbol{\psi} }  \ \ (2 \leq i \leq d),
$$
then the system $\boldsymbol{\psi}$ is regular.
\end{thm}
Let us denote
\begin{equation}
\label{def rho}
\rho_{d,i}(t) =  d \ 2^{4 i}  (i!)  t^2  \ \ (2 \leq i \leq d)
\end{equation}
so that for each $2 \leq i \leq d$, we have $\rho_{d,i}(t)$ is an increasing function, and
$$
\rho_{d,i}(R_{\boldsymbol{\psi} }  ,d) \geq d \ 2^{4 i}  (i!)  r_i R_{\boldsymbol{\psi} } .
$$

Note Theorem \ref{Schmidt main} is regarding a system of polynomials that does not contain any linear polynomials.
We prove Corollary \ref{cor Schmidt} for systems that contain linear forms as well.
Note the content of the following Corollary \ref{cor Schmidt} is essentially \cite[Corollary 3]{CM}.
\begin{cor}
\label{cor Schmidt}
Let $d>1$. Let $\boldsymbol{\psi} = (\boldsymbol{\psi}^{(d)}, ..., \boldsymbol{\psi}^{(1)})$ be a system of rational polynomials with notation as in Definition \ref{def regular}.
Suppose $\boldsymbol{\psi}^{(1)}$ only contains linear forms and that they are linearly independent over $\mathbb{Q}$.
We also let $\mathbf{f}^{(i)}$ be the system of degree $i$ portion of the polynomials $\boldsymbol{\psi}^{(i)} \ (1 \leq i \leq d)$.
We denote $r_i = | \boldsymbol{\psi}^{(i)} | = | \mathbf{f}^{(i)} | \ (1 \leq i \leq d)$, and $R_{\boldsymbol{\psi} }  = \sum_{i=1}^d r_i$.
For each $2 \leq i \leq d$, let $\rho_{d,i}(\cdot)$ be as in ~(\ref{def rho}).
If we have
$$
h(  \mathbf{f}^{(i)} ) \geq \rho_{d,i} (R_{\boldsymbol{\psi} }  - r_1) + r_1 \ \ (2 \leq i \leq d),
$$
then the system $\boldsymbol{\psi}$ is regular.
\end{cor}

\begin{proof}
We have $\boldsymbol{\psi}^{(1)} = \mathbf{f}^{(1)} = \{f_1^{(1)}, ..., f_{r_1}^{(1)} \}$. Let
$$
f_i^{(1)} = a_{i1}x_1+ ... + a_{in}x_n \ (1 \leq i \leq r_1),
$$
and denote the coefficient matrix of these linear forms to be $A = [a_{ij}]_{1\leq i \leq r_1, 1 \leq j \leq n}$.
Let $\mathbf{e}_{j}$ be the $j$-th standard basis of $\mathbb{R}^n \ (1 \leq j \leq n)$.
Since the linear forms $f_1^{(1)}, ..., f_{r_1}^{(1)}$ are linearly independent over $\mathbb{Q}$,
we can find an invertible linear transformation $T \in \ GL_n(\mathbb{Q})$,
where every entry of the matrix is in $\mathbb{Z}$, such that
$(f^{(1)}_i \circ T^{-1})(\mathbf{x}) = m_{n-i + 1 } x_{n-i+1}$, where $m_{n-i + 1 } \in \mathbb{Q} \backslash \{0\}  \ (1 \leq i \leq r_1)$.
For simplicity, let us denote $\mathbf{x}' = (x_{n-r_1+1}, ..., x_n)$.
Let
$$
Y:= V_{\mathbf{f}^{(1)}, \mathbf{0} } (\mathbb{R}) = \{ \mathbf{x} \in \mathbb{R}^n : \mathbf{f}^{(1)}(\mathbf{x}) = \mathbf{0} \} = \{ \mathbf{x} \in \mathbb{R}^n : A \circ \mathbf{x} = \mathbf{0} \} = \text{Ker}(A ),
$$
which is a subspace of codimension $r_1$. Since $T(Y) = \text{Ker}(A \circ T^{-1})$, it follows from our choice of $T \in \ GL_n(\mathbb{Q})$ that
$$
T(Y) = \mathbb{R} \mathbf{e}_1 + ... + \mathbb{R} \mathbf{e}_{n - r_1}.
$$
We also know there exit
$c' , C' > 0$ such that
$$
[-c'N, c'N]^n \subseteq T( [-N, N]^n) \subseteq   [-C'N,C'N]^n.
$$
Define $\boldsymbol{\psi}' = (\boldsymbol{\psi'}^{(d)}, ..., \boldsymbol{\psi'}^{(1)}) := \boldsymbol{\psi} \circ T^{-1}$, and let $\mathbf{f'}^{(i)}$ be the system of degree $i$ portion of the polynomials $\boldsymbol{\psi'}^{(i)} \ (1 \leq i \leq d)$. We then have $\mathbf{f'}^{(i)} = \mathbf{f}^{(i)} \circ T^{-1}$.
We can also verify that $V_{\boldsymbol{\psi}', \mathbf{0} } (\mathbb{R}) = T ( V_{\boldsymbol{\psi}, \mathbf{0} } (\mathbb{R}) ).$
Therefore, we obtain
$$
T( V_{\boldsymbol{\psi}, \mathbf{0} }(\mathbb{R}) \cap [-N, N]^n )  \subseteq V_{\boldsymbol{\psi}', \mathbf{0} }(\mathbb{R}) \cap [-C' N, C' N]^n,
$$
and since every entry of the matrix $T \in \ GL_n(\mathbb{Q})$ is in $\mathbb{Z}$, it follows that
\begin{eqnarray}
| V_{\boldsymbol{\psi}, \mathbf{0} } (\mathbb{Z})\cap [-N, N]^n   |
&\leq&  |V_{\boldsymbol{\psi}', \mathbf{0} }(\mathbb{Z}) \cap [-C' N, C' N]^n  |.
\label{eqn 1 in cor 3.3}
\end{eqnarray}
Let $\boldsymbol{\psi''} = (\boldsymbol{\psi'}^{(d)} |_{{\mathbf{x}' = \mathbf{0}}}, ..., \boldsymbol{\psi'}^{(2)}|_{{\mathbf{x}' = \mathbf{0}}} )$.
Since $\boldsymbol{\psi' }^{(1)} = \mathbf{0}$ is equivalent to $\mathbf{x}' = \mathbf{0}$, we have
\begin{equation}
\label{eqn 1' in cor 3.3}
|V_{\boldsymbol{\psi}', \mathbf{0} }(\mathbb{Z}) \cap [-C' N, C' N]^n  |
= |V_{\boldsymbol{\psi}'', \mathbf{0} }(\mathbb{Z}) \cap [-C' N, C' N]^{n-r_1}  |.
\end{equation}

Since the degree $i$ portion of $\boldsymbol{\psi'}^{(i)} |_{\mathbf{x}' = \mathbf{0}}$ is $\mathbf{f'}^{(i)} |_{{\mathbf{x}' = \mathbf{0}}}$ for each $2 \leq i \leq d$, we have by Lemma \ref{h ineq 1'} that
\begin{eqnarray}
\notag
h( \mathbf{f'}^{(i)} |_{{\mathbf{x}' = \mathbf{0}}} ) \geq   h( \mathbf{f'}^{(i)} ) - r_1 = h( \mathbf{f}^{(i)} ) - r_1 \geq \rho_{d,i}(R_{\boldsymbol{\psi}} - r_1, d).
\end{eqnarray}
Thus, it follows by Theorem \ref{Schmidt main} that
\begin{equation}
\label{eqn 2 in cor 3.3}
|V_{\boldsymbol{\psi}'', \mathbf{0} } (\mathbb{Z}) \cap [-C' N, C' N]^{n-r_1}  |
\ll N^{(n-r_1) - \sum_{i=2}^d i r_i }.
\end{equation}
Therefore, we obtain from ~(\ref{eqn 1 in cor 3.3}),  ~(\ref{eqn 1' in cor 3.3}), and ~(\ref{eqn 2 in cor 3.3}) that
\begin{eqnarray}
| V_{\boldsymbol{\psi}, \mathbf{0} }(\mathbb{Z}) \cap [-N, N]^n  | 
&\ll&
N^{ n - \sum_{i=1}^d i r_i }.
\notag
\end{eqnarray}

\end{proof}

Given $\mathbf{g} = \{g_1, ..., g_{r_d}  \} \subseteq \mathbb{Q}[x_1, ..., x_n]$, a system of forms of degree $d$, and a partition of variables $\mathbf{x} = (\mathbf{y}, \mathbf{z})$, we denote $\overline{\mathbf{g}}$ to be the system obtained by removing all the forms of $\mathbf{g}$ that depend only on the $\mathbf{z}$ variables. Clearly, if we have the trivial partition $\mathbf{x} = (\mathbf{y}, \mathbf{z})$, where $\mathbf{z} = \emptyset$, then $\overline{\mathbf{g}} = \mathbf{g}$.
For a form $g(\mathbf{x})$ over $\mathbb{Q}$, we define $h(g;\mathbf{z})$ to be the smallest number $h_0$ such that $g(\mathbf{x})$
can be expressed as
$$
g(\mathbf{x}) =  g(\mathbf{y}, \mathbf{z}) = \sum_{i=1}^{h_0} u_i v_i + w_0(\mathbf{z}),
$$
where $u_i, v_i$ are rational forms of positive degree $(1 \leq i \leq h_0)$, and
$w_0(\mathbf{z})$ is a rational form only in the $\mathbf{z}$ variables.
We also define $h(\mathbf{g}; \mathbf{z})$ to be
$$
h(\mathbf{g}; \mathbf{z}) = \min_{\boldsymbol{\lambda} \in \mathbb{Q}^{r_d} \backslash \{ \boldsymbol{0} \}}  h( \lambda_{1} g_1 + ... + \lambda_{r_d} g_{r_d}; \mathbf{z} ).
$$
If we have the trivial partition, then clearly we have $h(\mathbf{g}; \emptyset) = h(\mathbf{g}).$
We have the following lemma.
\begin{lem}[Lemma 2, \cite{CM}]
\label{Lemma 2 in CM}
Let $\mathbf{g} = \{g_1, ..., g_{r_d}  \} \subseteq \mathbb{Q}[x_1, ..., x_n]$ be a system of forms of degree $d$, and suppose we have a partition of variables $\mathbf{x} = (\mathbf{y}, \mathbf{z})$. Let $\mathbf{y}'$ be a distinct set of variables with the same number of variables as $\mathbf{y}$.
Then we have
$$
h(\mathbf{g}(\mathbf{y}, \mathbf{z}),  \mathbf{g}(\mathbf{y}', \mathbf{z}) ; \mathbf{z} ) = h( \mathbf{g} ; \mathbf{z} ).
$$
\end{lem}

The process in the following proposition is referred to as the regularization of systems in \cite{CM}, and it
is a crucial component of the method in \cite{CM}. 
\begin{prop}[Propositions 1 and 1', \cite{CM}]
\label{prop reg par}
Let $d>1$, and let $\boldsymbol{\mathcal{F}}$ be any collection of non-decreasing functions $\mathcal{F}_i: \mathbb{Z}_{\geq 0} \rightarrow \mathbb{Z}_{\geq 0} \ (2 \leq i \leq d)$. For a collection of non-negative integers $r_1, ..., r_d$, there exist constants
$$
C_1(r_1, ..., r_d, \boldsymbol{\mathcal{F}} ), ... , C_d(r_1, ..., r_d,  \boldsymbol{\mathcal{F}} )
$$
such that the following holds.

Given a system of integral forms $\mathbf{f} = ({\mathbf{f}}^{(d)}, ..., {\mathbf{f}}^{(1)}) \subseteq \mathbb{Z}[x_1, ..., x_n]$, where each
$\mathbf{f}^{(i)}$ is a system of $r_i$ forms of degree $i \ (1 \leq i \leq d)$, and a partition of variables $\mathbf{x} = (\mathbf{y}, \mathbf{z})$, there exists a system of forms
$\mathcal{R}( \mathbf{f}) = ({\mathbf{a}}^{(d)}, ..., {\mathbf{a}}^{(1)})$ satisfying the following.
Let $r'_i = |\mathbf{a}^{(i)}| \ (1 \leq i \leq d)$, and $R' = r'_1 + ... + r'_d$.
\newline

(1) Each form of the system $\mathbf{f}$ can be written as a rational polynomial expression in the forms of the system $\mathcal{R}( \mathbf{f})$.
In particular, the level sets of $\mathcal{R}( \mathbf{f})$ partition those of $\mathbf{f}$.

(2) For each $1 \leq i \leq d$, $r'_i$ is at most $C_i(r_1, ..., r_d, \boldsymbol{\mathcal{F}})$.

(3) The subsystem $({\mathbf{a}}^{(d)}, ..., { \mathbf{a} }^{(2)})$ satisfies $h({\mathbf{a}}^{(i)}) \geq F_i(R')$ for each $2 \leq i \leq d$. Moreover, the linear forms of subsystem ${ \mathbf{a} }^{(1)}$ are linearly independent over $\mathbb{Q}$.

(4) Let $\overline{\mathbf{a}}^{(i)}$ be the system obtained by removing from ${\mathbf{a}}^{(i)}$ all forms that depend only
on the $\mathbf{z}$ variables $(2 \leq i \leq d)$. Then the subsystem $({ \overline{\mathbf{a}}}^{(d)}, ..., { \overline{\mathbf{a}} }^{(2)})$,
satisfies $h({\overline{\mathbf{a}} }^{(i)} ; \mathbf{z}) \geq F_i(R')$ for each $2 \leq i \leq d$.
\end{prop}
We will be utilizing this proposition 
in Section \ref{section minor arc} to obtain minor arc estimates.

\section{Technical Estimates}
\label{technical estimate}

In this section, we provide results from \cite{S} related to Weyl differencing
that are necessary in obtaining estimates for the singular series in Section \ref{section singular series}.
The work here is similar to that of \cite{S}, which is in terms of forms instead of polynomials as in this section. It is stated in
\cite{S} with some explanation that similar results for polynomials also follow, but the details are not shown. We chose to present necessary details in order to make certain dependency of the constants explicit, which are crucial in our estimates. Let us denote $\mathfrak{B}_1 = [-1,1]^n$. We shall refer to $\mathfrak{B} \subseteq \mathbb{R}^n$ as a box, if $\mathfrak{B}$ is of the form
$$
\mathfrak{B} = I_1 \times ... \times I_n,
$$
where each $I_j$ is a closed or open or half open/closed interval $(1 \leq j \leq n)$.
Given a function $G(\mathbf{x})$, we define
$$
\Gamma_{d, G} (\mathbf{x}_1, ..., \mathbf{x}_d) = \sum_{t_1=0}^1 ... \sum_{t_d=0}^1 (-1)^{t_1 + ... + t_d} \
G( t_1 \mathbf{x}_1 + ...+ t_d \mathbf{x}_d ).
$$
Then it follows that $\Gamma_{d, G}$ is symmetric in its $d$ arguments, and that
$\Gamma_{d, G} (\mathbf{x}_1, ...,\mathbf{x}_{d-1}, \mathbf{0}) = 0$ \cite[Section 11]{S}.
It is clear from the definition that
$\Gamma_{d, G} + \Gamma_{d, G'} = \Gamma_{d, G + G'}.$
We also have that if $G$ is a form of degree $j$, where $d > j > 0$, then $\Gamma_{d, G}= 0$ \cite[Lemma 11.2]{S}.

For $\alpha \in \mathbb{R}$, let $\| \alpha \|$ denote the
distance from $\alpha$ to the closest integer. Given $\boldsymbol{\alpha} = (\alpha_1, ..., \alpha_n) \in \mathbb{R}^n$,
we let
$$
\| \boldsymbol{\alpha} \| = \max_{1 \leq i \leq n} \| \alpha_i \|.
$$

\begin{lem}\cite[Lemma 13.1]{S}
\label{Lemma 13.1 in S}
Suppose $G(\mathbf{x}) = G^{(0)} + G^{(1)}(\mathbf{x} ) + ... + G^{(d)}(\mathbf{x} )$, where $G^{(j)}$ is a form of degree $j$
with real coefficients $(1 \leq j \leq d)$, and $G^{(0)} \in \mathbb{R}$. Let $\mathfrak{B}$ be a box with sides $\leq 1$, let $P > 1$, and put
$$
S' = S'( G, P , \mathfrak{B} ) := \sum_{\mathbf{x} \in P \mathfrak{B} \cap \mathbb{Z}^n } e(G(\mathbf{x})).
$$
Let $\mathbf{e}_1, ... , \mathbf{e}_{n}$ be the standard basis vectors of $\mathbb{R}^n$.
Then for any $\varepsilon > 0$, 
we have
$$
|S'|^{2^{d-1}} \ll P^{ (2^{d-1} -d )n + \varepsilon} \sum \left(
\prod_{i=1}^n \min ( P, \|   \Gamma_{d, G^{(d)}} ( \mathbf{x}_1, ... , \mathbf{x}_{d-1}, \mathbf{e}_i  ) \|^{-1}   )
 \right),
$$
where the sum $\sum$ is over $(d-1)$-tuples of integer points $\mathbf{x}_1, ... , \mathbf{x}_{d-1}$ in $P \mathfrak{B}_1$,
and the implicit constant in $\ll$ depends only on $n,d,$ and $\varepsilon$.
\end{lem}

\begin{lem}\cite[Lemma 14.2]{S}
\label{Lemma 14.2 in S}
Make all the assumptions of Lemma \ref{Lemma 13.1 in S}. Suppose further that
$$
|S'| \geq P^{n-Q}
$$
where $Q>0.$
\iffalse
Then the number $N$ of $(d-1)$-tuples of integer points $\mathbf{x}_1, ..., \mathbf{x}_{d-1} \in P \mathfrak{B}_1$
with
$$
\| \Gamma_{d, G^{(d)}} ( \mathbf{x}_1, ... , \mathbf{x}_{d-1}, \mathbf{e}_i  ) \| < P^{-1 } \ (i=1,..., s)
$$
satisfies
$$
N \gg P^{ s(d-1) - 2^{d-1} Q - \varepsilon },
$$
where the implicit constant in $\gg$ depends only on $n,d, \varepsilon$.
\end{lem}

\begin{lem}\cite[Lemma 14.2]{S}
\label{Lemma 14.2 in S}
Make the same assumptions as in Lemma \ref{Lemma 13.2 in S}. Suppose $\eta > 0$, and
$$
\eta  \leq 1.
$$
\fi
Let $0 < \eta \leq 1.$
Then the number $N(\eta)$ of integral $(d-1)$-tuples
$$
\mathbf{x}_1, ..., \mathbf{x}_{d-1} \in P^{\eta} \mathfrak{B}_1
$$
with
$$
\|  \Gamma_{d, G^{(d)}} ( \mathbf{x}_1, ... , \mathbf{x}_{d-1}, \mathbf{e}_i  ) \| < P^{-d + (d-1) \eta } \ (i=1,..., n)
$$
satisfies
$$
N(\eta) \gg P^{ n(d-1)\eta - 2^{d-1} Q - \varepsilon },
$$
where the implicit constant in $\gg$ depends only on $n,d, \eta,$ and $\varepsilon$.
\end{lem}

Let $\boldsymbol{\psi} = \{ \psi_1, ..., \psi_{r_d}  \}$ be a system of rational polynomials of degree $d$.
Let $\mathbf{f} = \{ f_1, ..., f_{r_d} \}$ be the system of forms, where $f_i$ is the degree $d$ portion of
$\psi_i \ (1 \leq i \leq {r_d} )$.
We define the following exponential sum associated to  $\boldsymbol{\psi}$ and $\mathfrak{B}$,
\begin{equation}
\label{def of S 1}
S( \boldsymbol{\alpha}) = S( \boldsymbol{\psi},  \mathfrak{B} ;\boldsymbol{\alpha}) := \sum_{\mathbf{x} \in P \mathfrak{B} \cap \mathbb{Z}^n} e( \boldsymbol{\alpha}  \cdot \boldsymbol{\psi} (\mathbf{x}) ).
\end{equation}

Let $\mathbf{e}_1, ..., \mathbf{e}_n$ be the standard basis vectors of $\mathbb{C}^n$.
We define $\mathfrak{M}_d = \mathfrak{M}_d (\mathbf{f}^{}) $ to be the set of $(d-1)$-tuples $(\mathbf{x}_1, ..., \mathbf{x}_{d-1} ) \in (\mathbb{C}^n)^{d-1}$ for which the matrix
$$
[m_{ij}] = [ \Gamma_{d, f_{j}} ( \mathbf{x}_1, ... , \mathbf{x}_{d-1}, \mathbf{e}_i ) ] \ \ \ \  (1 \leq j \leq r_d, 1\leq i \leq n)
$$
has rank strictly less than $r_d$. For $R>0$, we denote $z_{R} (\mathfrak{M}_d)$ to be the number of integer points $(\mathbf{x}_1, ..., \mathbf{x}_{d-1} )$ on
$\mathfrak{M}_d$ such that $$\max_{1 \leq i \leq d-1} \max_{1 \leq j \leq n}  | x_{ij} | \leq R,$$
where $\mathbf{x}_i = (x_{i1}, ..., x_{in}) \ (1 \leq i \leq d-1)$.

Let $P > 1$, $Q > 0$, and $\varepsilon >0$ be given, and suppose that $d > 1$. We then have:
\begin{lem}\cite[Lemma 15.1]{S}
\label{Lemma 15.1 in S}
Given a box $\mathfrak{B}$ with sides $\leq 1$, define the sum  $S( \boldsymbol{\alpha})$ associated with $\boldsymbol{\psi}$ and $\mathfrak{B}$ as in ~(\ref{def of S 1}). Given $0 < \eta \leq 1$,
one of the following three alternatives must hold:

(i) $|  S( \boldsymbol{\alpha})  | \leq P^{n-Q}$.

(ii) there exists $n_0 \in \mathbb{N}$ such that
$$
n_0 \ll P^{r_d(d-1) \eta} \text{  and  } \|  n_0  \boldsymbol{\alpha} \| \ll P^{ -d + r_d (d-1) \eta}.
$$

(iii) $ z_{R} (\mathfrak{M}_d) \gg R^{ (d-1)n - 2^{d-1}(Q/ \eta) - \varepsilon}$
holds with $R = P^{\eta}$.
\newline
\newline
All implicit constants 
depend at most on
$n,d, r_d, \eta, \varepsilon$ and $\mathbf{f}$.
\end{lem}

\begin{proof}
Take $\boldsymbol{\alpha} \in \mathbb{R}^{r_d}$. Let
$\boldsymbol{\alpha} \cdot \boldsymbol{\psi}(\mathbf{x}) = G^{(0)}+ G^{(1)}(\mathbf{x}) + ... +  G^{(d)}(\mathbf{x})$, where
$G^{(j)}$ is a form of degree $j \ (1 \leq  j \leq d)$, and $G^{(0)} \in \mathbb{R}$.
Suppose $(i)$ fails, then we may apply Lemma \ref{Lemma 14.2 in S}. The number
$N(\eta)$ of integral $(d-1)$-tuples $\mathbf{x}_1$, ..., $\mathbf{x}_{d-1}$ in $P^{\eta} \mathfrak{B}_1$ with
\begin{equation}
\label{(15.2) in S}
\|   \Gamma_{d, G^{(d)}} ( \mathbf{x}_1, ... , \mathbf{x}_{d-1}, \mathbf{e}_i  ) \| < P^{-d + (d-1) \eta } \ (i=1,..., n)
\end{equation}
satisfies
$$
N(\eta) \gg R^{n(d-1) - 2^{d-1} (Q/ \eta) - \varepsilon },
$$
where $R = P^{\eta}$, and  the implicit constant in $\gg$ depends only on $n,d, \eta,$ and $\varepsilon$.

Recall $\boldsymbol{\psi} = \{ \psi_1, ...,  \psi_{r_d} \}$. Given $\mathbf{x}_1$, ..., $\mathbf{x}_{d-1}$
as above, we form the matrix
$$
[m_{ij}]_{\mathbf{x}_1, ... , \mathbf{x}_{d-1}} = [ \Gamma_{ d, \psi_j } ( \mathbf{x}_1, ..., \mathbf{x}_{d-1}, \mathbf{e}_i  ) ]
  \  \  (1 \leq i \leq n, 1 \leq j \leq r_d).
$$
Recall $f_j$ is the degree $d$ portion of $\psi_j \ (1 \leq j \leq r_d)$, and
$\mathbf{f}= \{ f_1, ...,  f_{r_d} \}$.
Since each $\psi_j$ is of degree $d$, it follows that $\Gamma_{d, \psi_j} = \Gamma_{d, f_j} \  (1 \leq j \leq r_d)$.
It is also clear that $G^{(d)}( \mathbf{x} ) = \boldsymbol{\alpha} \cdot \mathbf{f}(\mathbf{x})$.
Now if this matrix $[m_{ij}]_{\mathbf{x}_1, ... , \mathbf{x}_{d-1}}$ has rank less than $r_d$ for each of the $(d-1)$-tuples counted by $N(\eta)$, then by the definition of
$z_R(\mathfrak{M}_d)$ we have that
$$
z_R(\mathfrak{M}_d) \geq N(\eta) \gg R^{n(d-1) - 2^{d-1} (Q/ \eta) - \varepsilon },
$$
where again the implicit constant in $\gg$ depends only on $n,d, \eta,$ and $\varepsilon$.
Thus we have $(iii)$ in this case. Hence, we may suppose that at least one of these matrices, which we denote by $[m_{ij}]$, has rank $r_d$.
Without loss of generality, suppose the submatrix $M_0$ formed by taking the first $r_d$ columns of $[m_{ij}]$
has rank $r_d$. 

It follows from the definition of $\Gamma_{d, f_j}$ that every monomial occurring in
$\Gamma_{d, f_j} (\mathbf{x}_1, ..., \mathbf{x}_d)$ has some component of $\mathbf{x}_i$ as a factor for each $1 \leq i \leq d$ \cite[Proof of Lemma 11.2]{S}.
Also, the maximum absolute value of all coefficients of
$\Gamma_{d, f_j}$ is bounded by a constant dependent only on $d$ and the coefficients of $f_j$ \cite[Lemma 11.3]{S}.
Therefore, by the construction of $[m_{ij}]$ we have
$$
m_{ij} \ll R^{d-1},
$$
and hence
$$
n_0 := \det (M_0) \ll R^{r_d(d-1)} = P^{r_d(d-1)\eta},
$$
where the implicit constants in $\ll$ depend only on $r_d$ and $\mathbf{f}$.

We have
$$
\Gamma_{d, G^{(d)}} = \sum_{j=1}^{r_d} \Gamma_{d, \alpha_j f_j} = \sum_{j=1}^{r_d} \alpha_j \Gamma_{d, f_j}.
$$
Hence, from ~(\ref{(15.2) in S}) we may write
$$
\sum_{j=1}^{r_d} \alpha_j m_{ij} = c_i + \beta_i \ (1 \leq i \leq n),
$$
where the $c_i$ are integers and the $\beta_i$ are real numbers bounded by the right hand side of ~(\ref{(15.2) in S}).
Let $u_1, ..., u_{r_d}$ be the solution of the system of linear equations
\begin{equation}
\label{(15.3) in S}
\sum_{j=1}^{r_d} u_j m_{ij} = n_0 c_i \ (1 \leq i \leq r_d).
\end{equation}
Then
\begin{equation}
\label{(15.4) in S}
\sum_{j=1}^{r_d} (  n_0 \alpha_j - u_j  )m_{ij} = n_0 \beta_i \ (1 \leq i \leq r_d).
\end{equation}
By applying Cram\'{e}r's rule to ~(\ref{(15.3) in S}), it follows that the $u_j$ are integers. Also, by applying Cram\'{e}r's rule  to ~(\ref{(15.4) in S}),
we obtain that
\begin{eqnarray}
\| n_0 \alpha_j \| \leq | n_0 \alpha_j - u_j | \ll R^{ (d-1)(r_d - 1) }  P^{-d + (d-1) \eta} = P^{ - d + (d-1) r_d \eta } ,
\end{eqnarray}
where the implicit constant in $\ll$ depends only on $r_d$ and $\mathbf{f}$.
This completes the proof of Lemma \ref{Lemma 15.1 in S}
\end{proof}

We define $g_d( \mathbf{f} )$
to be the largest real number such that
\begin{equation}
\label{def gd}
z_P(\mathfrak{M}_d) \ll P^{n(d-1) - g_d( \mathbf{f} ) + \varepsilon}
\end{equation}
holds for each $\varepsilon >0$. It was proved in \cite[pp. 280, Corollary]{S} that
\begin{equation}
\label{h and g}
h(\mathbf{f}) < \frac{d!}{  (\log 2)^{d} }  \left( g_d( \mathbf{f} ) + (d-1)r_d (r_d - 1)  \right).
\end{equation}
Let
$$
\gamma_d = \frac{2^{d-1} (d-1) r_d}{ g_d( \mathbf{f} ) }
$$
when $g_d( \mathbf{f} ) > 0$.
We let $\gamma_d = + \infty$ if $g_d( \mathbf{f} ) = 0$.
We also define
\begin{equation}
\label{def gamma'}
\gamma'_d = \frac{ 2^{d-1} }{ g_d( \mathbf{f} ) } = \frac{ \gamma_d }{ (d-1) r_d }.
\end{equation}
\begin{cor}\cite[pp.276, Corollary]{S}
\label{cor 15.1 in S}
Given a box $\mathfrak{B}$ with sides $\leq 1$, we define the sum  $S( \boldsymbol{\alpha}) $ associated with $\boldsymbol{\psi}$ and $\mathfrak{B}$ as in ~(\ref{def of S 1}).
Suppose $\varepsilon' > 0$ is sufficiently small and $Q>0$ satisfies
$$
Q \gamma'_d < 1.
$$
Then one of the following alternatives must hold:

(i) $|  S( \boldsymbol{\alpha})  | \leq P^{n-Q}$.

(ii) there exists $n_0 \in \mathbb{N}$ such that
$$
n_0 \ll P^{Q \gamma_d + \varepsilon'} \text{  and  } \|  n_0 \boldsymbol{\alpha} \| \ll P^{ -d + Q \gamma_d + \varepsilon'},
$$
where the implicit constants in $\ll$ depend only on $n, d, r_d, \varepsilon', Q$ and $\mathbf{f}$.
\end{cor}
Note the fact that the implicit constant depends on $\mathbf{f}$, but not
on other lower order terms of $\boldsymbol{\psi}$ is an important feature which
we make use of in Section \ref{section singular series}.

\begin{proof}
Since $ Q \gamma'_d < 1$, we can choose $\varepsilon_1 > 0$ sufficiently small so that
$\eta = Q \gamma'_d + \varepsilon_1$ satisfies $0 < \eta \leq 1$.
Also with this choice of $\eta$, we have
\begin{eqnarray}
\frac{2^{d-1} Q} { \eta} &=& \frac{ 2^{d-1} Q}{ Q \gamma'_d + \varepsilon_1 }
\notag
\\
&=& \frac{ g_d( \mathbf{f} )}{ 1+ \varepsilon_1 g_d(\mathbf{f} ) / ( 2^{d-1} Q  ) }
\notag
\\
&<& g_d( \mathbf{f} ).
\notag
\end{eqnarray}
Then choose $\varepsilon_0 > 0$ such that $2^{d-1}Q / \eta + \varepsilon_0 < g_d( \mathbf{f} )$. By the definition of $g_d( \mathbf{f} )$ we have
$$
z_R(\mathfrak{M}_d) \ll R^{n(d-1) - 2^{d-1}Q / \eta - \varepsilon_0}.
$$
Thus in this case we see that the statement $(iii)$ in Lemma \ref{Lemma 15.1 in S}
can not occur with $0 < \varepsilon < \varepsilon_0$.
Also the equation $\eta = Q \gamma'_d + \varepsilon_1$ implies
$$
r_d (d-1) \eta = Q \gamma_d +  r_d (d-1) \varepsilon_1.
$$
Therefore, from Lemma \ref{Lemma 15.1 in S} (applying it with $0 < \varepsilon < \varepsilon_0$) we obtain our result with
$\varepsilon' = r_d (d-1) \varepsilon_1$.
\end{proof}

For the rest of this section, we assume $\boldsymbol{\psi}$ to be a system of integral polynomials of degree $d$.
When the polynomials $\boldsymbol{\psi}$ in question are over $\mathbb{Z}$, we consider the following.

Hypothesis ($\star$). Let $\mathfrak{B}$ be a box in $\mathbb{R}^n$. For any $\Delta > 0$,
there exists $P_1 = P_1(\mathbf{f}, \Omega,  \Delta, \mathfrak{B})$ such that for $P > P_1$,
each $\boldsymbol{\alpha} \in \mathbb{T}^{r_d}$ satisfies at least one of the following two alternatives. Either

(i)  $|S (\boldsymbol{\alpha}) | \leq P^{n - \Delta \Omega},$ or

(ii) there exists  $q = q(\boldsymbol{\alpha}) \in \mathbb{N}$ such that  
$$
q \leq P^{\Delta} \  \text{  and  } \  \|  q \boldsymbol{\alpha}  \| \leq P^{-d + \Delta}. 
$$

We will say that the restricted Hypothesis ($\star$) holds if the above condition holds for each $\Delta$ in $0 < \Delta \leq 1$.

The important thing to note here is that the lower bound for $P$ in Hypothesis ($\star$) only depends on $\mathbf{f}$, and not on $\boldsymbol{\psi}$. In other words, only the highest degree portion of the polynomials $\boldsymbol{\psi}$ play a role in this estimate.

\begin{prop}\cite[Proposition $\text{II}_0$]{S}
\label{prop omega and g}
Given a box $\mathfrak{B}$ with sides $\leq 1$, Hypothesis $(\star)$ is true for any $\Omega$ in
\begin{equation}
\label{(10.6) in S}
0 < \Omega < \frac{g_d( \mathbf{f} )}{2^{d-1} (d-1) r_d}.
\end{equation}

\end{prop}

\begin{proof}
It follows from ~(\ref{(10.6) in S}) that
$ \Omega \gamma_d < 1$.
We set $Q = \Delta \Omega$, and let $\varepsilon > 0$ be sufficiently small so that $Q \gamma_d + \varepsilon < \Delta$.
First, we suppose $\Delta \leq (d-1)r_d$. In this case, it follows that $Q \gamma'_d < 1$.
Thus it follows from Corollary \ref{cor 15.1 in S} that there exists $P_0 = P_0( \mathbf{f}, \Omega, \Delta)$ such that whenever $P > P_0$, either

$(i)$ $|S (\boldsymbol{\alpha})| \leq P^{n - \Delta \Omega}$, or

$(ii)$ there exists $q \in \mathbb{N}$ such that
$$
q \leq P^{\Delta} \text{ and } \|  q \boldsymbol{\alpha} \| \leq P^{-d + \Delta}.
$$
On the other hand, if $\Delta > (d-1)r_d$, then the case $(ii)$ above is always true by Dirichlet's Theorem on Diophantine approximation.
\end{proof}

For each $q \in \mathbb{N}$, we denote $\mathbb{U}_q$ as the group of units in $\mathbb{Z}/ q \mathbb{Z}.$ 
Given $\mathbf{m} \in \mathbb{U}_q^{r_d}$, we define
\begin{equation}
\label{def of E}
E(q^{-1} \mathbf{m}) = E( \boldsymbol{\psi}, q ; q^{-1} \mathbf{m} ) := q^{-n} \sum_{ \mathbf{x} \ (\text{mod }q) } e(q^{-1} \ \mathbf{m} \cdot \boldsymbol{\psi} (\mathbf{x}) ).
\end{equation}

\begin{lem}\cite[Lemma 7.1]{S}
\label{bound on E}
Suppose $\Omega$ satisfies ~(\ref{(10.6) in S}).
Then for $0 < Q < \Omega$, we have
\begin{equation}
\label{(7.1) in S}
| E (q^{-1} \mathbf{m} ) | \ll q^{- Q},
\end{equation}
where the implicit constant in $\ll$ depends only on $\mathbf{f}$, $Q$ and $\Omega$.
\end{lem}
Again the fact that the implicit constant depends on $\mathbf{f}$, but not
on other lower order terms of $\boldsymbol{\psi}$ becomes crucial when we apply this lemma in Section \ref{section singular series}.

\begin{proof}
Since $E(q^{-1} \mathbf{m} ) = q^{-n} S( \boldsymbol{\alpha} )$ with $\boldsymbol{\alpha}  = q^{-1} \mathbf{m}$, $P = q$ and $\mathfrak{B} = [0,1)^{r_d}$,
and with our choice of $\Omega$
we know that Hypothesis ($\star$) is satisfied by Proposition \ref{prop omega and g}. Thus we apply it with $\Delta = Q / \Omega < 1$. Let $q$ be sufficiently large, and suppose we are in case $(ii)$ of Hypothesis ($\star$). Then we know there
exists $q_0 \leq q^{\Delta} < q$ (when $q \not = 1$) with
$$
\|  q_0 q^{-1} \mathbf{m}  \| \leq q^{-d + \Delta} < q^{-1}.
$$
Since $(\mathbf{m}, q) = 1$, this is not possible.
Therefore, we must have case $(i)$ of Hypothesis ($\star$), which is precisely the inequality
~(\ref{(7.1) in S}).
\end{proof}

\section{Hardy-Littlewood Circle Method: Minor Arcs}
\label{section circle method}
\label{section minor arc}

For each $q \in \mathbb{N}$, recall we let $\mathbb{U}_q$ be the group of units in $\mathbb{Z}/ q \mathbb{Z}$. When $q=1$, we let $\mathbb{U}_1 = \{0\}$. For a given value of $C>0$ and an integer $q \leq (\log N)^C$, we define the \textit{major arc}
$$
\mathfrak{M}_{{m}, q}(C) = \{  {\alpha} \in [0,1] : | \alpha - m / q | \leq N^{-d} (\log N)^C  \}
$$
for each $m \in \mathbb{U}_q$. These arcs are disjoint for $N$ sufficiently large,
and we define
$$
\mathfrak{M}(C) = \bigcup_{q \leq (\log N)^C } \bigcup_{ {m} \in \mathbb{U}_q }  \mathfrak{M}_{{m}, q}(C).
$$
We then define the \textit{minor arcs} to be
$$
\mathfrak{m}(C) = [0,1] \backslash \mathfrak{M}(C).
$$

We obtain the following bound on the minor arcs.
\begin{prop}
\label{prop minor arc bound}
Let $b(\mathbf{x}) \in \mathbb{Z}[x_1, ..., x_n]$ be a polynomial of degree $d$. Let $T(b; \alpha)$
be defined as in ~(\ref{def of exp sum T}).
Then there exists a positive number $A_d$ dependent only on $d$ such that the following holds. Suppose $b(\mathbf{x})$
satisfies $h^{\star}(f_b) > A_d$.
Then, given any $c>0$, there exists $C>0$ such that
$$
\int_{\mathfrak{m}(C)}  T(b; \alpha)  \ d {\alpha}  
\ll
\frac{N^{n - d } }{(\log N)^{c}}.
$$
\end{prop}
\begin{proof}
For simplicity, we denote $f = f_b$ for the rest of the proof.
We let $h = h(f)$, and let $0 < M < h^{\star}(f) \leq h$ to be chosen later.
As explained in the paragraph before ~(\ref{h-inv decomp after linear transfn}),
by relabeling the variables if necessary we have
$$
f = (x_1 + \ell_1) v'_1 + ... + (x_M + \ell_M) v'_M + u'_{M+1}v'_{M+1} + ... + u'_h v'_h,
$$
where each $\ell_i$ is a linear form in $\mathbb{Q}[x_{M+1}, ..., x_n] \ (1 \leq i \leq M)$,
and each $u'_{i'}$ and $v'_j$ are rational forms of positive degree $(M+1 \leq i' \leq h, 1 \leq j \leq h)$.
We can then find a monomial $x_{i_1}x_{i_2}... x_{i_d}$, where $M < i_1 \leq i_2 \leq ... \leq i_d$, of $f$ with a non-zero coefficient.
This is the case, for otherwise it means that every monomial of $f$ is divisible by one of $x_1, ..., x_M$, and consequently that
$h = h(f)\leq M$, which is a contradiction.
We denote the distinct variables of $\{ x_{i_1}, x_{i_2}, ... , x_{i_d} \} \subseteq \{x_{M+1}, ..., x_n \}$
by $\{w_1, ..., w_K \}$, and let $\mathbf{w} = (w_1, ..., w_K)$. Clearly, we have $K \leq d$.
We selected these $K$ variables for the purpose of applying Weyl differencing later.
We also label $\mathbf{y} = (x_1,..., x_M) = (y_1, ..., y_M)$ for notational convenience,
let $\mathbf{z} = \{ x_{M+1}, ..., x_n \} \backslash \mathbf{w}$, and denote $\mathbf{z} = (z_1, ..., z_{n - M - K})$.
We note that each $\ell_i$ is a rational linear form only in the $\mathbf{w}$ and the $\mathbf{z}$ variables $(1 \leq i \leq h)$.

We define $g_M$ with respect to $f$ as in ~(\ref{def gM}).
By Lemma \ref{h ineq2}, we have
\begin{equation}
\label{main prop eqn 2 - 2}
{f}(\mathbf{x}) = f( \mathbf{w}, \mathbf{y}, \mathbf{z}) = g_M( \mathbf{w}, \mathbf{y}, \mathbf{z} ) + {f} (\mathbf{w}, (- \ell_1, ..., -\ell_M), \mathbf{z}),
\end{equation}
where
\begin{equation}
\label{main prop eqn 1 - 2}
h_{  }(g_M( \mathbf{w}, \mathbf{y}, \mathbf{z} ) ) \geq M \ \ \text{ and } \ \ h_{  } ({f} (\mathbf{w}, (- \ell_1, ..., -\ell_M), \mathbf{z})) = h-M.
\end{equation}
We then have
\begin{equation}
\label{gM is 0'}
{f}(\mathbf{0}, \mathbf{y}, \mathbf{z}) = g_M( \mathbf{0}, \mathbf{y}, \mathbf{z} ) + {f} (\mathbf{0}, (- \ell_1 |_{\mathbf{w} = \mathbf{0}}, ..., -\ell_M |_{\mathbf{w} = \mathbf{0}}),  \mathbf{z}).
\end{equation}
Let us denote
$$
f_M(\mathbf{z} ) = {f} (\mathbf{0}, (- \ell_1 |_{\mathbf{w} = \mathbf{0}}, ..., -\ell_M |_{\mathbf{w} = \mathbf{0}}), \mathbf{z}).
$$
Consequently, we obtain from Lemma \ref{h ineq 1} and ~(\ref{main prop eqn 1 - 2}) that
\begin{equation}
\label{h eqn 1}
h_{  }(g_M( \mathbf{0}, \mathbf{y}, \mathbf{z} ) ) \geq M-K \geq M-d,
\end{equation}
and
\begin{equation}
\label{h eqn 2}
h ( f_M(\mathbf{z} ) ) \geq h - M - K \geq h- M- d.
\end{equation}

Let
$$
b_M(\mathbf{z}) = {b}(\mathbf{0}, (- \ell_1 |_{\mathbf{w} = \mathbf{0}}, ..., -\ell_M |_{\mathbf{w} = \mathbf{0}}), \mathbf{z} ).
$$
It is clear that the degree $d$ portion of the polynomial ${b}(\mathbf{0}, \mathbf{y}, \mathbf{0})$
is $g_M(\mathbf{0}, \mathbf{y}, \mathbf{0})$.
Let use denote
\begin{eqnarray}
\label{big equation in decomposition1}
&&{b}(\mathbf{0}, \mathbf{y}, \mathbf{z}) - b_M(\mathbf{z})
\\
&=& \sum_{j=1}^{d-1} \ \sum_{1 \leq t_1 \leq ... \leq t_{j} \leq M}
\left( \sum_{k=0}^{d-j} \Psi^{(k)}_{t_1, ..., t_{j}} ( \mathbf{z} )  \right) y_{t_1} ... y_{t_{j}}
+ \left( \sum_{k=1}^{d} \Psi^{(k)}_{\emptyset} ( \mathbf{z} ) \right)  + g_M(\mathbf{0}, \mathbf{y}, \mathbf{0}),
\notag
\end{eqnarray}
where $\Psi^{(k)}_{t_1, ..., t_{j}} ( \mathbf{z} )$ and
$\Psi^{(k)}_{\emptyset} ( \mathbf{z} )$ are forms of degree $k$.
With these notations, we have the following decomposition,
\begin{eqnarray}
\label{decomp of b 1}
&&{b}(\mathbf{w}, \mathbf{y}, \mathbf{z})
\\
&=&
{b}(\mathbf{w}, \mathbf{0}, \mathbf{0})
+ \sum_{j=1}^{d-1}  \ \sum_{1 \leq i_1 \leq ... \leq i_j \leq K}
\left( \sum_{k=1}^{d-j} \Phi^{(k)}_{i_1, ..., i_{j}} ( \mathbf{y}, \mathbf{z} )  \right) w_{i_1} ... w_{i_j}
\notag
\\
&+&
\sum_{j=1}^{d-1} \ \sum_{1 \leq t_1 \leq ... \leq t_{j} \leq M}
\left( \sum_{k=0}^{d-j} \Psi^{(k)}_{t_1, ..., t_{j}} ( \mathbf{z} )  \right) y_{t_1} ... y_{t_{j}}
+ \left( \sum_{k=1}^{d} \Psi^{(k)}_{\emptyset} ( \mathbf{z} ) \right)  + g_M(\mathbf{0}, \mathbf{y}, \mathbf{0})
\notag
\\
&+& b_M(\mathbf{z}) - b(\mathbf{0}, \mathbf{0},\mathbf{0}),
\notag
\end{eqnarray}
which we describe below.
We note that $\Phi^{(k)}_{i_1, ..., i_{j}} ( \mathbf{y}, \mathbf{z} )$
are forms of degree $k$.
The above decomposition establishes the following. The term
$$
{b}(\mathbf{w}, \mathbf{0}, \mathbf{0})
+ \sum_{j=1}^{d-1}  \ \sum_{1 \leq i_1 \leq ... \leq i_j \leq K}
\left( \sum_{k=1}^{d-j} \Phi^{(k)}_{i_1, ..., i_{j}} ( \mathbf{y}, \mathbf{z} )  \right) w_{i_1} ... w_{i_j}
$$
consists of all the monomials of $b(\mathbf{x})$, which involve any variables of $\mathbf{w}$.
Consequently, we have
\begin{eqnarray}
{b}(\mathbf{0}, \mathbf{y}, \mathbf{z})
&=&
\sum_{j=1}^{d-1} \ \sum_{1 \leq t_1 \leq ... \leq t_{j} \leq M}
\left( \sum_{k=0}^{d-j} \Psi^{(k)}_{t_1, ..., t_{j}} ( \mathbf{z} )  \right) y_{t_1} ... y_{t_{j}}
+ \left( \sum_{k=1}^{d} \Psi^{(k)}_{\emptyset} ( \mathbf{z} ) \right) + g_M( \mathbf{0}, \mathbf{y}, \mathbf{0} )
\notag
\\
&+& b_M(\mathbf{z}),
\notag
\end{eqnarray}
and the degree $d$ portion of $b( \mathbf{0}, \mathbf{y}, \mathbf{z}) = f( \mathbf{0}, \mathbf{y}, \mathbf{z})$.
Clearly, the degree $d$ portion of $b_M(\mathbf{z})$ is
$$
f_M(\mathbf{z}) = {f} (\mathbf{0}, (- \ell_1 |_{\mathbf{w} = \mathbf{0}}, ..., -\ell_M |_{\mathbf{w} = \mathbf{0}}),  \mathbf{z}).
$$
It then follows from ~(\ref{gM is 0'}) and ~(\ref{big equation in decomposition1}) that the degree $d$ portion of
$$
\sum_{j=1}^{d-1} \ \sum_{1 \leq t_1 \leq ... \leq t_{j} \leq M}
\left( \sum_{k=0}^{d-j} \Psi^{(k)}_{t_1, ..., t_{j}} ( \mathbf{z} )  \right) y_{t_1} ... y_{t_{j}}
+ \left( \sum_{k=1}^{d} \Psi^{(k)}_{\emptyset} ( \mathbf{z} ) \right)
+ g_M( \mathbf{0}, \mathbf{y}, \mathbf{0} )
$$
is
$$
g_M( \mathbf{0}, \mathbf{y}, \mathbf{z} ) =  \sum_{j=1}^{d-1} \ \sum_{1 \leq t_1 \leq ... \leq t_{j} \leq M}
 \Psi^{(d-j)}_{t_1, ..., t_{j}} ( \mathbf{z} ) \ y_{t_1} ... y_{t_{j}}
+ \Psi^{(d)}_{\emptyset} ( \mathbf{z} )
+ g_M( \mathbf{0}, \mathbf{y}, \mathbf{0} ).
$$
We also know from ~(\ref{gM is 0'}) that $g_M( \mathbf{0}, (- \ell_1 |_{\mathbf{w} = \mathbf{0}}, ..., -\ell_M |_{\mathbf{w} = \mathbf{0}}), \mathbf{z} ) = 0$,
and consequently,
\begin{eqnarray}
\Psi^{(d)}_{\emptyset} ( \mathbf{z} )
&=&
 \left. \left( -\sum_{j=1}^{d-1} \ \sum_{1 \leq t_1 \leq ... \leq t_{j} \leq M} \Psi^{(d-j)}_{t_1, ..., t_{j}} ( \mathbf{z} ) \ y_{t_1} ... y_{t_{j}}
\right) \right|_{y_i = - \ell_i |_{\mathbf{w} = \mathbf{0} \ (1 \leq i \leq M)} }
\notag
\\
&&
\notag
\\
&-& g_M( \mathbf{0}, (- \ell_1 |_{\mathbf{w} = \mathbf{0}}, ..., -\ell_M |_{\mathbf{w} = \mathbf{0}}), \mathbf{0} ).
\notag
\end{eqnarray}
In other words, $\Psi^{(d)}_{\emptyset} ( \mathbf{z} ) $ can be expressed as a rational polynomial
in the forms $ \{ \Psi^{(d-j)}_{t_1, ..., t_{j}} ( \mathbf{z} ) :  1 \leq j \leq d-1, 1 \leq t_1 \leq ... \leq t_{j} \leq M \} \cup \{ \ell_i |_{\mathbf{w} = \mathbf{0}} : 1 \leq i \leq M \}$.

We denote by $\Phi = \{ \Phi^{(k)}_{i_1, ..., i_{j}} : 1 \leq j \leq d-1, 1 \leq i_1 \leq ... \leq i_j \leq K, 1 \leq k \leq d-j \}$.
Note every polynomial of $\Phi$ has degree strictly less than $d$, and involves only the $\mathbf{y}$ and the $\mathbf{z}$ variables.
Clearly, we have $|\Phi| \leq d^2 K^d \leq d^{d+2}$. We apply Proposition \ref{prop reg par} to the system $\Phi$ with respect
to the functions $\boldsymbol{\mathcal{F}} = \{\mathcal{F}_2, ..., \mathcal{F}_{d-1} \}$, where $\mathcal{F}_i(t) = {\rho}_{d,i} (2 + 2t) + 2t$ for $2 \leq i \leq d-1$, and obtain
$\mathcal{R}(\Phi) = ( \mathbf{a}^{(d-1)}, ..., \mathbf{a}^{(1)} )$.
For each form $a^{({s})}_i \in \mathbf{a}^{({s})} \ (1 \leq s \leq d-1, 1 \leq i \leq | \mathbf{a}^{({s})}|)$, we write
\begin{equation}
\label{defn of a's}
a^{({s})}_i(\mathbf{y}, \mathbf{z}) = \sum_{k=0}^{s} \sum_{1 \leq i_1 \leq ... \leq i_k \leq M} \widetilde{\Psi}^{(s - k)}_{{s} :i: i_1, ..., i_k}(\mathbf{z}) y_{i_1} ... y_{i_k},
\end{equation}
where each
$\widetilde{\Psi}^{(s - k)}_{{s} :i: i_1, ..., i_k}(\mathbf{z})$ is a form of degree $s - k$.
Thus each form $a^{({s})}_i$ introduces at most $({s}+1) M^{s} \leq d M^d$ forms in $\mathbf{z}$.
Also for each $1 \leq i \leq d-1$, we denote $\overline{\mathbf{a}}^{(i)}$
to be the system obtained by removing all forms that depend only on the $\mathbf{z}$ variables from $\mathbf{a}^{(i)}$.
Let $\overline{\mathcal{R}}(\Phi)= (\overline{\mathbf{a}}^{(d-1)}, ..., \overline{\mathbf{a}}^{(1)} )$, $R_2 = \sum_{i = 1}^{d-1} \ | \overline{\mathbf{a}}^{(i)} |$, and $D_2 = \sum_{i = 1}^{d-1} i \ | \overline{\mathbf{a}}^{(i)} |$. By relabeling if necessary, we denote the
elements of $\overline{\mathbf{a}}^{(s)}$ by $\overline{\mathbf{a}}^{(s)} = \{ a^{(s)}_i : 1 \leq i \leq | \overline{\mathbf{a}}^{(s)} | \}$
for each $1 \leq s \leq d-1$.

Let
\begin{eqnarray}
\Psi &=& \{  \Psi^{(k)}_{t_1, ..., t_{j}} (\mathbf{z}): 1 \leq j \leq d-1, 1 \leq t_1 \leq ... \leq t_{j} \leq M ,   0 \leq k \leq d-j  \}
\notag
\\
&\cup&  \  \{ {\Psi}^{(k)}_{\emptyset}(\mathbf{z}) : 1 \leq k < d \}
\notag
\\
&\cup&
\notag
\{ \ell_i |_{\mathbf{w} = \mathbf{0}} : 1 \leq i \leq M \}
\\
&\cup&  \  \{ \widetilde{\Psi}^{(s - k)}_{{s} :i: i_1, ..., i_k}(\mathbf{z}) :1 \leq {s} \leq d-1, 1 \leq i \leq |\mathbf{a}^{(s)}|, 1 \leq k \leq {s}, 1 \leq i_1 \leq ... \leq i_k \leq M  \}.
\notag
\end{eqnarray}
In other words, $\Psi$ is the collection of $\ell_i |_{\mathbf{w} = \mathbf{0}}$, and all $ \Psi^{(k)}_{t_1, ..., t_{j}} (\mathbf{z})$,
$\widetilde{\Psi}^{(s - k)}_{{s} :i: i_1, ..., i_k}(\mathbf{z})$, and
${\Psi}^{(k)}_{\emptyset}(\mathbf{z})$ except ${\Psi}^{(d)}_{\emptyset}(\mathbf{z})$. In particular, every polynomial of $\Psi$ has degree strictly less than $d$.
We can see that
$$
|\Psi| \leq d^2 M^d + d + M + |\mathcal{R}(\Phi) | d M^d.
$$
We let $\mathcal{R}(\Psi)$ be a regularization of
$\Psi$ with respect
to the functions $\boldsymbol{\mathcal{F}} = \{\mathcal{F}_2, ..., \mathcal{F}_d\}$, where again $\mathcal{F}_i(t) = {\rho}_{d,i} (2 + 2 t) + 2t$ for $2 \leq i \leq d-1$.
Let us denote $\mathcal{R}(\Psi) = (\mathbf{v}^{(d-1)}, ..., \mathbf{v}^{(1)} )$, $R_1 = \sum_{i = 1}^{d-1} \ | \mathbf{v}^{(i)} |$, and $D_1 = \sum_{i = 1}^{d-1} i \ | \mathbf{v}^{(i)} |$.

Let $\mathcal{R}^{(i)}(\Phi)$, $\Phi^{(i)}$, and $\mathcal{R}^{(i)}(\Psi)$ denote the degree $i$ forms of $\mathcal{R}^{}(\Phi)$, $\Phi$, and $\mathcal{R}^{}(\Psi)$, respectively.
From Proposition \ref{prop reg par}, we know that each $|\mathcal{R}^{(i)}(\Phi)| = |\mathbf{a}^{(i)}| \ (1 \leq i \leq d-1)$, and consequently $R_2$, is bounded by some constant dependent only on $\boldsymbol{\mathcal{F}}$, and $|\Phi^{(d-1)}|$, ..., $|\Phi^{(1)}|$. Thus we see that $R_2$ is bounded by a constant dependent only
on $d$.
We set
$$
M =  \rho_{d,d}( 2 + 2 R_2) + 2 R_2  + d,
$$
and note that $M$ is bounded by a constant dependent only on $d$.
By Proposition \ref{prop reg par} again,
we have that each $|\mathcal{R}^{(i)}(\Psi)| = |\mathbf{v}^{(i)}| \ (1 \leq i \leq d-1)$, and consequently $R_1$, is bounded by some constant dependent only
on $d$, $\boldsymbol{\mathcal{F}}$, $M$, and $|\Phi^{(d-1)}|$, ..., $|\Phi^{(1)}|$.
Thus $R_1$ is bounded by a constant dependent only on $d$ as well.

We define
\begin{equation}
\label{def of Ad}
A_d :=  \max\{ 2 \rho_{d,d}( 2 + 2 R_1) + 4 R_1 + 2d , \ 2 \rho_{d,d}( 2 + 2 R_2) + 4 R_2 + 2d , \ \frac{5 \cdot 2^{d-1} \cdot (d-1) \cdot d!}{(\log 2)^d} + 5d\},
\end{equation}
and suppose $h^{\star}(f) \geq A_d$. We note that the third term inside the maximum function above is not required in this section, but
this lower bound on $A_d$ becomes necessary in Section \ref{section major arcs}.
With this choice of $A_d$, we have from ~(\ref{h eqn 1}) and ~(\ref{h eqn 2}) that
\begin{equation}
\label{cond on h and M}
h( f_M( \mathbf{z} ) ) \geq h - M - d \geq \rho_{d,d}( 2 + 2 R_1) + 2 R_1,
\end{equation}
and
\begin{equation}
\label{cond on h and M'}
h( g_M(\mathbf{0}, \mathbf{y}, \mathbf{z} ) ) \geq M - d \geq  \rho_{d,d}( 2 + 2 R_2) + 2 R_2.
\end{equation}

For each $\mathbf{H} \in \mathbb{Z}^{R_1}$, we define the following set
$$
Z(\mathbf{H}) = \{ \mathbf{z} \in [0, N]^{n-M-K} \cap \mathbb{Z}^{n -M -K} : \mathcal{R}(\Psi) (\mathbf{z}) = \mathbf{H} \}.
$$
By Proposition \ref{prop reg par}, we know that each of the polynomials $\Psi^{(k)}_{t_1, ..., t_{j}} (\mathbf{z})$ and ${\Psi}^{(k)}_{\emptyset}(\mathbf{z})$ in ~(\ref{decomp of b 1}) can be expressed as a rational polynomial in the forms of $\mathcal{R}(\Psi)$. Let us denote
$$
\Psi^{(k)}_{t_1, ..., t_{j}} (\mathbf{z}) = {\hat{c} }^{(k)}_{ t_1, ..., t_{j}} ( \mathcal{R}(\Psi) ) \ \  \mbox{  and  } \  \
{\Psi}^{(k)}_{\emptyset} (\mathbf{z}) =  {\hat{c}}^{(k)}_{\emptyset} ( \mathcal{R}(\Psi) ),
$$
where ${\hat{c} }^{(k)}_{ t_1, ..., t_{j}} $ and  ${\hat{c}}^{(k)}_{\emptyset}$ are rational polynomials in $R_1$ variables.
Therefore, for any $\mathbf{z}_0 \in Z(\mathbf{H})$, we have
$$
\Psi^{(k)}_{t_1, ..., t_{j}}(\mathbf{z}_0) = {\hat{c} }^{(k)}_{ t_1, ..., t_{j}} ( \mathbf{H} ) \ \  \mbox{  and  } \  \
{\Psi}^{(k)}_{\emptyset}(\mathbf{z}_0) =  {\hat{c}}^{(k)}_{\emptyset} ( \mathbf{H} ).
$$

Since each of the forms $\widetilde{\Psi}^{(s - k)}_{{s} :i: i_1, ..., i_k}(\mathbf{z})$ in ~(\ref{defn of a's}) can be expressed as a rational polynomial
in the forms of $\mathcal{R}(\Psi)$, let us denote
$$
\widetilde{\Psi}^{(s - k)}_{{s} :i: i_1, ..., i_k}(\mathbf{z}) = \widetilde{c}^{(s - k)}_{{s} :i: i_1, ..., i_k}(\mathcal{R}(\Psi)),
$$
where each $\widetilde{c}^{(s - k)}_{{s} :i: i_1, ..., i_k}$ is a rational polynomial in $R_1$ variables.
Therefore, for each $a^{({s})}_i \in \mathcal{R}(\Phi) = ( \mathbf{a}^{(d-1)}, ..., \mathbf{a}^{(1)} )$,
where $1 \leq s \leq d-1$ and $1 \leq i \leq |\mathbf{a}^{(s)}|$,
we can write
\begin{equation}
\label{def of a's 2}
a^{({s})}_i(\mathbf{y}, \mathbf{z}) = \sum_{k=0}^{s} \sum_{1 \leq i_1 \leq ... \leq i_k \leq M} \widetilde{c}^{(s - k)}_{{s} :i: i_1, ..., i_k}(\mathcal{R}(\Psi)) y_{i_1} ... y_{i_k}.
\end{equation}
Consequently, we can define the following polynomial for each $1 \leq s \leq d-1$ and $1 \leq i \leq |\mathbf{a}^{(s)}|$,
\begin{equation}
\label{def of a's 3}
a^{({s})}_i(\mathbf{y}, Z(\mathbf{H})  ) = \sum_{k=0}^{s} \sum_{1 \leq i_1 \leq ... \leq i_k \leq M} \widetilde{c}^{(s - k)}_{{s} :i: i_1, ..., i_k}(\mathbf{H} ) y_{i_1} ... y_{i_k},
\end{equation}
so that given any $\mathbf{z}_0 \in Z(H)$, we have
$$
a^{({s})}_i(\mathbf{y}, \mathbf{z}_0  ) = a^{({s})}_i(\mathbf{y}, Z(\mathbf{H}) ).
$$
We also define
\begin{eqnarray}
\overline{R}(\Phi)(\mathbf{y}, Z(\mathbf{H}) ) = \{ a^{({s})}_i(\mathbf{y}, Z(\mathbf{H})  ) :
1 \leq s \leq d-1, 1 \leq i \leq |\mathbf{a}^{(s)}|,  \text{ and }  a^{({s})}_i(\mathbf{y}, \mathbf{z})  \in \overline{R}(\Phi)(\mathbf{y},\mathbf{z} )
\}.
\notag
\end{eqnarray}
For each $\mathbf{G} \in \mathbb{Z}^{R_2}$, we let
$$
Y(\mathbf{G};\mathbf{H}) = \{ \mathbf{y} \in [0, N]^{ M } \cap \mathbb{Z}^{M} : \overline{\mathcal{R}}(\Phi) (\mathbf{y}, Z(\mathbf{H}) ) = \mathbf{G} \}.
$$

Recall $\Phi$ is the collection of all $\Phi^{(k)}_{i_1, ..., i_{j}} ( \mathbf{y}, \mathbf{z} )$ in ~(\ref{decomp of b 1}), and
that each $\Phi^{(k)}_{i_1, ..., i_{j}} ( \mathbf{y}, \mathbf{z} )$ can be expressed as
a rational polynomial in the forms of $\mathcal{R}(\Phi)$.
Thus, it follows from this fact and ~(\ref{def of a's 3}) that each $\Phi^{(k)}_{i_1, ..., i_{j}} ( \mathbf{y}, \mathbf{z} )$
is constant on $(\mathbf{y}, \mathbf{z}) \in Y(\mathbf{G};\mathbf{H}) \times Z(\mathbf{H})$, and we denote this constant value by
${c}^{(k)}_{i_1, ..., i_{j}} ( \mathbf{G}, \mathbf{H} )$.
Therefore, for any choice of $\mathbf{z} \in Z(\mathbf{H})$ and $ \mathbf{y} \in  Y(\mathbf{G};\mathbf{H})$, the polynomial ${b}(\mathbf{x})$ takes the following
shape
\begin{eqnarray}
&&{b}( \mathbf{w}, \mathbf{y}, \mathbf{z} )
\label{decom of b with coeff}
\\
&=&
{b}( \mathbf{w}, \mathbf{0}, \mathbf{0})
+ \sum_{j=1}^{d-1}  \ \sum_{1 \leq i_1 \leq ... \leq i_j \leq K}
\left( \sum_{k=1}^{d-j} {c}^{(k)}_{i_1, ..., i_{j}} ( \mathbf{G}, \mathbf{H} )  \right) w_{i_1} ... w_{i_j}
\notag
\\
&+&
\sum_{j=1}^{d-1} \ \sum_{1 \leq t_1 \leq ... \leq t_{j} \leq M}
\left( \sum_{k=0}^{d-j} {\hat{c} }^{(k)}_{ t_1, ..., t_{j}} ( \mathbf{H} )  \right) y_{t_1} ... y_{t_{j}}
+
\left( \sum_{k=1}^{d} {\hat{c}}^{(k)}_{\emptyset} ( \mathbf{H} )  \right)
+ g_M( \mathbf{0}, \mathbf{y}, \mathbf{0} )
\notag
\\
&+& b_M(\mathbf{z}) - b(\mathbf{0}, \mathbf{0}, \mathbf{0}).
\notag
\end{eqnarray}

We label
\begin{eqnarray}
\mathfrak{C}_0 (\mathbf{w}, \mathbf{G}, \mathbf{H}  ) &=& {b}(\mathbf{w}, \mathbf{0}, \mathbf{0} )
+ \sum_{j=1}^{d-1}  \ \sum_{1 \leq i_1 \leq ... \leq i_j \leq K}
\left( \sum_{k=1}^{d-j} {c}^{(k)}_{i_1, ..., i_{j}} ( \mathbf{G}, \mathbf{H} )  \right) w_{i_1} ... w_{i_j},
\notag
\end{eqnarray}
and
$$
\mathfrak{C}_1 (\mathbf{y}, \mathbf{H} ) = \sum_{j=1}^{d-1} \ \sum_{1 \leq t_1 \leq ... \leq t_{j} \leq M}
\left( \sum_{k=0}^{d-j} {\hat{c}}^{(k)}_{ t_1, ..., t_{j}} ( \mathbf{H} )  \right) y_{t_1} ... y_{t_{j}}
+
\left( \sum_{k=1}^{d} {\hat{c}}^{(k)}_{\emptyset} ( \mathbf{H} )  \right)
+ g_M( \mathbf{0}, \mathbf{y}, \mathbf{0} ),
$$
so that for $\mathbf{z} \in Z(\mathbf{H})$ and $ \mathbf{y} \in  Y(\mathbf{G};\mathbf{H})$, we have
$$
{b}(\mathbf{w}, \mathbf{y}, \mathbf{z}  ) =  \mathfrak{C}_0 (\mathbf{w}, \mathbf{G}, \mathbf{H}  ) + \mathfrak{C}_1 (\mathbf{y}, \mathbf{H} ) +
b_M(\mathbf{z}) - b(\mathbf{0}, \mathbf{0}, \mathbf{0}).
$$

We define the following three exponential sums,
$$
S_0( {\alpha}, \mathbf{G}, \mathbf{H} ) = \sum_{ \mathbf{w} \in [0,N]^K \cap \mathbb{Z}^K } \Lambda (\mathbf{w})
\ e(  {\alpha} \cdot \mathfrak{C}_0 (\mathbf{w}, \mathbf{G}, \mathbf{H}  )  ),
$$
$$
S_1( {\alpha}, \mathbf{G}, \mathbf{H}) = \sum_{\mathbf{y} \in Y(\mathbf{G};\mathbf{H})} \Lambda (\mathbf{y}) \  e( {\alpha} \cdot \mathfrak{C}_1 (\mathbf{y}, \mathbf{H}  )  ),
$$
and
$$
S_2( {\alpha}, \mathbf{H} ) = \sum_{\mathbf{z} \in Z(\mathbf{H}) } \Lambda (\mathbf{z}) \ e(  {\alpha} \cdot   b_M(\mathbf{z}) - {\alpha} \cdot b(\mathbf{0}, \mathbf{0}, \mathbf{0}) ).
$$

Let
\begin{eqnarray}
\mathcal{L}_1(N) 
= \{ \mathbf{H} \in \mathbb{Z}^{R_1} : Z(\mathbf{H}) \not = \emptyset \},
\notag
\end{eqnarray}
and for each $\mathbf{H} \in \mathcal{L}_1(N)$,  let
\begin{eqnarray}
\mathcal{L}_2(N ; \mathbf{H}) 
= \{ \mathbf{G} \in \mathbb{Z}^{R_2} : Y(\mathbf{G}, \mathbf{H} ) \not = \emptyset \}.
\notag
\end{eqnarray}
It is clear that
$$
|\mathcal{L}_1(N)| \ll N^{D_1} \  \mbox{ and } \  |\mathcal{L}_2(N)| \ll N^{D_2}.
$$
Therefore, we obtain by applying the Cauchy-Schwartz inequality
\begin{eqnarray}
\label{minor arc ineq 1}
&& \Big{|} \int_{\mathfrak{m}(C)} T({b}; {\alpha} ) \ {d} {\alpha}  \Big{|}^2
\\
&\leq&
\Big{|}
\sum_{\mathbf{H} \in \mathcal{L}_1(N)} \sum_{\mathbf{G} \in \mathcal{L}_2(N ; \mathbf{H})} \int_{\mathfrak{m}(C)} \
\sum_{\substack{ \mathbf{w} \in [0,N]^K \cap \mathbb{Z}^K \\ \mathbf{z} \in Z(\mathbf{H}) \\  \mathbf{y} \in Y(\mathbf{G}; \mathbf{H}) }}
\Lambda(\mathbf{w}) \Lambda(\mathbf{y}) \Lambda(\mathbf{z}) \cdot
\notag
\\
&&\phantom{12345612345}
e( {\alpha} \cdot (\mathfrak{C}_0 (\mathbf{w}, \mathbf{G}, \mathbf{H}  ) + \mathfrak{C}_1 (\mathbf{y}, \mathbf{H} ) + {b}_M( \mathbf{z} ) - b(\mathbf{0}, \mathbf{0}, \mathbf{0})  \  ))
\  {d} {\alpha}  \Big{|}^2
\notag
\\
&\ll&
N^{D_1+ D_2}
\sum_{\mathbf{H} \in \mathcal{L}_1(N)} \sum_{\mathbf{G} \in \mathcal{L}_2(N ; \mathbf{H})} \int_{\mathfrak{m}(C)}
\Big{|}   S_0 ( {\alpha}, \mathbf{G}, \mathbf{H} )
S_1( {\alpha}, \mathbf{G}, \mathbf{H} )S_2( {\alpha}, \mathbf{H} )
\Big{|}^2 \  {d}  {\alpha}
\notag
\\
&\ll&
N^{D_1+D_2} \  \left( \sup_{\substack {\mathbf{H} \in \mathcal{L}_1(N) \\ \mathbf{G} \in \mathcal{L}_2(N ; \mathbf{H}) }}  \sup_{\alpha \in \mathfrak{m}(C) } | S_0 (\alpha, \mathbf{G}, \mathbf{H} ) |^2 \right) \cdot
\notag
\\
&&\phantom{1223344556677889123456}  \sum_{\mathbf{H} \in \mathcal{L}_1(N)} \sum_{\mathbf{G} \in \mathcal{L}_2(N ; \mathbf{H})}   \|S_1(\cdot, \mathbf{G}, \mathbf{H} ) \|_2^2 \ \| S_2(\cdot, \mathbf{H} ) \|_2^2,
\notag
\end{eqnarray}
where $\| \cdot \|_2$ denotes the $L^2$-norm on $[0,1]$.
By the orthogonality relation, it follows that
\begin{eqnarray}
\notag
\|S_1(\cdot, \mathbf{G}, \mathbf{H} ) \|_2^2 \ \| S_2(\cdot, \mathbf{H} ) \|_2^2
&\leq&
(\log N)^{2n - 2K} \mathcal{N}_1 (\mathbf{G}; \mathbf{H}) \mathcal{N}_2 (\mathbf{H}),
\notag
\end{eqnarray}
where
$$\mathcal{N}_1 (\mathbf{G}; \mathbf{H}) =  | \{ (\mathbf{y}, \mathbf{y}' )  \in Y(\mathbf{G}; \mathbf{H}) \times Y(\mathbf{G}; \mathbf{H}): \mathfrak{C}_1 (\mathbf{y}, \mathbf{H}  ) = \mathfrak{C}_1 (\mathbf{y}', \mathbf{H}  )   \} |,$$
and
$$
\mathcal{N}_2 (\mathbf{H}) =  |\{ (\mathbf{z}, \mathbf{z}' )  \in Z(\mathbf{H}) \times Z(\mathbf{H}) : {b}_M( \mathbf{z} ) = {b}_M( \mathbf{z}' ) \}|.
$$

With these notations, we may further bound ~(\ref{minor arc ineq 1}) as follows
\begin{eqnarray}
\label{minor arc ineq 2}
&&\Big{|} \int_{\mathfrak{m}(C)} T({b}; {\alpha} ) \ {d} {\alpha}  \Big{|}^2
\\
&\ll&
(\log N)^{2n - 2K}  N^{D_1+D_2} \ \left( \sup_{\substack {\mathbf{H} \in \mathcal{L}_1(N) \\ \mathbf{G} \in \mathcal{L}_2(N ; \mathbf{H}) }}  \sup_{\alpha \in \mathfrak{m}(C) } | S_0 (\alpha, \mathbf{G}, \mathbf{H} ) |^2 \right)
\ \mathcal{W},
\notag
\end{eqnarray}
where
$$
\mathcal{W} = \sum_{\mathbf{H} \in \mathcal{L}_1(N)} \sum_{\mathbf{G} \in \mathcal{L}_2(N; \mathbf{H})} \mathcal{N}_1 (\mathbf{G}; \mathbf{H})  \mathcal{N}_2 (\mathbf{H}).
$$
We can express $\mathcal{W}$ as the number of solutions $\mathbf{y}, \mathbf{y}'  \in [0,N]^M \cap \mathbb{Z}^M$
and $\mathbf{z}, \mathbf{z}' \in [0,N]^{n-M-K} \cap \mathbb{Z}^{n-M-K} $ of the system
\begin{eqnarray}
{\mathcal{R}}(\Psi) ( \mathbf{z}) &=& {\mathcal{R}}(\Psi) (\mathbf{z}') = \mathbf{H}
\label{first system}
\\
\overline{\mathcal{R}}(\Phi) (\mathbf{y},  Z(\mathbf{H})) &=& \overline{\mathcal{R}}(\Phi) (\mathbf{y}', Z(\mathbf{H}))= \mathbf{G}
\notag
\\
\mathfrak{C}_1 (\mathbf{y}, \mathbf{H}  )  &=&  \mathfrak{C}_1 (\mathbf{y}', \mathbf{H}  )
\notag
\\
{b}_M( \mathbf{z} ) &=& {b}_M( \mathbf{z}' )
\notag
\end{eqnarray}
for any $\mathbf{H}\in \mathcal{L}_1(N)$ and $\mathbf{G}\in \mathcal{L}_2(N; \mathbf{H})$.
We know that the system $\overline{\mathcal{R}}(\Phi) (\mathbf{y},  Z(\mathbf{H}))$ is identical to $\overline{\mathcal{R}}(\Phi) (\mathbf{y},  \mathbf{z}_0)$ for any choice of $\mathbf{z}_0 \in Z(\mathbf{H})$ and any $\mathbf{y} \in [0,N]^M \cap \mathbb{Z}^M$. Similarly, we know that the polynomial
$\mathfrak{C}_1 (\mathbf{y}, \mathbf{H})$ is identical to $b(\mathbf{0}, \mathbf{y}, \mathbf{z}_0) - b_M(\mathbf{z}_0)$ for
any choice of $\mathbf{z}_0 \in Z(\mathbf{H})$.
Therefore, since ${\mathcal{R}}(\Psi) (\mathbf{z}) = \mathbf{H}$
implies that $z \in Z(\mathbf{H})$, we can rearrange the system ~(\ref{first system}) and
deduce that $\mathcal{W}$ is the number of solutions $\mathbf{y}, \mathbf{y}'  \in [0,N]^M \cap \mathbb{Z}^M$
and $\mathbf{z}, \mathbf{z}' \in [0,N]^{n-M-K} \cap \mathbb{Z}^{n-M-K} $ of the following system
\begin{eqnarray}
\label{second system}
{\mathcal{R}}(\Psi) ( \mathbf{z}) &=& {\mathcal{R}}(\Psi) (\mathbf{z}')
\\
\overline{\mathcal{R}}(\Phi) (\mathbf{y},  \mathbf{z} ) &=& \overline{\mathcal{R}}(\Phi) (\mathbf{y}', \mathbf{z} )
\notag
\\
{b}(\mathbf{0}, \mathbf{y}, \mathbf{z}) - {b}_M (\mathbf{z} )  &=&  {b}(\mathbf{0}, \mathbf{y}', \mathbf{z}) - {b}_M (\mathbf{z} )
\notag
\\
{b}_M( \mathbf{z} ) &=& {b}_M( \mathbf{z}' ).
\notag
\end{eqnarray}

Our result follows from the following two claims.

Claim 1: Given any $c>0$, there exists $C > 0$ such that the following bound holds,
$$
\sup_{\substack {\mathbf{H} \in \mathcal{L}_1(N) \\ \mathbf{G} \in \mathcal{L}_2(N ; \mathbf{H}) }}  \sup_{\alpha \in \mathfrak{m}(C) } | S_0 (\alpha, \mathbf{G}, \mathbf{H} ) |
\ll \frac{N^K}{(\log N)^c}.
$$

Claim 2: We have the following bound on $\mathcal{W}$,
$$
\mathcal{W} \ll N^{2n - 2K - 2 d - D_1 - D_2}.
$$

By substituting the bounds from the above two claims into ~(\ref{minor arc ineq 2}), we obtain for any $c>0$
there exists $C>0$ such that
$$
\int_{\mathfrak{m}(C)} T({b}; {\alpha} ) \ {d} {\alpha}
\ll
\frac{N^{n - d} }{(\log N)^{c}},
$$
and this completes the proof of our proposition.
Therefore, we only need to establish Claims $1$ and $2$.
Claim $1$ is obtained via Weyl differencing. Since the set up for our Claim 1 is the same as that
of \cite{CM}, we omit the proof of Claim 1 and refer the reader to \cite[pp. 725]{CM}.

We now present the proof of Claim 2. From ~(\ref{second system}), we can write
$$
\mathcal{W} = \sum_{\mathbf{z} \in  [0,N]^{n-M-K} \cap \mathbb{Z}^{n-M-K}} T_1(\mathbf{z}) \cdot T_2(\mathbf{z}),
$$
where $T_1(\mathbf{z})$ is the number of solutions $\mathbf{y}, \mathbf{y}' \in [0,N]^{M} \cap \mathbb{Z}^{M}$
to the system
\begin{eqnarray}
{b}( \mathbf{0}, \mathbf{y}, \mathbf{z} ) &=&  {b}( \mathbf{0}, \mathbf{y}', \mathbf{z} )
\notag
\\
\overline{\mathcal{R}}(\Phi) (\mathbf{y}, \mathbf{z}) &=& \overline{\mathcal{R}}(\Phi) (\mathbf{y}', \mathbf{z}),
\notag
\end{eqnarray}
and $T_2(\mathbf{z})$ is the number of solutions $\mathbf{z}' \in  [0,N]^{n-M-K} \cap \mathbb{Z}^{n-M-K}$
to the system
\begin{eqnarray}
{b}_M( \mathbf{z} ) &=& {b}_M( \mathbf{z}' )
\notag
\\
{\mathcal{R}}(\Psi) ( \mathbf{z}) &=& {\mathcal{R}}(\Psi) (\mathbf{z}').
\notag
\end{eqnarray}
Define $\mathcal{W}_i :=\sum_{\mathbf{z}} T_i (\mathbf{z})^2 \ (i=1, 2)$ so that
we have $\mathcal{W}^2 \leq \mathcal{W}_1 \mathcal{W}_2$ by the Cauchy-Schwartz inequality.
We first estimate $\mathcal{W}_1$, which
we can deduce to be the number of solutions $\mathbf{y}, \mathbf{y}', \mathbf{u}, \mathbf{u}' \in [0,N]^M \cap \mathbb{Z}^M$
and $\mathbf{z} \in [0,N]^{n-M-K} \cap \mathbb{Z}^{n-M-K}$ satisfying the equations
\begin{eqnarray}
\label{system 1 in claim 2}
{b}( \mathbf{0}, \mathbf{y}, \mathbf{z} )  -  {b}( \mathbf{0}, \mathbf{y}', \mathbf{z} ) &=& 0
\\
{b}( \mathbf{0}, \mathbf{u}, \mathbf{z} ) - {b}( \mathbf{0}, \mathbf{u}', \mathbf{z} )  &=&  0
\notag
\\
\overline{\mathcal{R}}(\Phi) (\mathbf{y}, \mathbf{z}) - \overline{\mathcal{R}}(\Phi) (\mathbf{y}', \mathbf{z}) &=&  0
\notag
\\
\overline{\mathcal{R}}(\Phi) (\mathbf{u}, \mathbf{z}) - \overline{\mathcal{R}}(\Phi) (\mathbf{u}', \mathbf{z}) &=&  0.
\notag
\end{eqnarray}
We consider the $h$-invariant of the system of forms on the left hand side of ~(\ref{system 1 in claim 2}), and show that
it is a regular system. The first two equations of ~(\ref{system 1 in claim 2}) are the degree $d$ polynomials of the system, and let $h_d$ be the $h$-invariant of these two polynomials.
Suppose for some $\lambda, \mu \in \mathbb{Q}$, not both $0$, we have
\begin{eqnarray}
\lambda \cdot ({f}( \mathbf{0}, \mathbf{y}, \mathbf{z} )  -  {f}( \mathbf{0}, \mathbf{y}', \mathbf{z} ) )
+ \mu \cdot ({f}( \mathbf{0}, \mathbf{u}, \mathbf{z} ) - {f}( \mathbf{0}, \mathbf{u}', \mathbf{z} ) )
= \sum_{j=1}^{h_d} U_j \cdot V_j,
\notag
\end{eqnarray}
where $U_j = U_j( \mathbf{y}, \mathbf{y}', \mathbf{u}, \mathbf{u}', \mathbf{z} )$ and $V_j =
V_j( \mathbf{y}, \mathbf{y}', \mathbf{u}, \mathbf{u}', \mathbf{z} )$ are rational forms of positive degree $(1 \leq j \leq h_d)$.
Without loss of generality, suppose $\lambda \not = 0$.
Let $\boldsymbol{\ell} = (- \ell_1 |_{\mathbf{w} = \mathbf{0}}, ..., - \ell_M |_{\mathbf{w} = \mathbf{0}} )$. If we set
$\mathbf{u} = \mathbf{u}' = \mathbf{y}' = \boldsymbol{\ell}$,
then the above equation becomes
\begin{eqnarray}
{f}( \mathbf{0}, \mathbf{y}, \mathbf{z} )  -  {f}_M( \mathbf{z} )
= \frac{1}{\lambda} \sum_{j=1}^{h_d}  U_j (\mathbf{y}, \boldsymbol{\ell},\boldsymbol{\ell}, \boldsymbol{\ell}, \mathbf{z}) \cdot V_j (\mathbf{y}, \boldsymbol{\ell}, \boldsymbol{\ell},\boldsymbol{\ell}, \mathbf{z}).
\notag
\end{eqnarray}
Therefore, we obtain from ~(\ref{cond on h and M'})
\begin{eqnarray}
h_d &\geq& h ({f}( \mathbf{0}, \mathbf{y}, \mathbf{z} )  -  {f}_M( \mathbf{z} ) )
\notag
\\
&=& h ({g_M}( \mathbf{0}, \mathbf{y}, \mathbf{z} ) )
\notag
\\
&\geq& \rho_{d,d}( 2 + 2 R_2) + 2 R_2
\notag
\\
&\geq& \rho_{d,d}( 2 + 2 R_2 - 2 | { \overline{\mathbf{a} }}^{(1)}|) + 2 | { \overline{\mathbf{a} }}^{(1)}|.
\notag
\end{eqnarray}

For each $1 \leq i \leq d-1$, denote by
$$
\overline{\mathcal{R}}(\Phi)^{(i)} (\mathbf{y}, \mathbf{z}) - \overline{\mathcal{R}}(\Phi)^{(i)} (\mathbf{y}', \mathbf{z})
=
\{ {a}_j^{(i)}(\mathbf{y}, \mathbf{z}) - {a}_j^{(i)}(\mathbf{y}', \mathbf{z}) : 1 \leq j \leq |\overline{\mathbf{a}}^{(i)}| \},
$$
the system of degree $i$ polynomials of $\overline{\mathcal{R}}(\Phi) (\mathbf{y}, \mathbf{z}) - \overline{\mathcal{R}}(\Phi) (\mathbf{y}', \mathbf{z})$.
We also define
$$
\overline{\mathcal{R}}(\Phi)^{(i)} (\mathbf{u}, \mathbf{z}) - \overline{\mathcal{R}}(\Phi)^{(i)} (\mathbf{u}', \mathbf{z})
$$
in a similar manner. We apply Lemma \ref{Lemma 2 in CM} to estimate the $h$-invariant of the degree $i$ forms of the system ~(\ref{system 1 in claim 2}) for each $2 \leq i \leq d-1$,
\begin{eqnarray}
&&
h \left( \overline{\mathcal{R}}(\Phi)^{(i)} (\mathbf{y}, \mathbf{z}) - \overline{\mathcal{R}}(\Phi)^{(i)} (\mathbf{y}', \mathbf{z}),  \overline{\mathcal{R}}(\Phi)^{(i)} (\mathbf{u}, \mathbf{z}) - \overline{\mathcal{R}}(\Phi)^{(i)} (\mathbf{u}', \mathbf{z} ) \right)
\notag
\\
&\geq &h \left( \overline{\mathcal{R}}(\Phi)^{(i)} (\mathbf{y}, \mathbf{z}) - \overline{\mathcal{R}}(\Phi)^{(i)} (\mathbf{y}', \mathbf{z}),  \overline{\mathcal{R}}(\Phi)^{(i)} (\mathbf{u}, \mathbf{z}) - \overline{\mathcal{R}}(\Phi)^{(i)} (\mathbf{u}', \mathbf{z} ) ; \mathbf{z} \right)
\notag
\\
&=&
h \left( \overline{\mathcal{R}}(\Phi)^{(i)} (\mathbf{y}, \mathbf{z}) - \overline{\mathcal{R}}(\Phi)^{(i)} (\mathbf{y}', \mathbf{z}) ; \mathbf{z}  \right)
\notag
\\
&\geq&
h \left( \overline{\mathcal{R}}(\Phi)^{(i)} (\mathbf{y}, \mathbf{z}) , \overline{\mathcal{R}}(\Phi)^{(i)} (\mathbf{y}', \mathbf{z}) ; \mathbf{z}  \right)
\notag
\\
&\geq&
h \left( \overline{\mathcal{R}}(\Phi)^{(i)} (\mathbf{y}, \mathbf{z}) ; \mathbf{z}  \right)
\notag
\\
&\geq&
\rho_{d,i}( 2 + 2 R_2 ) + 2 R_2
\notag
\\
&\geq&
\rho_{d,i}( 2 + 2 R_2 - 2| \overline{{ \mathbf{a} }}^{(1)}| ) + 2 | \overline{{ \mathbf{a} }}^{(1)}|.
\notag
\end{eqnarray}

We also have to show that the linear forms of the system ~(\ref{system 1 in claim 2}) are linearly independent over $\mathbb{Q}$.
Recall the linear forms of $\overline{\mathcal{R}}(\Phi)^{(1)} (\mathbf{y}, \mathbf{z})$ are linearly independent over $\mathbb{Q}$,
and do not include any linear forms that depend only on the $\mathbf{z}$ variables, and similarly for $\overline{\mathcal{R}}(\Phi)^{(1)} (\mathbf{y}', \mathbf{z})$, $\overline{\mathcal{R}}(\Phi)^{(1)} (\mathbf{u}, \mathbf{z})$, and $\overline{\mathcal{R}}(\Phi)^{(1)} (\mathbf{u}', \mathbf{z})$.
We leave it as a basic exercise for the reader to verify that the linear forms of
$$
\overline{\mathcal{R}}(\Phi)^{(1)} (\mathbf{y}, \mathbf{z}) - \overline{\mathcal{R}}(\Phi)^{(1)} (\mathbf{y}', \mathbf{z}) \
\bigcup
\ \overline{\mathcal{R}}(\Phi)^{(1)} (\mathbf{u}, \mathbf{z}) - \overline{\mathcal{R}}(\Phi)^{(1)} (\mathbf{u}', \mathbf{z})
$$
are linearly independent over $\mathbb{Q}$.

Therefore, it follows from Corollary \ref{cor Schmidt} that
$$
\mathcal{W}_1 \ll N^{n + 3M - K - (2 d + 2 D_2)}.
$$

We now estimate $\mathcal{W}_2$,
which we can deduce to be the number of solutions $\mathbf{z}, \mathbf{z}', \mathbf{z}'' \in [0,N]^{n-M-K } \cap \mathbb{Z}^{n-M-K}$
satisfying the equations
\begin{eqnarray}
\label{system 2 in claim 2}
{b}_M( \mathbf{z} ) - {b}_M( \mathbf{z}' ) &=& 0
\\
{b}_M( \mathbf{z} ) - {b}_M(  \mathbf{z}'' ) &=& 0
\notag
\\
{\mathcal{R}}(\Psi) ( \mathbf{z}) - {\mathcal{R}}(\Psi) (\mathbf{z}') &=& 0
\notag
\\
{\mathcal{R}}(\Psi) ( \mathbf{z}) - {\mathcal{R}}(\Psi) (\mathbf{z}'') &=& 0.
\notag
\end{eqnarray}
We consider the $h$-invariant of the system of forms on the left hand side of ~(\ref{system 2 in claim 2}), and show that
it is a regular system. The first two equations of ~(\ref{system 2 in claim 2})
are the degree $d$ polynomials of the system, and let $h_d$ be the $h$-invariant of these two polynomials.
Suppose for some $\lambda, \mu \in \mathbb{Q}$, not both $0$, we have
\begin{eqnarray}
\label{eqn f_M U V}
\lambda \cdot ( {f}_M ( \mathbf{z} ) - {f}_M (  \mathbf{z}' ) )
+ \mu \cdot ( {f}_M ( \mathbf{z} ) - {f}_M ( \mathbf{z}'' ) )
= \sum_{j=1}^{h_d} U_j \cdot V_j,
\notag
\end{eqnarray}
where $U_j = U_j( \mathbf{z}, \mathbf{z}', \mathbf{z}'' )$ and $V_j =
V_j( \mathbf{z}, \mathbf{z}', \mathbf{z}'' )$ are rational forms of positive degree $(1 \leq j \leq h_d)$.
We consider two cases, $(\lambda + \mu ) \not = 0$ and $(\lambda + \mu ) = 0$.
Suppose $(\lambda + \mu ) \not = 0$. If we set $\mathbf{z}' = \mathbf{z}'' = \mathbf{0}$, then the above equation becomes
\begin{eqnarray}
(\lambda + \mu ) \cdot  {f}_M( \mathbf{z} )
=  \sum_{j=1}^{h_d}  U_j (\mathbf{z}, \mathbf{0}, \mathbf{0} ) \cdot V_j (\mathbf{z}, \mathbf{0}, \mathbf{0} ).
\notag
\end{eqnarray}
Thus we obtain
$$
h_d \geq h( {f}_M( \mathbf{z} ) ).
$$
On the other hand, suppose $(\lambda + \mu ) = 0$, then the above equation ~(\ref{eqn f_M U V}) simplifies to
$$
{f}_M(  \mathbf{z}' ) - {f}_M(  \mathbf{z}'' )
= \frac{-1}{\lambda} \sum_{j=1}^{h_d} U_j \cdot V_j.
$$
From this equation, 
we substitute $\mathbf{z}'' = \mathbf{0}$ to obtain
$$
h_d \geq h( {f}_M(  \mathbf{z}' ) ).
$$
Therefore, in either case we obtain from ~(\ref{cond on h and M}) that
$$
h_d \geq h( {f}_M( \mathbf{z} ) ) \geq \rho_{d,d}( 2 + 2  R_1) + 2R_1 \geq \rho_{d,d}( 2 + 2  R_1 - 2 | { \mathbf{v} }^{(1)}|) + 2 | { \mathbf{v} }^{(1)}|.
$$
Recall we defined ${\mathcal{R}} (\Psi) = (\mathbf{v}^{(d-1)}, ..., \mathbf{v}^{(1)})$, where $\mathbf{v}^{(i)} = \mathcal{R}^{(i)}(\Psi)$ are the degree $i$ forms of
${\mathcal{R}} (\Psi) \ (1 \leq i \leq d-1)$. Take $2 \leq i \leq d-1$. Let $m_i = | \mathbf{v}^{(i)} |$, and we label the forms of $\mathbf{v}^{(i)}$ to be $v^{(i)}_1, ...,  v^{(i)}_{m_i}$. 
Let $h_i$ be the $h$-invariant of the degree $i$ forms of the system ~(\ref{system 2 in claim 2}).
Then for some $\boldsymbol{\lambda}, \boldsymbol{\mu} \in \mathbb{Q}^{m_i}$, not both $\mathbf{0}$, we have
\begin{equation}
\label{eqn f_M U V 1}
\sum_{j=1}^{m_i} \lambda_j \cdot ( v^{(i)}_j (\mathbf{z}) - v^{(i)}_j (\mathbf{z}' ) ) + \sum_{j=1}^{m_i} \mu_j \cdot ( v^{(i)}_j (\mathbf{z}) - v^{(i)}_j (\mathbf{z}'' ) )= \sum_{t=1}^{h_i} U_t \cdot V_t,
\end{equation}
where $U_t = U_t( \mathbf{z}, \mathbf{z}', \mathbf{z}'')$ and $V_t =
V_t( \mathbf{z}, \mathbf{z}', \mathbf{z}'')$ are forms of positive degree $(1 \leq t \leq h_i)$.
We consider two cases, $(\boldsymbol{\lambda} + \boldsymbol{\mu}) \not = \mathbf{0}$ and
$(\boldsymbol{\lambda} + \boldsymbol{\mu}) = \mathbf{0}$.

Suppose $(\boldsymbol{\lambda} + \boldsymbol{\mu}) \not = \mathbf{0}$. In this case, we set $\mathbf{z}' = \mathbf{z}'' = \mathbf{0}$, and equation ~(\ref{eqn f_M U V 1}) simplifies to
$$
\sum_{j=1}^{m_i} ( \lambda_j + \mu_j ) \cdot  v^{(i)}_j (\mathbf{z}) = \sum_{t=1}^{h_i} U_t( \mathbf{z}, \mathbf{0}, \mathbf{0}) \cdot  V_t ( \mathbf{z}, \mathbf{0}, \mathbf{0}).
$$
Therefore, it follows that
$$
h_i \geq h( { \mathbf{v} }^{(i)} ) \geq \rho_{d,i}( 2 + 2 R_1) + 2 R_1 \geq \rho_{d,i}( 2 + 2 R_1 - 2| { \mathbf{v} }^{(i)}|) + 2| { \mathbf{v} }^{(1)}|.
$$
On the other hand, suppose $(\boldsymbol{\lambda} + \boldsymbol{\mu}) = \mathbf{0}$.
Then equation ~(\ref{eqn f_M U V 1}) simplifies to
$$
\sum_{j=1}^{m_i} - \lambda_j \cdot (   v^{(i)}_j (\mathbf{z}') -  v^{(i)}_j (\mathbf{z}'' ) )= \sum_{t=1}^{h_i} U_t \cdot V_t.
$$
From this equation, 
we substitute $\mathbf{z}'' = \mathbf{0}$ to obtain
$$
h_i \geq h( { \mathbf{v} }^{(i)} ) \geq \rho_{d,i}( 2 + 2 R_1) + 2 R_1 \geq \rho_{d,i}( 2 + 2 R_1 - 2| { \mathbf{v} }^{(i)}|) + 2| { \mathbf{v} }^{(1)}|.
$$
\iffalse
\begin{eqnarray}
h_i &\geq&
h(  \mathbf{v}^{(i)} )
\notag
\\
&=&
h( \mathcal{R}^{(i)}(\Psi) )
\notag
\\
&\geq&
\rho_{d,i}( 2 + 2 R_1) + 2 R_1.
\notag
\\
&\geq&
\rho_{d,i}( 2 + 2 R_1 - 2 | { \mathbf{v} }^{(1)}| ) + 2 | { \mathbf{v} }^{(1)}|.
\notag
\end{eqnarray}
\fi

We also have to show that the linear forms of the system ~(\ref{system 2 in claim 2}),
\begin{equation}
\label{system of linear forms2}
\{ \mathbf{v}^{(1)}(\mathbf{z}) - \mathbf{v}^{(1)}(\mathbf{z}') \} \cup \{ \mathbf{v}^{(1)}(\mathbf{z}) - \mathbf{v}^{(1)}(\mathbf{z}'') \},
\end{equation}
are linearly independent over $\mathbb{Q}$.
Recall the linear forms of $\mathbf{v}^{(1)}(\mathbf{z})$ are linearly independent over $\mathbb{Q}$.
The linear independence over $\mathbb{Q}$ of the system of linear forms ~(\ref{system of linear forms2}) follows from this fact, and we leave the verification
as a basic exercise for the reader.

Therefore, we obtain by Corollary \ref{cor Schmidt} that
$$
\mathcal{W}_2 \ll N^{3(n -M - K) - (2 d + 2 D_1)}.
$$

Combining the bounds for $\mathcal{W}_1$ and $\mathcal{W}_2$ together, we obtain
$$
\mathcal{W} \leq \mathcal{W}_1^{1/2} \mathcal{W}_2^{1/2} \ll N^{2n -2 K - (2d + D_1 + D_2)},
$$
which proves Claim 2.
\end{proof}

\section{Hardy-Littlewood Circle Method: Major Arcs}
\label{section major arcs}
Recall $f_b$ is the degree $d$ portion of the degree $d$ polynomial $b(\mathbf{x}) \in \mathbb{Z}[x_1, ..., x_n]$.
In this section we assume that $f_b$ satisfies $h(f_b) > A_d$, where $A_d$ is defined in ~(\ref{def of Ad}).
We define $g_d(f_b)$ as in ~(\ref{def gd}) with $\mathbf{f} = \{ f_b \}$ and $r_d = 1$.
It then follows from ~(\ref{h and g}) 
that
$$
A_d < h(f_b) \leq (\log 2)^{-d} \cdot d! \cdot g_d(f_b).
$$
From this bound and our choice of $A_d$ in ~(\ref{def of Ad}), we have
\begin{equation}
\label{g and h}
\frac{ 2^{d-1}}{g_d(f_b)} < \frac{d!  2^{d-1} }{(\log 2)^d A_d} < \frac{d! 2^{d-1} }{(\log 2)^d (A_d - 5d) } < \frac{1}{5}.
\end{equation}

We take $\Omega$ to be
$$
4 \ < \ \Omega  \ < \ 5 \ \leq  \ \frac{(A_d - 5 d) \cdot  (\log 2)^{d}}{ 2^{d-1} (d-1) d! } \ \leq \  \frac{g_d(f_b)}{ 2^{d-1} (d-1)}.
$$
Therefore, with this choice of $\Omega$, we have that $b(\mathbf{x})$ satisfies the Hypothesis ($\star$) with $\mathfrak{B}_0$ by Proposition \ref{prop omega and g}.
We then choose $Q$ to satisfy $0 < Q < \Omega$ and
\begin{equation}
\label{Q bound 1}
Q \cdot \frac{ 2^{d-1}}{g_d(f_b)} < 1.
\end{equation}
We fix these values of $\Omega$ and $Q$ throughout this section.
We note that with these choices of $\Omega$ and $Q$, we have
\begin{equation}
\label{omega bound 2}
0 \ < \ \Omega \  \leq \frac{ (A_d - d Q) \cdot  (\log 2)^{d}}{ 2^{d-1} (d-1) d! }.
\end{equation}
The work of this section is based on \cite{CM} and it is similar to their treatment of
the major arcs. However, we had to tailor their argument to be in terms of the $h$-invariant instead of the Birch rank.

We define the following sums,
\begin{equation}
\label{defn Stilde}
\widetilde{S}_{{m}, q } = \sum_{\mathbf{k} \in \mathbb{U}_q^n} e( {b}(\mathbf{k}) \cdot {m}/q ),
\end{equation}
$$
B( q ) = \sum_{{m} \in \mathbb{U}_q} \frac{1}{\phi(q)^n} \ \widetilde{S}_{ {m}, q }, 
$$
where $\phi$ is Euler's totient function,
and
$$
\mathfrak{S}( N) = \sum_{q \leq (\log N)^C}  B( q ).
$$

Recall we denote $\mathfrak{B}_0 = [0,1]^n$.
We invoke the following estimate on the major arcs, which is an immediate consequence of \cite[Lemma 6]{CM}.
We remark that we had to make a slight modification to their proof of \cite[Lemma 6]{CM}, where we chose $c'$ to be between $(n+2)C + c + 1$ and $2(n+2)C + 2c + 2$ 
(instead of between $C + c$ and $2C + 2c$) during the proof, to obtain the following.

\begin{lem}[Lemma 6, \cite{CM}]
\label{Lemma 6 in CM} Given any $c>0$, there exists $C>0$ such that we have
$$
\int_{\mathfrak{M}_{{m}, q}(C)} T({b}; {\alpha} ) \ {d}{\alpha}
=
\frac{1}{\phi(q)^n} \ \widetilde{S}_{{m}, q } \  J_0 + O\left( q^n (\log N)^C \frac{N^{n - d}}{(\log N)^{2nC + 3C +2 c + 1}} \right),
$$
where
$$
J_0 = \int_{|\tau| \leq N^{-d} (\log N)^C } \int_{\mathbf{u} \in N \mathfrak{B}_0 } e( \tau b(\mathbf{u}) ) \ \mathbf{d} \mathbf{u} \ d \tau.
$$
\end{lem}
Note $J_0$ is independent of ${m}$ and  $q$.
We simplify the expression for $J_0$.
Let
$$
\mathcal{I}( {\eta}) = \int_{ \mathfrak{B}_0 } e( \eta f( \boldsymbol{\xi}) ) \ \mathbf{d} \boldsymbol{\xi}
$$
For any $\varepsilon > 0$, the inner integral of $J_0$ can be expressed as
\begin{eqnarray}
\int_{\mathbf{u} \in N \mathfrak{B}_0 } e( \tau b(\mathbf{u})  ) \ \mathbf{d} \mathbf{u}
&=& \int_{\mathbf{u} \in N \mathfrak{B}_0 } e( \tau f(\mathbf{u}) ) \ \mathbf{d} \mathbf{u} + O(N^{n - 1 + \varepsilon})
\notag
\\
&=& N^n \int_{\boldsymbol{\xi} \in \mathfrak{B}_0 } e( N^d \tau f(\boldsymbol{\xi})   ) \ \mathbf{d} \boldsymbol{\xi} + O(N^{n - 1 + \varepsilon})
\notag
\\
&=& N^n \cdot \mathcal{I}( N^d \tau ) + O(N^{n - 1 + \varepsilon}),
\notag
\end{eqnarray}
where we used the change of variable $\mathbf{u} =  N \boldsymbol{\xi}$ to obtain the second equality above.

We define
$$
J(L) = \int_{|\eta| \leq L } \mathcal{I}(\eta) \ d \eta.
$$
Then we can simplify $J_0$ as
$$
J_0 = N^{n-d} \cdot J( (\log N)^C ) + O(N^{n-d-1 + \varepsilon} (\log N)^C).
$$
Since we have $\Omega > 2$ and the Hypothesis $(\star)$, and in particular the restricted Hypothesis $(\star)$,
it follows by \cite[Lemma 8.1]{S} that
\begin{equation}
\label{(3.9) is S}
\mathcal{I}( {\eta}) \ll \min (1 , |{\eta}|^{-2} ).
\end{equation}
As stated in \cite[Section 3]{S}, it follows from ~(\ref{(3.9) is S}) that
$$
\mu(\infty) = \int_{\mathbb{R}} \mathcal{I}( {\eta}) \ d {\eta}
$$
exists.
Furthermore, we have
\begin{equation}
\label{(3.9') is S}
\Big{|} \mu(\infty) - J(L) \Big{|} \ll L^{- 1}.
\end{equation}

Therefore, we obtain the following estimate as a consequence of the definition of the major arcs and Lemma \ref{Lemma 6 in CM}.
\begin{lem}
\label{lemma major arc estimate}
Suppose $h(f_b) > A_d$, where we define $A_d$ as in ~(\ref{def of Ad}). Then given any $c>0$, there exists $C>0$  such that we have
$$
\int_{\mathfrak{M}(C) }T({b}; {\alpha} ) \ {d} {\alpha}
=
\mathfrak{S}(N) \mu(\infty) N^{n - d} + O\left( \mathfrak{S}(N) \frac{ N^{n - d}}{(\log N)^C} +  \frac{N^{n - d}}{(\log N)^c} \right).
$$
\end{lem}

\subsection{Singular Series}
\label{section singular series}
We obtain the following estimate on the exponential sum $\widetilde{S}_{{m}, q }$ defined in ~(\ref{defn Stilde}).
\begin{lem}
\label{to bound local factor}
Suppose $h(f_b) > A_d$, where we define $A_d$ as in ~(\ref{def of Ad}).
Let $p$ be a prime and let $q = p^t$, $t \in \mathbb{N}.$ For $m \in \mathbb{U}_q$, we have the following bounds
\begin{eqnarray}
\notag
\widetilde{S}_{ {m}, q} \ll
\left\{
    \begin{array}{ll}
         q^{n-Q},
         &\mbox{if } t \leq d ,\\
         p^Q q^{n-Q},
         &\mbox{if } t > d,
    \end{array}
\right.
\end{eqnarray}
where the implicit constants are independent of $p$.
\end{lem}

\begin{proof}
We consider the two cases $t \leq d$ and $t > d$ separately.
We apply the inclusion-exclusion principle to bound $\widetilde{S}_{ {m}, q }$ when $q = p^t$ and $t \leq d$,
\begin{eqnarray}
\label{bound on S tilde}
\widetilde{S}_{ {m}, q } &=& \sum_{\mathbf{k} \in \mathbb{U}_q^n} e( {b}(\mathbf{k}) \cdot {m}/q )
\\
&=& \sum_{\mathbf{k} \in (\mathbb{Z} / q \mathbb{Z})^n} \prod_{i=1}^n \left( 1 -  \sum_{u_i \in \mathbb{Z} / p^{t-1} \mathbb{Z} } \mathbf{1}_{k_i = p u_i}  \right)
e( {b}(\mathbf{k}) \cdot {m}/q )
\notag
\\
\notag
&=& \sum_{ I \subseteq \{ 1, 2, ..., n \}} (-1)^{|I|}
\sum_{ \mathbf{u} \in  (\mathbb{Z} / p^{t-1} \mathbb{Z} )^{|I|} } \
\sum_{\mathbf{k} \in (\mathbb{Z} / q \mathbb{Z})^n}
F_I(\mathbf{k}; \mathbf{u}) e( {b}(\mathbf{k}) \cdot {m}/q ),
\notag
\end{eqnarray}
where
$\mathbf{1}_{k_i = p u_i}$ denotes a characteristic function and
$$
F_I(\mathbf{k}; \mathbf{u}) = \prod_{i \in I} \mathbf{1}_{k_i = p u_i}
$$
for $\mathbf{u} \in (\mathbb{Z} / p^{t-1} \mathbb{Z})^{|I|}$.
In other words, $F_I(\mathbf{k}; \mathbf{u})$ is the characteristic function of the set $H_{I, \mathbf{u}} = \{ \mathbf{k} \in (\mathbb{Z} / q \mathbb{Z})^{n} : k_i = p u_i \ (i \in I)\}$.
We now bound the summand in the final expression of ~(\ref{bound on S tilde}) by further considering two cases, $|I| \geq t Q$ and $|I| < t Q$.
In the first case $|I| \geq t Q$, we use the following trivial estimate
\begin{eqnarray}
\Big{|} \sum_{ \mathbf{u} \in  (\mathbb{Z} / p^{t-1} \mathbb{Z})^{|I|} } \
\sum_{\mathbf{k} \in (\mathbb{Z} / q \mathbb{Z})^n}
F_I(\mathbf{k}; \mathbf{u}) e( {b}(\mathbf{k}) \cdot {m}/q ) \Big{|}
&\leq&
p^{(t-1) |I|} (p^t)^{n - |I|}
\notag
\\
&=& q^{n - |I|/t}
\notag
\\
&\leq& q^{n - Q}.
\notag
\end{eqnarray}

On the other hand, suppose  $|I| < t Q$.
Let
$$
\mathfrak{g}_b(\mathbf{x}) = {b}(\mathbf{x})|_{x_i = p u_i ( i \in I )},
$$
or equivalently the polynomial $\mathfrak{g}_b(\mathbf{x})$ is obtained by
substituting $x_i = p u_i \ (i \in I)$ to the polynomial ${b}(\mathbf{x})$.
Thus $\mathfrak{g}_b(\mathbf{x})$ is a polynomial in $n - |I|$ variables.
We can also deduce easily that the degree $d$ portion of the polynomial $\mathfrak{g}_b(\mathbf{x})$, which we denote
$f_{\mathfrak{g}_b}$, is obtained by substituting $x_i = 0 \ (i \in I)$ to the degree $d$ portion of the polynomial ${b}(\mathbf{x})$.
Hence, we have
$$
f_{\mathfrak{g}_b } =  {f_b} |_{x_i = 0 \ (i \in I)}.
$$
Consequently, we obtain by Lemma \ref{h ineq 1} that
$$
h(f_{\mathfrak{g}_b }) \geq h( {f_b}) - |I| > h( {f_b}) - dQ > A_d - dQ.
$$
By our choice of $Q$ and $\Omega$, and from ~(\ref{h and g}) and ~(\ref{omega bound 2}), we have
$$
0 \ < \ Q \ < \ \Omega \  < \frac{ h(f_{\mathfrak{g}_b}) \cdot (\log 2)^{d}}{ 2^{d-1} (d-1) d! }
\leq \  \frac{g_d( f_{\mathfrak{g}_b} )}{2^{d-1} (d-1) }.
$$
Therefore, with these notations we have by Lemma \ref{bound on E} that
\begin{eqnarray}
\sum_{\mathbf{k} \in (\mathbb{Z} / q \mathbb{Z})^n} F_I(\mathbf{k}; \mathbf{u}) e( {b}(\mathbf{k}) \cdot {m}/q )
&=& \sum_{\mathbf{s} \in (\mathbb{Z} / q \mathbb{Z})^{n - |I|}} e( \mathfrak{g}_b(\mathbf{s}) \cdot {m}/q ).
\notag
\\
&=& q^{n - |I|} E ( \ \mathfrak{g}_b,  q ;{m}/q )
\notag
\\
&\ll& q^{n - |I| - Q}.
\notag
\end{eqnarray}
Thus, we obtain
$$
\sum_{ \mathbf{u} \in  (\mathbb{Z} / p^{t-1} \mathbb{Z})^{|I|} } \
\sum_{\mathbf{k} \in (\mathbb{Z} / q \mathbb{Z})^n}
F_I(\mathbf{k}; \mathbf{u}) e(  {b}(\mathbf{k}) \cdot  {m}/q )
\ll (p^{t-1})^{|I|} q^{n - |I| - Q} \leq q^{n-Q}.
$$
Consequently, combining the two cases $|I| \geq tQ$ and $|I| < tQ$ together, we obtain
$$
\widetilde{S}_{ {m}, q} \ll q^{n-Q}
$$
when $t \leq d$.

We now consider the case $q = p^t$ when $t>d$. By the definition of $\widetilde{S}_{ {m}, q }$, we have
\begin{eqnarray}
\label{widetile S part 3-1}
\widetilde{S}_{ {m}, q } &=& \sum_{\mathbf{k} \in \mathbb{U}_q^n} e(  {b}(\mathbf{k}) \cdot  {m}/q )
\\
&=& \sum_{\mathbf{k}' \in \mathbb{U}_p^n} \  \sum_{ \mathbf{s} \in (\mathbb{Z} / (p^{t-1} \mathbb{Z}) )^n  } e( {b}(\mathbf{k}' + p \mathbf{s}) \cdot {m}/q )
\notag
\\
&=& \sum_{\mathbf{k}' \in \mathbb{U}_p^n} \  \sum_{ \mathbf{s} \in [0,p^{t-1})^n   } e(  {b}(\mathbf{k}' + p \mathbf{s}) \cdot {m}/q ).
\notag
\end{eqnarray}
For each fixed $\mathbf{k}' \in \mathbb{U}_p^n$, we have
$$
b(\mathbf{k}' + p \mathbf{s}) = p^d  f_b( \mathbf{s}) + \chi_{p, \mathbf{k}'}(\mathbf{s}),
$$
where $\chi_{p, \mathbf{k}'}(\mathbf{x})$ is a polynomial of degree at most $d-1$ and its coefficients depend on $p$ and $\mathbf{k}'$.
We apply Corollary \ref{cor 15.1 in S} with $r_d=1$, $\psi(\mathbf{x}) = f_b( \mathbf{x}) + \frac{1}{p^d} \ \chi_{p, \mathbf{k}'}(\mathbf{x})$, $ {\alpha} = {m}/p^{t-d}$, $\mathcal{B} = [0,1)^{n}$, and $P = p^{t-1}$.
Let $\varepsilon' > 0$ be sufficiently small.
Recall from ~(\ref{Q bound 1}) that our choice of $Q>0$ satisfies $$Q \cdot \frac{2^{d-1}}{g_d( f_b)} < 1. $$
Let $\gamma_d$ and $\gamma_d'$ be as in the paragraph before Corollary \ref{cor 15.1 in S} with $\mathbf{f} = \{ f_b \}$ and $r_d = 1$.
Suppose the alternative $(ii)$ of Corollary \ref{cor 15.1 in S} holds. Then we know there exists $n_0 \in \mathbb{N}$ such that
$$
n_0 \ll (p^{t-1}-1)^{Q \gamma_d + \varepsilon'}
$$
and
\begin{equation}
\label{ineq of n0 in sing ser'}
\| n_0 ( {m}/p^{t-d})  \| \ll (p^{t-1}-1)^{-d + Q \gamma_d + \varepsilon'} \leq \left( \frac{1}{2}p^{t-1} \right)^{-d + Q \gamma_d + \varepsilon'}.
\end{equation}
However, this is not possible once $p^t$ is sufficiently large with respect to $n,d, \varepsilon', Q$, and $ {f_b}$, for the following reason. First note that $n_0$ can not be divisible by $p^{t-d}$ for $p^t$ sufficiently large, because
$Q \gamma_d + \varepsilon' < Q \gamma_d' < 1$. Since $n_0 \in \mathbb{N}$ is not divisible by $p^{t-d}$ and $(m,p)=1$, we have
$$
\| n_0 ( {m}/p^{t-d})  \| \geq \frac{1}{p^{t-d}},
$$
which contradicts ~(\ref{ineq of n0 in sing ser'}) for $p^t$ sufficiently large.

Thus by Corollary \ref{cor 15.1 in S}, we can bound the inner sum of ~(\ref{widetile S part 3-1}) by
\begin{eqnarray}
\sum_{ \mathbf{s} \in [0, p^{t-1})^n  } e \left( \left( f( \mathbf{s}) + \frac{1}{p^d} \ \chi_{p, \mathbf{k}'}(\mathbf{s}) \right) \cdot {m}/ p^{t-d} \right)
&\ll&
\notag (p^{t-1})^{n - Q},
\end{eqnarray}
where the implicit constant depends at most on $n,d, \varepsilon', Q$, and $ {f_b}$.
Therefore, we can bound ~(\ref{widetile S part 3-1}) as follows
\begin{eqnarray}
\widetilde{S}_{{m}, q } &\leq&
\sum_{\mathbf{k}' \in \mathbb{U}_p^n} \Big{|} \sum_{ \mathbf{s} \in [0, p^{t-1})^n   } e( (p^d  f_b( \mathbf{s}) + \chi_{p, \mathbf{k}'}(\mathbf{s})) \cdot  {m}/q ) \Big{|}
\notag
\\
&=&
\sum_{\mathbf{k}' \in \mathbb{U}_p^n} \Big{|} \sum_{ \mathbf{s} \in [0, p^{t-1})^n  } e \left( \left( f_b( \mathbf{s}) + \frac{1}{p^d} \ \chi_{p, \mathbf{k}'}(\mathbf{s}) \right) \cdot  {m}/ p^{t-d} \right) \Big{|}
\notag
\\
&\ll&
p^n (p^{t-1})^{n - Q}
\notag
\\
&=&
p^{Q} q^{n - Q}.
\notag
\end{eqnarray}

\end{proof}

For each prime $p$, we define
\begin{equation}
\label{def mu p}
\mu(p) =  1  + \sum_{t=1}^{\infty} B( p^t),
\end{equation}
which converges absolutely provided that $h(f_b) > A_d$ as we see in the following lemma.
As stated in \cite{CM}, by following the outline of L. K. Hua \cite[Chapter VII, \S 2, Lemma 8.1]{H} one can show that $B(q)$
is a multiplicative function of $q$. Therefore, we consider the following identity
\begin{equation}
\label{def sigular series}
\mathfrak{S}(\infty) := \lim_{N \rightarrow \infty} \mathfrak{S}(N) = \prod_{p \ \text{prime}} \mu(p).
\end{equation}

\begin{lem}
\label{singular series lemma}
There exists $\delta_1 > 0$ such that for each prime $p$, we have
$$
\mu(p) = 1 + O(p^{-1 + \delta_1}),
$$
where the implicit constant is independent of $p$.
Furthermore, we have
$$
\Big{|} \mathfrak{S}(N) -  \mathfrak{S}(\infty) \Big{|} \ll (\log N)^{-C \delta_2 }
$$
for some $\delta_2 > 0$.
\end{lem}
Therefore, the product in ~(\ref{def sigular series}) converges and the limit exists.
\begin{proof}
Recall our choice of $Q$ satisfies $Q > 4 \geq 2d/(d-1)$.
For any $t \in \mathbb{N}$, we know that $\phi(p^t) = p^t(1 - 1/p) \geq \frac12 p^t$.
Therefore, by considering the two cases as in Lemma \ref{to bound local factor}, we obtain
\begin{eqnarray}
| \mu(p) - 1 |
&\leq&
\sum_{1 \leq t \leq d} \Big{|} \sum_{{m} \in \mathbb{U}_{p^t}} \frac{1}{\phi(p^t)^n} \ \widetilde{S}_{ {m}, p^t }  \Big{|}
+
\sum_{ t > d} \Big{|} \sum_{{m} \in \mathbb{U}_{p^t}} \frac{1}{\phi(p^t)^n} \ \widetilde{S}_{ {m}, p^t } \Big{|}
\notag
\\
&\ll&
\sum_{1 \leq t \leq d} p^{t} p^{-nt} p^{nt - t Q }
+
\sum_{ t > d} p^{t} p^{-nt} p^{Q + nt - t Q }
\notag
\\
&\ll&
p^{1 - Q }
+
p^Q p^{-(d+1)(Q-1)}
\notag
\\
&\ll&
p^{1 - Q }
+
p^{-dQ + d +1 }
\notag
\\
&\ll&
p^{-1 + \delta_1},
\notag
\end{eqnarray}
for some $\delta_1 > 0$. We note that the implicit constants in $\ll$ are independent of $p$ here.

Let $q = p_1^{t_1} ... p_{v}^{t_{v}}$ be the prime factorization of $q \in \mathbb{N}$.
Without loss of generality, suppose we have $t_j \leq d \ (1 \leq j \leq v_0)$ and
$t_j > d \ (v_0 < j \leq v)$.
By the multiplicativity of $B(q)$, it also follows from Lemma \ref{to bound local factor} that
\begin{eqnarray}
B(q)
&=&
B(p_1^{t_1}) \ ... \  B(p_{v}^{t_{v}})
\notag
\\
&\ll&
\left( \prod_{j=1}^{v_0} p_j^{t_j} p_j^{-n t_j}  p_j^{t_j (n - Q)} \right) \cdot \left(  \prod_{j=v_0 + 1}^{v}  p_j^{t_j} p_j^{-n t_j} p_j^{Q} p_j^{t_j (n - Q)} \right)
\notag
\\
&=&
q^{1 - Q} \cdot  \left(  \prod_{j=v_0 + 1}^{v}  p_j^{ Q } \right)
\notag
\\
&\leq&
q^{1 - Q} \cdot  q^{Q/d}
\notag
\\
&\leq&
q^{- 1 - \delta_2},
\notag
\end{eqnarray}
for some $\delta_2 > 0$. Again we note that the implicit constant in $\ll$ is independent of $q$ here.
Therefore, we obtain
\begin{eqnarray}
\Big{|} \mathfrak{S}(N) -  \mathfrak{S}(\infty) \Big{|}
&\leq&
 \sum_{q > (\log N)^C } |  B(q) |
\notag
\\
&\ll&
\sum_{q > (\log N)^C } q^{- 1 - \delta_2}
\notag
\\
&\ll&
(\log N)^{- C \delta_2 }.
\notag
\end{eqnarray}
\end{proof}

Let $\nu_t(p)$ denote the number of solutions $\mathbf{x} \in (\mathbb{U}_{p^t})^n$
to the congruence
\begin{eqnarray}
b( \mathbf{x} ) \equiv 0 \  (\text{mod } p^t).
\end{eqnarray}
Then using the fact that
\begin{eqnarray}
\notag
\sum_{ {m} \in \mathbb{Z}/(p^t \mathbb{Z} ) } e \left(  a  \cdot {m}/p^t  \right) =
\left\{
    \begin{array}{ll}
         p^{t},
         &\mbox{if } p^t | a ,\\
         0,
         &\mbox{otherwise,}
    \end{array}
\right.
\end{eqnarray}
we deduce
\begin{eqnarray}
&&1 + \sum_{j=1}^t B(p^j)
\notag
\\
&=&
1 + \sum_{j=1}^t  \frac{1}{\phi(p^j)^n} \sum_{\mathbf{k} \in (\mathbb{U}_{p^j})^{n}}   \sum_{ {m} \in \mathbb{U}_{p^j} }
 e \left(  {b}(\mathbf{k})  \cdot {m}/p^j  \right)
\notag
\\
&=& \frac{1}{\phi(p^t)^n} \sum_{\mathbf{k} \in (\mathbb{U}_{p^t})^{n}}   \sum_{ {m} \in \mathbb{Z}/(p^t \mathbb{Z} ) } e \left(  {b}(\mathbf{k})  \cdot {m}/p^t  \right)
\notag
\\
&=&  \frac{p^{t}}{\phi(p^t)^n } \ \nu_t(p).
\notag
\end{eqnarray}
Therefore, provided  $h({b}) > A_{d}$ we obtain
$$
\mu(p) = \lim_{t \rightarrow \infty}  \frac{  p^{t } \ \nu_t(p) }{ \phi(p^t)^n }.
$$
At this point we refer the reader to \cite[pp. 704, 736]{CM} to conclude
$$
\mu(p) > 0
$$
if the equation ~(\ref{main system}) has a non-singular solution in $\mathbb{Z}_p^{\times}$, the units of $p$-adic integers.
It then follows from Lemma \ref{singular series lemma} that if the equation ~(\ref{main system}) has a non-singular solution in $\mathbb{Z}_p^{\times}$
for every prime $p$, then
$$
\prod_{p \ \text{prime}}\mu(p) > 0.
$$

\section{Conclusion and further remarks}
\label{sec concln}
As a consequence of Lemmas \ref{lemma major arc estimate} and \ref{singular series lemma}, and
Proposition \ref{prop minor arc bound}, we obtain the final form of the asymptotic formula.
It follows that given any $c > 0$, there exists $C>0$ such that we have
\begin{eqnarray}
\mathcal{M}_{{b}}(N)
&=& \int_0^1 T({b}; {\alpha} ) \ {d}{\alpha}
\notag
\\
&=& \int_{\mathfrak{M}(C) } T({b}; {\alpha} ) \ {d} {\alpha} + \int_{\mathfrak{m}(C)} T({b}; {\alpha} ) \ {d} {\alpha}
\notag
\\
&=&  \mathfrak{S}(\infty) \mu(\infty) N^{n - d} + O \left(  \frac{N^{n-d}}{ (\log N)^c }  \right),
\end{eqnarray}
which completes the proof of Theorem \ref{the main theorem}.

Some possible refinements to Theorem \ref{CM theorem}, the main result of B. Cook and \'{A}. Magyar in \cite{CM},
are provided in \cite[Section 8]{CM}. The list includes for example, in terms of our Theorem \ref{the main theorem}, to replace our assumption of largeness of $h^{\star}(b)$ with the $h$-invariant of $b(\mathbf{x})$. It also includes improving the upper bound for the Birch rank required in Theorem \ref{CM theorem}, which ``already exhibit(s) tower type behavior in $d$'' \cite{CM} for a single polynomial case. Our estimate of $A_d$ is comparable to this bound, and it is quite large. We refer the reader to \cite[Section 8]{CM} for more information on this topic. It would also be interesting to determine whether the methods of this
paper can be developed for system of polynomial equations.
\begin{thebibliography}{9}

\bibitem{B}  B. J. Birch,\textit{Forms in many variables}, Proc. Roy. Soc. Ser. A 265 1961/1962, 245–263.

\bibitem{BGS} J. Bourgain, A. Gamburd and P. Sarnak, \textit{Affine linear sieve, expanders, and sum-product}, Invent. Math. 179 (2010), no. 3, 559–644.

\bibitem{BP} T.D. Browning, and S.M. Prendiville, \textit{Improvements in Birch's theorem on forms in many variables},
J. Reine Angew. Math., to appear.

\bibitem{BDLW} J. Br\"{u}dern, R. Dietmann, J. Liu and T. D. Wooley,
\textit{A Birch-Goldbach theorem}, Arch. Math. (Basel) 94 (2010), no. 1, 53–58.

\bibitem{CM} B. Cook and {\'A}.  Magyar, \textit{Diophantine equations in the primes},
Invent. Math. {198} (2014), 701-737.

\bibitem{DRS} W. Duke, Z. Rudnick and P. Sarnak, \textit{Density of integer points on affine homogeneous varieties}, Duke Math. J. 71 (1993), no. 1, 143–179.

\bibitem{GPY}  D. A. Goldston,  J. Pintz and  C. Y. Y{\i}ld{\i}r{\i}m, \textit{Primes in tuples. I}, Ann. of Math. (2) 170 (2009), no. 2, 819–862.

\bibitem{GT} B. Green and  T. Tao, \textit{Linear equations in primes}, Ann. of Math. (2) 171 (2010), no. 3, 1753–1850.

\bibitem{GT1} B. Green and  T. Tao, \textit{The distribution of polynomials over finite fields, with applications to the Gowers norms}, Contrib. Discrete Math. 4 (2009), no. 2, 1–36.

\bibitem{H1} H. A. Helfgott, \textit{Major arcs for Goldbach's problem}, arXiv:1305.2897.

\bibitem{H2} H. A. Helfgott, \textit{Minor arcs for Goldbach's problem}, arXiv:1205.5252.

\bibitem{KL} T. Kaufman and S. Lovett, \textit{Worst case to average case reductions for polynomials}, 49th Annual IEEE Symposium on Foundations of Computer Science (2008), 166-175.

\bibitem{H}  L. K. Hua,  \textit{ Additive theory of prime numbers}, Translations of Mathematical Monographs, Vol. 13 American Mathematical Society,
Providence, R.I. (1965).

\bibitem{L} J. Liu, \textit{Integral points on quadrics with prime coordinates},
Monatsh. Math. 164 (2011), no. 4, 439–465.

\bibitem{LS}  J. Liu and P. Sarnak, \textit{Integral points on quadrics in three variables whose coordinates have few prime factors}, Israel J. of math., 178 (2010), 393–426.

\bibitem{M}  J. Maynard, \textit{Small gaps between primes}, Ann. of Math. (2) 181 (2015), no. 1, 383–413.

\bibitem{DamS} D. Schindler, \textit{A variant of Weyl's inequality for systems of forms and applications},	arXiv:1403.7156.

\bibitem{S} W.M. Schmidt, \textit{The density of integer points on homogeneous varieties},
Acta Math. {154} (1985), no. 3-4, 243-296.

\bibitem{V} I.M. Vinogradov. \textit{Representation of an odd number as
a sum of three primes}, Dokl. Akad. Nauk. SSR, 15:291–294, 1937

\bibitem{Z}  Y. Zhang, \textit{Bounded gaps between primes}, Ann. of Math. (2) 179 (2014), no. 3, 1121–1174.

\end{thebibliography}

\end{document}

