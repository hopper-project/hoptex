\documentclass[12pt]{amsart}
\usepackage[]{hyperref}
\usepackage{amssymb,amsfonts,amsmath,amsopn,amstext,amscd,latexsym, amsthm, enumerate,mathrsfs}

\usepackage[margin=36mm]{geometry}
\headheight=14pt
\parskip 1mm

\usepackage{bbm}
\usepackage[all,cmtip]{xy}

\usepackage{enumerate}
\usepackage[shortlabels]{enumitem}

\newtheorem{theorem}{Theorem}[section]
\newtheorem{thm}[theorem]{Theorem}
\newtheorem{lemma}[theorem]{Lemma}

\newtheorem*{thmA}{Theorem~A}
\newtheorem*{thmB}{Theorem~B}
\newtheorem*{thmC}{Theorem C}
\newtheorem*{example}{Examples}
\newtheorem{lem}[theorem]{Lemma}
\newtheorem{sublemma}[theorem]{Sublemma}
\newtheorem{proposition}[theorem]{Proposition}
\newtheorem{prop}[theorem]{Proposition}
\newtheorem{corollary}[theorem]{Corollary}
\newtheorem{cor}[theorem]{Corollary}
\newtheorem{definition}[theorem]{Definition}
\newtheorem*{question}{Question}
\theoremstyle{remark}

\newtheorem{remark}[theorem]{Remark}

\numberwithin{equation}{section}

\begin{document}

\title[Brauer characters of $q^\prime$-degrees]{Brauer characters of $q^\prime$- degrees}

\author[M. L. Lewis]{Mark L. Lewis}
\address{Department of Mathematical Sciences, Kent State University, Kent, OH 44242, USA}
\email{lewis@math.kent.edu}

\author[H. P. Tong-Viet]{Hung P. Tong Viet}
\address{Department of Mathematical Sciences, Kent State University, Kent, OH 44242, USA}
\email{htongvie@math.kent.edu}

\thanks{}

\subjclass[2010]{Primary 20C20; Secondary 20C15, 20B15}

\date{\today}

\date{Jan 6, 2016}

\keywords{Brauer character degrees; derangements}

\begin{abstract}
We show that if $p$ is a prime and $G$ is a finite $p$-solvable group satisfying the condition that a prime $q$ divides the degree of no irreducible $p$-Brauer character of $G$, then the normalizer of some Sylow $q$-subgroup of $G$ meets all the conjugacy classes of $p$-regular elements of $G$.
\end{abstract}

\maketitle

\section{Introduction}

Throughout this paper, $G$ will be a finite group and $p$ will be a prime. Let $\operatorname{Irr} (G)$ be the set of all complex  irreducible characters of $G$ and  let $\operatorname{IBr} (G)$ be the set of irreducible $p$-Brauer characters of $G$.  The celebrated It\^{o}-Michler theorem says that $p$ does not divide $\chi (1)$ for all $\chi \in \operatorname{Irr} (G)$ if and only if $G$ has a normal abelian Sylow $p$-subgroup.

Many variations of this theorem have been proposed and studied in the literature.  See the recent survey paper by G. Navarro on this topic in \cite{N1}.  One might ask whether there is any version of It\^{o}-Michler theorem for Brauer characters of finite groups. Now let $q$ be a prime and assume that $q$ divides the degree of no irreducible $p$-Brauer character of $G.$  Indeed, it is known that if $q=p,$ then $G$ has a normal Sylow $q$-subgroup.  (See Theorem 3.1 of \cite{N1}.)

This raises the question of whether there exists a similar result when $q\ne p$.  In particular, in Problem 3.2 of \cite{N1}, Navarro asks when $G$ is a $p$-solvable group and $q$ divides the degree of no irreducible $p$-Brauer character, is it true that every $p$-regular conjugacy class of $G$ intersects the normalizer of a Sylow $q$-subgroup of $G?$  In our first result, we prove that this is true.

\begin{thmA}\label{th:main1}
Let $p$ be a prime and let $G$ be a finite $p$-solvable group. Let $q$ be a prime and suppose that $q$ divides the degree of no irreducible $p$-Brauer character of $G.$ Then every $p$-regular conjugacy class of $G$ meets ${\mathbf{N}}_G (Q),$ where $Q$ is a Sylow $q$-subgroup of $G.$
\end{thmA}

The conclusion of Theorem A can be restated in the language of permutation group theory as follows. Let $H$ be a proper subgroup of a finite group $G.$ Following \cite{IKLM}, we say that an element $x\in G$ is an \emph{$H$-derangement} in $G$ if the conjugacy class $x^G$ containing $x$ does not meet $H.$ We write $\Delta_H(G)$ for the set of all $H$-derangements of $G.$ If $H$ is core-free in $G,$ then $G$ is a permutation group acting on the right coset space $\Omega=G/H$ with point stabilizer $H$ and $\Delta(G)=\Delta_H(G)$ is the set of all derangements or fixed-point-free elements of $G$ on $\Omega.$

A classical theorem due to Jordan \cite{Jordan} says that the set $\Delta_H(G)$ is non-empty. Notice that $$\Delta_H(G)=G\setminus \cup_{g\in G}H^g.$$  Derangements have many applications in topology and number theory (see \cite{Serre}). Derangements have been used in studying zeros of ordinary character theory. Now with this concept, Theorem A can be restated as follows.

\medskip
\emph{Let $p$ and $q$ be primes and let $G$ be a  finite $p$-solvable group and $Q$ a Sylow $q$-subgroup of $G.$ If $q$ divides the degree of no irreducible $p$-Brauer character of $G,$ then all ${\mathbf{N}}_G(Q)$-derangements of $G$ have order divisible by $p$.}

\medskip
As already mentioned in \cite{N1}, using the It\^{o}-Michler Theorem for Brauer characters and a result in \cite{FKS} which states that every finite permutation group of degree $>1$ contains a derangement of prime power order, it is easy to see that for a finite group $G,$ every $p$-Brauer character of $G$ has $p'$-degree if and only if every ${\mathbf{N}}_G(P)$-derangement of $G,$ for some Sylow $p$-subgroup $P$ of $G,$ has order divisible by $p.$ That is, Theorem A holds for all finite groups when $q=p.$ Unfortunately, this does not hold true when $q$ is different from $p$.  We will provide examples to show that the solvable assumption on $G$ in Theorem A is necessary.

Returning to Problem 3.2 of \cite{N1}, we note that Navarro asks for a characterization of $p$-solvable groups where $q \ne p$ and $q$ does not divide the degree of any irreducible $p$-Brauer character of $G$.  Our next goal is to obtain just such a characterization.  Manz and Wolf studied these groups in \cite{MW2}.  Many of the results of that paper can be found in Section $13$ of \cite{MW1}.  In Corollary $13.15$ of \cite{MW1}, they prove that ${\bf O}^{q'} (G)$ is solvable, and in Theorem $13.8$ of \cite{MW1}, they prove that in a $q$-series for $G$, the $q$-factors are abelian, the $q$-length of $G/{\bf O}_{p,q} (G)$ is at most $1$, and the Sylow $q$-subgroups are metabelian.  When $G$ is a finite group and $p$ is a prime, ${\mathbf{O}}_p(G)$ is contained in the kernel of all $p$-Brauer characters of $G.$  So there is no loss in assuming that ${\mathbf{O}}_p(G)=1.$  

We will provide examples of groups that meet the conclusion of Theorem A and the conditions of Manz and Wolf, yet have irreducible $p$-Brauer characters whose degrees are divisible by $q$.  Thus, to state our characterization, we need one further condition beyond the one stated in Theorem A and the conditions found by Manz and Wolf.  To state this condition, we introduce the following notation. Let $M\unlhd G$ and $N\unlhd M.$  We define $${\mathbf{C}}_G(M/N)=\{g\in G\,:\, [g,M]\subseteq N\}.$$  Note that we are not assuming that $N$ is normal in $G$.  We will prove that ${\mathbf{C}}_G(M/N)$ is a subgroup of $G$ containing $M$ whenever $M/N$ is abelian.  We will also prove that if $M \le K \le G$ and $K' \le N$, then $K \le {\mathbf{C}}_G (M/N)$; so in condition (5) of Theorem B, the fact that $Q'^g \le N$ implies that $Q' \le C$.  Note that condition (5)  is required to be true for all normal subgroups of ${\mathbf{O}}_q (L)$ whose quotient is cyclic.  We will need $C/{\mathbf{O}}_q (L)$ to satisfy the same conditions as $G$, and while it is obvious that conditions (2), (3), and (4) will be inherited from $G$, it is not clear that condition (1) is inherited from $G$, so we need to assume this for $C/{\mathbf{O}}_q (L)$ making condition (5) quite technical.

\begin{thmB}\label{th:main2}
Let $p$ and $q$ be distinct primes and suppose that $G$ is $p$-solvable with ${\mathbf{O}}_p(G)=1$. Let $L:={\mathbf{O}}^{q'}(G)$ and let $Q\leq L$ be a Sylow $q$-subgroup of $G$. Then $q\nmid\varphi(1)$ for all $\varphi\in\operatorname{IBr}(G)$ if and only if the following conditions hold:
\begin{enumerate}[$(1)$]
\item $x^G\cap {\mathbf{N}}_G(Q)\neq\emptyset$ for all $p$-regular elements $x\in G;$
\item In each $q$-series of $G,$ the $q$-factors are abelian and a Sylow $q$-subgroup of $G$ is metabelian.
\item $L$ is solvable;
\item $L/{\mathbf{O}}_{p,q}(L)$ has $q$-length at most $1;$
\item For every normal subgroup $N$ of ${\mathbf{O}}_q(L)$ with ${\mathbf{O}}_q(L)/N$ cyclic, there exists an element $g\in L$ such that $(Q^g)'\leq N$ and $$(x {\mathbf{O}}_q (L))^{C/{\mathbf{O}}_q(L)} \cap N_{C/{\mathbf{O}}_q (L)} (Q^g/{\mathbf{O}}_q (L)) \neq \emptyset$$ for all $p$-regular elements $x {\mathbf{O}}_q (L) \in C/{\mathbf{O}}_q (L),$ where $$C = {\mathbf{C}}_L ({\mathbf{O}}_q (L)/N).$$
\end{enumerate}

\end{thmB}

This gives a group theoretical characterization of finite $p$-solvable groups whose all irreducible $p$-Brauer characters have $q'$-degree. This answers positively Problem 3.2 in \cite{N1}.

\section{Proof of Theorem A}

Throughout this section, $p$ and $q$ are distinct primes and $G$ is a finite group. Suppose that $q$ divides the degree of no irreducible $p$-Brauer character of $G$ and let $N$ be a normal subgroup of $G.$ Since $\operatorname{IBr}(G/N)\subseteq\operatorname{IBr}(G)$ and by \cite[Corollary~8.7]{Navarro}, we see that both $N$ and $G/N$ satisfy this property.  In particular,  all $p$-Brauer characters of ${\mathbf{O}}^{q'}(G)$ have $q'$-degree.  We show the converse of this holds when $G/N$ is $p$-solvable.

\begin{lemma}\label{lem1} Let $p$ and $q$ be distinct primes and
suppose that $G/{\mathbf{O}}^{q'}(G)$ is $p$-solvable. Then $q\nmid \varphi(1)$ for all $\varphi\in\operatorname{IBr}(G)$ if and only if $q\nmid \beta(1)$ for all $\beta\in\operatorname{IBr}({\mathbf{O}}^{q^\prime}(G)).$
\end{lemma}

\begin{proof} By the discussion above, it suffices to show that  if all irreducible $p$-Brauer characters of $L:={\mathbf{O}}^{q^\prime}(G)$ have $q'$-degree, then $q\nmid \varphi(1)$ for all $\varphi\in\operatorname{IBr}(G)$. Let $\varphi\in\operatorname{IBr}(G)$ and let $\theta\in\operatorname{IBr}(L)$ be an irreducible constituent of $\varphi_L.$ By \cite[Theorem~8.30]{Navarro}, we have ${\varphi(1)}/{\theta(1)}$ divides $|G/L|.$ Since $G/L$ is a $q'$-group and $q\nmid \theta(1)$ by our assumption, we deduce that $q\nmid \varphi(1).$
\end{proof}

We next formally state the results of Theorem 13.8 and Corollary 13.15 of \cite{MW2}.

\begin{lemma}\label{lem2} Let $p$ and $q$ be distinct primes and let $G$ be a finite $p$-solvable group. Suppose that $q\nmid \varphi(1)$ for all $\varphi\in\operatorname{IBr}(G).$ Then the following hold.

\begin{enumerate}
\item ${\mathbf{O}}^{q'}(G)$ is solvable and so $G$ is $q$-solvable.
\item In each $q$-series of $G,$ the $q$-factors are abelian and the Sylow $q$-subgroup of $G$ is metabelian.
\item $G/{\mathbf{O}}_{p,q}(G)$ has $q$-length at most $1.$
\end{enumerate}

\end{lemma}

Fix a prime $p$. Let $G$ be a finite group and let $H$ be a proper subgroup of $G.$ For brevity, we say that the pair $(G,H)$ has property $\mathcal{D}_p$ if $x^G\cap H$ is not empty for all $p$-regular elements $x\in G$ or equivalently all $H$-derangements of $G$ have order divisible by $p.$

The following slightly generalizes Lemma 4.2 in \cite{BT}.

\begin{lemma}\label{lem:normal}
Let $H$ be a  proper subgroup of a finite group $G$ and $L$ be a normal subgroup of $G$ such that $G=HL$.  If $T$ is a proper subgroup of $L$ containing $H\cap L$,
then $\Delta_T(L)\subseteq \Delta_H(G)$.
\end{lemma}

\begin{proof}
Let $x\in \Delta_T(L)$ and assume that $x\not\in\Delta_H(G)$. Then  $x^{g}\in H$ for some $g\in G.$ Since $x\in L\unlhd G,$ we  have $x^{g}\in L,$ so $x^{g}\in H\cap L\le T.$  As $g\in G=HL=LH$, we can write $g=lh$ with $h\in H$ and $l\in L.$ Then $x^g=x^{lh}=(x^l)^h\in H$ which implies that $x^l\in H$ and since both $x$ and $l$ are in $L,$ we obtain that $x^l\in H\cap L\le T,$ which is a contradiction.
\end{proof}

We collect in the next lemma some properties of finite groups satisfying $\mathcal{D}_p$.

\begin{lemma}\label{D-property} Let $p$ be a prime, $G$ be a finite group and $H$ be a subgroup of $G.$ Let $L\unlhd G.$
\begin{enumerate}[$(1)$]
\item  If $G=HL$ and $(G,H)$ satisfies $\mathcal{D}_p$ then so does $(L,H\cap L).$

\item If $L$ is a $p$-group or $p'$-group, $(G,H)$ satisfies $\mathcal{D}_p$ and $G\neq HL,$ then $(G/L,HL/L)$ also satisfies $\mathcal{D}_p.$

\item If $H\le K< G$ and $(G,H)$ satisfies $\mathcal{D}_p$, then $(G,K)$ satisfies $\mathcal{D}_p.$

\item If $L\unlhd G$ such that $L\leq H$ and $(G/L,H/L)$ satisfies $\mathcal{D}_p$ then $(G,H)$ satisfies $\mathcal{D}_p.$
\end{enumerate}
\end{lemma}

\begin{proof}
(1) This follows immediately since $\Delta_{H\cap L}(L)\subseteq \Delta_H(G)$ by Lemma \ref{lem:normal}

\medskip
(2) Let $\overline{G}=G/L.$ Since $G\neq HL,$ we see that $\overline{H}$ is a proper subgroup of $\overline{G}.$ Let $\overline{x}\in\Delta_{\overline{H}}(\overline{G}).$ Then $\overline{x}^{\overline{G}}\cap \overline{H}=\emptyset$ which implies that $x^G\cap HL=\emptyset$ and thus $x^G\cap H=\emptyset$ or $x\in\Delta_H(G)$ so $p$ divides $|x|,$ the order of $x.$

If $L$ is a $p'$-group, then the order of $\overline{x}$ must be divisible by $p$ and we are done. So, assume that $L$ is a $p$-group and that the order of $\overline{x}$, say $n,$ is indivisible by $p$. As $L$ is a $p$-group and $p\nmid n,$ we see that $|x|=p^an$ for some integer $a.$ There exist integers $u,v$ such that $1=up^a+vn.$ Hence $x=(x^{p^a})^u(x^n)^v$ where $x^n\in L.$ Thus $\overline{x}=\overline{y}$ with $y=x^{up^a}$ and $|y|=n.$ We now have that $\overline{y}^{\overline{G}}\cap \overline{H}=\emptyset$ and thus $y^G\cap HL=\emptyset$ so $y\in\Delta_H(G)$ and hence $p$ divides $|y|,$ which is a contradiction.

\medskip

(3) is obvious since $\Delta_K(G)\subseteq \Delta_H(G).$

\medskip
(4) Let $x\in\Delta_H(G).$ Then $x^G\cap H=\emptyset$ and thus $\overline{x}^{\overline{G}}\cap \overline{H}=\emptyset$ since $\overline{H}=H/L.$ So $p$ divides $|x|$ as  $(G/L,H/L)$ satisfies $\mathcal{D}_p$. Clearly $p$ must divide $|x|$ and we are done.
\end{proof}

\iffalse
The following question might be of some interest:
\begin{question} Classify all finite groups $G$ with a proper subgroup $H$ such that all $H$-derangements of $G$ have order divisible by some fixed prime $p.$
\end{question}
\fi
We now apply Lemma \ref{D-property} for $H={\mathbf{N}}_G(Q),$ where $Q$ is a Sylow $q$-subgroup of $G.$ The next lemma asserts that the condition `${\mathbf{N}}_G(Q)$ meets every $p$-regular class of a finite group $G$'  is inherited to normal subgroups.

\begin{lemma}\label{lem6}
Let $p$ and $q$ be distinct primes. Let $Q$ be a Sylow $q$-subgroup of $G$ and let $L\unlhd G$. Suppose that $x^G\cap {\mathbf{N}}_G(Q)\neq\emptyset$ for all $p$-regular elements $x$ of $G.$ Then $x^L\cap {\mathbf{N}}_L(Q\cap L)\neq\emptyset$ for all $p$-regular elements $x$ of $L.$ In particular, if $Q\leq L,$ then $x^L\cap {\mathbf{N}}_L(Q)\neq\emptyset$ for all $p$-regular elements $x$ of $L.$
\end{lemma}

\begin{proof} Let $H={\mathbf{N}}_G(Q)$ and $U=Q\cap L.$ Then $U\unlhd Q\unlhd H\le {\mathbf{N}}_G(U)$ and $U\in{{\mathrm {Syl}}}_q(L).$ Since $L\unlhd G,$ we have $G={\mathbf{N}}_G(U)L$ by Frattini's argument.

If $U\unlhd L,$ then the conclusion is trivially true. So, we may assume that ${\mathbf{N}}_L(U)$ is a proper subgroup of $L$ which implies that both $H$ and ${\mathbf{N}}_G(U)$ are proper subgroups of $G.$ It suffices to show that the pair $(L,{\mathbf{N}}_L(U))$ satisfies $\mathcal{D}_p.$

Clearly, $(G,H)$ satisfies $\mathcal{D}_p$ by the hypothesis, so $(G,{\mathbf{N}}_G(U))$ satisfies $\mathcal{D}_p$ by Lemma \ref{D-property}$(3).$ Now part (1) of Lemma \ref{D-property} implies that $(L,L\cap {\mathbf{N}}_G(U))$ satisfies ${{\mathcal D}}_p$ or $(L, {\mathbf{N}}_L(U))$ satisfies ${{\mathcal D}}_p$ as wanted.
\end{proof}

We next prove Theorem A under the additional hypothesis that $G = Q{\mathbf{O}}_{q'} (G)$ where $Q$ is a Sylow $q$-subgroup of $G$.

\begin{lemma}\label{lem3}
 Let $p$ and $q$ be distinct primes and let $Q\in{{\mathrm {Syl}}}_q(G)$. Suppose that $G=Q{\mathbf{O}}_{q'}(G)$ and that $q\nmid \varphi(1)$ for all $\varphi\in\operatorname{IBr}(G).$ Let $K={\mathbf{O}}_{q'}(G)$ and $H={\mathbf{N}}_G(Q).$ Then
 \begin{enumerate}
 \item $Q$ is abelian and $x^K\cap{\mathbf{C}}_K(Q)\neq\emptyset$ for all $p$-regular elements $x\in K;$
 \item $x^G\cap {\mathbf{N}}_G(Q)$ is non-empty for all $p$-regular elements $x\in G$.
 \end{enumerate}
\end{lemma}

\begin{proof}
Assume first that $q\nmid \varphi(1)$ for all $\varphi\in\operatorname{IBr}(G).$  Since $G=QK,$ with $K\unlhd G$ and $Q\cap K=1,$ we deduce that $H=Q{\mathbf{C}}_K(Q).$

\medskip
$(0)$ $Q$ is abelian. This follows from Lemma \ref{lem2}$(ii).$

\medskip

$(1)$ Every $\theta\in\operatorname{IBr}(K)$ is $Q$-invariant. Let $\theta\in\operatorname{IBr}(K)$ and let $T=I_G(\theta).$ By Clifford correspondence \cite[Theorem~8.9]{Navarro}, if $\psi\in\operatorname{IBr}(T\vert\theta),$ then $\psi^G\in\operatorname{IBr}(G)$ and so $\psi^G(1)=|G:T|\psi(1).$ As $K\unlhd T\le G$ and $q\nmid \psi^G(1),$ we must have that $T=G$ or equivalently $\theta$ is $G$-invariant and hence it is $Q$-invariant.

\medskip
$(2)$ $Q$ stabilizes all $p$-regular conjugacy classes of $K.$ From $(1),$ $Q$ stabilizes all irreducible $p$-Brauer characters of $K$ and hence it stabilizes all the projective indecomposable characters $\Phi_\mu$ associated with $\mu\in\operatorname{IBr}(K).$ From \cite[Theorem~2.13]{Navarro}, the set \[\{\Phi_\mu \mid \mu \in \operatorname{IBr}(K)\}\] is a basis of the space of complex functions of $K$ vanishing off the set $K^\circ$ of all $p$-regular elements of $K.$ Therefore $Q$ stabilizes the characteristic functions of the $p$-regular conjugacy classes of $K$ and thus $Q$ stabilizes all the $p$-regular conjugacy classes of $K.$

\medskip
$(3)$ If $x\in K$ is $p$-regular, then $x^K\cap {\mathbf{C}}_K(Q)\neq\emptyset.$ Let ${{\mathcal C}}$ be a $p$-regular class of $K.$  By $(2),$ $Q$ acts coprimely on ${{\mathcal C}}$ and also $K$ acts transitively on ${{\mathcal C}}.$ By  Corollary ~1 of Theorem~4 in \cite{Glauberman}, we obtain that ${{\mathcal C}}\cap {\mathbf{C}}_K(Q)\neq\emptyset.$ This proves $(i).$

\medskip
$(4)$ $y^G\cap H\neq\emptyset$ for every $p$-regular element $y\in G.$ Let $y$ be a $p$-regular element of $G$. If $y\in K,$ then $y^K\cap {\mathbf{C}}_K(Q)\neq\emptyset$ so $y^G\cap H\neq\emptyset$ and we are done. So, we can assume that $y\not\in K.$

We can write $y=y_1y_2,$ where  $y_1$ and $y_2$ are powers of $y$ and  $y_1$ is chosen to have $q$-power order and $y_2$ is chosen to have $q'$-order. Replacing $Q$ by its conjugate if necessary, we can assume $y_1\in Q$, $y_2\in K$. Since $y$ is $p$-regular, we see that $y_2\in K$ is also $p$-regular and thus $y_2^k\in{\mathbf{C}}_K(Q)$ for some $k\in K.$ Hence $Q^{k^{-1}}\leq {\mathbf{C}}_G(y_2).$ By Sylow theorem, there exists $l\in {\mathbf{C}}_G(y_2)$ such that $Q^{k^{-1}}=Q^l$ so that $Q^{lk}=Q$ or equivalently $lk\in H.$

Since $y=y_1y_2$ and $y_1\in H,$ we deduce that $y_1^{lk}\in H.$ Moreover, as $l\in{\mathbf{C}}_G(y_2),$ we have $y_2^{lk}=y_2^k\in {\mathbf{C}}_K(Q)\leq H.$ Therefore,  $$y^{lk}=y_1^{lk}y_2^{lk}\in H.$$
So, we have shown that $y^G\cap H\neq\emptyset$ for all $p$-regular elements $y\in G.$ This completes the proof of $(ii).$
\end{proof}

We are now ready to prove Theorem A which we restate here.

\begin{theorem}
Let $p$ and $q$ be distinct primes and let $G$ be a finite $p$-solvable group. Suppose that $q\nmid \varphi(1)$ for all $\varphi\in\operatorname{IBr}(G).$ Then $x^G\cap {\mathbf{N}}_G(Q)\neq\emptyset$ for all $p$-regular elements $x\in G,$ where $Q$ is a Sylow $q$-subgroup of $G.$
\end{theorem}

\begin{proof} Let $Q\in{{\mathrm {Syl}}}_q(G)$ and $H={\mathbf{N}}_G(Q).$ Suppose that $q\nmid\varphi(1)$ for all $\varphi\in\operatorname{IBr}(G).$
We proceed by induction on $|G|.$

\medskip
(1) If $N\unlhd G$ is nontrivial, then $y^G\cap HN\neq\emptyset$ for all $p$-regular elements $y\in G.$  Since $G/N$ satisfies the hypothesis of the theorem with $|G/N|<|G|,$ by induction we deduce that ${\mathbf{N}}_{G/N}(QN/N)={\mathbf{N}}_G(Q)N/N=HN/N$ meets all the $p$-regular classes of $G/N.$ It follows that  $y^G\cap HN\neq\emptyset$ for every $p$-regular element $y\in G.$

\medskip
(2) ${\mathbf{O}}_p(G)=1={\mathbf{O}}_q(G)$ and $H_G=1.$  If $H_G\unlhd G$ is nontrivial, then the conclusion of the theorem holds by applying (1). So, we can assume $H_G=1$ and so ${\mathbf{O}}_q(G)=1.$

Suppose that ${\mathbf{O}}_p(G)$ is nontrivial. By (1) again, we see that $y^G\cap H{\mathbf{O}}_p(G)$ is nonempty for every $p$-regular element $y\in G.$ Replacing $y$ by its conjugate if necessary, we can assume $y\in H{\mathbf{O}}_p(G).$ Since $y$ is an element of $p'$-order and $H$ contains a Hall $p'$-subgroup $T$ of $H{\mathbf{O}}_p(G),$ some $H{\mathbf{O}}_p(G)$-conjugate of $y$ lies in $T\subseteq H$ and thus $y^G\cap H$ is nonempty.

\medskip

(3) Let $L={\mathbf{O}}^{q'}(G).$ Then $L=QK$ where $K={\mathbf{O}}_{q'}(L)$ is solvable. Since $L\unlhd G$ and ${\mathbf{O}}_p(G)=1={\mathbf{O}}_q(G)$ by (2), we deduce that ${\mathbf{O}}_p(L)={\mathbf{O}}_q(L)=1$ so that ${\mathbf{O}}_{p,q}(L)=1.$ Therefore, by Lemma \ref{lem2} $L$ is solvable and has $q$-length at most $1.$ Since $L={\mathbf{O}}^{q'}(L),$ we must have that $L=Q{\mathbf{O}}_{q'}(L)$ as wanted.

\medskip
(4) $x^K\cap {\mathbf{C}}_K(Q)\neq\emptyset$ for every $p$-regular element $x\in K.$ This follows from (3) and Lemma \ref{lem3}(i).

\medskip

(5) $y^G\cap H\neq\emptyset$ for every $p$-regular element $y\in G.$ Since $L=QK$ is solvable  by (3) and ${\mathbf{O}}_q(L)=1$ by (2), we see that $K$ is nontrivial and solvable. Notice that $K$ is a solvable normal subgroup of $G.$ Hence $K$ contains a minimal normal subgroup of $G,$ say $N.$ Then $N$ is an elementary abelian $r$-group for some prime $r$ different from both $p$ and $q.$

From (1), we have that $y^G\cap HN\neq\emptyset$ for all $p$-regular elements $y\in G.$ Hence it suffices to show that $y^G\cap H\neq\emptyset$ for all $p$-regular elements $y\in HN.$ Now fix a $p$-regular element $y\in HN.$ Then $y=hn$ for some $h\in H$ and $n\in N.$ If $n=1,$ then $y=h\in H$ and we are done. So we assume that $n$ is nontrivial. By (4), $n^k\in{\mathbf{C}}_K(Q)$ for some $k\in K.$ Hence $Q^{k^{-1}}\le {\mathbf{C}}_{QK}(n)$, and thus by Sylow theorem, $Q^{k^{-1}}=Q^l$ for some $l\in {\mathbf{C}}_{QK}(n)$ so $Q^{lk}=Q$ and hence $lk\in H.$ Since $n^{lk}=n^k$ and $h^{lk}\in H,$ we obtain that  \[y^{lk}=(hn)^{lk}=h^{lk}n^{lk}=h^{lk}n^k\in H.\] Therefore, we have shown that $y^G\cap H$ is not empty.
\end{proof}

\section{Proof of Theorem B}

Let $N\unlhd G$ and let $\theta\in\operatorname{IBr}(N).$ Denote by $\operatorname{IBr}(G \mid \theta)$ the set of all irreducible $p$-Brauer characters of $G$ lying over $\theta.$ We now consider the converse of Lemma \ref{lem3}.

\begin{lemma}\label{lem4} Let $p$ and $q$ be distinct primes and suppose that $G=QK$ where $Q$ is an abelian Sylow $q$-subgroup of $G$ and $K={\mathbf{O}}_{q'}(G).$  If $Q$ is abelian and $x^K\cap{\mathbf{C}}_{K}(Q)\neq\emptyset$ for all $p$-regular elements $x\in K,$ then $q\nmid\beta(1)$ for all $\beta\in\operatorname{IBr}(G).$
\end{lemma}

\begin{proof} As $Q\cong G/K$ is abelian, all $\varphi\in\operatorname{IBr}(G/K)$ are linear. Since $(|K|,|G:K|)=1$, by \cite[Theorem~8.13]{Navarro} it suffices to show that every $\beta\in\operatorname{IBr}(K)$ is $Q$-invariant and thus is $G$-invariant;  hence $\beta$ is extendible to $G$ and thus $q$ divides the degree of no irreducible $p$-Brauer character of $G.$

By Corollary 1 of Theorem 4 in \cite{Glauberman}, $Q$ fixes all the $p$-regular conjugacy classes of $K$ since $Q$ fixes some element in every $p$-regular class of $K.$ Reversing the argument in the proof of Lemma \ref{lem3}, we see that $Q$ fixes all the characteristic functions of the $p$-regular classes of $K$ and thus $Q$ fixes all the indecomposable projective characters $\Phi_\mu$ where $\mu\in\operatorname{IBr}(K).$ Hence $Q$ fixes all the $p$-Brauer characters of $K.$
\end{proof}

When a Sylow $q$-subgroup of $G$ is abelian, we can simplify Theorem B as follows.

\begin{theorem}\label{th:abelian}
Let $p$ and $q$ be distinct primes and suppose that $G$ is $p$-solvable and a Sylow $q$-subgroup $Q$ of $G$ is abelian. Then $q\nmid\varphi(1)$ for all $\varphi\in\operatorname{IBr}(G)$ if and only if the following conditions hold:
\begin{enumerate}[$(1)$]
\item $x^G\cap {\mathbf{N}}_G(Q)\neq\emptyset$ for all $p$-regular elements $x\in G;$
\item ${\mathbf{O}}^{q'}(G)$ is solvable;
\item ${\mathbf{O}}^{q'}(G)/{\mathbf{O}}_{p,q}({\mathbf{O}}^{q'}(G))$ has $q$-length at most $1.$
\end{enumerate}
\end{theorem}

\begin{proof}
The `only if' direction of the theorem holds by combining Lemma \ref{lem2} and Theorem A. We now focus on the `if' part of the theorem.

Suppose that $Q\in{{\mathrm {Syl}}}_q(G)$ is abelian and all conditions (1)-(3) hold.  In view of Lemmas \ref{lem1} and \ref{lem6}, we can assume that $G={\mathbf{O}}^{q'}(G).$ Notice that $Q\leq {\mathbf{O}}^{q'}(G)$ and (1)-(3) hold for ${\mathbf{O}}^{q'}(G).$ As ${\mathbf{O}}_p(G)$ is contained in the kernel of all irreducible $p$-Brauer characters of $G,$ we can assume that ${\mathbf{O}}_p(G)=1.$

As $G$ is $q$-solvable, we have that $${\mathbf{C}}_{G/{\mathbf{O}}_{q'}(G)}({\mathbf{O}}_{q',q}(G)/{\mathbf{O}}_{q'}(G))\le {\mathbf{O}}_{q',q}(G)/{\mathbf{O}}_{q'}(G).$$ Since $Q$ is abelian,  $[Q,{\mathbf{O}}_{q',q}(G)]\leq{\mathbf{O}}_{q'}(G)$ and so $Q\le {\mathbf{O}}_{q',q}(G)$ which implies that $G=Q{\mathbf{O}}_{q'}(G)$ as ${\mathbf{O}}^{q'}(G)=G.$ Now the result follows by Lemma \ref{lem4}.
\end{proof}

Let $M\unlhd G$ and $N\unlhd M.$  We now prove that ${\mathbf{C}}_G(M/N)$ is a subgroup of $G$.  Recall that $${\mathbf{C}}_G(M/N)=\{g\in G\,:\, [g,M]\subseteq N\}.$$ As we mentioned in the Introduction, we do not need to assume that $N$ is normal in $G$.

\begin{lemma}\label{lem5} Let $M\unlhd G$ and let $N\unlhd M.$

\begin{enumerate}[$(1)$]
\item ${\mathbf{C}}_G(M/N)$ is a subgroup of $G.$
\item If $M\le K\le G$ and $K'\leq N,$ then $K\le {\mathbf{C}}_G(M/N).$
\end{enumerate}

\end{lemma}

\begin{proof}
For (1), it suffices to show that ${\mathbf{C}}_G(M/N)$ is closed under multiplication. Observe that if $g_1,g_2\in {\mathbf{C}}_G(M/N)$ and $x\in M,$ then \[[g_1g_2,x]=[g_2,x^{g_1}][g_1,x]\in [g_2,M]\cdot [g_1,M]\subseteq N\cdot N=N.\] So $g_1g_2\in {\mathbf{C}}_G(M/N).$

For (2), suppose that  $M\leq K\le G$ with $K'\le N.$ Then $K\leq {\mathbf{C}}_G(M/N)$ since $[K,M]\le [K,K]=K'\le N.$
\end{proof}

We are now ready to prove Theorem B which we restate here.

\begin{theorem}\label{th:characterization}
Let $p$ and $q$ be distinct primes and suppose that $G$ is $p$-solvable with ${\mathbf{O}}_p(G)=1$. Let $L={\mathbf{O}}^{q'}(G)$ and let $Q\subseteq L$ be a  Sylow $q$-subgroup of $G$. Then $q\nmid\varphi(1)$ for all $\varphi\in\operatorname{IBr}(G)$ if and only if the following conditions hold:
\begin{enumerate}[$(1)$]
\item $x^G\cap {\mathbf{N}}_G(Q)\neq\emptyset$ for all $p$-regular elements $x\in G;$
\item In each $q$-series of $G,$ the $q$-factors are abelian and a Sylow $q$-subgroup of $G$ is metabelian.
\item $L$ is solvable;
\item $L/{\mathbf{O}}_{p,q}(L)$ has $q$-length at most $1;$
\item For every normal subgroup $N$ of ${\mathbf{O}}_q(L)$ with ${\mathbf{O}}_q(L)/N$ cyclic, there exists an element $g\in L$ such that $(Q^g)'\leq N$ and $$(x {\mathbf{O}}_q (L))^{C/{\mathbf{O}}_q(L)} \cap N_{C/{\mathbf{O}}_q (L)} (Q^g/{\mathbf{O}}_q (L)) \neq \emptyset$$ for all $p$-regular elements $x {\mathbf{O}}_q (L) \in C/{\mathbf{O}}_q (L),$ where $$C = {\mathbf{C}}_L ({\mathbf{O}}_q (L)/N).$$
\end{enumerate}
\end{theorem}

\begin{proof}

(\textbf{A}) Suppose that $q\nmid \varphi(1)$ for all $\varphi\in\operatorname{IBr}(G).$ We will show that $(1)-(5)$ hold. From Lemma \ref{lem1}, $q\nmid \varphi(1)$ for all $\varphi\in\operatorname{IBr}(L)$ since $L\unlhd G.$ The first four conditions follow from Lemma \ref{lem2} and Theorem A.

We now prove (5). Since ${\mathbf{O}}_p(G)=1,$ we have ${\mathbf{O}}_p(L)=1$ and so ${\mathbf{O}}_{p,q}(L)={\mathbf{O}}_q(L)$ and thus by (4), $L/{\mathbf{O}}_q(L)$ has a normal $q$-complement $K/{\mathbf{O}}_q(L)$ where $K={\mathbf{O}}_{q,q'}(L).$

Let $N\unlhd {\mathbf{O}}_q(L)$ with ${\mathbf{O}}_q(L)/N$ cyclic. As ${\mathbf{O}}_q(L)$ is an abelian $q$-group by (2), we have $$\operatorname{IBr}({\mathbf{O}}_q(L))=\operatorname{Irr}(L)\cong {\mathbf{O}}_q(L).$$ So, there exists a Brauer character $\theta\in\operatorname{IBr}({\mathbf{O}}_q(L))$ such that $N={{\mathrm {Ker}}}(\theta).$ Let $J=I_L(\theta).$ For every Brauer character $\beta\in\operatorname{IBr}(J \mid \theta)$, we have that $\beta^L\in\operatorname{IBr}(L)$ and $q\nmid \beta^L(1)=|L:J|\beta(1).$ So $J$ must contain a Sylow $q$-subgroup of $L$, hence $Q^g\leq J$ for some $g\in L$ and $q\nmid \beta(1).$ Now $J/{\mathbf{O}}_q(L)$ has a normal $q$-complement $J_1/{\mathbf{O}}_q(L)$ and $J=Q^gJ_1$ so $Q^g\cap J_1={\mathbf{O}}_q(L)$ and $J/J_1\cong Q^g/{\mathbf{O}}_q(L).$ We see that $\theta$ extends to $\lambda\in\operatorname{IBr}(J_1)$ and $\lambda$ is also extendible to $\theta_0\in\operatorname{IBr}(J)$ as $q\nmid \beta(1)$ for every $\beta\in\operatorname{IBr}(J \mid \theta)$. Since $\theta_0$ is linear and $Q^g/{\mathbf{O}}_q(G)$ is abelian, we deduce that $$(Q^g)'\leq {{\mathrm {Ker}}}(\theta_0)\cap {\mathbf{O}}_q(L)\le {{\mathrm {Ker}}}(\theta)=N.$$

\medskip
We next claim that $J={\mathbf{C}}_L({\mathbf{O}}_q(L)/N).$ Let $C:={\mathbf{C}}_L({\mathbf{O}}_q(L)/N).$ Suppose first that $g\in J.$ Then $\theta^g=\theta$ so $\theta^g(x)=\theta(x)$ for all $x\in {\mathbf{O}}_q(L).$ It follows that $gxg^{-1}x^{-1}\in {{\mathrm {Ker}}}(\theta)=N$ for all $x\in{\mathbf{O}}_q(L)$ or equivalently $[g^{-1},M]\subseteq N$ hence $g^{-1}\in C$ and thus $g\in C.$ Conversely, suppose that $g\in C.$ Then $g^{-1}\in C$ and thus $[g^{-1},{\mathbf{O}}_q(L)]\subseteq {{\mathrm {Ker}}}(\theta).$ Then $gxg^{-1}x^{-1}\in {{\mathrm {Ker}}}(\theta)$ for all $x\in {\mathbf{O}}_q(L)$ which implies that $\theta^g=\theta$ or $g\in J.$ Therefore $J={\mathbf{C}}_L({\mathbf{O}}_q(L)/N)$ as wanted.

\medskip
From the previous discussion, we know that $\theta$ extends to $\theta_0\in\operatorname{Irr}(J).$ So $\theta_0\mu\in\operatorname{IBr}(J \mid \theta)$ for all $\mu\in\operatorname{IBr}(J/{\mathbf{O}}_q(L))$ by \cite[Corollary 8.20]{Navarro}. As $(\theta_0\mu)^L\in\operatorname{IBr}(L)$ and $q\nmid (\theta_0\mu)^L(1)=|L:J|\mu(1)$, we deduce that $q\nmid \mu(1)$ for all $\mu \in \operatorname{IBr}(J/{\mathbf{O}}_q(L)).$ By Theorem A, we see that the pair $(J/{\mathbf{O}}_q(L),Q^g/{\mathbf{O}}_q(L))$ satisfies the last part of condition (5).  Notice since $Q^g\le J$ that $Q^g/{\mathbf{O}}_q (L)$ will be a Sylow $q$-subgroup of $J/{\mathbf{O}}_q (L)$.  This completes the proof of condition (5).

\medskip
(\textbf{B})
For the converse, suppose that (1)-(5) hold. We will show that $q\nmid\varphi(1)$ for all Brauer characters $\varphi \in \operatorname{IBr}(G).$ By Lemma \ref{lem1}, we can assume that $G=L={\mathbf{O}}^{q'}(G).$ From (3), $G$ is solvable. As ${\mathbf{O}}_p(G)=1,$ we have ${\mathbf{O}}_{p,q}(G)={\mathbf{O}}_q(G)$ and thus $G={\mathbf{O}}_{q,q',q}(G)$  by (4). It follows from (2) that ${\mathbf{O}}_q(G)$ and $G/{\mathbf{O}}_{q,q'}(G)$ are abelian.

Clearly, $G/{\mathbf{O}}_q(G)$ satisfies the last two conditions of Theorem \ref{th:abelian} since $G={\mathbf{O}}^{q'}(G).$ Furthermore, the first condition of that theorem holds for $G/{\mathbf{O}}_q(G)$  by Lemma \ref{D-property}(2). Thus $q\nmid \beta(1)$ for all $\beta\in\operatorname{IBr}(G/{\mathbf{O}}_q(G))$ by Theorem \ref{th:abelian}.

Now it suffices to show that every $\beta\in\operatorname{IBr}(G)$ which does not contain ${\mathbf{O}}_q(G)$ in its kernel has $q'$-degree. Let $\beta\in\operatorname{IBr}(G)$ be such a character and let $\theta\in\operatorname{IBr}({\mathbf{O}}_q(G))$ be an irreducible constituent of $\beta$ when restricted to ${{\mathbf{O}}_q (G)}$. Let  $N={{\mathrm {Ker}}}(\theta).$ Observe that $\theta$ is a nontrivial character of ${\mathbf{O}}_q(G).$ Since ${\mathbf{O}}_q(G)$ is abelian, the quotient  ${\mathbf{O}}_q(G)/{{\mathrm {Ker}}}(\theta)$ is cyclic.  By (5) we have $(Q^g)'\leq {{\mathrm {Ker}}}(\theta)$ for some $g\in G$ and $$I_G(\theta)/{\mathbf{O}}_q(G)={\mathbf{C}}_G({\mathbf{O}}_q(G)/N)/{\mathbf{O}}_q(G)$$ has a Sylow $q$-subgroup $Q^g/{\mathbf{O}}_q(G)$ satisfying condition (1).  Since $Q^g/{{\mathrm {Ker}}}(\theta)$ is abelian, $\theta$ extends to the Sylow $q$-subgroup of the quotient $J/{\mathbf{O}}_q(G).$  Furthermore, for every prime $r\neq q$ and $R/N$ a Sylow $r$-subgroup of $J/{\mathbf{O}}_q(G),$ we see that $\theta$ extends to $R$ since $(|{\mathbf{O}}_q(G)|,|R:{\mathbf{O}}_q(G)|)=1.$ (See \cite[Theorem 8.23]{Navarro}.)  By \cite[Theorem 8.29]{Navarro}, $\theta$ extends to $\theta_0\in\operatorname{IBr}(J)$ and thus all irreducible constituents of $\theta^J$ have the form $\theta_0\lambda$ where $\lambda\in\operatorname{IBr}(J/{\mathbf{O}}_q(G))$ by \cite[Corollary 8.20]{Navarro}. Applying \cite[Theorem 8.9]{Navarro}, we conclude that $\beta = (\theta_0\eta)^G$ for some $\eta \in \operatorname{IBr}(J/{\mathbf{O}}_q(G))$.

Now, the group $J/{\mathbf{O}}_q(G)$ satisfies all the hypotheses of Theorem \ref{th:abelian}, so $q\nmid \lambda(1)$ for all $\lambda\in\operatorname{IBr}(J/{\mathbf{O}}_q(G))$ and thus $q\nmid (\theta_0\eta)^G (1) = \beta (1)$.

Therefore, we can conclude that $q\nmid \varphi(1)$ for all $\varphi\in\operatorname{IBr}(G).$
\end{proof}

\section{Examples}

In the example below, we show that the solvable assumption on $G$ in Theorem A is necessary.

Let $p$ be a prime. We refer the readers to \cite{Lubeck} for some basic information on modular representation theory of $\operatorname{SL}_2(p^f)$ in defining characteristic $p,$ where $f\ge 1.$

It follows from Remark 4.5 \cite{Lubeck} that every $p$-modular irreducible representation of $\operatorname{SL}_2(p)$ has degree $k+1$ for some $k$ with $0\le k\le p-1$ and for odd $p,$ it is unfaithful if and only if $k+1$ is odd. The $p$-modular irreducible representations of $\operatorname{SL}_2(p^f)$ are then obtained by using Steinberg's tensor product theorem (see \cite[Theorem~2.2]{Lubeck}). It follows that every $p$-Brauer character degree of $\operatorname{SL}_2(p^f)$ is a product of at most $f$ Brauer character degrees of $\operatorname{SL}_2(p).$

Now, if $p=2,$ then all irreducible $p$-Brauer characters of $\operatorname{SL}_2(2^f)$ have $2$-power degrees since every irreducible $p$-Brauer characters of $\operatorname{SL}_2(2)$ has degree $1$ or $2.$

Denote by ${{\mathrm {cd}}}_p(G)$ the degrees of irreducible $p$-Brauer characters of $G.$
Assume $p>2.$ Then ${{\mathrm {cd}}}_p(\operatorname{SL}_2(p))$ consists of all integers from $1$ to $p;$ and ${{\mathrm {cd}}}_p(\operatorname{PSL}_2(p))$ consists of all odd integers from $1$ to $p.$ Finally, ${{\mathrm {cd}}}_p(\operatorname{SL}_2(p^2))$ consists of all integers of the form $ab$ where $a,b\in\{1,2,\cdots,p\}.$

With these results on $p$-Brauer character degrees of $\operatorname{SL}_2(p^f)$ and $\operatorname{PSL}_2(p)$, we have the following.

\begin{example}
Let $p$ and $q$ be distinct primes and let $f\ge 1$ be an integer.
\begin{enumerate}
\item Assume $f\ge 4$, $p=2$, and $q$ is a prime divisor of $2^{f}+1.$ Let $G=\operatorname{SL}_2(2^f)$ and $Q\in{{\mathrm {Syl}}}_q(G).$ Then ${\mathbf{N}}_G(Q)\cong {\rm{D}}_{2(2^f+1)} $ and all irreducible $2$-Brauer characters of $G$ have $2$-power degree.  In particular, $q\nmid \varphi(1)$ for all $2$-Brauer characters $\varphi\in\operatorname{IBr}(G)$, but ${\mathbf{N}}_G(Q)$ contains no element of order $2^f-1.$

\item Let $p\ge 5$ be a prime and let $q$ be a prime divisor of $p^2+1$ such that $q>p.$ Let $G=\operatorname{SL}_2(p^2).$ Then $q$ divides the degree of no irreducible $p$-Brauer character of $G.$ However, ${\mathbf{N}}_G(Q)\cong {\rm{D}}_{p^2+1}\cdot 2$ contains no $p$-regular element of order $p^2-1.$

\item Let $p$ be a prime of the form $2^f\pm 1\ge 17$ and let $G=\operatorname{PSL}_2(p).$ Then all irreducible $p$-Brauer characters of $G$ have odd degree and that the Sylow $2$-subgroup $Q$ of $G$ is maximal in $G.$ Then $2\nmid \varphi(1)$ for all $\varphi\in\operatorname{IBr}(G)$ but ${\mathbf{N}}_G(Q)=Q$ contains no odd $p$-regular element of $G.$
\end{enumerate}
\end{example}

Finally, we present examples of groups that satisfy the conclusion of Theorem A and the conditions of Manz and Wolf, yet have irreducible $p$-Brauer characters whose degrees are divisible by $q$.  We begin by noting that all $\{ p, q \}$-groups trivially satisfy the conclusion of Theorem A since the $p$-regular elements will have $q$-power order and hence necessarily be conjugate to elements of the given Sylow $q$-subgroup.  Also, by Burnside's theorem, we know that any $\{ p, q \}$-group is necessarily solvable.  Thus, it suffices to find a $\{ p, q \}$-group $G$ where in the $q$-series for $G$, the $q$-factors are abelian, the $q$-length of $G/{\bf O}_{p,q} (G)$ is at most $1$, and the Sylow $q$-subgroups are metabelian, and there exists a $p$-Brauer character whose degree is divisible by $q$. A specific example when $p = 3$ and $q = 2$ can be found by taking the semidirect product of $\operatorname{S}_3$ acting on two copies of the Klein $4$-group where the action found in $\operatorname{S}_4$.  Obviously, a Sylow $2$-subgroup is metabelian, the $2$-factors in $2$-series for $G$ will be abelian, and $G$ will have $2$-length $1$.  Finally, it is not difficult to see that there exist irreducible $3$-Brauer characters for $G$ that have degree $6$.  We note that there is nothing particular about $3$ and $2$ that are needed for an example.  We claim that for any two distinct primes $p$ and $q$, the iterated wreath product of ${{\mathbb Z}}_q$ by ${{\mathbb Z}}_p$ and then ${{\mathbb Z}}_q$ again will yield an example, but we leave the details to the reader. 

\begin{thebibliography}{100}

\bibitem{BT}  T.C. Burness, H.P. Tong-Viet, Derangements in primitive permutation groups, with an application to character theory, \emph{Q. J. Math.} \textbf{66} (2015), no. 1, 63--96.

\bibitem{FKS} B. Fein, W. Kantor and M. Schacher, Relative Brauer groups, II, \emph{J. Reine Angew. Math.} \textbf{328} (1981), 39--57.

\bibitem{Glauberman} G. Glauberman, Fixed points in groups with operator groups, \emph{Math. Z.} \textbf{84} (1964), 120--125.

\bibitem{IKLM} I. M. Isaacs, T. M. Keller, M. L. Lewis and A. Moret\'{o}, Transitive permutation groups in which all derangements are involutions, \emph{J. Pure Appl. Algebra} \textbf{207} (2006), 717--724.

\bibitem{Jordan} C. Jordan, Recherches sur les substitutions, \emph{J. Math. Pures Appl.} (Liouville) \textbf{17} (1872), 351--367.

\bibitem{Lubeck}  F. L\"{u}beck, Small degree representations of finite Chevalley groups in defining characteristic, \emph{LMS J. Comput. Math.} \textbf{4} (2001), 135--169.

\bibitem{MW1} O. Manz, T. R.  Wolf, \emph{Representations of solvable groups}, London Mathematical Society Lecture Note Series, \textbf{185}. Cambridge University Press, Cambridge, 1993.

\bibitem{MW2}  O. Manz, T. R. Wolf, Brauer characters of $q^\prime$-degree in $p$-solvable groups, \emph{J. Algebra} \textbf{115} (1988), no. 1, 75--91.

\bibitem{N1} G. Navarro, {Variations on the It\^{o}-Michler theorem of character degrees}, \emph{Rocky Mountain J. Math.}, to appear.

\bibitem{Navarro} G. Navarro, {\it Characters and Blocks of  Finite Groups,} LMS Lecture Note Series \textbf{250}, Cambridge University Press, Cambridge, 1998.

\bibitem{Serre} J. P. Serre, On a theorem of Jordan, \emph{Bull. Amer. Math. Soc.} \textbf{40} (2003), 429--440.

\end{thebibliography}

\end{document} 
