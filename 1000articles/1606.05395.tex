\documentclass[1 [leqno,11pt]{amsart}
\usepackage{amssymb, amsmath}
\usepackage{enumerate}
 \setlength{\oddsidemargin}{0mm}
\setlength{\evensidemargin}{0mm} \setlength{\topmargin}{-15mm}
\setlength{\textheight}{220mm} \setlength{\textwidth}{155mm}

\let\pa=\partial
\let\p=\partial
\let\al=\alpha
\let\b=\beta
\let\g=\gamma
\let\e=\varepsilon
\let\ep=\varepsilon
\let\z=\zeta
\let\lam=\lambda
\let\r=\rho
\let\s=\sigma
\let\f=\frac
\let\vf=\varphi
\let\om=\omega
\let{\nabla}=\nabla
\let\G= \Gamma
\let\D=\Delta
\let\Lam=\Lambda
\let\S=\Sigma
\let\Om=\Omega
\let\wt=\widetilde
\let\wh=\widehat
\let\convf=\leftharpoonup
\let\tri\triangle

\newtheorem{defi}{Definition}[section]
\newtheorem{thm}{Theorem}[section]
\newtheorem{lem}{Lemma}[section]
\newtheorem{rmk}{Remark}[section]
\newtheorem{col}{Corollary}[section]
\newtheorem{prop}{Proposition}[section]

\begin{document}
\title[Global regularity of 3-D density patch]
{On the global  regularity of three dimensional density patch for inhomogeneous
incompressible viscous flow}
 \author[X. Liao]{Xian Liao}
\address [X. Liao]{Mathematisches Institut, Universit\"at Bonn,
Endenicher Allee 60, 53115 Bonn, Germany.}
\email{xianliao@math.uni-bonn.de}
\author[Y. Liu]{Yanlin Liu}\address[Y. Liu]
 {Department of Mathematical Sciences, University of Science and Technology of China, Hefei 230026, China.}
\email{liuyanlin3.14@126.com}

\date{\today}
\maketitle

\begin{abstract}
Toward P.-L. Lions' open question  in \cite{Lions96} concerning the propagation of  
regularity for density patch, we prove that the boundary regularity of the 3-D density patch 
persists by time evolution for inhomogeneous incompressible viscous flow, with initial density 
given by $(1-\eta){\bf 1}_{\Om_0}+{\bf 1}_{\Om_0^c}$
for some small enough constant $\eta$ and some $W^{k+2,p}$ domain $\Om_0$, $p\in]3,\infty[$,
and with initial velocity in $W^{1,p}\cap L^3$ satisfying some smallness condition
and appropriate tangential regularities.
\end{abstract}

\noindent\keywords{\sl {Keywords:}} {Inhomogeneous incompressible Navier-Stokes equations; density patch;
striated distributions space}

\noindent {\sl AMS Subject Classification (2000):} 35Q30, 76D03  \

 \setcounter{equation}{0}
\section{Introduction}

We  consider the following inhomogeneous incompressible three dimensional Navier-Stokes equations:
\begin{equation}\label{1.1}
 \left\{\begin{array}{l}
\displaystyle \pa_t\rho+{{\rm div}\,}(\rho v)=0,\qquad (t,x)\in{\mathop{\mathbb R\kern 0pt}\nolimits}^+\times{\mathop{\mathbb R\kern 0pt}\nolimits}^3,\\
\displaystyle\rho(\pa_t v+v\cdot\nabla v)-\Delta v+\nabla\pi=0,\\
\displaystyle {{\rm div}\,} v=0,\\
\displaystyle  (\rho, v)|_{t=0}=(\rho_0, v_0),
\end{array}\right.
\end{equation}
where $\rho\in{\mathop{\mathbb R\kern 0pt}\nolimits}^+,$ $v\in{\mathop{\mathbb R\kern 0pt}\nolimits}^3$ and $\pi\in{\mathop{\mathbb R\kern 0pt}\nolimits}$ stand for the density, velocity field
and pressure of the fluid respectively.
System \eqref{1.1} describes an incompressible fluid
with variable density.
Basic examples are mixture of incompressible and non reactant flows,
 blood flows, models of rivers, fluids containing a melted substance, etc.

The existence and uniqueness issues of the solutions of \eqref{1.1} have been studied by numerous mathematicians.
We just cite here among many others
  \cite{AKM,Lions96, Simon}  for the construction of the weak solutions of \eqref{1.1}, and the works \cite{AP,AKM,CK,D1,LS}  for  the wellposedness result for the strong solutions.
Recently progresses have been made:
the smallness condition on the density fluctuation was successfully removed in e.g. \cite{AGZ2, CHW}, the small jump of the density across some interface was permitted in e.g. \cite{DM12, HPZ} and in the  energy framework, \cite{PZZ} considered the positive density which is only assumed to be bounded from up and below.

In \cite{Lions96} P.-L. Lions proposed the following density patch problem:
if the initial density  $\rho_0=\mathbf{1}_D$ for some smooth domain $D$, then whether or not the boundary regularity of $D$ will persist by time evolution?
In \cite{LZ, LZ2}, the first author and P. Zhang considered this  density patch problem  in space dimension two, away from vacuum.
More precisely in \cite{LZ},   the initial density $\rho_0$ is taken of the following form
\begin{equation}\label{rho0}
{\rho_0}={(1-\eta)\bf 1}_{\Om_0}+ {\bf 1}_{\Om_0^c},
\quad 1-\eta\in{\mathop{\mathbb R\kern 0pt}\nolimits}^+,
\end{equation}
where $|\eta|$ is a sufficiently small constant, and $\Omega_0$ is some  
simply connected $W^{k+2,p}({\mathop{\mathbb R\kern 0pt}\nolimits}^2)$,  $k\geqslant 1,$ $p\in ]2,4[$ bounded domain in ${\mathop{\mathbb R\kern 0pt}\nolimits}^2$.
Let $X_0 \in W^{k+1,p}({\mathop{\mathbb R\kern 0pt}\nolimits}^2)$
 be the divergence-free tangential vector field of ${\partial}\Omega_0$
\footnote{Let $g_0\in W^{k+2,p}({\mathop{\mathbb R\kern 0pt}\nolimits}^2)$ such that $\pa \Omega_0=g_0^{-1}(0)$ and $\nabla g_0$ does not vanish on $\pa \Omega_0$.
Then we can choose $X_0=\nabla^\bot g_0$.}
and denote by
 $\p_{X_0}f{\buildrel\hbox{\footnotesize def}\over =} X_0\cdot{\nabla} f={{\rm div}\,}(fX_0)$ the derivative of $f$ in the direction of $X_0$.
Then   they proved the following persistence result of the boundary regularity:
\begin{thm}\label{thm1.1}
{\sl
Suppose that the initial  velocity $v_0$ of \eqref{1.1}
satisfies the following conormal regularities for some $\epsilon\in ]0,1[$
\begin{equation}\label{LZv0}
\begin{split}
v_0\in W^{1,p}({\mathop{\mathbb R\kern 0pt}\nolimits}^2) {\quad\hbox{with}\quad} \p_{X_0}^\ell v_0\in
 W^{1-\f{\ell}k\epsilon,p}({\mathop{\mathbb R\kern 0pt}\nolimits}^2),
 \quad \ell=1,\cdots, k.
    \end{split}\end{equation}

Then the Cauchy problem \eqref{1.1}-\eqref{rho0}-\eqref{LZv0} has a unique global solution $(\r, v)$ so that
$\r(t,x)={(1-\eta)\bf 1}_{\Om(t)}+{\bf 1}_{\Om(t)^c}$ for some simply connected $W^{k+2,p}({\mathop{\mathbb R\kern 0pt}\nolimits}^2)$ domain $\Om(t)$.
}
\end{thm}

The smallness condition on $|\eta|$ was successfully  removed in \cite{LZ2}, 
with the initial velocity $v_0$, together with its conormal derivatives ${\partial}_{X_0}^\ell v_0$, taken as
$$v_0\in L^2\cap{\dot{B}}^{s_0}_{2,1},
\quad
  {\partial}_{X_0}^\ell v_0\in
L^2\cap{\dot{B}}^{s_\ell}_{2,1}\quad\mbox{for}\  \ell=1,\cdots,k,  $$
 where  $s_0\in ]0,1[,$ $ s_\ell{\buildrel\hbox{\footnotesize def}\over =} s_0-\epsilon_1{ \ell}/{k}$ for some fixed
$\epsilon_1\in ]0,s_0[,$  and $p\in
\bigl]2,{2}/{(1-s_k)}\bigr[.$
Then the authors there took a thorough use of time weighted energy estimates 
to propagate the boundary and the velocity regularity as in Theorem \ref{thm1.1},
and $v\in C({\mathop{\mathbb R\kern 0pt}\nolimits}^+;\dot B^{s_0}_{2,1})$. 

The purpose of this paper is to extend Theorem \ref{thm1.1} to three dimensional case, for the density patch problem with small jump \eqref{rho0} but not in the finite-energy framework.
 One firstly notices that,
the boundary of a two dimensional domain is one dimensional curve, whose tangent space
can be spanned by the tangent vector (e.g. by $X_0$ given above).
  But the boundary
of  three dimensional domain is  two dimensional surface, whose tangent space has dimension two,
hence in order to propagate   the boundary regularity of arbitrarily large order,
 we need to discuss differentiation in different  directions.
 

 Recall that the conormal (or striated) distribution spaces have been successfully used by J.-Y. Chemin \cite{Chemin91, Chemin93} for  studying the evolution of the boundary regularity  of the two-dimensional vortex patch problem for Euler equations (see also \cite{BC93}).
 Then the subsequent works \cite{D97,Hmidi05} considered the viscous case
 and the works \cite{GR,ZQ} extend the result to  three dimensional case.
 {}
We here follow \cite{GR} to   select ``good'' tangent vector directions to work with.
We first adopt the following definition of admissible systems introduced in \cite{GR}:
\begin{defi}[Definition 3.1 of \cite{GR}]\label{def1.1}
{\sl Any system $W=\{W_1,\cdots,W_N\}$ composed of N continuous vector fields is said to be \emph{admissible} if the function
$$[W]^{-1}{\buildrel\hbox{\footnotesize def}\over =}
\Bigl( \frac{2}{N(N-1)}\sum_{\mu<\nu}|W_\mu\wedge W_\nu|^2 \Bigr)^{-\frac 14}$$
is bounded.
Here, for any two vector fields $X=(X^1, X^2, X^3)$, $Y=(Y^1, Y^2, Y^3)$, the notation $X\wedge Y$ is defined as follows
\begin{equation*}
X\wedge Y=\left(
X^2 Y^3-X^3 Y^2,
X^3Y^1-X^1 Y^3,
X^1Y^2-X^2 Y^1
\right)^T.
\end{equation*}
}
\end{defi}

Now we can select five divergence-free tangential vector fields for the two dimensional boundary according to the following result from \cite{GR}:
\footnote{Porposition 3.2 in \cite{GR} concerns the framework of H\"older spaces, however the proof also works in the framework of Sobolev spaces.}
\begin{prop}[Proposition 3.2 of \cite{GR}]\label{prop1.1}
For any $W^{k+2,p}({\mathop{\mathbb R\kern 0pt}\nolimits}^3)$
two dimensional compact submanifold $\Sigma$ of ${\mathop{\mathbb R\kern 0pt}\nolimits}^3$,
we can find an admissible system
consisting of five $W^{k+1,p}({\mathop{\mathbb R\kern 0pt}\nolimits}^3)$, divergence-free vector fields tangent to $\Sigma$.
\end{prop}

Before we go to the statement of density patch problem in three space dimension, let us introduce the following notations which will be used freely in the following context.
For any system $W=\{W_1,\cdots,W_N\}$
composed of $N$ continuous vector fields,
and for any multi-index $\alpha=(\alpha_1,\cdots,\alpha_m)$ of length $m$ with $\alpha_1,\cdots,\alpha_m\in\{1,\cdots,N\}$, we denote
\begin{equation*}\label{1.3}
\pa_W^\alpha=\pa_W^{(\alpha_1,\cdots,\alpha_m)}
=\pa_{W_{\alpha_1}}\cdots\pa_{W_{\alpha_m}}.
\end{equation*}
We emphasize that the order of differentiation is important.
We furthermore denote
\begin{equation*}\label{1.4}
\pa_W^m=\bigl(\pa_W^\alpha\bigr)_\alpha
\end{equation*}
to be an $N^m$ dimensional vector, where $\alpha$ takes over all the  multi-index with length $m$ and with elements taking integer  values between $1$ and $N$.

Now we can state   the density patch problem in ${\mathop{\mathbb R\kern 0pt}\nolimits}^3$.
Let  us take  the initial density $\rho_0$  of the form \eqref{rho0} where
 $\Omega_0$ is a simply connected bounded domain of ${\mathop{\mathbb R\kern 0pt}\nolimits}^3$,
such that its boundary ${\partial}\Omega_0$ is $W^{k+2,p}({\mathop{\mathbb R\kern 0pt}\nolimits}^3),$
  $k\geqslant 1,$ $p\in ]3,\infty[,$ two dimensional compact submanifold in ${\mathop{\mathbb R\kern 0pt}\nolimits}^3$.
By Proposition \ref{prop1.1}, we can find an admissible system
as follows
\begin{equation}\label{W0}
\begin{split}
&W_0=\{X_{1,0},X_{2,0},X_{3,0},X_{4,0},X_{5,0}\},
\\
&\quad\hbox{with } X_{i,0}\in W^{k+1,p}({\mathop{\mathbb R\kern 0pt}\nolimits}^3),
\,k\geq 1, \,p\in ]3,\infty[,
\quad{{\rm div}\,} X_{i,0}=0,
\quad i=1,\cdots,5.
\end{split}
\end{equation}
We assume  the following smallness condition on the initial velocity
\begin{equation}\label{HPZv0}
\begin{split}
 & \|v_0 \|_{{\dot{B}}^{-1+\frac 3{p_1}}_{{p_1},r}} \leq c_0,
 \hbox{ for some } 1<p_1<3
 \hbox{ and } 1<r<\min\{\frac{2{p_1}}{3({p_1}-1)},\, \frac{4p}{2p-3},\,3 \},
    \end{split}
 \end{equation}
as well as the following (conormal) regularities on $v_0$ for some $\varepsilon\in ]0,1[$ \begin{equation}\label{v0}
\begin{split}
&v_0\in W^{1,p} \cap L^3,
\quad\p_{W_0}^\ell v_0\in
 W^{1-\f{\ell}k\varepsilon,p},\quad \ell=1,\cdots, k.
    \end{split}
 \end{equation}

Under the smallness condition on the initial data,  we have the following theorem from \cite{HPZ} to guarantee the existence of  global weak solutions to \eqref{1.1}.
\begin{thm}[Theorem 1.1 of \cite{HPZ}]\label{thm2.1}
{\sl
Let the initial data $(\rho_0, v_0)$ satisfy \eqref{rho0} and \eqref{HPZv0}.
 If $|\eta|, c_0$ are sufficiently small, then \eqref{1.1}  has a  global  weak solution $(\rho, v)$, and there exists a positive constant $C$  such that
\begin{equation}\label{vLr}\begin{split}
\|(\Delta v, \nabla\pi)\|_{L^r({\mathop{\mathbb R\kern 0pt}\nolimits}^+;L^{\frac{3r}{3r-2}})}&+\|\nabla v\|_{L^{2r}({\mathop{\mathbb R\kern 0pt}\nolimits}^+;L^{\frac{3r}{2r-1}})}\leq C c_0.
\end{split}\end{equation}
}
\end{thm}

In the following, we consider the propagation
of the boundary regularity for the  density patch problem.
The main result of this paper states as follows:
\begin{thm}\label{thm1.2}
{\sl Let the initial data $(\rho_0, v_0)$ satisfy \eqref{rho0}-\eqref{W0}-\eqref{HPZv0}-\eqref{v0}, with $|\eta|, c_0$ sufficiently small.
 Then \eqref{1.1} has a unique global
solution $(\r, v)$ so that $\r(t,x)={(1-\eta)\bf 1}_{\Om(t)}+{\bf
1}_{\Om(t)^c}$ for some simply connected $W^{k+2,p}({\mathop{\mathbb R\kern 0pt}\nolimits}^3)$ bounded domain
$\Om(t).$ }
\end{thm}

\begin{rmk}
\begin{itemize}
\item[(i)]According to Theorem 1.1 of \cite{HPZ}, the smallness condition on the initial data \eqref{rho0} and \eqref{HPZv0}
can be replaced by the following more general one
\begin{equation*}
\begin{split}
 \bigl( \|\rho_0-1\|_{L^\infty} +\|v_0^{{\rm h}}\|_{{\dot{B}}^{-1+\frac 3{p_1}}_{{p_1},r}}\bigr)\cdot
\exp\bigl(C_r\|v_0^3\|^{2r}_{{\dot{B}}^{-1+\frac 3{p_1}}_{{p_1},r}}\bigr)\leq c_0.
    \end{split}
 \end{equation*}
Here $C_r$ is some positive constant and  $v_0^{{\rm h}}=(v_0^1, v_0^2)$ denotes the initial horizontal velocity.

 \item[(ii)]We mention that the initial vorticity $\omega_0=c\mathbf{1}_{\Omega_0}$ (i.e. vortex patch)
 with $c$ sufficiently small implies  the assumptions \eqref{HPZv0} and \eqref{v0} on $v_0$.
 Indeed,  noting that $\Omega_0$ is bounded, we have
 $\|\omega_0\|_{L^q}=c|\Omega_0|^{\frac1q},\ \forall q\in]1,\infty[$.
Hence we can take the divergence-free initial velocity $v_0=(-\Delta)^{-1}\nabla\times\omega_0$ such that  $\|\nabla v_0\|_{L^q}  \leq C\frac{q^2}{q-1}\|\omega_0\|_{L^q}$
and
 $v_0\in L^s,\ \forall s\in]\frac32,\infty]$.
In particular,   the smallness condition \eqref{HPZv0} holds true with $p_1=2$
$$\|v_0\|_{\dot B^{\frac 12}_{2,1}}
\leq C\|v_0\|_{W^{1,2}}\leq Cc,$$
provided $c$ is small enough.
On the other side, thanks to $\p_{W_0}^\ell \omega_0\equiv 0$,  the conformal regularities $\p_{W_0}^\ell v_0\in W^{1-\f{\ell}k\varepsilon,p},\ \ell=0,\cdots, k$
were shown in Remark 1.1 of \cite{LZ}.
\end{itemize}\end{rmk}

Recall in  \cite{LZ} that in the two dimensional case, if the velocity of the flow $v\in L^1_{{\rm loc}\,}({\mathop{\mathbb R\kern 0pt}\nolimits}^+, W^{2,p})$, then to prove the persistence of  $W^{k+2,p}({\mathop{\mathbb R\kern 0pt}\nolimits}^2)$ regularity of the domain $\Omega_0$ is equivalent  to show that the tangential vector field $X$ which is transported by the flow from $X_0$    satisfies the following (conormal) regularities:
$$
X,\,{\partial}_X X, \cdots,{\partial}_X^{k-1}X\in W^{2,p}({\mathop{\mathbb R\kern 0pt}\nolimits}^2),
\ \mbox{where}\ {{\rm div}\,} X=0
\hbox{ and } {\partial}_X f{\buildrel\hbox{\footnotesize def}\over =} X\cdot\nabla f={{\rm div}\,}(fX).
$$
Here we consider the system $W(t)=\{X_1(t),\cdots,X_5(t)\}$ satisfying
for any $1\leq i\leq 5$,
 \begin{equation}\label{W}
 \left\{\begin{array}{l}
\displaystyle \pa_t X_i+v\cdot\nabla X_i=X_i\cdot{\nabla} v,\\
\displaystyle X_i(0,x)=X_{i,0}(x),
\end{array}\right. \end{equation}
and we claim that \\
\noindent\textbf{Claim:} Theorem \ref{thm1.2} follows, if the following fact holds true for any $1\leq i\leq 5$,
\begin{equation}\label{claim}
v\in L^1_{{\rm loc}\,}({\mathop{\mathbb R\kern 0pt}\nolimits}^+, W^{2,p}({\mathop{\mathbb R\kern 0pt}\nolimits}^3)),
\quad\pa_W^{\ell-1}X_i\in L^\infty_{{\rm loc}\,}({\mathop{\mathbb R\kern 0pt}\nolimits}^+; W^{2,p}({\mathop{\mathbb R\kern 0pt}\nolimits}^3)),\quad
\forall   \ell\in \{1,\cdots, k\}.
\end{equation}

The rest of this section is contributed to prove this claim and next section is devoted to show  \eqref{claim}.
The procedure of the proofs is similar to the one in \cite{LZ} but we pay  attention on the differences caused by the change of the space dimension.

Assume that \eqref{claim} is correct.
Then  the velocity field $v\in L^1_{{\rm loc}\,}({\mathop{\mathbb R\kern 0pt}\nolimits}^+;{{\rm Lip}\,})$, and the uniqueness result of the weak solutions in Theorem \ref{thm2.1} can be proved by using Lagrangian formulation of \eqref{1.1} as  in \cite{DM12, HPZ}.
We omit the details here.

Let $\psi(t,x)$ be the flow associated with the velocity field $v$, that is,
\begin{equation}\label{psi}
 \left\{\begin{array}{l}
\displaystyle \pa_t\psi(t,x)=v(t,\psi(t,x)),\\
\displaystyle \psi(0,x))=x,
\end{array}\right.
\end{equation}
then by view of \eqref{claim},   we have
\begin{equation*}
\psi(t,\cdot)-Id\in L^\infty_{{\rm loc}\,}({\mathop{\mathbb R\kern 0pt}\nolimits}^+;W^{2,p}),
\end{equation*}
and hence $\Omega(t){\buildrel\hbox{\footnotesize def}\over =}\psi(t,\Omega_0)$ is a $W^{2,p}({\mathop{\mathbb R\kern 0pt}\nolimits}^3)$ domain. Moreover,
the first equation of \eqref{1.1} gives
\begin{equation}\label{rho}
\rho(t,x)={(1-\eta)\bf 1}_{\Om(t)}+{\bf 1}_{\Om(t)^c},
\quad 1-\eta\in {\mathop{\mathbb R\kern 0pt}\nolimits}^+.
\end{equation}

On the other side, there exists a finite number of charts   $\{{\mathbf{V}}^\beta\}_{\beta=1}^m$ covering  the two dimensional $W^{k+2,p}({\mathop{\mathbb R\kern 0pt}\nolimits}^3)$ compact submanifold $\pa\Omega_0$
such that we can parametrize anyone of them, say ${\mathbf{V}}^1$,  as follows
 \begin{equation*}\label{1.6}
\phi_1: {\mathbf{U}}^1\rightarrow   {\mathbf{V}}^1,
\quad\hbox{via}\quad (r,s)\mapsto\phi_1(r,s),
\quad \phi_1\in W^{k+2,p}({\mathbf{U}}^1),
 \end{equation*}
 where ${\mathbf{U}}^1$ is an open set on ${\mathop{\mathbb R\kern 0pt}\nolimits}^2$ and ${\mathbf{V}}^1$ is ${\partial}\Omega_0$-open set in ${\mathop{\mathbb R\kern 0pt}\nolimits}^3$.
In order to show that $\pa\Omega(t)=\psi(t,\pa\Omega_0)\in W^{k+2,p}$,  $k\geq 1$,
it suffices to show (without loss of generality)
\begin{equation*}
\pa_r^{k_1}\pa_s^{k_2}\psi(t,\phi_1(r,s))\in L^\infty_{{\rm loc}\,}(W^{2,p}({\mathbf{U}}^1)),\ \forall k_1+k_2=k.
\end{equation*}
Hence we only need to verify that
\begin{equation}\label{YZ}
(\pa_{Y_0}^{k_1}\pa_{Z_0}^{k_2}\psi)(t,\cdot)\in L^\infty_{{\rm loc}\,}(W^{2,p}({\mathbf{V}}^1)),\ \forall k_1+k_2=k,
\end{equation}
where the tangent vector fields $Y_0, Z_0\in W^{k+1,p}({\mathbf{V}}^1;{\mathop{\mathbb R\kern 0pt}\nolimits}^3)$ are defined by
 $$Y_0(\phi_1(r,s))=\pa_r\phi_1(r,s),
 \quad
 Z_0(\phi_1(r,s))=\pa_s\phi_1(r,s).
 $$
Indeed,   a direct calculation gives for any $(r,s)\in {\mathbf{U}}^1$,
\begin{align*}
&\pa_r\psi(t,\phi_1(r,s))
=Y_0(\phi_1(r,s))\frac{\pa\psi(t,\phi_1(r,s))}{\pa x}
=(\pa_{Y_0}\psi)(t,\phi_1(r,s)),
\\
& \pa_s\psi(t,\phi_1(r,s))
=Z_0(\phi_1(r,s))\frac{\pa\psi(t,\phi_1(r,s))}{\pa x}
=(\pa_{Z_0}\psi)(t,\phi_1(r,s)),
\end{align*}
and hence by an induction argument we achieve
\begin{equation*}\label{5.6}
\pa_r^{k_1}\pa_s^{k_2}\psi(t,\phi_1(r,s))
=(\pa_{Y_0}^{k_1}\pa_{Z_0}^{k_2}\psi)(t,\phi_1(r,s)),
\hbox{ with }\phi_1\in W^{k+2,p}({\mathbf{U}}^1).
\end{equation*}

Since the initial system $W_0$ given in \eqref{W0} is admissible,
for any $(\widetilde r, \widetilde s)\in{\mathbf{U}}^1$, there exists a ${\partial}\Omega_0$-open set ${\mathbf{V}}_0\subset {\mathbf{V}}^1$ containing $\phi_1(\tilde{r},\tilde{s})$   such that
(without loss of generality)
\begin{equation*}\label{5.9}
\inf_{x\in{\mathbf{V}}_0 }|X_{1,0}\wedge X_{2,0}|(x) >0.
\end{equation*}
Thus we can decompose $Y_0$, $Z_0$ as a linear combination of $X_{1,0}$ and $X_{2,0}$ on ${\mathbf{V}}_0$, namely
\begin{equation*}\label{5.10}
Y_0=c_1 X_{1,0}+c_2 X_{2,0},\quad Z_0=d_1 X_{1,0}+d_2 X_{2,0},
\end{equation*}
where the coefficients are defined by
\begin{equation*}\label{5.11}\begin{split}
&c_i=\frac{(Y_0,X_{i,0})|X_{j,0}|^2-(Y_0,X_{j,0})(X_{1,0},X_{2,0})}{|X_{1,0}\wedge X_{2,0}|^2},\\
&d_i=\frac{(Z_0,X_{i,0})|X_{j,0}|^2-(Z_0,X_{j,0})(X_{1,0},X_{2,0})}{|X_{1,0}\wedge X_{2,0}|^2},
\quad i,j=1,2,\, i\neq j.
\end{split}\end{equation*}
Furthermore, in view of the fact $X_{1,0}, X_{2,0}, Y_0, Z_0\in W^{k+1,p}({\mathbf{V}}^1)$, $k\geq 1$, $p>3$,
we know that the coefficients $c_i, d_i$, $i=1,2$ belong to $W^{k+1,p}({\mathbf{V}}_0)$, $k\geq 1$.
Hence without loss of generality,  to prove \eqref{YZ}  reduces to prove
\begin{equation*}\label{5.14}
(\pa_{c_1 X_{1,0}+c_2 X_{2,0}}^{k_1}\pa_{d_1 X_{1,0}+d_2 X_{2,0}}^{k_2}\psi)(t,\cdot)\in L^\infty_{{\rm loc}\,}(W^{2,p}({\mathbf{V}}_0)),\ \forall k_1+k_2=k,
\end{equation*}
and it suffices to show
\begin{equation}\label{5.15}
\pa_{(X_{1,0},X_{2,0})}^{k}\psi\in L^\infty_{{\rm loc}\,}(W^{2,p}({\mathbf{V}}_0)).
\end{equation}
To do this, let us recall the definition \eqref{psi} of the stream function $\psi$,
thus the vector field $X_i(t,x)$ defined by \eqref{W} can be written as
\begin{equation}\label{5.17}
X_i(t,x)=(\pa_{X_{i,0}}\psi)(t,\psi^{-1}(t,x)),
\quad   i=1,2.
\end{equation}
Hence for any function $f(t,x)\equiv g(t,\psi^{-1}(t,x))$, there holds
\begin{equation*}\label{5.18}\begin{split}
(\pa_{X_i}&f)(t,x)=(\pa_{X_{i,0}}\psi)(t,\psi^{-1}(t,x))
\cdot\nabla\psi^{-1}(t,x)\cdot(\nabla g )(t,\psi^{-1}(t,x))\\
&=X_{i,0}(t,\psi^{-1}(t,x))\cdot(\nabla\psi)(t,\psi^{-1}(t,x))
\cdot\nabla\psi^{-1}(t,x)\cdot(\nabla g )(t,\psi^{-1}(t,x))\\
&=X_{i,0}(t,\psi^{-1}(t,x))\cdot(\nabla g )(t,\psi^{-1}(t,x))\\
&=(\pa_{X_{i,0}} g)(t,\psi^{-1}(t,x)),
\quad i=1,2.
\end{split}\end{equation*}
Applying the above formula repeately  on \eqref{5.17} yields that
\begin{equation*}\label{5.19}\begin{split}
(\pa_{(X_1,X_2)}^{\alpha(k-1)}X_{\alpha_k})(t,x)
= (\pa_{(X_{1,0},X_{2,0})}^{\alpha(k)}\psi)\,(t,\psi^{-1}(t,x)),
\end{split}\end{equation*}
for any multi-index $\alpha(k)=(\alpha_1,\cdots,\alpha_k)$ with $\alpha_1,\cdots,\alpha_k\in\{1,2\}$,
and $\alpha(k-1)$ denotes $(\alpha_1,\cdots,\alpha_{k-1})$.
Hence in order to prove \eqref{5.15}, we only need to show
\begin{equation}\label{5.16}
\pa_{(X_{1},X_{2})}^{k-1}(X_{1},X_{2})\in L^\infty_{{\rm loc}\,}\bigl(W^{2,p}(\mathbf{V}(t))\bigr),
\quad \mathbf{V}(t){\buildrel\hbox{\footnotesize def}\over =}\psi(t,{\mathbf{V}}_0).
\end{equation}
As \eqref{claim} guarantees \eqref{5.16},  we conclude that
$\pa\Omega(t)\in W^{k+2,p}({\mathop{\mathbb R\kern 0pt}\nolimits}^3)$ for any $t\in{\mathop{\mathbb R\kern 0pt}\nolimits}^+$ and hence Theorem \ref{thm1.2} follows.
 This completes the proof of the claim.

 

\section{The proof of  \eqref{claim}}\label{sec3}

Similarly as in \cite{LZ},  in order to prove \eqref{claim}, we consider
\begin{equation}\label{J0}
\begin{split}
J_0(t){\buildrel\hbox{\footnotesize def}\over =} 1&+\|({\partial}_t v,\Delta v,\nabla\pi)\|_{L^{r_0}_t(L^p)}
+\|\nabla v\|_{L^{\sigma_r}_t(L^\infty)\cap L^{s_0}_t(L^p)}
 +\|v\|_{L^\infty_t(L^3)\cap L^{\sigma_s}_t(L^\infty)} ,
\end{split}
\end{equation}
and the following inductively defined quantities $J_\ell(t)$,  $ \ell=1,\cdots,k,$
\begin{equation}\label{Jell}
\begin{split}
J_{\ell}(t){\buildrel\hbox{\footnotesize def}\over =}
&J_{\ell-1}(t)
 + \|({\partial}_t {\partial}_W^{\ell } v,\Delta{\partial}_W^{\ell} v, \nabla{\partial}_W^{\ell
}\pi)\|_{L^{r_{\ell }}_t(L^p)}
+\|\nabla{\partial}_W^{\ell } v\|_{L^{r_{\ell }}_t(L^\infty) \cap
L^{s_{\ell }}_t(L^p)} \\
&+\|{\partial}_W^{\ell } v\|_{L^{s_{\ell }}_t(L^\infty)\cap L^\infty_t(L^p)}
+ \|{\partial}_t {\partial}_W^{\ell-1}W\|_{L^{s_{\ell }}_t(W^{1,p})} +
\|{\partial}_W^{\ell-1} W\|_{L^\infty_t(W^{2,p})},
\end{split} \end{equation}
where  $r_\ell, s_\ell, \sigma_r, \sigma_s$ can be taken freely as long as
\begin{equation}\label{r}
r_\ell \in
\bigl]1,\f{2k}{k+\ell\e}\bigr[,
\quad
s_\ell\in
\bigl]2,\f{2k}{\ell\e}\bigr[,
\quad\sigma_r\in
\bigl]\f{2p}{p+3}, \frac {2p}{3}\bigr[,
\,\sigma_s\in\bigl]\frac{4p-6}{p}, \infty[,
\quad  \ell=0,\cdots,k.
\end{equation}
Our aim in this section is to show the following estimates
  \begin{equation}\label{bound}
J_\ell(t)\leq
{{\mathcal H}}_\ell(t){\buildrel\hbox{\footnotesize def}\over =} C_0\underbrace{\exp\cdots\exp}_{\ell\hbox{ times }}(C_0 t),
\quad
\forall \ell=0,\cdots,k,
\quad \forall t\in{\mathop{\mathbb R\kern 0pt}\nolimits}^+,
\end{equation}
where $C_0$ denotes some positive constant which depends only on the initial data and may vary from lines to lines in the following context.

It is obvious that \eqref{claim} can be deduced from the above estimates \eqref{bound}.
The following context  is devoted to the proof of \eqref{bound}.
In particular, Propositions   \ref{prop3.1}, \ref{prop3.2} and \ref{prop4.1} in the following give \eqref{bound} for  the case $\ell=0$, $\ell=1$ and $\ell\geq 2$ respectively.

\smallbreak

Let us first notice the following fact.
 \begin{prop}\label{prop2.1}
{\sl Assume  the same hypothesis as in Theorem \ref{thm2.1} and $v_0\in L^3$.
Then for the global weak solution given in Theorem \ref{thm2.1}, there exists a positive constant $C_0$ such that
\begin{equation}\label{vL3}
\|v(t)\|_{L^3}^3+\|\nabla |v|^{\f32}\|_{L^2_t(L^2)}^2\leq
C_0(\|v_0\|_{L^3}^3+1),
\quad\forall t\in{\mathop{\mathbb R\kern 0pt}\nolimits}^+.
\end{equation}
}
\end{prop}

\begin{proof}
It suffices to prove \eqref{vL3} for smooth solutions of the equation \eqref{1.1}
\footnote{Indeed, we can take approximated smooth solutions $(\rho_n, v_n, \pi_n)$ of the equation \eqref{1.1} accompanied with the mollified initial data $(1+S_n(\rho_0-1), S_nv_0)$ such that \eqref{vL3} holds uniformly in $n$. Then a passage to the limit implies Proposition \ref{prop2.1}.}.
Taking $L^2$ inner product between the momentum equation in \eqref{1.1} with $v|v|$ gives
\begin{equation}\label{2.6}
\f13\f{d}{dt}\|\rho^{\f13}v(t)\|_{L^3}^3
-\int_{{\mathop{\mathbb R\kern 0pt}\nolimits}^3}\Delta v\cdot v|v|dx
+\int_{{\mathop{\mathbb R\kern 0pt}\nolimits}^3}\nabla\pi\cdot v|v|dx=0.
\end{equation}
Integrating by parts, we have
\begin{equation}\label{2.7}
\begin{split}
-\int_{{\mathop{\mathbb R\kern 0pt}\nolimits}^3}\Delta v\cdot v|v|dx
&=\int_{{\mathop{\mathbb R\kern 0pt}\nolimits}^3} |\nabla v|^2 |v|dx
+\int_{{\mathop{\mathbb R\kern 0pt}\nolimits}^3} \nabla |v|\cdot\nabla v\cdot v dx
\\
&\geq \frac12\int_{{\mathop{\mathbb R\kern 0pt}\nolimits}^3} \nabla|v|\cdot \nabla|v|^2 dx
=\|\frac 23\nabla |v|^{\f32}\|_{L^2}^2.
\end{split}
\end{equation}
Applying the three dimensional interpolation inequality that
$$\|f\|_{L^q({\mathop{\mathbb R\kern 0pt}\nolimits}^3)}
\leq C\|f\|_{L^2({\mathop{\mathbb R\kern 0pt}\nolimits}^3)}^{\f3q-\frac 12}\|\nabla f\|_{L^2({\mathop{\mathbb R\kern 0pt}\nolimits}^3)}^{\frac 32-\f3q}
\quad \mbox{for}\quad \forall q\in[2,6],$$
and Young's inequality, we obtain for $r\in [1,3]$
\begin{equation}\label{2.8}
\begin{split}
\int_{{\mathop{\mathbb R\kern 0pt}\nolimits}^3}\nabla\pi\cdot v|v|dx
&\leq\|\nabla\pi\|_{L^{\f{3r}{3r-2}}}\||v|^{\f32}\|_{L^{2r}}^{\f43}\\
&\leq\|\nabla\pi\|_{L^{\f{3r}{3r-2}}}\||v|^{\f32}\|_{L^2}^{2\cdot\f{3-r}{3r}}
\|\nabla |v|^{\f32}\|_{L^2}^{2\cdot\f{r-1}{r}}\\
&\leq\frac 19\|\nabla |v|^{\f32}\|_{L^2}^2
+C\|\nabla\pi\|_{L^{\f{3r}{3r-2}}}^{r}\||v|^{\f32}\|_{L^2}^2+C\|\nabla\pi\|_{L^{\f{3r}{3r-2}}}^{r}.
\end{split}\end{equation}
Substituting \eqref{2.7}, \eqref{2.8} into \eqref{2.6}, we get
\begin{equation}\label{2.9}
\f13\f{d}{dt}\|\rho^{\f13}v(t)\|_{L^3}^3+\frac 13\|\nabla |v|^{\f32}\|_{L^2}^2\leq
C\|\nabla\pi\|_{L^{\f{3r}{3r-2}}}^{r}\|v\|_{L^3}^3+C\|\nabla\pi\|_{L^{\f{3r}{3r-2}}}^{r}.
\end{equation}
Then we take use of Gronwall's inequality,   the estimate \eqref{vLr} and \eqref{rho}
to achieve \eqref{vL3}.
\end{proof}

We will also use frequently the following two lemmas
\begin{lem}\label{lem3.1}
{\sl Let $p\in [\frac 32,\infty[$, $r\in ]1,2[$, $s\in ]2, \infty]$ and $q\in
\bigl]\f{4p}{2p+3}, \frac{4p}{3}\bigr[.$ Let $v_0\in W^{1,p}$ and $
v_L(t){\buildrel\hbox{\footnotesize def}\over =} e^{t\D}v_0.$ Then there exists some positive constant $C$ such that
\begin{equation}\label{3.1}
\|\D v_L\|_{L^{r}_t(L^p)}+\bigl\|{\nabla} v_L\bigr\|_{L^{s}_t(L^p)}+\|{\nabla}
v_L\|_{L^q_t(L^{2p})}\leq C\|v_0\|_{W^{1,p}},
\quad \forall t\in {\mathop{\mathbb R\kern 0pt}\nolimits}^+.
\end{equation}}
\end{lem}

\begin{lem}\label{lem3.2}
{\sl Let $p\in [\frac 32,\infty[$ and $r\in ]1,2[$, then  there exists some positive constant $C$ such that
\begin{equation}\label{3.2}
\begin{split}
\Bigl\| \int^t_0 \Delta e^{(t-t')\Delta}& f(t')\,dt'
\Bigr\|_{L^r_T (L^p)}+
\Bigl\| \int^t_0 \nabla e^{(t-t')\Delta} f(t')\,dt'
\Bigr\|_{L^{\frac{2r}{2-r}}_T (L^p)}\\
&+\Bigl\| \int^t_0 \nabla
e^{(t-t')\Delta} f(t')\,dt' \Bigr\|_{L^{q}_T (L^{2p})}
\leq C\|f\|_{L^r_T(L^p)},
\quad\forall T\in {\mathop{\mathbb R\kern 0pt}\nolimits}^+,
\end{split}
\end{equation} for $q$ given by $\f1q=\f1r- \bigl(\frac 12-\f{3}{4p}\bigr).$}
\end{lem}
\begin{proof}
Lemma \ref{lem3.1} and Lemma \ref{lem3.2}  can be
 proved exactly along the same lines of the proofs of Lemma 4.1, Lemma 4.2 in
\cite{LZ}, and Lemma 7.3 in \cite{LPG}.
Hence we just sketch their proofs for the reader's convenience.

The fact that $v_0\in W^{1,p}$ ensures
$$
\nabla^2 v_0\in \dot B^{-1}_{p,\infty}\cap \dot B^{-2}_{p,\infty}
\hookrightarrow \dot B^{-1-\sigma}_{p,1},
 \quad \nabla v_0\in \dot B^{0}_{p,\infty}\cap \dot B^{-1}_{p,\infty}
\hookrightarrow \dot B^{-\sigma}_{p,1}
\hookrightarrow \dot B^{-\sigma-\frac{3}{2p}}_{2p,1},
\quad \forall \sigma\in ]0,1[.
$$
Hence by the characterisation of the Besov spaces with negative index, we know
$\forall \sigma\in ]0,1[$,
\begin{align*}
\|t^{\frac 12+\frac \sigma2-\frac 1r} e^{t\Delta}\nabla^2 v_0\|_{L^r({\mathop{\mathbb R\kern 0pt}\nolimits}^+;L^p)}
&+
\|t^{\frac \sigma2-\frac 1s} e^{t\Delta}\nabla v_0\|_{L^s({\mathop{\mathbb R\kern 0pt}\nolimits}^+;L^p)}
\\
&+\|t^{\frac \sigma2+\frac{3}{4p}-\frac 1q}e^{t\Delta}\nabla  v_0\|_{L^q({\mathop{\mathbb R\kern 0pt}\nolimits}^+;L^{2p})}
\leq C\|\nabla v_0\|_{W^{1,p}}.
\end{align*}
This together with the fact $\|e^{t\Delta }\nabla v_0\|_{L^p}\leq C\|\nabla v_0\|_{L^p}$
ensures \eqref{3.1}.

To prove \eqref{3.2} it suffices to write
$$
e^{(t-t')\Delta} f(t')=\frac{1}{\sqrt{t-t'}^3}K(\frac{\cdot}{\sqrt{t-t'}})\ast f(t', \cdot),
$$
with $K$ denoting the inverse Fourier transform of $e^{-|\xi|^2}$,
and then to take Young's inequality first in the space variable and  then in the time variable.
\end{proof}

Now we come to the proof of \eqref{bound} and we first consider   $J_0(t)$.
\begin{prop}\label{prop3.1}
{\sl
Assume the hypothesis in Theorem \ref{thm1.2}.
Then the estimate \eqref{bound} holds true for $\ell=0$:
\begin{equation}\label{3.4}
\begin{split}
J_0(t)\leq C_0,
\quad \forall t\in{\mathop{\mathbb R\kern 0pt}\nolimits}^+.
\end{split}
\end{equation}}
\end{prop}

\begin{proof} We follow the idea in the proof of Proposition 4.2 in \cite{LZ}.

We first introduce
$a{\buildrel\hbox{\footnotesize def}\over =} 1/\rho-1$ so that the system \eqref{1.1} translates into the system for the unknowns $(a, v, \nabla\pi)$
\begin{equation}\label{2.1}
 \quad\left\{\begin{array}{lll}
\displaystyle \pa_t a + v \cdot {\nabla} a=0 \qquad (t,x)\in{\mathop{\mathbb R\kern 0pt}\nolimits}^+\times{\mathop{\mathbb R\kern 0pt}\nolimits}^3,  \\
\displaystyle \pa_t v + v \cdot {\nabla} v+ (1+a)({\nabla}\pi-\D v)=0, \\
\displaystyle {\mathop{\rm div}\nolimits}\, v = 0, \\
 \displaystyle (a, v)|_{t=0}=(a_0, v_{0}).
\end{array}\right.
\end{equation}

For any ${\lambda}>0$, for any function $g(t)$,  we denote
\begin{equation*} \label{3.5}
g_{\lambda}(t){\buildrel\hbox{\footnotesize def}\over =}
g(t)\exp\bigl(-{\lambda}\int_0^t V(t')\,dt'\bigr),
\quad \hbox{ with }V(t){\buildrel\hbox{\footnotesize def}\over =}
\|v(t)\|_{L^{2p}}^{\f{4p}{2p-3}}\geq 0.
\end{equation*}
Then by virtue of \eqref{2.1}, $v_{\lambda}$ satisfies
\begin{equation}\label{vlambda}
{\partial}_t v_\lambda+ \lambda f(t)v_\lambda=({\partial}_t v)_\lambda
=\Delta v_\lambda
+(-v\cdot\nabla
v_{\lambda} +a\Delta v_{\lambda}-(1+a)\nabla\pi_{\lambda}),
\end{equation}
and also
\begin{equation} \label{3.6}
v_{\lambda}(t)=e^{t\Delta}v_{0,\lambda} +\int^t_0
e^{-{\lambda}\int_{t'}^t V(t'')\,dt''} e^{(t-t')\Delta}
\Bigl( -v\cdot\nabla
v_{\lambda} +a\Delta v_{\lambda}-(1+a)\nabla\pi_{\lambda} \Bigr)(t')\,dt'.
\end{equation}
Taking space divergence to  \eqref{vlambda} gives
 $$
 \D\pi_{\lambda}=-{\mathop{\rm div}\nolimits}(v\cdot{\nabla}
v_{\lambda})+{\mathop{\rm div}\nolimits}\bigl(a(\D v_{\lambda}-{\nabla}\pi_{\lambda})\bigr),
$$
from which and the fact that
$\|a\|_{L^\infty}=\frac{|\eta|}{1-|\eta|}$ is sufficiently small, we infer
\begin{equation}\label{3.7}
\|{\nabla}\pi_{\lambda}(t)\|_{L^p}\leq C\bigl(\|v\cdot{\nabla}
v_{\lambda}(t)\|_{L^p}+|\eta|\|\D v_{\lambda}(t)\|_{L^p}\bigr).
 \end{equation}
In view of
\eqref{3.6}, we get, by applying
Lemma \ref{lem3.1}, Lemma \ref{lem3.2} and \eqref{3.7},
that \begin{eqnarray*}
\begin{split}
\|\D v_{\lambda}&\|_{L^{r_0}_t(L^p)}
+\|{\nabla} v_{\lambda}\|_{L^{\f{2r_0}{2-r_0}}_t(L^p)}
+\|{\nabla} v_{\lambda}\|_{L^{q_0}_t(L^{2p})}
\\
\leq &\|\D v_L \|_{L^{r_0}_t(L^p)}
+\|{\nabla} v_L \|_{L^{\f{2r_0}{2-r_0}}_t(L^p)}
+\|{\nabla} v_L \|_{L^{q_0}_t(L^{2p})}
\\
&+C\biggl(\int_0^te^{-{\lambda} r_0\int_{t'}^t V(t'')\,dt''}\Bigl(\|v\cdot{\nabla}
v_{\lambda}(t')\|_{L^p}^{r_0}+\|a(t')\|_{L^\infty}^{r_0}\|\D
v_{\lambda}(t')\|_{L^p}^{r_0}\\
&\qquad\qquad\qquad\qquad\qquad\qquad\qquad\quad+\bigl(1+\|a(t')\|_{L^\infty}^{r_0}\bigr)\|{\nabla}\pi(t')\|_{L^p}^{r_0}\Bigr)\,dt'\biggr)^{\f{1}{r_0}}\\
\leq& C\biggl(\|v_0\|_{W^{1,p}}+|\eta|\|\D
v_{\lambda}\|_{L^{r_0}_t(L^p)} +\Bigl(\int_0^te^{-{\lambda}
r_0\int_{t'}^t  V(t'')\,dt''}\|v\cdot{\nabla}
v_{\lambda}(t')\|_{L^p}^{r_0}\,dt'\Bigr)^{\f{1}{r_0}}\biggr),
\end{split}
\end{eqnarray*}
 where $v_L=e^{t\Delta}v_0$,  $\f{1}{q_0}=\f{1}{r_0}- \bigl(\frac 12-\f{3}{4p}\bigr)$
 and $r_0$ is defined in \eqref{r}.
Since
\begin{eqnarray*}
\begin{split}
\Bigl(\int_0^t&e^{-{\lambda} r_0\int_{t'}^t V(t'')\,dt''}\|v\cdot{\nabla}
v_{\lambda}(t')\|_{L^p}^{r_0}\,dt'\Bigr)^{\f{1}{r_0}}\\
\leq
&\Bigl(\int_0^te^{-\f{4p }{2p-3}\lambda
\int_{t'}^t V(t'')\,dt''}\|v(t')\|_{L^{2p}}^{\f{4p}{2p-3}}\,dt'\Bigr)^{\f{2p-3}{4p}}\|{\nabla}
v_{\lambda}\|_{L^{q_0}_t(L^{2p})}
\leq \f{C}{{\lambda} ^{\f{2p-3}{4p}}}\|{\nabla} v_{\lambda}\|_{L^{q_0}_t(L^{2p})}.
\end{split}
\end{eqnarray*}
 we take $|\eta|$ small enough and $\lambda$ large enough  to achieve
\begin{equation} \label {3.9}
\|\D v_{\lambda}\|_{L^{r_0}_t(L^p)}
+\|{\nabla}
v_{\lambda}\|_{L^{\f{2r_0}{2-r_0}}_t(L^p)}
+\|{\nabla}
v_{\lambda}\|_{L^{q_0}_t(L^{2p})}
\leq C\|v_0\|_{W^{1,p}} .
\end{equation}
By use of  the estimates \eqref{vLr} and \eqref{vL3},
we  deduce from the interpolation inequality that
\begin{equation}\label{3.12}
\bigl(\int^t_0 V(t')\,dt'\bigr)^{\frac{2p-3}{4p}}
\equiv\|v\|_{L_t^{\f{4p}{2p-3}}(L^{2p})}
\leq
C\|v\|_{L_t^{\infty}(L^3)}^{1-\theta}\|\Delta v\|_{L_t^r(L^{\f{3r}{3r-2}})}^\theta
\leq C_0,
\end{equation}
where $\theta=\f{(2p-3)r}{4p}\in]0,1[$.
Thus we deduce from \eqref{3.9} that
\begin{equation}\label{3.13}
\begin{split}
\|\D v\|_{L^{r_0}_t(L^p)}
+&\|{\nabla}
v\|_{L^{\f{2r_0}{2-r_0}}_t(L^p)}+\|{\nabla} v\|_{L^{q_0}_t(L^{2p})}\\
\leq& C\|v_0\|_{W^{1,p}}\cdot e^{{\lambda} \int_0^t V(t')\,dt'}\leq C\|v_0\|_{W^{1,p}}\cdot e^{{\lambda}
C_0}\leq C_0.
\end{split} \end{equation}
This together with \eqref{3.7} and \eqref{3.12} ensures that
\begin{equation}\label{3.14}
\begin{split} \|{\nabla}\pi\|_{L^{r_0}_t(L^p)}\leq
C\Bigl(\|v\|_{L^{\f{4p}{2p-3}}_t(L^{2p})}\|{\nabla}
v\|_{L^{q_0}_t(L^{2p})}+|\eta|\|\D v\|_{L^{r_0}_t(L^p)}\Bigr)\leq C_0,
\end{split}
\end{equation}
and hence we deduce from the velocity equation in \eqref{2.1} that
\begin{equation}\label{3.14b}
\begin{split} \|{\partial}_t v\|_{L^{r_0}_t(L^p)} \leq C_0.
\end{split}
\end{equation}

Moreover, for any $p>3,\ r_0\in]1,2[$, there holds
\begin{equation}\label{3.15}
\|\nabla v\|_{L_t^{\f{2pr_0}{2p+3r_0-pr_0}}(L^\infty)}\leq C\|{\nabla}
v\|_{L^{\f{2r_0}{2-r_0}}_t(L^p)}^{1-\f3p}\|\D v\|_{L^{r_0}_t(L^p)}^{\f3p}\leq C_0.
\end{equation}
It is easy to observe that when $r_0$ varies from 1 to 2, we have
$\sigma_r{\buildrel\hbox{\footnotesize def}\over =}\f{2pr_0}{2p+3r_0-pr_0}\in\bigl]\f{2p}{p+3},\f{2p}{3} \bigr[$.
Similarly  we deduce
\begin{equation}\label{3.16}
\|v\|_{L_t^{\f{2p-3}{p}\cdot\f{2r_0}{2-r_0}}(L^\infty)}\leq
C\|v\|_{L_t^{\infty}(L^3)}^{\f{p-3}{2p-3}}\|\nabla v\|_{L_t^{\f{2r_0}{2-r_0}}(L^p)}^{\f{p}{2p-3}}\leq C_0,
\end{equation}
and $\sigma_s{\buildrel\hbox{\footnotesize def}\over =} \f{r_0(4p-6)}{p(2-r_0)} \in\bigl]\f{4p-6}{p},\infty \bigr[$.

By view of \eqref{J0}, we deduce  \eqref{3.4} from the  estimates \eqref{vL3}, \eqref{3.13}, \eqref{3.14}, \eqref{3.14b}, \eqref{3.15}, \eqref{3.16}.
 Hence the proposition follows.
\end{proof}

Next we shall prove the estimate \eqref{bound} for $\ell=1$.
\begin{prop}\label{prop3.2}
{\sl Assume the hypothesis in Theorem \ref{thm1.2}.
Then the estimate \eqref{bound} holds true for $\ell=1$:
\begin{equation}\label{3.22}
\begin{split}
J_{1}(t) \leq{{\mathcal H}}_1(t),
\quad\forall t\in{\mathop{\mathbb R\kern 0pt}\nolimits}^+.
\end{split} \end{equation}
}
\end{prop}
\begin{proof}
This proposition can be proved exactly along the same lines of the proof of Proposition 5.1 in \cite{LZ}.
We sketch the proof here for the reader's convenience.

Firstly, we know from the initial density assumptions \eqref{rho0} and \eqref{W0} that
$$
{\partial}_{W_0}^\ell \rho_0={\partial}_{W_0}^\ell a_0\equiv 0,
\hbox{ with }a_0=\rho_0^{-1}-1, \quad \ell=1,\cdots,k.
$$
On the other side, the equation \eqref{W} ensures that the operators ${\partial}_{X_i}$ and
 $({\partial}_t+v\cdot\nabla)$ are commutative, and hence by view of \eqref{1.1}, ${\partial}_{X_i}^\ell \rho, {\partial}_{X_i}^\ell a$ satisfy the free transport equations
$$
({\partial}_t+v\cdot\nabla) ({\partial}_{X_i}^\ell\rho)=0,
\quad ({\partial}_t+v\cdot\nabla) ({\partial}_{X_i}^\ell a)=0,\quad \forall\ell=1,\cdots,k,
\quad \forall i=1,\cdots,5.
$$
Therefore we derive that
\begin{equation}\label{bound:a}
{\partial}_{W}^\ell \rho\equiv0,
\quad {\partial}_W^\ell a\equiv 0,
\quad \forall \ell=1,\cdots,k.
\end{equation}

We next deduce from the equation \eqref{W} and Proposition \ref{prop3.1} that
\begin{equation}\label{X:W1p}
\begin{split}
\| X_i(t)\|_{W^{1,p}}
\leq
\|X_{i,0}\|_{W^{1,p}}
\exp\Bigl(C\int_0^t\|{\nabla} v(t')\|_{W^{1,p}}\,dt'\Bigr)
\leq {{\mathcal H}}_1(t),
\quad \forall i=1,\cdots,5,
\end{split} \end{equation}
and that
\begin{equation}\label{X:W2p}
\begin{split}
\| X_i(t)\|_{W^{2,p}}
&\leq\Bigl(
\|X_{i,0}\|_{W^{2,p}}+\|\Delta {\partial}_{X_i} v\|_{L^1_t(L^p)} \Bigr)
\exp\Bigl(C\int_0^t\|{\nabla} v(t')\|_{W^{1,p}}\,dt'\Bigr)
\\
&\leq {{\mathcal H}}_1(t)(1+\|\Delta {\partial}_{X_i} v\|_{L^1_t(L^p)} ),
\quad \forall i=1,\cdots,5.
\end{split} \end{equation}

We perform the operator ${\partial}_{X_i}$, $1\leq i\leq 5$ to the velocity equation in \eqref{2.1} to get the equation for ${\partial}_{X_i} v$:
\begin{align}\label{4.13}
{\partial}_t ({\partial}_{X_i} v)
&+v\cdot\nabla({\partial}_{X_i} v)
+(1+a)(\nabla{\partial}_{X_i} \pi-\Delta {\partial}_{X_i} v)\notag
\\
&=F_1(v,\pi,(i)){\buildrel\hbox{\footnotesize def}\over =}
(1+a)\bigl(\nabla X_i\cdot\nabla \pi-\Delta X_i\cdot\nabla v
-2\nabla X_i:\nabla^2 v\bigr),
\end{align}
and hence similar as \eqref{3.6} we write
\begin{align*}
({\partial}_{X_i} v)(t)=
&e^{t\Delta}({\partial}_{X_{i,0}} v_0)
\\
&+\int^t_0 e^{(t-t')\Delta}
\Bigl( -v\cdot\nabla({\partial}_{X_i} v)
+a\Delta ({\partial}_{X_i} v)
-(1+a)\nabla({\partial}_{X_i} \pi)+F_1(v,\pi,(i)) \Bigr)(t')\,dt'.
\end{align*}
Furthermore, a direct calculation shows that $({\partial}_{X_i} \pi)$ satisfies (noticing that ${{\rm div}\,} X_{i}=0$)
\begin{align*}
{{\rm div}\,}((1+a)\nabla({\partial}_{X_i} \pi))
&={{\rm div}\,}\Bigl( -{\partial}_t X_i\cdot\nabla v-{\partial}_t v\cdot\nabla X_i-v\cdot\nabla({\partial}_{X_i} v)
\\
&\qquad
+\Delta X_i\cdot\nabla v+2\nabla X_i:\nabla^2 v
+\Delta v\cdot\nabla X_i+a\Delta({\partial}_{X_i} v)+F_1(v,\pi,(i))\Bigr).
\end{align*}
By use of Proposition \ref{prop3.1} and \eqref{X:W1p} we get the following estimate
\begin{align}\label{bound:pi1}
\|(v\cdot\nabla({\partial}_{X_i}v), &\nabla ({\partial}_{X_i} \pi), F_1(v,\pi,(i)))\|_{L^{r_1}_t(L^p)}
\leq {{\mathcal H}}_1(t)
+C|\eta|\|\Delta ({\partial}_{X_i} v)\|_{L^{r_1}_t(L^p)}\notag
\\
&\quad
+C\Bigl(\int_0^t
 \bigl(\|\nabla v(t')\|_{W^{1,p}}^{r_1}
 + \|(v\otimes\nabla v(t'),
{\nabla}\pi(t'))\|_{L^p}^{r_1} \bigr)
\|\nabla X_i(t')\|_{W^{1,p}}^{r_1}\Bigr)^{\f1{r_1}},
\end{align}
with $r_1$ defined in \eqref{r}.
Therefore noticing $({\partial}_{X_{i,0}} v_0 )\in W^{1-\frac \varepsilon k, p}$,
we take use of Lemma \ref{lem3.1} and Lemma \ref{lem3.2} to achieve
\begin{align*}
&\|({\partial}_t ({\partial}_{X_i} v), \Delta ({\partial}_{X_i} v), \nabla ({\partial}_{X_i} \pi))\|_{L^{r_1}_t(L^p)}
+\|\nabla ({\partial}_{X_i} v)\|_{L^{\frac{2r_1}{2-r_1}}_t(L^p)}
\\
&\leq {{\mathcal H}}_1(t)   +C\Bigl(\int_0^t
 \bigl(\|\nabla v(t')\|_{W^{1,p}}^{r_1}
 + \|(v\otimes\nabla v(t'),
{\nabla}\pi(t'))\|_{L^p}^{r_1} \bigr)
\|\nabla X_i(t')\|_{W^{1,p}}^{r_1}\Bigr)^{\f1{r_1}}.
\end{align*}
We substitute the above estimate into \eqref{X:W2p} and take use of
Proposition \ref{prop3.1} and Gronwall's  inequality to arrive at
$$
\|X_i\|_{L^\infty_t(W^{2,p})}\leq {{\mathcal H}}_1(t),
\quad \forall i=1,\cdots,5.
$$
By the definition \eqref{Jell}, the estimate \eqref{3.22} follows correspondingly.
\end{proof}

Before we consider the estimate \eqref{bound} for the case $\ell\geq 2$, we state the following commutator estimates which can viewed as
generalisations of Lemma 6.1 in \cite{LZ}.
\begin{lem}\label{lem4.1}
{\sl For any $\ell\in\{1,\ldots,k\}$ and any system $W=\{X_1, \cdots, X_N\}$,
 let $\alpha(\ell)=(\alpha_1,\ldots,\alpha_\ell)$ be a
multi-index of length $\ell$ with
 indices taking value in $\{1,\ldots,N\}$,
and we denote
$\widehat\alpha(i)=(\alpha_{\ell-i+1},\ldots,\alpha_\ell)$
for $i=0,\cdots,\ell$.
\\

Then  there exists a positive constant $C$ such that $\forall\,r_\ell \in
\bigl]1,\f{2k}{k+\ell\e}\bigr[$, $s_\ell\in\bigl]2,\f{2k}{\ell\e}\bigr[$, $\forall\,X\in W$,
$\forall 1\leq i\leq\ell$,
\begin{equation}\label{4.3}\begin{split}
&\bigl\|{\partial}_W^{\alpha(i)}\nabla {\partial}_W^{\widehat\alpha(\ell-i)} X-\nabla {\partial}_W^{\alpha(\ell)}
X\|_{L^\infty_t(W^{1,p})}+\bigl\|{\partial}_W^{\alpha(i)}\nabla^2{\partial}_W^{\widehat\alpha(\ell-i)} X-\nabla^2 {\partial}_W^{\alpha(\ell)}
X\bigr\|_{L^\infty_t(L^p)}\\
&\qquad\qquad+\bigl\|{\partial}_W^{\alpha(i)}{\partial}_t{\partial}_W^{\widehat\alpha(\ell-i)}
X-{\partial}_t{\partial}_W^{\alpha(\ell)} X\bigr\|_{L^{s_\ell}_t(W^{1,p})} \leq
CJ_{\ell}^{\ell+1},
\end{split} \end{equation}
and when $i\neq \ell$, one has
\begin{equation} \label{4.4}\begin{split}
&\bigl\|{\partial}_W^{\alpha(i)} \nabla {\partial}_W^{\widehat\alpha(\ell-i)} v-\nabla{\partial}_W^{\alpha(\ell)} v\bigr\|
_{L^{r_{\ell-1}}_t(L^\infty)\cap
L^{s_{\ell-1}}_t(L^p)}\\
&+\bigl\|{\partial}_W^{\alpha(i)} \nabla^2 {\partial}_W^{\widehat\alpha(\ell-i)}
v-\nabla^2{\partial}_W^{\alpha(\ell)} v\bigr\| _{L^{r_{\ell-1}}_t(L^p)}
+\bigl\|{\partial}_W^{\alpha(i)} \nabla {\partial}_W^{\widehat\alpha(\ell-i)} \pi-\nabla{\partial}_W^{\alpha(\ell)}
\pi\bigr\|_{L^{r_{\ell-1}}_t(L^p) }\\
&+\bigl\|{\partial}_W^{\alpha(i)} {\partial}_t{\partial}_W^{\widehat\alpha(\ell-i)}
v-{\partial}_t{\partial}_W^{\alpha(\ell)} v\| _{L^{r_{\ell-1}}_t(L^p) } \leq C
J_{\ell-1}^{\ell+1} ,
\end{split} \end{equation}
and  when $i=\ell$, there holds
\begin{equation}\label{4.5}\begin{split}
&\bigl\|{\partial}_W^{\alpha(\ell)}\nabla v-\nabla{\partial}_W^{\alpha(\ell)}
v +{\partial}_W^{\alpha(\ell-1)}\nabla X_{\alpha_\ell}\cdot\nabla v
\bigr\|_{L^{r_{\ell-1}}_t(L^\infty)\cap L^{s_{\ell-1}}_t(L^p)}
\leq C J_{\ell-1}^{\ell+1},
\\
&\bigl\|{\partial}_W^{\alpha(\ell)} \Delta   v - \Delta{\partial}_W^{\alpha(\ell)} v +{\partial}_W^{\alpha(\ell-1)}
\Delta X_{\alpha_\ell}\cdot\nabla v +2{\partial}_W^{\alpha(\ell-1)}\nabla X_{\alpha_\ell}:\nabla^2
v\bigr\|_{L^{r_{\ell-1}}_t(L^p)} \leq C J_{\ell-1}^{\ell+1},
\\
&\bigl\|{\partial}_W^{\alpha(\ell)} \nabla   \pi-\nabla{\partial}_W^{\alpha(\ell)} \pi
+{\partial}_W^{\alpha(\ell-1)}\nabla X_{\alpha_\ell}\cdot\nabla\pi\bigr\| _{L^{r_{\ell-1}}_t(L^p)
} \leq C J_{\ell-1}^{\ell+1},\\
&\bigl\|{\partial}_W^{\alpha(\ell)} {\partial}_t v-{\partial}_t{\partial}_W^{\alpha(\ell)} v +{\partial}_W^{\alpha(\ell-1)}{\partial}_t
X_{\alpha_\ell}\cdot\nabla v\bigr\| _{L^{r_{\ell-1}}_t(L^p) } \leq C
J_{\ell-1}^{\ell+1}.
\end{split} \end{equation}
Moreover, it follows  from \eqref{Jell}, \eqref{4.3}, \eqref{4.4} and \eqref{4.5} that
for any $0\leq i\leq\ell$,
\begin{equation}\label{4.6}\begin{split}
&\|{\partial}_W^{\alpha(i)}\nabla {\partial}_W^{\widehat\alpha(\ell-i)} X \|_{L^\infty_t(W^{1,p})}
+\|{\partial}_W^{\alpha(i)}\nabla^2{\partial}_W^{\widehat\alpha(\ell-i)} X \|_{L^\infty_t(L^p)}\\
&\qquad\qquad+\|{\partial}_W^{\alpha(i)}{\partial}_t{\partial}_W^{\widehat\alpha(\ell-i)} X \|_{L^{s_{\ell+1}}_t(W^{1,p})} \leq C
J_{\ell+1}^{\ell+1},
\end{split} \end{equation}
and
\begin{equation}\label{4.7}
\begin{split}
&\|{\partial}_W^{\alpha(i)} \nabla {\partial}_W^{\widehat\alpha(\ell-i)} v \| _{L^{r_{\ell }}_t(L^\infty)\cap
L^{s_{\ell}}_t(L^p)} +\|{\partial}_W^{\alpha(i)} \nabla^2 {\partial}_W^{\widehat\alpha(\ell-i)} v
\|_{L^{r_{\ell }}_t(L^p)}
\\
&\qquad\qquad\qquad+\|{\partial}_W^{\alpha(i)} \nabla {\partial}_W^{\widehat\alpha(\ell-i)} \pi \|
_{L^{r_{\ell }}_t(L^p) }+\|{\partial}_W^{\alpha(i)} {\partial}_t{\partial}_W^{\widehat\alpha(\ell-i)} v \|
_{L^{r_{\ell }}_t(L^p) } \leq C J_{\ell}^{\ell+1}.
\end{split} \end{equation}
}
\end{lem}

\begin{proof} The calculation is similar to Lemma 6.1 of \cite{LZ}, so we just sketch it here.

Firstly, for any $X,Y\in W$, it is easy to observe that
\begin{eqnarray*}
\begin{split}
&\|\p_Y{\nabla}
X-{\nabla}\p_YX\|_{L^\infty_t(W^{1,p})}+\|\p_Y{\nabla}^2X-{\nabla}^2\p_YX\|_{L^\infty_t(L^p)}\\
&\qquad\qquad+\|\p_Y\p_tX-\p_t\p_YX\|_{L^{s_1}_t(W^{1,p})}\\
&\leq C\|{\nabla} X\|_{L^\infty_t(W^{1,p})}\bigl(\|{\nabla}
Y\|_{L^\infty_t(W^{1,p})}+\|\p_tY\|_{L^{s_1}_t(W^{1,p})}\bigr)\leq
CJ_1^2.
\end{split}
\end{eqnarray*}
This shows that \eqref{4.3} holds for $\ell=1.$
It is also easy to see that
\eqref{4.4} and \eqref{4.5} hold trivially for $\ell=1.$
Hence Lemma \ref{lem4.1} holds for $k=1.$

Let us now assume that \eqref{4.3}-\eqref{4.7} hold for
$\ell\leq j-1$ with $j\leq k.$ We are going to prove that
\eqref{4.3}-\eqref{4.5} also hold for $\ell=j,$  which will
imply immediately \eqref{4.6}-\eqref{4.7} for $\ell=j.$
In the following,
 for $n\leq m$ and for the $m$-length multi-index $(l_1,\cdots,l_m)$ such that
 $$
 (l_1,\cdots,l_m)\in L^n_m=\{(l_1,\cdots,l_m)\,|\,
 l_1<\cdots<l_n,\,l_{n+1}<\cdots<l_{m},\, \{l_1,\cdots,l_m\}=\{1,\cdots,m\}\},
 $$
  we denote $\alpha^l(n)=(\alpha_{l_1}, \cdots, \alpha_{l_n})$
 and $\widehat\alpha^l(m-n)=(\alpha_{l_{n+1}}, \cdots, \alpha_{l_m})$.

For any positive integer $i\leq j-1$, a direct calculation and the induction assumptions give
\begin{equation}\label{4.8}
\begin{split}
&\bigl\|{\partial}_W^{\alpha(i+1)}\nabla{\partial}_W^{\widehat\alpha(j-i-1)}
X-\nabla{\partial}_W^{\alpha(j)} X\bigr\|_{L^\infty_t(W^{1,p})}\\
&=\bigl\|\sum_{m=0}^i{\partial}_W^{\alpha(m)}
(\pa_{X_{\alpha_{m+1}}}\nabla
-\nabla{\partial}_{X_{\alpha_{m+1}}}){\partial}_W^{\widehat\alpha(j-m-1)}
X\bigr\|_{L^\infty_t(W^{1,p})}\\
&=\bigl\|\sum_{m=0}^i{\partial}_W^{\alpha(m)}\bigl(\nabla X_{\alpha_{m+1}}\cdot\nabla{\partial}_W^{\widehat\alpha(j-m-1)}
X\bigr)\bigr\|_{L^\infty_t(W^{1,p})}\\
&\leq
C\sum_{m=0}^{i}\sum_{n=0}^{m}\sum_{(l_1,\cdots,l_m)\in L^n_m}
 \|{\partial}_W^{\alpha^l(n)}\nabla X_{\alpha_{m+1}}\|_{L^\infty_t(W^{1,p})}
\|\pa_W^{\widehat\alpha^l(m-n)}\nabla{\partial}_W^{\widehat\alpha(j-m-1)}
X\|_{L^\infty_t(W^{1,p})}
\\
&\leq C\sum_{m=0}^{i}\sum_{n=0}^{m}
 J_{n+1}^{n+1}  J_{j-n }^{j-n}
\leq CJ_j^{j+1}.
\end{split} \end{equation}
We follow the same lines as above to obtain
\begin{equation}\label{4.9}
\bigl\|{\partial}_W^{\alpha(i+1)}\pa_t{\partial}_W^{\widehat\alpha(j-i-1)}
X-\pa_t{\partial}_W^{\alpha(j)} X\bigr\|_{L^{s_j}_t(W^{1,p})}\leq CJ_j^{j+1},
\end{equation}
and
\begin{equation}\label{4.10}\begin{split}
&\bigl\|{\partial}_W^{\alpha(i+1)}\nabla^2{\partial}_W^{\widehat\alpha(j-i-1)} X-\nabla^2 {\partial}_W^{\alpha(j)}
X\bigr\|_{L^\infty_t(L^p)}\\
&\leq  C\sum_{m=0}^{i}\sum_{n=0}^{m} \sum_{(l_1,\cdots,l_m)\in L^n_m}
\Bigl( \|{\partial}_W^{\alpha^l(n)}\nabla^2 X_{\alpha_{m+1}}\|_{L^\infty_t(L^{p})}
\|\pa_W^{\widehat\alpha^l(m-n)}\nabla{\partial}_W^{\widehat\alpha(j-m-1)}X\|_{L^\infty_t(L^\infty)}
\\
&\qquad\qquad\qquad
+\|{\partial}_W^{\alpha^l(n)}\nabla X_{\alpha_{m+1}}\|_{L^\infty_t(L^{\infty})}
\|\pa_W^{\widehat\alpha^l(m-n)}\nabla^2{\partial}_W^{\widehat\alpha(j-m-1)}X\|_{L^\infty_t(L^p)}\Bigr)
\\
&\leq C \sum_{m=0}^{i}\sum_{n=0}^{m}
 J_{n+1}^{n+1}  J_{j-n }^{j-n}
\leq CJ_j^{j+1}.
\end{split} \end{equation}
The estimates \eqref{4.8}, \eqref{4.9} and \eqref{4.10} show that \eqref{4.3} holds for $\ell=j$.

The same argument to achieve \eqref{4.3}  for $\ell=j$ yield \eqref{4.4} and \eqref{4.5} for $\ell=j$. We complete the proof of Lemma \ref{lem4.1} by the  induction argument.
\end{proof}

 Now we come to the estimate  \eqref{bound} for the case $\ell\geq 2$:
\begin{prop}\label{prop4.1}
{\sl Assume the hypothesis in Theorem \ref{thm1.2}.
Then the estimate \eqref{bound} holds true for $\ell\geq 2$:
\begin{equation} \label{4.2}
J_\ell(t)\leq {{\mathcal H}}_\ell(t), \quad \forall\ \ell
=2,\cdots, k,
\quad \forall t\in{\mathop{\mathbb R\kern 0pt}\nolimits}^+. \end{equation} }
\end{prop}
\begin{proof}
The proof is similar to the one of Proposition 6.1 in \cite{LZ} and we sketch it.

For $\ell=2,\ldots,k$, and any multi-index $\alpha(\ell)=(\alpha_1,\ldots,\alpha_{\ell})$,
we take the operator $\pa_W^{\alpha(\ell-1)}$ to the equation \eqref{4.13} for $({\partial}_{X_{\alpha_\ell}}v)$  to get
\begin{equation} \label{4.14}\begin{split}
&{\partial}_t{\partial}_W^{\alpha(\ell)} v + v\cdot\nabla {\partial}_W^{\alpha(\ell)} v -  (1+a)\bigl(\Delta{\partial}_W^{\alpha(\ell)} v -
\nabla {\partial}_W^{\alpha(\ell)} \pi\bigr){\buildrel\hbox{\footnotesize def}\over =} F_\ell(v,\pi,\alpha(\ell)).
\end{split} \end{equation}
Here $F_\ell(v,\pi,\alpha(\ell))$ is given by induction
\begin{equation*}\label{4.17}
F_\ell(v,\pi,\alpha(\ell))=\pa_{X_{\alpha_1}}F_{\ell-1}(v,\pi,\widehat\alpha(\ell-1))
+F_{1}(\pa_W^{\widehat\alpha(\ell-1)}v,\pa_W^{\widehat\alpha(\ell-1)}\pi,(\alpha_1)),
\end{equation*}
and hence  by view of  the definition of $F_1$ in \eqref{4.13}  we arrive at
\begin{equation*}\label{4.18}
\begin{split}
&F_\ell(v,\pi,\alpha(\ell))
=\sum_{i=0}^{\ell-1}\pa_W^{\alpha(\ell-1-i)}
F_{1}(\pa_W^{\widehat\alpha(i)}v,\pa_W^{\widehat\alpha(i)}\pi,(\alpha_{\ell-i}))
\\
&=(1+a)\sum_{i=0}^{\ell-1}\pa_W^{\alpha(\ell-1-i)}
\Bigl(  \nabla X_{\alpha_{\ell-i}}\cdot\nabla \pa_W^{\widehat\alpha(i)}\pi
- \Delta  X_{\alpha_{\ell-i}}\cdot\nabla \pa_W^{\widehat\alpha(i)}v
-2\nabla X_{\alpha_{\ell-i}}:\nabla^2 \pa_W^{\widehat\alpha(i)}v
 \Bigr).
\end{split}
\end{equation*}
By use of Lemma \ref{lem4.1}, we arrive at the following estimate
\begin{equation}\label{4.19}
\begin{split}
\|  F_{\ell}(v,\pi,\alpha(\ell))\|_{L^{r_{\ell}}_t(L^p)}
 &\leq
 CJ_{\ell-1}^{\ell+1} \\
&+C\Bigl(\int_0^t
 \bigl(\|\nabla v(t')\|_{W^{1,p}}^{r_\ell}
 + \| {\nabla}\pi(t')\|_{L^p}^{r_\ell} \bigr)
\|\nabla {\partial}_W^{\ell-1}W(t')\|_{W^{1,p}}^{r_\ell}\Bigr)^{\f1{r_\ell}}.
\end{split} \end{equation}

 Similar as the proof of \eqref{bound:pi1} and by an inductive argument
(as the same method for proving Lemma 6.3 in \cite{LZ}), we obtain
\begin{equation}\label{4.20}\begin{split}
 \| {\nabla}\p_W^\ell&\pi\|_{L^{r_{\ell}}_t(L^p)} \leq
 C J_{\ell-1}^{\ell+2}
 +|\eta|\|\D\p_W^\ell
v\|_{L^{r_\ell}_t(L^p)}\\
&\quad+\Bigl(\int_0^t
 \bigl(\|\nabla v(t')\|_{W^{1,p}}^{r_\ell}
 + \|(v\otimes\nabla v(t'),
{\nabla}\pi(t'))\|_{L^p}^{r_\ell} \bigr)
\|\nabla {\partial}_W^{\ell-1}W(t')\|_{W^{1,p}}^{r_\ell}\Bigr)^{\f1{r_\ell}}.
\end{split} \end{equation}

Now noticing $({\partial}_{X_{i,0}}^\ell v_0 )\in W^{1-\frac  \ell k\varepsilon, p}$,
we take use of Lemma \ref{lem3.1}, Lemma \ref{lem3.2} and the estimates \eqref{4.19}, \eqref{4.20} and Proposition \ref{prop3.1} to achieve  \eqref{4.2} by induction argument.
\end{proof}

\smallskip

\noindent {\bf Acknowledgments.} This work was done when we were
visiting Morningside Center of the Academy of Mathematics and
Systems Sciences, CAS. We would like to thank Professor Ping Zhang
for introducing this interesting problem to us.
\medskip

\begin{thebibliography}{9999}

\bibitem{AGZ2} H. Abidi, G. Gui and P. Zhang,
 On the wellposedness of $3-$D inhomogeneous Navier-Stokes equations
in the critical spaces,  {\it   Arch. Ration. Mech. Anal.},  {\bf
204}  (2012), 189--230.

\bibitem{AP} H. Abidi and M. Paicu, Existence globale pour un
fluide inhomog\`ene,  {\it Ann. Inst. Fourier}, {\bf 57}  (2007),
883--917.

\bibitem{AKM}
S.N. Antontsev, A.V. Kazhikhov and V.N. Monakhov,
{\it Boundary value problems in mechanics of nonhomogeneous fluids},
Studies in Mathematics and its Applications 22,
North-Holland Publishing Co., Amsterdam,
1990.

   \bibitem{BC93} A.-L. Bertozzi and P. Constantin,  Global regularity for
vortex patches, {\it Comm. Math. Phys.}, {\bf 152} (1993), 19-28.

\bibitem {Chemin91}
   { J.-Y. Chemin},
  {Sur le mouvement des particules d'un fluide parfait
              incompressible bidimensionnel},
  {\it Invent. Math.},
  {\bf 103} {(1991)},
  {599--629}.

\bibitem {Chemin93}
   { J.-Y. Chemin},
 Persistance de structures g\'eom\'etriques dans les
fluides incompressibles bidimensionnels,  {\it Ann. Sci. \'Ecole
Norm. Sup.}, {\bf  26} (1993), 517-542.

\bibitem{CK}
H. Choe and H. Kim,
Strong solutions of the {N}avier-{S}tokes equations for  nonhomogeneous incompressible fluids,
{\it Comm. Partial Differential Equations},
28(2003), 1183--1201.

\bibitem{CHW}
W. Craig, X. Huang and Y. Wang,
Global wellposedness for the {3D} inhomogeneous incompressible {N}avier-{S}tokes equations,
{\it J. Math. Fluid Mech.}, 15(2013), 747--758.

 \bibitem{D97} R.  Danchin,  Poches de tourbillon visqueuses, {\it J. Math. Pures Appl. (9)},
 {\bf  76} (1997),  609-647.

\bibitem{D1} R. Danchin, Density-dependent incompressible viscous fluids in critical spaces,
{\it  Proc. Roy. Soc. Edinburgh Sect. A}, {\bf 133}  (2003),
1311--1334.

  \bibitem{DM12}
R. Danchin and P.  Mucha,  A Lagrangian approach for the
incompressible Navier-Stokes equations with variable density, {\it
Comm. Pure Appl. Math.}, {\bf 65} (2012),  1458-1480.

\bibitem {HPZ}
   {J. Huang, M. Paicu and P. Zhang},
  {Global well-posedness of incompressible inhomogeneous fluid
              systems with bounded density or non-{L}ipschitz velocity},
 {\it Arch. Ration. Mech. Anal.},
   {\bf 209}
(2013),
 {631--682}.

 \bibitem {GR}
   {P. Gamblin and X. Saint Raymond},
  {On three-dimensional vortex patches},
 {\it Bull. Soc. math. France},
   {\bf 123}
(1995),
 {375--424}.

\bibitem {Hmidi05} T. Hmidi,   {R\'egularit\'e H\"old\'erienne des poches de tourbillon
              visqueuses},
  {\it J. Math. Pures Appl. (9)},
  {\bf 84}
(2005),   {1455--1495}.

\bibitem {Hmidi06} T. Hmidi,  Poches de tourbillon singuli\`eres
 dans un fluide faiblement visqueux, {\it Rev. Mat. Iberoam.}, {\bf 22} (2006),  489-543.

 \bibitem{LS} O.-A.~  Ladyzhenskaja and  V.-A.~ Solonnikov,  The unique solvability
of an initial-boundary value problem for viscous incompressible
inhomogeneous fluids. (Russian) Boundary value problems of
mathematical physics, and related questions of the theory of
functions, 8, {\it Zap. Nau{c}n. Sem. Leningrad. Otdel. Mat. Inst.
Steklov. (LOMI)}, {\bf 52} (1975), 52-109, 218-219.

\bibitem{LZ}
X. Liao and P. Zhang,
{  On the global regularity of  two-dimensional  density patch for inhomogeneous incompressible viscous flow},  {\it Arch. Ration. Mech.
Anal.},  {\bf 220} (2016),  937-981.

\bibitem{LZ2}
X. Liao and P. Zhang,
Global  regularities of two-dimensional density patch for inhomogeneous
incompressible viscous flow with  general density,  arXiv:1604.07922.

\bibitem{Lions96} P.-L.  Lions, {\it Mathematical topics in fluid mechanics. Vol.
1. Incompressible models.} Oxford Lecture Series in Mathematics and
its Applications, {\bf 3}. Oxford Science Publications. The
Clarendon Press, Oxford University Press, New York, 1996.

\bibitem{LPG} P.~G. Lemari{\'e}-Rieusset,
{\it Recent developments in the {N}avier-{S}tokes problem},
  Chapman \& Hall/CRC Research Notes in Mathematics,
    {\bf 431},  2002.

    \bibitem{PZZ}
{M. Paicu, P. Zhang and Z. Zhang},
{Global unique solvability of inhomogeneous {N}avier-{S}tokes equations with bounded density},
{\it Comm. Partial Differential Equations},
{\bf 38} (2013),
{1208-1234}.

\bibitem{ZQ} { P. Zhang  and Q.  Qiu},
 {Propagation of higher-order regularities of the boundaries of
              {$3$}-{D} vortex patches},
 {\it Chinese Ann. Math. Ser. A},
 {\bf 18} {(1997)},
  {381--390}.

\bibitem{Simon} J. Simon, Nonhomogeneous viscous incompressible fluids: existence of velocity, density, and pressure,
 {\em SIAM J. Math. Anal.}, {\bf 21} (1990),
 1093--1117.

\end{thebibliography}

\end{document}

