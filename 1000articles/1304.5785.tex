
\documentclass[10pt]{amsart}
\addtolength\headheight{4pt}
\usepackage{amssymb}
\usepackage{graphicx}
\usepackage[all,cmtip]{xy}
\setlength{\oddsidemargin}{5pt} \setlength{\evensidemargin}{5pt}
\setlength{\textwidth}{440pt}
\setlength{\topmargin}{-50pt}
\setlength{\textheight}{24cm}

\newtheorem{proposition}{Proposition}
\newtheorem{theorem}[proposition]{Theorem}
\newtheorem{definition}[proposition]{Definition}
\newtheorem{lemma}[proposition]{Lemma}
\newtheorem{conjecture}[proposition]{Conjecture}
\newtheorem{corollary}[proposition]{Corollary}
\newtheorem{remark}[proposition]{Remark}
\newtheorem{question}[proposition]{Question}

\hyphenation{me-tric Rie-man-nian}

\begin{document}

\title{A remark on the Reeb Flow for Spheres}

\subjclass[2010]{Primary: 53D10.}

\keywords{contact structures, Reeb flow.}

\author{Roger Casals}
\address{Instituto de Ciencias Matem\'aticas -- CSIC.
C. Nicol\'as Cabrera, 13--15, 28049, Madrid, Spain.}
\email{casals.roger@icmat.es}

\author{Francisco Presas}
\address{Instituto de Ciencias Matem\'aticas -- CSIC.
C. Nicol\'as Cabrera, 13--15, 28049, Madrid, Spain.}
\email{fpresas@icmat.es}

\begin{abstract}
We prove the non--triviality of the Reeb flow for the standard contact spheres ${\mathbb{S}}^{2n+1}$, $n\neq3$, inside the fundamental group of their contactomorphism group. The argument uses the existence of homotopically non--trivial $2$--spheres in the space of contact structures of a $3$--Sasakian manifold. 
\end{abstract}
\maketitle

\noindent Let $(M,\xi)$ be a closed contact manifold. Consider the space ${\mathcal{C}}(M,\xi)$ of contact structures isotopic to $\xi$. This space has been studied in special cases. See \cite{El} for the $3$--sphere and \cite{Bo}, \cite{Ge} for torus bundles. In the present note we prove the non--triviality of its second homotopy group for $3$--Sasakian manifolds, see \cite{BG}. 
\begin{theorem}\label{thm:lin}
Let $(M,\xi)$ be a $3$--Sasakian manifold, then ${\operatorname{rk}}(\pi_2({\mathcal{C}}(M,\xi)))\geq1$.
\end{theorem}
\noindent Let $({\mathbb{S}}^{4n+3},\xi_0=\ker\alpha_0)$ be the standard contact sphere with the standard contact form. The non--trivial spheres in ${\mathcal{C}}({\mathbb{S}}^{4n+3},\xi_0)$ allow us to answer a question posed in \cite{Gi}:\\

\noindent {\it \noindent Remarque 2.10: On peut se demander s'il n'y a pas, dans ${\operatorname{Cont}}({\mathbb{S}}^{2n+1},\xi_0)$, un lacet positif contractile plus simple que dans ${\mathbb{P}} U(n,1)$ et par exemple si le lacet $\rho_t$, $t\in {\mathbb{S}}^1$, n'est pas contractile. C'est peu probable mais je n'en ai pas la preuve.}\\

\noindent The answer we provide is the following
\begin{corollary}\label{cor:Gi}
The class in $\pi_1({\operatorname{Cont}}({\mathbb{S}}^{2n+1},\xi_0))$ generated by the Reeb flow of $\alpha_0$ is a non--trivial element of infinite order for $n\neq3$.
\end{corollary}

\noindent In Section \ref{sec:pre} we introduce the objects of interest and necessary notation. The geometric construction underlying the results is explained in Section \ref{sec:lin}. It is a generalization to higher dimensions of ideas found in \cite{Ge}. Theorem \ref{thm:lin} is concluded. Section \ref{sec:gi} contains the argument deducing Corollary \ref{cor:Gi}. Section \ref{sec:gener} extends the results to higher homotopy groups.\\

\noindent{\bf Acknowledgements}: We are grateful to V. Ginzburg for useful discussions. The first author thanks O. Sp\'a$\check{\mbox{c}}$il for valuable remarks.

\section{Preliminaries}\label{sec:pre}
\subsection{Contact structures}
\begin{definition}
Let $M^{2n+1}$ be a smooth manifold. A codimension--$1$ regular distribution $\xi$ is a contact distribution if there exists a $1$--form $\alpha \in \Omega^1(M)$ such that $\ker \alpha = \xi$ and $\alpha \wedge d\alpha^{n}$ is a volume form.
\end{definition}
\noindent The structure described above is known as a cooriented contact structure. Since the non--coorientable case is not considered in this article, we refer to a cooriented contact structure simply as a contact structure. The smooth manifold $M$ will be assumed to be oriented. The contact structures to be considered will be positively cooriented, i.e. the induced orientation coincides with that prescribed on $M$.\\

\noindent The definition is independent of the choice of $1$--form $\alpha'= e^f \alpha$, for any $f\in C^{\infty}(M, {\mathbb{R}})$. Let ${\operatorname{Cont}}(M,\xi)=\{s\in{\operatorname{Diff}}(M):ds_*\xi=\xi\}$ be the space of diffeomorphisms that preserve the contact structure. These diffeomorphisms are called contactomorphisms. The connected component of the identity of ${\operatorname{Cont}}(M,\xi)$ will be denoted by ${\operatorname{Cont}}_0(M,\xi)$. ${\mathcal{C}}(M,\xi)$ will stand for the space of positive contact structures in $M$ isotopic to $\xi$. The unique vector field $R$ such that
$$i_R \alpha=1,\quad i_R d\alpha=0,$$
is called the Reeb vector field associated to $\alpha$.\\

\noindent A vector field $X\in\Gamma(TM)$ preserves the contact structure if it satisfies the following pair of equations
\begin{eqnarray*}
i_X \alpha & = & H, \\
i_X d\alpha &= & -dH + (i_R dH)\alpha,
\end{eqnarray*}
for a choice of $\alpha$ and a function $H\in C^{\infty}(M,{\mathbb{R}})$. Such a function is called the Hamiltonian associated to the vector field. This correspondance defines a linear isomorphism between the space of vector fields $\Gamma_{\xi}(TM)$ preserving the contact structure $\xi$ and the vector space of smooth functions $C^{\infty}(M, {\mathbb{R}})$. By definition, a contactomorphism $\phi \in {\operatorname{Cont}}_0(M, \xi)$ admits an expression as $\phi=\phi_1$ for a time dependent flow $\{ \phi_t \}_{t\in[0,1]}$ generated by a time dependent family $X_t\in \Gamma_{\xi}(TM)$. Therefore, its flow $\{{\phi}_t\}$ can be generated by a time dependent family of smooth functions $\{H_t \}$.\\

\subsection{Contact fibrations}
\noindent A smooth fibration $\pi: X \longrightarrow B$ is said to be contact for a codimension--$1$ distribution $\xi \subset TX$ if for any fiber $F_p= \pi^{-1}(p)\stackrel{e}{\hookrightarrow}X$, the restriction of the distribution $e^* \xi$ is a contact structure on the fibre. We assume that the distribution $\xi$ is cooriented. Any $\alpha\in\Omega^1(X)$ such that $\xi=\ker\alpha$ will be referred to as a fibration form.\\

\noindent Let $\pi:X \longrightarrow B$ be a smooth fibration. The vertical subbundle $V\subset TX$ is defined fiberwise by $V_x= \ker d\pi(x), \forall x\in X$. An {\em Ehresmann connection} is a smooth choice of a fiberwise complementary linear space $H_x$ for $V_x$ inside $T_xX$. Therefore, the map $d\pi_x: H_x \longrightarrow TB_{\pi(x)}$ is a linear isomorphism and there is a well-defined notion of parallel transport. \\

\noindent There is a canonical connection once a contact fibration $(\pi,\xi=\ker \alpha)$ is fixed. The connection $H$ is defined at a point $x\in X$ to be the annihilator of the vector subspace $V_x \cap \xi_x$ with respect to the quadratic form $(\xi, d\alpha)$. It is complementary to $V_x$ since $V_x \cap \xi_x$ is a symplectic space for the $2$--form $d\alpha$.  The connection is independent of the choice of fibration form $\alpha$. See \cite{Pr} for details on the following facts.
\begin{lemma}
The parallel transport of the canonical connection associated to a contact fibration is by contactomorphisms.
\end{lemma}
\begin{lemma} \label{lem:parallel}
Let $(F, \ker \alpha_0)$ be a closed contact manifold. Let $\pi: F\times{\mathbb{D}}^2 \longrightarrow {\mathbb{D}}^2$ be a contact fibration with fibration distribution defined by the kernel of $\alpha=\alpha_0 + Hd\theta$, for some function $H:F\times{\mathbb{D}}^2 \longrightarrow {\mathbb{R}}$ satisfying $|H|=O(r^2)$. Fix a loop $\gamma:{\mathbb{S}}^1 \longrightarrow{\mathbb{D}}^2$, defined as $\gamma(\theta)=\gamma(r_0, \theta)$ in polar coordinates. Then, the contactomorphism of the fiber $F \times (r_0, 0)$ defined by the parallel transport along $\gamma$ is generated by the family of Hamiltonian functions $\{G_{\theta}(p)=- H(p,r_0, \theta) \}_{\theta \in [0, 2\pi]}$.
\end{lemma}
\noindent Let us study general contact fibrations over a $2$--disk ${\mathbb{D}}^2$. Fix a contact fibration $\pi:X \longrightarrow{\mathbb{D}}^2$ with distribution $\xi=\ker\alpha$. Consider the radial vector field $Y= \partial_r$, defined on ${\mathbb{D}}^2 \setminus \{0\}$. It can be lifted to $X$ by using the canonical contact connection. This produces a vector field $\widetilde{Y}:X\setminus F_0\longrightarrow TX$. Once an angle $\theta_0$ is fixed it can be uniquely extended to $0\in{\mathbb{D}}^2$. In such a case, denote by $\phi_{r,\theta_0}: F_0 \longrightarrow F_{(r, \theta_0)}$ the associated flow at time $r$. It identifies via contactomorphisms the fibers over $0\in{\mathbb{D}}^2$ and over $(r,\theta_0)\in{\mathbb{D}}^2$. Define the diffeomorphism:
\begin{eqnarray*}
\Phi: F_0 \times D^2 & \longrightarrow & X \\
(p,r, \theta) & \longmapsto & \phi_{r, \theta} (p).
\end{eqnarray*}
Then the definition of the contact connection implies $\Phi^* \alpha = e^g(\alpha_0 +H d\theta)$, where $g:M \times D^2 \longrightarrow {\mathbb{R}}$ and $H: M \times D^2 \longrightarrow {\mathbb{R}}$ are arbitrary smooth functions. We can choose as fibration form $\alpha'= e^{-g} \alpha$ and trivialize the fibration using $\Phi$. Then we obtain the expression
\begin{equation}
\Phi^* \alpha' = (\alpha_0 +H d\theta). \label{eq:radial_triv}
\end{equation}
Given a contact fibration over the disk, the trivialization constructed above is called {\it radial}. It is convenient to observe that the radial trivialization construction can be made parametric for families of contact fibrations over the disk.

\subsection{Loops at infinity}
\noindent Fix a contact fibration $\pi:X \longrightarrow {\mathbb{S}}^2$ with distribution $\xi$, fibre $F$ and a point $N\in  {\mathbb{S}}^2$. This point will be referred to as North pole or infinity. Define the restriction fibration $\pi_N: X \setminus \pi^{-1}(N) \longrightarrow {\mathbb{S}}^2 \setminus N \simeq{\mathbb{D}}^2$. Trivialize the contact fibration $\pi_N$ {\it radially} from $S= \{ 0 \} \in{\mathbb{D}}^2$ to obtain a new contact fibration $\hat{\pi}: F \times{\mathbb{D}}^2\longrightarrow{\mathbb{D}}^2$ with fibration form $\alpha_0+ Hd\theta$. Denoting by $\Phi: F\times{\mathbb{D}}^2 \longrightarrow X\setminus \pi^{-1}(N)$ the trivialization map, we obtain $\Phi^* \xi = \ker\{\alpha_0+ Hd\theta\}$. Therefore, the map is connection--preserving. Consider the family of loops
\begin{eqnarray*}
\gamma_r: {\mathbb{S}}^1 & \longrightarrow & D^2 \\
\theta & \longmapsto & (r, \theta).
\end{eqnarray*}
Composing with the embedding ${\mathbb{D}}^2 \hookrightarrow {\mathbb{S}}^2$, for $r \longrightarrow1$, they are smaller and smaller loops around the North pole $N \in {\mathbb{S}}^2$. By Lemma \ref{lem:parallel}, the parallel transport associated to the loop $\gamma_r$ is generated by a family of Hamiltonians $\{ÊG_{\theta}^r \}_{\theta \in {\mathbb{S}}^1}$, defined by $G_{\theta}^r(p)= -H(p,r, \theta)$. The limit function
$$G_{\theta}= \lim_{r\longrightarrow 1} G_{\theta}^r$$
exists because the connection associated to $\xi$ is a smooth connection over ${\mathbb{S}}^2$. It is clear that $\{ G_{\theta} \}$ defines a loop in ${\operatorname{Cont}}(M, \xi_0 = \ker \alpha_0)$. This will be called the {\it loop at infinity} associated to $(\pi, \xi)$. Continuous families of contact fibrations with marked fibre produce continuous families of loops at infinity.

\begin{definition} A contact sphere is a smooth map $e: {\mathbb{S}}^2 \longrightarrow{\mathcal{C}}(M, \xi)$.
\end{definition}
\noindent There is a canonical contact fibration over ${\mathbb{S}}^2$ associated to any contact sphere $e$. It is defined as
$$X= M\times {\mathbb{S}}^2\longrightarrow{\mathbb{S}}^2,$$
with the distribution at $(p,z)\in M \times {\mathbb{S}}^2$ being $\xi^e(p,z)= e(z)_p \oplus T_z {\mathbb{S}}^2 \subset T_pM \oplus T_z {\mathbb{S}}^2$. \\

\noindent Denote by $C^{\infty}({\mathbb{S}}^2, {\mathcal{C}}(M, \xi))$ the space of smooth maps from ${\mathbb{S}}^2$ to ${\mathcal{C}}(M,\xi)$. The smooth loop space of ${\operatorname{Cont}}_0(M,\xi)$ is denoted as $\Omega({\operatorname{Cont}}_0(M,\xi), id)$.
\begin{lemma} \label{lem:north}
The previous construction induces a continuous map
$$ C^{\infty}({\mathbb{S}}^2, {\mathcal{C}}(M, \xi)) \longrightarrow \Omega({\operatorname{Cont}}_0(M,\xi), id). $$
Therefore, it provides a morphism
$$ \pi_2({\mathcal{C}}(M, \xi)) \longrightarrow \pi_1({\operatorname{Cont}}_0(M,\xi)). $$
\end{lemma}

\subsection{Homotopy sequence}
The group ${\operatorname{Diff}}_0(M)$ acts transitively on ${\mathcal{C}}(M,\xi)$ because of Gray's Stability Theorem. It is a Serre fibration with homotopy fibre  ${\operatorname{Cont}}(M,\xi)\cap{\operatorname{Diff}}_0(M)$. This homotopy fibre might be disconnected. Its identity component is denoted by ${\operatorname{Cont}}_0(M,\xi)$. Hence the fibration induces a long exact sequence
\begin{equation} \label{eq:seq}
\ldots\longrightarrow\pi_2({\operatorname{Diff}}_0(M))\longrightarrow\pi_2({\mathcal{C}}(M,\xi))\stackrel{\partial_2}{\longrightarrow} \pi_1({\operatorname{Cont}}_0(M,\xi))\longrightarrow\pi_1({\operatorname{Diff}}_0(M))\longrightarrow\ldots. 
\end{equation}
The map $\partial_2$ is the one provided by Lemma \ref{lem:north}. The study of this sequence will provide Corollary \ref{cor:Gi}. \\

\noindent Note that a geometric lifting map
\begin{equation}
\pi_j({\mathcal{C}}(M,\xi))\stackrel{\partial_j}{\longrightarrow} \pi_{j-1}({\operatorname{Cont}}_0(M,\xi))
\label{eq:lifting}
\end{equation}
can be analogously defined. It provides a geometric representative of the connecting morphism. This generalizes the previous constructions. It will be used in Section \ref{sec:gener}.
\section{Spheres in ${\mathcal{C}}(M,\xi)$}\label{sec:lin}
\subsection{Almost contact structures}
\noindent Let $M$ be an oriented $(2n+1)$--dimensional manifold. Denote by $Dist(M)$ the space of smooth codimension--$1$ regular cooriented distributions on $M$. Concerning orientations, an almost complex structure on a cooriented distribution will be positive if the induced orientation coincides with the prescribed one. Define the space of almost contact structures as
$${\mathcal{A}}(M)=\{(\xi,{\mathfrak{j}}):\xi\in Dist(M),{\mathfrak{j}}\in{\operatorname{End}}(\xi),{\mathfrak{j}}^2=-\mbox{id},{\mathfrak{j}}\mbox{ positive}\}.$$
Given a contact structure $\xi=\ker \alpha$, an almost complex structure ${\mathfrak{j}}\in{\operatorname{End}}(\xi)$ is said to be compatible with $\alpha$ if it is compatible with the symplectic form on the symplectic space $(\xi,d\alpha)$. The space ${\mathcal{A}}(M)$ has a subset defined by
$${\mathcal{A}}{\mathcal{C}}(M, \xi)=\{(\eta,{\mathfrak{j}}):\eta \in {\mathcal{C}}(M, \xi),{\mathfrak{j}}\in{\operatorname{End}}(\eta),{\mathfrak{j}}^2=-\mbox{id}, {\mathfrak{j}} \mbox{ compatible with }\alpha\mbox{ such that }\eta=\ker\alpha\}.$$
The space of almost complex structures compatible with a fixed symplectic form is contractible. Thus, the forgetful map ${\mathcal{A}}{\mathcal{C}}(M, \xi)\longrightarrow{\mathcal{C}}(M, \xi)$ has a contractible homotopy fibre. Hence there exists a homotopy inverse $\i:{\mathcal{C}}(M, \xi)\longrightarrow{\mathcal{A}}{\mathcal{C}}(M, \xi)$ provided by the choice of a compatible almost complex structure on the contact distribution.\\

\noindent Fix a point $p\in M$ and an oriented framing ${\tau}:T_pM\stackrel{\simeq}{\longrightarrow}{\mathbb{R}}^{2n+1}$. 
Define the evaluation map
$$e_{(p,\tau)}:{\mathcal{A}}(M)\longrightarrow{\mathcal{A}}({\mathbb{R}}^{2n+1}),\quad e_{(p,\tau)}(\xi, {\mathfrak{j}})= (\tau_{*}\xi_p, \tau_* {\mathfrak{j}}_p).$$
This is a continuous map and thus induces $\widetilde{e}_{(p,\tau)}:\pi_2({\mathcal{A}}(M))\longrightarrow\pi_2({\mathcal{A}}({\mathbb{R}}^{2n+1}))$. Therefore, we obtain
$$\varepsilon_{(p,\tau)}=\widetilde{e}_{(p,\tau)}\circ\i_*:\pi_2({\mathcal{C}}(M,\xi))\longrightarrow\pi_2({\mathcal{A}}({\mathbb{R}}^{2n+1}))$$
\begin{lemma}
$\pi_2({\mathcal{A}}({\mathbb{R}}^{2n+1}))\cong{\mathbb{Z}}$.
\end{lemma}
\begin{proof}
The space ${\mathcal{A}}({\mathbb{R}}^{2n+1})$ is isomorphic to the homogeneous space $SO(2n+1)/U(n)$. The standard inclusion $SO(2n)\longrightarrow SO(2n+1)$ descends to a map
$$SO(2n)/U(n)\longrightarrow SO(2n+1)/U(n)$$
with homotopy fibre ${\mathbb{S}}^{2n}$. The long exact sequence for a homotopy fibration implies that
$$\pi_2(SO(2n)/U(n))\cong\pi_2(SO(2n+1)/U(n)),\quad n\geq2.$$ It is simple to show that $SO(2n+1)/U(n)$ is also isomorphic to $SO(2n+2)/U(n+1)$. Since $SO(4)/U(2)$ is a $2$--sphere, the statement follows.
\end{proof}
\noindent Thus the evaluation map can be seen as an integer--valued map for $\pi_2({\mathcal{C}}(M,\xi))$.
\begin{lemma}
The map $\varepsilon_{(p,\tau)}:\pi_2({\mathcal{C}}(M,\xi))\longrightarrow\pi_2({\mathcal{A}}({\mathbb{R}}^{2n+1}))$ is independent of the choice of $p$ and $\tau$.
\end{lemma}
\begin{proof}
\noindent Let $p,q\in M$ and $\tau_p$,$\tau_q$ be oriented framings of $T_pM$, $T_qM$ respectively. Consider a continuous path of pairs $\{(p_t,\tau_t)\}$ connecting $(p,\tau_p)$ and $(q,\tau_q)$. The continuous family of maps
$$e_{(p_t,\tau_t)}:{\mathcal{A}}(M)\longrightarrow{\mathcal{A}}({\mathbb{R}}^{2n+1}),\quad e_{(p_t,\tau_t)}(\xi, {\mathfrak{j}})= (\tau_{t_{*}}\xi_p, \tau_{t_*} {\mathfrak{j}}_p)$$
provides a homotopy between $e_{(p,\tau_p)}$ and $e_{(q,\tau_q)}$.
\end{proof}
\subsection{Linear Contact Spheres}
\begin{definition}
A linear contact sphere is a contact sphere $\iota:{\mathbb{S}}^2\longrightarrow{\mathcal{C}}(M,\xi)$ such that there exist three contact forms $(\alpha_0,\alpha_1,\alpha_2)$ satisfying
$$\iota(p)=\ker(e_0\alpha_0+e_1\alpha_1+e_2\alpha_2)$$
for the standard embedding $(e_0,e_1,e_2):{\mathbb{S}}^2\longrightarrow{\mathbb{R}}^3$.
\end{definition}
\begin{remark}
Such spheres can only exist in a $(4n+3)$--dimensional manifold. The fact that $\alpha$ and $-\alpha$ do not induce the same volume form in dimensions congruent to $1$ modulo $4$ yields an obstruction for their existence.
\end{remark}
\noindent Note that for a $3$--fold the triple $(\alpha_0,\alpha_1,\alpha_2)$ constitutes a framing of the cotangent bundle.
\begin{lemma}
Let $M$ be a $3$--fold and $S$ a linear contact sphere. The class $[S]\in\pi_2({\mathcal{C}}(M,\xi))$ is non--trivial and has infinite order.
\end{lemma}
\begin{proof}
Let $p\in M$ be a point and consider the framing $\tau=(\alpha_0,\alpha_1,\alpha_2)_p$. In the three--dimensional case ${\mathcal{A}}({\mathbb{R}}^3)$ is homotopic to a $2$--sphere. This homotopy can be realized by projection $\pi$ onto the space of cooriented $2$--plane distributions. The degree of the evaluation map is computed via
$${\mathbb{S}}^2\stackrel{\varepsilon_{(p,\tau)}}{\longrightarrow}{\mathcal{A}}(T_pM)\stackrel{\pi}{\longrightarrow}Dist({\mathbb{R}}^3)\cong{\mathbb{S}}^2$$
$$z\longmapsto e_0(z)\alpha_0(p)+e_1(z)\alpha_1(p)+e_2(z)\alpha_2(p)\longmapsto (e_0(z),e_1(z),e_2(z)).$$
Being the identity, this map has degree $1$.
\end{proof}
\subsection{$3$--Sasakian manifolds}
Let us define a class of contact manifolds with natural linear contact spheres.

\begin{definition}
Let $(M^{4n+3},g)$ be a Riemannian manifold. It is said to be $3$--Sasakian if the holonomy group of the metric cone $(C(M),\bar{g})=(M\times{\mathbb{R}}^+, r^2g+dr\otimes dr)$ reduces to $Sp(n+1)$.
\end{definition}
\noindent This implies that $(C(M),\bar{g})$ is a hyperk\"ahler manifold $(C(M),\bar{g},I,J,K)$. The hyperk\"ahler structure induces a $2$--sphere of complex structures
$${\mathbb{S}}^2(\bar{g})=\{e_0I+e_1J+e_2K: e_0^2+e_1^2+e_2^2=1\}.$$

\noindent Any such complex structure ${\mathfrak{j}}\in{\mathbb{S}}^2(\bar{g})$ endows $(M\times{\mathbb{R}}^+,\bar{g})$ with a K\"ahler structure, providing $(M,g)$ with a Sasakian structure. The vertical vector field $\partial_r$ on $M\times{\mathbb{R}}^+$ is orthogonal to $M\times\{1\}$ and the form $\alpha$ defined by $\alpha_{\mathfrak{j}}(v)=g(v,{\mathfrak{j}}\partial_r)$ is a contact structure. Thus, a $3$--Sasakian structure provides a linear contact sphere $\{\alpha_{\mathfrak{j}}\}_{{\mathfrak{j}}\in{\mathbb{S}}^2(\bar{g})}$ generated by $\alpha_I,\alpha_J$ and $\alpha_K$.

\begin{theorem}\label{thm:linSas}
Let $M^{4n+3}$ be a $3$--Sasakian manifold. The class of the associated linear contact sphere is an element of infinite order in $\pi_2({\mathcal{C}}(M,\ker(\alpha_I)))$.
\end{theorem}
\begin{proof}
Let $p\in M$ and note that the $4n$--distribution $\eta=\ker\alpha_I\cap\ker\alpha_J\cap\alpha_K$ is $(I,J,K)$--invariant. Thus, it can be identified with the quaternionic vector space ${\mathbb{H}}^n$ by fixing a quaternionic framing $v=\{v_1,\ldots,v_n\}$. This induces a real framing $\tau=\{v,Iv,Jv,Kv\}$ for $\eta$, identifying it with ${\mathbb{R}}^{4n}$ endowed with the standard quaternionic structure.\\

\noindent Consider the Reeb vector fields $R_I,R_J,R_K$ associated to $\alpha_I,\alpha_J,$ and $\alpha_K$. Extend the framing $\tau$ to $\widetilde\tau=\{\tau,R_I,R_J,R_K\}$. Interpret the space ${\mathcal{A}}({\mathbb{R}}^{4n+3})$ as pairs of $(v,{\mathfrak{j}})$, where $v\in{\mathbb{S}}^{4n+2}\subset{\mathbb{R}}^{4n+3}$ is a unit vector and ${\mathfrak{j}}$ an almost complex structure in its orthogonal space. Define
\begin{equation}\label{eq:h}
h: {\mathcal{A}}({\mathbb{R}}^{4n+3})\longrightarrow\mathcal{J}({\mathbb{R}}^{4n+3}\oplus{\mathbb{R}}),\quad (v,{\mathfrak{j}})\longmapsto\{\widetilde{\mathfrak{j}}:\langle v\rangle^{\perp}\oplus\langle v\rangle\oplus\langle\partial_t\rangle\longrightarrow \langle v\rangle^{\perp}\oplus\langle v\rangle\oplus\langle\partial_t\rangle\}
\end{equation}
where the almost complex structure is $\widetilde{\mathfrak{j}}=\left(\begin{array}{ccc}
{\mathfrak{j}} & 0 & 0\\
0 & 0 & -1\\
0 & 1 & 0
\end{array}\right)$. This induces a morphism of second homotopy groups. Through the above identification the linear contact sphere generated by $(\alpha_I,\alpha_J,\alpha_K)$ evaluates in a sphere $\langle(\xi_I,I),(\xi_J,J),(\xi_K,K)\rangle\in{\mathcal{A}}({\mathbb{R}}^{4n+3})$. This sphere maps via (\ref{eq:h}) to the sphere of complex structures generated by the triple $(I,J,K)$ in $\mathcal{J}({\mathbb{R}}^{4n+4})$.\\

\noindent It is left to prove that the class of that sphere is an infinite order element of $\pi_2(SO(4n+4)/U(2n+2))$. Let us write $m=n+1$ to ease the notation. The homotopy fibration
$$U(2m)\longrightarrow SO(4m)\longrightarrow SO(4m)/U(2m)$$
induces an injection $\pi_2(SO(4m)/U(2m))\longrightarrow\pi_1(U(2m))\cong{\mathbb{Z}}$.\\

\noindent Let $(\theta,\phi)\in[0,2\pi]\times[0,\pi]$ be spherical angles. Define
$$J_\theta=\cos\theta J+\sin\theta K,\quad \widetilde{I}=\cos\phi I+\sin\phi J_\theta,\quad P_{\theta,\phi}=\cos(\phi/2)I+\sin(\phi/2)J_\theta.$$
The sphere is represented by $\widetilde{I}$, we shall compute its image under the boundary morphism. Note that $P_{\theta,\phi}\in SO(4m)$ and $\widetilde{I}=P_{\theta,\phi}^tIP_{\theta,\phi}$ . Further $P_{\theta,\pi}=J_\theta=(cos\theta\cdot id+\sin\theta I)J$, with $cos\theta\cdot id+\sin\theta I\in U(2m)$ and $J\in SO(4m)$. This decomposition provides a representative in $\pi_2(SO(4m)/U(2m))$. Thus the loop in $\pi_1(U(2m))$ is provided by $cos\theta\cdot id+\sin\theta I$ with $\theta\in[0,2\pi]$. Since the identification $\pi_1(U(2m))\cong{\mathbb{Z}}$ is given by the complex determinant, the degree of the sphere is $2m$.
\end{proof}
\noindent The argument above applies to a broader class of manifolds:
\begin{definition}
A contact manifold $(M,\xi_0)$ is said to possess an almost--quaternionic sphere if it admits a sphere ${\mathbb{S}}^2\stackrel{\xi}{\longrightarrow}{\mathcal{C}}(M,\xi_0)$ such that:
\begin{itemize}
\item[1)] There exists a family ${\{{\mathfrak{j}}_p\}}_{p\in{\mathbb{S}}^2}$ compatible with the contact distributions $\xi_p=\xi(p)$,
\item[2)] There exists a point $q\in M$ and a framing $\tau$ for $T_qM$ such that $e_{q,\tau}(\xi({\mathbb{S}}^2))$ becomes the linear sphere associated to $\langle(\xi_I,I),(\xi_J,J),(\xi_K,K)\rangle\in{\mathcal{A}}({\mathbb{R}}^{4n+3})$.
\end{itemize}
\end{definition}
\begin{corollary}
An almost--quaternionic sphere inside a contact manifold $(M,\xi)$ generates a class of infinite order in $\pi_2({\mathcal{C}}(M,\xi))$.
\end{corollary}
\section{Reeb Flow for Spheres}\label{sec:gi}

\noindent Let us prove Corollary \ref{cor:Gi}. The standard contact sphere will be denoted $({\mathbb{S}}^{2n+1},\xi)$. The relevant case is that of the spheres ${\mathbb{S}}^{2k+1}$ with $k$ odd. Indeed, for the spheres ${\mathbb{S}}^{2k+1}$ with $k=2n$ the Reeb flow is non--trivial in $\pi_1(SO(4n+2))\hookrightarrow\pi_1({\operatorname{Diff}}_0({\mathbb{S}}^{4n+1}))$. Thus it cannot be contractible in ${\operatorname{Cont}}_0(M,\xi)\subset{\operatorname{Diff}}_0({\mathbb{S}}^{4n+1})$. In order to conclude the case ${\mathbb{S}}^{4n+3}$ we detail the construction in Sections \ref{sec:pre} and \ref{sec:lin}.\\

\noindent Consider the endomorphisms $I,J,K$ of ${\mathbb{R}}^{4(n+1)}$ obtained by direct sum of the corresponding endomorphisms $i,j,k$ of ${\mathbb{R}}^4$, satisfying the quaternionic relations
$$i^2=j^2=k^2=ijk=-1.$$
The endomorphisms $I,J,K$ anti--commute and hence any of their linear combinations is a complex structure. Let $e=(e_0,e_1,e_2):{\mathbb{S}}^2\longrightarrow{\mathbb{R}}^3$ be the standard embedding of the $2$--sphere in Euclidean $3$--space with azimuthal angle $\theta$ and polar angle $\phi$:
$$e_0=\cos\theta\sin\phi,\quad e_1=\sin\theta\sin\phi,\quad e_2=\cos\phi,\quad (\theta,\phi)\in[0,2\pi]\times[0,\pi].$$

\noindent A complex structure ${\mathfrak{j}}\in End({\mathbb{R}}^{4n+4})$ induces the real $(4n+2)$--distribution
$$\xi_{\mathfrak{j}}=T{\mathbb{S}}^{4n+3}\cap{\mathfrak{j}} T{\mathbb{S}}^{4n+3}$$ of ${\mathfrak{j}}$--complex tangencies on the sphere ${\mathbb{S}}^{4n+3}$. There exists a unique, up to scaling, $U({\mathfrak{j}},n)$--invariant $1$--form $\alpha_{\mathfrak{j}}$ such that $\ker\alpha_{\mathfrak{j}}=\xi_{\mathfrak{j}}$. It is given by $\alpha(z)=z^t{\mathfrak{j}} dz$. We use the following three $1$--forms
$$\alpha_0=\alpha_I,\quad\alpha_1=\alpha_J,\quad \alpha_2=\alpha_K.$$
Their respective Reeb vector fields $R_0$, $R_1$ and $R_2$ are linearly independent and their flows are given by the family of rotations generated by $I$, $J$ and $K$. Consider the $1$--form $\alpha=e_0\alpha_0+e_1\alpha_1+e_2\alpha_2$. The form $\alpha$ is a contact form on ${\mathbb{S}}^{4n+3}$ for each value of $e$. Although not used in the rest of the article, it is simple to prove the following
\begin{lemma}
$({\mathbb{S}}^2\times{\mathbb{S}}^{4n+3},\ker\alpha)$ is a contact manifold.
\end{lemma}

\noindent Let us compute the loop at infinity for the trivial contact fibration
$${\mathbb{S}}^2\times{\mathbb{S}}^{4n+3}\longrightarrow{\mathbb{S}}^{2}, (e,p)\longmapsto e.$$
In the spherical coordinates above, we will obtain the loop at infinity corresponding to $\phi=\pi$. The contact connection allows us to lift a vector field $X$ in the base ${\mathbb{S}}^2$. The lift $\widetilde{X}$ is the unique vector field on ${\mathbb{S}}^2\times{\mathbb{S}}^{4n+3}$ conforming the two conditions
$$\alpha(\widetilde{X})=0,\qquad d\alpha(\widetilde{X},V)=0,\quad\mbox{with }V\mbox{ an arbitrary vertical vector field}.$$
Since uniqueness is provided once a solution is found, the following assertion can be readily verified
\begin{lemma}
The lift of the polar vector field $\partial_\phi$ to the contact connection given by $\alpha$ is
$$\widetilde{X}_\phi=\partial_\phi+\frac{1}{2}\left(-\sin\theta R_0+\cos\theta R_1\right).$$
\end{lemma}
\noindent The Hamiltonian will appear once we pull--back the contact form $\alpha$ with the $\pi$--time flow of the lift $\widetilde{X}_\phi$. Consider the linear endomorphism $F_\theta=\frac{1}{2}\left(-\sin\theta I+\cos\theta J\right)$. The flow associated to $\widetilde{X}_\phi$ induces a diffeomorphism between the central fibre $\{\phi=0\}$ and the fibre at an arbitrary $\phi$. This diffeomorphism can be expressed as
$$\varphi_\phi:{\mathbb{S}}^{4n+3}\longrightarrow{\mathbb{S}}^{4n+3},\quad \varphi(p)=e^{F_\theta\phi}p.$$
This is understood as a map in complex space ${\mathbb{C}}^{2n+2}$ restricted to the sphere. The theory explained in Section \ref{sec:pre}, in particular formula (\ref{eq:radial_triv}), implies that the pull--back will be of the form $\alpha_2+H(p,\phi)d\theta$. A computation yields
\begin{lemma}
$\varphi_\phi^*(\alpha)=\alpha_2+\sin^2(\phi/2)d\theta$
\end{lemma}
\noindent The loops correspond to the flow of the vector field associated to $G=-\sin^2(\phi/2)$. The loop at infinity has Hamiltonian $G|_{\phi=\pi}\equiv-1$. Thus it is the Reeb flow.\\

\noindent We have geometrically realized the boundary map of the long exact homotopy sequence (\ref{eq:seq}). The non--contractibility of the Reeb flow will follow from an understanding of the contact sphere above and the group $\pi_2({\operatorname{Diff}}_0({\mathbb{S}}^{4n+3}))$. Regarding the former we have the following

\begin{lemma}\label{lem:acs} Let $S$ be the sphere of complex structures
$$S=\{e_0I+e_1J+e_2K:e\in{\mathbb{S}}^2\}\subset SO(4n+4)/U(2n+2).$$
\begin{itemize}
\item[1)] $S$ represents a non--trivial element of $\pi_2(SO(4n+4)/U(2n+2))\cong{\mathbb{Z}}$.
\item[2)] The image of $S$ in ${\mathcal{C}}({\mathbb{S}}^{4n+3},\xi)$ generates an infinite cyclic subgroup in $\pi_2({\mathcal{C}}({\mathbb{S}}^{4n+3},\xi))$.
\end{itemize}
\end{lemma}
\begin{proof}
Both statements follow from the argument provided in the proof of Theorem \ref{thm:linSas}.
\end{proof}

\noindent Concerning the group ${\operatorname{Diff}}_0({\mathbb{S}}^{4n+3})$, the following lemma will suffice.
\begin{lemma}\label{lem:tor}
$\pi_2({\operatorname{Diff}}({\mathbb{S}}^{4n+3}))\otimes{\mathbb{Q}}=0$ for $n\geq 2$.
\end{lemma}
\begin{proof}
This is a result in algebraic topology. Let ${\operatorname{Diff}}_0({\mathbb{D}}^l,\partial)$ be the group of diffeomorphisms of the $l$--disk restricting to the identity at the boundary. Note the homotopy equivalence
$${\operatorname{Diff}}_0({\mathbb{S}}^l)\simeq SO(l+1)\times{\operatorname{Diff}}_0({\mathbb{D}}^l,\partial)$$
and that $\pi_2(SO(l+1))=0$ since $SO(l+1)$ is a Lie group. Let $\phi(l)=\min\{(l-4)/3,(l-7)/2\}$. In the stable concordance range $0\leq j<\phi(l)$ we have
\begin{equation}\label{fh}
\pi_j({\operatorname{Diff}}_0({\mathbb{D}}^l,\partial))\otimes{\mathbb{Q}}=0,\quad\mbox{if }l\mbox{ even or }4\not|j+1.
\end{equation}
See \cite{We} for details. In particular $\pi_2({\operatorname{Diff}}_0({\mathbb{D}}^l,\partial))\otimes{\mathbb{Q}}=0$ for $l>11$. We are thus able to conclude
$$\pi_2({\operatorname{Diff}}_0({\mathbb{S}}^{4n+3}))\otimes{\mathbb{Q}}\cong\pi_2({\operatorname{Diff}}_0({\mathbb{D}}^{4n+3},\partial))\otimes{\mathbb{Q}}=0,\quad n>2.$$
For the case $n=2$ we provide a more {\it ad hoc} argument. Let $C({\mathbb{D}}^{11})$ be the space of pseudo--isotopies for the disk ${\mathbb{D}}^{11}$. There exists a homotopy fibration
$${\operatorname{Diff}}_0({\mathbb{D}}^{12},\partial)\longrightarrow C({\mathbb{D}}^{11})\longrightarrow{\operatorname{Diff}}_0({\mathbb{D}}^{11},\partial)$$
Algebraic $K$--theory implies $\pi_1C({\mathbb{D}}^{11})\otimes{\mathbb{Q}}=\pi_2C({\mathbb{D}}^{11})\otimes{\mathbb{Q}}=0$. Observe that (\ref{fh}) implies that $\pi_1({\operatorname{Diff}}({\mathbb{D}}^{12},\partial))$ is a torsion group. The long exact homotopy sequence of the above fibration gives
$$\ldots\longrightarrow\pi_2(C({\mathbb{D}}^{11}))\stackrel{\rho_2}{\longrightarrow}\pi_2({\operatorname{Diff}}_0({\mathbb{D}}^{11},\partial))\stackrel{\partial}{\longrightarrow}\pi_1({\operatorname{Diff}}_0({\mathbb{D}}^{12},\partial))\stackrel{i_1}{\longrightarrow}\pi_1(C({\mathbb{D}}^{11}))\longrightarrow\ldots$$
This implies the short exact sequence of Abelian groups
$$0\longrightarrow A\longrightarrow\pi_2({\operatorname{Diff}}_0({\mathbb{D}}^{11},\partial))\longrightarrow B\longrightarrow0,$$
where $A=\ker\partial={\operatorname{im}}\rho_2$ and $B={\operatorname{im}} i_1={\operatorname{coker}}\rho_2$. Thus $\pi_2({\operatorname{Diff}}_0({\mathbb{D}}^{11},\partial))$ is a torsion group.
\end{proof}
\begin{remark}
The Smale conjeture ${\operatorname{Diff}}_0({\mathbb{S}}^3)\simeq SO(4)$ holds for ${\mathbb{S}}^3$, see \cite{Ha}.
\end{remark}

\noindent In order to conclude Corollary \ref{cor:Gi} for ${\mathbb{S}}^{4n+3}$ consider the class of the Reeb loop in $\pi_1({\operatorname{Cont}}_0({\mathbb{S}}^{4n+3},\xi))$. The construction explained above shows that it lies in the image of the boundary morphism
$$\partial_2:\pi_2({\mathcal{C}}({\mathbb{S}}^{4n+3},\xi))\longrightarrow\pi_1({\operatorname{Cont}}_0({\mathbb{S}}^{4n+3},\xi)).$$
If the Reeb class were to be zero the sphere $S$ would lie in the image of $\pi_2({\operatorname{Diff}}_0({\mathbb{S}}^{4n+3}))$ in (\ref{eq:seq}). Lemma \ref{lem:tor} implies that such a sphere needs to be a torsion class if $n\geq 2$. Lemma \ref{lem:acs} contradicts this statement. Thus proving Corollary \ref{cor:Gi}.\\

\section{Higher homotopy groups} \label{sec:gener}
\noindent The previous arguments can be modified for $n$--dimensional homotopy spheres. This allows us to conclude properties of the higher homotopy type of the contactomorphism group. Consider the evalution map
$$
e_{p,\tau}: {\mathcal{A}}(M) \longrightarrow {\mathcal{A}}({\mathbb{R}}^{2n+1}).
$$
Composition with the homotopy inverse $\i: {\mathcal{C}}(M) \to {\mathcal{A}}{\mathcal{C}}(M)$
defines higher homotopy maps
$$
\pi_k(e_{p,\tau} \circ \i): \pi_{k}({\mathcal{C}}(M)) \longrightarrow \pi_k({\mathcal{A}}({\mathbb{R}}^{2n+1})),\quad k \geq 1.
$$
Let us provide a simple application. Define the natural inclusion
$$
i_{{\mathcal{J}}}: {{\mathcal{J}}}({\mathbb{R}}^{2n+2}) \longrightarrow {\mathcal{C}}({\mathbb{S}}^{2n+1},\xi),\quad\mbox i_{{\mathcal{J}}}({\mathfrak{j}})= T{\mathbb{S}}^{2n+1} \cap {\mathfrak{j}} T{\mathbb{S}}^{2n+1}.$$
\begin{lemma} \label{lem:inclu}
The map $i_{{\mathcal{J}}}$ is a homotopy inclusion.
\end{lemma}
\begin{proof}
Consider the following chain of maps
$$
c: {{\mathcal{J}}}({\mathbb{R}}^{2n+2}) \stackrel{i_{{\mathcal{J}}}}{\longrightarrow} {\mathcal{C}}({\mathbb{S}}^{2n+1},\xi) \stackrel{e_{p,\tau} \circ \i}{\longrightarrow} {\mathcal{A}}({\mathbb{R}}^{2n+1}) \stackrel{h}{\longrightarrow} {{\mathcal{J}}}({\mathbb{R}}^{2n+2}).
$$
The definition of each map implies $c= id$. Therefore, it induces the identity in homotopy:
$$
\pi_k(c)=id: \pi_k({{\mathcal{J}}}({\mathbb{R}}^{2n+2})) \stackrel{\pi_k(i_{{\mathcal{J}}})}{\longrightarrow} \pi_k({\mathcal{C}}({\mathbb{S}}^{2n+1}),\xi) \stackrel{\pi_k(e_{p,\tau}\circ \i)}{\longrightarrow} \pi_k({\mathcal{A}}({\mathbb{R}}^{2n+1})) \stackrel{\pi_k(h)}{\longrightarrow} \pi_k({{\mathcal{J}}}({\mathbb{R}}^{2n+2})).
$$
Thus the map $i_{{\mathcal{J}}}$ induces an injection $\pi_k(i_{{\mathcal{J}}})$, $\forall k\geq0$.
\end{proof}
\noindent This lemma can be combined with results on the homotopy type of the group ${\operatorname{Diff}}({\mathbb{S}}^{2n+1})$. We can then conclude the existence of infinite order elements in certain homotopy groups of ${\operatorname{Cont}}({\mathbb{S}}^{2n+1},\xi)$. Among many others, a simple instance is the following
\begin{lemma}
The group $\pi_5({\operatorname{Cont}}({\mathbb{S}}^{2n-1},\xi))$ has an element of infinite order, for $n \geq 12$.
\end{lemma}
\begin{proof}
Using the connecting map $\partial_6$, as described in equation (\ref{eq:lifting}), the statement is reduced to the following two assertions:
\begin{itemize}
\item[-] $\pi_6({{\mathcal{J}}}({\mathbb{R}}^{2n}))= \pi_6(SO(2n)/U(n))={\mathbb{Z}}$ and therefore, by Lemma \ref{lem:inclu}, ${\operatorname{rk}}(\pi_6({\mathcal{C}}({\mathbb{S}}^{2n-1},\xi)))\geq 1$.
\item[-] $\pi_6({\operatorname{Diff}}({\mathbb{S}}^{2n-1})) \otimes {\mathbb{Q}}=0$, for $n\geq 12$. This is again a consequence of the results in \cite{We}.
\end{itemize}
\end{proof}
\noindent Bott Periodicity Theorem allows us to apply the same argument to infinitely many other homotopy groups of ${\operatorname{Cont}}({\mathbb{S}}^{2n-1},\xi)$. These techniques can be adapted for general contact manifolds as long as there is a partial understanding of the homotopy type of their group of diffeomorphisms.
\begin{thebibliography}{xxxx}
\bibitem[BG]{BG} C.P. Boyer, K. Galicki, $3$--Sasakian manifolds, Surveys in Differential Geometry: Essays on Einstein Manifolds, Surv. Differ. Geom. VI. Boston Int. Press (1999), 123--184.
\bibitem[Bo]{Bo} F. Bourgeois, Contact homology and homotopy groups of the space of contact structures, Math. Res. Lett. 13 (2006), 71--85.
\bibitem[El]{El} Y. Eliashberg, Contact $3$--manifolds twenty years since J. Martinet's work, Ann. Inst. Fourier 46 (1992), 165--192.
\bibitem[Ge]{Ge} H. Geiges, J. Gonzalo, On the topology of the space of contact structures on torus bundles, Bull. London Math. Soc. 36 (2004), 640--646.
\bibitem[Gi]{Gi} E. Giroux, Sur la g\'eom\'etrie et la dynamique des transformations de contact, S\'eminaire Bourbaki (2009), n.1004.
\bibitem[Ha]{Ha} A. E. Hatcher, A proof of the Smale conjecture, Ann. of Math. 117 (1983), 553--607.
\bibitem[Pr]{Pr} F. Presas, A class of non--fillable contact structures, Geometry$\&$Topology 11 (2007), 2203--2225.
\bibitem[We]{We} M. Weiss, B. Williams, Automorphisms of manifolds, Surveys on Surgery Theory. Vol. II Papers Dedicated to C.T.C Wall, Annals of Mathematics Studies 149 (2001).
\end{thebibliography}
\end{document}
