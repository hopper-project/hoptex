

\documentclass{amsart}
\usepackage{latexsym,amsxtra,amscd,ifthen}
\usepackage{amsfonts}
\usepackage{verbatim}
\usepackage{amsmath}
\usepackage{amsthm}
\usepackage{amssymb}

\numberwithin{equation}{section}

\tolerance=500
\unitlength=1mm

\theoremstyle{plain}
\newtheorem{theorem}{Theorem}[section]
\newtheorem{theorema}[theorem]{Theorem}
\newtheorem{theordef}[theorem]{Theorem--Definition}
\newtheorem{prop}[theorem]{Proposition}
\newtheorem{lemma}[theorem]{Lemma}
\newtheorem{cor}[theorem]{Corollary}
\newtheorem{corollary}[theorem]{Corollary}
\newtheorem{corollarium}[theorem]{Corollary}
\newtheorem{conj}[theorem]{Conjecture}

\theoremstyle{definition}
\newtheorem{defi}[theorem]{Definition}
\newtheorem{definition}[theorem]{Definition}
\newtheorem{definitia}[theorem]{Definition}
\newtheorem{exam}[theorem]{Example}
\newtheorem{probl}[theorem]{Problem}
\newtheorem{quest}[theorem]{Question}
\newtheorem{rema}[theorem]{Remark}

           

          

\font \maius =cmcsc10

\begin{document}

\title{Noncommutative Grassmannian of codimension two has coherent coordinate ring}

\author{Dmitri Piontkovski}

      \address{Department of Mathematics for Economics,
Myasnitskaya str. 20, State University `Higher School of Economics', Moscow 101990, Russia
}

\thanks{This study was supported by ``The National Research University Higher School of Economics Academic Fund Program'' in 2013--2014 (research grant 12-01-0134).  
It is also supported by RFBR project 14-01-00416.}
\email{piont@mccme.ru}

\subjclass[2000]{14A22; 16W50; 16S38}

\keywords{Calabi-Yau algebra, coherent ring, noncommutative scheme}

\date{\today}

\begin{abstract}
A noncommutative Grassmannian $A ={\mathop{\mathrm{NGr}}\nolimits}(m,n)$ is introduced by Efimov, Luntz, and Orlov in
 {\em Deformation theory of objects in homotopy and derived categories III: Abelian categories}
 as a noncommutative algebra associated to an exceptional collection of $n-m+1$ coherent sheaves on ${\ensuremath{\mathbb P}}^n$.
 It is a graded Calabi--Yau ${\ensuremath{\mathbb Z}}$-algebra of dimension $n-m+1$. 
 We show that this algebra is coherent provided that the codimension $d=n-m$ of the Grassmannian is two. According to {\em op.~cit.}, this gives a $t$-structure on the 
 derived category of the coherent sheaves on the noncommutative Grassmannian.
 
 

 The proof is quite different from the recent proofs of the coherence of some graded 3-dimensional Calabi--Yau algebras and is based on  properties of a PBW-basis of the algebra $A$. 
\end{abstract}

\maketitle

\section{Introduction}

Recall that 
a connected  {\ensuremath{\mathbb N}}-graded algebra of the form $A = A_0 \oplus
A_1 \oplus \dots $ such that  $A_0$ is a copy of the basic field  $k$ is called {\em regular}  if
it has finite global dimension (say, $d$) and satisfies the
following Gorenstein property:
$$
{\mbox{Ext\,}}^i_A(k,k) \cong\left\{ \begin{array}{ll}
k^* [l] \mbox{ for some } l\in {\ensuremath{\mathbb Z}} ,& i =d \\
 0, & i\ne d .
\end{array}  \right.
$$
The same notion of regularity is extend (following Bondal and Polishchuk~\cite{bp}) to a slightly more general case of a ${\ensuremath{\mathbb Z}}$-algebra $A$, see Subsection~\ref{subs:z} below.

 
Regular algebras play the roles of coordinate rings of noncommutative projective spaces 
in a version of noncommutative projective geometry~\cite{pol,BvdB} which generalizes 
the well-known approach of
Artin and Zhang~\cite{AZ}. Namely, suppose that a regular algebra of global dimension $d$ is graded coherent.
 Consider the quotient category ${\mathop{\mathrm{qgr}}\nolimits} A = {\mathop{\mathrm{cmod}}\nolimits} A / {\mathop{\mathrm{tors}}\nolimits} A$ 
 of the category  ${\mathop{\mathrm{cmod}}\nolimits} A$ of finitely
presented (=graded coherent)  right graded $A$-modules by its subcategory ${\mathop{\mathrm{tors}}\nolimits} A$ of finite-dimensional  modules.
This category  ${\mathop{\mathrm{qgr}}\nolimits} A $
plays the role 
of the category of coherent sheaves 
on a noncommutative $(d-1)$-dimensional projective space.

Here we are interested in the case $d=3$, that is, in the case of noncommutative planes. The famous classification of 3-dimensional regular  algebras $A$ of polynomial growth is obtained Artin and Shelter~\cite{AS}. Particularly, they have shown that these algebras  are Noetherian (hence, coherent). 
It is not hard to construct also non-Noetherian 3-dimensional regular algebras (e.g., one may follow the approach of~\cite[Sections~2 and~3]{AS}).
In contrast, it is often not easy to prove that such an algebra is coherent. 
However, there are important examples for which the coherence is established. These are the octonion algebra of P.~Smith~\cite{smi}, 
the Yang--Mills algebra introduced by Movshev and Swartz~\cite{ms} which is coherent by a theorem of Herscovich~\cite{hers} (the Yang--Mils algebra introduced by Connes and Dobios--Violette~\cite{cdv} is a particular case of it), and 3-Calabi--Yau algebras which are Ore extensions of 2-Calaby--Yau ones by He, Oystaeyen, and Zhang~\cite{hoz}. 

In all these cases, the coherence property is proved using the same 
lemma~\cite[Prop.~3.2]{Pi1}. It states that if a non-trivial two-sided ideal $I$ in a graded algebra $A$ is free as a left module and the quotient algebra $A/I$ is right Noetherian, then $A$ is graded coherent. 
So, the known examples of coherent regular algebras are, in a sense, extensions of Noetherian algebras along 
free modules. 

In this paper, we prove the coherence property of another 3-dimensional regular algebra. In contrast to the previous cases, 
it seems  that the approach based on the above lemma fails for this algebra.\footnote{At least,  for this algebra $A$ calculations of Hilbert series for several ideals $I$ such that the quotient algebra $A/I$ is Noetherian  by natural reasons 
show that $I$ cannot be projective as a left module.}  
This algebra is introduced by Efimov, Luntz, and Orlov~\cite{elo} under the name {\em Noncommutative Grassmannian} (see Subsection~\ref{subs:gras} below for an explicit definition).  According to~\cite[Section~7]{elo}, a noncommutative Grassmannian ${\mathop{\mathrm{NGr}}\nolimits}(m,n)$ 
`is a true noncommutative moduli space of the structure sheaves ${\mathcal O_{{\ensuremath{\mathbb P}} (W)} \in D^b_{coh}({\ensuremath{\mathbb P}}^n)}$', where $W \subset k^n$ runs the vector subspaces of dimension $m$. The connection with structure sheaves is based on the theory of helices in derived categories by A.Bondal and A.Polishchuk~\cite{bondal, bp}. The construction of~\cite{elo} gives a description (by constructing a $t$-structure) of the derived category of ${\mathop{\mathrm{qgr}}\nolimits} {\mathop{\mathrm{NGr}}\nolimits}(m,n)$ provided that the algebra ${\mathop{\mathrm{NGr}}\nolimits}(m,n)$  of coherent.  It is pointed out in~\cite[Remark~8.23]{elo} 
that ${\mathop{\mathrm{NGr}}\nolimits}(m,n)$ is coherent if the codimension $d = n-m$ of the Grassmannian is equal to 1 (since the algebra ${\mathop{\mathrm{NGr}}\nolimits}(m,n)$ has global dimension 2 in this case). The first non-trivial case is the Grassmannian of codimension $d= 2$, when the algebra has global dimension three. The main result of this paper is the following.

\begin{theorem}
\label{th:main_intro}
The noncommutative Grassmannian algebra ${\mathop{\mathrm{NGr}}\nolimits}(m,n)$ is coherent provided that $n-m=2$.
\end{theorem}

The paper is organized as follows. In Section~\ref{sec:defs} we briefly remind necessary facts about ${\ensuremath{\mathbb Z}}$-algebras and, in particular, about the noncommutative Grassmannian algebra ${\mathop{\mathrm{NGr}}\nolimits}(m,n)$.  
In Subsection~\ref{subs: hat A},  we note that the ${\ensuremath{\mathbb Z}}$--algebra $A ={\mathop{\mathrm{NGr}}\nolimits}(m,n)$ with $n-m=2$ is 3-periodic, so, its properties are essentially the same as the properties of the corresponding algebra $\hat A$ over a triangle quiver. We immediately calculate here the Hilbert series of $\hat A$. This obviously gives also the Hilbert series of $A$. 
In Subsection~\ref{subs:PBW}, we show that $A$ is a PBW algebra as a 6-periodic ${\ensuremath{\mathbb Z}}$-algebra (in particular, it admits a quadratic Gr\"obner basis of relations). Note that we do not know if $\hat A$ is PBW or not. 
In Proposition~\ref{prop:3-proc}, we prove that $A$ satisfies a property of bounded processing~\cite{pi01}, 
that is, the structure of the multiplication of paths in this quiver algebra is essentially depend 
only on bounded segments of the multipliers. We recall necessary definitions in  
Subsection~\ref{subs:proc}.
Using a result of~\cite{pi01}, we finally  deduce in Corollary~\ref{cor:main} that the algebra $A$ is coherent. A stronger consequence of bounded processing is briefly discussed in Remark~\ref{rem:uni-coh}.

\subsection{Acknowledgement}

I am grateful to Alexei Bondal, Alexander Efimov, and Dmitri Orlov for stimulating questions and fruitful discussions.

\section{Background and notations}

\label{sec:defs}

\subsection{{\ensuremath{\mathbb Z}}-algebras}
\label{subs:z}

Recall that a ${\ensuremath{\mathbb Z}}$-algebra is a path
algebra (with relations) over the infinite line quiver $\dots \to
-1 \to 0 \to 1
 \to 2\to \dots$ with  multiple arrows. 
 We refer the reader to~\cite{bp} or~\cite[Ch.~4, Sect.~9--10]{PP} for the basic definitions of quadratic, Koszul and PBW ${\ensuremath{\mathbb Z}}$--algebras. 
We say that
a ${\ensuremath{\mathbb Z}}$-algebra $A = \oplus_{i,j \in {\ensuremath{\mathbb Z}}}$ (where $A_{i,j} = 0$ for $i<j$ and $A_{i,i}=k$) is {\em regular} of dimension $d$ if each irreducible module 
$k_i = P_i / \oplus_{j> i}A_{i,j}$  has global dimension $d$ (where $P_i = \oplus_{j\ge i}A_{i,j}$ is the corresponding projective module)
and the Exts of these modules  satisfy an analogous Gorenstein condition~\cite[Sect.~4]{bp}
$$
{\mbox{Ext\,}}^i (k_s, P_t)\cong\left\{ \begin{array}{ll} k^* \mbox{ if } t = s+l ,& i =d, 
\\
0, & \mbox{ otherwise} 
\end{array}  \right.
$$
for some $l\in {\ensuremath{\mathbb Z}}$. Note that regular algebras are sometimes also called {\em AS-regular}. 

\subsection{Noncommutative Grassmannian}

\label{subs:gras}

A noncommutative Grassmannian is defined by Efimov, Lunts, and
Orlov~\cite[Part~3]{elo} as a noncommutative scheme associated to
the  following algebra. 
  Given two positive integers $m<n$ and an $n$-dimensional
vector space $V$, let $A = A^{m,V}$ be a quadratic ${\ensuremath{\mathbb Z}}$-algebra
with $A_{ij} = k$ (a basic field) and generators
$$
A_{i,i+1} = \left\{\begin{array}{ll} {{\Lambda}}^d V, & i|(d+1),\\
 V^*, &\mbox{ otherwise, }
\end{array} \right.
$$
where $d=n-m$. The quadratic relations of $A$ are defined via the
natural exact sequences
$$
0\to \Lambda^{d-1} V \to A_{i+1,i+2}\otimes A_{i,i+1} \to
A_{i,i+2} \to 0 \mbox{ for } (d+1)|i,i+1
$$
and
$$
0\to \Lambda^2 V^* \to A_{i+1,i+2}\otimes A_{i,i+1} \to A_{i,i+2}
\to 0, \mbox{ otherwise}.
$$

Obviously, $A$ is $(d+1)$-periodic (that is, $A_{i,j}$ is
naturally isomorphic to $A_{i+d+1,j+d+1}$).
By \cite[Prop.~8.18]{elo}, $A$ isomorphic to the automorphism ${\ensuremath{\mathbb Z}}$-algebra of the helix
generated by the exceptional collection of $(d+1)$ coherent sheaves on ${\ensuremath{\mathbb P}} (k^n)$
$$
E = ({\mathcal O}_{{\ensuremath{\mathbb P}} (k^n)}(m-n), \dots, {\mathcal O}_{{\ensuremath{\mathbb P}} (k^n)}(-1), {\mathcal O}_{{\ensuremath{\mathbb P}} (k^n)} ). 
$$
 It follows from~\cite{bp} that 
$A$ is Koszul and
Gorenstein of global dimension $d+1$. 

Note that $A$ is a so-called graded Calabi--Yau algebra
of dimension 3 (in the sense of~\cite{bock}). 
This follows from the same property of the corresponding algebra $\hat A$,
see Subsection~\ref{subs: hat A} below. 

 It is pointed out
in~\cite[Remark~8.23]{elo} that the description of the derived
category of ${\mathop{\mathrm{QMod}}\nolimits} A$ (``quasicoherent sheaves'' on the
noncommutative Grassmannian) can the transferred to $D^b({\mathop{\mathrm{qmod}}\nolimits} A)$
(derived category of the ``coherent sheaves'') provided that the
category ${\mathop{\mathrm{qmod}}\nolimits} A$ is Abelian, that is, $A$ is coherent. 
Namely, 
in this case a $t$-structure on the derived 
category of the finite-dimensional modules over a finite-dimensional algebra 
$\oplus_{1\le i,j \le n} A_{i,j}$ constructed in~\cite[Section~8]{elo} induces a $t$-structure 
on $D^b({\mathop{\mathrm{qmod}}\nolimits} A)$. 

\section{The results}

\subsection{Algebra $\hat A$ and its Hilbert series}

\label{subs: hat A}

Since $A$ is $(d+1)$-periodic (in terms of~\cite[Section~4]{bp}, it is geometric of period $d+1$),
 its categories of graded modules
(all , finitely generated, finitely presented, finite
dimensional\dots) are equivalent to the ones of the algebra 
$\hat
A = \oplus_{i=0}^d \oplus_{j\in {\ensuremath{\mathbb Z}}} A_{ij}$ considered as a path
algebra over a cyclic quiver of length $d+1$  with multiple arrows. 
It follows that $\hat A$ is Koszul and Gorenstein of
global dimension $d+1$ as well as $A$ is. The components of $\hat A$ are
indexed by the pairs $(i,j)$ with $i \in {\ensuremath{\mathbb Z}}/ (d+1){\ensuremath{\mathbb Z}} $ and $j\in
{\ensuremath{\mathbb Z}}$, where elements of $\hat A_{i,j}$ are considered as path from
$i$ to $(j-i) {\mbox{mod\,}} (d+1)$. The surjection $A\to \hat A$ is induced by the
natural surjection of  quiver algebras.

In our case $d=2$, the algebra $\hat A$ is a path algebra (=quiver algebra with relations) over
the quiver $Q:0\to 1\to 2\to 0$ where the arrows are multiple,
namely, with $n(n-1)/2$ arrows   $0\to 1$, $n$ arrows $1\to 2$ and $n$
arrows $2\to 0$. 
 Note that it follows from~\cite[Theorem~3.1]{bock} that 
 $\hat A$ is a graded Calabi--Yau algebra of dimension 3.  
 In 
the notations of Subsection~\ref{subs:PBW} below, the corresponding superpotential is equal to a cyclic element represented by 
$$
\sum_{t\in 3{\ensuremath{\mathbb Z}}} \sum_{i,j\in [1..n], i\ne j} x^{t-1}_i e^t_{ij} x^{t+1}_j .
$$

Recall  that the Hilbert series $H_{\hat A}$ of the graded path
algebra $\hat A$ over the above  quiver with 3 vertices
 is a $3 \times 3$ matrix $H_{\hat A}= (h_{ij})$ 
over ${\ensuremath{\mathbb Z}}[[t]]$  defined by
$$
h_{ij} = \sum_{m \in {\ensuremath{\mathbb Z}}} (\mbox{the number of paths $i\to j$ of
length } m) t^m = \sum_{n\in Z} t^{3n+j-i} {\mbox{dim\,}} \hat A_{i,3n+j}.
$$

\begin{prop}
\label{prop:Hilb_ser}
 The Hilbert series $H_{\hat A}$ of the  path
algebra $\hat A$ is
$$
H_{\hat A} = \left(
       \begin{array}{ccc}
           1 -t^3& \binom{n}{2} t & n t^2 \\
            n t^2 &  1-t^3           & -n   t         \\
            -n t        &     \binom{n}{2} t^2     &  1-t^3\\
           \end{array}
          \right)^{-1}
          $$
          $$ =
 \left( \begin {array}{ccc}
  1+ \frac{4-5{n}^{2}+{n}^{4}}{4}
                           {t}^{3}
        &  \frac{n(n-1)}{2} t
            &n \left( {n}^{2}-n-2 \right)
{t}^{ 2}/2\\
   \noalign{\medskip} \frac{n (n+1)}{2}
  {t}^{2}
         & 1+\frac{4-5{n}^{2}+{n}^{4}}{4}
  {t}^{3}
             &nt\\
\noalign{\medskip} nt&
  n \frac{{n}^{2}-n-2}{2}
{t}^{2}&
  1 + \frac{2+{n}^{4}-{n}^{3}-4{n}^{2}}{2}
{t}^{3}
  \end {array}
 \right) + O(t^4).
$$
\end{prop}

\begin{proof}
Note that the Koszul dual algebra of $\hat A$ has the following
components:
$$
\hat A_{ii}^! = k \mbox{ for all } i =0, 1,2,
$$
$$
\hat A_{i,i+1}^! = (\hat A_{i,i+1})^* \approx
    \left\{
       \begin{array}{ll}
        \left( {\Lambda}^2 V \right)^*,
              & i=0,\\
              V, & i=1,2,
           \end{array}
 \right.
$$
$$
\hat A_{i,i+2}^! = (\mbox{relations of $A$ in} A_{i+1,i+2}\otimes
A_{i,i+1})^* \approx
    \left\{
       \begin{array}{ll}
                   \left( \Lambda^{d-1} V \right)^* = V^*,   & i=0,2\\
                    \left( \Lambda^2 V^* \right)^*,   & i=1,
           \end{array}
 \right.
$$
and, by the Gorenstein property,
$$
\hat A_{i,i+3}^! \approx k.
$$
All other components of $\hat A^!$ vanish.

It follows that the Hilbert series of $\hat A^!$ is the following
matrix of order $d+1 = 3$
$$
H_{\hat A^!} (t) = \left(
       \begin{array}{ccc}
           1 +t^3& \binom{n}{2} t & n t^2 \\
            \binom{n}{2} t^2 &  1+t^3           & n   t         \\
            n t        &    n  t^2     &  1+t^3\\
           \end{array}
          \right)
$$
Since $\hat A$ is Koszul, we have $H_{\hat A}(t) H_{\hat A}(-t) = {\mbox{Id\,}}$, thus,  
$$
H_{\hat A}(t) =H_{\hat A}(-t)^{-1}.
$$
 \end{proof}

\subsection{PBW property}

\label{subs:PW}
\label{subs:PBW}

Let us fix a pair of dual bases $x = \{x_1,\dots, x_n \}$ in $V^*$
and $e = \{e_1,\dots, e_n \}$ in $V$. Given a finite sequence
$\alpha = (\alpha_1, \dots, \alpha_s)$, we denote by $e_\alpha$
the product $e_{\alpha_1} \wedge \dots \wedge e_{\alpha_s} \in
\Lambda^s V$, so that the set $e=\{ e_{ij} | i<j \}$ is a basis of
$\Lambda^2 V$. For $t\in {\ensuremath{\mathbb Z}}$, we denote by $x^t=\{x_1^t,\dots,
x_n^t \}$ and $e^t = \{e_{ij}^t |  i<j  \}$ the
corresponding bases of $A_{t,t+1} $ in the cases $3 |t$ and $3
\not| t$, respectively. We will sometimes omit the upper indexes
for elements of these bases. 

In this notations, the relations of the algebra $A$ of the grading component
$(t,t+2)$ has the following form (where for $i>j$ we use the sign
$e_{ij}:=-e_{ji}$):
\begin{equation}
\label{eq:rel_A}
\begin{array}{lll}
f_i = f_i^t:= \sum\limits_{j \ne i} e_{ij}^t x_j^{t+1}, & i = 1..n, & t\in 3{\ensuremath{\mathbb Z}},\\
c_{ij} = c_{ij}^t:= x_i^t x_j^{t+1} - x_j^t x_i^{t+1}, & 1\le i <
j \le n, & t\in 3{\ensuremath{\mathbb Z}} +1 ,\\
g_i = g_i^t:= \sum\limits_{j \ne i} x_j^{t} e_{ji}^{t+1} , & i = 1..n, & t\in 3{\ensuremath{\mathbb Z}} +2.\\
\end{array}
\end{equation}

Let us fix orderings of the elements  of the bases $x^t$ and $e^t$
in the following (6-periodic) way:
$$
\begin{array}{lll}
    e_{ij}>e_{kl}  & \mbox{ if } i<k \mbox{ or  } i=k,j<l  & \mbox{ for } t \in
    6{\ensuremath{\mathbb Z}},\\
    e_{ij} > e_{kl}  & \mbox{ if } i>k \mbox{ or  } i=k,j>l  & \mbox{ for } t \in
    6{\ensuremath{\mathbb Z}} +3,\\
    x_{1}<\dots <x_n & &\mbox{ for } t \in
    6{\ensuremath{\mathbb Z}} \pm 1,\\
     x_{1}>\dots >x_n & &\mbox{ for } t \in
    6{\ensuremath{\mathbb Z}} \pm 2.
             \end{array}
$$

Let us introduce the reverse lexicographical order on the paths of
the quiver $Q$, that is, for two paths $v=v_1\dots v_t$ and
$w=w_1\dots w_t$ of length $t$ with  the same head and the
same tail we set $v<w$ iff $v_j<w_j , v_{j+1} = w_{j+1}, \dots $
and $ v_{t} = w_{t}$ for some $j$.

\begin{prop}
\label{prop: A is PBW}
The above quadratic relations of the algebra $\hat A$ form a
Gr\"obner basis of the ideal of relations with respect to the above
order, that is, the algebras
 $\hat A$ and $A$ are PBW algebras.
 \end{prop}

\begin{proof}
The leading monomials of the relations~(\ref{eq:rel_A}) w.~r.~t. the above
reverse lexicographical order are
\begin{equation}
\label{eq:lead_terms}
\begin{array}{lll}
e_{in} x_n \mbox{ for } 1\le i <n, e_{n-1,n}x_{n-1},  & t \in 6 {\ensuremath{\mathbb Z}},\\
x_i x_j \mbox{ for } n\ge i > j \ge 1, & t \in 6 {\ensuremath{\mathbb Z}}+1, \\
x_1 e_{1i} \mbox{ for } n\ge i>1, x_{2} e_{12},  & t \in 6 {\ensuremath{\mathbb Z}}+2,\\
e_{1i} x_1 \mbox{ for } n\ge i>1, e_{12} x_{2} ,  & t \in 6 {\ensuremath{\mathbb Z}}+3,\\
x_i x_j \mbox{ for } 1 \le i < j \le n, & t \in 6 {\ensuremath{\mathbb Z}}+4, \\
x_n e_{in} \mbox{ for } 1\le i <n, x_{n-1} e_{n-1,n},  & t \in 6 {\ensuremath{\mathbb Z}}+5,\\
\end{array}
\end{equation}

Let $ B $ be the $Z$-algebra defined by the same generators as $A$
and the above monomial relations. We have ${\mbox{dim\,}} B_{ij} = {\mbox{dim\,}}
A_{ij}$ for $j=i, j=i+1, j=i+2$ and ${\mbox{dim\,}} B_{ij} \ge  {\mbox{dim\,}} A_{ij}$
for $j\ge i+3$.   By~\cite[Prop.~10.1 of Ch.~4]{PP}, to show that
$A$ is PBW  it is sufficient to check the equalities ${\mbox{dim\,}} B_{i,
i+3} = {\mbox{dim\,}} A_{i, i+3}$ for all $i$.

Because of the symmetry of the monomial relations above, we have
${\mbox{dim\,}} B_{i, i+3} = {\mbox{dim\,}} B_{i+3s, i+3s+3} $ for all $s\in {\ensuremath{\mathbb Z}}$.
  Since the algebra $A$ is
3-periodic, it is enough to show that the equalities  ${\mbox{dim\,}} B_{i,
i+3} = {\mbox{dim\,}} A_{i, i+3}$ hold for $i=0,1,2$. Here  ${\mbox{dim\,}} A_{i,i+3}
= {\mbox{dim\,}}\hat A_{i {\mbox{mod\,}} 3,i+3}$ is the coefficient of $t^3$ in the
$(i {\mbox{mod\,}} 3 ,i {\mbox{mod\,}} 3)$-th entry of the Hilbert series given in
Proposition~\ref{prop:Hilb_ser}, that is,
$$
{\mbox{dim\,}} \hat A_{0,3} = {\mbox{dim\,}} \hat A_{1,4} =
1-5/4\,{n}^{2}+1/4\,{n}^{4}
$$
or
$$
{\mbox{dim\,}} \hat A_{2,5} =  1+1/2\,{n}^{4}-1/2\,{n}^{3}-2\,{n}^{2}. 
$$

To find ${\mbox{dim\,}} B_{t, t+3}$, let us calculate the nonzero paths of
length 3 in the algebra $B$. The integers $i,j,k,l$ below  belong to the interval $[1,\dots,n]$.  We have
$$
{\mbox{dim\,}} B_{0, 3} $$
$$
= {\mathop{\mathrm{Card}}} \{ e_{ij}x_k x_l|i<j, k\le l\} - {\mathop{\mathrm{Card}}}
\{ e_{in}x_n x_l|i<n \} -{\mathop{\mathrm{Card}}}  \{ e_{n-1,n}x_{n-1} x_l|l \le
n-1\} $$
$$= \frac{n(n-1)}{2}\frac{n(n+1)}{2} - n(n-1)-(n-1) =
\frac{n^4}{4} -\frac{5n^2}{4}+1,
$$
$$
{\mbox{dim\,}} B_{1, 4} = {\mbox{dim\,}} B_{0, 3} = \frac{n^4}{4} -\frac{5n^2}{4}+1
\mbox{ by symmetry}
$$
and
$$
{\mbox{dim\,}} B_{2, 5} = {\mathop{\mathrm{Card}}} \{ x_i e_{jk} x_l |j<k \}
 - {\mathop{\mathrm{Card}}} \{ x_n e_{in} x_l|i<n \}
 -  {\mathop{\mathrm{Card}}} \{ x_{n-1} e_{n-1,n} x_l \}
 $$
 $$
  - {\mathop{\mathrm{Card}}} \{ x_i e_{1k} x_1|k>1\} -  {\mathop{\mathrm{Card}}} \{ x_i e_{12} x_2 \}
 +{\mathop{\mathrm{Card}}} \{ x_n e_{1n} x_1 \} $$
 $$=
 n^2\frac{n(n-1)}{2} - \frac{n(n-1)}{2} - n
 - \frac{n(n-1)}{2} - n +1 = \frac{n^4}2 -\frac{n^3}2-2n^2+1.
$$
We obtain the equality  ${\mbox{dim\,}}  B_{t, t+3} = {\mbox{dim\,}} \hat
A_{t,t+3}$ for each $t=0,1,2$, thus,  the algebra $A$ is PBW.
\end{proof}

\begin{rema}
There is another proof of Proposition~\ref{prop: A is PBW} based on the 
  Diamond Lemma and the Buchberger criterion for
Gr\"obner bases.
\end{rema}

\begin{rema}
Note that while the algebra $A$ is 3-periodic, we have shown only 
that $A$ is PBW as a 6-periodic algebra. We do not know whether  $A$
is  PBW as a 3-periodic algebra, that is, whether $\hat A$ is PBW or not.
\end{rema}

\subsection{Bounded processing}

\label{subs:proc}

 Given an algebra $R$ defined by a set of generators $X$ and a
Gr\"obner basis of relations $G$ (given an admissible order of monomials 
on $X$), one can identify each element of
the algebra with a linear combinations of the words on $X$ which are
normal (=irreducible) with respect to $G$.
Recall that the multiplication of normal words after this identification is defined as follows.
We may assume that the Gr\"obner basis $G$ is reduced, that is, each its element 
$g\in G$ has the form $g = \hat g - \bar g$, where $\hat g$ is its leading monomial and $-\bar g$
is a linear combination of lower monomials. 
Given two normal words $u$ and $v$, one applies (if possible) to the concatenation $uv$ a {\it reduction} 
by some element of the Gr\"obner 
$g\in G$, that is, one replaces a subword $\hat g$ in $uv$ by the noncommutative polynomial $(-\bar g)$. 
In the resulted noncommutative polynomial, one applies to all its nonzero terms additional reductions, etc.
After a finite number of steps, all nonzero terms of the 
 resulted noncommutative polynomial $u*v$ became irreducible.  By the definition of Gr\"obner basis,  
 the linear combination 
 of the normal words $u*v$ is defined uniquely and is 
 identified with the product of $u$ and $v$ in the algebra $R$. 
 
 Thus, one can consider the calculation of the product $u*v$ as a processing of some ``machine''
 (like a Turing machine, if $G$ is finite) which takes a concatenation $uv$ as an input,
 finds the first occurrence  of a leading monomial $\hat g$ of some element of the Gr\"obner basis, 
 replaces it by $-\bar g$, etc. Since the words $u$ and $v$ are normal, the first replaced subword $\hat g$
 should overlap the both parts $u$ and $v$ of $uv$. In the next steps, the region of processing in each subword should  overlap one of the words from some $\bar g$ which appeared in a previous step. 
 An algebra $R$ is said to be an {\em algebra of $r$-processing}~\cite{pi01} for some $r>0$ if the region of processing 
 do not spread beyond $r$ letters to the right from the beginning of the right part $v$ of the initial word $uv$, that is, for each pair of normal words $u$ and $v = ws$, where the word $w$ has length at least $r$,
 we have
 $$
     u*v = (u*w)s.
 $$

Note that the above definition is compatible with the standard assumptions of the Gr\"obner basis
theory for ideals in path algebras~\cite{ffg}. In this theory, it is assumed that the above
 set of generators $X$
of a path algebra $kQ$ of a quiver $Q$ consists of two parts, $X = V \cup E$, where $V$ is the set of vertices and $E$ is the set of arrows of the quiver $Q$. By definition, the normal words are the paths of the corresponding quiver, that is, the paths of length 0 which are  the vertices and the paths of positive length which are sequences of arrows.

 We see that the definition of length of a normal word in the quiver algebra 
 is slightly different from the general definition of length as the number of letters in word used above. 
However, if we try to check the property of $r$-processing (with $r\ge 1$) for a quotient of the path algebra by an ideal $I$ generated by linear combination of paths of positive length (quiver algebra $R = kQ /I$), then we can assume that the words $u,v$, and $w$ are paths of positive length. 
For these words $u,v$, and $w$, the both definitions of lengths coincide, so,
 one can use the second one. 
 
 In particular, 
the path algebra $kQ$ of any quiver $Q$ is an algebra of 1-processing. 
Another example of a quiver algebra with $r$-processing is our algebra $A$.
 

\begin{prop}
\label{prop:3-proc}
The algebra $A$ is an algebra of 3-processing with respect to the generators and the Gr\"obner basis introduced in Subsection~\ref{subs:PBW}. 
\end{prop}

\begin{proof}
Let $uv = \dots e_{ab}x_ix_je_{kl}x_sx_t\dots$ be a product of two normal words $u$ and $v$. Suppose that 
the right subword $x_sx_t\dots$ belongs to the second part $v$.
 It is sufficient to show that in 
each stage of the processing, the subword which begins with $x_t$ is stable. Since in each stage of the processing the region of processing is extended by at most one letter (because all leading monomials of the elements of the Gr\"obner basis have length 2), it is sufficient to show that this region does not rich $x_t$.

In the reductions w.~r.~t. the elements of the Gr\"obner basis $G$ from Subsection~\ref{subs:PBW}, any 
 two-letter word $c$ is reduced to a linear  combinations of another two-letter words $c'$ with the following properties:

(1)   each letter $e$ is replaced by some $e$, and each $x$ is replaced by some $x$;

(2) if $c = yz$ and $c' = y'z'$  for some letters $y,z,y',z'$, then $z\ge z'$.

In particular, we have 

(3') if $c= e_{\alpha \beta} x_{s}$, then $c'= e_{\alpha' \beta'} x_{s'}$ with $x_{s'} \le x_s$.

It follows from observation of  the list of the leading monomials of elements of $G$ given in~(\ref{eq:lead_terms})
that

(4) if the monomial $x_sx_t$ is irreducuble (i.~e., normal), then for each $x_{s'} \le x_s$ the monomial $x_{s'}x_t$ is irreducible too.

Since the monomial $x_sx_t$ is a subword of the normal word $v$, it is irreducible. 
 By~(1), in each stage of the processing this subword will be replaced by a linear combination of the words of the form $x_{s'}x_{t'}$, where $x_{s'} \le x_s$ by~(3'). By~(4), it follows that $x_{t'} = x_t$, that is,
 the region of processing does not reach $x_t$. It follows that $x_t$ and all letters to the right of it will be stable in each stage of the processing.   
\end{proof}

\begin{rema}
Due to the right--left symmetry in list of the leading monomials of the Gr\"obner basis, $A$ is an algebra of left $3$-processing as well.
\end{rema}

\begin{rema}
Note that the sufficient condition for $r$--processing given in~\cite[Prop.~2]{pi01} does not 
hold for the algebra $A$.
\end{rema}

\subsection{Coherence}

\begin{cor}
\label{cor:main}
The algebra $A$ is right and left coherent.
\end{cor}

\begin{proof}
The right coherence of $A$ follows from Proposition~\ref{prop:3-proc} and~\cite[Th.~8]{pi01}.
The left coherence 
follows from symmetry. 
\end{proof}

Note that it follows that $A$ is coherent both in graded and non-graded sense. 

\begin{rema}
\label{rem:uni-coh}
The property of $3$--processing implies also the following estimate for the degrees of relations of ideals~\cite[Prop.~7]{pi01}: 
If a right sided  ideal $I$ in $A$ is generated in degreed $\le d$ for some $d$, then its relations are concentrated in degrees~$\le d+6$. 
It follows that $A$ is universally coherent in terms of~\cite{pi4}, see also~\cite[Prop.~4.10]{pi4}. 
Similar linear estimates for the generators of the entries 
of the  minimal projective resolution for each
 finitely presented $A$--module follow from~\cite[Prop.~4.3]{pi4}. 
\end{rema}

\begin{thebibliography}{Biblio}

\bibitem[AS]{AS} M.~Artin and W.F. Schelter, {\it Graded algebras of global dimension 3,}
Adv. Math., {\bf 66} (1987), pp.~171--216

\bibitem[AZ]{AZ} M. Artin, J.  J. Zhang,
{\it Noncommutative projective schemes,} Adv. Math.,  {\bf  109}  (1994),  2,
p.~228--287

\bibitem[Bock]{bock} Bocklandt R., {\em Graded Calabi Yau algebras of dimension 3}, Journal of pure and applied algebra, {\bf 212} (2008), 1, pp.~14--32 (with Appendix by M. Van den Bergh)

 
  

\bibitem[Bon]{bondal} Bondal A. I.,  {\em Representation of associative algebras and coherent sheaves},  Izv. Akad. Nauk SSSR Ser. Mat. {\bf 53} (1989), 1, pp.~25--44 [Russian]; translation in
{\bf 34} (1990), 1, pp.~23–42

\bibitem[BP]{bp} A. I. Bondal, A. Polishchuk, {\em Homological properties of associative algebras: the method of helices} (Russian. Russian summary), Izv. Ross. Akad. Nauk Ser. Mat. 57 (1993), no. 2, 3--50; translation in
Russian Acad. Sci. Izv. Math. 42 (1994), 2, pp.~219--260

\bibitem[BVdB]{BvdB} A. Bondal, M. Van den Bergh, {\it  Generators and representability of
functors in commutative and noncommutative geometry,}  Mosc. Math.
J.,  {\bf 3}  (2003),  1, pp.~1--36, 258

\bibitem[CDV]{cdv} A. Connes, M. Dubois-Violette, {\em Yang-Mills algebra},
 Lett. Math. Phys., {\bf 61} (2002), 2, pp.~149--158

\bibitem[ELO]{elo}  A. I. Efimov, V. A. Lunts, D. O. Orlov, {\it Deformation theory of objects in homotopy and derived categories III: Abelian categories}, Adv. Math., {\bf 226} (2011), 5, pp.~3857--3911

\bibitem[FFG]{ffg} D. R. Farkas, C. D. Feustel, E. L. Green, 
{\em Synergy in the theories of Gr\"obner bases and path algebras}, Canadian Journal of Mathematics, 
{\bf 45} (1993), 4, pp.~727--739

\bibitem[H]{hers} Herscovich E., {\em  Representations of Super Yang-Mills Algebras}, Communications in Mathematical Physics, {\bf 320} (2012), 3,  pp.~783--820

\bibitem[HOZ]{hoz} He J., Van Oystaeyen F., Zhang Y., {\em Graded 3-Calabi-Yau algebras as Ore extensions of 2-Calabi-Yau algebras}, arXiv:1303.5293 (2013)

\bibitem[MS]{ms}
Movshev, M., Schwarz, A., {\em Algebraic structure of Yang-Mills theory}, The unity of mathematics, Progr.
Math., Vol. 244, Birkhauser Boston, 2006, pp.~473--523
 

\bibitem[P01]{pi01} D. Piontkovski, {\it Noncommutative Groebner bases,
          coherence of associative algebras, and divisibility
                 in semigroups,} Fundam. Prikl. Mat.,
           {\bf 7} (2001), 2, pp.~495--513 [Russian]

\bibitem[P05]{pi4} D. Piontkovski, {\it Linear equations over noncommutative graded rings,}
            J. Algebra,  {\bf 294}  (2005),  2, pp.~346--372

\bibitem[P08]{Pi1}	D. Piontkovski, {\it Coherent algebras and noncommutative projective lines},
Journal of Algebra, {\bf 319} (2008), 8, pp.~3280--3290

\bibitem[PP]{PP} Polishchuk, A. Positselski, L. {\em Quadratic algebras.} University Lecture Series, 37. American Mathematical Society, Providence, RI, 2005. xii+159 pp.

\bibitem[Po]{pol} A. Polishchuk, {\it Noncommutative proj and coherent algebras},
  Math. Res. Lett., {\bf   12}  (2005),  1, pp.~63--74

\bibitem[S]{smi} S. P. Smith, {\em A 3-Calabi-Yau algebra with $G_2$ symmetry constructed from the octonions},  	arXiv:1104.3824 (2011)

\end{thebibliography}

\end{document}

Note that it follows that $A$ is coherent both in graded and non-graded sense. 

The property of $3$--processing implies also an estimate for the degrees of relations of ideals in $A$, see~\cite[Prop.~7]{pi01}. 

\begin{cor}
If a right sided  ideal $I$ in $A$ is generated in degreed $\le d$ for some $d$, then its relations are concentrated in degrees~$\le d+6$. 
\end{cor}

It follows that $A$ is universally coherent in terms of~\cite{pi4}, see also~\cite[Prop.~4.10]{pi4}. 

