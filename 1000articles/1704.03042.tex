\documentclass[12pt]{amsart}
\usepackage{amsmath}
\usepackage{amsfonts,xcolor}
\usepackage{amssymb}
\usepackage[all]{xy}
\usepackage{amssymb}
\usepackage{subfigure}
\usepackage{graphicx}
\usepackage{chngcntr}
\counterwithin{figure}{section}
\numberwithin{equation}{section}

\makeatletter
\newtheorem*{rep@theorem}{{Theorem} \ref##}

\makeatother

   
   

   
   

   
   

   
   
   
   
   
   
   
   
   
   
   
   
   
   
   
   
   
   
   
   
   
   
   
   
   
   
   
   
   
   
   
   
   
   

   
   
   

   
   
   
   
   
   

   
   
   
   
 
   
   
   
   
   
   
   
   

   
   
   
   
   
   
   
   
        
   
   
   
   
    
   
   
   
   

   
   
   
   
   
   
   
   
   
   
   
   
   
   
                      
                      
                  
   
                      
   
   
                      
   
   
   
        
   
   
   

   
   
   
   
   
   
   
   
   
   
   
   
   
  

   
   
   
   
   
   
   

   
   
        
   
   
   
        
        
   
   
   

   
   

   
   
   
   
   
   
   
   
   

        
   
   

  
   
   
  
   
   
   
   

   
   
   

   
   

   
   
   
   
   
   
        
   
   
   
   

   
   
   
   
   
   
   
   
   
   
   
   
   
   
   

   

   

   

   

   

   

   

   
   

 
   
   

   
   
   
   
   

   
   
   
   
   
   
                     
   
   
                 

   
   
   
   
   
   
   
   
   
   
   
   
   
   
   

   

   
   
   
   
   
   
   
   
   
   
   
   
   
   
   
   
   
   
   
   
   
   

   
   
   
   
   

   
   
   
   
   
   
   
   
   
   
   
   
   
   
   
   
   

   
   
   

   
   

   
   
   

   

   

   
   

   
   
   

   
   
   
   
   
   
   
   

   
   
   
   
   
   

   
  

   
   

   

   

   

   
   
   

   
   

   
   
   
   
   
   

   

   
   

   
   
   
   
   
   
   
   
   
   
   
   
   
   
   
   
   
   
   
   
   
   
   
   
   

   
   
   
   
   
   
   
   
   
   
   
   
   
   
   
   
   
   
   

   
   
   

           
   
                       

   
                        
   
   
   
   

   
   
   
   
   
   
   
   

   
   
   
   
   
   
   
 
   
   
   
   
   
   
   

              
   
   
          
   
   

   
   

   
   
   
   
   
   

   

   
   
   

   
   
   
   
   
   
   
   
   
   
   
   
   
   
   

   
   
   
   
   
   

   

   

   
   
   
   
   
   
   
   
   
   
   
   
   

   
   
   

   
  
   

   

   
   
   
   
   

   

   
   
   
   
   
   

   
   

   
   
   

   

   
   
   

   
   

   
   
   
   

   

   
   
   

   
   
   
   
   
   
   

   
   
   
   
   
   
   
   
   

   
   

   
   
   
   
   
   

   
   
   
   
   
   
  
   
   
   
   

   
   
   
   
   
   
   
   
   
   
   

   
   
   
   

   
   
   
   
   
   

  
   
   
   
   
   
   

   
   
   
   

   
   

   
   
   
   
   
   
   
   
   
   
   
   
   
   
   
   
   
   
   
   
   
   
   
   
   
   
   
   
   
   
   
   
   
   
   
   
   
   
   
   
   
   
   
   
   
   
   
   
   
   
   
   

   
   
   
   
   
   
   
   
   
   
   

   
   
   
   
   
   
   
   
   
   
   
   
   
   
   
   
   
   
   

   
   
   

   
   
   
   

   

   

   
              
   
   

   
   

   
   
   
   
   
   
   
   
   
   
   
   
   
   
   
   
   
   
   
   

   
   
   
   
   
   
   
   
   
   
   
   

   
   
   

   
   
   
   

   
   
  
   
   
   
   
  
  
   
   
   
   
  
   
   

   
   
   
   
   
   

   
   
   
   
   
   
   
                 
                
   
   

   
   
   
   
   
   
   
   
   
   
   
   
   
   
   
   
   

   
   
   

   
   

   
   
   
   
   
   

   
   
   
   
   

   
   
 
   
   
   
   
   
   
   
   
   
   
   
   

   
   
   
   
   
   
   
   
   
   
   
   

   

   
   
   
   
   
   
   
   

   
   

   

   
   
   
   
   
   
   
   

   
   
   
   
   

   
   
   
   
   
   
   
   
   
   
   
   
   
   
   
   

   
   

   
   
   
  
   
   
   
   
   
   
   
   
   
         
   
   
   
   
   
   
   
   
   
   
   
   
\def\mv1{{{\boldsymbol M}}_v^1}   
   
   
   
   

   
   

   
   

   
   
   
   
   
   
   
   

   
   

   
   
   
   

   
   
   

   

   
   
   

   
   
   
   
   
   

   

   
                   
   
   
   
   
   
   
   
   
   
   
   
   
   
   
   
   
   
   
   
   
   
         
   
   
   
   
   
   
   
   
   
   
   
   
   
   
   
   
   
   
   

   
   

   

   

   
   

   
   
   
   

   
   
   
   
   
   
   
   

   
   

   

   
   

   
   
   
   
   
   
   
   
   
   
   

   

   

   
   
   
   
   
   
   
   
   
   
   
   
   
   
   
   
   
   
   
   
   
   
   
   
   
   
   
   
   
   
   
   
   
   
   
   
   
   
   
   
   
   
   
   
   
   
   
   
   
               
   
   
   
   
   
   
   
   

   
   
   
   
   
   
   
   
   
   
   
   
   
   
   
   
   
   
   
   
   
   
   
   

   
   
   
   
   
   
   
   
   
   
   

   

   
   
   
   

   
   

          

          
                  
   
   
                  
   
   
   
\def{\sqcup \hspace{-0.15em}\sqcup}1{{\makebox[2.3ex][s]{$\sqcup$\hspace{-0.15em}\hfill $\sqcup$}}}   
   
   
   
   
   
   
   
   
   
   
   
   
   
   
   
   
   
   
   

   
   
  
   

   
   
   
   

   

   
   
   
   
   

   

   
   
   
   

   
   
  
   
   
   
   
   
   
   
   
   
   
   
   
   
   
   
   
   
   
   
   
   
   
   
   
   
   
   
   
   
   
   
   
   
   
   

   
   
   
   
   

   
   
   
   
   
   
   
   

   

   
   
   
   
   

   

   

  

   
   
   
   
   
   

   
   
                
  
   
   
   
   
   
   

   
   
   
   
   
   

   
   

   
   
   
   
   
  
   
  
   
   
   
   
   
   
   
   
   
   
   
   
   
   
   
   
   
                  
   
   
   
   
   
   
   
   
   
   
   

   
   
   
   
   
   
   
   
   

   
   
   
   
   
   
   
   

   
   
   
   
   

   

   
   

   
   
   
   
   
   
   

   
   
   
   
   
   

   
   
   
   
   
   
   
   
   
   
   
   
   
   

   
   
   
               

   

   

   
   
   
   
   
   
   
   
   
   
   
   
   
   
   
   
   
   
   
   
   
   
    
   
   
   
   
   
   
   
   
   
   
   
   
   
   
   
   
   
   
   
   
   
   
   
   
   
   
   
   
   
   
   
   
   
   
   
   
   

   
   
   
   
   
   
   
   
   
   
   
   
   
   
   
   
   
   
   
   
   
   
   
   
   
   
   
   
   
   
   
   
   
   
   
   
   
   
   
   

   
   
   
   
   
   
   

   
   
   
   
   
  
   
   
   
   
   
   
   
   
   

   
   
   
   
   
   
   
   
   
   

   

   

   

   
   
   
   
   
   
   

   

   
   
   
   
   
   
   
   

   
   
   
   

   
   
   
   

\addtolength{\oddsidemargin}{-2cm}
\addtolength{\evensidemargin}{-2cm}
\addtolength{\textwidth}{3.5cm}
\addtolength{\topmargin}{-1cm}
\addtolength{\textheight}{2cm}
\linespread{1.2}

{
\newenvironment{rep{theorem}}[1]{
 }
 \begin{rep@theorem}}
 \end{rep@theorem}}
\newtheorem{lemma}{Lemma}[section]
\newtheorem{theorem}[lemma]{Theorem}
\newtheorem{coro}[lemma]{Corollary}
\newtheorem{prop}[lemma]{Proposition}
\newtheorem{proposition}[lemma]{Proposition}
\newtheorem{claim}[lemma]{Claim}
\newtheorem{rem}[lemma]{Remark}
\newtheorem{remark}[lemma]{Remark}
\newtheorem{definition}[lemma]{Definition}
\newtheorem{example}[lemma]{Example}

\subjclass{}
\keywords{}
\thanks{
L. D. A. was supported by the Austrian Science Fund (FWF): START-project FLAME (”Frames and Linear Operators for 
Acoustical Modeling and Parameter Estimation”, Y 551-N13). K.\ G.\ was supported in part by the  project P26273 - N25 
of the Austrian Science Fund (FWF), and J. L. R. gratefully acknowledges support from the Austrian Science Fund (FWF): P 
29462 - N35.}

\begin{document}
\title{Harmonic analysis in phase space and finite Weyl-Heisenberg ensembles}
\author{Lu\'{\i}s Daniel Abreu}
\address{Acoustics Research Institute\\ Austrian Academy of Sciences\\ Wohllebengasse 12-14, Vienna,
1040, Austria}
\author{Karlheinz Gr\"{o}chenig}
\address{Faculty of Mathematics \\
University of Vienna \\
Oskar-Morgenstern-Platz 1 \\
A-1090 Vienna, Austria}
\author{Jos\'{e} Luis Romero}
\address{Acoustics Research Institute\\ Austrian Academy of Sciences\\ Wohllebengasse 12-14, Vienna,
1040, Austria}
\date{}

\begin{abstract}
Weyl-Heisenberg ensembles are determinantal point processes associated with the Schr\"odinger
representation of the Heisenberg group, and include as examples the Ginibre ensemble
and the  polyanalytic ensembles, which are related to the  higher
Landau levels in physics. We introduce finite
versions of the Weyl-Heisenberg  ensembles and show that they behave
analogously to the finite Ginibre ensembles. 
The construction does not rely on explicit formulas but rather on phase-space methods, and recovers
asymptotically the finite polyanalytic ensembles introduced by Haimi and Hedenmalm. As an
application we obtain quantitative and non-asymptotic estimates for
the one-point intensities of the finite polyanalytic ensembles.
\end{abstract}

\maketitle

\section{Introduction}
\subsection{Weyl-Heisenberg ensembles}

We study the class of {determinantal point process} es on $\mathbb{R}^{2d}$ whose correlation kernel is given as
\begin{equation} \label{eq:l1}
{{K^g}}((x,\xi),(x',\xi '))=\int_{\mathbb{R}^{d}}e^{2\pi i(\xi^{\prime }
-\xi)t}g(t-x^\prime) \overline{g(t-x )}dt
\end{equation}
for some non-zero function $g\in L^2({{{\mathbb{{R}}}}^d} )$ and $(x,\xi), (x',\xi') \in \mathbb{R}^{2d} $. These
{determinantal point process} es are called Weyl-Heisenberg ensembles (WH ensembles)  and have been introduced recently
in \cite{APRT}. They form a large class of translation-invariant hyperuniform point
processes \cite{PHRE2009, ghosh1}.

The prototype of a Weyl-Heisenberg ensemble is the complex
Ginibre ensemble. Choosing $g$ in~\eqref{eq:l1} to be
the Gaussian  $g(t)=2^{1/4}e^{-\pi t^{2}}$ and writing $z=x+i\xi, z'= x'+i\xi' $, the
resulting kernel is then

\begin{equation}
\label{eq_intro_kg}
{{K^g}}(z,z')=e^{i\pi (x'\xi'-x\xi )}
e^{-\frac{\pi}{2}({\ensuremath{\left| {z} \right| }}^{2}+{\ensuremath{\left| {z'} \right| }}^{2})}
e^{\pi \overline{z} z'}.
\end{equation}
Thus, up to a gauge transformation (conjugation with a unimodular phase factor),
the identification ${{\mathbb{{C}}}}^d={{\mathbb R}}^{2d}$,
and the change of variables $(z,z') \mapsto (\overline{z},\overline{z'})$,
${{K^g}}$ coincides with the kernel of  the \emph{infinite
Ginibre ensemble}  $K_\infty (z,z') = e^{-\frac{\pi}{2} (\left\vert z\right\vert
^{2}+\left\vert z'\right\vert ^{2})}e^{\pi z\overline{z'}}$.
Another important class of examples arises by choosing $g$ to be
a Hermite function. In this case one obtains the so-called
polyanalytic Ginibre ensembles studied in~\cite{Dunne, HendHaimi, SHIRAI}, which model the
electron density in the higher Landau levels (see Section \ref{sec_pure_pol} for some background).

In this article, we study finite dimensional versions of Weyl-Heisenberg ensembles.
The Ginibre ensemble with kernel
$K_\infty$ arises as limit of corresponding
processes with $N$ points, whose kernels
\begin{equation}
K_{N}(z,z')=e^{-\frac{\pi}{2} (\left\vert z\right\vert ^{2}+\left\vert
z'\right\vert ^{2})}\sum_{j=0}^{N-1}\frac{\left( \pi z\overline{z'}\right) ^{j}
}{j!}, \label{finite}
\end{equation}
are obtained simply by truncating the expansion of the exponential
 $e^{\pi z\overline{z'}}$. It is not obvious how to obtain the
 analogous finite-dimensional process for a general {Weyl-Heisenberg }\ ensemble
 \eqref{eq:l1}, because for
most choices of $g\in L^{2}({\mathbb{R}}^d)$ there is no
explicit formula available for ${{K^g}}$.
Our goal is to present a canonical
construction of finite Weyl-Heisenberg ensembles with properties similar to the finite Ginibre
ensemble. 

Our contribution is two-fold: on the one hand, we introduce  a new
class of determinantal point processes and investigate selected
properties, on the other hand, we introduce a new set of tools, namely
the harmonic analysis of phase space, to the theory of point
processes. 

\subsection{Construction of finite WH ensembles}
\label{sec_intro_cons}
In the absence of explicit formulas, we rely on methods from
harmonic analysis on phase space \cite{folland89, MR2226126} and on the spectral analysis of
phase-space localization operators. Write $z=(x,\xi) \in {{{{\mathbb{{R}}}}^{2d}}}, z' = (x',\xi ') \in {{{{\mathbb{{R}}}}^{2d}}} $
for a point in phase space and
\begin{equation}
  \label{eq:l2}
\pi (z)f(t):=e^{2\pi i\xi t}f(t-x)
\end{equation}
for the phase-space shift by $z$. Then the kernel in \eqref{eq:l1} is given by
\begin{equation}
  \label{eq:l3}
  {{K^g}}(z,z') = \langle \pi (z')g, \pi (z)g\rangle.
\end{equation}
Let us now describe the construction of the finite point processes
associated with the kernel ${{K^g}}$. For normalized $g\in L^2({{{\mathbb{{R}}}}^d} )$, $\|g\|_2 =
1$, the integral operator with kernel ${{K^g}}$ is an orthogonal projection
(see for example ~\cite[Chapter 1]{folland89}, \cite[Chapter 9]{Charly}). Its range
${\mathcal{V}}_g \subseteq L^2({{{{\mathbb{{R}}}}^{2d}}} )$ is the closed subspace
\[
{\mathcal{V}}_g
=\big\{F \in L^2({{{{\mathbb{{R}}}}^{2d}}} ): F(z) = \langle f, \pi (z) g\rangle,
\mbox{ for } f \in L^2({{{\mathbb{{R}}}}^d})\,\big\} \subseteq {L^{2}({\mathbb{R}^{2d}})},
\]
and the waveforms $F$ are phase-space representations of functions $f$ defined on the
configuration space $\mathbb{R}^d$.

\emph{Step 1: Concentration as a smooth restriction}. Let $\mathcal{X}^g$ be a WH ensemble
(with correlation kernel ${{K^g}}$) and let $\Omega \subseteq {{{{\mathbb R}}^{2d}}}$ be a measurable
set. The restriction of $\mathcal{X}^g$ to $\Omega$ is a
determinantal point process (DPP) $\mathcal{X}^g_{|\Omega}$ with
correlation kernel
\begin{align}
\label{eq_intro_ker_o}
{{K^g}}_{|\Omega}(z,z') = 1_\Omega(z) {{K^g}}(z,z') 1_\Omega(z').
\end{align}
An expansion of the kernel ${{K^g}}_{|\Omega}$ can be obtained as follows.
We consider the \emph{Toeplitz operator} on ${\mathcal{V}}_g$ defined by
\begin{align}
\label{Toe0}
M_{\Omega }^{g}F(z) &=\int_{\Omega }F(z'){{K^g}}(z,z')\,dz'
\\
&=
\label{Toe}
\int_{{{{\mathbb R}}^{2d}}} F(z')
\left[
\int_{{{{\mathbb R}}^{2d}}} {{K^g}} (z,z'') 1_\Omega (z'')  {{K^g}} (z'',z') \, dz'' \right] \, dz'.
\end{align}
The identity of \eqref{Toe0} and  \eqref{Toe} holds for  $F \in
{\mathcal{V}}_g$. On the other hand, if  $F \in {\mathcal{V}}_g^\perp$, then
the expression in \eqref{Toe} vanishes. 
For $\Omega \subseteq {{{{\mathbb{{R}}}}^{2d}}}$ of finite measure,
$M^g_\Omega $ is a compact positive (self-adjoint) operator on
$L^2({{{{\mathbb{{R}}}}^{2d}}} )$; see for example ~\cite{cogr03}. By the spectral theorem,
$M^g_\Omega$ is diagonalized by an orthonormal set $\{p_{g,j}^{\Omega }: j \in {{\mathbb{{N}}}} \}$ of
eigenfunctions, with corresponding eigenvalues $\lambda _j =
\lambda_{j}^{\Omega }$ (ordered non-increasingly):
\begin{equation}
\label{eq_eigenexp}
M_{\Omega}^{g} = \sum_{j \geq 1} \lambda_{j}^{\Omega } \,
p_{g,j}^{\Omega } \otimes p_{g,j}^{\Omega }.
\end{equation}
The key property is that the eigenfunctions $p_{g,j}^{\Omega }$ are \emph{doubly-orthogonal}:
since ${\ensuremath{\left<{M_{\Omega}^{g} F},{F}\right>}} = \int_\Omega {\ensuremath{\left| {F} \right| }}^2$,
\begin{align*}
{\ensuremath{\left<{p_{g,j}^{\Omega }},{p_{g,j'}^{\Omega }}\right>}}_{L^2(\Omega)}
={\ensuremath{\left<{M_{\Omega}^{g} p_{g,j}^{\Omega}},{p_{g,j'}^{\Omega }}\right>}}_{L^2({{{{\mathbb R}}^{2d}}})}
= \lambda_{j}^{\Omega } \delta_{j,j'},
\end{align*}
and consequently the restricted kernel can be orthogonally expanded as
\begin{align}
\label{eq_intro_ker_o2}
{{K^g}}_{|\Omega}(z,z') = \sum_{j \geq 1} \left( p_{g,j}^{\Omega}(z)1_\Omega(z) \right)
\cdot
\left( \overline{p_{g,j}^{\Omega}(z')}1_\Omega(z') \right);
\end{align}
see Section \ref{sec_duality} for details. Note that in \eqref{eq_intro_ker_o2},
the functions $p_{g,j}^{\Omega}(z)1_\Omega(z)$ are not normalized. In fact,
\begin{align} \label{eq:o11}
\int_\Omega {\ensuremath{\left| {p_{g,j}^{\Omega}(z)} \right| }}^2 dz = \lambda_{j}^{\Omega }.
\end{align}
Thus, while in \eqref{eq_intro_ker_o2} the basis functions
are restricted to the domain $\Omega$, the expansion of the
Toeplitz operator \eqref{eq_eigenexp} involves the non-truncated functions $p_{g,j}^{\Omega}(z)$
weighted by the measure of their concentration on $\Omega$. We call the DPP with correlation
kernel corresponding to \eqref{Toe} 
the \emph{concentration} of the full WH ensemble to $\Omega$
and denote it by $\mathcal{X}^{g,\rm{con}}_{\Omega}$. This process is thus a smoother variant of
the restricted process $\mathcal{X}^{g}_{|\Omega}$.
The construction of DPPs from the spectrum of self-adjoint operators has been suggested as an
analogue of the construction of DPPs from the spectral measure of a group \cite{BO2}.

\emph{Step 2: Spectral truncation}. The eigenvalues $\lambda_{j}^{\Omega }$ describe the best
possible simultaneous phase-space concentration of waveforms within $\Omega$.
Indeed, since ${\ensuremath{\left<{M_{\Omega}^{g} F},{F}\right>}} = \int_\Omega {\ensuremath{\left| {F} \right| }}^2$, by the min-max principle,
\begin{align}
\label{eq_minimax}
\lambda^\Omega_j =
\max \left\{\int_\Omega {\ensuremath{\left| {F(z)} \right| }}^2 dz: {\lVert{F}\rVert}_2=1, F \in {\mathcal{V}}_g,
F \perp p_{g,1}^{\Omega }, \ldots, p_{g,j-1}^{\Omega}
\right\}.
\end{align}
It is well-known that there are $\approx {\ensuremath{\left| {\Omega} \right| }}$ large $\lambda_{j}^{\Omega }$.
For example, for any $\delta \in (0,1)$,
\begin{equation}
  \label{eq:l4}
\left\vert \#\{j:\lambda _{j}^{\Omega }>1-\delta \}-\left\vert \Omega
\right\vert \right\vert \leq C_{g,\delta}
{{\ensuremath{\left| {\partial {\Omega}} \right| }}_{2d-1}},
\end{equation}
where ${{\ensuremath{\left| {\partial {\Omega}} \right| }}_{2d-1}}$ is the perimeter of $\Omega$ (the
surface measure of its boundary),
and $C_{g,\delta} $ is a constant depending explicitly on $g$ and $\delta$
(see for instance \cite[Proposition ~3.4]{AGR}).

\begin{figure}
\centering
\subfigure{
\includegraphics[scale=1]{./dom.png}
}
\subfigure
{
\includegraphics[scale=0.3]{./eigenvals.png}
}
\caption{A plot of the eigenvalues of the Toeplitz operator $M^g_\Omega$,
with $g$ a Gaussian window and $\Omega$ of area $\approx$ 18.}
\label{fig_dom}
\end{figure}

\begin{figure}
\centering
\includegraphics[scale=0.4]{./funcs.png}
\caption{The eigenfunctions \# 1, 7, 18 corresponding to the operator in Fig. \ref{fig_dom}}
\end{figure}

We now look into the concentrated process $\mathcal{X}^{g,\rm{con}}_{\Omega}$ introduced in
Step 1.
The Toeplitz operator $M^g_\Omega$ is not a projection. However, the corresponding
DPP can be realized as a random mixture of DPP's associated with projection kernels \cite[Theorem
4.5.3]{DetPointRand}. Indeed, if $I_j \sim \mbox{Bernoulli}(\lambda_{j}^{\Omega })$ are independent
(taking the value $1$ or $0$ with probabilities $\lambda_{j}^{\Omega }$ and $1-\lambda_{j}^{\Omega
}$ respectively),
then $\mathcal{X}^{g,\rm{con}}_{\Omega}$ is generated by the kernel corresponding to the random
operator
\begin{equation}
M_{\Omega}^{g, \mathrm{ran}} = \sum_{j \geq 1} I_j \cdot p_{g,j}^{\Omega } \otimes p_{g,j}^{\Omega
}.
\end{equation}
Precisely, this means that one first chooses a realization of the $I_j$'s and then a realization
of the DPP with the kernel above. Because of \eqref{eq:l4}, the first eigenvalues $\lambda_j$ are
close to $1$ and thus the corresponding $I_j$ will most
likely be $1$. Similarly, for $j \gg {\ensuremath{\left| {\Omega} \right| }}$, the corresponding $I_j$ will most likely be $0$.
As a finite-dimensional model for WH ensembles, we propose replacing the
random Bernoulli mixing coefficients with
\begin{eqnarray}
\label{eq_outcome}
\left\{
\begin{aligned}
&1, &\mbox{ for } j \leq {\ensuremath{\left| {\Omega} \right| }},
\\
&0, &\mbox{ for } j > {\ensuremath{\left| {\Omega} \right| }}.
\end{aligned}
\right.
\end{eqnarray}

\begin{definition}
\label{def_intro_wh}
Let $g \in L^2({{{{\mathbb R}}^d}})$ be of norm 1, let $\Omega \subseteq {{{{\mathbb R}}^{2d}}}$
with non-empty interior and
finite measure and perimeter, and let
$N_{\Omega }=\left\lceil \left\vert \Omega
\right\vert \right\rceil $  the least integer greater than or equal to  the Lebesgue
measure of $\Omega $. The  \emph{finite
Weyl-Heisenberg ensemble} is the {determinantal point process}\  $\mathcal{X}^g_\Omega$ with
correlation kernel \footnote{We do not denote this kernel by $K^g_{\Omega }$ in order to avoid a
possible confusion
with the
restricted kernel ${{K^g}}_{|\Omega}$. Note also the notational difference between the finite
ensemble
$\mathcal{X}^g_\Omega$
and the restriction of the infinite ensemble ${\mathcal{X}^g}_{|\Omega}$.}
\begin{equation*}
K_{g,\Omega }(z,z')=\sum_{j=1}^{N_{\Omega }}p_{g,j}^{\Omega }(z)
\overline{p_{g,j}^{\Omega }(z')}.
\end{equation*}
\end{definition}

To illustrate the construction, consider $g(t) = 2^{1/4} e^{-\pi t^2}$ and $\Omega = D_R = \{
z\in {{\mathbb{{C}}}} : |z| \leq R\} $. In this case the Toeplitz operator $M_\Omega ^g$ is unitarily equivalent
both to
a Toeplitz operator on Bargmann-Fock space and to the anti-Wick (Berezin) quantization of
the symbol $1_\Omega $~\cite[Chapter 2]{folland89}. The eigenfunctions of $M_{B_R}^g$ are
explicitly given as $p_{g,j}^{B_R}(\overline{z}) = e^{\pi i x \xi} (\pi ^{j}/j!)^{\frac{1}{2}}z^{j}
e^{-\pi |z|^2/2}$, $z=x+i\xi$. Remarkably, they are independent of the radius $R$ of the disk, and
choosing $R$ such that ${\ensuremath{\left| {D_R} \right| }}=N$, the corresponding finite WH ensemble is precisely
the finite Ginibre ensemble given by~\eqref{finite}. See Corollary \ref{coro_ident_3} for details.

\subsection{Approximation and the thermodynamical limit}
We now discuss how finite WH ensembles behave when the number of points tends to infinity.
Let
\[\rho _{g,\Omega }(z) = K_{g,\Omega }
  (z,z) = \sum_{j=1}^{N_{\Omega }}|p_{g,j}^{\Omega }(z) |^2\]
  be the
  one-point intensity of a finite Weyl-Heisenberg ensemble, so
  that
$$
\int _D \rho _{g,\Omega } (z) dz =
\mathbb{E}\left[ \mathcal{X}^g_\Omega(D) \right]  \,
$$
is the expected number of points to be found in $D \subseteq {{{{\mathbb{{R}}}}^{2d}}}$
(see Section \ref{sec_det}). The following describes the
asymptotic behavior of these one-point
intensities under scaling.
\begin{theorem} \label{tl1}
 Let $\Omega \subset {\mathbb{R}^{2d}}$ be compact.
Then the $1$-point intensity of the finite Weyl-Heisenberg ensemble satisfies
\begin{equation}\label{l5}
\rho _{g,m\Omega }(m\cdot )\longrightarrow 1_{\Omega },
\end{equation}
in $L^{1}({\mathbb{R}^{2d}})$, as $m\longrightarrow +\infty $.
\end{theorem}

\begin{figure}
\centering
\includegraphics[scale=0.5]{./onepoint.png}
\caption{The one-point intensity of a WH ensemble plotted over the domain
in Fig. \ref{fig_dom}.}
\label{fig_onepoint}
\end{figure}

Theorem~\ref{tl1} follows immediately from \cite[Theorem 1.3]{AGR}, once the one-point intensity
$\rho_{g,\Omega}$ is recognized as the so-called accumulated spectrogram \cite[Definition
1.2]{AGR}. In the context of {determinantal point process} es, Theorem~\ref{tl1} amounts to a generalization of the circular
law for the Ginibre ensemble. See Figure \ref{fig_onepoint} for an illustration.

(i) When $g(t) = 2^{1/4} e^{-\pi t^2}$ and $\Omega$ is a disk of area $N$, Theorem~\ref{tl1}
expresses nothing but the circular law for the Ginibre ensemble.

(ii)  The asymptotics are not restricted to disks, but hold
for arbitrary sets $\Omega $ with finite measure and also hold in
arbitrary dimension, not just for planar {determinantal point process} es.

(iii) The limit distribution in~\eqref{l5} is independent of the
parameterizing function $g$. This can be seen as yet another instance
of the universality phenomenon~\cite{Deift, TVK}.

There are several ways to quantify the error in the circular law, see, e.g., \cite[Theorem
1.4]{AMH2}, \cite[Theorem 1.4]{AMH1}, \cite{TVK}. In view of Theorem~\ref{tl1} we will measure the
error in the $L^1$-norm. Under a mild assumption on the phase-space concentration of $g$, one can
show a non-asymptotic error estimate, which was proved in \cite{APR} improving on a result from
\cite{AGR}.
\begin{theorem}
\label{th_quant_one}
Let $\rho _{g,\Omega }$ be the one-point intensity of the finite
Weyl-Heisenberg ensemble. If $g$ satisfies the condition
$$
\lVert g\rVert _{M^{\ast }}^{2}:=\int_{\mathbb{R}^{2d}}\left\vert
z\right\vert \left\vert \langle g, \pi (z)g\rangle \right\vert ^{2}dz<+\infty,
$$
$\Omega$ has finite perimeter and ${{\ensuremath{\left| {\partial {\Omega}} \right| }}_{2d-1}} \geq 1$,
then
\begin{equation} \label{l7}
\lVert \rho _{g,\Omega}-1_{\Omega}\rVert _{1}\leq C_{g}
{{\ensuremath{\left| {\partial {\Omega}} \right| }}_{2d-1}}
\end{equation}
  with a constant depending only on $\|g\|_{M^*}$.
\end{theorem}

This error rate is optimal - see \cite[Theorem 1.6]{APR}. Intuitively, in
\eqref{l7} we compare the
continuous function $\rho _{g,\Omega}$ with the characteristic function
$1_{\Omega}$. Thus along every point of the boundary of $\Omega$ (of
surface measure  ${{\ensuremath{\left| {\partial {\Omega}} \right| }}_{2d-1}}$) we
accumulate a pointwise  error of  $\mathcal{O} (1)$, leading to a total $L^1$-error
at least of order ${{\ensuremath{\left| {\partial {\Omega}} \right| }}_{2d-1}}$.

\subsection{Planar Hermite ensembles}
The complex Hermite polynomials are given by
\begin{equation}
H_{j,r}(z,\overline{z})=\left\{
\begin{tabular}{l}
${\sqrt{\frac{r!}{j!}}\pi ^{\frac{j-r}{2}}z^{j-r}L_{r}^{j-r}\left( \pi
\left\vert z\right\vert ^{2}\right), \qquad
j>r \geq 0,}$
\\
${\left( -1\right) ^{r-j}\sqrt{\frac{j!}{r!}}\pi ^{\frac{r-j}{2}}\overline{
z}^{r-j}L_{j}^{r-j}\left( \pi \left\vert z\right\vert ^{2}\right), \qquad 0 \leq j\leq r}$,
\end{tabular}
\right. \label{ComplexHermite}
\end{equation}
where $L_{r}$ denotes the Laguerre polynomials
\begin{align}
\label{eq_lag}
&L_{j}^{\alpha}(x)=\sum\limits_{i=0}^{j}(-1)^{i}\binom{j+\alpha }{j-i}\frac{x^{i}}{i!},
\qquad x\in {\mathbb{R}}, \qquad j \geq 0, j+\alpha \geq 0.
\end{align}
Complex Hermite polynomials satisfy the doubly-indexed orthogonality
\begin{equation*}
\int_{\mathbb{C}}H_{j,r}(z,\overline{z})\overline{H_{j\prime ,r\prime }(z,\overline{z})}e^{-\pi
\left\vert z\right\vert ^{2}}dz={\delta }_{jj\prime }{\delta }_{rr\prime },
\end{equation*}
and provide a basis for the space $L^{2}\left( \mathbb{C},
e^{-\pi \left\vert z\right\vert ^{2}}\right)$ \cite{Ismail, MR3537240}
\footnote{Perelomov \cite{Perelomovbook} mentions that \eqref{ComplexHermite} has been used by
Feynman and Schwinger
as the explicit expression for the matrix elements of the displacement operator in Fock space.}.
This relation allows one to
consider a variety of
associated ensembles.
\begin{definition}
Let $J \subseteq{{\mathbb{{N}}}}_{0}\times {{\mathbb{{N}}}}_{0}$. The planar Hermite ensemble based on $J$
is the {determinantal point process}\ with the  correlation kernel
\begin{equation}
K(z,z')=e^{-\frac{\pi}{2} ({\left\vert z\right\vert ^{2}+\left\vert z'\right\vert
^{2}})}\sum_{j,r\in J}H_{j,r} \left(z,\overline{z} \right)
\overline{
H_{j,r} \left( z',\overline{z'}
\right)}.
\label{complexhermiteensembles}
\end{equation}
\end{definition}
Several important {determinantal point process} es arise as special cases
of (\ref{complexhermiteensembles}). First note that, since
$H_{j,0}(z,\overline{z})=(\pi^{j}/j!)^{\frac{1}{2}}z^{j}$, selecting the set
$J=\{0,\ldots,N-1\}\times \{0\}$ in
\eqref{complexhermiteensembles}, leads to the kernel of the
Ginibre ensemble \eqref{finite}. A second important example arises if we consider
$J := \{(j,r): 0 \leq j \leq n-1, r=m-n+j\}$ with $n,m \in \mathbb{N}$. The corresponding
one-point intensity
\begin{align*}
\sum_{j =0}^{n-1} \frac{j!}{(m-n+j)!}
\big(\pi {\ensuremath{\left| {z} \right| }}^2 \big)^{m-n} L^{m-n}_j\big(\pi{\ensuremath{\left| {z} \right| }}^2 \big)^2 e^{-\pi {\ensuremath{\left| {z} \right| }}^2}
\end{align*}
is a radial version of the marginal p.d.f. of the unordered eigenvalues of a complex Gaussian
Wishart matrix - after the change of variables $t \to \pi {\ensuremath{\left| {z} \right| }}^2$ - see, e.g. \cite[Theorem
2.17]{verdu}.

\subsection{Finite pure polyanalytic ensembles}
The complex Hermite polynomials are an example of \emph{polyanalytic functions} - that is,
polynomials in $\overline{z}$ with analytic coefficiets (see Section \ref{app_poly}).
Haimi and Hedenmalm \cite{HendHaimi,HaiHen2} introduced polyanalytic versions of the finite Ginibre
ensemble and decomposed them into orthogonal components known as (finite) \emph{pure polyanalytic
ensembles} (see Section \ref{sec_pure_pol}). Finite pure polyanalytic ensembles are finite
versions of the polyanalytic ensembles introduced by Shirai \cite{SHIRAI},
which model the distribution of electrons in higher Landau levels.

By \cite[Proposition 2.1]{HendHaimi}, pure polyanalytic ensembles are planar Hermite ensembles with
$J=\{0,\ldots,N-1\}\times \{r\}$. Concretely, the finite \emph{$(r,N)$-pure polyanalytic ensemble}
is the
DPP with correlation kernel \begin{equation}
\label{purekernel}
K_{r,N}(z,z')= e^{-\frac{\pi}{2} (\left\vert z\right\vert ^{2}+\left\vert
z'\right\vert ^{2})}
\sum_{j=0}^{N-1} H_{j,r}
\left(z,\overline{z} \right)
\overline{H_{j,r}\left(z',\overline{z'}\right)}.
\end{equation}
To clarify the relation to the finite WH ensembles, we consider again a gauge transformation and
the change of variables
$f^*(z):=f(\overline{z})$, $z \in {{\mathbb{{C}}}}^d$. Given an operator $T: L^2({{{{\mathbb R}}^{2d}}}) \to
L^2({{{{\mathbb R}}^{2d}}})$ we denote:
\begin{align}
\label{eq_tilde_1}
\left[\widetilde{T} f \right]^* := \overline{m} \cdot T(f^* \cdot m), \qquad m(x,\xi) := e^{-\pi i x
\xi}.
\end{align}
Hence, if $T$ has integral kernel $K$, then $\widetilde{T}$ has integral kernel
\begin{align}
\label{eq_tilde_2}
\widetilde{K}(z,z') = e^{\pi i (x' \xi' - x \xi)} K \left(\overline{z},\overline{z'} \right),
\qquad z=x+ i \xi,\, z'=x'+i\xi'.
\end{align}
We call the operation $K \mapsto \widetilde K$ a renormalization of the kernel $K$.
With this notation, if ${{K^g}}$ is the kernel in \eqref{eq_intro_kg}
and $g$ is the Gaussian window, then $\widetilde{K}_g$ is the infinite Ginibre kernel. In addition,
the DPP's on ${{\mathbb{{C}}}}^d$ associated with the kernels $K$ and $\widetilde{K}$ are related by the
transformation $z \mapsto \overline{z}$.

After this preparation, let the window $g$ be a Hermite function
\begin{align}
\label{eq_hermite}
h_{r}(t) = \frac{2^{1/4}}{\sqrt{r!}}\left(\frac{-1}{2\sqrt{\pi}}\right)^r
e^{\pi t^2} \frac{d^r}{dt^r}\left(e^{-2\pi t^2}\right), \qquad r \geq 0.
\end{align}
The corresponding kernel $K_{h_r}$ describes (after the renormalization above) the orthogonal
projection onto the so-called Fock space of pure polyanalytic functions of type $r$ (see Section
\ref{app_poly}).

Let us consider a Toeplitz operator on $L^2(\mathbb{R}^2)$ with a circular domain $\Omega =
D_R$. By means of an argument based on phase-space symmetries (more precisely, the symplectic
covariance of Weyl's quantization) we show in Section \ref{sec_hd} that the eigenfunctions
$\{\widetilde p_{h_r,j}^{D_R}: j \geq 1\}$ of
$\widetilde{M}^{h_r}_{D_R}$ are the  normalized complex
Hermite polynomials $H_{j,r}(z,\bar{z}) e^{-\frac{\pi}{2} |z|^2}$.
In particular, as with the Ginibre ensemble, the eigenfunctions are
independent of the radius $R$. Choosing $R$ such
that $N_{D_R} = N$, and recalling that we order the eigenvalues of $M_{B_R}^{h_r}$ by magnitude,
we obtain a map ${\sigma}: {{\mathbb N}} _0 \to {{\mathbb N}}_0$, such that
\begin{align*}
\widetilde p_{h_r,j}^{D_R}=H_{{\sigma}(j),r}(z,\bar{z})
e^{-\frac{\pi}{2} |z|^2}.
\end{align*}
Thus, the finite WH ensemble associated with $h_r$ and $D_R$ is
a planar Hermite ensemble, with correlation kernel described by
\begin{equation}
\label{eq_a}
\widetilde K_{h_r,D_{R}}(z,z')=e^{-\frac{\pi}{2} (\left\vert z\right\vert ^{2}+\left\vert
z'\right\vert ^{2})}\sum_{j=1} ^{N_{D_{R}} } H_{{\sigma} (j),r}(z,\overline{z})
\overline{H_{{\sigma} (j),r}(z',\overline{z'})}.
\end{equation}
Comparing the correlation kernels of the finite pure polyanalytic ensemble \eqref{purekernel}
on the one hand and the finite (renormalized) WH ensemble with a
Hermite window \eqref{eq_a} on the other, we see that in each case different subsets of the complex
Hermite basis intervene: in one case functions are ordered according to their Hermite index, while
in the other they are ordered according to the magnitude of their eigenvalues.

Figure \ref{fig_val} shows the eigenvalues of $\widetilde{M}^{h_1}_{D_R}$, as a function of $R$,
corresponding to the eigenfunctions $H_{0,1}(z,\overline{z}) e^{-\tfrac{\pi}{2}{\ensuremath{\left| {z} \right| }}^2}$ and $H_{1,1}(z,\overline{z}) 
e^{-\tfrac{\pi}{2}{\ensuremath{\left| {z} \right| }}^2}$. For small values of $R>0$, the eigenvalue corresponding
to $H_{1,1}$ is bigger than the one corresponding to $H_{1,0}$, and thus for small $N$, the kernels
in \eqref{purekernel} and \eqref{eq_a} do not coincide. In this article, we show that this
difference is asymptotically
negligible. Thus, the finite pure polyanalytic ensemble - defined by a lexicographic criterion - is
asymptotically equivalent to a finite WH ensemble - defined by optimizing phase-space concentration.
To show this, we use Weyl's correspondence and account for the
difference between \eqref{eq_a} and \eqref{purekernel} as the error introduced by using two
different variants of Berezin's quantization rule (anti-Wick calculus).

\begin{figure}
\centering
\includegraphics[scale=0.4]{./vals.png}
\caption{A plot of the eigenvalues
$\lambda=\widetilde{M}^{h_1}_{D_R}
\left(H_{j,1}(z,\overline{z}) e^{-\tfrac{\pi}{2}{\ensuremath{\left| {z} \right| }}^2} \right)$,
as a function of $R$, corresponding to $j=0$ (blue, solid) and $j=1$ (red, dashed)}
\label{fig_val}
\end{figure}

Combining Theorem \ref{th_quant_one}  with the approximate description of pure
polyanalytic Ginibre ensembles as finite WH ensembles, we obtain the following quantitative
circular law for pure polyanalytic Ginibre ensembles.

\begin{theorem}
\label{asy_pure}
Let $\rho_{r,N}$ be the one-point intensity of the finite $(r,N)$-pure
polyanalytic Ginibre ensemble and let $R>0$ be such that $N_{D_R}={\ensuremath{\lceil {{\ensuremath{\left| {D_R} \right| }}} \rceil}}=N$.
Then
\begin{equation}
\lVert \rho_{r,n}-1_{D_{R}}\rVert _{1}\leq C_{r} R \asymp \sqrt{N}.
\label{est1.1}
\end{equation}
\end{theorem}

\subsection{Simultaneous observability}
The independence of the eigenfunctions of $M_{D_R}^{h_r}$ of the radius $R$ yields another
property of the (finite and infinite) $r$-pure polyanalytic ensembles.

\begin{theorem}
\label{th_intro_sym}
The restrictions $\{ p_{h_r,j} \big| _{D_R} : j\in {{\mathbb{{N}}}} \}$
are orthogonal on $L^2(D_R)$ for all $R>0$. In the terminology of
{determinantal point process} es this means that the family of disks $\{D_R: R>0\}$ is simultaneously
observable for all $r$-pure polyanalytic ensembles.
\end{theorem}

This recovers and slightly extends a result from Shirai \cite{SHIRAI}. As an application, one
obtains an extension of Kostlan's theorem \cite{Kostlan} on the absolute values of the
points of the Ginibre ensemble of dimension $N$.

\begin{theorem}
\label{eq_thm4}
The set of absolute values of the points distributed according to the $r$-pure polyanalytic Ginibre
ensemble has the same distribution as $\{Y_{1,r},\ldots,Y_{n,r}\}$, where the $Y_{j}$'s are
independent and have density
\begin{equation*}
f_{Y_{j}}(x):=2 \frac{\pi^{j-r+1} r!}{j!} x^{2(j-r)+1}\left[ L_{r}^{j-r}(\pi x^2)\right]^{2}e^{-\pi
x^2},
\end{equation*}
where $L_{j}^{\alpha}$ are the Laguerre polynomials of~\eqref{eq_lag}.
(Hence, $Y^2_j$ is distributed according to a generalized Gamma density function.)
\end{theorem}

\subsection{Organization}
Section \ref{sec_ps} presents tools from phase-space analysis, including the short-time Fourier
transform and
Weyl's correspondence.
Section \ref{sec_wh} introduces finite WH ensembles and more technical variants
required for the identification of finite polyanalytic ensembles as WH ensembles with Hermite
windows. This  identification is carried out in Section \ref{sec_hd} by means of symmetry arguments.
The approximate identification of finite polyanalytic ensembles with finite WH ensembles is
finished in Section
\ref{sec_comp} and  gives  a comparison of the processes defined by truncating the complex Hermite
expansion on the one hand, and by the abstract concentration and spectral truncation method on the
other. We explain the deviation between the two ensembles as stemming from two
different quantization rules. The proof resorts to a Sobolev
embedding for certain symbol classes known modulation spaces. Some of the technical details are
postponed to
the appendix. In Section \ref{sec_do} we apply the symmetry argument from Section \ref{sec_hd} to
rederive the so-called simultaneous observability of polyanalytic ensembles. We also clarify the
relation between the spectral expansions of the restriction and Toeplitz kernels. Finally,
the appendix provides brief background material on determinantal point processes, certain symbol
class for pseudo-differential operators, functions of bounded variation, and polyanalytic spaces.

\section{Harmonic analysis in Phase space}
\label{sec_ps}
In this section we briefly  discuss our tools. The methods from harmonic analysis are new in the
study of  determinantal
point processes. 
\subsection{The short-time Fourier transform}
Given a window function $g\in L^{2}({\mathbb{R}^{d}})$, the short-time Fourier transform of $f \in
L^2({{{{\mathbb R}}^d}})$ is
\begin{equation}
  \label{eq_stft}
V_{g}f(x,\xi )=\int_{{{{\mathbb R}}^d}} f(t) \overline{g(t-x)} e^{-2\pi i \xi t} dt, \qquad (x,\xi) \in {{{{\mathbb R}}^{2d}}}.
\end{equation}
The short-time Fourier transform is closely related to the
Schr\"odinger representation of the Heisenberg group,
which is implemented by the operators
\begin{equation*}
T(x,\xi ,\tau )g(t)=e^{2\pi i\tau }e^{-\pi ix\xi }e^{2\pi i\xi t}g(t-x),
\qquad (x, \xi) \in {{{{\mathbb R}}^d}}, \tau \in {{\mathbb R}}.
\end{equation*}
The corresponding representation coefficients are
\begin{equation*}
\left\langle f,T(x,\xi ,\tau )g\right\rangle =e^{-2\pi i\tau }e^{\pi ix\xi
}\left\langle f,e^{2\pi i\xi \cdot}g(\cdot-x)\right\rangle
= e^{-2\pi i\tau } e^{\pi ix\xi}V_g f(x,\xi).
\end{equation*}
Thus, the short-time Fourier transform eliminates the central variable in the Schr\"odinger
representation coefficients.
We identify a pair $(x,\xi )\in {\mathbb{R}^{2d}}$ with the complex vector $z=x+i\xi \in
{\mathbb{C}^{d}}$. In terms of the phase-space shifts in \eqref{eq:l2},
the short-time Fourier transform is $V_{g}f(z):=\left\langle f,\pi (z)g\right\rangle$.
The phase-space shifts satisfy the commutation relations
\begin{equation}
\pi (x,\xi )\pi (x^{\prime },\xi ^{\prime })=e^{-2\pi i\xi ^{\prime }x}\pi
(x+x^{\prime },\xi +\xi ^{\prime }),\qquad (x,\xi ),(x^{\prime },\xi
^{\prime })\in {\mathbb{R}^{d}}\times {\mathbb{R}^{d}},  \label{com}
\end{equation}
and the short-time Fourier transform satisfies the following \emph{orthogonality
relations}~\cite[Proposition 1.42]{folland89} \cite[Theorem 3.2.1]{Charly},
\begin{equation}
\left\langle V_{g_{1}}f_{1},V_{g_{2}}f_{2}\right\rangle _{L^2({{{{\mathbb{{R}}}}^{2d}}} )} =\left\langle
f_{1},f_{2}\right\rangle _{L^2({{{\mathbb{{R}}}}^d} )} \overline{\left\langle
  g_{1},g_{2}\right\rangle  }_{L^2({{{\mathbb{{R}}}}^d} )}.
\label{eq_ortrel}
\end{equation}
In particular, when $\lVert g\rVert _{2}=1$, the map $V_{g}$ is an
isometry between $L^{2}({\mathbb{R}^{d}})$ and a closed subspace of
$L^{2}({\mathbb{R}^{2d}})$:
\begin{equation*}
\lVert V_{g}f\rVert _{{L^{2}({\mathbb{R}^{2d}})}}=\lVert f\rVert_{{L^{2}({\mathbb{R}^{d}})}},\qquad
f\in {L^{2}({\mathbb{R}^{d}})}.
\end{equation*}
The commutation rule (\ref{com}) implies the \emph{covariance property} of
the short-time Fourier transform:
\begin{equation*}
V_{g}(\pi (x,\xi )f)(x^{\prime },\xi ^{\prime })=e^{-2\pi ix(\xi ^{\prime
}-\xi )}V_{g}f(x^{\prime }-x,\xi ^{\prime }-\xi ),\qquad (x,\xi ),(x^{\prime
},\xi ^{\prime })\in {\mathbb{R}^{d}}\times {\mathbb{R}^{d}}.
\end{equation*}
\subsection{Special windows}
If we choose the Gaussian function $h_{0}(t)=2^{\frac{1}{4}}e^{-\pi t^{2}}$, $t \in {{\mathbb R}}$,
as a window in \eqref{eq_stft}, then
a simple calculation shows that
\begin{equation}
e^{-i \pi x\xi + \frac{\pi}{2} \left\vert z\right\vert ^{2}}V_{h_{0}}f(x,-\xi
)=2^{1/4} \int_{\mathbb{R}}f(t)e^{2\pi tz-\pi t^{2}-\frac{\pi }{2}z^{2}}dt=\mathcal{B}f(z),
\label{Bargmann}
\end{equation}
where $\mathcal{B}f(z)$ is the \emph{Bargmann transform} of $f$
\cite[Chapter~1.6]{folland89}. The Bargmann transform $\mathcal{B}$ is a unitary isomorphism from
$L^{2}({{\mathbb{{R}}}})$ onto the Bargmann-Fock space $\mathcal{F}({{\mathbb{{C}}}})$ consisting of all entire functions
satisfying
\begin{equation}
\left\Vert F\right\Vert _{\mathcal{F}(\mathbb{C})}^{2}=
\int_{\mathbb{C}}\left\vert F(z)\right\vert ^{2}e^{-\pi \left\vert z\right\vert
^{2}}dz<\infty .  \label{Focknorm}
\end{equation}
We now explain the relation between polyanalytic Fock spaces and
time-frequency analysis with Hermite windows $\{h_r: r\geq0\}$. The $r$-\emph{pure
polyanalytic Bargmann transform} \cite{Abreusampling} is the map
$\mathcal{B}^{r}:L^{2}({\mathbb{R}})\rightarrow L^{2}({\mathbb{C}},e^{-\pi \left\vert z\right\vert
^{2}})$
\begin{equation}
\label{eq_relation}
\mathcal{B}^{r}f(z):=e^{-i\pi x\xi +\tfrac{\pi }{2}\left\vert z\right\vert
^{2}}V_{h_{r}}f(x,-\xi ),\qquad z=x+i\xi .
\end{equation}
This map defines an isometric
isomorphism between $L^{2}({\mathbb{R}})$ and the pure polyanalytic-Fock
space $\mathcal{F}^{r}(\mathbb{C})$ (see Section \ref{sec_pure_pol}).
The orthogonality relations
\eqref{eq_ortrel} show that for $r\not=r^{\prime }$, $V_{h_{r}}f_{1}$ is
orthogonal to $V_{h_{r^{\prime }}}f_{2}$ for all $f_1,f_2 \in
L^{2}({\mathbb{R}})$. The key relation between
time-frequency analysis and polyanalytic functions is the so-called \emph{Laguerre connection}
\cite[Chapter 1.9]{folland89}:
\begin{equation} \label{l9b}
V_{h_{r}} h_j (x,-\xi ) = e^{i\pi x\xi -\tfrac{\pi }{2}{\ensuremath{\left| {z} \right| }}^2} H_{j,r}(z,\bar{z}),
\end{equation}
which, in terms of the polyanalytic Bargmann transform reads as 
\begin{equation} \label{l9}
\mathcal{B}^{r}h_{j}(z)=H_{j,r}(z,\bar{z}),
\end{equation}
see also \cite{Abreusampling}.

\subsection{The range of the short-time Fourier transform}

For $\lVert g\rVert _{2}=1$, the short-time Fourier transform $V_{g}$
defines an isometric map $V_{g}:{L^{2}({\mathbb{R}^{d}})}\rightarrow {L^{2}({\mathbb{R}^{2d}})}$
with range 
\begin{equation*}
{{\mathcal{V}}_{g}}:=\big\{\,V_{g}f\,:\,f\in {L^{2}({\mathbb{R}^{d}})}\,\big\}
\subseteq {L^{2}({\mathbb{R}^{2d}})}.
\end{equation*}
The adjoint of $V_{g}$ is $V_{g}^{\ast }:L^{2}({\mathbb{R}^{2d}})\rightarrow
L^{2}({\mathbb{R}^{d}})$
\begin{equation*}
V_{g}^{\ast }F(t)=\int_{{\mathbb{R}^{d}}\times {\mathbb{R}^{d}}}F(x,\xi
)g(t-x)e^{2\pi i\xi t}dxd\xi = \int _{{{{\mathbb{{R}}}}^{2d}}} F(z) \pi (z)g(t)  \, dz ,\qquad t\in {\mathbb{R}^{d}}.
\end{equation*}
The orthogonal projection $P_{{\mathcal{V}}_{g}}:L^{2}({\mathbb{R}^{2d}}
)\rightarrow {{\mathcal{V}}_{g}}$ is then $P_{{\mathcal{V}}_{g}}=V_{g}V_{g}^{
\ast }$. Explicitly, $P_{{\mathcal{V}}_{g}}$ is the integral operator
\begin{equation*}
P_{{\mathcal{V}}_{g}}F(z)=\int_{\mathbb{R}^{2d}}
{{K^g}}(z,z^{\prime })
F(z^{\prime })
dz^{\prime },\qquad z=(x,\xi )\in {\mathbb{R}^{2d}},
\end{equation*}
where the \emph{reproducing kernel} ${{K^g}}$ is given by \eqref{eq:l1}.
Every function $F\in {{\mathcal{V}}_{g}}$ is continuous and satisfies the reproducing
formula $F(z)=\int F(z^{\prime }){{K^g}}(z,z^{\prime })dz^{\prime }$.

\subsection{Metaplectic rotation}
We will make use of a rotational symmetry argument in phase space. Let $R_{\theta }:=
\big[\begin{smallmatrix}
\cos (\theta ) & -\sin (\theta ) \\
\sin (\theta ) & \cos (\theta )
\end{smallmatrix}
\big]$ denote the rotation with angle $\theta \in {\mathbb{R}}$.
The \emph{metaplectic rotation} is the operator given in the Hermite basis
$\left\{h_{r}:r\geq 0\right\} $ by
\begin{equation}
\metrtf=\sum_{r\geq 0}e^{i r\theta }\left\langle
f,h_{r}\right\rangle h_{r},\qquad f\in L^{2}(\mathbb{R}) \, ,  \label{rotation}
\end{equation}
in particular, $\mu (R_\theta ) h_r = e^{i r\theta } h_r$. 
The standard and metaplectic rotations are related by 
\begin{equation}
V_{g}f(R_{\theta }(x,\xi ))=e^{\pi i(x\xi -x{^{\prime }}\xi {^{\prime }})}V_{{\mu(R_{-\theta})}
g}{\mu(R_{-\theta})} f(x,\xi ),\mbox{ where }(x{^{\prime }},\xi {^{\prime
}})=R_{\theta
}(x,\xi ).  \label{eq_cov}
\end{equation}
This formula is a special case of the symplectic covariance of Schr\"odinger's representation;
see \cite[Chapters 1 and 2]{folland89}, \cite[Chapter 9]{Charly}, or \cite[Chapter 15]{dego11}) for
background and
proofs.

\subsection{Time-frequency localization and Toeplitz operators}
\label{sec_toep}
Let us consider $g$ with $\lVert g\rVert _{2}=1$. For $m\in
L^{\infty }({\mathbb{R}^{2d}})$, the \emph{Toeplitz operator}
$M_{m}^{g}:{{\mathcal{V}}_{g}}\rightarrow {{\mathcal{V}}_{g}}$ is
\begin{equation*}
M_{m}^{g}F:=P_{{\mathcal{V}}_{g}}(m\cdot F),\qquad F\in {{\mathcal{V}}_{g}},
\end{equation*}
and its integral kernel is given by \eqref{Toe}. (The operator $M_{m}^{g}$ is defined
on ${\mathcal{V}}_g$; the kernel in \eqref{Toe} represents the extension of $M_{m}^{g}$ to
$L^2({{{{\mathbb R}}^{2d}}})$ that is $0$
on ${\mathcal{V}}_g^\perp$.)
Clearly, $\lVert M_{m}^{g}\rVert _{{{\mathcal{V}}_{g}}\rightarrow {{\mathcal{V}}_{g}}}\leq \lVert m\rVert
_{\infty }$. In addition, it is easy to see that
if $m\geq 0$, then $M_{m}^{g}$ is a positive operator. The \emph{time-frequency localization
operator} with window $g$ and symbol $m$ is $H_{m}^{g}:=V_{g}^{\ast
}M_{m}^{g}V_{g}:{L^{2}({\mathbb{R}^{d}})}\rightarrow {L^{2}({\mathbb{R}^{d}})}$. Hence $M_{m}^{g}$
and $H_{m}^{g}$ are unitarily
equivalent. \footnote{The operator $H_{m}^{g}$ should not be confused with the complex Hermite polynomial $H_{j,r}$.}
The situation is depicted in the following diagram.
\begin{align}
\label{eq_diagram}
\xymatrix{ L^2({{{\mathbb{{R}}}}^d}) \ar[d]_{V_g} \ar[r]^{H^g_m} &L^2({{{\mathbb{{R}}}}^d}) \ar[d]^{V_g}
\\ {\mathcal{V}}_g \ar[rd]_{m \cdot} \ar[r]^{M^g_m} & {\mathcal{V}}_g \\ & L^2({{{{\mathbb{{R}}}}^{2d}}})
\ar[u]_{P_{{\mathcal{V}}_g}} }
\end{align}

Explicitly, the time-frequency localization operator applies a mask to the
short-time Fourier transform:
\begin{equation*}
H_{m}^{g}f:=\int_{{\mathbb{R}^{2d}}}m(z)V_{g}f(z)\pi (z)g\,dz,
\qquad f \in L^2({{{{\mathbb{{R}}}}^{2d}}}).
\end{equation*}
As we will use the connection between time-freqency localization on
${{{\mathbb{{R}}}}^d} $ and Toeplitz operators on ${{{{\mathbb{{R}}}}^{2d}}} $ in a crucial argument, we
write ~\eqref{eq_diagram} as a formula
\begin{align}
  \langle H_m^g f, u\rangle &= \langle V_g (V_g^* M_m^g V_gf),
  V_gu\rangle \notag  \\
&= \langle P_{\mathcal{V}_g} (m \, V_gf), V_gu\rangle \notag  \\
&= \langle m \, V_gf ,V_gu \rangle \, .  \label{eq:o12}
\end{align}
This formula makes sense for $f,u\in L^2({{{\mathbb{{R}}}}^d})$ and $m\in L^\infty({{{{\mathbb{{R}}}}^{2d}}}
)$, but also for many other assumptions~\cite{cogr03}. 

TF localization operators are useful in signal processing because they model time-varying filters.
For Gaussian windows, they have been studied in signal processing by Daubechies
\cite{Daubechies} and as Toeplitz operators on spaces of analytic
functions by Seip \cite{Seip}; see also \cite[Section 1.4]{AGR}.
When $m\in L^{1}({\mathbb{R}^{2d}})$, $H_{m}^{g}$ is
trace-class and 
\begin{equation}
  \label{eq:o9}
\mathrm{trace}(H_{m}^{g})=\int_{{{{\mathbb{{R}}}}^{2d}}}  m(z) dz \, ,  
\end{equation}
see for example \cite{heil1, heil2, cogr03}.
A similar property holds for $M_{m}^{g}$, because it is unitarily
equivalent to $H_{m}^{g}$. When $m=1_{\Omega }$, the indicator function of a
set $\Omega $, we write $M_{\Omega }^{g}$ and $H_{\Omega }^{g}$. In this
case, the positivity property implies that $0\leq M_{\Omega }^{g}\leq I$.

\subsection{The Weyl correspondence}

The \emph{Weyl transform} of a distribution $\sigma \in \mathcal{S}^{\prime
}({\mathbb{R}^{d}}\times {\mathbb{R}^{d}})$ is an operator $\sigma ^{w}$
that is formally defined on functions $f:{\mathbb{R}^{d}}\rightarrow {\mathbb{C}}$ as
\begin{equation*}
\sigma ^{w} f(x):=\int_{{\mathbb{R}^{d}}\times {\mathbb{R}^{d}}}\sigma
\left( \frac{x+y}{2},\xi \right) e^{2\pi i(x-y)\xi }f(y)dyd\xi ,\qquad x\in {\mathbb{R}^{d}}.
\end{equation*}
Every continuous linear operator $T:\mathcal{S}({\mathbb{R}^{d}})\rightarrow
\mathcal{S}^{\prime }({\mathbb{R}^{d}})$ can be represented in a unique way
as $T=\sigma ^{w}$, and $\sigma $ is called its \emph{Weyl symbol}
(see \cite[Chapter 2]{folland89}). The
Wigner distribution of a test function $g\in \mathcal{S}({\mathbb{R}^{d}})$
and a distribution $f\in \mathcal{S}^{\prime }({\mathbb{R}^{d}})$ is
\begin{equation*}
{W(f,g)}(x,\xi )=\int_{{\mathbb{R}^{2d}}}f(x+\tfrac{t}{2})\overline{g(x-\tfrac{t}{2})}e^{-2\pi it\xi
}dt.
\end{equation*}
The integral has to be understood distributionally. The map $(f,g)\mapsto
{W(f,g)}$ extends to other function classes, for example, for $f,g\in L^{2}({\mathbb{R}^{d}})$,
${W(f,g)}$ is well-defined and
\begin{equation}
\label{eq_norm_wigner}
{\lVert{W(f,g)}\rVert}_2 = {\lVert{f}\rVert}_2 {\lVert{g}\rVert}_2.
\end{equation}
The Wigner distribution is
closely related to the short-time Fourier transform:
\begin{equation*}
{W(f,g)}(x,\xi )=2^{d}e^{4\pi ix\cdot \xi }V_{\tilde{g}}f(2x,2\xi ),
\end{equation*}
where $\tilde{g}(x)=g(-x)$.
The action of $\sigma ^{w}$ on a distribution
can be easily described in terms of the Wigner distribution:
\begin{equation*}
\left\langle \sigma ^{w}f,g\right\rangle =\left\langle \sigma,{W(g,f)}\right\rangle .
\end{equation*}
Time-frequency localization operators have a simple description in
terms of the Weyl calculus:
\begin{equation}
\label{eq_tf_weyl}
H_{m}^{g}=\left( m\ast {W(g,g)} \right) ^{w}.
\end{equation}

\section{Finite Weyl-Heisenberg ensembles}
\label{sec_wh}

\subsection{Definitions}
To define finite Weyl-Heisenberg processes, we consider a domain $\Omega
\subseteq {\mathbb{R}^{2d}}$ with non-empty interior, finite measure and finite perimeter, i.e.,
the characteristic function of $\Omega $ has bounded variation
(see Section \ref{sec_bv}). Since $M_{\Omega }^{g}$ is
trace-class, the Toeplitz operator $M_{\Omega }^{g}$ can be diagonalized as
\begin{equation}
\label{eq_diag_toep}
M_{\Omega }^{g}=\sum_{j\geq 1}{\lambda_{j}^{\Omega }}\,
p_{g,j}^{\Omega }
\otimes p_{g,j}^{\Omega },\qquad f\in {L^{2}({\mathbb{R}^{2d}})},
\end{equation}
where $\left\{ {\lambda_{j}^{\Omega }}:j\geq 1\right\} $ are the non-zero
eigenvalues of $M_{\Omega }^{g}$ in decreasing order and the corresponding
eigenfunctions $\left\{ p_{g,j}^{\Omega }:j\geq 1\right\} $ are normalized in $L^{2}$.
The operator $M_{\Omega }^{g}$ may have a non-trivial kernel, but it is known
that it always has infinite rank \cite[Lemma 5.8]{doro14}, therefore, the sequences
$\{ {\lambda_{j}^{\Omega }}:j\geq 1\} $
and $\{p_{g,j}^{\Omega }: j \geq 1\}$
are indeed infinite.
In addition, as follows from \eqref{eq:o9}, we have 
\begin{equation} \label{eq:o10}
0\leq {\lambda_{j}^{\Omega }}\leq 1,\mbox{ and }\sum_{j \geq 1}{\lambda_{j}^{\Omega }}=\left\vert \Omega \right\vert.
\end{equation}
We remark that the eigenvalues ${\lambda_{j}^{\Omega }}$ do depend on the window function $g$. When we need
to stress this dependence we write $\lambda_{j}(\Omega,g)$.

\emph{The finite Weyl-Heisenberg ensemble} $\mathcal{X}^g_\Omega$ is given by Definition
\ref{def_intro_wh}.
For technical reasons, we will also consider a more general class of
WH ensembles depending on an extra ingredient. Given a subset $I\subseteq {\mathbb{N}}$,
we let ${\mathcal{X}^g_{\Omega,I}}$ be the determinantal point process with correlation kernel
\begin{equation*}
K_{g,\Omega ,I}(z,z')=\sum_{j\in I}p_{g,j}^{\Omega }(z)\overline{p_{g,j}^{\Omega }(z')}.
\end{equation*}
When $I=\{1,\ldots ,N_{\Omega }\}$ we obtain the finite WH ensemble $\mathcal{X}^g_\Omega$,
while for
$I={\mathbb{N}}$ we obtain the infinite ensemble. (In the latter case, the resulting point-process
is independent of domain $\Omega $.) Later  we need to analyze the properties of the
ensemble ${\mathcal{X}^g_{\Omega,I}}$ under variations of the index set $I$. When no subset $I$ is specified, we
always refer to the ensemble $\mathcal{X}^g_\Omega$
associated with $I=\{1,\ldots,N_{\Omega }\}$.

\begin{rem} {\rm 
The process ${\mathcal{X}^g_{\Omega,I}}$ is well-defined due to the
Macchi-Soshnikov theorem (see Section \ref{sec_det}). Indeed, since the kernel $K_{g,\Omega ,I}$
represents an orthogonal projection, we only need to verify that it is locally trace-class. This
follows easily from the facts that
$0 \leq K_{g,\Omega ,I}(z,z) \leq {{K^g}}(z,z)=1$
and that the restriction operators are positive (see Section \ref{sec_duality}).}
\end{rem}

\subsection{Universality and quantitative limit laws}
The one-point intensity associated with a Weyl-Heisenberg ensemble
${\mathcal{X}^g_{\Omega,I}}$ is
\begin{equation*}
\rho _{g,\Omega,I}(z):=\sum_{j \in I}\left\vert p_{g,j}^{\Omega
}(z)\right\vert ^{2}.
\end{equation*}
For $\mathcal{X}^g_\Omega$, the intensity $\rho_{g,\Omega}$ has been studied in the realm of signal
analysis, where it is known as the \emph{accumulated spectrogram} \cite{AGR,APR}.
(Another interesting connection between DPP's and signal analysis is the completeness
results of  Ghosh \cite{ghosh2}.) The results in
\cite{AGR,APR} imply Theorems \ref{tl1} and \ref{th_quant_one}, which apply to the finite
Weyl-Heisenberg ensembles $\mathcal{X}^g_\Omega$. For the general ensemble
${\mathcal{X}^g_{\Omega,I}}$ we have the following lemma.

\begin{lemma}
\label{lemma_one_point_I}
Let $\rho_{g,\Omega,I}$ be the one-point intensity of a WH ensemble ${\mathcal{X}^g_{\Omega,I}}$ with $\#I < \infty$.
Then
\begin{align*}
\lVert \rho_{g,\Omega,I}-1_\Omega\rVert_{L^1({\mathbb{R}^{2d}})} =
\#I-\left| \Omega \right| + 2 \sum_{j \notin I} {\lambda_{j}^{\Omega }}.
\end{align*}
\end{lemma}
\begin{proof}
Using that $0\leq \rho_{g,\Omega,I} \leq 1$ and \eqref{eq:o11} and
\eqref{eq:o10}, we first calculate 
\begin{align*}
\lVert \rho_{g,\Omega,I}-1_\Omega\rVert_{L^1(\Omega)}=\int_\Omega
\left(1-\rho_{g,\Omega,I}(z)\right)\, dz = {|\Omega|} - \sum_{j \in I}
{\lambda_{j}^{\Omega }}= \sum_{j \notin I} {\lambda_{j}^{\Omega }}.
\end{align*}
Second,
\begin{align*}
\lVert \rho_{g,\Omega,I}-1_\Omega\rVert_{L^1({\mathbb{R}^{2d}}
\setminus\Omega)}&=\int_{{\mathbb{R}^{2d}}\setminus\Omega}
\rho_{g,\Omega,I}(z) \, dz = \sum_{j \in I}
\left(1-{\lambda_{j}^{\Omega }}\right) \\
&= \#I - \sum_{j \in I} {\lambda_{j}^{\Omega }}=\#I - \left| \Omega \right| +
\sum_{j \notin I} {\lambda_{j}^{\Omega }}.
\end{align*}
The conclusion follows by adding both estimates.
\end{proof}

\section{Hermite windows and polyanalytic ensembles}
\label{sec_hd}

\subsection{Eigenfunctions of Toeplitz operators}
We first investigate the eigenfunctions of Toeplitz operators with Hermite windows
$\{h_r: r \geq 0\}$ and circular domains.

\begin{prop}
\label{prop_hn_disks} Let $D_R \subseteq {\mathbb{R}}^2$ be a disk centered
at the origin. Then  the family of Hermite functions is a complete set
of eigenfunctions for $H^{h_r}_{D_R}$. (In particular, every eigenfunction
of $H^{h_r}_{D_R}$ is a  Hermite function.)

As a consequence, the set $\{H_{j,r}(z,\overline{z}) e^{-\pi |z|^2/2}:
j \geq 0\}$  forms  a complete set of eigenfunctions
for
$\widetilde M_{D_R}^{h_n}$.
(where $\widetilde M_{D_R}^{h_r}$ is related to $M_{D_R}^{h_r}$ by \eqref{eq_tilde_1}.)
\end{prop}

\begin{proof}
Consider the metaplectic rotation $R_{\theta }$ with angle $\theta \in {\mathbb{R}}$ defined in
\eqref{rotation}. For $f,u\in L^{2}({\mathbb{R}})$,
we use first \eqref{eq:o12} and then  the covariance property in \eqref{eq_cov} and the rotational
invariance of $D{_{R}}$ to compute:
\begin{align*}
\left\langle {\mu(R_\theta)}^{\ast }H_{D_R}^{h_r}
\metrtf,u\right\rangle & =\left\langle H_{D_R}^{h_r}\metrtf,\metrtu\right\rangle =\left\langle 1_{{D_{R}}}V_{h_{r}}\metrtf,V_{h_{r}}\metrtu\right\rangle \\
& =\left\langle 1_{D{_{R}}}V_{\metrth_{r}}
{\mu(R_\theta)} f,V_{\metrth_{r}}\metrtu\right\rangle
\\
&=\left\langle 1_{D{_{R}}}V_{h_{r}}f(R_{-\theta} \,\cdot),
V_{h_{r}}u(R_{-\theta }\,\cdot)\right\rangle \\
& =\int_{D{_{R}}}V_{h_{r}}f(z)V_{h_{r}}u(z)dz =\left\langle H_{D_R}^{h_r}f,u\right\rangle.
\end{align*}
We conclude that ${\mu(R_\theta)}^{\ast }H_{D_R}^{h_r}{\mu(R_\theta)}=H_{D_R}^{h_r}$, for all $\theta \in
{\mathbb{R}}$. Applying this identity to a Hermite function gives
\begin{align*}
{\mu(R_\theta)}^{\ast} H_{D_R}^{h_r} h_j &=
{\mu(R_\theta)}^{\ast} H_{D_R}^{h_r}{\mu(R_\theta)}
\left( e^{-ij \theta}  h_j \right)
\\
&=
e^{-ij \theta} {\mu(R_\theta)}^{\ast} H_{D_R}^{h_r}{\mu(R_\theta)} h_j
=
e^{-ij \theta} H_{D_R}^{h_r} h_j.
\end{align*}
Thus, $H_{D_R}^{h_r} h_j$ is an eigenfunction of ${\mu(R_\theta)}^{\ast}$
with eigenvalue $e^{-ij \theta}$.
For irrational $\theta$, the numbers $\{e^{-ij \theta}: j \geq 0\}$ are all different,
and, therefore, the eigenspaces of ${\mu(R_\theta)}^{\ast}$ are one-dimensional. Hence,
$H_{D_R}^{h_r} h_j$ must be a multiple of $h_j$. Thus, we have shown that each Hermite
function is an eigenfunction of $H_{D_R}^{h_r}$. Since the Hermite family is complete, the
conclusion follows. The statement about the complex Hermite
polynomials follows from \eqref{l9b} and 
\eqref{eq_diagram}; the extra phase-factors and conjugation bars disappear due to the
renormalization $M_{D_R}^{h_r} \mapsto \widetilde M_{D_R}^{h_r}$.
\end{proof}

\subsection{Eigenvalues of Toeplitz operators}
As a second step to identify polyanalytic ensembles as WH ensembles, we inspect the eigenvalues
of Toeplitz operators.

\begin{lemma}
\label{lemma_hn_eig}
Let $R>0$. Then the eigenvalue of $H_{D_{R}}^{h_{r}}$ corresponding
to $h_{j}$ and the eigenvalue of $\widetilde M_{D_R}^{h_r}$
corresponding to $H_{j,r}(z,\overline{z}) e^{-\pi |z|^2/2}$ are 
\begin{equation}
\label{eq_muR}
\mu_{j,R}^{r}:=
{\ensuremath{\left<{H_{D_{R}}^{h_{r}} h_{j}},{h_j}\right>}} =\int_{D_{R}}\left\vert
H_{r,j}(z,\bar{z})\right\vert ^{2}e^{-\pi \left\vert z\right\vert ^{2}}dz.
\end{equation}
In particular, $\mu_{j,R}^{r}\not=0$, for all $j,r\geq 0$ and $R>0$, and
\begin{align}
H_{D_{R}}^{h_{r}} &= \sum_{j \geq 0} \mu^r_{j,R} \, h_j \otimes h_j.
\end{align}
\end{lemma}
\begin{proof}
\eqref{eq_muR} follows immediately by from the definitions. Since
$H_{r,j}$ vanishes only on a set of measure zero - cf. \eqref{ComplexHermite} -
we conclude that $\mu_{j,R}^{r}\not=0$.  The diagonalization follows from Proposition \ref{prop_hn_disks}.
\end{proof}
\begin{remark}
\label{rem_mult_1}
Figure \ref{fig_val} shows a plot of $\mu^1_{0,R}$ (solid, blue) and $\mu^1_{1,R}$ (dashed, red) as
a function of $R$. Note that for a certain value of $R$, the eigenvalue $\mu^1_{0,R}=\mu^1_{1,R}$
is multiple.
\end{remark}

\subsection{Identification as a WH ensemble}
We can now identify finite pure polyanalytic ensembles as WH ensembles.
\begin{prop}
\label{prop_ident_1}
Let $J \subseteq {{\mathbb N}}_0$ and $R>0$, then there exist a set $I \subseteq {\mathbb{N}}$ with $\#I=\#J$
such that
\begin{align}
\label{eq_set}
  \left\{ V_{h_r} h_{j}: j \in J \right\} =\left\{p_{h_{r},j}^{D_{R}}:j\in I\right\}.
\end{align}
\end{prop}
\begin{proof}
By Proposition \ref{prop_hn_disks} every Hermite function $h_j$ is an eigenfunction of
$H^{h_r}_{D_R}$. In addition, by Lemma \ref{lemma_hn_eig}, the corresponding eigenvalue
$\mu^r_{j,R}$ is non-zero. Hence $V_{h_r} h_j$ is one of the functions $p_{h_r,j'}^{D_R }$
in the diagonalization \eqref{eq_diag_toep}. The set $I := \{j': j \in J\}$
satisfies \eqref{eq_set}.
\end{proof}
As a consequence, we obtain the following.
\begin{prop}
\label{prop_ident_2}
The $r$-pure polyanalytic Ginibre ensemble with $N$ particles and 
with kernel $K_{r,N}$ in \eqref{purekernel} can be identified with a finite WH ensemble in the following way.
Let $D_{R_{N}}\subset {\mathbb{C}}$ be the disk with area $N$.
Let $I_{r,N} \subseteq {\mathbb{N}}$ be a set such that
\begin{align}
\label{eq_ident_2}
\left\{ V_{h_r} h_{0}, \ldots, V_{h_r} h_{N-1} \right\} =\left\{p_{h_{r},j}^{D_{R}}:j\in
I_{r,N} \right\},
\end{align}
and $\# I_{r,N}=N$, whose existence is granted by Proposition \ref{prop_ident_1}.
Then $\widetilde K_{h_r, D_{R_{N}}, I_{r,N}}=K_{r,N}$, and the corresponding point processes
coincide. In particular
\begin{align}
\label{eq_related}
\rho_{r,N}(z)=\rho_{h_{r},D_{R_{N}},I_{r,N}}(z), \qquad z \in {{\mathbb{{C}}}}.
\end{align}
\end{prop}
\begin{proof}
Since $\# I_{r,N}=N$, we can write
\begin{align*}
K_{h_r, D_{R_{N}}}(z,z') = \sum_{j \in I_{r,N}} p_{h_{r},j}^{D_{R_N}}(z)
\overline{p_{h_{r},j}^{D_{R_N}}(z')}
= \sum_{j=0}^{N-1} V_{h_r} h_{j}(z) \overline{V_{h_r} h_{j}(z')}.
\end{align*}
Using \eqref{eq_tilde_2} and \eqref{l9b} we conclude that
\begin{align*}
\widetilde K_{h_r, D_{R_{N}}}(z,z') =
\sum_{j=0}^{N-1}
H_{j,r}(z,\overline{z}) e^{-\pi |z|^2/2}
\overline{H_{j,r}(z',\overline{z'})} e^{-\pi |z'|^2/2}
= K_{r,N}(z,z'),
\end{align*}
as desired. This implies that the point processes corresponding to
$K_{h_r, D_{R_{N}}}$ and $K_{r,N}$ are related by transformation $z \mapsto \overline{z}$.
Since $H_{j,r}(z,\overline{z})=\overline{H_{j,r}(\overline{z},z)}$,
the intensities of the pure $(r,N)$-polyanalytic ensemble are invariant under the map
$z \mapsto \overline{z}$ and the conclusion follows.

\end{proof}
While Proposition \ref{prop_ident_2} identifies finite pure polyanalytic ensembles with WH
ensembles in the generalized sense of Section \ref{sec_wh}, this is just a technical step. Our
final goal is to compare finite polyanalytic ensembles with finite WH ensembles in the sense of
Definition \ref{def_intro_wh}, where the index set is $I_{r,N}=\{1, \ldots, N\}$. Before proceeding
we note that for the Gaussian $h_0$ such comparison is in fact an exact identification.

\begin{coro}
\label{coro_ident_3}
For $r=0$, the set $I_{0,N}$ from Proposition \ref{prop_ident_2} is
$I_{0,N}=\{0, \ldots, N-1\}$. Thus, the Ginibre ensemble with $N$ particles has the same
distribution as the finite WH
ensemble $\mathcal{X}^{h_0}_{D_{R_N}}$, and
\begin{align}
\label{eq_related_3}
\rho_{0,N}(z)=\rho_{h_{0},D_{R_{N}}}(z), \qquad z \in {{\mathbb{{C}}}}.
\end{align}
\end{coro}
\begin{proof}
The claim amounts to saying that the eigenvalues $\mu^0_{j,R}$ in \eqref{eq_muR} are decreasing for
all $R>0$, so that the ordering of the eigenfunctions in \eqref{eq_diag_toep} coincides with the
indexation of the complex Hermite polynomials. The explicit formula in \eqref{eq_muR}
in the case $r=0$ gives the sequence of \emph{incomplete Gamma functions}:
\begin{align*}
\mu^0_{j,R} = \frac{1}{j!} \int_0^{\pi R^2} t^j e^{- t} dt
= 1- e^{-\pi R^2} \sum_{k=0}^{j} \frac{\pi^k}{k!} R^{2k},
\end{align*}
which is decreasing in $j$ (see for example \cite[Eq. 6.5.13]{special}).
\end{proof}

\section{Comparison between finite WH and polyanalytic ensembles}
\label{sec_comp}
Having identified finite pure polyanalytic ensembles as WH ensembles associated
with a certain subset of eigenfunctions $I$, we now investigate how much this choice deviates from
the standard one $I=\{1, \ldots, N\}$. Thus, we compare finite pure polyanalytic ensembles
to the finite WH ensembles of  Definition \ref{def_intro_wh}.

\subsection{Change of quantization}

\label{sec_change} As a main technical step, we show that the change of the window of a
time-frequency localization operator affects the distribution of the
corresponding eigenvalues in a way that is controlled by the perimeter of
the localization domain. When $g$ is a Gaussian, the map $m\mapsto H_{m}^{g}$
is called \emph{Berezin's quantization}
or \emph{anti-Wick calculus} \cite[Chapter 2]{folland89} or \cite{lerner03}. The results in this section show
that if Berezin's quantization is considered with respect to more general
windows, the resulting calculus enjoys  similar asymptotic spectral
properties. We consider the function class
\begin{equation}
\label{eq_def_m1}
{M^{1}}({\mathbb{R}^{d}}):=\big\{\,f\in L^{2}({\mathbb{R}^{d}})\,:\,\lVert
f\rVert _{{M^{1}}}:=\lVert V_{\phi }f\rVert _{L^{1}({\mathbb{R}^{2d}}
)}<+\infty \,\big\},
\end{equation}
where $\phi (x)=2^{d/4}e^{-\pi \left\vert x\right\vert ^{2}}$. The class $M^{1}$ is one of the
\emph{modulation spaces} used in signal processing. It
is also important as a symbol-class for pseudo-differential operator. Indeed,
the following lemma, whose proof can be found in \cite{gr96}, gives a
trace-class estimate in terms of the $M^{1}$ norm of the Weyl symbol
(see also \cite{heil1, heil2, cogr03}).

\begin{prop}
\label{prop_s1} Let $\sigma \in {M^1}({\mathbb{R}^{2d}})$. Then $\sigma ^{w}$
is a trace-class operator and
\begin{equation*}
\lVert \sigma ^{w}\rVert _{S^{1}}\lesssim \lVert \sigma \rVert _{{M^1}},
\end{equation*}
where ${\lVert{\cdot}\rVert}_{S^{1}}$ denotes the trace-norm.
\end{prop}

The next lemma will allow us to exploit cancellation properties in the $M^1$-norm. Its proof is postponed to Section \ref{sec_mod}.

\begin{lemma}[A Sobolev embedding for $M^1$]
\label{lemma_M1_sobolev} Let $f\in L^{1}({\mathbb{R}^{d}})$ be such that
$\partial _{x_{k}}f\in {M^1}({\mathbb{R}^{d}})$, for $k=1,\ldots ,d$. Then
$f\in {M^1}({\mathbb{R}^{d}})$ and $\lVert f\rVert _{{M^1}}\lesssim \lVert
f\rVert _{L^{1}}+\sum_{k=1}^{d}\lVert \partial _{x_{i}}f\rVert _{{M^1}}$.
\end{lemma}
We can now derive the main technical result. Its statement uses the space
of $\mathrm{BV}({{{{\mathbb R}}^{2d}}})$ of (integrable) functions of bounded variation; see Section \ref{sec_bv}
for some background.
\begin{theorem}
\label{th_change} Let $g_{1},g_{2}\in \mathcal{S}({\mathbb{R}^d})$ with $\lVert g_i\rVert_{2}=1$
and $m\in \mathrm{BV}({\mathbb{R}^{2d}})$. Then
\begin{equation*}
\lVert H_m^{g_1} - H_m^{g_2}\rVert_{S^{1}}\leq C_{g_{1},g_{2}}\mathit{var}(m),
\end{equation*}
where $C_{g_{1},g_{2}}$ is a constant that only depends on $g_1$ and $g_2$.

In particular, when $m=1_\Omega$ we obtain that
\begin{equation*}
\lVert H_\Omega^{g_1} - H_\Omega^{g_2}\rVert_{S^{1}}\leq
C_{g_{1},g_{2}}
{{\ensuremath{\left| {\partial {\Omega}} \right| }}_{2d-1}}.
\end{equation*}
\end{theorem}

\begin{proof}[Proof of Theorem \protect\ref{th_change}]
Let us assume first that $m$ is smooth and compactly supported. We use the
description of time-frequency localization operators as Weyl operators. By \eqref{eq_tf_weyl},
$H_{m}^{g_{i}}=(m\ast W(g_{i},g_i))^{w}$.

Let $h:=W(g_{1},g_{1})-W(g_{2},g_{2})$. Then $h\in \mathcal{S}$
- see, e.g., \cite[Proposition 1.92]{folland89} -
and $\int h=\lVert g_{1}\rVert
_{2}^{2}-\lVert g_{2}\rVert _{2}^{2}=0$ by~\eqref{eq_norm_wigner}. Hence, by
Proposition \ref{prop_s1},
\begin{equation*}
\lVert H_{m}^{g_{1}}-H_{m}^{g_{2}}\rVert _{S^{1}}=\lVert (m\ast h)^{w}\rVert
_{S^{1}}\lesssim \lVert m\ast h\rVert _{M^1},
\end{equation*}
Therefore, it suffices to prove that $\lVert m\ast h\rVert _{M^1}\lesssim
\mathit{var}(m)$. We apply Lemma \ref{lemma_M1_sobolev} to this end. First
note that $\partial _{x_{i}}(m\ast h)=\partial _{x_{i}}m\ast h$ and,
consequently,
\begin{equation*}
\lVert \partial _{x_{i}}(m\ast h)\rVert _{M^1}\lesssim \lVert \partial
_{x_{i}}m\rVert _{L^{1}}\lVert h\rVert _{M^1}\lesssim \mathit{var}(m).
\end{equation*}

Second, we exploit the fact that $\int h=0$ to get
\begin{align*}
(m\ast h)(z)& =\int_{\mathbb{R}^d}m(z^{\prime })h(z-z^{\prime })dz^{\prime
}=\int_{{\mathbb{R}^d}}(m(z^{\prime })-m(z))h(z-z^{\prime })dz^{\prime } \\
& =\int_{{\mathbb{R}^d}}\int_{0}^{1}\left\langle \nabla (m)(tz^{\prime
}+(1-t)z),z^{\prime }-z\right\rangle dt\,h(z-z^{\prime })dz^{\prime },
\end{align*}
and consequently
\begin{align*}
\int_{{\mathbb{R}^d}}\left\vert m\ast h(z)\right\vert dz& \leq
\int_{0}^{1}\int_{{\mathbb{R}^d}}\int_{{\mathbb{R}^d}}\left\vert \nabla
(m)(tz^{\prime }+(1-t)z)\right\vert \left\vert z^{\prime }-z\right\vert
\left\vert h(z-z^{\prime })\right\vert dz^{\prime }dzdt
\\
& =\int_{0}^{1}\int_{{\mathbb{R}^d}}\int_{{\mathbb{R}^d}}\left\vert \nabla(m)(tw+z)\right\vert
\left\vert w\right\vert \left\vert h(-w)\right\vert
dwdzdt \\
& =\lVert \nabla m\rVert_{L^{1}}\int_{0}^{1}\int_{{\mathbb{R}^d}}\left\vert
w\right\vert \left\vert h(w)\right\vert dwdt= \lVert \nabla
m\rVert_{L^{1}}\int_{{\mathbb{R}^d}}\left\vert w\right\vert \left\vert
h(w)\right\vert dw.
\end{align*}
Since $h\in \mathcal{S}$ the last integral is finite. We conclude that $\lVert
m*h\rVert_{L^{1}}\lesssim \lVert \nabla m\rVert_{L^{1}}=\mathit{var}(m)$.

This completes the argument for smooth, compactly supported $m$. For general
$m \in \mathrm{BV}({\mathbb{R}^d})$, there exists a sequence of smooth,
compactly supported functions $\left \{ m_k: k \geq 0 \right \} $ such that
$m_k \rightarrow m$ in $L^1$, and $\mathit{var}(m_k) \rightarrow
\mathit{var}(m)$, as $k \rightarrow +\infty$ (see for example
\cite[Sec.~5.2.2, Theorem 2]{evga92}.) By Proposition \ref{prop_s1}, $H^{g_i}_{m_k}
\rightarrow H^{g_i}_m$ in trace norm, and the conclusion follows by a
continuity argument.
\end{proof}

\subsection{The one-point intensity of finite polyanalytic ensembles}
\begin{proof}[Proof of Theorem \ref{asy_pure}]
We use Proposition \ref{prop_ident_2} to identify the $(r,N)$-polyanalytic ensemble
with a Weyl-Heisenberg ensemble with parameters $(h_{r},D_{R_{N}},I_{r,N})$.
By Lemma \ref{lemma_hn_eig},
\begin{equation*}
H_{D_{R_{N}}}^{h_{r}}=\sum_{j\geq 0}\mu_{j,R_N}^{r} \, h_j \otimes h_j.
\end{equation*}
Using Theorem \ref{th_change}, we compare the eigenvalues
of $H_{D_{R_{N}}}^{h_{0}}$ (related to the Ginibre ensemble) to those of
$H_{D_{R_{N}}}^{h_{r}}$.   We conclude that
\begin{equation}
\lVert \mu ^{0}-\mu ^{r}\rVert _{\ell ^{1}}=\lVert
H_{D_{R_{N}}}^{h_{0}}-H_{D_{R_{N}}}^{h_{r}}\rVert _{S^{1}}\leq
C_{r} {{\ensuremath{\left| {\partial {D_{R_{N}}}} \right| }}_{1}} \asymp  R_N 
\asymp \sqrt{N}. 
\label{eq_mus}
\end{equation}
The  estimate for the one-point intensity follows from 
\begin{align*}
&\lVert \rho_{r,N}-1_{D_{R_{N}}}\rVert _{L^{1}({\mathbb{C}})}
=\lVert \rho_{h_{r},D_{R_{N}},I_{r,N}}-1_{D_{R_{N}}}\rVert _{L^{1}({\mathbb{C}})}
&\qquad \mbox{by \eqref{eq_related}}
\\
&\qquad=2\sum_{j\notin I_{r,N}}\lambda_{j}(D_{R_{N}},h_{r})
&\qquad \mbox{by Lemma \ref{lemma_one_point_I}}
\\
&\qquad=2\sum_{j \geq N}\mu _{j}^{r}
&\qquad \mbox{by \eqref{eq_ident_2}}
\\
&\qquad \leq 2\sum_{j \geq N}\mu _{j}^{0}+2\lVert \mu^{0}-\mu ^{r}\rVert _{\ell ^{1}}
&\qquad \mbox{}
\\
&\qquad \leq 2\sum_{j\geq N}\mu _{j}^{0}+C_{r}\sqrt{N}
&\qquad \mbox{by \eqref{eq_mus}}
\\
&\qquad = 2\sum_{j\notin I_{0,N}}\lambda
_{j}(D_{R_{N}},h_{0})+C_{r}\sqrt{N}
&\qquad \mbox{by \eqref{eq_ident_2}}
\\
&\qquad = \lVert \rho_{h_{0},D_{R_{N}},I_{0,N}}-1_{D_{R_{N}}}\rVert
_{L^{1}({\mathbb{C}})}+C_{r}\sqrt{N}
&\qquad \mbox{by Lemma \ref{lemma_one_point_I}}
\\
&\qquad = \lVert \rho_{h_{0},D_{R_{N}}}-1_{D_{R_{N}}}\rVert _{L^{1}({\mathbb{C}})}+C_{r}\sqrt{N}
&\qquad \mbox{by Corollary \ref{coro_ident_3}}
\\
&\qquad \lesssim \sqrt{N} &\qquad \mbox{by Theorem \ref{th_quant_one}.}
\end{align*}
This completes the proof.
\end{proof}
Note that the proof combines our main  insights: the identification
of the finite polyanalytic ensembles with certain WH ensembles, the
analysis of the spectrum of time-frequency localization operators and
Toeplitz operators, and the non-asymptotic estimates of the
accumulated spectrum. 

\section{Double orthogonality}
\label{sec_do}

\subsection{Restriction versus localization}
\label{sec_duality}
Let $\mathcal{X}^g$ be an infinite WH ensemble on ${{{{\mathbb R}}^{2d}}}$ and $\Omega \subseteq {{{{\mathbb R}}^{2d}}}$
of finite measure and non-empty interior. We consider the \emph{restriction operator} $T^g_\Omega:
L^2({{{{\mathbb R}}^{2d}}}) \to
L^2({{{{\mathbb R}}^{2d}}})$,
\begin{align*}
T^g_\Omega F := 1_\Omega P_{{\mathcal{V}}_g} (1_\Omega \cdot F),
\end{align*}
and the \emph{inflated Toeplitz operator} $S^g_\Omega: L^2({{{{\mathbb R}}^{2d}}}) \to L^2({{{{\mathbb R}}^{2d}}})$,
\begin{align*}
S^g_\Omega F := P_{{\mathcal{V}}_g} (1_\Omega \cdot P_{{\mathcal{V}}_g} F).
\end{align*}
In view of  the decomposition 
$L^2({{{{\mathbb R}}^{2d}}}) = {\mathcal{V}}_{g} \oplus {\mathcal{V}}_{g}^\perp$, $S^g_\Omega$
and $M_{\Omega}^{g}$ are related by
\begin{align*}
S^g_{\Omega} = \left[
\begin{array}{cc}
M_{\Omega}^{g} & 0 \\
0 & 0
\end{array}
\right],
\end{align*}
and therefore share the same non-zero eigenvalues and the corresponding eigenspaces coincide. The
integral representation of $S^g$ is given by \eqref{Toe}.

Since $P_{{\mathcal{V}}_g}$ and $F \mapsto F \cdot 1_\Omega$ are both orthogonal projections,
$T^g_\Omega$ and $S^g_\Omega$ are both self-adjoint operators
with spectrum contained in $[0,1]$. The integral kernel of $T^g_\Omega$
is given by \eqref{eq_intro_ker_o} and $\int {{K^g}}_{|\Omega}(z,z) dz = {\ensuremath{\left| {\Omega} \right| }} <+\infty$.
Therefore, $T^g_\Omega$ is trace-class (see e.g. \cite[Theorems 2.12 and 2.14]{simonsbook}). It is an elementary fact that
$T^g_\Omega$ and $S^g_\Omega$ have the same non-zero eigenvalues with the same
multiplicities (this is true for $PQP$ and $QPQ$ whenever $P$ and $Q$ are orthogonal projections).
Moreover, the corresponding eigenspaces are related, for $\lambda \not=0$, by
the isometry 
\begin{align*}
{\ensuremath{\left \{ {F \in L^2({{{{\mathbb R}}^{2d}}}): S^g_\Omega F = \lambda F} \right \}}} &\longrightarrow
{\ensuremath{\left \{ {F \in L^2({{{{\mathbb R}}^{2d}}}): T^g_\Omega F = \lambda F} \right \}}}
\\
F &\longmapsto \frac{1}{\sqrt{\lambda}} 1_\Omega F.
\end{align*}
Therefore, if $M^g_\Omega$ is diagonalized as in \eqref{eq_eigenexp}, then
$T^g_\Omega$ can be expanded as in \eqref{eq_intro_ker_o2}. This justifies the discussion
in Section \ref{sec_intro_cons}.

\subsection{Simultaneous observability}
\label{sec_so}
Let $\mathcal{X}$ be a {determinantal point process}\ (with a Hermitian locally trace-class correlation kernel). We say
that a family of sets $\left\{ \Omega_\gamma:\gamma \in \Gamma\right\}$ is
\emph{simultaneously observable} for $\mathcal{X}$,  if the following happens. Let $\Omega=\bigcup_{\gamma \in \Gamma
}\Omega_\gamma$. There is an orthogonal basis $\{\varphi_{j}:j\in J\}$ of the closure of the range
of the restriction operator
$T_\Omega$ consisting of eigenfunctions of $T_{\Omega}$ such that for each $\gamma \in \Gamma$,
the set $\{\varphi_{j}|_{\Omega_\gamma}:j\in J\}$ of the restricted functions is
orthogonal. This is a slightly relaxed version of the notion in \cite[pg. 69]{DetPointRand}:
in the situation of the definition, the functions
$\{\varphi _{j}|_{\Omega_\gamma}:j\in J\} \setminus {\ensuremath{\left \{ {0} \right \}}}$ form an orthogonal basis of
the closure of the range of $T_{\Omega_\gamma}$, but we avoid making claims about the kernel of
$T_\Omega$. As explained in \cite[pg. 69]{DetPointRand}, the motivation for this terminology comes
from quantum mechanics, where two physical quantities can be  measured
simultaneously if the
corresponding operators commute (or, more concretely, if they have common eigenfunctions).

\begin{theorem}
\label{th_gab_sym} Let $\mathcal{D} = \left \{ D_R: R \in {\mathbb{R}}^+ \right \} $ be the
family of all disks of ${\mathbb{R}}^2$ centered at the origin. Then

\begin{itemize}
\item[(i)] $\mathcal{D}$ is simultaneously observable for the infinite
Weyl-Heisenberg ensemble with window $h_{r}$.

\item[(ii)] Let $D_{R_0}$ be a disk and $I\subseteq {\mathbb{N}}$. Then
$\mathcal{D}$ is simultaneously observable for the Weyl-Heisenberg ensemble
$\mathcal{X}^{h_r}_{D_{R_0}, I}$.
\end{itemize}
\end{theorem}

\begin{proof}
Let us prove (i). Since the definition of simultaneous observability involves the orthogonal
complement of the kernels of the restriction
operators $T^g_{D_R}$, $\mathrm{ker}\, T_{D_R}^g)^\perp $,  the discussion in
Section \ref{sec_duality} implies that it suffices to show that the Toeplitz operators
$M_{D_R}^{h_r}$ have a common basis of eigenfunctions. Since
$V_{h_r}^{\ast }M_{D_R}^{h_r}V_{h_r}=H_{D_R}^{h_r}$, and, by Proposition
\ref{prop_hn_disks}, the Hermite basis diagonalizes $H_{D_R}^{h_r}$ for all $R>0$, the conclusion
follows.

Let us now prove (ii). The  ensemble $\mathcal{X}^{h_r}_{D_{R_0}, I}$  is constructed by
selecting the eigenfunctions of the Toeplitz operator
$M_{D_{R_0}}^{h_r}:{{\mathcal{V}}_{h_r}}\rightarrow
{{\mathcal{V}}_{h_r}}$ corresponding to the indices in $I$:
\begin{equation*}
K^{h_r}_{D_{R_0},I}(z,z')=\sum_{j\in I}p^{D_{R_0}}_{h_r,j}(z)\overline{p^{D_{R_0}}_{h_r,j}(z')}.
\end{equation*}
Since, by part (i), the functions $p^\Omega_{g,j}$ are orthogonal when restricted to disks, the
conclusion follows.
\end{proof}

As a consequence, we obtain Theorem \ref{th_intro_sym}, which we restate for convenience.

\begin{reptheorem}{th_intro_sym}
The family $\mathcal{D} = \left \{ D_R: r \in {\mathbb{R}}^+ \right \} $ of all disks of
${\mathbb{C}}$ centered at the
origin is
simultaneously observable for every finite and infinite pure-type
polyanalytic ensemble.
\end{reptheorem}

\begin{proof}
This follows immediately from Proposition \ref{prop_ident_2} and Theorem \ref{th_gab_sym}.
(This slightly extends a result originally derived by Shirai \cite{SHIRAI}.)
\end{proof}

\subsection{An extension of Kostlan's theorem}

Theorem \ref{eq_thm4} is a consequence of the following slightly more general result.

\begin{theorem}
\label{th_poly_kostlan} Let $\mathcal{X}$ be the determinantal point process
associated with the $(r,J)$-pure polyanalytic ensemble, with $J\subseteq {{\mathbb N}}_{0}$ finite. Then the
point process on $[0,+\infty )$ of absolute values $\left\vert \mathcal{X}\right\vert $ has the same
distribution as the process generated by $\{Y_{j}: j\in J\}$ where the
$Y_{j}$'s are independent random variables with density
\begin{equation*}
f_{Y_{j}}(x):=2 \frac{\pi^{j-r+1} r!}{j!} x^{2(j-r)+1}\left[ L_{r}^{j-r}(\pi x^2)\right]^{2}e^{-\pi x^2}.
\end{equation*}
(Hence, $Y^2_j$ is distributed according to a generalized Gamma density function.)
\end{theorem}

\begin{proof}
We want to show that the point processes $\left\vert \mathcal{X}\right\vert
:=\sum_{x\in \mathcal{X}}\delta _{\left\vert x\right\vert }$ and $\mathcal{Y}
:=\sum_{j\in J}\delta _{Y_{j}}$ have the same distribution. Let
$I_{k}=[r_{k},R_{k}]$, $k=1,\ldots N$, be a disjoint family of subintervals
of $[0,+\infty )$. Then
\begin{equation*}
\left( \mathcal{Y}(I_{1}),\ldots ,\mathcal{Y}(I_{N})\right) \overset{d}{=}
\sum_{j\in J}{\zeta}_{j},
\end{equation*}
where the ${\zeta}_{j}$ are independent,
$\mathbb{P}({\zeta}_{j}=e_{k})=\int_{r_{k}}^{R_{k}}f_{Y_{j}}(x)dx$,
and
$\mathbb{P}({\zeta}_{j}=0)=\int_{{{\mathbb R}} \setminus \cup_{k}[r_{k},R_{k}]}f_{Y_{j}}(x)dx$.
On the other hand, Theorem \ref{th_intro_sym}
implies that the annuli $A_k:=
{\ensuremath{\left \{ {z \in {{\mathbb{{C}}}} : r_k \leq {\ensuremath{\left| {z} \right| }} \leq R_k} \right \}}}$ are simultaneously observable for $\mathcal{X}$. Hence,
by \cite[Proposition 4.5.9]{DetPointRand} - which is still applicable for the slightly more general
definition of simultaneous observability in  Section \ref{sec_so}, we have
\begin{equation*}
\left( \left\vert \mathcal{X}\right\vert (I_{1}),\ldots ,\left\vert \mathcal{
X}\right\vert (I_{N})\right) =\left( \mathcal{X}(A_{1}),\ldots ,\mathcal{X}
(A_{N})\right) \overset{d}{=}\sum_{j\in J}{\zeta} _{j}^{\prime },
\end{equation*}
where the ${\zeta}_{j}^{\prime }$ are independent,
$\mathbb{P}({\zeta}_{j}^{\prime }=e_{k})=
\int_{A_k}\left\vert H_{j,r}(z,\overline{z})\right\vert^{2}
e^{-\pi \left\vert z\right\vert^{2}} dz$,
and $\mathbb{P}({\zeta}_{j}^{\prime }=0)
=
\int_{{{\mathbb{{C}}}} \setminus \cup_k A_k}\left\vert H_{j,r}(z,\overline{z})\right\vert^{2}
e^{-\pi \left\vert z\right\vert^{2}} dz$. A direct calculation, together with
the identity
\begin{equation*}
\frac{(-x)^{k}}{k!}L_{r}^{k-r}(x)=\frac{(-x)^{r}}{r!}L_{k}^{r-k}(x)
\end{equation*}
shows that $\left({\zeta}_{j}: j \in J \right) \overset{d}{=} \left({\zeta}_{j}^{\prime}: j \in J \right)$
and the conclusion follows.
\end{proof}

\begin{rem}
{\rm Let $n(R)$ denote the number of points of a point process in the disk of
radius $R$ centered at the origin. An immediate consequence of Theorem \ref{th_poly_kostlan} is the following formula for the probability of finding
such a disk void of zeros, when the points are distributed according to the
a polyanalytic  Ginibre ensemble of the pure type:
\begin{equation*}
\mathbb{P}\left[ n(R)=0\right] =\prod_{j}P\left(  Y_{j} \geq R \right)
\end{equation*}
This is known as the hole probability (see \cite[Section 7.2]{DetPointRand}
for applications in the case of the Ginibre ensemble).}
\end{rem}

\appendix\section{Additional background material}
\label{sec_app}

\subsection{Determinantal point processes and intensities}
\label{sec_det}
We follow the presentation of \cite{Bor00, DetPointRand}. Let
$K: {{\mathbb R}}^d \times {{\mathbb R}}^d \to
{{\mathbb{{C}}}}$ be a locally trace-class Hermitian kernel with spectrum contained in $[0,1]$, and consider the
functions
\begin{align}
\rho_{n}(x_1, \ldots, x_n) := \det \left( K(x_j, x_k) \right)_{j,k=1,\ldots,d},
\qquad x_1, \ldots, x_n \in {{{{\mathbb R}}^d}}.
\end{align}
The Macchi-Soshnikov theorem implies that there exist a point process $\mathcal{X}$ on
${{\mathbb R}}^d$ such that for every family of disjoint measurable sets $\Omega_1, \ldots \Omega_n
\subseteq {{\mathbb R}}^d$,
\begin{align*}
\mathbb{E} \left[ \prod_{j=1}^n \mathcal{X}(\Omega_j) \right]
= \int_{\prod_j \Omega_j} \rho_{n}(x_1,\ldots,x_n)
dx_{1} \ldots dx_n,
\end{align*}
where $\mathcal{X}(\Omega)$ denotes the number of points of $\mathcal{X}$ to be found in $\Omega$.
The functions $\rho_{n}$ are known as correlation functions or intensities and $\mathcal{X}$ is
called a determinantal point process. The one-point intensity
$\rho$ is simply the diagonal of the correlation kernel
\begin{equation*}
\rho(z) = \rho_{1}(z)=K(x,x),
\end{equation*}
and allows one to compute the expected number of points to be found on a domain $\Omega$:
\begin{equation*}
\mathbb{E}\left[ \mathcal{X}(\Omega)\right] =\int_{\Omega}\rho (x)dx.
\end{equation*}
The one-point intensity can also be used to evaluate expectations of linear
statistics:
\begin{equation*}
\tfrac{1}{n}\mathbb{E}
\left[f(x_{1})+\ldots+f(x_{n})\right]=\mathbb{E}\left[ f(x_{1})\right]
=\int_\Rdstf(x)\rho(x)dx.
\end{equation*}

\subsection{Functions of bounded variation}
\label{sec_bv}
A real-valued function $f\in L^{1}({\mathbb{R}^{d}})$ is said to have \emph{bounded variation},
$f\in \mathrm{BV}({\mathbb{R}^{d}})$, if its
distributional partial derivatives are finite Radon measures. The variation
of $f$ is defined as
\begin{equation*}
\mathit{var}(f):=\sup \left\{ \int_{\mathbb{R}^{d}}f(x)\,
  \mathrm{div}\, \phi
(x)dx:\phi \in C_{c}^{1}({\mathbb{R}^{d}},{\mathbb{R}^{d}}),\left\vert \phi
(x)\right\vert _{2}\leq 1\right\},
\end{equation*}
where $C_{c}^{1}({\mathbb{R}^{d}},{\mathbb{R}^{d}})$ denotes the class of
compactly supported $C^{1}$-vector fields and $\mathrm{div}$ is the divergence
operator. If $f$ is continuously differentiable, then $f\in \mathrm{BV}({\mathbb{R}^{d}})$ simply
means that $\partial _{x_{1}}f,\ldots $, $\partial_{x_{d}}f\in L^{1}({\mathbb{R}^{d}})$, and
$\mathit{var}(f)=\int_{\mathbb{R}^{d}}\left\vert \nabla f(x)\right\vert _{2}dx=\lVert \nabla
f\rVert _{L^{1}}$. A set $\Omega \subseteq {\mathbb{R}^{d}}$ is said to have \emph{finite perimeter}
if its characteristic function $1_{\Omega}$ is of bounded variation, and the perimeter of $\Omega$ is defined
as  ${{\ensuremath{\left| {\partial {\Omega}} \right| }}_{d-1}} :=\mathit{var}(1_{\Omega}) $. If $\Omega$ has a smooth boundary, then
${{\ensuremath{\left| {\partial {\Omega}} \right| }}_{d-1}}$ is just the $(d-1)$-Hausdorff measure of the topological
boundary. See \cite[Chapter 5]{evga92} for an extensive discussion of $\mathrm{BV}$.

\subsection{Properties of modulation spaces}
\label{sec_mod}
Recall the definition of the modulation space $M^1$ in \eqref{eq_def_m1}.
It is well-known that, instead of the Gaussian function $\phi$, any
non-zero Schwartz function can be used to define ${M^{1}}$, giving an
equivalent norm ~\cite{fe81-2}, \cite[Chapter 9]{Charly}. Using this fact, the following lemma follows easily.

\begin{lemma}
\label{lemma_M1} Let $f\in {L^{2}({\mathbb{R}^{d}})}$. Then:

\begin{itemize}
\item[(i)] $f\in {M^1}({\mathbb{R}^{d}})$ if and only if $\hat{f}\in {M^1}({\mathbb{R}^{d}})$,
where $\hat{f}$ is the Fourier transform of $f$. In this case: $\lVert f\rVert _{{M^1}}\asymp \lVert \hat{f}\rVert _{{M^1}}$.

\item[(ii)] If $f$ is supported on
$D_1(0)=\{x:{\ensuremath{\left| {x} \right| }}\leq 1\}$ and $\hat{f}\in L^{1}({\mathbb{R}^{d}})$, then $f\in {M^1}({\mathbb{R}^{d}})$ and $\lVert f\rVert _{{M^1}}\lesssim \lVert \hat{f}\rVert _{L^{1}}$.

\item[(iii)] If $f\in {M^1}({\mathbb{R}^{d}})$ and $m\in C^{\infty }({\mathbb{R}^{d}})$ has bounded
derivatives of all orders, then $m\cdot f\in {M^1}({\mathbb{R}^{d}})$, and $\lVert m\cdot f\rVert
_{{M^1}}\leq C_{m}\lVert
f\rVert _{{M^1}}$, where $C_{m}$ is a constant that depends on $m $.
\end{itemize}
\end{lemma}

We now prove the Sobolev embedding lemma that was used in Section \ref{sec_change}.
\begin{proof}[Proof of Lemma \ref{lemma_M1_sobolev}]
Let $g$ be such that $\hat{g}=f$. By Lemma \ref{lemma_M1}, it suffices to
show that $g\in {M^{1}}(\mathbb{R})$ and satisfies a suitable norm estimate.
Let $\eta \in C^{\infty }(\mathbb{R})$ be such that $\eta (\xi )\equiv 0$
for $\left\vert \xi \right\vert \leq 1/2$ and $\eta (\xi )\equiv 1$ for $\left\vert \xi \right\vert
>1$. We write $\eta (\xi )=\sum_{k=1}^{d}\xi
_{k}\eta _{k}(\xi )$, where $\eta _{k}\in C^{\infty }(\mathbb{R})$ has
bounded derivatives of all orders. We set $g_{1}:=\eta \cdot g$ and $g_{2}:=(1-\eta )\cdot g$. Then
$g_{1}(\xi )=\sum_{k=1}^{d}\eta _{k}(\xi )\xi
_{k}g(\xi )$. Since $\xi _{k}g(\xi )=\tfrac{1}{2\pi i}\widehat{\partial
_{x_{k}}f}(\xi )$ is in $M^1$ by Lemma~\ref{lemma_M1}(i)  and $\eta _{k}$ has bounded derivatives of all orders, we
conclude from Lemma \ref{lemma_M1}(iii)  that $g_{1}\in {M^{1}}(\mathbb{R})$ and
that
\begin{equation*}
{\lVert{g_1}\rVert}_{M^1} \asymp
{\lVert{\widehat{g}_1}\rVert}_{M^1}
\lesssim
\sum_{k=1}^{d}
{\lVert{\xi_k \widehat{g}}\rVert}_{M^1}
\asymp \sum_{k=1}^{d}\lVert \partial
_{x_{i}}f\rVert _{{M^{1}}}.
\end{equation*}
On the other hand, since $g$ has an integrable Fourier transform, so does $g_{2}=(1-\eta )\cdot g$
and $\lVert \widehat{g_{2}}\rVert _{L^{1}}\lesssim
\lVert f\rVert _{L^{1}}$. In addition, $g_{2}$ is supported on $D_{1}(0)$.
Therefore, by Lemma \ref{lemma_M1}, $g_{2}\in {M^{1}}$ and $\lVert
g_{2}\rVert _{{M^{1}}}\lesssim \lVert f\rVert _{L^{1}}$. Hence $g=g_{1}+g_{2}\in {M^{1}}$,
and it satisfies the stated  estimate.
\end{proof}

\subsection{Polyanalytic Fock spaces}
\label{app_poly}

A complex-valued function $F(z,\overline{z})$ defined on a subset of
$\mathbb{C}$ is said to be \emph{polyanalytic of order }$q-1$, if it 
satisfies  the generalized Cauchy-Riemann equations
\begin{equation}
\left( \partial _{\overline{z}}\right) ^{q}F(z,\overline{z})=\frac{1}{2^{q}}
\left( \partial _{x}+i\partial _{\xi }\right) ^{q}F(x+i\xi ,x-i\xi )=0
\, .
\label{eq:c1}
\end{equation}
 Equivalently,
$F$ is a polyanalytic function of order $q-1$ if it can be written as
\begin{equation}
F(z,\overline{z})=\sum_{k=0}^{q-1}\overline{z}^{k}\varphi _{k}(z),
\label{polypolynomial}
\end{equation}
where the coefficients $\{\varphi _{k}(z)\}_{k=0}^{q-1}$ are analytic
functions. The \emph{polyanalytic Fock space}  $\mathbf{F}^{q}(\mathbb{C})$ consists of all the
polyanalytic functions of order $q-1$ contained in the  Hilbert space
 $L^{2}(\mathbb{C},e^{-\pi \left\vert z\right\vert ^{2}})$. The
reproducing kernel of the polyanalytic Fock space $\mathbf{F}^{q}(\mathbb{C}) $ is
\begin{equation*}
\mathbf{K}^{q}(z,z')=L_{q}^{1}(\pi \left\vert z-z'\right\vert ^{2})e^{\pi z \overline{z'}}.
\end{equation*}
Polyanalytic Fock spaces appear naturally in vector-valued time-frequency analysis
\cite{Abreusampling}, \cite{CharlyYurasuper} and signal multiplexing
\cite{balan00}.

Within $\mathbf{F}^{q}(\mathbb{C})$ we distinguish the \emph{$N$-dimensional polynomial space}
\begin{equation*}
Pol_{\pi ,q,N}=span\{z^{j}\overline{z}^{l}:0\leq j\leq N-1,0\leq l\leq q-1\},
\end{equation*}
 with the Hilbert space structure of
$L^{2}(\mathbb{C},e^{-\pi \left\vert z\right\vert ^{2}})$.
The polyanalytic Ginibre ensemble, introduced in \cite{HendHaimi}, is the DPP with correlation
kernel corresponding to the orthogonal projection onto $Pol_{\pi ,q,N}$
(weighted with the Gaussian measure). In \cite[Proposition
2.1]{HendHaimi} it is shown that
\begin{equation*}
Pol_{\pi ,q,N}=span\{H_{j,r}(z,\overline{z}):0\leq j\leq N-1,0\leq r\leq
q-1\},
\end{equation*}
where $H_{j,r}$ are the complex Hermite polynomials \eqref{ComplexHermite}. Thus, the
reproducing kernel of $Pol_{\pi ,q,N}$ can be written as
\begin{equation}
\mathbf{K}_{\pi ,N}^{q}(z,z')=\sum_{r=0}^{q-1}\sum_{j=0}^{N-1}H_{j,r}(z,\overline{z})
\overline{H_{j,r}(z',\overline{z'})}.
\label{eq_ker_poly_fock_qn}
\end{equation}

\subsection{Pure polyanalytic-Fock spaces}
\label{sec_pure_pol}

The pure polyanalytic Fock spaces $\mathcal{F}^{r}(\mathbb{C})$ have been introduced by Vasilevski in \cite{VasiFock},
under the name of true polyanalytic spaces. They are spanned by the complex Hermite polynomials of
fixed order $r$\ and can be defined as the set of polyanalytic functions $F$ integrable in
$L^{2}(\mathbb{C},e^{-\pi \left\vert z\right\vert ^{2}})$ and such that, for some entire function
$H$ \cite{Abreusampling},
\begin{equation*}
F(z)=\left( \frac{\pi ^{r}}{r!}\right) ^{\frac{1}{2}}e^{\pi \left\vert
z\right\vert ^{2}}\left( \partial _{z}\right) ^{r}\left[ e^{-\pi \left\vert
z\right\vert ^{2}}H(z)\right].
\end{equation*}
 Vasilevski~\cite{VasiFock} obtained the following decomposition of the
polyanalytic Fock space $\mathbf{F}^{q}(\mathbb{C})$ into pure
components 
\begin{equation}
\mathbf{F}^{q}(\mathbb{C})=\mathcal{F}^{0}(\mathbb{C})\oplus ...\oplus
\mathcal{F}^{q-1}(\mathbb{C}).  \label{orthogonal}
\end{equation}
Pure polyanalytic spaces are important in signal analysis \cite{Abreusampling} and in connection to
theoretical physics \cite{AoP, HendHaimi}. Indeed,
they parameterize the so-called \emph{Landau levels}, which are the eigenspaces of the
Landau Hamiltonian and model
the distribution of electrons in high energy states (see e.g. \cite[Section 2]{SHIRAI}, \cite[Section 4.1]{APRT}).

The complex Hermite polynomials \eqref{ComplexHermite} provide a natural way of defining a
polynomial subspace of the true polyanalytic space:
\begin{equation*}
\mathcal{P}ol_{\pi ,r,N}=span\{H_{j,r}(z,\overline{z}):0\leq j\leq N-1\}.
\end{equation*}
Thus,
\begin{equation*}
Pol_{\pi ,q,N}=\mathcal{P}ol_{\pi ,0,N}\oplus ...\oplus \mathcal{P}ol_{\pi
,q,N}.
\end{equation*}
The reproducing kernels of $Pol_{\pi ,r,N}$ is therefore
\begin{equation*}
\mathcal{K}_{r,\pi ,N}(z,z')=\sum_{j=0}^{N-1}H_{j,r}(z,\overline{z})
\overline{H_{j,r}(z',\overline{z'})},
\end{equation*}
and the corresponding determinantal point processes have been introduced
in \cite{HendHaimi}.

\begin{thebibliography}{10}

\bibitem{special}
M.~Abramowitz.
\newblock {\em Handbook of Mathematical Functions, With Formulas, Graphs, and
  Mathematical Tables,}.
\newblock Dover Publications, Incorporated, 1974.

\bibitem{Abreusampling}
L.~D. Abreu.
\newblock Sampling and interpolation in {B}argmann-{F}ock spaces of
  polyanalytic functions.
\newblock {\em Appl. Comput. Harmon. Anal.}, 29(3):287--302, 2010.

\bibitem{AoP}
L.~D. Abreu, P.~Balazs, M.~de~Gosson, and Z.~Mouayn.
\newblock Discrete coherent states for higher {L}andau levels.
\newblock {\em Ann. Physics}, 363:337--353, 2015.

\bibitem{AGR}
L.~D. Abreu, K.~Gr\"ochenig, and J.~L. Romero.
\newblock On accumulated spectrograms.
\newblock {\em Trans. Amer. Math. Soc.}, 368(5):3629--3649, 2016.

\bibitem{APR}
L.~D. {A}breu, J.~M. {P}ereira, and J.~L. {R}omero.
\newblock Sharp rates of convergence for accumulated spectrograms.
\newblock  {\em https://arxiv.org/pdf/1704.02266.pdf}

\bibitem{APRT}
L.~D. {A}breu, J.~M. {P}ereira, J.~L. {R}omero, and S.~{T}orquato.
\newblock {T}he {W}eyl-{H}eisenberg ensemble: hyperuniformity and higher
  {L}andau levels.
\newblock {\em J. Stat. Mech. Theor. Exp.}, To appear. Ar{X}iv:1612.00359,
  2016.

\bibitem{AMH1}
Y.~Ameur, H.~Hedenmalm, and N.~Makarov.
\newblock Fluctuations of eigenvalues of random normal matrices.
\newblock {\em Duke Math. J.}, 159(1):31--81, 2011.

\bibitem{AMH2}
Y.~Ameur, H.~Hedenmalm, and N.~Makarov.
\newblock Random normal matrices and {W}ard identities.
\newblock {\em Ann. Probab.}, 43(3):1157--1201, 2015.

\bibitem{balan00}
R.~V. Balan.
\newblock Multiplexing of signals using superframes.
\newblock In {\em International Symposium on Optical Science and Technology},
  pages 118--129. International Society for Optics and Photonics, 2000.

\bibitem{Bor00}
A.~Borodin and G.~Olshanski.
\newblock Distributions on partitions, point processes, and the hypergeometric
  kernel.
\newblock {\em Comm. Math. Phys.}, 211(2):335--358, 2000.

\bibitem{BO2}
A.~Borodin and G.~Olshanski.
\newblock Representation theory and random point processes.
\newblock In {\em European {C}ongress of {M}athematics}, pages 73--94. Eur.
  Math. Soc., Z\"urich, 2005.

\bibitem{cogr03}
E.~Cordero and K.~Gr\"ochenig.
\newblock Time-frequency analysis of localization operators.
\newblock {\em J. Funct. Anal.}, 205(1):107--131, 2003.

\bibitem{Daubechies}
I.~Daubechies.
\newblock Time-frequency localization operators: a geometric phase space
  approach.
\newblock {\em IEEE Trans. Inform. Theory}, 34(4):605--612, 1988.

\bibitem{dego11}
M.~A. de~Gosson.
\newblock {\em Symplectic methods in harmonic analysis and in mathematical
  physics}, volume~7 of {\em Pseudo-Differential Operators. Theory and
  Applications}.
\newblock Birkh\"auser/Springer Basel AG, Basel, 2011.

\bibitem{Deift}
P.~Deift.
\newblock Universality for mathematical and physical systems.
\newblock In {\em International {C}ongress of {M}athematicians. {V}ol. {I}},
  pages 125--152. Eur. Math. Soc., Z\"urich, 2007.

\bibitem{doro14}
M.~D\"orfler and J.~L. Romero.
\newblock Frames adapted to a phase-space cover.
\newblock {\em Constr. Approx.}, 39(3):445--484, 2014.

\bibitem{Dunne}
G.~V. Dunne.
\newblock Edge asymptotics of planar electron densities.
\newblock {\em Int. J. Modern Phys. B}, 8(11n12):1625--1638, 1994.

\bibitem{evga92}
L.~C. Evans and R.~F. Gariepy.
\newblock {\em Measure theory and fine properties of functions}.
\newblock Studies in Advanced Mathematics. CRC Press, Boca Raton, FL, 1992.

\bibitem{fe81-2}
H.~G. Feichtinger.
\newblock On a new {S}egal algebra.
\newblock {\em Monatsh. Math.}, 92(4):269--289, 1981.

\bibitem{folland89}
G.~B. Folland.
\newblock {\em Harmonic analysis in phase space}, volume 122 of {\em Annals of
  Mathematics Studies}.
\newblock Princeton University Press, Princeton, NJ, 1989.

\bibitem{MR2226126}
G.~B. Folland.
\newblock The abstruse meets the applicable: some aspects of time-frequency
  analysis.
\newblock {\em Proc. Indian Acad. Sci. Math. Sci.}, 116(2):121--136, 2006.

\bibitem{ghosh2}
S.~Ghosh.
\newblock Determinantal processes and completeness of random exponentials: the
  critical case.
\newblock {\em Probab. Theory Related Fields}, 163(3-4):643--665, 2015.

\bibitem{ghosh1}
S.~Ghosh and J.~L. Lebowitz.
\newblock Fluctuations, large deviations and rigidity in hyperuniform systems:
  a brief survey.
\newblock {\em arXiv:1608.07496 [math.PR]}.

\bibitem{gr96}
K.~Gr\"ochenig.
\newblock An uncertainty principle related to the {P}oisson summation formula.
\newblock {\em Studia Math.}, 121(1):87--104, 1996.

\bibitem{Charly}
K.~Gr\"ochenig.
\newblock {\em Foundations of time-frequency analysis}.
\newblock Applied and Numerical Harmonic Analysis. Birkh\"auser Boston, Inc.,
  Boston, MA, 2001.

\bibitem{CharlyYurasuper}
K.~Gr\"ochenig and Y.~Lyubarskii.
\newblock Gabor (super)frames with {H}ermite functions.
\newblock {\em Math. Ann.}, 345(2):267--286, 2009.

\bibitem{HendHaimi}
A.~Haimi and H.~Hedenmalm.
\newblock The polyanalytic {G}inibre ensembles.
\newblock {\em J. Stat. Phys.}, 153(1):10--47, 2013.

\bibitem{HaiHen2}
A.~Haimi and H.~Hedenmalm.
\newblock Asymptotic expansion of polyanalytic {B}ergman kernels.
\newblock {\em J. Funct. Anal.}, 267(12):4667--4731, 2014.

\bibitem{heil2}
C.~{H}eil, J.~{R}amanathan, and P.~{T}opiwala.
\newblock {\em {A}symptotic {S}ingular {V}alue {D}ecay of {T}ime-frequency
  {L}ocalization {O}perators}, volume 2303, pages 15--24.
\newblock {O}ctober 1994.

\bibitem{heil1}
C.~Heil, J.~Ramanathan, and P.~Topiwala.
\newblock Singular values of compact pseudodifferential operators.
\newblock {\em J. Funct. Anal.}, 150(2):426--452, 1997.

\bibitem{DetPointRand}
J.~B. Hough, M.~Krishnapur, Y.~Peres, and B.~Vir\'ag.
\newblock {\em Zeros of {G}aussian analytic functions and determinantal point
  processes}, volume~51 of {\em University Lecture Series}.
\newblock American Mathematical Society, Providence, RI, 2009.

\bibitem{Ismail}
M.~E.~H. Ismail.
\newblock Analytic properties of complex {H}ermite polynomials.
\newblock {\em Trans. Amer. Math. Soc.}, 368(2):1189--1210, 2016.

\bibitem{MR3537240}
M.~E.~H. Ismail and R.~Zhang.
\newblock Kibble-{S}lepian formula and generating functions for 2{D}
  polynomials.
\newblock {\em Adv. in Appl. Math.}, 80:70--92, 2016.

\bibitem{Kostlan}
E.~Kostlan.
\newblock On the spectra of {G}aussian matrices.
\newblock {\em Linear Algebra Appl.}, 162/164:385--388, 1992.
\newblock Directions in matrix theory (Auburn, AL, 1990).

\bibitem{lerner03}
N.~Lerner.
\newblock The {W}ick calculus of pseudo-differential operators and some of its
  applications.
\newblock {\em Cubo Mat. Educ.}, 5(1):213--236, 2003.

\bibitem{Perelomovbook}
A.~M. Perelomov.
\newblock {\em Generalized coherent states and their applications}.
\newblock Springer-Verlag, Berlin, 1986.

\bibitem{PHRE2009}
A.~Scardicchio, C.~E. Zachary, and S.~Torquato.
\newblock Statistical properties of determinantal point processes in
  high-dimensional {E}uclidean spaces.
\newblock {\em Phys. Rev. E (3)}, 79(4):041108, 19, 2009.

\bibitem{Seip}
K.~Seip.
\newblock Reproducing formulas and double orthogonality in {B}argmann and
  {B}ergman spaces.
\newblock {\em SIAM J. Math. Anal.}, 22(3):856--876, 1991.

\bibitem{SHIRAI}
T.~Shirai.
\newblock Ginibre-type point processes and their asymptotic behavior.
\newblock {\em J. Math. Soc. Japan}, 67(2):763--787, 2015.

\bibitem{simonsbook}
B.~Simon.
\newblock {\em Trace ideals and their applications}, volume 120 of {\em
  Mathematical Surveys and Monographs}.
\newblock American Mathematical Society, Providence, RI, second edition, 2005.

\bibitem{TVK}
T.~Tao and V.~Vu.
\newblock Random matrices: universality of {ESD}s and the circular law.
\newblock {\em Ann. Probab.}, 38(5):2023--2065, 2010.
\newblock With an appendix by Manjunath Krishnapur.

\bibitem{verdu}
A.~M. Tulino and S.~Verd\'{u}.
\newblock Random matrix theory and wireless communications.
\newblock {\em Foundations and Trends® in Communications and Information
  Theory}, 1(1):1--182, 2004.

\bibitem{VasiFock}
N.~L. Vasilevski.
\newblock Poly-{F}ock spaces.
\newblock In {\em Differential operators and related topics, {V}ol. {I}
  ({O}dessa, 1997)}, volume 117 of {\em Oper. Theory Adv. Appl.}, pages
  371--386. Birkh\"auser, Basel, 2000.

\end{thebibliography}
\end{document}

