\documentclass[12pt]{amsart}

\usepackage{amsmath,amsthm,amssymb,mathrsfs,amsfonts,verbatim,enumitem}
\usepackage{etoolbox} 
\usepackage{bbm}
\usepackage[all,tips]{xy}
\usepackage{graphicx,ifpdf}
\ifpdf
   \DeclareGraphicsRule{*}{mps}{*}{}
\fi

\newtheorem{thm}{Theorem}[section]
\newtheorem{lem}[thm]{Lemma}
\newtheorem{cor}[thm]{Corollary}
\newtheorem{prop}[thm]{Proposition}
\newtheorem{prob}[thm]{Problem}

\theoremstyle{definition}
\newtheorem{de}[thm]{Definition}
\newtheorem{ex}[thm]{Example}
\newtheorem{xca}[thm]{Exercise}

\theoremstyle{remark}
\newtheorem{rem}[thm]{Remark}
\newtheorem{assume}[thm]{Assumption}

\numberwithin{equation}{section}

\begin{document}

\title{Pointwise equidistribution  for    one parameter diagonalizable group action on homogeneous space}

\author{Ronggang Shi}
\address{School of Mathematical Sciences, Tel Aviv University, Tel Aviv 69978, Israel, and
School of Mathematical Sciences, Xiamen University, Xiamen 361005, PR China}
 
 \email{ronggang@xmu.edu.cn}
\thanks{The author is supported by NSFC (11201388), NSFC (11271278),
 ERC starter grant DLGAPS 279893.}

\subjclass[2000]{Primary   28A33; Secondary 37C85, 37A30.}

\date{}

\keywords{homogeneous dynamics,  equidistribution, ergodic theorem}

\begin{abstract}
\noindent
Let $\Gamma$ be a lattice of a semisimple Lie group $L$. 
 Suppose that one parameter  Ad-diagonalizable subgroup  
$\{g_t\}$ of  $L$ acts ergodically on $L/\Gamma$ 
with respect to the probability Haar measure $\mu$.
For certain proper subgroup $U$ of 
the unstable horospherical subgroup  of $\{g_t\}$ we show that given $x\in L/\Gamma$
for almost every  $u\in U$   the trajectory
$\{g_tux: 0\le t\le T\}$ is uniformly distributed   with respect to $\mu$ as $T\to \infty$.
\end{abstract}

\maketitle

\markright{}

\section{introduction}\label{sec;intro}

Let $(X, \mathcal B,\mu, T)$ be a probability measure preserving system, i.e. 
$\mu$ is a probability measure on the measurable space $(X, \mathcal B)$ and
the measurable map
$T:X\to X$ preserves $\mu$. 
The Birkhoff  ergodic theorem 
says that 
if $T$ is ergodic then  for every $f\in L^1_\mu(X)$ 
\begin{equation}\label{eq;birkhoff}
\lim_{N\to \infty}\frac{1}{N} \sum_{n=0}^{N-1}f(T^nx)=\int_Xf\, d\mu
\end{equation}
for  almost every $x\in X$. 

Suppose that   $X$ is a locally compact second countable  
 Hausdorff 
 topological space and $\mathcal B$ is the Borel sigma algebra
of $X$. 
Given $x\in X$ the condition 
 that   (\ref{eq;birkhoff}) holds for  every  $f$ belonging to   the set $C_c(X)$ of 
   continuous functions with compact support
is equivalent to 
\begin{equation}\label{eq;question}
\lim_{N\to \infty}\frac{1}{N} \sum_{n=0}^{N-1}\delta_{T^nx}=\mu
\end{equation}
in the space of finite measures on $X$ under the weak$^*$ topology. 
Here $\delta_y$ denotes the Dirac measure supported on $y\in X$.
A Radon measure $\nu$ on $X$ is said to be $(T,\mu)$ generic if   (\ref{eq;question}) holds
  for $\nu$ almost every $x\in X$. 
 A natural question is whether a measure $\nu$ (usually singular to $\mu$) is $(T,\mu)$ generic.
  
This question is studied by several authors for natural dynamical systems on $X=\mathbb R/\mathbb Z$. 
Let $m,n$ be coprime positive integers greater than or equal to $2$.
Suppose that     $\mu_X$ is the Lebesgue measure on $X$ and  $T_n=\times n $ modulo $\mathbb Z$.
Host \cite{h95} shows that
any  $T_m$ invariant and ergodic probability measure $\nu$ on
$X$ with positive entropy
  is  $(T_n, \mu_X)$ generic. This result  is strengthened by 
 Hochman and  Shmerkin \cite{hs} where they prove that for any $C^2$ diffeomorphism  $\varphi: \mathbb R\to \mathbb R$, the push forward of $\nu$ modulo $\mathbb Z$
is $(T_n, \mu_X)$ generic. The reader can find detailed references of related results in \cite{hs}.

The aim of this paper is to address this question for one parameter  flows in homogeneous space. 
Let $\Gamma$ be a  lattice
of a   Lie group  $L$.
Every subgroup $H$ of $L$
 acts on $L/\Gamma$ by left translations and this action preserves the probability Haar measure $\mu_{L/\Gamma}$. 
 We use $(H, L/\Gamma)$ to denote this measure preserving system. 
 There are two basic types of one parameter subgroup $t\to g_t\in L$ in terms of its image under the adjoint 
representation $Ad: L\to GL(\mathfrak l)$ where $\mathfrak l$ is the Lie algebra of $L$.
If  $Ad(g_t)$  is unipotent, then according to  Ratner's uniform distribution theorem \cite{r912}  
    the Dirac measure $\delta_x$ of any point $x\in L/\Gamma$ is generic 
with respect to some $\{g_t:t\in \mathbb R\}$ ergodic homogeneous probability measure.
If  the 
one parameter subgroup is
 Ad-diagonalizable, i.e.~$Ad(g_t)$
 is diagonalizable over $\mathbb R$, the unstable horospherical subgroup
 of $\{g_t: t\in \mathbb R\}$  is defined by 
\[
U^+_L=\{h\in  L: g_t^{-1}hg_t\to \mathbf e \mbox{ as } t\to \infty\}.
\]
Here and 
throughout the paper we use   the bold faced letter 
  $\mathbf e$ to denote  neutral element of group.
A   variant of Birkhoff ergodic theorem says that 
if  
   $(\{g_t :t\in \mathbb R\}, L/\Gamma)$ is ergodic then  
 given any $x\in L/\Gamma$ and any    $f\in C_c(L/\Gamma)$
\begin{equation}\label{eq;generic}
\lim_{T\to \infty}\frac{1}{T}\int_0^T f(g_tux)\, dt=\int_{L/\Gamma} f \, d\mu_{L/\Gamma}
\end{equation}
holds for almost every  $u\in U_L^+$ with respect to the Haar measure of $U_L^+$.
Suppose that $\mu$  is a  $\{g_t: t\in\mathbb R\}$ invariant  probability measure  on $L/\Gamma$.
We say that a   Radon measure  $\nu$ on $U^+_L$  or 
more generally on $L$  is  $(g_t, \mu)$ generic  at $x\in L/\Gamma$
if for any  $f\in C_c(L/
\Gamma)$ and   $\nu$ almost every $u $ we have  (\ref{eq;generic}) holds.
We remark here that the property of being $(g_t, \mu)$ generic only depends on the equivalence class 
of the measure  $\nu$. 

Unlike one parameter Ad-unipotent subgroup few results are known for Ad-diagonalizable subgroup when
 $\nu$ on $U_L^+$  is singular to the Haar measure. 
 We do know many examples of  probability measure $\nu$ whose pushforward  image 
 under $g_t$ as $t\to \infty$ or  trajectory under $\{g_t:0\le t\le T\}$ as $T\to\infty$
  is 
 equidistributed with respect to some probability homogeneous measure.
The   reader can find precise description of these  measures  for asymptotic results in Shah \cite{s96}\cite{s09}\cite{s092}\cite{s093}\cite{s10}, 
Shah and Weiss \cite{sw96}; and for average results by author in  
\cite{shi12}\cite{shi122}.
 

 
 
 
We investigate pointwise equidistribution for measures studied in \cite{s96} and \cite{sw96} above. 
Let $G\le L$ be a connected semisimple Lie group  without compact factors.  Ratner's theorem \cite{r912} implies that 
for any $x\in L/\Gamma$ 
the orbit closure $\overline {Gx}$ is a finite volume homogeneous space, i.e. $\overline {Gx}=Hx$
 where $H=\{g\in L: g\overline {Gx}=\overline {Gx}\}$ and there is a unique $H$ invariant 
 probability measure (denoted by $\mu_{\overline{Gx}}$) supported on $\overline {Gx}$. 
The main result of this paper is:
\begin{thm}\label{thm;1}
Let  $\{g_t :t\in \mathbb R\}$ be an  Ad-diagonalizable 
one parameter subgroup of a connected semisimple Lie group  $G$ without compact factors.
Suppose  that the projection of $g_t$ to each simple factor of $G$ is nontrivial. 
Let $\Gamma$ be a lattice of a Lie group  $L$
which contains $G$. Then for every $x\in L/\Gamma$  the  Haar measure of $U^+_G$ is 
 $(g_t, \mu_{\overline {Gx}})$ generic at $x$.
\end{thm}

Our result is new in the following simple case: $G=
\left (\begin{array}{cc}
SL_2(\mathbb R) & 0 \\
0 & 1
\end{array}
\right )$, $g_t=\mbox{diag}(e^t, e^{-t},1), 
L=SL_3(\mathbb R), \Gamma=SL_3(\mathbb Z)$.
The key property we use for the group $U_G^+$ is the $ g_1 $   expanding property which 
we describe now. 
Let $\{g_t :t\in \mathbb R\}$ and $G$ be as in Theorem \ref{thm;1}.
 
Every representation $\rho$ of $G$ on a finite dimensional real  vector space $V$ splits
into a direct sum $V^+\oplus V^0\oplus V^-$  of $\rho(g_1)$ invariant subspaces so that 
restrictions of $\rho(g_1)$ to the spaces 
 $V^+, V^0, V^-$ have eigenvalues  $>,=,<1$ respectively. Let   $\pi_+$  be the projection from $V$ to $V^+$. 
A connected subgroup $U$ of $G$ normalized by $g_t$ is said to be  $ g_1 $ expanding if
 for every nontrivial
irreducible  representation $\rho$ of $G$ on $V$ and every nonzero vector $v\in V$ one has that the map 
\[
 U\to V \quad \mbox{given by}\quad  u\to  \pi_+(\rho(u)v)
\]
is not identically zero. 
It can be proved that    $U$ is $g_1$ expanding if and only if 
$U\cap U_G^+$ is $g_1$ expanding.

One family of $ g_1  $ expanding subgroups comes from epimorphic subgroups of algebraic groups
introduced by  Bien and Borel \cite{bb}.
Suppose that $G$ is the connected component of real points of some
 semisimple linear algebraic group defined over $\mathbb R$.
Let  $S\le G$ be a one dimensional $\mathbb R$ split algebraic torus and let $U$ be a unipotent algebraic 
subgroup of $G$ normalized by $S$.  
Let   $H$  be the subgroup  generated by $S$  and $U$.
The group $H$ is  epimorphic in $G$ if 
any $H$ fixed  vector of 
 an   algebraic  representation of $G$
   is also  fixed by $G$. 
 It is proved in  \cite{sw96} Proposition 2.2 that if $H$ is an epimorphic subgroup of $G$ then $U$ is  $ g_1 $ expanding for some choice of the parameterization of the connected component of $S$.
 
 
 Under an additional abelian assumption for $g_1$ expanding  subgroup  $U$ we  prove the following 
\begin{thm}\label{thm;2}
Let  $\{g_t :t\in \mathbb R\}$ be an  Ad-diagonalizable
one parameter subgroup of a connected semisimple Lie group  $G$ without compact factors.
Let $\Gamma$ be a lattice of a Lie group  $L$ which
contains $G$.  
Suppose that  $U\le U_G^+$ is a connected  $ g_1 $ expanding abelian  subgroup of $G$.
Then for every  $x\in L/\Gamma$   the  Haar measure of $U$ is 
 $(g_t, \mu_{\overline {Gx}})$ generic at $x$.
\end{thm}

Here we give some concrete examples that Theorem \ref{thm;2} applies. Let $m,n$
be two positive integers and let $\mathbf v=(a_1, \ldots, a_m, -b_1, \ldots,- b_n)$
where $a_i, b_j>0$ and $a_1+\cdots+ a_m=b_1+\cdots+b_n$. 
For every $\xi\in M_{mn}$ where  $ M_{mn}$ is the set   of $m\times n$
matrices, we let 
$u(\xi)=\left(
\begin{array}{cc}
I_m  & \xi\\
0    & I_n
\end{array}
\right)$
where $I_m$ and $I_n$ are identity matrices of order $m$ and $n$ respectively.
Let  diagonal matrix  $g_{t\mathbf v}=\mathrm{diag}(e^{a_1t}, \ldots, e^{a_mt}, e^{-b_1t},\ldots, e^{-b_nt})$. 
It follows from Kleinbock and Weiss \cite{kw08} Proposition 2.4 that the group
$U=\{u(\xi): \xi\in M_{mn}\}$ is $g_{\mathbf v}$ expanding. Therefore as a special case 
of Theorem \ref{thm;2} we have 
\begin{cor}
Let $\Gamma$ be a lattice of $G=SL(m+n, \mathbb R)$ and let $\mu$
be the probability Haar measure on $G/\Gamma$. Then for every $x\in G/\Gamma$ the 
additive Haar measure of $U=\{u(\xi): \xi\in M_{mn}\}$ is $(g_{t\mathbf v}, \mu)$
generic at $x$.
\end{cor}

The abelian assumption of Theorem \ref{thm;2} for the group $U$ might be superfluous. 
The only place where we essentially need  it is the shadowing Lemma \ref{lem;shade} and its variant
Lemma \ref{lem;new eq} which 
are links between random walks and flows. 
We do not know how to get shadowing lemma and  simultaneously the  contraction property
Lemma \ref{lem;estimate} 
 even in the case where $U$ is the two step  Heisenberg group. 
 This is also the main obstruction that  we cannot apply our method 
 to  the case of volume measures of  curves studied by  \cite{s09}\cite{s092}\cite{s093}\cite{s10}, 
e.g.~nonplanar analytic  curves
 in $U_G^+$ where $G=SL(n, \mathbb R)$ and $g_t=\mbox{diag}(e^{(n-1)t}, e^{-t},\ldots, e^{-t})$. 
Theorem \ref{thm;1} is deduced from Theorem \ref{thm;2} and the asymptotic  equidistribution of measures
proved in  \cite{sw96}. 
This type of deduction  might  be able to  prove pointwise equidistribution in some 
other cases where $U$ is not abelian.

The proof of Theorem \ref{thm;2} is based on
  quantitative estimate of the
$\{g_t: 0\le t\le T\}$ trajectory of  measures. The method is inspired by 
 Chaika and  Eskin \cite{ce} where they prove 
Birkhoff type ergodic theorem for Teichmuller geodesic flows on moduli spaces and 
 by Benoist and Quint \cite{bq132} where they prove almost everywhere equidistribution of Random walks on 
homogeneous space. 

\textbf{Acknowledgements:}
We would like to thank  Barak Weiss for suggesting  this problem and generously 
sharing his ideas. 
We also would like to thank Yves Benoist, Jean-Francois Quint, 
Alex Eskin and Alexander Gorodnik for discussions related to this work. 

\section{Outline of the proof}\label{s;outline}

We first  outline the proof of Theorem \ref{thm;2} and leave details of the proof of Proposition 
\ref{prop;inv}, \ref{prop;nescape} and \ref{prop;singular} for later sections. 
Let $ G, g_t, U$ be as in Theorem \ref{thm;2}. 
In particular 
$U\le U_G^+$ is a connected abelian $g_1$ expanding  subgroup   of $G$. 
It follows from Ratner \cite{r90} Proposition 1.3 that 
$U$ is  simply 
 connected. We fix an isomorphism of Lie groups 
 \begin{equation}\label{eq;isomorphism}
 u:\mathbb R^m\to U
 \end{equation}
 so that there are positive real numbers $b_1, \ldots, b_m$ 
 such that  for  standard basis $\{\mathbf e_i\}_{1\le i\le m}$
of $\mathbb R^m $ one has 
\begin{equation}\label{eq;Recall}
g_tu(\mathbf e_i)g_{-t}=u(e^{tb_i}\mathbf e_i).
\end{equation}

It is not hard to see that  Theorem \ref{thm;2} follows  from 
\begin{thm}\label{thm;proof}
Let  $\{g_t :t\in \mathbb R\}$ be an  Ad-diagonalizable
one parameter subgroup of a connected semisimple Lie group  $G$ without compact factors.
Let $\Gamma$ be a lattice of a Lie group  $L$ which
contains $G$.  
Suppose that  $U\le U_G^+$ is a connected  $ g_1 $ expanding abelian  subgroup of $G$.
Let $x\in L/\Gamma$, let the interval $I=[-1, 1]$ and let $u$ be a fixed isomorphism as in (\ref{eq;isomorphism})
so that (\ref{eq;Recall}) holds. Then 
 for almost every $w\in I^m$ 
  \begin{equation}\label{eq;goalpoint}
\lim_{T\to \infty}  \frac{1}{T}\int_0^T g_tu(w)\delta_{x} \, dt=\mu_{\overline{Gx}}.
\end{equation}
\end{thm}
 

Here $g_t u(w)\delta_x$ is the pushforward of the Dirac measure $\delta_x$  by $g_tu(w)$ and 
it is equal to $\delta_{g_t u(w)x}$.
In the rest of this section we assume the notation and assumption in Theorem \ref {thm;proof}.
To prove it we first establish unipotent invariance:
\begin{prop}\label{prop;inv}
For almost every $w\in I^m$, if $\nu_w$ is any weak$^*$ limit point of
$\frac{1}{T}\int_0^T g_t u(w) \delta_xdt$ as $T\to \infty$, then $\nu_w$ is invariant under $U$.
\end{prop}

 

Next we prove  nonescape  of mass (Corollary \ref{cor;nescape})
using    quantitative nonescape of mass for $\{g_t: 0\le t\le T\}$ trajectory of the measure  associated to
the   Lebesgue measure on $I^m$. For every measurable subset $K$ of $X$, positive real number $T$ and 
$w\in I^m$ we use $\mathcal A_K^T(w)$ to denote the proportion of the trajectory $\{g_tu(w)x: 0\le t\le T\}$ in $K$. 
More precisely,
\begin{equation}\label{eq;average}
\mathcal A_K^T(w):=\frac{1}{T}\int_0^T \mathbbm 1 _K(g_tu(w)x)\,dt
\end{equation}
where $\mathbbm 1_K$ is the characteristic function of $K$. 
For every measurable subset $J$ of $\mathbb R^m$ we let $|J|$ to denote 
the Lebesgue measure of $J$.
\begin{prop}\label{prop;nescape}
For every $0<\epsilon <1 $, there is a compact subset 
 $K$ of $L/\Gamma$ and a positive real  number  $c<1$ such that 
 \begin{equation}\label{eq;pnes}
\left |\{w\in I^m: \mathcal A_K^T(w)\le 1- \epsilon   \}\right |\le  c^T
 \end{equation}
 for every $T\ge 0$.
\end{prop}

\begin{cor}\label{cor;nescape}
For almost every $w\in I^m$, any weak$^*$ limit point of 
$\frac{1}{T}\int_0^T g_tu(w)\delta_{x} dt$ as $T\to \infty $ is a probability measure.
\end{cor}
\begin{proof}
Given $0< \epsilon <1$,  according to Proposition \ref{prop;nescape} there exists   a compact subset $K$ of $L/\Gamma $
 and a positive number  $c<1$ so that  (\ref{eq;pnes}) holds
  as $T$ runs through all the positive integers.
So
 Borel-Cantelli Lemma implies  that 
 \begin{equation}\label{eq;liminfeasy}
 \liminf_{n\to \infty, n\in \mathbb N}\mathcal A_K^n(w)\ge 1- \epsilon
 \end{equation}
 for almost every  $w\in I^m$. It follows from (\ref{eq;liminfeasy})  that  for almost every $w\in I^m$
 \[
 \liminf_{T\to \infty} \mathcal A_{K}^T(w)\ge 1- \epsilon . 
 \]
 Since we can take  $ \epsilon $  arbitrarily  close to zero, the 
 conclusion follows.
 \end{proof}

Let $H$ be the group generated by $\{g_t :t\in \mathbb R\} $ and $U$. It follows from 
Mozes \cite{m95} Theorem 1 that any $H$ invariant probability measure on $L/\Gamma$ is
$G$ invariant. 
A closed subset $Y$ of $L/\Gamma$ is said to be a finite volume homogeneous subspace 
if  a closed subgroup $L'$ of $L$ acts transitively on $Y$ and $L'$ preserves a probability 
measure $\mu_Y$ on $Y$. We say $Y$ is $G$ ergodic if $G$ acts ergodically on 
$(Y, \mu_Y)$.
Let $C_L(G)$ be the group of centralizers of $G$ in $L$.
It follows from  \cite{bq132} Proposition 2.1 that $G$ ergodic probability measures on $L/\Gamma$
 is at most a countable union of the set  
 \[
 C_L(G)\mu_Y:=\{g\mu_Y: g\in C_L(G)\}
 \]  where  $Y $ is a $G$ ergodic finite volume homogeneous subspace. 
 Without loss of generality we may assume that $\overline{Gx}=L/\Gamma$. 
We show that for almost every $w\in I^m$ any weak$^*$ limit $\nu_w$ 
 of 
$\frac{1}{T}\int_0^T g_tu(w)\delta_{x} dt$ as $T\to \infty $
 does not 
put any mass on 
$C_L(G)Y$ for any proper $G$ ergodic  finite volume homogeneous  subspace $Y$.
 This is  proved by  a similar quantitative result
for $\{g_t: 0\le t\le T\}$ trajectory of the measure associated to the  Lebesgue measure of $I^m$.
\begin{prop}\label{prop;singular}
Suppose that $Gx$ is dense in $L/\Gamma$. 
Let $Y$ be a proper $G$ ergodic finite volume homogeneous subspace. 
For any  compact subset  $F$ of $C_L(G)$ and any $ \epsilon_1 >0$, there exists a 
compact subset $K_1$ of $L/\Gamma$
with $K_1\cap FY=\emptyset$
 and a positive number  $c_1<1$ such that 
 \begin{equation*}
\left |\{w\in I^m: \mathcal A_{K_1}^T(w)\le 1- \epsilon_1   \}\right |\le  c_1^T
 \end{equation*}
 for every $T\ge 0$.
\end{prop}
\begin{cor}\label{cor;nobad}
Suppose that $Gx$ is dense in $L/\Gamma$.  
Let $Y$ be a proper $G$ ergodic finite volume homogeneous subspace. 
Then for almost every $w\in I^m$ one has $\nu_w(C_L(G)Y)=0$ for 
any
 weak$^*$ limit $\nu_w$ of $\frac{1}{T}\int_0^T g_tu(w)\delta_x\,dt$
 as $T\to \infty $.
\end{cor}
The proof  uses Proposition \ref{prop;singular} and  is the same as that of Corollary \ref{cor;nescape}, so we 
omit the details here. 

\begin{proof}[Proof of Theorem \ref{thm;proof}]
It follows from  Ratner's  orbit closure  theorem \cite{r912} that  $Gx$ 
is dense in a $G$ ergodic finite volume homogenous subspace of $L/\Gamma$.
So we can without loss of generality assume that 
$Gx$ is dense in $L/\Gamma$. 
  
It follows from Proposition \ref{prop;inv}, Corollary  \ref{cor;nescape}  and Corollary 
\ref{cor;nobad} that there exists a  subset  $J$ of $I^m$ with full measure such that  for
any $w\in J$,  any weak$^*$ limit $\nu_w$ of 
$\frac{1}{T}\int_0^T g_tu(w)\delta_xdt$ 
  as $T\to \infty$ has the following properties: 
\begin{itemize}
\item $\nu_w$ is invariant under $U$ and hence invariant under $G$;
\item $\nu_w$ is a probability measure;
\item $\nu_w(C_L(G)Y)=0$ for any proper $G$ ergodic  finite volume homogeneous subspace $Y$.
\end{itemize}
Therefore for any $w\in J$ we have (\ref{eq;goalpoint})
holds. This completes the proof.
\end{proof}

Here  we describe a general strategy of using Theorem \ref{thm;2} to prove 
pointwise equidistribution for other $g_1$ expanding subgroups not necessarily abelian.  In particular we derive 
Theorem \ref{thm;1} from it.  We need to use the following 
\begin{thm}[\cite{s96},\cite{sw96}]\label{thm;old}
Let $U'$ be a connected Ad-unipotent $ g_1 $ expanding subgroup of $G$. Suppose that $\mu$ is a
probability measure on $U'$ absolutely continuous with respect to the Haar measure. 
Let $\mu_ x$ be the push forward of $\mu$ to $L/\Gamma$
with respect to the map $u\in U'\to ux$.
Then 
\[
g_t \mu_ x\to \mu_{\overline{Gx}} \quad \mbox{ as } t \to \infty.
\] 
\end{thm}
This result is not explicitly stated in both of the papers but it is a simple consequence.
Let $H$ be the subgroup of $G$ generated by $\{g_t: t\in \mathbb R\}$ and $U'$.
It is easy to see that $g_1$ expanding property implies that the Zariski closure of $\rho(H)$ is 
an epimorphic subgroup of $\rho(G)$ for any finite dimensional real representation $\rho$ of 
$G$. Furthermore the ray $\{g_t: t>0\}$ is contained in the cone of \cite{sw96} Lemma 2.1 for any
nontrivial 
irreducible representation $\rho$. Therefore Theorem \ref{thm;old} follows from \cite{sw96} Theorem 1.4.

In case $U'$ is abelian and $g_1$ expanding
  we 
obtain in  Theorem \ref{thm;2} that the Haar measure of $U'$ is generic at $x\in L/\Gamma$. 
This seems to be true  when $U'$  is not assumed to be abelian. In view of  Theorem \ref{thm;old} 
it suffices to  show that for almost every $u\in U'$
\[
\lim_{T\to \infty}\frac{1}{T}\int_0^T g_tu \delta_{x} \,dt
\] 
exists in the space of probability measures on $L/\Gamma$. 
Theorem \ref{thm;2} will give this
almost everywhere existence 
 if there is an abelian subgroup  subgroup $U_a$ of $U'$ 
normalized by $g_t$ and a connected semisimple subgroup $G_1$ of $G$ without compact factors 
so that the following holds:
\begin{itemize}
\item[$(*)$]
$\{g_t: t\in \mathbb R\}$ is a subgroup of
$G_1$ and 
$U_a$ is a $ g_1 $ expanding subgroup of $G_1$.  
\end{itemize}

For example, if $G$ is the real rank one group $SU(2,1)$ then
the unstable horospherical  subgroup $U_G^+$ is not abelian. 
But we can find a subgroup with Lie algebra $\mathfrak{sl}_2(\mathbb R)$ containing the group 
$\{g_t: t\in \mathbb R\}$. More generally we have
\begin{lem}\label{lem;abelian}
Under the assumption of Theorem \ref{thm;1} there is a connected
 semisimple  subgroup $G_1\subset G$ without compact factors 
and an abelian subgroup $U_a$ of $U_G^+$ such that property $(*)$ holds.
\end{lem}
The proof of this lemma uses  strongly orthogonal system of simple 
root systems
and will be given in the appendix.  
Lemma \ref{lem;abelian} together with 
Theorem \ref{thm;2} and  Theorem \ref{thm;old}  proves
Theorem \ref{thm;1}.

  
  

 

\section{Some auxiliary results}
\subsection{Large deviation}
In this section we prove a large deviation result. 
Our argument is inspired by 
\cite{bq13} and \cite{a}.

Let $(W, \mathcal B, \mu)$ be a standard Borel  space with probability measure $\mu$. 
The  conditional expectation of 
a  nonnegative   Random variable 
$\tau$ (i.e.~a measurable  map $\tau: W\to [0, \infty]$)
with respect to   
  a sub sigma algebra
$\mathcal A$ of $\mathcal B$     is an $\mathcal A$ measurable function 
$E(\tau| \mathcal A)$  such that for any $A\in \mathcal A$ one has
$\int_A\tau(w)\, d\mu(w)=\int_AE(\tau| \mathcal A)(w)\, d\mu(w)$.
The conditional probability of $A\in \mathcal B$  is  the function
$
\mu(A|\mathcal A):=E(\mathbbm 1_A|\mathcal A)
$ 
where $ \mathbbm 1 _A$ is the characteristic function of $A$. 
For a nonnegative  random variable $\tau$ and $a\in \mathbb R$
we will follow the the convention of probability theory to write 
$\mu(\tau \ge a)$ for $\mu(\{w\in W: \tau(w)\ge a\})$ and $E(\tau)$
for $\int _W \tau (w)\, dw$

In this paper the set of natural numbers is  $\mathbb N=\{0, 1, 2, \ldots\}$.
A measurable map $\tau: W \to { \mathbb N}\cup \{\infty\}$ with $\tau(w)<\infty$ almost surely  is called  ${\mathbb N}$ valued random variable. 
A sequence of random variables $(\tau_i)_{i\in \mathbb N}$ is said to be  
increasing if $\tau_i(w)\ge \tau_{i-1}(w)$ for any $i\ge 1$ and almost every $w\in W$. 
A sequence of   sub sigma algebras  $(\mathcal A_i)_{i\in \mathbb N}$
of  $\mathcal B$ is said to 
be a filtration if 
$\mathcal A_{i-1}\subseteq \mathcal A_{i}$.
In the rest of  this section the relations $=$ and $\le $ for functions on $W$ are meant
to hold almost surely. 

\begin{lem}\label{lem;main}
Let $(\tau_i)_{i\in \mathbb N}$ be an  increasing sequence  of
${\mathbb N}$ valued  Random variables  on $W$.  Let $(\mathcal A_i)_{i\in \mathbb N}$ be a sequence of 
filtrations of sub sigma algebras of $\mathcal B$ such that $\tau_i$ is $\mathcal A_i$ measurable. 
Suppose that there exits $\vartheta_0>0$ and $Q_0 >0$ such that  
\begin{equation}\label{eq;basic exp}
\mu(\tau_{i}-\tau_{i-1}\ge  q|\mathcal A_{i-1})\le e^{-\vartheta_0 q}
\end{equation}
for every $q\ge Q_0$ and $i\ge 1$.
Then  for every $\epsilon>0$
there exists  $\vartheta>0$ such that for 
all sufficiently large $Q$
and  any positive integer  $n$  we have
\begin{equation}\label{eq;goaldev}
\mu\left( \frac{1}{n}\sum_{i=1}^n \mathbbm 1 _Q ( \tau_{i}(w)-\tau_{i-1}(w))\ge \epsilon\right)
\le e^{-\vartheta n}
\end{equation}
where 
$ \mathbbm 1 _Q: {\mathbb N}\to  {\mathbb N} $ is defined by \begin{equation}\label{eq;truncation}
 \mathbbm 1 _Q(q)=\left\{
\begin{array}{cl}
q & \mbox{ if } q\ge Q \\
0 & \mbox{otherwise.}
\end{array}
\right.
\end{equation}
\end{lem}
Remark:
It can be seen from the proof below that $Q$ and $\vartheta$ only depend on $Q_0,\epsilon$ and $\vartheta_0$ but not 
on 
the probability space, 
the the sequence of random variables or the filtration of sigma algebras. 

\begin{proof}
We will show that if $\vartheta=\epsilon\vartheta_0/4$ and $Q\ge Q_1$ where 
\begin{equation}\label{eq;choiceQ}
Q_1=\max\left\{
\frac{2\log[(e^{\epsilon\vartheta_0/4}-1)(1-e^{-\vartheta_0/2})]}
{-\vartheta_0}, Q_0\right\}
\end{equation}
then (\ref{eq;goaldev}) holds. 

For every positive integer  $n$ we define a function $f_n$ on
$W $ by
 \[
 f_n(w)=\exp\left (\frac{\vartheta_0}{2}{\sum_{i=1}^n ( \mathbbm 1 _Q  ( \tau_{i}(w)-\tau_{i-1}(w))} \right).
 \]
 By monotone convergence theorem for conditional expectations of nonnegative random variables 
 we have 
 \begin{eqnarray*}
E(f_n|\mathcal A_{n-1}) & \le &  f_{n-1}\left [1+
\mathcal \sum_{q\ge Q} e^{\vartheta_0 q/2}
\mu (\tau_n-\tau_{n-1}=q|\mathcal A_{n-1})
\right] \\
 & \le & f_{n-1}\frac{1-e^{-\vartheta_0/2}+e^{-Q\vartheta_0/2}}{1-e^{-\vartheta_0/2}}.
 \end{eqnarray*}
By induction on $n$ one gets
 \[
E(f_n)\le \left (\frac{1-e^{-\vartheta_0/2}+e^{-Q\vartheta_0/2}}{1-e^{-\vartheta_0/2}}
\right )^n.
 \]
On the other hand by Chebyshev inequality 
\[
E(f_n)\ge
 e^{\epsilon n\vartheta_0/2}\mu \left(\frac{1}{n}\sum_{i=1}^n \mathbbm 1 _Q  ( \tau_{n}-\tau_{n-1})\ge\epsilon
\right).
\]
Therefore 
\begin{equation}\label{eq;derivation}
\mu\left(\frac{1}{n}\sum_{i=1}^n \mathbbm 1 _Q  ( \tau_{n}-\tau_{n-1})\ge \epsilon
\right)
\le \left(\frac{1-e^{-\vartheta_0/2}+e^{-Q\vartheta_0/2}}
{e^{\epsilon \vartheta_0/2}(1-e^{-\vartheta_0/2})}
\right)^n.
\end{equation}
In view of  the   first  lower bound of $Q_1$ in (\ref{eq;choiceQ}) 
the conclusion follows. 
\end{proof}

\subsection{unipotent invariance}
The aim of this section is to prove Proposition \ref{prop;inv}. 
Our argument is modeled on  \cite{ce} \S 3. 
Let $L, \Gamma, G, g_t, U,x$ be   as  in Theorem \ref{thm;proof}.

We observe that there exists a countable
dense subset of $C_c(L/\Gamma)$  consisting of  smooth functions.
   Also if $s_1, s_2$ are linearly independent over 
$\mathbb Q$, then the closure of the group $<u(s_1\mathbf e_j), u(s_2\mathbf e_j): 1\le j\le m>$ is $U$. 
Therefore Proposition \ref{prop;inv} will follow if  we can show that for 
every  
 $\psi \in C_c^\infty(L/\Gamma), s>0$ and $1\le i\le m$
we have for almost every $w\in I^m$
\begin{equation}\label{eq: goal}
\frac{1}{T}\int_0^T \psi_t(w)\, dw\to 0\quad \mbox{as } T\to\infty
\end{equation}
where 
\[
\psi_t(w)  =  \psi[g_tu(w)x]-\psi[u(s\mathbf e_i)g_tu(w)x].
\]

We will prove (\ref{eq: goal}) using  law of large numbers. The key is the the 
following estimate: 
\begin{lem}\label{lem;decay}
There exists $\vartheta>0$ and $C>0$ such that for any $t,l>0$
\begin{equation}\label{eq: cross}
\int_{I^m}\psi_t(w)\psi_l(w)\,dw\le C e^{-\vartheta |l-t|}.
\end{equation}
\end{lem}

Lemma \ref{lem;decay} allows us to use the following lemma  to complete the proof of 
 (\ref{eq: goal}) and hence Proposition \ref{prop;inv}.

\begin{lem}[\cite{ce} Lemma 3.4]
Suppose that $\psi_t:I^m\to \mathbb R$ are bounded functions satisfying (\ref{eq: cross}) (for some $C>0$ and $\vartheta>0$).  Additionally, assume that $\psi_t(w)$ are Lipschitz functions of $t$ for each $w\in I^m$. Then
 (\ref{eq: goal}) holds
 for almost 
every $w\in I^m$.
\end{lem}

\begin{proof}[Proof of Lemma \ref{lem;decay}]

 We fix a right invariant Riemannian metric on $L$
and let $d(\cdot, \cdot)$ be the induced  distance function.
We note that the function  $\psi $ is Lipschitz, 
i.e.~$|\psi(gy)-\psi(hy)|\ll d(g, h) $ for any $g, h\in L$ and $y\in L/\Gamma$. 

Without loss of generality we assume that $l>t$ and $i=1$.
 Let 
 $b=b_1>0$ which is defined in the beginning of \S \ref{s;outline}, i.e. 
$
 {g_1}u(\mathbf e_1) g_1^{-1}=u(e^b\mathbf e_1).
$
Then 
\[
\psi_t(w)=\psi[g_tu(w)x]-\psi[g_tu(w+se^{-bt }\mathbf e_1)x].
\]
We will show that for $\vartheta=b/2$ there exists $C>0$ so that (\ref{eq: cross}) holds.

 We divide $[-1, 1]$ consecutively  into intervals of 
the form 
\[
I(r)=[r-e^{-(l+t)b/2}, r+e^{-(l+t)b/2}]
\]
except for the last part which will not affect the validity of (\ref{eq: cross})
since it has length  
less than $2e^{-(l+t)b/2}$.
For every $s_1\in \mathbb R$ with $|s_1|\le e^{-(l+t)b/2}$
we have
 \[
d(g_tu(s_1 \mathbf e_1),g_t)=d(u(e^{bt}s_1\mathbf e_1), \mathbf e)\ll e^{-(l-t)b/2}.
\]
As noted above that the function $\psi$ is Lipschitz,
so for every 
$s_1\in I(r)$ and 
$w\in \{0\}\times I^{m-1}$
 one has
\[
|\psi_t(s_1\mathbf e_1+w)-\psi_t(r\mathbf e_1+w)|\ll e^{-(l-t)b/2}.
\]
Therefore for any $w\in \{0\}\times I^{m-1}$
 \begin{eqnarray}\label{eq;inv1}
  & &\frac{1}{|I(r)|}\int_{I(r)} \psi_l(s_1\mathbf e_1+w)\psi_t(s_1\mathbf e_1+w)\, ds_1  \\
 & =&\frac{\psi_t(r\mathbf e_1+w)}{|I(r)|}
\int_{I(r)}\psi_l (s_1\mathbf e_1+w)\, ds_1 + O(e^{-(l-t)b/2}).\notag
\end{eqnarray}
Since the interval $I(r)$ and $I(r)+se^{-bl}$ have overlaps except for ends whose   length 
are $se^{-bl}$,
we have 
\begin{equation}\label{eq;inv2}
\frac{1}{|I(r)|}
\int_{I(r)}\psi_l (s_1\mathbf e_1+w)\, ds_1\ll 2se^{-(l-t)b/2}.
\end{equation}
We sum up the integral of $\psi_t\psi_l$ over a covering of $[-1,1]$ by consecutive intervals of the form $I(r)$, then 
(\ref{eq;inv1}), (\ref{eq;inv2}) and Fubini theorem with respect to $I\times I^{m-1}$ give  (\ref{eq: cross}).
\end{proof}

\subsection{Linear representations}
Let $ G, g_t, U$ be   as  in Theorem \ref{thm;proof}
and  $H$ be the subgroup  of $G$ generated by $U$ and $\{ g_t: t\in \mathbb R\}$.  The main result of this section is

 \begin{lem}\label{lem;estimate}
Let   normed  real vector spaced  $V$ be a finite dimensional representation of 
 $G$ without  nonzero  $G$ invariant vectors.  
 Then there exists 
 $\vartheta_0>0$ so that the following holds:
  for every $0<\vartheta<\vartheta_0$  and every $a>0$ there exists $T_0>0$ such that 
 if  $\tau: I^m\to \mathbb R_{\ge 0}$ is any measurable function  with
 $\inf_{w\in I^m} \tau(w)\ge T_0$ then
  \begin{equation}\label{eq;goal estimate}
\sup _{ \|v\|=1}\int_{I^m} \frac{dw}{\|g_{\tau(w)} u(w)v\|^\vartheta}\le  a
 \end{equation}
 where $\|\cdot \|$ is the norm on $V$. 
\end{lem}

We decompose   $V$ as
 $V^+\oplus V^0\oplus V^-$ according to the  eigenvalues
of $g_1$, i.e.~$V^+$ is the  sum of eigen spaces  of $g_1$
whose eigenvalues are greater than one, etc. 
Let $ \pi_+$ be the  projection from $V$ 
to $V^+$.
 For every $v\in V, r>0$
we set 
\[
D^+(v, r)=\{w\in I^m: \|\pi_+(u(w)v)\|\le r\}.
\]

\begin{lem}\label{lem;good}
Let $V$ be as in Lemma \ref{lem;estimate}.
Then there exists $0<\vartheta_0 \le 1$ such that
\begin{equation}\label{eq;uniform}
C:=\sup _{ \|v\|=1, r>0}\frac{|D^+(v,r)|}{r^{\vartheta_0}}<\infty.
\end{equation}
\end{lem}
\begin{proof}
Recall that   $U$ is  assumed to be $ g_1 $ expanding in $G$. 
When $v$ varies in the unit sphere of $V$ the family of  maps  which send
$w\in I^m\to\pi_+(u(w)v)$ are polynomials in $w$ with  degree uniformly bounded from above and
maximum of coefficients in absolute value uniformly bounded  from below
by some positive constant.  So
the lemma follows from the $(C, \alpha )$-good property of polynomial functions  proved in 
\cite{bkm01} Lemma 3.2.
\end{proof}

\begin{proof}[Proof of Lemma \ref{lem;estimate}]
Our proof basically follows that of  \cite{emm98} Lemma 5.1. 
We take  $\vartheta_0>0$ so that (\ref{eq;uniform}) holds. 
For fixed  $a>0$ and $0<\vartheta<\vartheta_0$  we need to find  $T_0$ 
so that (\ref{eq;goal estimate}) holds for any $\tau$ whose value on $I_m$
is bounded from below by $T_0$. 

First we need some preparation.
As $V$ is finite dimensional there exists $C_1>1$ such that for every vector $v_1\in V$ one has
$\|\pi_+(v_1)\| \le C_1\|v_1\|$. Let $b>0$ so that 
 $e^b$ is the smallest eigenvalue of $g_1$ in $V^+$.
Let $C$
 be the constant in (\ref{eq;uniform}) and let
 \[
 r=\sup_{\|v\|=1, w\in I^m} \|\pi_+(u(w)v)\|.
 \]
 We will show that  for $T_0>0$ which satisfies
 \begin{equation}\label{eq;esti aug1}
\frac{2CC_1r^{\vartheta_0-\vartheta}}{1-2^{\vartheta-\vartheta_0}}e^{-bT_0\vartheta} =a
 \end{equation}
the conclusion holds.

We fix a unit vector $v\in V$, function $\tau$ on $I^m$ with $\inf \tau \ge T_0$ and estimate the integral of 
\[
{f_{\tau,v}(w):=\|g_{\tau(w)}u(w)v\|^{-\vartheta}}.
\]  
Since 
  $
\|g_{\tau(w)}u(w)v\|\ge C_1^{-1} e^{bT_0}\| \pi_+[u(w)v]\|
$
one has   
\begin{equation}\label{eq;simplified}
f_{\tau,v}(w)\le C_1e^{-bT_0\vartheta}\| \pi_+[u(w)v]\|^{-\vartheta}
\end{equation}
 for every $w\in I^m$.
For every nonnegative integer $n$, 
 (\ref{eq;uniform}) and (\ref{eq;simplified}) imply  that
\begin{equation}\label{eq;sieve}
\int\limits_{D^+(v, r2^{-n})\setminus D^+(v, r2^{-n-1})}f_{\tau,v}(w)\, dw\le e^{-bT_0\vartheta}2CC_1r^{\vartheta_0-\vartheta}2^{-n(\vartheta_0-\vartheta)}.
\end{equation}
We write 
\[
I^m=D^+(v, 0)\bigcup\cup_{n\ge 0}[ D^+(v,2^{-n}r)\setminus D^+(v, 2^{-n-1}r)].
\]
Since $|D^+(v, 0)|=0$, we have
\begin{eqnarray*}
\int\limits_{I^m}f_{\tau,v}(w)\, dw  & = &\sum_{n=0}^\infty\int\limits_{D^+(v, r2^{-n})\setminus D^+(v, r2^{-n-1})}f_{\tau, v}(w)\, dw \\
\mbox{by (\ref{eq;sieve})}\qquad&\le & 
\frac{2CC_1r^{\vartheta_0-\vartheta}}{1-2^{\vartheta-\vartheta_0}}e^{-bT_0\vartheta} \\
\mbox{by (\ref{eq;esti aug1})}\qquad & = & a.
\end{eqnarray*}
\end{proof}

\section{Nonescape of mass}
The aim of this section is to prove Proposition \ref{prop;nescape}. 
Let $L, \Gamma, G, g_t, U,x$ be  as  in Theorem \ref{thm;proof} and let  $X=L/\Gamma$.
The main tool is the contraction property of a function $\alpha$ (we call it height function)
on $X$ which measures whether  points in  $X$  
 are close to  $\infty$. 
The height function with the contraction property  on homogeneous space  
is introduced by 
Eskin, Margulis and Mozes \cite{emm98}.
A significant   improvement  is given by Benoist and
 Quint \cite{bq12} which  will be used in this paper. 

\subsection{Existence of height function}

\begin{lem}\label{lem;contract}
Given a compact subset  $Z$ of $ X$ and a positive number $a<1$, for 
$t $ sufficiently large (depending on $a$)
 there exists a lower semicontinuous   function  $\alpha: X\to [0, \infty]$  and $b>0$ with the following properties:
\begin{enumerate}
\item
For every  every $y\in X$
\begin{equation}\label{eq;linear}
\int _{I^m}\alpha(g_t u(w)y)\, dw \le a \alpha (y)+b
\end{equation}
where $dw$ is the usual Lebesgue measure;

\item $\alpha $ is finite  on $ G Z$;

\item $\alpha$ is Lipschitz, i.e.~for every compact subset $F_0$
of $G$ there exists $C>0$ such that $\alpha(gy)\le C \alpha(y)$ for every $y\in X$ and $g\in F_0$; 

\item $\alpha$ is proper, i.e.~if $\alpha(Z_0)$ is bounded for some subset $Z_0$ of $X$ then
$Z_0$ is relatively compact.  
\end{enumerate}
\end{lem}
Remark: Here lower semicontinuity implies that for every positive number $M$ the subset  $\alpha^{-1}([0, M])$ is  closed and  hence compact by (4).

We first deal  with the case where  $\Gamma$ is arithmetic.
For the moment we assume that   $L=SL_d(\mathbb R)$
and $\Gamma =SL_d(\mathbb Z)$ where  $d\ge 2$.  It is well known that the space 
$X=SL_d(\mathbb R)/SL_d(\mathbb Z)$ can be identified with the set
of unimodular lattices in $\mathbb R^d$. For every $y\in X$, let $\Lambda_y$ be
the lattice in $\mathbb R^d$ corresponding to it, i.e.~$\Lambda_y=g\mathbb Z^d$ 
if $y=gSL_d(\mathbb Z)$.
A vector 
\[
v\in \wedge^* \mathbb R^d
:=\oplus_{0\le i\le d} \wedge^i \mathbb R^d
\]
 is monomial if $v=v_1\wedge \cdots \wedge v_i$ 
where $v_1, \ldots, v_i\in \mathbb R^d$. We say $v$ is $y$-integral monomial   if
we can take $v_i\in \Lambda_y$.

We review the height function defined in \cite{bq12}.  Since $G$ is a 
connected  semisimple Lie group contained in $L=SL_d(\mathbb R)$, 
it is the connected component of  real points of some real algebraic group.  
We fix a maximal connected diagonalizable  subgroup 
$A$ of $G$ containing $\{g_t: t\in \mathbb R\}$. Let $\Phi(G, A) $ be the 
 relative root system, i.e.~ the set of 
nonzero  weights   of $A$
appeared in the adjoint representation.
We fix a positive system $\Phi(G, A) ^+$ such that $\lambda (g_1)\ge 1$ for 
every $\lambda\in \Phi(G, A) ^+$.  
We endow a partial order on the set $P$ of algebraic characters  of  $A$ by 
$
\lambda \le \mu$  
if and only if 
$ \mu-\lambda$ 
 is  nonnegative linear combination  of  $\Phi(G, A) ^+$.
 For any   irreducible finite dimensional real 
representation of $G$,
the set of weights of $A$ in this representation 
 has a unique maximal element called
 highest weight of the representation. 
Let $P^+$ be the set of all the highest weights appearing in $\wedge^*\mathbb R^d$.

 For each $\lambda\in P^+$, let $q_\lambda$ be the projection 
from $\wedge^* \mathbb R^d$ to the subspace consisting of all the irreducible sub representations with
highest weight $\lambda$. 
Let  $\|\cdot\|$ be the usual Euclidean norm on $\wedge^*\mathbb R^d $.
One of the key ingredients of \cite{bq12} is the following Mother Inequality:
\begin{lem}[\cite{bq12} Proposition 3.1]\label{lem;mo}
There exists $C_1>0$ such that for any monomials $u, v, w$ in $\wedge ^*\mathbb R^d$
one has the inequality 
\[
\|q_\lambda(u)\|\cdot \|q_\mu(u\wedge v\wedge w)\|\le C_1 \max_{\nu, \rho\in P^+ \atop 
\nu+\rho\ge \lambda +\mu} \|q_\nu(u\wedge v)\|\cdot
\|q_\rho(u\wedge w)\|.
\]
\end{lem}

We fix the following index:
\[
\delta_i=(d-i)i \quad\mbox{and}\quad \delta_\lambda= \log( \lambda( g_1))
\]
where $0\le i\le d$ and $\lambda\in P^+\setminus 0$ where $0$ is the trivial  character of $A$. Recall that
$U$ is $ g_1 $ expanding, so for
 $\lambda \in P^+\setminus 0$  we have
$\delta_\lambda>0$.
 Also we take
\begin{equation}\label{eq;kappa}
\kappa=(\min_{\lambda\in P^+\setminus 0} \delta_\lambda)^{-1}
\quad\mbox{and}\quad \kappa_1=(\max_{\lambda\in P^+\setminus 0} \delta_\lambda)^{-1}.
\end{equation}

Let $\varepsilon>0$ and $0<i< d$. 
Following \cite{bq12}
for every  
$v\in \wedge^i \mathbb R^d$  we let
\[
\varphi_{ \varepsilon }(v)=\left\{
\begin{array}{ll}
\min _{\lambda\in P^+\setminus 0}  \varepsilon ^{\frac{\delta _i}
{\delta _\lambda}}\|q_\lambda(v)\|^{\frac{-1}{\delta _\lambda}} & \mbox{ if }
\|q_0(v)\|<\varepsilon ^{\delta _i}\\
0 & \mbox{otherwise.}
\end{array}
\right.
\]
We remark here 
that  $\varphi_{ \varepsilon }(v)=\infty $
 if
 $v=q_0(v)$ and $\|v\|<  \varepsilon ^{\delta _i}$. 

\begin{lem}\label{lem;restate}
There exits $\vartheta_1>0$ such that 
for every $\vartheta $ with $0<\vartheta< \vartheta_1$ and $0<a<1$
the following holds: for $t$ sufficiently large and 
 for every  $v\in \wedge ^i\mathbb R^d$ with $0<i<d$ one has
\begin{equation}\label{eq;swim}
\int_{I^m} \varphi_{ \varepsilon }^\vartheta(g_tu(w)v)\, dw\le a\varphi_{ \varepsilon }^\vartheta
(v) \quad \mbox{for any } 0<\varepsilon <1.
\end{equation}
\end{lem}

\begin{proof}
Let $V$ be the   subspace  of $\wedge^* \mathbb R^d$ complementary to the subspace of $G$
invariant 
vectors.
For the representation $G$ on $V$ we fix $\vartheta_0$ given by the conclusion of Lemma \ref{lem;estimate}
and take $\vartheta_1 = \vartheta_0/\kappa$.

There are two trivial cases: 
 if either 
 $\|q_0(v)\|\ge \varepsilon^{\delta _i}$ or    $q_0(v)=v$ and $\|v\|< \varepsilon^{\delta _i} $,
  then both sides of (\ref{eq;swim}) are either $0$ or  $\infty$ respectively.
In general if $v\neq q_0(v)$ and $\|q_0(v)|< \varepsilon ^{\delta_i}$, then
the conclusion  follows from Lemma \ref{lem;estimate}
and the fact that the integral  of the  minimum of finite functions is less than or equal to the minimum of integrals. 
\end{proof}

Following \cite{bq12} we define  $\alpha_{\varepsilon }: X\to [0, \infty]$ by 
\[
\alpha_{ \varepsilon }(y)=\max\varphi_{\varepsilon }(v)
\]
where the maximum is taken over all the non-zero $y$-integral monomials $v\in \wedge ^i\mathbb R^d$
with $0<i<d$. 

\begin{lem}\label{lem;inequality}
Given  $\vartheta>0$ sufficiently small and  $0< a<1$, for every $t$ sufficiently large (depending on $\vartheta $ and $a$)
and 
 $\varepsilon >0$ sufficiently small (depending on $t$) there exists 
$b>0$ such that 
\begin{equation}\label{eq;swim5}
\int_{I^m} \alpha_{ \varepsilon }^{\vartheta}(g_tu(w)y) \,dw\le  a\alpha_{ \varepsilon }^\vartheta(y)+b
\end{equation}
for every $y\in X$.
\end{lem}
\begin{proof}
We fix $t>0$ sufficiently large so that   according to 
 Lemma \ref{lem;restate}  
one has
\[
\int_{I^m}\varphi^\vartheta_{\varepsilon}(g_tu(w)v)\, dw\le \frac{a}{2d}\varphi^\vartheta_{\varepsilon}(v)
\]
for every $0<\varepsilon<1$ and  $v\in \wedge^i\mathbb R^d$  with $0<i<d$.

Let $C_0=\sup\{\|g_tu(w)\|+\|(g_tu(w))^{-1}\|: w\in I^m\}\ge 1$
where $\|\cdot\|$ is the operator norm for elements of $G$ acting on $\wedge^*\mathbb R^d$.
We take $\varepsilon $ small enough so that 
\[
C_0^{2\kappa}(C_1\varepsilon)^{\kappa_1/2} <1
\]
where $C_1$ is the constant given in Lemma \ref{lem;mo} and $\kappa,\kappa_1$
are defined in (\ref{eq;kappa}).
Let 
\[
b_{1}=\sup \varphi_{\varepsilon }(v)<\infty
\]
where the supremum is taken over all the monomials $v\in\wedge^* \mathbb R^d$
with $\|v\|\ge 1$. We will show that for 
\[
b=2^m(C_0^{\kappa}\max\{b_1,C_0^{2\kappa}\})^\vartheta
\]
(\ref{eq;swim5}) holds.

 It follows from the definition of $C_0$ that  for every monomial
 $v\in \wedge^i\mathbb R^d$ with $0< i <d$ one has 
\[
C_0^{-\kappa}\varphi_{\varepsilon}(v)\le \varphi_{\varepsilon}(g_tu(w)v)\le C_0^{\kappa} \varphi_{\varepsilon}(v).
\]
If $\alpha_{ \varepsilon }(y)\le \max\{b_1, C_0^{2\kappa}\}$, then
\[
\int_{I^m} \alpha_{ \varepsilon }^\vartheta(g_tu(w)y)\, dw\le b.
\]

Let $\Psi$ be the finite set of primitive $y$-integral and monomial elements $v$ 
of $\wedge^*\mathbb R^d$ with degree in $(0,d)$ such that 
\[
\varphi_{ \varepsilon }(v)\ge C_0^{-2\kappa} \alpha_{\varepsilon}(y).
\]
It follows from
claim (5.9) 
 in the  proof of \cite{bq12} Proposition 5.9  that
 if  $\alpha_{ \varepsilon }(y)>\max\{b_1, C_0^{2\kappa}\}$ 
 then 
 $\Psi$ contains at most one element up to sign change in each degree $i$. 
Therefore
in this case one has
\[
\int_{I^m} \alpha_{ \varepsilon }^\vartheta(g_tu(w)y) \,dw\le \sum _{v\in \Psi}
\int_{I^m} \varphi_{ \varepsilon }^\vartheta(g_tu(w)v)\, dw\le 
\frac{a}{2d}\sum _{v\in \Psi} \varphi_{ \varepsilon }^\vartheta(v)\le a  \alpha_{ \varepsilon }^\vartheta(y). 
\]
\end{proof}

We fix $\vartheta $ and $\varepsilon$ sufficiently small so that 
$\alpha_\varepsilon^\vartheta$
is finite on $Z$ and  Lemma \ref{lem;inequality} holds. 
It is easy to see  that $\alpha=\alpha_\varepsilon^\vartheta $ satisfies properties (1)-(4) of Lemma \ref{lem;contract}. 
Therefore   we have proved Lemma \ref{lem;contract} in the case where
$L=SL_d(\mathbb R)$ and $\Gamma=SL_d(\mathbb Z)$. The general case will 
be reduced to this case 
and the real rank one case. 
We need the following 
lemma which is straightforward to check  so we omit the details of proof. 

\begin{lem}\label{lem;straight}
Let $\Gamma_1$ be a lattice of a connected Lie group $L_1$. Let $\varphi: L\to L_1$ be a surjective
homomorphism of Lie groups so that $\varphi(G)$ is nontrivial. Suppose that $\varphi(\Gamma)\subset 
\Gamma_1$ and the induced map $X=L/\Gamma\to L_1/\Gamma_1$ is proper. If Lemma \ref{lem;contract}
holds for  $L_1/\Gamma_1, \varphi(g_t), \varphi(U)$ or it holds for 
$L/\Gamma', g_t, U$ where $\Gamma'$ is a finite index subgroup of $\Gamma$, then it holds for $X, g_t, U$. 
\end{lem}

\begin{proof}[Proof of Lemma \ref{lem;contract}]
Let $\mathfrak {r}$ be the largest amenable ideal of the Lie algebra $\mathfrak l$  of $L$, $\mathfrak s:=\mathfrak l/\mathfrak r$,
$S:=Aut(\mathfrak s)$.
Let $R$ be the kernel of the  adjoint representation $Ad_{\mathfrak s}: L\to S$.
 It follows from  \cite{bq12} Lemma 6.1  that 
$\Gamma\cap R$ is a cocompact lattice  in $R$ and the image group $\Gamma_S:=Ad_{\mathfrak s}(\Gamma)$
is a lattice in $S$. Therefore the map
$
L/\Gamma\to S/\Gamma_S
$
is proper. So according to Lemma \ref{lem;straight}  it suffices to prove the case where $L$ is a connected semisimple center free Lie group without compact 
factors. 

Under this assumption we can write $L=\prod_{i=1}^qL_i$ as a direct product  of connected semisimple
Lie groups such that $L_i\cap \Gamma$ is an irreducible lattice in $L_i$.
We can 
 assume that $\Gamma=\prod_{i=1}^q L_i\cap \Gamma$ since the latter 
 has finite index in $\Gamma$. 
 Let $\pi_i: L\to L_i$ be the natural quotient map. 
If $\pi_i(G)$ is nontrivial then  $\pi_i(g_t)$ is a nontrivial  Ad-diagonalizable  one parameter subgroup 
of $\pi_i(G)$
and $\pi_i(U)$ is  $\pi_i(g_1)$ expanding.
Suppose that  Lemma \ref{lem;contract} holds for every $L_i/\Gamma_i$ with $\pi_i(G)$ nontrivial. 
Let $\alpha_i: L_i/\Gamma_i\to [0, \infty]$ be a lower semicontinuous
function   associated to the compact subset $\pi_i(Z)\subset L_i/\Gamma_i$,   
$0<a<1$ and $t>0$. If $\pi_i(G)$ is trivial, we set $\alpha_i=(1-\mathbbm 1_{\pi_i(Z)})\cdot \infty$.
Then the function $\alpha$ on $X$ define by
\[
\alpha(y_1, \ldots, y_q)= \alpha_1(y_1)+\cdots+\alpha_q(y_q)\quad
\mbox{where} \quad y_i\in L_i/\Gamma_i
\]
satisfy properties (1)-(4) of Lemma \ref{lem;contract} with respect to $Z, a$ and $t$.
Therefore it suffices  to prove  the case where $L$ is a connected  center free
semisimple Lie group without compact factors  and $\Gamma$ is an irreducible lattice. 

 If the real rank of $L$ is bigger than or equal to two, then Margulis arithmeticity theorem (see e.g. \cite{z} Theorem 6.1.2)
implies that there is an injective map \[
\varphi: L\to SL_d(\mathbb R)
\]
such that $\varphi(\Gamma)$ is commensurable with $\varphi(L)\cap SL_d(\mathbb Z)$.
So Lemma \ref{lem;contract} follows from Lemma \ref{lem;straight} and the case
where $L=SL_d(\mathbb R)$  and $\Gamma=SL_d(\mathbb Z)$.

Otherwise $L$ has real rank one.
If $X=L/\Gamma$ is compact, then we take $\alpha(y)=1$ for any $y\in X$.
Suppose that  $X$ is noncompact. 
 It follows from \cite{gr70} (cf.~\cite{kw} Proposition 3.1
and \cite{bq12} page 54) and the proof of  \cite{em04} Proposition 2.7 that there 
exists a finite dimensional real representation $V$ of $G$ with norm $\|\cdot\|$ and finite nonzero 
vectors  $v_1, \ldots, v_r$ of $V$
with the following properties: 

\begin{enumerate}[label=(\alph*)]
\item $\Gamma v_i$ is closed and hence discrete in $V$ for $1\le i\le r$;
\item For any $F\subset L$, the set $F\Gamma\subset L/\Gamma$ is relatively compact if and only if 
there exists $c>0$ such that $\|g\gamma v_i\|>c$ for any $\gamma\in \Gamma, g\in F$
 and $1\le i\le r$;
\item There exists $c_0>0$ such that for 
any $g\in L$ there exists at most one $v\in \bigcup_{1\le i \le r} \Gamma v_i$  such that $\|gv\|<c_0$;
\item There exists $C'>0$ such that for every $1\le i\le r$ and every $g\in L$ one has
 $||\pi (gv_i)\|\ge C'\|gv_i\|$ where $\pi$ is the projection to the subspace  complementary  to the subspace of
  $G$
 invariant vectors.
\end{enumerate}
Let
\[
\tilde \alpha_\vartheta(g\Gamma)=\max_{1\le i\le r}\max_{\gamma\in \Gamma}\|g\gamma v_i\|^{-\vartheta}.
\]
In this case
Lemma 
\ref{lem;contract}
follows from  properties (a)-(d) listed above and 
Lemma \ref{lem;estimate} by taking $\alpha=\tilde \alpha_\vartheta$ for some  $\vartheta $ sufficiently small.
\end{proof}

\subsection{Exponential recurrence to cusp}\label{s;rec}
For $Z=\{x\}$ and $a=\frac{1}{4}$ we  choose $t>0$, $b>0$ and $\alpha:X\to [0, \infty]$ so that
Lemma \ref{lem;contract} holds. 
We first  use inequality (\ref{eq;linear}) to study discrete  trajectory
\begin{equation}\label{eq;distrajectory}
\{g_{nt}u(w)x:  n\in \mathbb N\}
\end{equation}
where $\mathbb N=\{0, 1, 2, \ldots\}$.
Recall that $\{\mathbf e_i: 1\le i\le m\}$ is the standard basis of $\mathbb R^m$ and $b_i>0$ satisfies 
(\ref{eq;Recall}). Let 
\[
w=\sum_{i=1}^m a_i\mathbf e_i\quad \mbox{and }\quad w'=\sum_{i=1}^m a'_i \mathbf e_i.
\]
If $|a_i-a'_i|\le  2e^{-ntb_i}$, then two points
 $g_{nt}u(w)x$ and $g_{nt}u(w')x$
 can always be translated to each other by elements in a fix compact subset of $G$.
In view of property (3) of $\alpha$ we consider them as at the same height. 
The following lemma plays a key role to link random walks with respect to $g_tu(I^m)$ and
 trajectory  (\ref{eq;distrajectory}).

\begin{lem}[Shadowing Lemma]\label{lem;shade}
For $1\le i\le m$ let $J_i\subset [-1, 1]$ be an interval with length  $ |J_i|\ge e^{-ntb_i}$.
Then for any nonnegative measurable  function $\psi$ on $X$ and $J=\prod_{i=1}^m J_i$ one has
\begin{equation}
\int_J \psi(g_{(n+1)t}u(w)x)\, dw \le \int_J\int_{I^m} \psi(g_{t}u(w_1)g_{nt}u(w)x) \, dw_1dw.
\end{equation}
\end{lem}
\noindent
 The proof is  an elementary exercise of calculus using  change of  variables 
\[
(w_1, w)=( (  s'_i),( s_i))\to ((s_i'),(s_i+s_i'e^{-ntb_i})).
\]
Here abelian assumption is essential to us. If we drop the abelian  assumption, then we need
to change the domain of the integral for $w_1$ to something that depends on $J$. In that case it is 
not clear to the author how to get (\ref{eq;goal estimate})
uniformly in terms of $T_0$ for various domains determined by $J$
 and hence the contraction property
(\ref{eq;linear}).

For every positive integer $n$
we need to  divide the interval $[-1,1]$ into intervals of size $e^{-ntb_i}$ for each component 
of $I^m$ to form a box so that the above shadowing lemma holds and $g_{nt}u(w)x$
are bounded for $w$ in each box.
We can do this consecutively except for the last interval  which we allow
 to have length bigger than $e^{-ntb_i}$ but 
no more than $2 e^{-ntb_i}$. 
We want  the
 partition for   $n+1$  to be a refinement of that  for  $n$ so 
we do this by induction on $n$. 
The first step we divide $I^m$ into boxes of the form
\[
\prod_{1\le i\le m}[-1 +je^{-tb_i}, -1+(j+1)e^{-tb_i})
\]
with slight modifications for the end  intervals. 
For every $w\in I^m$ we use $I_1(w)$ to denote  the box containing $w$. 
In the second step we  divide each box above into  smaller boxes of the form
\[
\prod_{1\le i\le m}[-1 +je^{-tb_i}+ke^{-2tb_i}, -1+je^{-tb_i}+(k+1)e^{-2tb_i})
\]
and we use
 $I_2(w)$  to denote the one containing $w$. 
By the same construction we do it for all $n$ and define $I_n(w)$ accordingly.
We also take $I_0(w)=I^m$ and $I_\infty(w)=\{w\}$ for every $w\in I^m$.
We fix a compact subset $F_0$ of $G$
 so that for any $n\in \mathbb N$,   $w\in I^m$ and $w'\in I_n(w)$ one has
\begin{equation}\label{eq;compactG}
g_{nt}u(w')x=hg_{nt}u(w)x\quad \mbox{for some}\quad h\in F_0.
\end{equation}

For every $n\in \mathbb N$ let $\mathcal B_n$ be the smallest sigma algebra of $I^m$ such that $I_j(w)\in
\mathcal B_n$ for every  $0\le j\le n$ and  $w\in I^m$. 
It is not hard to see that the atom of $w$ in $\mathcal B_n $ is $I_n(w)$
and the sequence $(\mathcal B_n)_{ n\in \mathbb N}$ is a filtration of sigma algebras. 

\begin{lem}\label{lem;basic ineq}
For every  $J\in \mathcal B_n$ where $n\in \mathbb N$  one has
\[
\int_{J}\alpha(g_{{(n+1)}t}u(w)x)\, dw\le \frac{1}{4}\int_{J}\alpha(g_{nt}u(w)x)\, dw +b|J|.
\]
\end{lem}
\begin{proof}
It follows from  shadowing Lemma \ref{lem;shade} and  the linear inequality (\ref{eq;linear}).
\end{proof}

Let us fix a positive real number $l_0$ with the following properties:
\begin{enumerate}[label=(\roman*)]
\item   $b/l_0<1/4$; 
 \item $x\in X_{l_0}$ where $X_{l_0}=\{y\in X: \alpha(y)\le l_0\}$;
\end{enumerate}
We 
 define a sequence of measurable  functions $\sigma_i: I^m\to \mathbb N\cup\{\infty\}$ which represents 
$i$th 
return time to the compact subset $X_{l_0}$.
To begin with we set $\sigma_0(w)=0$.
To apply shadowing lemma we  want $\{w\in I^m: \sigma_i(w)=n\}$ to be 
$\mathcal B_n$ measurable. 
 The formal definition 
is
\begin{equation}\label{eq;return time}
\sigma_i(w)=\inf\{ n>\sigma_{i-1}(w): g_{nt}u(w_1)x\in X_{l_0} \mbox{ for some } w_1\in I_n(w)\}.
\end{equation}
If $\sigma_i(w)=\infty $ for some $i$ then we set $\sigma_j(w)=\infty$ for every $j>i$.
It follows from the definition that $\{w\in I^m: \sigma_i(w)> n\}$ is $\mathcal B_n$ measurable.  
To simplify notation we set
\[
I(\sigma_n,w):=I_{\sigma_n(w)}(w).
\]

\begin{lem}\label{lem;pre expon}
There exists   $Q_0>0$ and $\vartheta_0>0$ such that for any integer $q\ge Q_0, n\in \mathbb N$ and $w_0\in I^m$
with
$\sigma_n(w_0)<\infty$
the measure of the  set 
\begin{equation}\label{eq;expjump}
J_{n,q}(w_0)=\{w\in I(\sigma_n, w_0): \sigma_{n+1}(w)-\sigma_n(w)\ge q\}
\end{equation}
is less than or equal to 
$ e^{-\vartheta_0 q}|I(\sigma_n, w_0)|$.
\end{lem}
Remark: It follows from Lemma \ref{lem;pre expon} that  $\sigma_n(w)<\infty$ almost surely for every $n\in \mathbb N$. 
\begin{proof}
We fix $w_0$, $n$ and write  $\sigma_n=\sigma_n(w_0), J_q={J_{n,q}}(w_0)$ for simplicity.
Let 
\[
\quad s_q:=\int_{J_{q+1}}\alpha(g_{(\sigma_n+q)t}u(w)x)\, dw
\le  \int_{J_{q}}\alpha(g_{(\sigma_n+q)t}u(w)x)\, dw.
\]
Since $J_{q}$ is $\mathcal B_{\sigma_n+q-1}$ measurable,  Lemma \ref{lem;basic ineq}  implies
\[
s_q\le \frac{1}{4}s_{q-1}+b|J_{q}|\le \left(\frac{1}{4}+\frac{b}{l_0}\right)s_{q-1}\le \frac{1}{2}s_{q-1}.
\]
A simple induction on $q$ implies  that 
$
s_q\le s_0 2^{-q}.
$
  Chebyshev inequality  
  and  property (3) of $\alpha$ in Lemma \ref{lem;contract}
  implies  that there exists $C_1>0$ not depending on $w_0$ or $n$ such that  $|J_{q+1}|\le 2^{-q }C_1 
  |I(\sigma_n, w_0)|$. 
  Hence  the existence of $Q_0$ and $\vartheta_0$ follows.
\end{proof}

Recall that the proportion of the trajectory $\{g_tu(w)x: 0\le t\le T\}$ in a subset $K$ of $X$ 
is defined in (\ref{eq;average}).  Similarly, a discrete version of this function  is 
defined as
\begin{equation}\label{eq;daverage}
\mathcal D_{K}^n(w):= \frac{1}{n}\sum_{i=0}^{n-1} \mathbbm 1_{K}(g_{it}u(w)x)
\end{equation}
where $n$ is a positive integer and $\mathbbm 1_K$ is the characteristic function of $K$.

\begin{lem}\label{lem;discrete}
For every $0<\epsilon_0<1$ 
there exists a compact subset $K_0$ of $X$ and $0<c_0<1$ so that
\begin{equation}\label{eq;gotoUK}
\left| \left\{
w\in I^m:\mathcal D_{K_0}^n(w)\le 1-\epsilon_0\}
\right\}
\right|\le c_0^n
\end{equation}
for every positive integer $n$.
\end{lem}
\begin{proof}
We choose $l_0>0$ so that  properties ({\romannumeral {1}}) and ({\romannumeral {2}}) listed after Lemma \ref{lem;basic ineq} hold.
It follows from Lemma \ref{lem;pre expon} that there exists a positive integer  $Q_0$
such that for every $q\ge Q_0$ we get the exponential decay for the measure of  the set $J_{n, q}(w)$.
It follows from Lemma \ref{lem;main} that there exists $Q>0, 0<c_0<1$ and integer $N_0\ge 1$ such that  for $n\ge N_0$ the measure of the  set 
\[
J_n=\left\{w\in I^m:\frac{1}{n} \sum_{i=1}^n  \mathbbm 1 _Q (\sigma_i(w)-\sigma_{i-1}(w))\ge \epsilon_0
\right \}
\]
is less than or equal to $c_0^n$. Here  $ \mathbbm 1 _Q$ is the truncation of the  identity function    defined in (\ref{eq;truncation}).

We claim that the lemma holds for 
\[ K_0= \bigcup_{0\le s\le (Q+N_0)t}g_sF_0X_{l_0}.
\]  
To see this we  note  that $x\in X_{l_0}$.  Therefore    if $g_{jt}u(w)x\not \in K_0$ for some $j\in \mathbb N $,
 then $\sigma_i(w)-\sigma_{i-1}(w)\ge Q$
for some $i\le  j$.
 So if $w\in I^m$ satisfies  $\mathcal D_{K_0}^n(w)\le 1-\epsilon_0$ then $w\in J_n$.
It is easy to see that for  $n<N_0$ we have $\mathcal D_{X_l}^n(w)=1$ for any $w\in I^m$.
Therefore the conclusion follows from the exponential decay of the measure $|J_n|$.
\end{proof}

The following lemma allows us to deduce the continuous version of exponential decay from 
the discrete version in Lemma \ref{lem;discrete}. 

\begin{lem}\label{lem;continuousd}
Let $0< \epsilon_0,  c_0<1$  and let ${K_0}$ be a compact subset of $X$. Suppose that 
(\ref{eq;gotoUK}) holds for every positive integer $n$.
Then there exists a positive number  $ c<1$ and a compact subset $ K$ of $X$ such that 
\begin{equation}\label{eq;booktrain}
\left |\{w\in I^m: \mathcal A_{ K}^T(w)\le 1- 2\epsilon_0   \}\right |\le   c^T.
\end{equation}
\end{lem}
\begin{proof}
We fix $T_0>0$ so that for $T\ge T_0$ we have 
\begin{equation}\label{eq;easyread}
\left(1-\frac{t}{T}\right)\left(1-{\epsilon_0}\right)\ge 1-2\epsilon_0 \quad
\mbox{and}\quad T\ge  2t.
\end{equation}
Let $K'$ be a compact subset of $X$ so that 
 for any $w\in I^m$ and $0\le s\le T_0$ we have $g_{s}u(w)x\in K'$. 
 We claim that the compact subset 
    $ K=\left (\bigcup_{s\in [0,t]}g_sK_0\right) \cup K'$ and $ c=c_0^{\frac{1}{2t}}$ satisfies (\ref{eq;booktrain}).
 
 To prove  the claim it suffices to consider the case $T\ge T_0$. 
 Given $i\in \mathbb N$,
 if
 $g_{it}u(w)x\in  K_0 $ then $g_su(w)x\in K$ for $s\in [it, (i+1)t]$. Therefore  in view of
 the first inequality of  (\ref{eq;easyread}) we have if
  $\mathcal A_{ K}^T(w)\le 1-2\epsilon_0$ then $\mathcal D_{K_0}^{\lfloor T/t\rfloor}(w)\le 1-{\epsilon_0}$ 
 where ${\lfloor T/t\rfloor} $ is the biggest  integer less  than or equal to $T/t$.
 A simple calculation using 
 second inequality of (\ref{eq;easyread}) gives (\ref{eq;booktrain}).
\end{proof}

\begin{proof}[Proof of Proposition \ref{prop;nescape}]
It follows from  Lemma \ref{lem;discrete} and Lemma  \ref{lem;continuousd}.
\end{proof}

\section{Exponential recurrence to  singular subspace}
The aim of this section is to prove Proposition \ref{prop;singular}. 
Let $L, \Gamma, G, g_t, U,x$ be   as  in Theorem \ref{thm;proof}, let $Y, K_1 , C_L(G), F,\epsilon_1$ be
 as in Proposition \ref{prop;singular} and let $X=L/\Gamma$.
Let $S=\{g\in L: gY=Y\}$ and  let
 $\mathfrak s, \mathfrak c,\mathfrak l, \mathfrak g$ be the Lie algebras of $S,C_L(G), L$ and  $G$, respectively.
Let $\mathfrak t$ be  a $ G $ invariant subspace of $\mathfrak l$ complementary to 
$\mathfrak s+\mathfrak c$ with respect to the adjoint action.

We first set up some constants and review some results from previous section.
For every $w\in I^m$ and $n\in \mathbb N$ we let $I_n(w)$ to be the box defined in \S \ref{s;rec}.
Let $F_0$ be a compact subset of $G$ so that (\ref{eq;compactG}) holds.
 For  $a=1/4, Z=\{x\}$ we fix $t>0$ 
 so that  there exists $\alpha: X\to [0,\infty]$ and $b>0$ satisfying  Lemma \ref{lem;contract}. 
We fix $l_0>0$ so that properties ({\romannumeral {1}}) and ({\romannumeral {2}}) listed after Lemma \ref{lem;basic ineq} hold.
We let    $\sigma_i: I^m\to {\mathbb N}\cup \{\infty\}$
be the $i$th return time to $X_{l_0}$  defined in (\ref{eq;return time}).
By  Lemma \ref{lem;pre expon}
 there exists  $Q_0$ and $\vartheta_0>0$ such that
 for $q\ge Q_0$
  the measure of $J_{n,q}(w)$
 defined in (\ref{eq;expjump}) is less than or equal to $e^{-\vartheta_0q}|I(\sigma_n,w)|$. 

We fix a norm $\|\cdot\|$ on $\mathfrak g$ and use 
$\|g\|$ to denote the   operator norm of $g\in G$ with respect to
the adjoint representation. 
There exits $\vartheta'>0$ such that
\begin{equation}\label{eq;expgrow}
 \max{( \|g_{nt}u(w)\|, \|(g_{nt}u(w))^{-1}\|)}\le e^{n\vartheta'}
\end{equation}
for every $w\in I^m$ and positive integer  $n$. 
Let $0<\vartheta<1$ be sufficiently small so that $\frac{\vartheta_0}{2}-\vartheta'\vartheta>0$. 
According to Lemma \ref{lem;estimate}, by 
 making $\vartheta$ smaller we can find a positive integer $p$
 such that for any measurable map $\tau: I^m \to
{ \mathbb N}\cup\{\infty\}$ with $\inf_{w\in I^m} \tau(w) \ge pt$ and $\tau(w)<\infty$ almost surely 
\begin{equation}\label{eq;lcon}
\sup_{\|v\|=1}\int_{I^m}\frac{dw}{\|g_{\tau(w)}u(w)v\|^{\vartheta}}<\frac{1}{4}.
\end{equation}
To use contraction property (\ref{eq;lcon}) we need to modify $i$th return function $\sigma_i$ and define 
inductively 
\[
\kappa_0(w)=0\quad \mbox{and}\quad  \kappa_i(w)=\min\{\sigma_n(w): 
\sigma_n(w)\ge \kappa_{i-1}(w)+p\}.
\]
To simplify notation we set
\[
I(\kappa_n, w):= I_{\kappa_n(w)}(w).
\]
It follows from Lemma \ref{lem;pre expon} that for
any $n\in \mathbb N, w_0\in I^m$ and 
 $q\ge Q_0+2p$ we have
\begin{equation}\label{eq;lhard}
|\{w\in I(\kappa_n, w_0): \kappa_{n+1}(w)-\kappa_n(w)\ge q\}
|\le e^{-q\vartheta_0/2}|I(\kappa_n, w_0)|.
\end{equation}

Following  \cite{bq13} \S 6.8 we will define 
a height function $\beta:X\to [0, \infty]$ 
which roughly speaking measures wether 
elements of a fixed compact subset  are close to $FY$. 
Let $N\supset F_0$ be a relatively   compact open  neighborhood of identity  in $G$.
 We choose a positive number $\varepsilon\le 1 $, an open neighborhood $O$ of identity in $C_L(G)$
 and finite number of elements 
$f_1, \ldots, f_k\in F$ with $F\subset Of_1\cup \cdots \cup O f_k$
  so that the following holds:  for any $y\in NX_{l_0}:=\bigcup_{h\in N}hX_{l_0}$ and any $f_i$ 
  there exists at most one $v\in \mathfrak t$ with $\|v\|\le \varepsilon$ and $y\in \exp (v)Of_iY$. 
By shrinking $O$ we also assume that $x\not \in \overline O F Y$
where $\overline O$ is the closure of $O$.
For any $y\in NX_{l_0}$ and $1\le i\le k$, we set
\[
\beta_i(y)=\left\{\begin{array}{cl}
\|v\|^{-\vartheta} & \mbox{if }y\in \exp(v)Of_i Y \mbox{ with } v \in \mathfrak t \mbox{ and } \|v\|\le\varepsilon \\
\varepsilon ^{-\vartheta} & \mbox{otherwise} 
\end{array}
\right .
\]
and $\beta(y)=\beta_1(y)+\cdots+\beta_k(y)$.
We also set  $\beta(y)=0$ if $y\not\in NX_{l_0}$.
It is easy to see that $\beta$ satisfies the following properties:
\begin{enumerate}[label=(\Roman*)]
\item $\beta$ is lower semicontinous;
\item $\beta $ is Lipschitz on $NX_{l_0}$, i.e.~for every 
compact subset $F_2$ of $G$, there exists $C_2>1$ such that $\beta(gy)\le C_2\beta(y)$
for any $y\in NX_{l_0}$;
\item $\beta(y)=\infty$ if and only if $y\in NX_{l_0}\cap (\cup Of_i)Y$.

\end{enumerate}

Our strategy is in principle the same as that of  previous section. The key ingredient is 
Lemma \ref{lem;new eq} which is a varaint of Lemma \ref{lem;basic ineq}. The following
lemma is 
a preparation for the proof of Lemma \ref{lem;new eq}.

\begin{lem}\label{lem;lcontract}
For every $y\in NX_{l_0}$ and bounded measurable function $r: I^m\to \mathbb Z_{\ge p}$,
 there exists $b'>0$ depending on 
the upper bound of the function $r$ such that
\[
\int_{I^m}\beta(g_{r(w)t} u(w)y)\, dw\le  \frac{1}{4} \beta (y)+b'.
\]
\end{lem}
\begin{proof}
Since  the function $r$ is bounded, there exists $C\ge 1$ such  that 
$\max\{\|g_{r(w)t}u(w)\|,\|(g_{r(w)t}u(w))^{-1}\| \}\le C$ for every $w\in I^m$. 
Let $$J_i=\{w\in I^m: \beta_i (g_{r(w)t}u(w)y)\ge (C\varepsilon^{-1} )^{\vartheta} \}$$ 
and $J'_i=I^m \setminus J_i$.
If $w\in J_i$, then $y=\exp({v_i}) Of_i Y$ with $\|v_i\|=\beta_i (y)$ and
 $\beta_i (g_{r(w)t}u(w)y)=\|g_{r(w)t}u(w)v_i\|$. Therefore  according to (\ref{eq;lcon})
\[
\int_{J_i}\beta_i(g_{r(w)t} u(w)y)\,dw\le  \frac{1}{4}\beta _i(y).
\]
The lemma follows by taking 
$b'=2^mk(C\varepsilon^{-1} )^{\vartheta}$.
\end{proof}

\begin{lem}\label{lem;new eq}
There exists  $b''>0$ such that  for  any  $n\in \mathbb N$, $w_0\in I^m$ with $\kappa_n(w_0)<\infty$
 and $J=I(\kappa_n,w_0)$ one has 
\begin{equation}\label{eq;goalsingular}
\int_{J}\beta (g_{\kappa_{n+1}(w)t}u(w)x)\,dw\le \frac{1}{3} \int_{J}
\beta(g_{\kappa_n(w)t}u(w)x)\,dw +b''|J|.
\end{equation}
\end{lem}
Remark: Let $\mathcal C_n$ be the smallest sigma algebra of $I^m$ generated by   $I(\kappa_i, w)$  for
  $0\le i\le n$ and $w\in I^m$ with $\kappa_i(w)<\infty$.
Since modulo null sets every element  of $\mathcal C_n$ is  countable or finite disjoint  union of sets of the form $I(\kappa_n, w)$, the lemma
also holds for $J\in \mathcal C_n$.

\begin{proof}
Since the function $\kappa_n(w)$ is fixed on $J$ we simply write $\kappa_n$ for $\kappa_n(w)$.
Here  $\kappa_{n+1}(w)-\kappa_n$ varies for different $w$ and might be unbounded, so we can not
 use the  idea of shadowing Lemma \ref{lem;shade} 
directly. To overcome this difficulty we fix a positive integer $Q\ge  p$ which will be specified 
afterwards  and 
 define a truncation  for the function $\kappa_{n+1}(w)-\kappa_n$ by 
\[
 r(w)=
\left\{
\begin{array}{cl}
\kappa_{n+1}(w)-\kappa_n & \mbox{if } w\in J \mbox{ and } \kappa_{n+1}(w)-\kappa_n<  Q \\
Q & \mbox{otherwise.}
\end{array}
\right.
\]

It follows from Lipschitz property  of $\beta$ that there exists
$C_2\ge 1$ such that  
\begin{equation}\label{eq;proof1}
\beta(g_{\kappa_nt}u(w_0)x)/C_2\le \beta(g_{\kappa_nt}u(w)x)\le C_2 \beta(g_{\kappa_nt}u(w_0)x) 
\end{equation}
 for any 
$w\in J$.
 We take $Q $ to be the smallest integer greater than or equal 
to 
\begin{equation}\label{eq;upperQ1}
\max\left\{Q_0+2p, \frac{\log (12C_2^2)-\log(1-e^{\vartheta_0/2-\vartheta'\vartheta}) }{\vartheta_0/2-\vartheta'\vartheta}\right\}.
\end{equation}
Let $b'>0$ be the constant given by Lemma \ref{lem;lcontract} with respect
 to the truncated   function $r$.  We will show that (\ref{eq;goalsingular}) holds
 for $b''=b'$.
 Note that $C_2$ and hence $b'$ does not depend on $n$ or $w_0$.
 
 
We divide $J$ into two sets: 
 \[
J_1=\{w\in J:  r(w)
<  Q\}\quad 
\mbox{and}\quad J_2=J\setminus J_1=\{w\in J:  r(w)= Q\}.
\quad
\]
Let $w_n'=(e^{-\kappa_ntb_1}, \ldots, e^{-\kappa_ntb_m})$ where $b_i>0$
satisfies (\ref{eq;Recall}) and let  $w_1\cdot w_n'$ 
be the usual inner product on $\mathbb R^m$. We have
\begin{eqnarray*}
 \int_{J_1}\beta (g_{\kappa_{n+1}(w)t}u(w)x)\,dw     &  \le & \int_{J}\beta (g_{r(w)t+\kappa_nt}u(w)x)\,dw   
\\
 \empty& \le &
\int_{J}\int_{I^m}\beta(g_{ r( w+w_1 \cdot w_n')t}u(w_1)g_{\kappa_nt}u(w)x)\,dw_1dw \\
\text{by Lemma \ref{lem;lcontract}\hspace{1.5cm}}&  \le &   \frac{1}{4}\int_{J}
\beta(g_{\kappa_nt}u(w)x)\,dw+b'|J|.
\end{eqnarray*}

Let $B_q=\{w\in J: \kappa_{n+1}(w)-\kappa_n=q\}$.
In view of  (\ref{eq;expgrow}) we have
\begin{eqnarray*}
\int_{J_2}\beta (g_{\kappa_{n+1}(w)}u(w)x)\,dw & \le &  
\sum _{q\ge  Q}\int_{B_q}e^{q\vartheta'\vartheta} 
\beta(g_{\kappa_{n}t}u(w)x)\, dw \\
\mbox{by (\ref{eq;proof1})}\qquad& \le &  
\sum _{q\ge  Q}\int_{B_q}e^{q\vartheta'\vartheta} C_2
\beta(g_{\kappa_{n}t}u(w_0)x) \,dw \\
\mbox{by (\ref{eq;lhard})}\qquad
&\le &\sum_{q\ge Q} e^{-q(\vartheta_0/2-\vartheta'\vartheta)}C_2\int_J\beta(g_{\kappa_nt}u(w_0)x)\,dw \\
\mbox{by (\ref{eq;proof1})}\qquad
&\le & \frac{ e^{-Q(\vartheta_0/2-\vartheta'\vartheta)}} {1-e^{\vartheta_0/2-\vartheta'\vartheta}}
C_2^2\int_J\beta(g_{\kappa_nt}u(w)x)\,dw.
\end{eqnarray*}
The second lower bound for   $Q$ in (\ref{eq;upperQ1})  implies that 
\[
\frac{ e^{-Q(\vartheta_0/2-\vartheta'\vartheta)}} {1-e^{\vartheta_0/2-\vartheta'\vartheta}}C_2^2\le \frac{1}{12}
\]
which  completes the proof.
\end{proof}

We fix a positive number $l$ with $\beta(x)<l$ and  $b''/l<1/12$.
 Let  
\begin{equation}\label{eq;similarY}
X^Y_{l}=\{y\in \overline{N} X_{l_0}: \beta(y)\le l\}.
\end{equation}
The  $i$th return time to $X_{l}^Y$  is  the function $\tau_i:I^m\to { \mathbb N}\cup \{\infty\}$ defined
inductively  as follows:
  $\tau_0(w)=0$ and 
 \begin{multline*}
\tau_i(w)=\inf\{n: n>\tau_{i-1}(w) , \kappa_n(w)<\infty
\mbox{ and }  \\
  g_{\kappa_n(w)t}u(w_1)x\in X_{l}^Y   
   \mbox{ for some } w_1\in I(\kappa_n, w)\}.
\end{multline*}
We make it convention that $\tau_i(w)=\infty$ if the set where 
we take  infimum  is empty. 
It will also be convenient to set $\kappa_{\infty}(w)=\infty$. 
We  set \[
 \kappa_{\tau_n(w)}:=\kappa_{\tau_n(w)}(w)
 \quad \mbox{and}\quad 
 I(\kappa_{\tau_n}, w):=I_{\kappa_{\tau_n(w)}}(w)
\]
to simplify the notation.
The following two lemmas are preparations for the proof of Lemma \ref{lem;combine}
which is similar to Lemma \ref{lem;pre expon}.

\begin{lem}\label{lem;new exp}
There exists   $Q_1>0$ and $\vartheta_1>0$ such that for any $q\ge Q_1, n\in \mathbb N$ and $w_0\in I^m$
with $\tau_n(w_0)<\infty$ the measure of the  set 
\[
B_{n,q}(w_0):=\{w\in I(\kappa_{\tau_n}, w_0): \tau_{n+1}(w)-\tau_n(w)\ge q\}
\]
is less than or equal to 
$ e^{-\vartheta_1 q}|I(\kappa_{\tau_n},w_0)|$.
\end{lem}
Remark: It follows from Lemma \ref{lem;new exp} that for almost every $w\in I^m$
we have $\tau_n(w)<\infty$ for any $n\in \mathbb N$.
\begin{proof}
We fix $n, w_0$ and set  $B_q =B_{n, q}(w_0), i=\tau_n(w_0)$. It is easy to see that
\[
s_q:=\int_{B_{q+1}}\beta(g_{\kappa_{i+q}(w)t}u(w)x)\,dw\le \int_{B_{q}}\beta(g_{\kappa_{i+q}(w)t}u(w)x)\,dw.
\] 
Note that $B_{q}\in\mathcal C_{i+q-1}$ where $\mathcal C_{i+q-1}$ is defined 
in the remark of Lemma \ref{lem;new eq}. So Lemma \ref{lem;new eq} implies that
 \begin{equation}
s_q  \le  \frac{1}{3}\int_{B_{q}}\beta(g_{\kappa_{i+q-1}(w)t}u(w)x)\,dw+b''|B_{q}|\le \frac{1}{2}s_{q-1}.
\end{equation}
The rest of proof is the same as that of Lemma \ref{lem;pre expon}.
\end{proof}

\begin{lem}\label{lem;morelem}
There exists $Q_2>0$  and $\vartheta_2>0$ such that for any inegers
$i\ge 0, j>0 $  and  any  $w_0\in I^m$ with $\kappa_i(w_0)<\infty$
the measure of the set 
\begin{equation}\label{eq;indexplane}
C_{i, j}(w_0):=\{w\in I(\kappa_i,w_0): \kappa_{i+j}(w)-\kappa_i(w)\ge Q_2j\}
\end{equation}
is less than or equal to $e^{-\vartheta_2 j}|I(\kappa_i,w_0)|$.
\end{lem}
\begin{proof}
It follows from 
(\ref{eq;lhard}),
 Lemma \ref{lem;main} and its remark that there exists $\vartheta>0$ and $Q\ge Q_0+2p$ such that
 the measure of the set 
 \[
C_{i,j}' (w_0):= \left \{w\in I(\kappa_i,w_0): \sum_{s=1}^j\mathbbm{1}_Q
 (\kappa_{i+s}(w)-\kappa_{i+s-1}(w))\ge j\right\} 
 \]
is less than or equal to $e^{-\vartheta j}|I(\kappa_i,w_0)|$.
Suppose that $\kappa_{i+j}(w)-\kappa_i(w)\ge 2Qj$ for some 
$w\in I(\kappa_i,w_0)$, then $w\in C_{i,j}' (w_0)$.
Therefore the Lemma follows 
by taking $Q_2=2Q$ and $\vartheta_2=\vartheta$.
\end{proof}

\begin{lem}\label{lem;combine}
There exists   $Q_3>0$,  $\vartheta_3>0 $  such that for any $n\in \mathbb N, q\ge Q_3$ and $w_0\in I^m$ with $\tau_n(w_0)<\infty$
the measure of the  set 
\begin{equation}\label{eq;really need}
A_{n,q}(w_0):=\{w\in I(\kappa_{\tau_n},w_0): \kappa_{\tau_{n+1}(w)}-\kappa_{\tau_n(w)}\ge q\}
\end{equation}
 is less than or equal to 
$ e^{-\vartheta_3 q}|I(\kappa_{\tau_n},w_0)|$.
\end{lem}
\begin{proof}
We fix positive numbers  $Q_1,\vartheta_1$ and $Q_2, \vartheta_2$ so that Lemma \ref{lem;new exp}
and Lemma \ref{lem;morelem} hold respectively. 
We show that for 
\[
\vartheta_3=\frac{1}{2Q_2}\min 
\left \{{\vartheta_1}, {\vartheta_2}\right \}
\quad \mbox{and} \quad Q_3=\max\left\{(Q_1+2)Q_2 , \frac{\log 2}{\vartheta_3}
\right\}
\]
the conclusion holds. 

We fix $n, w_0$ and $q\ge Q_3$. Let  ${\lfloor q/Q_2\rfloor}$  is the biggest  integer less  than or equal to $q/Q_2$
 and let 
\[
A'=\left\{w\in A_{n,q}(w_0):\tau_{n+1}(w)-\tau_n(w)\ge\lfloor {q}/{Q_2} 
\rfloor\right\}.
\]
It follows from Lemma \ref{lem;new exp} that
\[
 |A'|\le e^ { {-\vartheta_1}{\lfloor q/Q_2\rfloor} }| I(\kappa_{\tau_n},w_0)|. 
 \]
 On the other hand 
 it is easy to see that  $ A_{n,q}(w_0)\setminus A'\subset C_{\tau_n(w_0), {\lfloor q/Q_2\rfloor} }$
 where the latter is defined in (\ref{eq;indexplane}).
 So Lemma \ref{lem;morelem} implies that 
 \[
 |A_{n,q}(w_0)\setminus A'|\le e^ {{-\vartheta_2}{\lfloor q/Q_2 \rfloor}}| I(\kappa_{\tau_n},w_0)|.
 \]
A simple calculation shows that  (\ref{eq;really need}) holds.
\end{proof}

\begin{proof}[Proof of Proposition \ref{prop;singular}]
It follows from Lemma \ref{lem;combine}  and Lemma \ref{lem;main} that there exists positive number
 $c_0<1$, positive  integers $Q$ and $ M_0$ such that for every integer $n\ge M_0$
the measure of the set 
\[
J_n=
\left \{w\in I^m: \frac{1}{n} \sum_{i=1}^n: \mathbbm{1}_Q
(\kappa_{\tau_i(w)}-\kappa_{\tau_{i-1}(w)})\le 1-\frac{\epsilon_1}{2}
\right \}
\]
is less than or equal to $c_0^n$. 
Therefore by taking 
\[
K_0=\bigcup_{0\le s\le (Q+M_0)t}g_s\overline N X_l^Y,
\]
the same proof as  that of Lemma \ref{lem;discrete} shows that 
\begin{equation}\label{lem;finaleq}
\left| \left\{
w\in I^m:\mathcal D_{K_0}^n(w)\le 1-\epsilon_1/2\}
\right\}
\right|\le c_0^n
\end{equation}
for any positive integer $n$.
The conclusion follows from 
 (\ref{lem;finaleq}) and Lemma \ref{lem;continuousd}.
\end{proof}

\appendix
\section{}
The aim of this section is to prove  Lemma \ref{lem;abelian}. 
Let $G, g_t, U_G^+$ be as  in  Theorem \ref{thm;1}.
We first give a characterization of $g_1$ expanding subgroup.
\begin{lem}\label{lem;characterize}
Let $U$ be a connected  Ad-unipotent subgroup of $G$ normalized 
by $\{g_t:t\in \mathbb R\}$.
 Suppose that for any nonzero $U$ stabilized  vector $v$ in  a  
 nontrivial 
irreducible finite dimensional representation of $G$
  the projection 
 $\pi_+(v)$
 to the expanding subspace of $g_1$ is nonzero, 
 then $U$ is a $g_1$ expanding subgroup of $G$.
 \end{lem}
 Remark: It is easy to see that  the converse of Lemma \ref{lem;characterize} also holds. 
 \begin{proof}
 We fix a nonzero vector $v$ in  a  nontrivial 
 finite dimensional irreducible  representation $V$ of $G$. 
 Let $W=\{v_1\in V: uv_1=v _1\mbox{ for any } u\in U\}$. It is easy
 to see that $W$ is invariant under the one parameter subgroup $\{g_t: t\in \mathbb R\}$. 
 According to the assumption we have that $W\subset V^+$. 
 Let $W'$ be a $\{g_t: t\in \mathbb R\}$ invariant   complement of $W$ in $V$
 and let $\pi_W: V\to W$ be the projection to $W$ with respect to $W'$. 
 It follows from \cite{s96} Lemma 5.1 that $\pi_W(Uv)$ is not identically zero. 
 Therefore $\pi_+(Uv)$ is not identically zero. So 
 $U$ is a $g_1$ expanding subgroup of $G$ according to the definition.
 \end{proof}

The key ingredient of the proof of Lemma \ref{lem;abelian} is the following 
result about abstract root systems.

\begin{lem}\label{lem;root}
Let $\Delta$ be an irreducible abstract root system  and let
 $E=\mathrm{span}_{\mathbb R}\Delta $.
 Suppose that $E$  has dimension $n$ and 
 $(\cdot, \cdot)$ is the inner product of $E$ invariant under the Weyl group of $\Delta$.    Let $\Delta^+\subset \Delta$ be a positive system dominated by 
some $\alpha\in E$, i.e.~$(\alpha, \beta)\ge 0 $ for any $\beta\in \Delta^+$. Then there exists a basis  $\beta_1, \ldots, \beta_n\in \Delta^+$ of $E$   such that 
\begin{equation}\label{eq;linearly}
\alpha=c_1\beta_1+\cdots +c_n\beta_n
\end{equation}
where  $c_i\ge 0$ and $\beta_i+\beta_j\not\in \Delta^+$ for any $i,j$.
\end{lem}

\begin{proof}
According to the classification of irreducible abstract  root systems, see e.g.~\cite{knapp} Chapter 2,
it suffices to consider the case where  $\Delta $ is reduced. 
 Let $\Pi=\{\alpha_1, \ldots, \alpha_n\}$ be simple roots 
determined by  $\Delta^+$  and let $A$ be the associated  Cartan matrix.

 Recall that two roots $\beta$ and $\gamma$ are said to be  strongly
orthogonal if $( \beta, \gamma)=0$ and 
a subset $\mathcal O$ of $\Delta^+$ is called strongly orthogonal system if elements of $\mathcal O$
are pairwise strongly orthogonal. 
It follows from Oh \cite{oh98} that
if $\Delta$ is of  type 
 $B, C,E_7, E_8, F_4, G_2, D_n (n \mbox{ is even})$ then
 there is a strongly orthogonal system $\mathcal O$  consisting of $n$ elements.
 In these cases
$\alpha$ is a linear combination
 of elements in $\mathcal O$ satisfying the conclusion  of the lemma. 

We will prove the rest cases one by one. 
Let $\|\cdot\|$ be the induced norm on $E$.   We assume without loss of generality that 
 $\|\alpha_i\|=1$ if $\Delta$ is of type $A_n, D_n$ or $ E_6$.
It follows from  Lusztig and Tits \cite{lt92} that $A^{-1}$ has positive rational entries. So  we have
\[
\alpha=a_1\alpha_1+\cdots+a_n\alpha_n
\]
where $a_i\in \mathbb R_{>0}$.

{\bf Case}   $\mathbf {A_n}$.
\[ 
\xy 
(-15,0)*{\cir<0pt>{}}; (0,0)*{\cir<4pt>{}}; **\dir{--};
(0,0)*{\cir<4pt>{}}; (15,0)*{\cir<4pt>{}}; **\dir{-};
(15,0)*{\cir<4pt>{}}; (45,0)*{\cir<4pt>{}}; **\dir{--};
(45,0)*{\cir<4pt>{}}; (60,0)*{\cir<4pt>{}}; **\dir{-};
(0,-6)*{\alpha_1};
(15, -6)*{\alpha_{2}};
(45, -6)*{\alpha_{k-1}};
(60, -6)*{\alpha_{k}};
\endxy 
\]
We assume that our simple roots are ordered so that  $a_1\ge a_2$
and the corresponding Dynkin  diagram is as above. 
  Since $\alpha$
is dominated we have
\[
(\alpha, \alpha_2)=a_2-\frac{1}{2}a_1-\frac{1}{2}a_3\ge 0
\]
which implies  $a_2\ge a_3$. Hence a simple induction implies that  $a_i\ge a_{i+1}$ for $i\le k-1$. 
Therefore
 $\Pi$ can be rearranged so that  $a_i\ge a_{i+1}$  and $\alpha_{i+1}$ is connected with one
 of $\{\alpha_1, \ldots, \alpha_i\}$ in the Dynkin diagram.  
So  if we take  $\beta_i=\alpha_1+\cdots+\alpha_i\in \Delta ^+$ the conclusion of the lemma follows.

 {\bf Case $\mathbf {D_n}$ where $\mathbf n$ is odd}. 
In this case the strongly orthogonal $\mathcal O$ constructed in \cite{oh98} contains 
$n-1$ elements and the highest root is not in $\mathcal O$. 
If we take there elements as $\beta_1,\ldots, \beta_n$, then
they satisfy $\beta_i+\beta_j\not \in \Delta$ but it is not clear to 
the author how to prove $c_i\ge 0$.

 
\[ 
\xy 
(0,0)*{\cir<4pt>{}}; (15,0)*{\cir<4pt>{}}; **\dir{-};
(15,0)*{\cir<4pt>{}}; (45,0)*{\cir<4pt>{}}; **\dir{--};
(45,0)*{\cir<4pt>{}}; (60,0)*{\cir<4pt>{}}; **\dir{-};
(70,10)*{\cir<4pt>{}}; (60,0)*{\cir<4pt>{}}; **\dir{-};
(70,-10)*{\cir<4pt>{}}; (60,0)*{\cir<4pt>{}}; **\dir{-};
(0,-6)*{\alpha_1};
(15, -6)*{\alpha_{2}};
(45, -6)*{\alpha_{n-3}};
(60, -6)*{\alpha_{n-2}};
(80, -10)*{\alpha_{n-1}};
(80, 10)*{\alpha_n};
\endxy 
\]
We  assume that $\Pi$ is ordered so that its Dynkin diagram is as above. 
There is 
a complete list of $\Delta^+ $ in terms of $\Pi$ with $E=\mathbb R^n$  given in 
Knapp  \cite{knapp} Appendix C
as follows:
$\alpha_i=\mathbf e_i- \mathbf e_{i+1}$ for $i<n$ and $\alpha_n=\mathbf e_{n-1}+\mathbf e_n$; 
$\Delta^+=\{ \mathbf  e_i\pm \mathbf  e_j: i<j\}$ where $\{\mathbf e_1, \ldots, \mathbf  e_n \}$ is
the standard basis of $\mathbb R^n$.

Since $\alpha $ is dominated we have 
 \[
2(\alpha, \alpha_1)=2a_{1}-a_{2}\ge 0.
\]
Assume that $(i+1)a_i-ia_{i+1}\ge 0$ for $i\le n-4$. Then
\[
2(i+1)(\alpha, \alpha_{i+1})+(i+1)a_i-ia_{i+1}=(i+2)a_{i+1}-(i+1)a_{i+2}\ge 0.
\]
Therefore  we have
\begin{equation}\label{eq;mosquito}
(i+1)a_i-ia_{i+1}\ge 0 \quad \mbox{for } 1\le i\le n-3.
\end{equation}
 By calculating  inner products of $\alpha$ with $\alpha_{n-1}$ and $\alpha_n$ we have
 \begin{equation}\label{eq;coefficient}
 \left\{
\begin{array}{lcl}
2a_{n} & \ge & {a_{n-2}} \\
2a_{n-1} & \ge &{a_{n-2}} .
\end{array}
\right.
\end{equation}
It follows form (\ref{eq;coefficient})  and $(\alpha, \alpha_{n-2})\ge 0$ that
\begin{equation*}
a_{n-3}\le a_{n-2}.
\end{equation*}
Using $(\alpha, \alpha_{n-2-i})\ge 0$ for $1\le i\le n-4$ it can be shown inductively that
\begin{equation}\label{eq;coefficient1}
a_{n-2-i}\ge a_{n-3-i}.
\end{equation}

For $1\le i\le  (n-3)/2$  we take 
 \[\left\{
\begin{array}{lcl}
\beta_{2i-1} &=& \alpha_{2i-1}\\
\beta_{2i} &=& \alpha_{2i-1}+2(\alpha_{2i}+\cdots+\alpha_{n-2})+\alpha_{n-1}+\alpha_n.
\end{array}
\right.
\]
It follows from (\ref{eq;mosquito}), (\ref{eq;coefficient}) and (\ref{eq;coefficient1})  that there are nonnegative  integers $c_1, \ldots, c_{n-3}, b_{n-2}, b_{n-1}, b_n$ such that \[
\alpha-c_1\alpha_1-\cdots-c_{n-3}\alpha_{n-3}=b_{n-2}\alpha_{n-2}+b_{n-1}\alpha_{n-1}+b_n\alpha_n.
\]

If $b_{n-2}\ge b_{n-1}\ge b_n$ we take
$\beta_{n-2} =  \alpha_{n-2}+\alpha_{n-1}+\alpha_n$,
 $\beta_{n-1}=\alpha_{n-2}+\alpha_{n-1}$ and $\beta_n=\alpha_{n-2}$.
It is easy to see the existence of nonnegative numbers $c_{n-2} ,c_{n-1}, c_n$.
The fact that 
$\beta_i+\beta_j\not \in \Delta^+$ follows from  the list of $\Delta^+$
and 
 \[
 \left\{
\begin{array}{lcll}
\beta_{2i-1} &=&  \mathbf e_{2i-1}- \mathbf e_{2i} &  \mbox{for }1\le i\le  (n-3)/2\\
\beta_{2i} &=&  \mathbf e_{2i-1}+ \mathbf e_{2i}               & \mbox{for }1\le i\le  (n-3)/2\\
\beta_{n-2}& = &  \mathbf e_{n-2}+ \mathbf e_{n-1} &  \\
\beta_{n-1}& = &  \mathbf e_{n-2}- \mathbf e_{n}  & \\
\beta_{n}& = &  \mathbf e_{n-2}- \mathbf e_{n-1}.  & 
\end{array}
\right.
\]

The rest cases can be proved similarly by taking 
$\beta_{n-2} =  \alpha_{n-2}+\alpha_{n-1}+\alpha_n$, so we only list the choices
 of $\beta_{n-1}$ and $\beta_n$.
If $b_{n-2}\ge b_{n}> b_{n-1}$ we take $\beta_{n-1}=\alpha_{n-2}+\alpha_{n}$ and $\beta_n=\alpha_{n-2}$;
if $b_{n-1}> b_{n-2}\ge b_n$ we take $\beta_{n-1}=\alpha_{n-2}+\alpha_{n-1}$ and $\beta_n=\alpha_{n-1}$;
if $b_{n}> b_{n-2}\ge b_{n-1}$ we  take $\beta_{n-1}=\alpha_{n-2}+\alpha_{n}$ and $\beta_n=\alpha_{n}$;
if $ b_{n}>b_{n-2}$ and
$ b_{n-1} >b_{n-2}$ we take $\beta_{n-1}=\alpha_{n-1}$ and $\beta_n=\alpha_{n}$.

 {\bf  Case $\mathbf {E_6}$}.

\[ 
\xy 
(30,0)*{\cir<4pt>{}}; (45,0)*{\cir<4pt>{}}; **\dir{-};
(45,0)*{\cir<4pt>{}}; (60,0)*{\cir<4pt>{}}; **\dir{-};
(60,0)*{\cir<4pt>{}}; (75,0)*{\cir<4pt>{}}; **\dir{-};
(75,0)*{\cir<4pt>{}}; (90,0)*{\cir<4pt>{}}; **\dir{-};
(60,0)*{\cir<4pt>{}}; (60,15)*{\cir<4pt>{}}; **\dir{-};
(30, -6)*{\alpha_6};
(45, -6)*{\alpha_5};
(60, -6)*{\alpha_4};
(75, -6)*{\alpha_3};
(90, -6)*{\alpha_1};
(60,21)*{\alpha_2};
\endxy 
\]
We assume that $\Pi$ is ordered so that its Dynkin diagram is as above. 
 A simple calculation using  $(\alpha, \alpha_i)\ge 0$ implies
\[\left\{
\begin{array}{rcl}
2a_1 &\ge  &a_3 \\
2a_6 & \ge & a_5  \\
3a_3 & \ge & 2a_4\\
3a_5 & \ge & 2a_4 \\
2a_4 & \ge & 3a_2.
\end{array}
\right.
\]
Let  $\beta_1 = \alpha_1+2\alpha_2+2\alpha_3+3\alpha_4+2\alpha_5+\alpha_6$ 
to be the highest root. 
Then it follows from the above calculation that
there exist nonnegative numbers $b_1, b_3, b_4, b_5, b_6$ such that 
\[
\alpha-\frac{a_2}{2}\beta_1=b_1\alpha_1+b_3\alpha_3+\cdots+b_6\alpha_6.
\]
We take $\beta_i$ to be the linear combination of $\alpha_1, \alpha_3, \alpha_4, \alpha_5, \alpha_6$ with coefficients 
$0$ or $1$ depending on the decreasing  order  of $b_i$. For example if 
$b_1\ge b_3\ge b_5 \ge b_6\ge b_4$,  then we take 
$\beta_2=\alpha_1+\alpha_3+\alpha_4+\alpha_5+\alpha_6$, $\beta_3=\alpha_5+\alpha_6$, $\beta_4=\alpha_1+\alpha_3$,
$\beta_5=\alpha_5$ and $\beta_6=\alpha_1$.
The existence of nonnegative coefficients $c_i$ follows easily. 
We can check the condition that $\beta_i+\beta_j\not \in \Delta$ by simply 
list all the roots. 
\end{proof}

\begin{proof}[Proof of Lemma \ref{lem;abelian}]
Let $\mathfrak a$ be a maximal $\mathbb R$ split Cartan  subalgebra of the Lie algebra $\mathfrak g$ of 
$G$ so that for some 
  $v\in \mathfrak a$ we have  $g_t=\exp tv$.
  Let $B$ be the Killing form of $\mathfrak g$ and let $\theta$
  be the Cartan involution of $\mathfrak g$ with $\mathfrak a$ belonging to the eigenspace of $-1$.
  The inner product of $w, w'\in \mathfrak a$ is defined by 
  $(w, w')=-B(w, \theta w')$. 
  For every $w\in \mathfrak g$ we associate $\alpha_w\in a^*$ where
$\alpha_w(w')=(w, w')$.
This defines an isomorphism of real vector spaces $\mathfrak a\to \mathfrak a^*$.
The inner product on $\mathfrak a$ can be transferred to 
 an inner product on $\mathfrak a^*$ via this isomorphism. 
 
  
Let    $\Phi(\mathfrak g, \mathfrak a) ^+$ be a positive system dominated by $v$
in the relative  root system
$\Phi(\mathfrak g, \mathfrak a) \subset \mathfrak a^*$. 
Suppose that $\mathfrak a$ has dimension $n$. 
It follows from Lemma \ref{lem;root} that there exist nonnegative real numbers  $c_1, \ldots, c_n$
and $\beta_1,\ldots,  \beta_n  \in \Phi(\mathfrak g, \mathfrak a)^+$ such that 
\[
\alpha_v=c_1\beta_1+\ldots+c_n\beta_n\quad \mbox{and }\quad
\beta_i+\beta_j\not\in \Phi(\mathfrak g, \mathfrak a).
\] 
For each $i$ with $c_i>0$ we choose some $w_i$ in the root space of  $\beta_i$
and fix a natural  $\mathfrak {sl}_2$ triple  $(v_i, w_i, w_i^-)$ where $w_i^-$ belongs
to the root space of $-\beta_i$ and $v_i\in \mathfrak a$ satisfies $\alpha_{v_i}=b_i\beta_i$
for some $b_i> 0$, cf.~\cite{knapp} Proposition 6.52.
Let  $G_1\subset G$ be the connected Lie group whose 
 Lie algebra $\mathfrak g_1 $ is generated by 
these $\mathfrak {sl}_2$ triples. 
It is straightforward to check using Lemma \ref{lem;characterize} that 
  the connected  subgroup $U_a$ whose  lie algebra is generated by 
 $\{w_i: c_i\neq 0\}$ satisfies property ($*$).
\end{proof}

\begin{thebibliography}{99}

\bibitem{a}
J. Athreya, 
Quantitative recurrence and large deviations
for Teichmuller geodesic flow, Geom. Dedicata (2006) 119, 121-140. 

\bibitem{bq12}
Y. Benoist and J-F Quint, Random walks on finite volume homogeneous spaces, Invent. math. 
187 (2012), 37-59.

\bibitem{bq13}
Y. Benoist and J-F Quint, Stationary measures and invariant subsets of homogeneous spaces (II), 
J. Amer. Math. Soc. 26 (2013), 659-734.

\bibitem{bq132}
Y. Benoist and J-F Quint, Stationary measures and invariant subsets of homogeneous spaces (III),
Ann. Math. 178 (2013), 1017-1059. 

\bibitem{bkm01}
V. Bernik, D. Kleinbock, and G. A.  Margulis, Khintchine type theorems on manifolds: The convergence case 
for standard multiplicative versions, Internat. Math. Res. Notices 9 (2001), 453-486.

\bibitem{bb}
F. Bien and A. Borel,
Sous-groupes Žpimorphiques de groupes linŽaires algŽbriques  I, (French) [Epimorphic subgroups of linear algebraic groups. I] C. R. Acad. Sci. Paris SŽr. I Math. 315 (1992), 649-653.

\bibitem{ce}
J. Chaika and A. Eskin, Every flat surface is Birkhoff and oscelledets 
generic in almost every direction, preprint. 

\bibitem{em04}
A. Eskin and G. A.  Margulis, Recurrence properties of random walks on finite volume homogeneous
manifolds,  In: Random Walks and Geometry, de Gruiter, Berlin
(2004),  431-444. 

\bibitem{emm98}
A. Eskin, G. A. Margulis,  and S. Mozes, Upper bounds and asymptotics in a quantitative version of the 
Oppenheim conjecture, Ann. of Math. 147 (1998), 93-141.

\bibitem{gr70}
H. Garland and M. S. Raghunathan, Fundamental domains for lattices in $\mathbb R$-rank $1$ semisimple 
Lie groups, Ann. Math. 92 (1970), 279-326.

\bibitem{hs}
M. Hochman and P. Shmerkin, Equidistribution from fractal measures, preprint.

\bibitem{h95}
B. Host, 
 Nombres normaux, entropie, translations,  Israel J. Math. 91 (1995), 419-428.

\bibitem{kw08}
D.  Kleinbock and B. Weiss,  Dirichlet's theorem on Diophantine approximation and homogeneous flows,
 J. Mod. Dyn. 2 (2008), 43-62.

\bibitem{kw}
D. Kleinbock and B. Weiss, Modified Schmidt games and a conjecture of Margulis, 
J. Mod. Dyn. 7 (2013), 429-460.

\bibitem{knapp}
 A. Knapp, Lie groups beyond an introduction, 2nd Edition, Birkhauser. 
 
 \bibitem{lt92}
G. Lusztig and J. Tits, The inverse of a Cartan matrix, An. Univ. Timisoara Ser. Mat. 30 (1992), 17-23. 
 
\bibitem{m95}
S. Mozes, Epimorphic subgroups and invariant measures, 
Ergodic Theory Dynam. Systems 15 (1995), 1207-1210.

\bibitem{oh98}
H. Oh, Tempered subgroups and representations with minimal decay of matrix coefficients, 
Bull. Soc. math. France 126 (1998), 355-380.

\bibitem{r90}
M. Ratner, Strict measure rigidity for unipotent subgroups of solvable groups, Invent. Math. 101
(1990), 449-482.

\bibitem{r912}
M. Ratner, Raghunathan's topological conjecture and distributions of unipotent flows, Duke
Math. J . 63 (1991), 235-280.

\bibitem{s96}
N. A. Shah, Limit distributions of expanding translates of certain orbits on homogeneous
spaces, Proc. Indian Acad. Sci. Math. Sci. 106 (1996), 105-125.

\bibitem{s09}
N. A. Shah. Limiting distributions of curves under geodesic flow on hyper-
bolic manifolds, Duke Math. J. 148 (2009), 251-279.

\bibitem{s092}
N. A. Shah, Asymptotic evolution of smooth curves under geodesic flow on
hyperbolic manifolds, Duke Math. J. 148 (2009), 281-304.

\bibitem{s093}
N. A. Shah, Equidistribution of expanding translates of curves and Dirichlet's
theorem on Diophantine approximation, Invent. Math. 177 (2009), 509-532.

\bibitem{s10}
N. A. Shah, Expanding translates of curves and Dirichlet-Minkowski theorem 
on linear forms, J. Amer. Math. Soc. 23 (2010), 563-589.

\bibitem{sw96}
N. A. Shah and B. Weiss, On actions of epimorphic subgroups on homogeneous spaces, Ergod. Th. and Dynam. 
Sys. 16 (1996), 1-28.

\bibitem{shi12}
R. Shi, Equidistribution of expanding measures with local maximal dimension and Diophantine
Approximation, Monatsh. Math.  165 (2012), 513-541.

\bibitem{shi122}
R. Shi, Convergence of measures under diagonal actions on homogeneous spaces, Adv. Math.
 229 (2012),  1417-1434.

\bibitem{z}
R. Zimmer, ergodic theory and semisimple groups, Birkhauser, 1984.

\end{thebibliography}

\end{document}
\endinput 

