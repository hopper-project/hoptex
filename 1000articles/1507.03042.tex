

\documentclass{amsart}

\usepackage{amsfonts}
\usepackage{amsmath}
\usepackage{amsthm}
\usepackage{amssymb}
\usepackage[english]{babel}
\usepackage{graphics}
\usepackage{latexsym}
\usepackage{longtable} 
\usepackage{mathrsfs} 
\usepackage{graphicx}
\usepackage{wrapfig} 
\usepackage[pdftex,colorlinks=true]{hyperref}

\newtheorem{theorem}{Theorem}
\newtheorem{corollary}[theorem]{Corollary}
\newtheorem{lemma}[theorem]{Lemma}
\newtheorem{proposition}[theorem]{Proposition}
\newtheorem{conjecture}[theorem]{Conjecture}
\newtheorem{statement}[theorem]{Statement}
\newtheorem{claim}[theorem]{Claim}
\newtheorem{example}[theorem]{Example}

\theoremstyle{definition}
\newtheorem{definition}[theorem]{Definition}
\newtheorem{remark}[theorem]{Remark}
\newtheorem{step}[theorem]{Step}

\numberwithin{equation}{section}

\begin{document}

\title[Equivalence between weak and ${\mathcal{D}}$-solutions for hyperbolic systems]{Equivalence between weak and ${\mathcal{D}}$-solutions for symmetric hyperbolic first order PDE systems}

\author{Nikos Katzourakis}
\address{Department of Mathematics and Statistics, University of Reading, Whiteknights, PO Box 220, Reading RG6 6AX, Berkshire, UK}
\email{n.katzourakis@reading.ac.uk}

\date{}

\keywords{Symmetric hyperbolic first order PDE systems, generalised solutions, fully nonlinear systems, distributional solutions, Young measures.}

\begin{abstract} In a recent paper the author introduced a new theory of generalised solutions for fully nonlinear PDE systems which allows for merely measurable solutions to systems of any order and with perhaps discontinuous coefficients. This approach is a duality-free alternative to distributions and is based on the probabilistic representation of limits of difference quotients via Young measures over compactifications of the ``state space". Herein we show that for  symmetric hyperbolic systems with constant coefficients whereat both weak solutions as well as ${\mathcal{D}}$-solutions apply, the notions actually coincide.
\end{abstract} 

\maketitle

\section{Introduction} \label{section1}

Let $n,N\in {\mathbb{N}}$ and $T>0$ and consider the system of first order PDE
\begin{equation} \label{1.1}
D_t u\ +\ {\textbf{A}} {\!:\!} D u\, =\, f,\ \ \ \ \ \ u\,:\, (0,T){\times}{\mathbb{R}}^n{\longrightarrow} {\mathbb{R}}^N,
\end{equation}
where ${\textbf{A}} : {\mathbb{R}}^{Nn}{\longrightarrow} {\mathbb{R}}^N$ is  a constant map given by
\begin{equation} \label{1.2}
 {\textbf{A}} {\!:\!} Q\ :=\, \sum_{{\alpha},{\beta}=1}^N\sum_{i=1}^n\big({\textbf{A}}_{{\alpha}{\beta} i}\,Q_{{\beta} i}\big)\, e^{\alpha} \, \in {\mathbb{R}}^N, \ \ \ \ Q\in {\mathbb{R}}^{Nn},
\end{equation}
and $f : (0,T){\times}{\mathbb{R}}^n{\longrightarrow} {\mathbb{R}}^N$. We obviously use the notation $Du=(D_1u,...,D_nu)$ for the spatial gradient with respect to $x\in {\mathbb{R}}^n$, $D_t$ denotes the partial temporal derivative with respect to $t>0$ and $\{e^1,...,e^N\}$ is the standard basis of ${\mathbb{R}}^N$ and we will assume that the unknown map $u$ and the right hand side $f$ are both in $L^2\big((0,T)\!{\times}{\mathbb{R}}^n,{\mathbb{R}}^N \big)$.  In index form the system \eqref{1.1} reads
\[
D_t u_{\alpha}\ +\ \sum_{{\beta}=1}^N\sum_{i=1}^n{\textbf{A}}_{{\alpha} {\beta} i}\, D_iu_{\beta}\, =\, f_{\alpha}, \ \ \ \ \ \ {\alpha}\,=\, 1,...,N.
\]
We will throughout this paper assume the \emph{hyperbolicity hypothesis}
\begin{equation}   \label{1.3}
{\textbf{A}}_{{\alpha} {\beta} i} \ =\ {\textbf{A}}_{{\beta} {\alpha} i}
\end{equation}
for all indices ${\alpha},{\beta}=1,...,N$ and all $i=1,...,n$. This implies that for each index $i$ the following matrix is symmetric
\begin{equation}  \label{1.3A}
{\textbf{A}}_i \, := \sum_{{\alpha},{\beta}=1}^N  {\textbf{A}}_{{\alpha}{\beta} i} \ e^{\alpha} {\otimes} e^{\beta} \ \, \in\ {\mathbb{R}}_s^{N{\times} N}
\end{equation}
and hence for each $i=1,...,n$, the space ${\mathbb{R}}^N$ has an orthonormal basis of eigenvectors $\{\eta^{(i)1},...,\eta^{(i)N}\}$ with respective eigenvalues ${\sigma}({\textbf{A}}_i)=\{c^{(i)1},....,c^{(i)N}\}$. The symmetric hyperbolic system \eqref{1.1} plays an important role in many contexts for both theory and applications.

In this paper we consider the question of equivalence between two completely different notions of generalised solutions to \eqref{1.1} and by assuming a commutativity hypothesis on the matrices ${\textbf{A}}_i$ we prove their equivalence. The first one in the usual notion of weak solutions defined by duality
\[
\int_{(0,T){\times} {\mathbb{R}}^n} \Bigg\{ u_{\alpha} \,D_t \phi\ +\, \sum_{{\beta}=1}^N\sum_{i=1}^n{\textbf{A}}_{{\alpha} {\beta} i}\, u_{\beta}\,D_i \phi \ +\ f_{\alpha}\, \phi\Bigg\} \, =\, 0, 
\]
for all $\phi \in C^0_c\big((0,T)\!{\times}{\mathbb{R}}^n \big)$. This is the standard natural notion of generalised solution defined in terms of integration-by-parts. The second notion of solution has very recently been proposed by the author in \cite{K8,K9} and is a duality-free notion of solution which applies to general \emph{fully nonlinear} PDE systems of any order. The a priori regularity required for this sort of solutions is just measurability and the nonlinearities are also allows to be discontinuous. More precisely, let  ${\Omega}{\subseteq} {\mathbb{R}}^m$ be an open set and
\begin{equation}   \label{1.4}
F\ : \ {\Omega} {\times} \Big({\mathbb{R}}^M{\times} {\mathbb{R}}^{Mm}{\times} {\mathbb{R}}^{Mm^2}_s {\times} \cdots {\times} {\mathbb{R}}^{Mm^p}_s \Big) {\longrightarrow} {\mathbb{R}}^d
\end{equation}
a Carath\'eodory map (that is $F$ is measurable with respect to the first argument and continuous with respect to the second). Here ${\mathbb{R}}^{Mm^p}_s$ is the space of symmetric tensors 
\[
\begin{split}
 \Big\{ & \,  {\textbf{X}}  \in  \, {\mathbb{R}}^{Mm^p} \ \, \big| \ \, {\textbf{X}}_{{\alpha} i_1 ...i_k... i_l ... i_p} \, =\ {\textbf{X}}_{{\alpha}  i_1 ...i_l... i_k ... i_p} ,
\\ 
& {\alpha} =1,...,M,  \, i_s =1,...,m,\, s=1,...,p,\, 1<k<l<p\Big\}
 \end{split}
\]
wherein the $p$th order derivative 
\[
D^p u\ =\ \left( D^p_{i_1 ... i_p}u_{\alpha}\right)_{i_1,...,i_p \in \{1,...,m\}}^{{\alpha}=1,...,M}
\]
of (smooth) mappings $u:{\Omega} {\subseteq} {\mathbb{R}}^m {\longrightarrow} {\mathbb{R}}^M$ is valued. Our new theory applies to measurable solutions of the $p$th order PDE system
\begin{equation}  \label{1.5}
F\big(\cdot,u,Du,...,D^pu\big)\, =\, 0,\ \  \text{ on }{\Omega}.
\end{equation}
Since in this approach we do not need to assume that the solutions are locally integrable on ${\Omega}$, the derivatives $Du$, ..., $D^pu$ may not exist not even in the sense of distributions.

The starting point of our notion in not based either on duality or on integration-by-parts. Instead, it relies on the probabilistic representation of the limits of difference quotients by using \emph{Young} (or \emph{parameterised}) measures. The Young measures have been introduced in the first half the previous century (\cite{Y}) in order to show existence of some sort of ``relaxed" solutions to nonconvex minimisation problems for which the infimum is not  attained at a function. Today they are indispensable tools in Calculus of Variations, PDE theory and they have also be studied in a very abstract topological setting (see e.g.\ \cite{E, M, P, FL, CFV, FG, V}). The standard use of Young measures so far has been to quantify the failure of weak convergence to be strong in approximating sequences. Typically this is a result of the combination of phenomena of oscillations and/or concentrations (e.g.\ \cite{DPM, KR}).

In our setting, Young measures are utilised in order to define generalised solutions of \eqref{1.5} by applying it to the \emph{difference quotients} of the candidate solution. Let us motivate the notion in the first order case $p=1$ of \eqref{1.5} which is the only case needed in this paper. Suppose that $u : {\Omega} {\subseteq} {\mathbb{R}}^m {\longrightarrow} {\mathbb{R}}^M$ is a $W^{1,1}_{\text{loc}}({\Omega},{\mathbb{R}}^M)$ strong a.e.\ solution of the PDE system
\begin{equation}   \label{1.6}
F\big(x,u(x),Du(x) \big)\,=\, 0, \quad \text{ a.e.\ }x\in {\Omega}.
\end{equation}
Then, observe that $u$ is a solution if and only if for any compactly supported function $\Phi \in C^0_c\big( {\mathbb{R}}^{Mm} \big)$, we have
\[
\int_{{\mathbb{R}}^{Nn}} \Phi(X)\, F\big( x,u(x),X\big)\, d[{\delta}_{Du(x)} ](X)\, =\, 0, \quad \text{ a.e. }x\in {\Omega}.
\] 
Hence, if we view the weak derivative $Du : {\Omega}{\subseteq} {\mathbb{R}}^m {\longrightarrow}{\mathbb{R}}^{Mm}$ as a probability-valued map ${\delta}_{Du} : {\Omega}{\subseteq} {\mathbb{R}}^m {\longrightarrow} {\mathscr{P}}({\mathbb{R}}^{Mm})$ given by the Dirac mass at the weak gradient, this gives the idea that we may relax the requirement to be a concentration measure and allow instead more general probability-valued maps arising as limits of difference quotients for non-differentiable mappings. In order to derive such a notion which extends strong solutions, we suppose that $u$ is in $W^{1,1}_{\text{loc}}({\Omega},{\mathbb{R}}^M)$ and solves \eqref{1.6}. By the equivalence between weak and strong $L^1$ derivatives, we have that along infinitesimal sequences $(h_\nu)_{\nu=1}^\infty$, $u$ satisfies
\[
F\Big(x,u(x),\lim_{\nu{\rightarrow} \infty}D^{1,h_\nu}u(x)\Big)\, =\, 0, \ \ \ \text{ a.e. }x\in {\Omega},
\]
where $D^{1,h}$ denotes the usual first difference quotients operator. Since $F$ is assumed continuous with respect to the gradient variable, this is equivalent to
\[
\lim_{\nu{\rightarrow} \infty}\, F\Big(x,u(x),D^{1,h_\nu}u(x)\Big)\, =\, 0, \ \ \ \text{ a.e. }x\in {\Omega}.
\]
Note now that the above statement makes sense if $u$ is merely measurable whereas the limit may exist even if the initial limit does not. In order to represent it, we view the difference quotients $D^{1,h}u$ as probability-valued maps ${\delta}_{D^{1,h}u}: {\Omega} {\longrightarrow} \mathscr{P}\big( \smash{\overline{\mathbb{R}}}^{Mm} \big)$ over the Alexandroff compactification
\[
\smash{\overline{\mathbb{R}}}^{Mm}\, :=\ {\mathbb{R}}^{Mm}\cup \{\infty\},
\]
that is as an element of the set of Young measures ${\mathscr{Y}}\big( {\Omega}, \smash{\overline{\mathbb{R}}}^{Mm} \big)$ which consists of measurable probability-valued maps ${\Omega} {\longrightarrow} \mathscr{P}\big( \smash{\overline{\mathbb{R}}}^{Mm} \big)$ (see Section \ref{section2} for the precise definitions and for the properties of Young measures). By the weak* compactness of ${\mathscr{Y}}\big( {\Omega}, \smash{\overline{\mathbb{R}}}^{Mm} \big)$, even if $u$ is merely measurable there always exist probability-valued maps ${\mathcal{D}} u : {\Omega} {\longrightarrow} \mathscr{P}\big( \smash{\overline{\mathbb{R}}}^{Mm} \big)$ such that along subsequences we have
\begin{equation}  \label{1.7}
{\delta}_{D^{1,h_\nu}u} \, \overset{^*}{\smash{-\!\!\!\!-\!\!\!\!\rightharpoonup}}\, \, {\mathcal{D}} u \quad \text{ in }\ {\mathscr{Y}}\big( {\Omega}, \smash{\overline{\mathbb{R}}}^{Mm} \big), \ \ \ \text{ as }\nu {\rightarrow} \infty.
\end{equation}
Then, by a simple convergence argument it follows that strong solutions satisfy
\begin{equation}  \label{1.8}
\int_{\smash{\overline{\mathbb{R}}}^{Mm}} \Phi(X)\, F\big( x,u(x),X\big)\, d[{\mathcal{D}} u(x)](X)\, =\, 0, \quad \text{ a.e. }x\in {\Omega},
\end{equation}
 for any compactly supported ``test" function $\Phi \in C^0_c\big( {\mathbb{R}}^{Mm} \big)$ and any ``diffuse derivative" ${\mathcal{D}} u$. We stress that \eqref{1.7} and \eqref{1.8} are \emph{independent} of the regularity of $u$. In the event that $u$ is weakly differentiable, then ${\mathcal{D}} u$ is unique and coincides with the Dirac mass ${\delta}_{Du}$ a.e.\ on ${\Omega}$. Up to a small adaptation of the idea (we need to take difference quotients with respect to special orthonormal bases depending on the coefficients) \emph{\eqref{1.7} and \eqref{1.8} essentially constitute the definition of \textbf{diffuse derivatives} of the map $u : {\Omega} {\subseteq} {\mathbb{R}}^m {\longrightarrow} {\mathbb{R}}^M$ and of \textbf{${\mathcal{D}}$-solutions} to the PDE system \eqref{1.6}} respectively.

In this paper we restrict our attention to the case of the $1$st order linear hyperbolic system \eqref{1.1} with constant coefficients and compare the notion of ${\mathcal{D}}$-solutions to the usual notion of weak solutions for \eqref{1.1}. The main result of this paper is Theorem \ref{theorem10} (and Corollary \ref{corollary11}) which says that \emph{${\mathcal{D}}$-solutions and weak solutions for \eqref{1.1} coincide if the matrices ${\textbf{A}}_i$ of \eqref{1.3A} commute}. The latter hypothesis is trivially satisfied if either we are in one spatial dimension ($n=1$) or in the scalar case  ($N=1$). We also obtain extra information regarding the \emph{partial regularity} of the solution to the system (in either senses) along certain rank-one lines of ${\mathbb{R}}^{N+Nn}$ which we explain below in the idea of the proof.

 Theorem \ref{theorem10} is particularly significant for the following reasons. Firstly, 
 \eqref{1.1} is an archetypal linear hyperbolic model of two classes of nonlinear PDE of major importance, that is of systems of conservation laws and of systems of Hamilton-Jacobi equations. Herein we show that when both weak solutions and ${\mathcal{D}}$-solutions apply, these two independent approaches actually coincide on their common domain of applicability. This is further strong evidence (on top of the existence-uniqueness results obtained in \cite{K8,K9} which we summarise below) that ${\mathcal{D}}$-solutions are a ``proper" duality-free counterpart of distributions, since they appear to be an efficient framework in which well-posedness of the Cauchy problem for fully nonlinear systems such as
\[
D_tu\, =\, F(\cdot,u,Du), \ \ \ \ u\, :\ (0,T){\times}{\mathbb{R}}^n{\longrightarrow} {\mathbb{R}}^N,
\]
can be considered and studied. 

Secondly, we provide new insights on the analytic structure of the newly proposed generalised objects complementing those of \cite{K8}. As we demonstrate later in Remark \ref{remark9}, unlike distributional derivatives, \emph{diffuse derivatives in general do not behave linearly}.  Without extra hypotheses, the sum of two ${\mathcal{D}}$-solutions to a linear equation may not be a ${\mathcal{D}}$-solution unless one of them is regular. As a byproduct of Theorem \ref{theorem10}, it follows that ${\mathcal{D}}$-solutions become a ``linear" notion for linear hyperbolic systems and hence a linear combination of ${\mathcal{D}}$-solutions to \eqref{1.1} is a ${\mathcal{D}}$-solution itself. Further, 
although it is fairly obvious from the motivations that ${\mathcal{D}}$-solutions are compatible with strong and classical solutions, there is no direct connection between ${\mathcal{D}}$-solutions and nonsmooth distributional solutions. \emph{Formally}, however, it can be easily seen that the \emph{distributional derivatives coincide with the barycentres of the diffuse derivative measures} (Remark \ref{remark9}). The results herein in a sense confirm this claim rigorously.

The essential idea of the proof of Theorem \ref{theorem10} is the following. First, by using  approximation and some \emph{partial regularity estimates}, we show that a map $u\in L^2((0,T){\times}{\mathbb{R}}^n,{\mathbb{R}}^N)$ is a weak solution to \eqref{1.1} if and only if the distributional space-time gradient $(D_t u,Du)$ has a projection along a certain subspace of ${\mathbb{R}}^{N+Nn}$ which is given by an $L^2$ matrix-valued map (Lemmas \ref{lemma12}, \ref{lemma14}). We follow the ideas of \cite{K8} and formalise this by using an extension of the classical Sobolev spaces which we call ``fibre spaces". The fibre space consists of \emph{partially} differentiable maps only along certain rank-one directions of ${\mathbb{R}}^{N+Nn}$ which correspond to the \emph{directions of non-degeneracy of \eqref{1.1}}. The commutativity hypothesis on the matrices ${\textbf{A}}_i$ gives the necessary condition that the subspace of non-degeneracies of the system (on which we have estimates) is spanned by rank-one directions. Then, we use the machinery of ${\mathcal{D}}$-solutions in order to characterise this partially regular object in the fibre space as a ${\mathcal{D}}$-solution to \eqref{1.1} (Lemma \ref{lemma13}, Remark \ref{remark15}).
 

Our motivation to introduce and study generalised solutions primarily comes from the necessity to study rigorously the equations arising in \emph{vectorial Calculus of Variations in the space $L^\infty$} and in particular the model of the $\infty$-Laplace system. Calculus of Variations in  $L^\infty$ concerns the study of variational problems of the so-called supremal functionals as well as of its associated analogues of the Euler-Lagrange equations.  The field has a relatively short history which was started in the 1960s by Aronsson (\cite{A1}-\cite{A5}) who was the first to consider variational problems for functionals of the form
\begin{equation}  \label{1.9}
E_\infty(u,{\Omega})\, :=\, \big\| H(\cdot,u,Du) \big\|_{L^\infty({\Omega})}
\end{equation}
and to study its associated equations. The study of \eqref{1.9} is inherently interesting both from the viewpoint of applications as well as from the purely mathematical side. On the one hand, minimisation of the maximum ``energy" provides more realistic models as opposed to the classical case of minimisation of the average ``energy" (integral). For instance, it is more realistic to know the maximum tolerance of a material to destructive forces rather than the average. On the other hand, the equations in $L^\infty$ are \emph{nondivergence} and highly degenerate. Until recently, the study of \eqref{1.9} was restricted exclusively to the scalar case. In the scalar case, the theory of viscosity solutions of Crandall-Ishii-Lions proved to be the proper setting for the $L^\infty$ equations (see \cite{K} for a pedagogical introduction). In the vectorial case, though, the absence of an efficient theory of generalised solutions hindered the rigorous study of the \emph{systems} arising in $L^\infty$. The vectorial case of \eqref{1.9} has been pioneered by the author in a series of recent papers  (\cite{K1}-\cite{K6}, \cite{K8,K9}). In the simplest model case of the functional
\begin{equation}  \label{1.10}
E_\infty(u,{\Omega})\, =\, \big\| |Du|^2 \big\|_{L^\infty({\Omega})}
\end{equation}
applied to Lipschitz maps $u : {\Omega} {\subseteq} {\mathbb{R}}^m {\longrightarrow} {\mathbb{R}}^M$ (where $|Du|$ is the Euclidean norm on ${\mathbb{R}}^{Mm}$), the analogue of the Euler-Lagrange equation is the $\infty$-Laplace system:
\begin{equation} \label{1.11}
{\Delta}_\infty u \, :=\, \Big(Du {\otimes} Du + |Du|^2[Du]^\bot \! {\otimes} I \Big):D^2u\, =\, 0.
\end{equation}
In \eqref{1.11}, the symbol $[Du]^\bot$ denotes the orthogonal projection on the orthogonal complement of the range of the gradient
\[
[Du(x)]^\bot\,:=\, \text{Proj}_{(R(Du(x)))^\bot}
\]
and in index form the $\infty$-Laplacian reads
\[
\sum_{{\beta}=1}^M\sum_{i,j=1}^m \Big(D_iu_{\alpha} \, D_ju_{\beta} +\, |Du|^2 [Du]_{{\alpha} {\beta}}^\bot \, {\delta}_{ij}\Big)\, D_{ij}^2u_{\beta}\, =\, 0, \ \ \ \ {\alpha}=1,...,M.
\]
In the recent paper \cite{K8}, among other things we proved existence of ${\mathcal{D}}$-solutions to the Dirichlet problem for the $\infty$-Laplace system and in \cite{K9} we proved existence for the equations arising from the functional \eqref{1.9} but in one spatial dimension.
The methods used therein varied and were based on the analytic ``counterpart" of Gromov's convex integration (the Dacorogna-Marcellini Baire category method \cite{DM}) for differential inclusions and on a priori estimates and $L^p$ approximations as $p{\rightarrow} \infty$. In the paper \cite{K8} we also considered the Dirichlet problem for fully nonlinear degenerate elliptic systems
\[
\left\{
\begin{split}
F(\cdot,D^2u) \, & =\, f, \ \text{ in }{\Omega},\\
u\, &=\, 0, \ \text{ on }{\partial}{\Omega},
 \end{split}
 \right.
\]
and we proved existence and uniqueness  of a ${\mathcal{D}}$-solution to the problem (which in general may not be even once weakly differentiable).
 

\section{Theory of ${\mathcal{D}}$-solutions for fully nonlinear systems} \label{section2}

\subsection{Preliminaries}  \label{subsection2.1} We begin with some notation and some basic facts which will be used throughout the paper, perhaps without explicitly quoting this subsection. 
{\medskip}

{\noindent} \textbf{Basics.} Our measure theoretic and function space notation is either standard, e.g.\ as in \cite{E2,EG} or self-explanatory. The norms $|\cdot|$ appearing will always be the Euclidean, while the Euclidean inner products will be denoted by either ``$\cdot$" on ${\mathbb{R}}^m,{\mathbb{R}}^M$ or by ``$:$" on matrix spaces, e.g.\ on ${\mathbb{R}}^{Mm}$ we have 
\[
|X|^2\, = \,\sum_{{\alpha}=1}^M\sum_{i=1}^m X_{{\alpha} i}X_{{\alpha} i} \, \equiv\, X{\!:\!} X ,
\]
etc. The symbol ``$:$" will also be used to denote higher order contractions as e.g.\ in \eqref{1.1}, \eqref{1.2} but the exact meaning will be clear form the context. The standard bases on ${\mathbb{R}}^m$, ${\mathbb{R}}^M$, ${\mathbb{R}}^{Mm}$ will be denoted by $\{e^i | \, i\}$, $\{e^{\alpha} | \, {\alpha}\}$ and $\{ e^{\alpha} {\otimes} e^i |\, {\alpha},i\}$ respectively. For the main part of the paper, we will take $M=N$ and $m=n+1$ and ${\mathbb{R}}^{Mm}={\mathbb{R}}^{N+Nn}$. 

We will follow the convention of denoting vector subspaces of ${\mathbb{R}}^{Mm}$ as well as the orthogonal projections on them by the same symbol. For example, if $\Pi {\subseteq} {\mathbb{R}}^{Mm}$ is a subspace, we denote the projection $\text{Proj}_\Pi : {\mathbb{R}}^{Mm} {\longrightarrow} {\mathbb{R}}^{Mm}$ by just $\Pi$.  

We will also use the $1$-point compactification of ${\mathbb{R}}^{Mm}$ which will be denoted by 
\[
\smash{\overline{\mathbb{R}}}^{Mm} \, :=\, {\mathbb{R}}^{Mm} \cup \{\infty\}.
\]
Its metric distance will be the usual one which makes it isometric to the sphere of the same dimension (via the stereographic projection which identifies the point infinity $\{\infty\}$ with the north pole of the sphere). ${\mathbb{R}}^{Mm}$ will be viewed as a metric vector space isometrically contained into its compactification $\smash{\overline{\mathbb{R}}}^{Mm}$. We note that balls and distances taken in ${\mathbb{R}}^{Mm}$ will be the \emph{Euclidean} and we will employ orthonormal expansions with respect to the usual inner product of ${\mathbb{R}}^{Mm}$ (see the next paragraph) but this will cause no confusion since the underlying topology is the same.

{\medskip}

{\noindent} \textbf{Derivatives \& difference quotients with respect to general bases.} In the sequel we will express derivatives and difference quotients with respect to non-standard orthonormal bases. Let $\{E^1,...,E^M\}$ be an orthonormal basis of ${\mathbb{R}}^M$ and suppose that for each ${\alpha}=1,...,M$ we have an orthonormal basis $\{E^{({\alpha})1},...,E^{({\alpha})m}\}$ of ${\mathbb{R}}^m$. Given such bases, we will equip ${\mathbb{R}}^{Mm}$ with the induced orthonormal basis
\begin{equation} \label{2.1}
\begin{split}
 {\mathbb{R}}^{Mm} =\, {\textrm{span}}[ \big\{ E^{{\alpha} i} \,| \, {\alpha},i \big\}],  \ \ \ \  E^{{\alpha} i} := \, E^{\alpha} {\otimes} E^{({\alpha})i}.
\end{split}
\end{equation}
Then, if $D_{E^{({\alpha})i}}$ denotes the directional derivative along $E^{({\alpha})i}$, the gradient $Du$ of a regular map $u:{\Omega} {\subseteq} {\mathbb{R}}^m {\longrightarrow} {\mathbb{R}}^M$ defined on an open set ${\Omega}$ can be written as
\begin{equation} \label{2.2}
Du\, =\, \sum_{{\alpha}=1}^M\sum_{i=1}^m\Big( E^{{\alpha} i} : Du \Big) \, E^{{\alpha} i}\, =\, \sum_{{\alpha}=1}^M\sum_{i=1}^m \Big( D_{E^{({\alpha})i}}(E^{\alpha} \cdot u)\Big) \, E^{{\alpha} i}.
\end{equation}
Further, let $u:{\Omega}{\subseteq} {\mathbb{R}}^m {\longrightarrow} {\mathbb{R}}^M$ be any measurable map which we understand to be extended by zero on ${\mathbb{R}}^m{\setminus}{\Omega}$. For any $a\in {\mathbb{R}}^n$ with $|a|=1$ and $h\in {\mathbb{R}}{\setminus} \{0\}$, the difference quotients of $u$ along the direction $a$ at $x$ will be denoted by
\begin{equation}  \label{2.3}
D^{1,h}_a u(x)\, :=\, \frac{1}{h}\big[u(x+ha)-u(x)\big].
\end{equation} 
Given any infinitesimal sequence $(h_\nu)_{\nu=1}^\infty {\subseteq} {\mathbb{R}}{\setminus}\{0\}$ with $h_\nu{\rightarrow} 0$ as $\nu{\rightarrow} \infty$, we define the \textbf{difference quotients of $u$} (with respect to the fixed reference frames) arising from $(h_\nu)_{\nu=1}^\infty$ as
\begin{equation} \label{2.4}
\begin{split}
 D^{1,h_\nu}u \, :\, \ {\Omega} {\subseteq} {\mathbb{R}}^m {\longrightarrow} {\mathbb{R}}^{Mm},  \ \ \ \  D^{1,h_\nu}u  \, :=  \, \sum_{{\alpha}=1}^M\sum_{i=1}^m  \left[ D^{1,h_\nu}_{E^{({\alpha})i} }(E^{\alpha} \cdot u) \right] \, E^{{\alpha} i} .
\end{split}
\end{equation}
In the above, $D^{1,h_\nu}_{E^{({\alpha})i} }u$ is as in \eqref{2.3} and $E^{{\alpha} i}$ as in \eqref{2.1}.

{\medskip}

{\noindent} \textbf{Young Measures.} Let $E{\subseteq} {\mathbb{R}}^m$ be a (Lebesgue) measurable and consider the $L^1$ space of strongly measurable maps valued in $C^0\big(\smash{\overline{\mathbb{R}}}^{Mm}\big)$, that is in the space of continuous functions over $\smash{\overline{\mathbb{R}}}^{Mm}$:
\[
L^1\big( E, C^0\big(\smash{\overline{\mathbb{R}}}^{Mm}\big)\big).
\]
For background material on this space we refer e.g.\ to \cite{FL,FG,Ed,V} (and references therein) as well as to \cite{K8,K9}. The elements of this space can be identified with the Carath\'eodory functions
\[
\Phi\ :\ E {\times} \smash{\overline{\mathbb{R}}}^{Mm} {\longrightarrow} {\mathbb{R}}, \quad (x,X)\mapsto \Phi(x,X)
\]
in the sense that $E\ni x\mapsto \Phi(x,\cdot) \in C^0\big(\smash{\overline{\mathbb{R}}}^{Mm}\big)$. The norm of the space is given by
\[
\| \Phi \|_{L^1( E, C^0(\smash{\overline{\mathbb{R}}}^{Mm}))}\, :=\, \int_E \left(\!\! \max_{\ \ {\overline{\mathbb{R}}}^{Mm}} \big|\Phi(x,\cdot)\big|\right)\, dx.
\]
The dual space is given by 
\[
\left( L^1\big( E, C^0\big(\smash{\overline{\mathbb{R}}}^{Mm}\big)\big) \right)^* \, =\, L^\infty_{w^*}\big( E,{\mathcal{M}} \big(\smash{\overline{\mathbb{R}}}^{Mm}\big) \big).
\]
Here ${\mathcal{M}} \big(\smash{\overline{\mathbb{R}}}^{Mm}\big)$ is the space of Radon measures equipped with the total variation norm. The dual space $L^\infty_{w^*}\big( E,{\mathcal{M}} \big(\smash{\overline{\mathbb{R}}}^{Mm}\big)\big)$ consists of \emph{weakly* measurable} measure-valued mappings
\[
E\ \ni \ x \longmapsto \vartheta(x)  \ \in\, {\mathcal{M}} \big(\smash{\overline{\mathbb{R}}}^{Mm}\big),
\]
that is, for any $\vartheta$ in the space and any fixed $\Phi \in C^0\big(\smash{\overline{\mathbb{R}}}^{Mm}\big)$, the function 
\[
E\, \ni\  x\, {\longmapsto}  \int_{ \smash{\overline{\mathbb{R}}}^{Mm} } \Phi(X)\, d[\vartheta(x)](X) \ \in \, {\mathbb{R}}
\]
is Lebesgue measurable. The duality pairing between the spaces is given by
\[
\begin{split}
\langle\cdot,\cdot\rangle\ :\  \ & L^\infty_{w^*}\big( E,{\mathcal{M}} \big(\smash{\overline{\mathbb{R}}}^{Mm}\big) \big) {\times} L^1\big( E, C^0\big(\smash{\overline{\mathbb{R}}}^{Mm}\big)\big) \, {\longrightarrow} \,{\mathbb{R}},\\
&\langle \vartheta, \Phi \rangle\, :=\, \int_E \int_{ \smash{\overline{\mathbb{R}}}^{Mm} } \Phi(x,X)\, d[\vartheta(x)] (X)\, dx
\end{split}
\]
and the norm of the space $L^\infty_{w^*}\big( E,{\mathcal{M}} \big(\smash{\overline{\mathbb{R}}}^{Mm}\big) \big)$ is the $L^\infty$-norm of the total variation:
\[
\| \vartheta \|_{L^\infty_{w^*} ( E,{\mathcal{M}} (\smash{\overline{\mathbb{R}}}^{Mm} ) )}\, :=\, \underset{x\in E}{{\textrm{ess}}\,\sup}\, \left\|\vartheta(x) \right\|\big(\smash{\overline{\mathbb{R}}}^{Mm}\big).
\]
\begin{definition}[Young measures] The space of Young measures is the set of all probability-valued mappings  $E{\subseteq} {\mathbb{R}}^m {\longrightarrow} {\mathscr{P}}\big( \smash{\overline{\mathbb{R}}}^{Mm} \big)$ which are weakly* measurable. Hence, the set of Young measures can be identified with a subset of the unit sphere of $L^\infty_{w^*}\big( E,{\mathcal{M}} \big(\smash{\overline{\mathbb{R}}}^{Mm}\big) \big)$:
\[
{\mathscr{Y}}\big(E,\smash{\overline{\mathbb{R}}}^{Mm}\big)\, :=\, \Big\{\, \vartheta\, \in \, L^\infty_{w^*}\big( E,{\mathcal{M}} \big( \smash{\overline{\mathbb{R}}}^{Mm} \big) \big)\, : \, \vartheta(x) \in {\mathscr{P}} \big( \smash{\overline{\mathbb{R}}}^{Mm} \big),\text{ a.e. }x\in E \, \Big\}.
\]
We will equip the Young measures with the induced weak* topology (which is metrisable and sequentially precompact on bounded sets because $L^1\big( E, C^0\big(\smash{\overline{\mathbb{R}}}^{Mm}\big)\big)$ is separable).
\end{definition}

\begin{remark}[Properties of ${\mathscr{Y}}\big(E,\smash{\overline{\mathbb{R}}}^{Mm}\big)$] \label{remark2} The next standard facts about Young measures will be used systematically (see e.g.\ \cite{FG}): 
\smallskip

i) [\textbf{Weakly* compact space}] ${\mathscr{Y}}\big(E,\smash{\overline{\mathbb{R}}}^{Mm}\big)$ is a convex and sequentially weakly* compact set.  

\smallskip

ii) [\textbf{Functions as Young measures}]  The set of  measurable mappings $v : E{\subseteq} {\mathbb{R}}^m {\longrightarrow} \smash{\overline{\mathbb{R}}}^{Mm}$ can be embedded into ${\mathscr{Y}}\big(E,\smash{\overline{\mathbb{R}}}^{Mm}\big)$ via the Dirac mass $v \mapsto {\delta}_v$ given by ${\delta}_v(x):= {\delta}_{v(x)}$ (and the embedding is weakly* dense).  

\end{remark}

The next fact is a essentially a classical result but it plays a fundamental role in our framework because it will guarantee compatibility of ${\mathcal{D}}$-solutions and strong solutions (its simple proof can be found e.g.\ in \cite{K8}).

\begin{lemma} \label{lemma2} Let  $U^\nu,U^\infty : E{\subseteq} {\mathbb{R}}^m{\longrightarrow} \smash{\overline{\mathbb{R}}}^{Mm}$ be measurable maps, $\nu\in {\mathbb{N}}$. Then, there are subsequences $(\nu_k)_1^\infty$, $(\nu_l)_1^\infty$ such that:
\[
\begin{split}
& (1) \quad U^\nu {\longrightarrow} U^\infty \ \text{ a.e.\ on }E\hspace{31pt} \Longrightarrow \ \ {\delta}_{U^{\nu_k}} \overset{_*}{-\!\!\!\!-\!\!\!\!\rightharpoonup}{\delta}_{U^\infty} \text{ in }{\mathscr{Y}}\big(E,\smash{\overline{\mathbb{R}}}^{Mm}\big),\\
& (2)\quad  {\delta}_{U^{\nu}} \overset{_*}{-\!\!\!\!-\!\!\!\!\rightharpoonup}{\delta}_{U^\infty} \text{ in } {\mathscr{Y}}\big(E,\smash{\overline{\mathbb{R}}}^{Mm}\big) \ \ \Longrightarrow \ \ U^{\nu_l} {\longrightarrow} U^\infty \ \text{ a.e.\ on }E.
\end{split}
\]
\end{lemma}

{\noindent} \textbf{Main definitions.}  We restrict our attention only to the first order case which is relevant to the system \eqref{1.1} we consider herein. For the general case we refer to \cite{K8}.

\begin{definition}[Diffuse derivatives]  \label{definition6}
Suppose we are given some fixed reference frames as in \eqref{2.1}. For any measurable map $u : {\Omega}{\subseteq} {\mathbb{R}}^m{\longrightarrow}{\mathbb{R}}^M$, we define the \textbf{diffuse gradients ${\mathcal{D}} u$} of $u$ as the limits of the difference quotients $D^{1,h_\nu}u$ (see \eqref{2.3}-\eqref{2.4}) in the spaces of Young measures ${\mathscr{Y}}\big(E,\smash{\overline{\mathbb{R}}}^{Mm}\big)$ which arise along infinitesimal subsequences $(h_{\nu_k})_1^\infty {\subseteq} (h_{\nu})_1^\infty$:
\[
\begin{split}
&{\delta}_{D^{1,h_{\nu_k}}u} \, \overset{^*}{\smash{-\!\!\!\!-\!\!\!\!\rightharpoonup}}\ {\mathcal{D}} u\  \ \ \, \text{ in }\,{\mathscr{Y}}\big({\Omega},\smash{\overline{\mathbb{R}}}^{Mm}\big), \ \ \quad  \text{ as }k{\rightarrow} \infty.
\end{split}
\]
\end{definition} 

By the weak* compactness of  ${\mathscr{Y}}\big(E,\smash{\overline{\mathbb{R}}}^{Mm}\big)$, every measurable map possesses diffuse derivatives, in fact at least one for every infinitesimal sequence. In general diffuse gradients \emph{may not be unique} for nonsmooth maps but they are compatible with weak derivatives: 
\begin{lemma}[Compatibility of weak and diffuse gradients] \label{lemma10} Suppose $u \in W^{1,1}_{\text{loc}}({\Omega},{\mathbb{R}}^M)$. Then, the diffuse gradient ${\mathcal{D}} u$ is unique and we have
${\delta}_{Du}\, =\, {\mathcal{D}} u$ a.e.\ on ${\Omega}$.
\end{lemma}

The proof of Lemma \ref{lemma10} is an immediate consequence of Lemma \ref{lemma2} and of the standard equivalence between weak and strong $L^1$ derivatives. Now we give our notion of solution to fully nonlinear first order PDE systems.

\begin{definition}[${\mathcal{D}}$-solutions of $1$st order systems] \label{definition11} Let ${\Omega}{\subseteq} {\mathbb{R}}^m$ be open,
\[
F\ : \ {\Omega} {\times} \left({\mathbb{R}}^M{\times} {\mathbb{R}}^{Mm} \right) {\longrightarrow} {\mathbb{R}}^d
\]
a Carath\'eodory map and $u : {\Omega}{\subseteq} {\mathbb{R}}^m {\longrightarrow} {\mathbb{R}}^M$ a map in $W^{1,1}_{\text{loc}}({\Omega},{\mathbb{R}}^M)$. Suppose we have fixed some reference frames as in \eqref{2.1} and consider the PDE system
\begin{equation} \label{2.11}
F\big(x,u(x),Du(x)\big)\, =\,0, \ \ \text{ on }{\Omega}.
\end{equation}
We say that $u$ is a \textbf{${\mathcal{D}}$-solution of \eqref{2.11}} when for any diffuse gradient of $u$ arising from any infinitesimal sequence along subsequences (Definition \ref{definition6})
 \[
 {\delta}_{D^{1,h_{\nu_k}}u} \, \overset{^*}{\smash{-\!\!\!\!-\!\!\!\!\rightharpoonup}}\, \, {\mathcal{D}} u \ \ \ \text{ in }\ \mathscr{Y}\big({\Omega}, \smash{\overline{\mathbb{R}}}^{Mm} \big),\ \ \text{ as }k{\rightarrow} \infty
\]
and for any $\Phi \in C^0_c\big( {\mathbb{R}}^{Mm}\big)$, we have
\[
\int_{\smash{\overline{\mathbb{R}}}^{Mm}} \Phi(X)\, F\big(x,u(x),X\big)\, d[{\mathcal{D}} u(x) ](X)\, =\, 0, \ \ \ \text{ a.e.\ }x\in {\Omega}.
\] 
 \end{definition} 

The following result asserts the fairly obvious fact that ${\mathcal{D}}$-solutions are compatible with strong solutions. 

\begin{proposition}[Compatibility of ${\mathcal{D}}$-solutions with strong solutions] \label{proposition14} Let $F$ be a Carath\'eodory map as above, $u:{\Omega}{\subseteq} {\mathbb{R}}^m {\longrightarrow} {\mathbb{R}}^M$ a mapping in $W^{1,1}_{\text{loc}}({\Omega},{\mathbb{R}}^M)$ and consider the PDE system \eqref{2.11}. Then, $u $ is a ${\mathcal{D}}$-solution on ${\Omega}$ if and only if $u$ is a strong a.e.\ solution on ${\Omega}$.
\end{proposition}

The proof of Proposition \ref{proposition14}  is an immediate consequence of Lemma \ref{lemma10} and of the motivation of the notions.

\begin{remark}[Nonlinearity of diffuse derivatives and relation to distributions] \label{remark9} 

We summarise here some of the discussions of \cite{K8}. In the context of the usual notions of solution (smooth, strong, weak, distributional), it is standard that the generalised derivative is a linear operation. However, without extra assumptions this may be false for diffuse derivatives;  \textit{${\mathcal{D}}$-solutions are a genuinely nonlinear  approach even when we apply them to linear PDE.} More precisely, let $T_a : {\mathbb{R}}^{Mm} {\rightarrow}  {\mathbb{R}}^{Mm}$ denote the translation by $a$. Given a Young measure $\vartheta \in \mathscr{Y}\big({\Omega}, \smash{\overline{\mathbb{R}}}^{Mm} \big)$, we define $\vartheta \circ T_a \in \mathscr{Y}\big({\Omega}, \smash{\overline{\mathbb{R}}}^{Mm} \big)$  by
\[
\big[(\vartheta \circ T_a) (x)\big] (B) \, :=\, [\vartheta(x)] \big((B{\setminus} \{\infty\}) - a \big)\ +\ [\vartheta(x)]  \big(\{\infty\}\big),
\]
for any Borel set $B{\subseteq} \smash{\overline{\mathbb{R}}}^{Mm}$ and a.e.\ $x\in{\Omega}$. Hence, we translate the Euclidean part of $B$ while ``infinity" is left intact (it can be seen as a push-forward measure as well). It can be easily proved that (see \cite{K8}, also \cite{FL}), given any  two maps $u,v : {\Omega}{\subseteq} {\mathbb{R}}^m{\longrightarrow} {\mathbb{R}}^M$ for which one of them, say $v$, is in addition weakly differentiable, we have
\[
{\mathcal{D}}(u+v)\, =\, Du \circ T_{Dv},  \ \ \ \text{ a.e.\ on } {\Omega}.
\]
Regarding the relation between distributional and diffuse derivatives, we can informally say that \emph{the barycentres of the diffuse derivatives are the distributional derivatives}. This can be seen as follows: if $u\in L^1({\mathbb{R}}^m)$, then we have $D^{1,h}u \overset{^*}{\smash{-\!\!\!\!-\!\!\!\!\rightharpoonup}}\, Du $ in the distributions as $h{\rightarrow} 0$. Since $D^{1,h}u$ is the barycentre of the measure ${\delta}_{D^{1,h}u}$ and $ {\delta}_{D^{1,h_\nu }u} \overset{^*}{\smash{-\!\!\!\!-\!\!\!\!\rightharpoonup}}\, {\mathcal{D}} u$ in the Young measures, we roughly have $\text{bar}({\mathcal{D}} u)=Du$. However, this formal relation lacks precision primarily because it does not take into account the loss of mass at $\infty$ which is the central ``nonlinear" feature.
\end{remark}

\section{Equivalence between weak and ${\mathcal{D}}$-solutions for hyperbolic systems} \label{section3}

\subsection{Fibre spaces and the main result} \label{subsection4.2} Before stating our main result we need some preparation. Given the map ${\textbf{A}}$ of \eqref{1.2} which we assume satisfies the hyperbolicity hypothesis \eqref{1.3}, we define the linear map
\begin{equation} \label{3.1}
{\underline{\textbf{A}}} \ :\ \ \ {\mathbb{R}}^{N +Nn}\, {\longrightarrow} \, {\mathbb{R}}^N, \ \ \ \ {\underline{\textbf{A}}} {\!:\!} \underline{X}\, :=\, X_0\, +\, {\textbf{A}}\!:\!X,
\end{equation}
that is ${\underline{\textbf{A}}}$ is given by
\[
{\mathbb{R}}^{N +Nn} \ni \ \underline{X} \equiv [X_0|X] \ {\longmapsto} \ \sum_{{\alpha}=1}^N\Bigg(X_{{\alpha} 0} \, +\, \sum_{{\beta}=1}^N\sum_{i=1}^n{\textbf{A}}_{{\alpha} {\beta} i}\, X_{{\beta} i}\Bigg) e^{\alpha} \ \in {\mathbb{R}}^N.
\]
The nullspace of ${\underline{\textbf{A}}}$ will be denoted as  
\begin{equation} \label{3.2}
N({\underline{\textbf{A}}})\, :=\, \Big\{\underline{X} \in {\mathbb{R}}^{N+Nn} \ \big| \ X_{ 0} \, +\, {\textbf{A}}{\!:\!} X \, =\, 0\Big\}
\end{equation}
and we consider its orthogonal complement
\begin{equation} \label{3.3}
\begin{split}
 \Pi \, :=\ N({\underline{\textbf{A}}})^\bot  \   {\subseteq}\  {\mathbb{R}}^{N +Nn}.
\end{split}
\end{equation}
By using \eqref{3.1}-\eqref{3.3} and the notation
\begin{equation}	 \label{3.4}
\left\{
\ \ \ \begin{split}
\underline{x} \, & :=\, (x_0,x)\, \equiv \, (t,x) \ \ \ \, \in (0,T){\times}\, {\mathbb{R}}^n, \\
{\underline{D}} u(\underline{x})\, & :=\, \big( D_tu(\underline{x}),  Du(\underline{x}) \big) \ \ \in \, {\mathbb{R}}^{N+Nn},
\end{split}
\right.
\end{equation}
when $u :  (0,T){\times} {\mathbb{R}}^n {\longrightarrow} {\mathbb{R}}^N$ is a measurable map, we may rewrite the PDE system \eqref{1.1} as
\begin{equation} \label{3.5}
{\underline{\textbf{A}}} {\!:\!} {\underline{D}} u(\underline{x})\ =\ f(\underline{x}), \ \ \ \ \underline{x}\in \, (0,T){\times}{\mathbb{R}}^n,
\end{equation}
when $f\in L^2\big(  (0,T){\times} {\mathbb{R}}^n,{\mathbb{R}}^N\big)$. We will also use the obvious notation 
 \[
 {\underline{D}}_{\underline{a}}^{1,h}u(\underline{x}) \, =\, \frac{u(\underline{x}+h \underline{a})-u(\underline{x})}{h} 
 \]
for the space-time difference quotients of $u$, the obvious analogues of the expansions \eqref{2.1}-\eqref{2.4} with respect to bases of ${\mathbb{R}}^{1+n}$, ${\mathbb{R}}^N$, ${\mathbb{R}}^{N+Nn}$ and the notation $\underline{\mathcal{D}} u$ for the diffuse space-time gradient
\[
{\delta}_{{\underline{D}}^{1,h_{\nu_k}}u} \, \overset{^*}{\smash{-\!\!\!\!-\!\!\!\!\rightharpoonup}}\,\, \underline{\mathcal{D}} u\ \ \ \text{ in }\, {\mathscr{Y}}\Big((0,T){\times} {\mathbb{R}}^n, \, \smash{\overline{\mathbb{R}}}^{N+Nn}\Big), \ \ \text{ as }k {\rightarrow} \infty
\]
of a measurable mapping $u :  (0,T){\times} {\mathbb{R}}^n {\longrightarrow} {\mathbb{R}}^N$ along infinitesimal subsequences $(h_{\nu_k})_1^\infty{\subseteq} (h_{\nu})_1^\infty$.
 
 {\medskip}

{\noindent} \textbf{The fibre Sobolev space.}  For ${\underline{\textbf{A}}}$ as in \eqref{3.1}, let $\Pi$ be given by \eqref{3.3} and suppose that \emph{$\Pi$ is spanned by rank-one directions $\eta {\otimes} \underline{a}$ of ${\mathbb{R}}^{N +Nn}$}, $\underline{a}=[a_0|a]\in {\mathbb{R}}^{1+n}$. A sufficient condition regarding when this happens is when the matrices ${\textbf{A}}_i$ of \eqref{1.3A} commute, something we will require later. For simplicity, we treat only the $L^2$ first order case needed in this paper (for extensions see \cite{K8}). Let us begin by identifying the Sobolev space $W^{1,2}\big((0,T)\!{\times}\! {\mathbb{R}}^n,{\mathbb{R}}^N\big)$ with its isometric image $\tilde{W}^{1,2}\big((0,T)\!{\times}\! {\mathbb{R}}^n,{\mathbb{R}}^N\big)$ into a product of $L^2$ spaces:
\[
\tilde{W}^{1,2}\Big((0,T)\!{\times}\! {\mathbb{R}}^n,{\mathbb{R}}^N\Big)\, \underset{\, ^{\rightarrow}}{\subset} \, L^2\Big((0,T)\!{\times}\! {\mathbb{R}}^n,\, {\mathbb{R}}^N \! {\times} {\mathbb{R}}^{N+Nn}\Big)
\]
via the mapping $u\mapsto(u,{\underline{D}} u)$. We define the \textbf{fibre space} $\mathscr{W}^{1,2}\big( (0,T)\!{\times}\! {\mathbb{R}}^n,{\mathbb{R}}^N \big)$ (associated to ${\textbf{A}}$ and \eqref{1.1}) as the Hilbert space
\begin{equation} \label{3.7}
\mathscr{W}^{1,2} \Big((0,T)\!{\times}{\mathbb{R}}^n,{\mathbb{R}}^N\Big)\, :=\, \overline{\, \text{Proj}_{ L^2\big((0,T){\times}  {\mathbb{R}}^n,{\mathbb{R}}^N {\times} \Pi \big) }\,  \tilde{W}^{1,2} \Big((0,T)\!{\times}{\mathbb{R}}^n,{\mathbb{R}}^N\Big) \, }^{\|\cdot\|_{ L^2} }
\end{equation}
with the natural induced norm (written for $W^{1,2}$ maps)
\[
\|u\|_{\mathscr{W}^{1,2}((0,T){\times} {\mathbb{R}}^n )} \, :=\ \|u \|_{L^2((0,T){\times} {\mathbb{R}}^n)}+\, \big\|\Pi\, {\underline{D}} u \big\|_{L^2((0,T){\times}{\mathbb{R}}^n)}.
\]
We remind that we use the same letter to denote both the vector space $\Pi$ as well as the orthogonal projection on it. By employing the Mazur theorem,  \eqref{3.7} can be characterised in the following way:
\[
\mathscr{W}^{1,2}  \Big((0,T) {\times} {\mathbb{R}}^n,{\mathbb{R}}^N\Big) = \left\{ 
\begin{array}{l}
 \big(u,  \underline{G}  (u)\big) \in\, L^2\Big( (0,T)\!{\times}\! {\mathbb{R}}^n,\, {\mathbb{R}}^N\! {\times} \Pi \Big) \, \Big|\  \ \exists\ (u^\nu)_1^\infty   {\subseteq}  \\
W^{1,2} \Big((0,T)\!{\times} {\mathbb{R}}^n,{\mathbb{R}}^N\Big) :\,  \left(u^\nu , \Pi\, {\underline{D}} u^\nu \right) {-\!\!\!\!-\!\!\!\!\rightharpoonup} \big(u, \underline{G} (u) \big)\\
 \text{weakly in }L^2\Big((0,T)\!{\times} {\mathbb{R}}^n,\, {\mathbb{R}}^N\!{\times} \Pi\Big),\ \ \text{ as }\nu {\rightarrow}\infty  
\end{array}
\!\!\right\}.
\]
We will call $\underline{G} (u) \in L^2\big((0,T)\!{\times}\! {\mathbb{R}}^n,\Pi\big)$ the \textbf{fibre (space-time) gradient} of $u$. By using integration by parts and the hypothesis that $\Pi$ is spanned by rank-one directions, it can be easily seen that the measurable map $\underline{G} (u)$ \emph{depends only on $u \in L^2\big((0,T)\!{\times}\! {\mathbb{R}}^n,{\mathbb{R}}^N\big)$ and not on the approximating sequence}. Further, by the  equivalence between weak and strong $L^2$ directional derivatives, $\underline{G} (u)$ can be characterised as a ``fibre" derivative of $u$:
\[
\eta {\otimes} \underline{a} \, \in  \Pi{\subseteq} {\mathbb{R}}^{N+Nn} \ \ \, \Longrightarrow \ \ \ \underline{G} (u) : (\eta {\otimes} \underline{a} )\, =\, D_{\underline{a}} (\eta \cdot u), \ \text{ a.e.\ on }(0,T){\times} {\mathbb{R}}^n.
\]
In general, the fibre spaces are strictly larger than their ``non-degenerate" counterparts since it is very easy to find singular examples which are not even once weakly differentiable. 

We may now state our main result. 

\begin{theorem}[Equivalence of weak and ${\mathcal{D}}$-solutions to linear hyperbolic systems with constant coefficients \& partial regularity] \label{theorem10} Let ${\textbf{A}}$ be as in \eqref{1.2}, \eqref{1.3} and consider the system
\begin{equation} \label{3.8}
D_tu\ +\ {\textbf{A}} {\!:\!} Du\, =\, f, \, \ \ \text{ in }(0,T){\times} {\mathbb{R}}^n,
\end{equation}
where $f\in L^2\big(  (0,T){\times} {\mathbb{R}}^n,{\mathbb{R}}^N\big)$ and $T>0$. We suppose that the commutator of the matrices ${\textbf{A}}_1,...,{\textbf{A}}_n$ of \eqref{1.3A} vanishes:
\[
[{\textbf{A}}_i,{\textbf{A}}_j]\, :=\, {\textbf{A}}_i {\textbf{A}}_j - {\textbf{A}}_j{\textbf{A}}_i\, =\, 0, \ \ \ \ i,j\,=\,1,..,n.
\]
Then, a map $u : (0,T){\times} {\mathbb{R}}^n {\longrightarrow} {\mathbb{R}}^N$ is a weak solution to \eqref{3.8} in the space
\begin{equation} \label{3.9}
\mathfrak{X}\, :=\, \Big\{u \in L^2\big((0,T),L^2({\mathbb{R}}^n,{\mathbb{R}}^N) \big)\ : \  D_t u \in L^2\big((0,T),W^{-1,2}({\mathbb{R}}^n,{\mathbb{R}}^N) \big)\Big\}
\end{equation}
if and only if $u : (0,T){\times} {\mathbb{R}}^n {\longrightarrow} {\mathbb{R}}^N$ is a ${\mathcal{D}}$-solution in the fibre Sobolev space $\mathscr{W}^{1,2}\big((0,T){\times} {\mathbb{R}}^n,{\mathbb{R}}^N\big)$ associated to ${\textbf{A}}$ (see \eqref{3.7}). By Definition \ref{definition11}, the latter means there exist orthonormal bases of ${\mathbb{R}}^{1+n}$, of ${\mathbb{R}}^N$ and of ${\mathbb{R}}^{N+Nn}$ (depending only on ${\textbf{A}}$) such that,  for any diffuse (space-time) gradient of $u$ arising along infinitesimal subsequences
\[
{\delta}_{{\underline{D}}^{1,h_{\nu_k}}u} \, \overset{^*}{\smash{-\!\!\!\!-\!\!\!\!\rightharpoonup}}\,\, \underline{\mathcal{D}} u\ \ \ \text{ in }\, {\mathscr{Y}}\Big((0,T){\times} {\mathbb{R}}^n, \, \smash{\overline{\mathbb{R}}}^{N+Nn}\Big), \ \ \text{ as }k {\rightarrow} \infty
\]
and for any $\Phi \in C^0_c({\mathbb{R}}^{N+Nn})$, we have
\[
\int_{ \smash{\overline{\mathbb{R}}}^{N+Nn} } \Phi(\underline{X})\, \Big(X_0\, +\, {\textbf{A}}\:X\, -\, f(t,x) \Big)\,d\big[  \underline{\mathcal{D}} u(t,x) \big](\underline{X})\,=\,0, 
\]
for a.e.\ $(t,x)\in (0,T){\times} {\mathbb{R}}^n$. 

Moreover, any ${\mathcal{D}}$-solution $u$ to \eqref{3.8} (and hence any weak solution) has the following regularity: the projection of the space-time gradient $(D_t u,Du)$ on the subspace $\Pi {\subseteq} {\mathbb{R}}^{N+Nn}$ associated to ${\textbf{A}}$ exists in $L^2$, $\Pi$ is spanned by rank-one matrices and for any such direction $\eta {\otimes} \underline{a}\in \Pi$, we have $D_{\underline{a}} (\eta \cdot u) \in L^2((0,T)\!{\times} \!{\mathbb{R}}^n)$.
\end{theorem}

The commutativity hypothesis is always satisfied if either $n=1$ (one spatial dimension) or $N=1$ (scalar case). By standard results on hyperbolic systems, we readily have the following consequence of Theorem \ref{theorem10}:

\begin{corollary}[Well-posedness of the Cauchy problem for ${\mathcal{D}}$-solutions] \label{corollary11} In the setting of Theorem \ref{theorem10} and under the same assumptions, the Cauchy problem
\[
\left\{
\begin{split}
D_tu\ +\ {\textbf{A}} {\!:\!} Du\, &=\, f, \, \ \ \ \text{ in }(0,T){\times} {\mathbb{R}}^n,\\
u(0,\cdot)\, &=\, u_0, \  \ \text{ on }\{0\}{\times} {\mathbb{R}}^n,
\end{split}
\right.
\]
has a unique ${\mathcal{D}}$-solution in the fibre space $\mathscr{W}^{1,2}\big((0,T)\! {\times} \! {\mathbb{R}}^n,{\mathbb{R}}^N\big)$, for any given $f\in L^2\big(  (0,T)\! {\times} \!{\mathbb{R}}^n,{\mathbb{R}}^N\big)$, $u_0\in L^2 ({\mathbb{R}}^n,{\mathbb{R}}^N )$ and $T>0$. 
\end{corollary}

{\medskip \noindent \textbf{Proof of Theorem} } \ref{theorem10}. The proof consists of three lemmas. In the first one below we show that the commutativity hypothesis on the matrices ${\textbf{A}}_1,...,{\textbf{A}}_n$ implies that the vector space $\Pi$ of \eqref{3.3} has an orthonormal basis of rank-one directions which can be completed to an orthonormal basis of rank-one directions spanning ${\mathbb{R}}^{N+Nn}$.

\begin{lemma} \label{lemma12} Given the map ${\textbf{A}} : {\mathbb{R}}^{Nn}{\longrightarrow} {\mathbb{R}}^N$ of \eqref{1.2}, let ${\underline{\textbf{A}}} : {\mathbb{R}}^{N+Nn}{\longrightarrow} {\mathbb{R}}^N$ be given by \eqref{3.1}, its nullspace $N({\underline{\textbf{A}}})$ by \eqref{3.2} and its orthogonal complement $\Pi$ by \eqref{3.3}. 

Then, if the matrices ${\textbf{A}}_1,..,{\textbf{A}}_n$ of \eqref{1.3A} commute, ${\mathbb{R}}^{N+Nn}$ has an orthonormal basis of rank-one directions such that $N$-many of them span the subspace $\Pi {\subseteq} {\mathbb{R}}^{N+Nn}$ and the rest $Nn$-many of them span the nullspace $N({\underline{\textbf{A}}}) {\subseteq} {\mathbb{R}}^{N+Nn}$:
\begin{equation} \label{3.10}
\left\{\ \ \ 
\begin{split}
{\mathbb{R}}^{N+Nn} \, &=\, {\textrm{span}}\big[\Big\{E^{{\alpha} i} \, :\, {\alpha}=1,...,N,\, i=0,1,...,n\Big\} \big], \\ 
\Pi \, &=\, {\textrm{span}}\big[\Big\{E^{{\alpha} 0} \, :\, {\alpha}=1,...,N\Big\} \big], \\ 
N({\underline{\textbf{A}}}) \, &=\, {\textrm{span}}\big[\Big\{E^{{\alpha} i} \, :\, {\alpha}=1,...,N,\, i=1,...,n\Big\} \big].
\end{split} 
\right.
\end{equation}
In addition, the basis $\{E^{{\alpha} i}\}$ arises in the following way: there is an orthonormal basis $\{E^1,...,E^N\}{\subseteq} {\mathbb{R}}^N$ and for each ${\alpha}=1,...,N$ an orthonormal basis $\{E^{({\alpha})0}, E^{({\alpha})1},...,E^{({\alpha})n}\}{\subseteq} {\mathbb{R}}^{1+n}$ such that
\begin{equation} \label{3.11}
E^{{\alpha} i} \,= \, E^{\alpha} {\otimes} E^{({\alpha})i},\ \ \ \ {\alpha}\,=\,1,...,N,\ i\,=\,0,1,...,n.
\end{equation}
Moreover, ${\underline{\textbf{A}}}$ satisfies the lower bound estimate
\begin{equation} \label{3.12}
\exists\, c>0\ : \ \ \ \big|{\underline{\textbf{A}}} {\!:\!} \underline{X}\big|\, \geq\, c\, |\Pi\, \underline{X}|,\ \ \ \ \ \ \ \ \underline{X} \in {\mathbb{R}}^{N+Nn},\\
\end{equation}
and the invariance property under the projection on $\Pi$
\begin{equation} \label{3.13}
{\underline{\textbf{A}}} {\!:\!}   \underline{X} \, =\,  {\underline{\textbf{A}}} {\!:\!} (\Pi\, \underline{X}) , \ \ \ \ \ \underline{X} \in {\mathbb{R}}^{N+Nn}.
\end{equation}
Finally, the fibre Sobolev space associated to ${\textbf{A}}$ (given by \eqref{3.7}) satisfies the desired rank-one spanning property. 
\end{lemma}

{\medskip \noindent \textbf{Proof of Lemma} } \ref{lemma12}. We begin by observing that directly from the definitions of $\Pi$ and of $\Pi^\bot=N({\underline{\textbf{A}}})$, we have
\begin{equation} \label{3.14}
\left\{\ \ \ 
\begin{split}
\Pi^\bot &=\, \Big\{ \underline{X}=\big[\!-{\textbf{A}}{\!:\!} X \big| X \big] \ \Big| \ X\in {\mathbb{R}}^{Nn} \Big\} ,\\
\Pi\ &=\, \Big\{ \underline{Y}=[Y_0|Y]  \in {\mathbb{R}}^{N+Nn} \ \Big| \, Y_0 \cdot(\!-{\textbf{A}}\:X) \, +\, Y{\!:\!} X\,=\,0,\  X\in {\mathbb{R}}^{Nn} \Big\} .
\end{split}
\right.
\end{equation}
Next, by standard linear algebra results (\cite{L}) we obtain that the commutativity hypothesis of the (symmetric) matrices $\{{\textbf{A}}_1,...,{\textbf{A}}_n\}{\subseteq} {\mathbb{R}}^{N{\times} N}_s$ is equivalent to the requirement that there exists an orthonormal basis $\{\eta^1,...,\eta^N\}{\subseteq} {\mathbb{R}}^N$ which diagonalises all the matrices $\{{\textbf{A}}_1,...,{\textbf{A}}_n\}$ simultaneously, namely there is a common set of eigenvectors for perhaps different eigenvalues $\{c^{(i)1},...,c^{(i)N} \}$ of ${\textbf{A}}_i$. Thus, for any $i=1...n$, we have 
\[
{\textbf{A}}_i \, \eta^{\alpha} \ =\ c^{(i){\alpha}}\, \eta^{\alpha}, \ \ \ \ \ {\alpha}\,=\,1,...,N,
\]
or, in index form (see \eqref{1.3A})
\[
\sum_{{\gamma}=1}^N{\textbf{A}}_{{\beta} {\gamma} i}\, \eta^{\alpha}_{\gamma} \, =\, c^{(i){\alpha} }\, \eta_{\beta}^{\alpha},  \ \ \ \ \ {\alpha},{\beta}\,=\,1,...,N,
\] 
whereas ${\sigma}({\textbf{A}}_i)=\{c^{(i)1},...,c^{(i)N} \}$. We rewrite the above as
\[
\sum_{{\gamma}=1}^N \sum_{j=1}^n{\textbf{A}}_{{\beta} {\gamma} j}\, \big(\eta^{\alpha}_{\gamma} \, e^i_j \big)\, +\, \big(\!-c^{(i){\alpha} }\, \eta_{\beta}^{\alpha}\big)\, =\, 0,  \ \ \ \ \ {\alpha},{\beta}\,=\,1,...,N, \ i\,=\,1,...,n,
\] 
and in view of \eqref{1.3A} we may write it as
\begin{equation} \label{3.15}
{\textbf{A}}{\!:\!} \big(\eta^{\alpha} {\otimes} e^i\big)\, +\, \big(\!-c^{(i){\alpha} }\, \eta^{\alpha}\big)\, =\, 0  \ \ \ \ \ {\alpha}\,=\,1,...,N,\ i\,=\,1,...,n.
\end{equation} 
We now define
\begin{equation} \label{3.16}
N^{{\alpha} i}\, :=\, \eta^{\alpha} {\otimes} 
\left[
\begin{array}{c}
-c^{(i){\alpha}}\\
\hline e^i
\end{array}
\right]\, =\, \eta^{\alpha} {\otimes} \Big( \underset{\quad\quad \ \ \ \ \widehat{\text{(1+i)-position}}}{ \big[-c^{(i){\alpha}},0,...,0,1,0,...0\big]^\top}  \Big),
\end{equation}
for ${\alpha}=1,...,N$, $i=1,...,n$, and also
\begin{equation}    \label{3.17}
N^{{\alpha} 0}\, :=\, \eta^{\alpha} {\otimes} 
\left[
\begin{array}{c}
1\\
\hline c^{\alpha}
\end{array}
\right]\, =\, \eta^{\alpha} {\otimes} \Big(\big[1,c^{(1){\alpha}},...,c^{(n){\alpha}}\big]^\top\Big), 
\end{equation}
where
\[
c^{\alpha}\, :=\, \big[c^{(1){\alpha}},...,c^{(n){\alpha}}\big]^\top
\]
is the ${\alpha}$-th eigenvalue vector of the matrices $\{{\textbf{A}}_1,...,{\textbf{A}}_n\}$. The definition of $N^{{\alpha} i}$ and \eqref{3.15} with \eqref{3.1} immediately give that
\[
{\underline{\textbf{A}}} : N^{{\alpha} i}\, =\, 0, \ \ \ \ \ {\alpha}\,=\,1,...,N,\ i\,=\,1,...,n,
\]
and hence $N^{{\alpha} i} \in N({\underline{\textbf{A}}})=\Pi^\bot$. Moreover, by \eqref{3.14} and the fact that the $(Nn)$-many matrices $\{\eta^{\alpha} {\otimes} e^i \,| \, {\alpha},i\}$ are an orthonormal basis of ${\mathbb{R}}^{Nn}$, we have that
\[
\begin{split}
\underline{Y} \,  \in \, \Pi \   &\Longleftrightarrow \ \  Y_0 \cdot(\!-{\textbf{A}}\:X) \, +\, Y{\!:\!} X\,=\,0,\hspace{65pt}   X\in {\mathbb{R}}^{Nn}, \\
&\Longleftrightarrow \  \,  Y_0 \cdot\big(\!- \! {\textbf{A}}{\!:\!} (\eta^{\alpha} {\otimes} e^i)\big) \, +\, Y{\!:\!}(\eta^{\alpha} {\otimes} e^i)\,=\,0,\  {\alpha}=1,...,N,\ i=1,...,n,\\
&\overset{ \eqref{3.15} }{\Longleftrightarrow}  \   Y_0 \cdot\big( \!-c^{(i){\alpha}}\eta^{\alpha}\big) \, +\, Y{\!:\!}(\eta^{\alpha} {\otimes} e^i)\,=\,0,\ \ \ \ \ \ \  {\alpha}=1,...,N,\ i=1,...,n,\\
&\overset{ \eqref{3.16} }{\Longleftrightarrow}  \    [Y_0|Y] : N^{{\alpha} i}\,=\,0,\hspace{106pt}  {\alpha}=1,...,N,\ i=1,...,n.
\end{split}
\]
Hence, $\underline{Y} \, \bot\, N({\underline{\textbf{A}}})$ if and only if $\underline{Y} \, \bot\, N^{{\alpha} i}$ for all ${\alpha}=1,...,N$ and $i=1,...,n$. Since $N({\underline{\textbf{A}}})=\Pi^\bot$, this proves that
\begin{equation}   \label{3.18}
N({\underline{\textbf{A}}})\, =\, {\textrm{span}}[\Big\{ N^{{\alpha} i}\, \big|\, {\alpha}=1,...,N,\ i=1,...,n\Big\}].
\end{equation}
Moreover, the matrices $N^{{\alpha} i}$ spanning $N({\underline{\textbf{A}}})$ are linearly independent and hence exactly $Nn$-many. Indeed, for ${\alpha},{\beta}=1,...,N$ and $i,j=1,...,n$, by \eqref{3.16} we have
\[
\begin{split}
N^{{\alpha} i} : N^{{\beta} j} \, & =\,  \Big(\eta^{\alpha} {\otimes} 
\left[
\begin{array}{c}
-c^{(i){\alpha}}\\
\hline e^i
\end{array}
\right]  \Big)
 :   \Big(\eta^{\beta} {\otimes} 
\left[
\begin{array}{c}
-c^{(i){\beta}}\\
\hline e^i
\end{array}
\right]  \Big)
\\
&=\, {\delta}_{{\alpha} {\beta}}\left(c^{(i){\alpha}}c^{(j){\beta}} \, +\, {\delta}_{ij}\right).
\end{split}
\]
It follows that for any ${\alpha}\neq {\beta}$, $N^{{\alpha} i}$ is orthogonal to $N^{{\beta} j}$. Moreover, for any for ${\alpha}=1,...,N$ and $i\neq j$ in $\{1,...,n\}$, by \eqref{3.16} we have
\[
\begin{split}
\frac{N^{{\alpha} i}}{| N^{{\alpha} i} |} : \frac{N^{{\alpha} j}}{| N^{{\alpha} j} |}  \, & 
=\,  \frac{  c^{(i){\alpha}}c^{(j){\alpha}}  }{\sqrt{1+(c^{(i){\alpha}})^2} \sqrt{1+(c^{(j){\alpha}})^2} } \, \in \, (-1,+1)
\end{split}
\]
and hence for each ${\alpha}$ the set of matrices $\{N^{{\alpha} i}|\,i\}$ is linearly independent. Further, by \eqref{3.16},  \eqref{3.17} we have that
\[
\begin{split}
N^{{\alpha} 0} : N^{{\beta} i} \, & =\, \Big(\eta^{\alpha} {\otimes} 
\left[
\begin{array}{c}
1\\
\hline c^{\alpha}
\end{array}
\right]  \Big) :   \Big(\eta^{\beta} {\otimes} 
\left[
\begin{array}{c}
-c^{(i){\beta}}\\
\hline e^i
\end{array}
\right]  \Big)
\\
& =\, \big(\eta^{\alpha} \cdot \eta^{\beta}\big) \Big\{ \big[1,c^{(1){\alpha}},...,c^{(n){\alpha}}\big] \cdot  \underset{\quad\quad \ \ \ \ \ \ \widehat{\text{(1+i)-position}}}{ \big[-c^{(i){\beta}},0,...,0,1,0,...0\big] } \Big\}\\
&=\, {\delta}_{{\alpha} {\beta}} \left( -c^{(i){\beta}} + c^{(i){\alpha}}\right)\\
&=\, 0,
\end{split}
\]
for all ${\alpha},{\beta}=1,...,N$ and $i=1,...,n$. Moreover, by \eqref{3.17} we have
\[
\begin{split}
N^{{\alpha} 0} : N^{{\beta} 0}\, &=\, \left(\eta^{\alpha} {\otimes} 
\left[
\begin{array}{c}
1\\
\hline c^{\alpha}
\end{array}
\right]\right) : \left(\eta^{\beta} {\otimes} 
\left[
\begin{array}{c}
1\\
\hline c^{\beta}
\end{array}
\right] \right)\\
& =\, {\delta}_{{\alpha} {\beta}}\big(1\,+\,c^{\alpha} \cdot c^{\beta} \big)
\end{split}
\]
and as a consequence the matrices $\{N^{{\alpha}0}|\, {\alpha}\}$ form an orthogonal set of $N$-many elements which is orthogonal to $N({\underline{\textbf{A}}})$. Since the dimension of the space is $N+Nn$, all the above together with \eqref{3.14}, \eqref{3.16}, \eqref{3.17} prove that
\begin{equation}   \label{3.19}
\Pi \, =\,  {\textrm{span}}[\Big\{ N^{{\alpha} 0}\, \big|\, {\alpha}=1,...,N \Big\}].
\end{equation}
We now show that the frame $\{N^{{\alpha} i}|\, {\alpha},i\}$ can be modified in order to be made an orthonormal basis and still consisting of rank-one matrices. First note that the matrices spanning $\Pi$ are orthogonal and we only need to fix their length. Further, note that $\Pi^\bot$ can be decomposed as the following direct sum of mutually orthogonal subspaces
\[
\Pi^\bot=\ \bigoplus_{{\alpha}=1}^N \,{\textrm{span}}[\Big\{ N^{{\alpha} i}\, \big| \, i=1,...,n \Big\} ]\ =:\  \bigoplus_{{\alpha}=1}^N \,\mathbb{W}_{\alpha} .
\]
Since
\[
\mathbb{W}_{\alpha} \, =\, \eta^{\alpha} {\otimes} {\textrm{span}}[\left\{  \left[
\begin{array}{c}
-c^{(i){\alpha}}\\
\hline e^i
\end{array}
\right]  \, : \ i=1,...,n \right\} ] ,
\]
by the Gram-Schmidt method, we can find an orthonormal basis of $\mathbb{W}_{\alpha}$ consisting of matrices of the form
\begin{equation}
\tilde{N}^{{\alpha} i}\, =\ \eta^{\alpha} {\otimes} \underline{a}^{({\alpha})i},\ \ \ \ \ \underline{a}^{({\alpha})i}\cdot \underline{a}^{({\alpha})j}=\, {\delta}_{ij}. 
\end{equation}
Finally, we define
\[
\begin{split}
E^{{\alpha} 0}\, &:=\, \frac{N^{{\alpha} 0}}{|N^{{\alpha} 0}|}\ =\ \eta^{\alpha} {\otimes} \left( \frac{1}{\sqrt{1+|c^{\alpha}|^2}}     \left[
\begin{array}{c}
1\\
\hline c^{\alpha}
\end{array}
\right] \right) \ \in {\mathbb{R}}^{N+Nn},\\
E^{{\alpha} i}\, &:=\, \tilde{N}^{{\alpha} i}\, =\ \eta^{\alpha} {\otimes} \underline{a}^{({\alpha})i} \hspace{93pt} \in {\mathbb{R}}^{N+Nn},
\end{split}
\]
and also
\[
\begin{split}
E^{\alpha}\, &:=\  \eta^{\alpha}  \hspace{145pt} \in {\mathbb{R}}^N, \\
 E^{({\alpha})0}\, &:=\  \frac{1}{\sqrt{1+|c^{\alpha}|^2}}  \left[
\begin{array}{c}
1\\
\hline c^{\alpha}
\end{array}
\right],\ E^{({\alpha})i}\, :=\, \underline{a}^{({\alpha})i} \ \in {\mathbb{R}}^{1+n},
\end{split}
\]
where ${\alpha}=1,...,N$ and $i=1,...,n$. By the previous it follows that $\{E^{{\alpha} i}|\, {\alpha}=1,...,N,\,i=0,1,...,n\}$ is an orthonormal basis of ${\mathbb{R}}^{N+Nn}$ consisting or rank-one directions such that $\{E^{{\alpha} 0}|\, {\alpha}=1,...,N\}$ span the subspace $\Pi$ and $\{E^{{\alpha} i}|\, {\alpha}=1,...,N,\,i=1,...,n\}$ span its complement $\Pi^\bot$. Moreover, $E^{{\alpha} i}=E^{\alpha} {\otimes} E^{({\alpha})i}$. We conclude the proof of the lemma by noting that \eqref{3.12}, \eqref{3.13} follow by the definition of $\Pi$ and standard linear algebra results.     \qed

{\medskip}

Next, we employ the orthonormal frames constructed in Lemma \ref{lemma12} and the properties \eqref{3.12}, \eqref{3.13} of ${\textbf{A}}$ in order to characterise weak solutions to \eqref{3.8} as mappings in the fibre space \eqref{3.7} which solve the equation in a pointwise ``strong fibre-wise" sense: the equation is satisfied a.e.\ on $(0,T)\! {\times} {\mathbb{R}}^n $ if we substitute the distributional gradient $(D_t u, Du)$ (which can be interpreted only via duality) with the pointwise fibre gradient $\underline{G}(u)$. 

\begin{lemma} \label{lemma14} In the setting of Theorem \ref{theorem10} and under the same assumptions, we have that a map $u :  (0,T)\! {\times} \!{\mathbb{R}}^n {\longrightarrow} {\mathbb{R}}^N $ in the fibre space \eqref{3.7} (associated to ${\textbf{A}}$) satisfies 
\[
{\underline{\textbf{A}}} {\!:\!} \underline{G}(u)\, =\, f, \ \ \ \text{ a.e.\ on }(0,T){\times} {\mathbb{R}}^n
\]
(where ${\underline{\textbf{A}}}$ is given by \eqref{3.1} and $\underline{G}(u)$ is the fibre gradient of $u$) if and only if $u :  (0,T)\! {\times} \!{\mathbb{R}}^n {\longrightarrow} {\mathbb{R}}^N $ is a weak solution to \eqref{3.8} in the space $\mathfrak{X}$ given by \eqref{3.9}.
\end{lemma}

{\medskip \noindent \textbf{Proof of Lemma} } \ref{lemma14}. Suppose first that $u$ is a weak solution of \eqref{3.8}. By mollifying in the standard way, for any ${\varepsilon}>0$ there are $u^{\varepsilon},f^{\varepsilon} \in C^\infty\big(({\varepsilon},T-{\varepsilon})\!{\times}\!{\mathbb{R}}^n,{\mathbb{R}}^N\big)$ such that $u^{\varepsilon} {\longrightarrow} u$ and $f^{\varepsilon} {\longrightarrow} f$ as ${\varepsilon}{\rightarrow} 0$ in $L^2\big(({\delta},T-{\delta}){\times}{\mathbb{R}}^n,{\mathbb{R}}^N\big)$ for any ${\delta}\geq{\varepsilon}$ and also
\[
D_t u^{\varepsilon} \ +\ {\textbf{A}}\:Du^{\varepsilon}\, =\, f^{\varepsilon},\ \ \text{ on }({\delta},T-{\delta}){\times}{\mathbb{R}}^n.
\]
By \eqref{3.1}, \eqref{3.4},\eqref{3.5} and  \eqref{3.13}, we rewrite this as
\begin{equation} \label{3.21}
{\underline{\textbf{A}}}{\!:\!} \big(\Pi\, {\underline{D}} u^{\varepsilon}\big)\,  =\, f^{\varepsilon},\ \ \text{ on }({\delta},T-{\delta}){\times}{\mathbb{R}}^n,
\end{equation}
for any ${\delta}\geq {\varepsilon}>0$. Hence, by \eqref{3.12} this gives the estimate
\[
\big\| \Pi\, {\underline{D}} u^{\varepsilon} \big\|_{L^2(({\delta},T-{\delta}){\times}{\mathbb{R}}^n)}\, \leq\, C \|f\|_{L^2(({\delta},T-{\delta}){\times}{\mathbb{R}}^n)},
\]
which is uniform in ${\varepsilon},{\delta}>0$. By the definition  of the fibre space \eqref{3.7} and the above estimate together with the fact that $u^{\varepsilon}{\longrightarrow} u$ as ${\varepsilon}{\rightarrow}0$ in $L^2$, we obtain that $u\in \mathscr{W}^{1,2}\big((0,T){\times}{\mathbb{R}}^n,{\mathbb{R}}^N\big)$ and in addition  $ \Pi\, {\underline{D}} u^{\varepsilon}  {\longrightarrow} \underline{G}(u)$ in $L^2$. Thus, by passing to the limit in \eqref{3.21} as ${\varepsilon}{\rightarrow}0$ and as ${\delta}{\rightarrow}0$, we obtain that
\begin{equation} \label{3.22}
{\underline{\textbf{A}}}{\!:\!} \underline{G}(u)\, =\, f,\ \ \text{ a.e.\ on }(0,T){\times}{\mathbb{R}}^n,
\end{equation}
as desired.  Conversely, suppose that \eqref{3.22} holds. Then, by the definition \eqref{3.7} of the fibre space there are approximating sequences $u^\nu {\longrightarrow} u$ and $\Pi \, {\underline{D}} u^\nu {\longrightarrow} \underline{G}(u)$, both in $L^2$ as $\nu {\rightarrow} \infty$. Hence, we have
\[
\begin{split}
{\underline{\textbf{A}}}{\!:\!} \big(\Pi\, {\underline{D}} u^\nu\big)\, -\,f &=\,   {\textbf{A}}{\!:\!}\big(\Pi\,{\underline{D}} u^\nu-\underline{G}(u) \big)\\
&=\ o(1),
\end{split}
\]
as $\nu {\rightarrow} \infty$, in $L^2$. By the above, \eqref{3.13} and \eqref{3.1}-\eqref{3.5}, we have
\[
\begin{split}
D_tu^\nu\, +\ {\textbf{A}}\:Du^\nu\, -\, f\ &=\ {\underline{\textbf{A}}}{\!:\!}  {\underline{D}} u^\nu \, -\, f\\
&=\ {\underline{\textbf{A}}}{\!:\!} \big(\Pi\, {\underline{D}} u^\nu\big) \, -\, f\\
&=\ o(1),
\end{split}
\]
as $\nu {\rightarrow} \infty$, in $L^2$. Hence, for any $\phi \in C^\infty_c\big((0,T)\!{\times}{\mathbb{R}}^n\big)$, we have
\[
\int_{(0,T)\,{\times}\,{\mathbb{R}}^n} \Big\{u^\nu D_t\phi \, +\ {\textbf{A}}{\!:\!}\big(u^\nu {\otimes} D\phi\big)\, +\, f\Big\}\,=\ o(1),
\]
as $\nu {\rightarrow} \infty$. By passing to the limit, we deduce that $u$ is a weak solution of \eqref{3.8}, as claimed. The lemma ensues.     \qed

{\medskip}

Finally, we characterise ${\mathcal{D}}$-solutions to \eqref{3.8} in the fibre space \eqref{3.7} as mappings which solve the equation in the pointwise ``strong fibre-wise" sense of Lemma \ref{lemma14}. This result completes the proof of Theorem \ref{theorem10}.

\begin{lemma} \label{lemma13} In the setting of Theorem \ref{theorem10} and under the same assumptions, we have that a map $u :  (0,T)\! {\times} \!{\mathbb{R}}^n {\longrightarrow} {\mathbb{R}}^N $ is a ${\mathcal{D}}$-solution to \eqref{3.8} in the fibre space \eqref{3.7} (associated to ${\textbf{A}}$) if and only if it satisfies 
\[
{\underline{\textbf{A}}} {\!:\!} \underline{G}(u)\, =\, f, \ \ \ \text{ a.e.\ on }(0,T){\times} {\mathbb{R}}^n, 
\]
where ${\underline{\textbf{A}}}$ is given by \eqref{3.1} and $\underline{G}(u)$ is the fibre gradient of $u$.
\end{lemma}

{\medskip \noindent \textbf{Proof of Lemma} } \ref{lemma13}. We begin by supposing that ${\underline{\textbf{A}}} {\!:\!} \underline{G}(u)=f$ a.e.\ on $(0,T) {\times}  {\mathbb{R}}^n$. Then, by the properties of the fibre space \eqref{3.7} and the equivalence between weak and strong $L^2$ directional derivatives, we have that
\[
{\underline{D}}_{ \underline{a} }^{1,h}(\eta\cdot u) {\longrightarrow} (\eta {\otimes} \underline{a}) {\!:\!} \underline{G}(u), \ \ \ \ \text{ in }L^2\big( (0,T){\times} {\mathbb{R}}^n \big) \text{ as }\, h{\rightarrow}0,
\] 
for any rank-one direction $\eta {\otimes} \underline{a} \in \Pi {\subseteq} {\mathbb{R}}^{N+Nn}$. By Lemma \ref{lemma12}, $\Pi$ has an orthonormal basis consisting of rank-one matrices which can be complemented to an orthonormal basis of rank-one matrices of ${\mathbb{R}}^{N+Nn}$. Thus, in view of \eqref{2.1}-\eqref{2.4} we have
\begin{equation} \label{3.20}
\Pi \, {\underline{D}}^{1,h}u {\longrightarrow} \underline{G}(u), \ \ \ \ \text{ in }L^2\big( (0,T){\times} {\mathbb{R}}^n,\Pi \big) \text{ as }\, h{\rightarrow}0,
\end{equation} 
where ${\underline{D}}^{1,h}$ denotes the difference quotient operator taken with respect to these bases. By \eqref{3.20} and \eqref{3.13} we obtain that
\begin{equation} \label{3.21a}
{\underline{\textbf{A}}} {\!:\!}  {\underline{D}}^{1,h}u   {\longrightarrow} {\underline{\textbf{A}}}{\!:\!} \underline{G}(u), \ \ \ \ \text{ in }L^2\big( (0,T){\times} {\mathbb{R}}^n,{\mathbb{R}}^N \big) \text{ as }\, h{\rightarrow}0.
\end{equation}
Further, for any fixed measurable set $E{\subseteq}  (0,T){\times} {\mathbb{R}}^n$ with  finite measure and any $\Phi\in C^0_c({\mathbb{R}}^{N+Nn})$, by using our hypothesis $ {\underline{\textbf{A}}} : \underline{G}(u)=f$, we have the estimate
\begin{equation} \label{3.21ab}
\begin{split}
\Big\| \Phi\big({\underline{D}}^{1,h}u\big) & \Big( {\underline{\textbf{A}}} {\!:\!} {\underline{D}}^{1,h}u \,-\, f \Big) \Big\|_{L^1(E)}\\ 
&\leq\, 
\sqrt{|E|}\,\|\Phi\|_{C^0({\mathbb{R}}^{N+Nn})} \, 
\Big\|  {\underline{\textbf{A}}} {\!:\!} {\underline{D}}^{1,h}u \,-\,  {\underline{\textbf{A}}}{\!:\!}\underline{G}(u)\Big\|_{L^2((0,T){\times} {\mathbb{R}}^n)}.
\end{split}
\end{equation}
Hence, \eqref{3.21a} and \eqref{3.21ab} imply
\begin{equation} \label{3.22a}
\Phi\big({\underline{D}}^{1,h}u\big)\Big( {\underline{\textbf{A}}} {\!:\!} {\underline{D}}^{1,h}u \,-\, f \Big){\longrightarrow} 0, \ \ \ \ \text{ in }L^1 ( E,{\mathbb{R}}^N ) \text{ as }\, h{\rightarrow}0.
\end{equation}
Moreover, the Carath\'eodory  function
\begin{equation} \label{3.23a}
\Psi(\underline{x},\underline{X})\, :=\,  \Big|\Phi\big( \underline{X} \big)\Big( {\underline{\textbf{A}}} {\!:\!} \underline{X}\,-\, f(\underline{x}) \Big) \Big| \, \chi_{E}(\underline{x})
\end{equation}
is an element of the space
\[
L^1\Big((0,T)\!{\times} {\mathbb{R}}^n,C^0\big( \smash{\overline{\mathbb{R}}}^{N+Nn} \big)\Big)
\]
because
\[
\begin{split}
\|\Psi \|_{L^1 \big((0,T){\times} {\mathbb{R}}^n,C^0 ( \smash{\overline{\mathbb{R}}}^{N+Nn} )\big)} 
&\leq\, 
|E|\, \Bigg(\max_{\underline{X} \in {\textrm{supp}}(\Phi)}\big| \Phi\big( \underline{X} \big)  {\underline{\textbf{A}}} {\!:\!} \underline{X} \big|\Bigg)\\
&\ \ \ +\, \sqrt{|E|}\, \Bigg(\max_{\underline{X} \in {\textrm{supp}}(\Phi)}\big| \Phi\big( \underline{X}\big) \big|\Bigg)\, \|f\|_{L^2((0,T){\times} {\mathbb{R}}^n)}.
\end{split}
\]
Let now $(h_\nu)_1^\infty {\subseteq} {\mathbb{R}}{\setminus}\{0\}$ be any infinitesimal sequence. Then, there is a subsequence $h_{\nu_k}{\rightarrow} 0$ such that
\begin{equation} \label{3.24b}
{\delta}_{{\underline{D}}^{1,h_{\nu_k}}u} \, \overset{^*}{\smash{-\!\!\!\!-\!\!\!\!\rightharpoonup}}\,\, \underline{\mathcal{D}} u\ \ \ \text{ in }\, {\mathscr{Y}}\Big((0,T){\times} {\mathbb{R}}^n, \, \smash{\overline{\mathbb{R}}}^{N+Nn}\Big), \ \ \text{ as }k {\rightarrow} \infty.
\end{equation}
By the weak*-strong continuity of the duality pairing between 
\[
L^1\Big((0,T) \!{\times} {\mathbb{R}}^n,C^0\big( \smash{\overline{\mathbb{R}}}^{N+Nn} \big)\Big) \!{\times} L^\infty_{w*}\Big((0,T)\!{\times} {\mathbb{R}}^n,{\mathcal{M}}\big( \smash{\overline{\mathbb{R}}}^{N+Nn} \big)\Big) {\longrightarrow} {\mathbb{R}}
\]
and by \eqref{3.22a}-\eqref{3.24b}, we have that
\begin{equation} \label{3.25a}
\begin{split}
\int_E  \Big|\Phi\big( {\underline{D}}^{1,h_{\nu_k}}u\big)\Big( {\underline{\textbf{A}}} {\!:\!} {\underline{D}}^{1,h_{\nu_k}}u\,-\, f \Big) \Big|&\,=\, \int_E \Psi \big(\cdot,{\underline{D}}^{1,h_{\nu_k}}u\big)\\
&\!\!{\longrightarrow}  \int_E \int_{ \smash{\overline{\mathbb{R}}}^{N+Nn} } \Psi \big(\cdot, \underline{X} \big)\, d[\underline\mDu ]( \underline{X} ) \\
&=\, \int_E  \int_{ \smash{\overline{\mathbb{R}}}^{N+Nn} }  \Big|\Phi\big( \underline{X} \big)\Big( {\underline{\textbf{A}}} {\!:\!}  \underline{X}\,-\, f \Big) \Big|  \, d[\underline\mDu ]( \underline{X} ),
\end{split}
\end{equation}
as $k{\rightarrow} \infty$. By \eqref{3.25a} and \eqref{3.22a} we conclude that
\[
\int_{ \smash{\overline{\mathbb{R}}}^{N+Nn} }  \Big|\Phi\big( \underline{X} \big)\Big( {\underline{\textbf{A}}} {\!:\!}  \underline{X}\,-\, f(\underline{x}) \Big) \Big|  \, d\big[\underline\mDu(\underline{x})\big]( \underline{X} )\,=\,0,\ \ \ \text{ a.e.\ }\underline{x}\in E.
\]
Since $E{\subseteq} (0,T) \!{\times} {\mathbb{R}}^n$ is an  arbitrary set of finite measure, $\Phi$ is an arbitrary function in $C^0_c({\mathbb{R}}^{N+Nn})$ and $\underline{\mathcal{D}} u$ an arbitrary diffuse gradient as in \eqref{3.24b}, it follows that $u$ is a ${\mathcal{D}}$-solutions of \eqref{3.8} on $(0,T) \!{\times} {\mathbb{R}}^n$, as desired.

{\medskip}

Conversely, suppose that $u$ is a ${\mathcal{D}}$-solution of \eqref{3.8} in the fibre space \eqref{3.7}. Then, for any diffuse gradient as in \eqref{3.24b} and any $\Phi \in C^0_c({\mathbb{R}}^{N+Nn})$, it follows that
\begin{equation} \label{3.26a}
\int_{ \smash{\overline{\mathbb{R}}}^{N+Nn} }   \Phi\big( \underline{X} \big)   \, d\big[\vartheta (\underline{x})\big]( \underline{X} )\,=\,0,\ \ \ \text{ a.e.\ }\underline{x}\in (0,T) \!{\times} {\mathbb{R}}^n,
\end{equation}
where for a.e.\ $\underline{x}\in (0,T) {\times} {\mathbb{R}}^n$, $\vartheta (\underline{x})$ is a locally finite Radon measure which is absolutely continuous with respect to the restriction of the diffuse gradient measure $\big[ \underline\mDu(\underline{x}) \big]{\text{\LARGE$\llcorner$}} {\mathbb{R}}^{N+Nn}$  and it is given by
\begin{equation} \label{3.27a}
\begin{split}
\vartheta (\underline{x})\, &\in \, {\mathcal{M}}_{\text{loc}}\big({\mathbb{R}}^{N+Nn}\big) ,\\
\big[\vartheta (\underline{x})\big](B)\, :=\, \int_{B}&\Big( {\underline{\textbf{A}}} {\!:\!}  \underline{X}\,-\, f(\underline{x}) \Big)\, d\big[\underline\mDu(\underline{x})\big]( \underline{X} ),
\end{split}
\end{equation}
for any Borel set $B{\subseteq} {\mathbb{R}}^{N+Nn}$. From \eqref{3.25a} and \eqref{3.27a} it follows that $\vartheta (\underline{x})=0$ for a.e.\ $\underline{x}\in (0,T) {\times} {\mathbb{R}}^n$. This implies that for any such point, the support of the measure $\big[ \underline\mDu(\underline{x}) \big]{\text{\LARGE$\llcorner$}} {\mathbb{R}}^{N+Nn}$ lies inside the closed set
\[
\mathscr{L}_{ \underline{x} }\, :=\, \Big\{ \underline{X} \in {\mathbb{R}}^{N+Nn}\ \Big| \ {\underline{\textbf{A}}}{\!:\!}\underline{X}\,=\,f(\underline{x}) \Big\}.
\]
Since
\[
\mathscr{L}_{ \underline{x} }\, :=\, \Big\{ \underline{X} \in {\mathbb{R}}^{N+Nn}\ \Big| \ \big|{\underline{\textbf{A}}}{\!:\!}\underline{X}\,-\,f(\underline{x})\big|\,=\,0 \Big\}
\]
and $\Phi $ has compact support in ${\mathbb{R}}^{N+Nn}$, it follows that 
\begin{equation} \label{3.28a}
\int_{ \smash{\overline{\mathbb{R}}}^{N+Nn} }  \big|\Phi\big( \underline{X} \big)\big|\, \Big| {\underline{\textbf{A}}} {\!:\!}  \underline{X}\,-\, f(\underline{x})  \Big|  \, d\big[\underline\mDu(\underline{x})\big]( \underline{X} )\,=\,0,
\end{equation}
for a.e.\ $\underline{x}\in (0,T)\! {\times} {\mathbb{R}}^n$. By considering again the function $\Psi$ of \eqref{3.23a} and invoking \eqref{3.24b}-\eqref{3.25a} and \eqref{3.28a}, we deduce that
\begin{equation}  \label{3.29a}
\lim_{k{\rightarrow} \infty} \int_E  \Big|\Phi\big( {\underline{D}}^{1,h_{\nu_k}}u(\underline{x})\big)\Big( {\underline{\textbf{A}}} {\!:\!} {\underline{D}}^{1,h_{\nu_k}}u(\underline{x})\,-\, f(\underline{x}) \Big) \Big|\,d\underline{x}\, =\, 0.
\end{equation}
We now fix $R>0$ and choose $\Phi  \geq \chi_{\overline{\mathbb{B}}_R(0)}$, where $\overline{\mathbb{B}}_R(0)$ is the closed $R$-ball of ${\mathbb{R}}^{N+Nn}$ centred at the origin. Then,  \eqref{3.29a} gives
\begin{equation}  \label{3.30a}
\lim_{k{\rightarrow} \infty} \int_{E\cap \big\{ \big| {\underline{D}}^{1,h_{\nu_k}}u  \big|\leq R\big\}}  \Big|  {\underline{\textbf{A}}} {\!:\!} {\underline{D}}^{1,h_{\nu_k}}u(\underline{x})\,-\, f (\underline{x}) \Big|\,d\underline{x}\, =\, 0,
\end{equation}
for any $R>0$. We now set
\[
E^R\,:=\, \Big\{ \underline{x} \in (0,T) {\times} {\mathbb{R}}^n \ \Big|\ \mathscr{L}_{\underline{x}} \cap \overline{\mathbb{B}}_R(0) \neq \,\emptyset \Big\}
\]
and 
\begin{equation}   \label{3.31a}
\text{T}^R\big(\underline{x},\underline{X}\big)\, :=\, 
\left\{
\begin{split}
& \underline{X}, \ \ \ \ \ \, \text{ for } \big|\underline{X} \big|\leq R ,\ \underline{x} \in E^R\\
& \underline{O}(\underline{x}),\ \  \text{ for } \big|\underline{X} \big|> R ,\ \underline{x} \in E^R,
\end{split}
\right.
\end{equation}
where $\underline{x} \mapsto \underline{O}(\underline{x} )$ is a measurable selection of the set-valued mapping with closed non-empty values
\[
E^R \, \ni\ \underline{x} \, {\longmapsto} \,\mathscr{L}_{\underline{x}} \cap \overline{\mathbb{B}}_R(0) \ {\subseteq} \, {\mathbb{R}}^{N+Nn}.
\]
This means that 
\[
{\underline{\textbf{A}}} {\!:\!} \underline{O}(\underline{x})\,=\, f (\underline{x}) \ \text{ and} \ \ \big| \underline{O}(\underline{x})\big|\leq R, \ \ \ \text{ a.e.\ }\underline{x}\in E^R.
\]
Such selections exist for large enough $R>0$ by Aumann's measurable selection theorem (see e.g.\ \cite{FL}), but in this specific case they can also be constructed explicitly because of the simple structure of the multi-valued mapping. By using 
\eqref{3.31a}, \eqref{3.30a} implies that
\[
\lim_{k{\rightarrow} \infty} \int_{E^R }  \Big|  {\underline{\textbf{A}}} : \text{T}^R\big(\underline{x},{\underline{D}}^{1,h_{\nu_k}}u(\underline{x})\big)\,-\, f (\underline{x}) \Big|\,d\underline{x}\, =\, 0 
\]
and by recalling \eqref{3.13}, we rewrite this as
\begin{equation}   \label{3.32}
\lim_{k{\rightarrow} \infty} \int_{E^R }  \Big|  {\underline{\textbf{A}}} : \text{T}^R\Big(\underline{x},\Pi\, {\underline{D}}^{1,h_{\nu_k}}u(\underline{x})\Big)\,-\, f (\underline{x}) \Big|\,d\underline{x}\, =\, 0.
\end{equation}
Hence, \eqref{3.32} implies that
\[
\begin{split}
\int_{E^R }  \Big|  {\underline{\textbf{A}}} : \underline{G}(u) \,-\, f  \Big| \
&\leq \, \int_{E^R }  \Big|  {\underline{\textbf{A}}} : \text{T}^R\Big(\cdot,\Pi\,   {\underline{D}}^{1,h_{\nu_k}}u \Big)\,-\, f   \Big|  \\ 
&\ \ \ \ + \,\int_{E^R }  \Big|  {\underline{\textbf{A}}} : \text{T}^R\Big(\cdot,\Pi\, {\underline{D}}^{1,h_{\nu_k}}u \Big)\,-\,  {\underline{\textbf{A}}} : \underline{G}(u)  \Big|  \\
& \leq \, o(1)\ +\ |{\underline{\textbf{A}}} |\int_{E^R }  \Big|   \text{T}^R\Big(\cdot,\Pi\, {\underline{D}}^{1,h_{\nu_k}}u \Big)\,-\,  \underline{G}(u) \Big| 
\end{split}
\]
as $k{\rightarrow} \infty$, and as a consequence we have
\begin{equation}      \label{3.33}
\begin{split}
\int_{E^R }  \Big|  {\underline{\textbf{A}}} : \underline{G}(u) \,-\, f  \Big| \
& \leq \,  |{\underline{\textbf{A}}} |\int_{E^R }  \Big|   \text{T}^R\Big(\cdot,\Pi\, {\underline{D}}^{1,h_{\nu_k}}u \Big)\,-\, \text{T}^R\big(\cdot,\underline{G}(u)\big) \Big|  \\
&\ \ \ \ + |{\underline{\textbf{A}}} |\int_{E^R }  \Big|   \text{T}^R\big(\cdot,\underline{G}(u)  \big)\,-\,  \underline{G}(u) \Big|\ +\ o(1),
\end{split}
\end{equation}
as $k{\rightarrow} \infty$, for any $R>0$. Moreover, by assumption $u$ is in the fibre space \eqref{3.7}. Hence by invoking \eqref{3.21a}, the Dominated convergence theorem, the fact that $|E|<\infty$ and \eqref{3.31a}, we may pass to the limit in \eqref{3.33} as $k{\rightarrow} \infty$ to obtain
\[
\int_{E^R }  \Big|  {\underline{\textbf{A}}} : \underline{G}(u) \,-\, f  \Big| \, \leq \, \ |{\underline{\textbf{A}}} |\int_{E^R }  \Big|   \text{T}^R\big(\cdot,\underline{G}(u)  \big)\,-\,  \underline{G}(u) \Big|,
\]
for any $R>0$. Finally, we let $R{\rightarrow} \infty$ and recall the arbitrariness of the set $E{\subseteq} (0,T){\times}{\mathbb{R}}^n$ and \eqref{3.31a} to infer that $ {\underline{\textbf{A}}} : \underline{G}(u) =f$, a.e.\ on $ (0,T){\times}{\mathbb{R}}^n$. The lemma has been established.     \qed

{\medskip}

The proof of Theorem \ref{theorem10} is now complete.             \qed

{\medskip}

\begin{remark}[Functional representation of the diffuse gradients] \label{remark15} In a sense, Lemma \ref{lemma13} says that all the diffuse gradients of the ${\mathcal{D}}$-solution $u$ when restricted on the subspace of non-degeneracies have a certain ``functional" representation \emph{inside the coefficients}, given by $\underline{G}(u)$. Namely, if we decompose ${\mathbb{R}}^{N+Nn}=\Pi \oplus \Pi^\bot$, the restriction of any diffuse space-time gradient $\underline\mDu \in {\mathscr{Y}}\big({\Omega}, \smash{\overline{\mathbb{R}}}^{N+Nn}\big)$ on $\Pi$ is given by the fibre space-time gradient:
\[
\underline\mDu(t,x)\,{\text{\LARGE$\llcorner$}} \, \Pi\  =\ {\delta}_{\,\underline{G}(u)(t,x)},\ \ \ \text{ a.e.\ }(t,x) \in (0,T){\times} {\mathbb{R}}^n.  
\]
This is a statement of ``partial regularity type" for ${\mathcal{D}}$-solutions: although not all of the diffuse gradient is a Dirac mass, certain restrictions of it on subspaces are concentration measures.
\end{remark}

{\medskip}

{\noindent} \textbf{Acknowledgement.} I would like to thank Tristan Pryer for our inspiring scientific discussions.

\bibliographystyle{amsplain}
\begin{thebibliography}{30}

\bibitem[A1]{A1} G. Aronsson, \emph{Minimization problems for the functional $sup_x F(x,
f(x), f'(x))$}, Arkiv f\"ur Mat. 6 (1965), 33 - 53.

\bibitem[A2]{A2} G. Aronsson, \emph{Minimization problems for the functional $sup_x F(x,
f(x), f'(x))$ II}, Arkiv f\"ur Mat. 6 (1966), 409 - 431.

\bibitem[A3]{A3} G. Aronsson, \emph{Extension of functions satisfying Lipschitz conditions}, Arkiv f\"ur Mat. 6 (1967), 551 - 561.

\bibitem[A4]{A4} G. Aronsson, \emph{On the partial differential equation $u_x^2 u_{xx} + 2u_x u_y u_{xy} + u_y^2 u_{yy} = 0$}, Arkiv f\"ur Mat. 7
(1968), 395 - 425.

\bibitem[A5]{A5} G. Aronsson, \emph{Minimization problems for the functional $sup_x F(x,
f(x), f'(x))$ III}, Arkiv f\"ur Mat. (1969), 509 - 512.

\bibitem[CFV]{CFV} C. Castaing, P. R. de Fitte, M. Valadier, \emph{Young Measures on Topological spaces with Applications in Control Theory and Probability Theory}, Mathematics and Its Applications, Kluwer Academic Publishers, 2004.

\bibitem[Co]{Co} J.F. Colombeau, \emph{New Generalized Functions and Multiplication of distributions}, North Holland, 1983.

\bibitem[D]{D} B. Dacorogna,  \emph{Direct Methods in the Calculus of Variations}, $2$nd Edition, Volume 78, Applied Mathematical Sciences, Springer, 2008.

\bibitem[DM]{DM} B. Dacorogna,  P. Marcellini, \emph{Implicit Partial Differential Equations}, Progress in Nonlinear Differential Equations and Their Applications, Birkh\"auser, 1999.

\bibitem[DPM]{DPM} R.J. DiPerna, A.J. Majda,  \emph{Oscillations and concentrations in weak solutions of the incompressible fluid equations}, Commun. Math. Phys. 108, 667 - 689 (1987).

\bibitem[Ed]{Ed} R.E. Edwards, \emph{Functional Analysis: Theory and Applications}, Dover Books on Mathematics,  2003.

\bibitem[E]{E} L.C. Evans, \emph{Weak convergence methods for nonlinear partial differential equations}, Regional conference series in mathematics 74, AMS,  1990.

\bibitem[E2]{E2} L.C. Evans, \emph{Partial Differential Equations}, AMS, Graduate Studies in Mathematics Vol. 19, 1998.

\bibitem[EG]{EG} L.C. Evans, R. Gariepy, \emph{Measure theory and fine properties of functions}, Studies in advanced mathematics, CRC press, 1992.

\bibitem[FG]{FG} L.C. Florescu, C. Godet-Thobie, \emph{Young measures and compactness in metric spaces}, De Gruyter, 2012.

\bibitem[FL]{FL} I. Fonseca, G. Leoni, \emph{Modern methods in the Calculus of Variations: $L^p$ spaces}, Springer Monographs in Mathematics, 2007.

\bibitem[K]{K} N. Katzourakis, \emph{An Introduction to viscosity Solutions for Fully Nonlinear PDE with Applications to Calculus of Variations in $L^\infty$}, Springer Briefs in Mathematics, 2015, DOI 10.1007/978-3-319-12829-0.

\bibitem[K1]{K1} N. Katzourakis,  \emph{$L^\infty$-Variational Problems for Maps and the Aronsson PDE system}, J.\ Differential Equations, Volume 253, Issue 7 (2012), 2123 - 2139.

\bibitem[K2]{K2} N. Katzourakis,  \emph{Explicit $2D$ $\infty$-Harmonic Maps whose Interfaces have Junctions and Corners}, Comptes Rendus Acad. Sci. Paris, Ser.I, 351 (2013) 677 - 680.

\bibitem[K3]{K3} N. Katzourakis,  \emph{On the Structure of $\infty$-Harmonic Maps}, Communications in PDE, Volume 39, Issue 11 (2014), 2091 - 2124.

\bibitem[K4]{K4} N. Katzourakis, \emph{$\infty$-Minimal Submanifolds}, Proceedings of the Amer. Math. Soc., 142 (2014) 2797-2811.

\bibitem[K5]{K5} N. Katzourakis,  \emph{Nonuniqueness in Vector-valued Calculus of Variations in $L^\infty$ and some Linear Elliptic Systems}, Communications on Pure and Applied Analysis,  Vol. 14, 1, 313 - 327 (2015). 

\bibitem[K6]{K6} N. Katzourakis,   \emph{Optimal $\infty$-Quasiconformal Immersions}, ESAIM Control, Opt. and Calc. Var., to appear (2015) DOI: http://dx.doi.org/10.1051/cocv/2014038. 

\bibitem[K7]{K7} N. Katzourakis,  \emph{On Linear Degenerate Elliptic PDE Systems with Constant Coefficients}, Adv. in Calculus of Variations, DOI: 10.1515/acv-2015-0004, published online June 2015.

\bibitem[K8]{K8} N. Katzourakis,  \emph{Generalised solutions for fully nonlinear PDE systems and existence-uniqueness theorems}, ArXiv preprint, \url{http://arxiv.org/pdf/1501.06164.pdf}. 

\bibitem[K9]{K9} N. Katzourakis,  \emph{Existence of generalised solutions to the equations of vectorial Calculus of Variations in $L^\infty$}, ArXiv preprint, \url{http://arxiv.org/pdf/1502.01179.pdf}. 

\bibitem[L]{L} P. D. Lax, \emph{Linear Algebra and Its Applications}, Wiley-Interscience, 2nd edition, 2007.

\bibitem[KR]{KR} J. Kristensen, F. Rindler, \emph{Characterization of generalized gradient Young measures generated by sequences in $W^{1,1}$ and $BV$}, Arch. Rational Mech. Anal. 197, 539 - 598 (2010) and \emph{erratum} Arch. Rational Mech. Anal. 203, 693 - 700 (2012).

\bibitem[M]{M} S. M\"uller, \emph{Variational models for microstructure and phase transitions}, Lecture Notes in Mathematics 1783, Springer, 85-210, 1999.

\bibitem[P]{P} P. Pedregal, \emph{Parametrized Measures and Variational Principles}, Birkh\"auser, 1997.

\bibitem[V]{V} M. Valadier, \emph{Young measures}, in ``Methods of nonconvex analysis", Lecture Notes in Mathematics 1446, 152-188 (1990).

\bibitem[Y]{Y} L.C. Young, \emph{Generalized curves and the existence of an attained absolute minimum in the calculus of variations}, Comptes Rendus de la Societe des Sciences et des Lettres de Varsovie, Classe III 30, 212 - 234 (1937).

\end{thebibliography}

\end{document}

