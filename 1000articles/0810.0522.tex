

\documentclass[regno,12pt]{amsart}
\usepackage{amsmath,amscd,amssymb,amsthm,hyperref,bm}

\newtheorem{theorem}{Theorem}[section]
\newtheorem{lemma}[theorem]{Lemma}
\newtheorem{corollary}[theorem]{Corollary}

\theoremstyle{definition}
\newtheorem{assumption}[theorem]{Assumption}
\newtheorem{definition}[theorem]{Definition}

\theoremstyle{remark}
\newtheorem{remark}[theorem]{Remark}
\newtheorem{example}[theorem]{Example}

\title[Boundary estimates for positive solutions]{Boundary estimates for positive solutions to second order elliptic equations}

\thanks{The work  was partially supported by NSF Grant DMS-9971052}
\author{Mikhail V. Safonov}

\address{School of Mathematics, University of Minnesota}
\email{safonov@math.umn.edu}
\keywords{Second-order elliptic equations, measurable coefficients, boundary Hopf lemma, estimates for ratios of solutions}

\subjclass[2000]{35J15, 35J67 (Primary). 35B05 (Secondary)}

\begin{document}

\begin{abstract}
Consider positive solutions to second order elliptic equations with measurable coefficients in a bounded domain, which vanish on a portion of the boundary. We give simple necessary and sufficient geometric conditions on the domain, which guarantee the Hopf-Oleinik type estimates and the boundary Lipschitz estimates for solutions. These conditions are sharp even for harmonic functions.
\end{abstract}

\maketitle

{\section{{Introduction. Formulation of main results}}
\setcounter{equation}{0}}\label{S.1}

Let $\Omega$ be a bounded open set in ${{\mathbb R}^n}$. Consider a second order elliptic operator
    \begin{equation}\label{1.1}
        Lu:= \sum_{i,j} a_{ij}D_{ij}u + \sum_i b_i D_iu
    \end{equation}
in $\Omega$, where $D_iu:={\partial} u/{\partial} x_i,\; D_{ij}u:=D_iD_j u,\; a_{ij}=a_{ji}\in L^{\infty}({{\mathbb R}^n}),\;
b_i\in L^{\infty}({{\mathbb R}^n})$, and $a_{ij}$ satisfy the {\textit{{uniform ellipticity condition}}}
    \begin{equation}\label{1.2}
        \nu |\xi|^2\le \sum_{i,j}a_{ij}\xi_i\xi_j\le \nu^{-1}|\xi|^2
        {\quad\text{{for all}}\quad} \xi=(\xi_1,\ldots,\xi_n)\in{{\mathbb R}^n},
    \end{equation}
with a constant $\nu\in (0,1]$. In 1952, E. Hopf \cite{H52} and O.A. Oleinik \cite{O52} independently proved the following {\textit{{boundary point lemma}}}.

\begin{lemma}\label{L1.1}
    Suppose that $\Omega$ satisfies an \emph{interior sphere condition} at $x_0\in{\partial\Omega}$, i.e. there exists a ball
    \[B:=B_{r_0}(y_0):=\{x\in{{\mathbb R}^n}:\;|x-y_0|<r_0\}{\subset}\Omega,\]
     with $x_0\in ({\partial\Omega})\cap ({\partial} B)$. Then for any function $u\in C^2(\Omega)\cap C({\overline{\Omega}})$ satisfying $\,u>0,\;Lu\le 0\,$ in $\Omega$, and $u(x_0)=0$, we have
    \begin{equation}\label{1.3}
        \liminf_{t\to 0^+} \frac{u(x_0+t{\bm{{l}}})}{t}>0.
    \end{equation}
     where ${\bm{{l}}}$ is an arbitrary interior vector to $B$ at  the point $x_0$, which means $x_0+t{\bm{{l}}}\in B$ for all $t$ in an interval $(0,t_0)$.
\end{lemma}

In a particular case when $L=\Delta$ - the Laplacian, this result was established in 1910 by M.S. Zaremba \cite{Z10}. In the beginning of 1930s, G. Giraud \cite{G33} has got a similar result for domains $\Omega$ with the boundary ${\partial\Omega}\in C^{1,\alpha},\,0<\alpha<1$, and operators $L$ with coefficients satisfying some continuity assumptions. See bibliographical notes in \cite{PW}, Ch. 2, and \cite{GT}, Ch. 3, for early references on this subject.
\smallskip

On the other hand, it is well known (see, e.g. \cite{CH}, IV.7.3) that an \emph{exterior sphere condition} at $x_0\in{\partial\Omega}$, together with the boundary condition $u=0$ near $x_0$, guarantees the boundedness of the ratio $u(x)/|x-x_0|$ in $\Omega$. In a ``model'' case, this property can be formulates as follows.

\begin{lemma}\label{L1.2}
     Suppose that $\Omega$ satisfies an \emph{exterior sphere condition} at a point $x_0\in{\partial\Omega}$, i.e. there exists a ball
     $B:=B_{r_0}(y_0)$, such that $\Omega\cap B={\emptyset}$, and $x_0\in ({\partial\Omega})\cap ({\partial} B)$.
     Let $u\in C^2(\Omega)\cap C({\overline{\Omega}})$ satisfy $\,u>0,\;Lu\ge 0\,$ in $\Omega$, and
     \[ u=0 {\quad\emph{{on}}\quad} ({\partial\Omega})\cap(B_{{\varepsilon}_0}(x_0),
     {\quad\text{{where}}\quad} {\varepsilon}_0={{\rm const}}>0.\]
     Then
     \begin{equation}\label{1.4}
        \sup_{\Omega} \frac{u(x)}{|x-x_0|}<{\infty}.
     \end{equation}
\end{lemma}

The proofs of Lemmas \ref{L1.1} and \ref{L1.2} and their generalizations are usually based on the classical \emph{comparison principle} (\cite{GT}, Theorem 3.3).

\begin{theorem}[Comparison principle]\label{T1.3}
    Let $\Omega$ be a bounded open set in ${{\mathbb R}^n}$, and let $u_1,u_2$ be functions in $C^2(\Omega)\cap C({\overline{\Omega}})$ satisfying $Lu_1\ge Lu_2$ in $\Omega$, and $u_1\le u_2$ on ${\partial\Omega}$. Then $u_1\le u_2$ in $\Omega$.
\end{theorem}

We give short proofs of Lemmas \ref{L1.1} and \ref{L1.2}, which contain some elements of the proofs of our main results, Theorems \ref{T1.8}  and \ref{T1.9}. For this purpose, we need the following elementary lemma, which will also be useful later, in the proof of Lemma \ref{L2.3}.

\begin{lemma}\label{L1.4}
    The functions $v(x):=|x|^{-\lambda}$ satisfies the inequality\\
     $\sum a_{ij}D_{ij}v\ge 0$ in ${{\mathbb R}^n}{\setminus}\{0\}$, provided the constant $\lambda=\lambda(n,\nu)>0$ is large enough.
\end{lemma}

\begin{proof}
    We have
    \[ \begin{split}
      \sum_{i,j}a_{ij}D_{ij}\big(|x|^{-\lambda}\big) &
      =\lambda |x|^{-\lambda-2}\cdot\bigg[(\lambda+2)\sum_{i,j}\frac{a_{ij}x_ix_j}{|x|^2}
      -{{\rm tr}\,} a \bigg]\\
        & \ge \lambda |x|^{-\lambda-2}\cdot \big[(\lambda+2)\nu- n\nu^{-1}\big]
        \ge 0{\quad\text{{for}}\quad} x\ne 0,
    \end{split} \]
provided $\lambda>0$ and $\lambda+2\ge n\nu^{-2}$.
\end{proof}

\begin{remark}\label{R1.5}
    The previous lemma says that $L\big(|x|^{-\lambda}\big)\ge 0$ for $x\ne 0$, where $L$ is an operator in {\eqref{{1.1}}} with $b_i{\equiv} 0$. One can easily adjust the proof of this lemma to the case $|b_i|\le K={{\rm const}}$, with $\lambda=\lambda(n,\nu,K,{{\rm diam}\,}\Omega)>0$.
\end{remark}

\noindent
\emph{Proof of Lemma \ref{L1.1}.}
    We have $u\ge c={{\rm const}}>0$ on the set ${\partial} B_{r_0/2}(y_0)$, which is a compact subset of $\Omega$.
    Following the argument in \S 1.3 of the book by E.M. Landis \cite{L71}, consider the function
    \[ u_1(x):=c_1\,\big(|x-y_0|^{-\lambda}-r_0^{-\lambda}\big)
    {\quad\text{{in}}\quad} \Omega_1:=B_{r_0}(y_0){\setminus} B_{r_0/2}(y_0)
    {\subset}\Omega, \]
    where $c_1:=(2^{\gamma}-1)^{-1}r_0^{\gamma}\,c>0$.
    Then $u_1=c\le u$ on ${\partial} B_{r_0/2}(y_0)$, and $u_1=0\le u$ on ${\partial} B_{r_0}(y_0)$, i.e. $u_1\le u$ on ${\partial}\Omega_1$. Moreover, $Lu_1\ge 0\ge Lu$ in $\Omega_1$. By the comparison principle, we have $u_1\le u$ in $\Omega_1$. It is easy to see that {\eqref{{1.3}}} holds true for the function $u_1$, hence it is also true for the given function $u$.
\hfill $\Box$
\medskip

\noindent
\emph{Proof of Lemma \ref{L1.2}.}
    We adjust the argument in \S IV.7.3 of the book by R. Courant and D. Hilbert \cite{CH}. Replacing the ball $B$ by a smaller ball if necessary, one can assume that it lies at a positive distance from $({\partial\Omega}){\setminus} B_{{\varepsilon}_0}(x_0)$. Then it is possible to choose a constant $R_0>r_0$ close to $r_0$, such that the set $({\partial\Omega})\cap \big(B_{R_0}(y_0){\setminus} B_{r_0}(y_0)\big)$ is a subset of $({\partial\Omega})\cap B_{{\varepsilon}_0}(x_0)$.
    Consider the function
    \[ u_2(x):=c_2\,\big(r_0^{-\lambda}-|x-y_0|^{-\lambda}\big)
    {\quad\text{{in}}\quad} \Omega_2:=\Omega\cap\big(B_{R_0}(y_0){\setminus} B_{r_0}(y_0)\big).\]
    Here $c_2>0$ is a large enough constant, such that
    \[ u\le c_2\,\big(r_0^{-\lambda}-R_0^{-\lambda}\big)=u_2
    {\quad\text{{on}}\quad} \Omega\cap {\partial} B_{R_0}(y_0).\]
    On the remaining part of ${\partial\Omega}_1$, which is a subset of $({\partial\Omega})\cap B_{{\varepsilon}_0}(x_0)$, we have $u=0\le u_2$. This means $u\le u_2$ on ${\partial\Omega}_2$. Moreover, $Lu\ge 0\ge Lu_2$ in $\Omega_2$.  By the comparison principle, we have $u\le u_2$ in $\Omega_2$. Since $u_2$ is a Lipschitz function on $\Omega_2$, and $u_2(x_0)=0$, the ratio
    \[ \frac{u(x)}{|x-x_0|}\le \frac{u_2(x)}{|x-x_0|}\le N={{\rm const}}
    {\quad\text{{in}}\quad} \Omega_2.\]
    On the complementary set $\Omega{\setminus} \Omega_2$, the function $u\in C({\overline{\Omega}})$ is bounded, and $|x-x_0|\ge R_0-r_0>0$. This implies the desired estimate {\eqref{{1.4}}}.
\hfill $\Box$
\medskip

In the formulations of Lemmas \ref{L1.1} and \ref{L1.2}, one cannot replace an exterior or interior sphere condition by a corresponding cone condition, as the following simple example shows.

\begin{example}\label{E1.6}
    (i) Fix a constant $\theta_1\in (0,\pi/2)$ and denote
    \[ \Omega_1:=\{ x=(x_1,x_2)\in {{\mathbb{R}}}^2:
    \;|x|<1,\; x_2>K\cdot |x_1|\},\]
    where $K:=\cot \theta_1>0$. In the polar coordinates $x_1=\rho\sin\theta,\; x_2=\rho\cos\theta$, we have
    \[ \Omega_1:=\{0<\rho<1,\;|\theta|<\theta_1\},
    {\quad\text{{and}}\quad} z:=ix_1+x_2=\rho e^{i\theta}.\]
    The function
    \[ u_1(x_1,x_2):={{\rm Re}\,}\big(z^{\gamma_1}\big)
    =\rho^{\gamma_1}\cos(\gamma_1\theta),
    {\quad\text{{where}}\quad} \gamma_1:=\frac{\pi}{2\theta_1}>1,\]
    belongs to $C^{\infty}(\Omega_1)\cap C({\overline}{\Omega_1})$ and satisfies $\,u_1>0,\,\Delta u_1=0\,$ in $\Omega_1$, and $u_1(0)=0$.
    It is easy to see that $u_1$ does not satisfy the strict inequality {\eqref{{1.3}}} (we have an equality) at the point $x_0=0\in{\partial\Omega}_1$, where ${\bm{{l}}}$ is an arbitrary interior vector to $\Omega_1$.
    \smallskip

    (ii) The set
    \[ \Omega_2:=\{ x=(x_1,x_2)\in {{\mathbb{R}}}^2:
    \;|x|<1,\; x_2>-K\cdot |x_1|\},\]
    can be described in a similar way with $\theta_2:=\pi-\theta_1\in (\pi/2,\pi)$ in place of $\theta_1$. The function
    \[ u_2(x_1,x_2):={{\rm Re}\,}\big(z^{\gamma_2}\big)
    =\rho^{\gamma_2}\cos(\gamma_2\theta),
    {\quad\text{{where}}\quad} \gamma_2:=\frac{\pi}{2\theta_2}\in (0,1),\]
    belongs to $C^{\infty}(\Omega_2)\cap C({\overline}{\Omega_2})$ and satisfies $\,u_2>0,\,\Delta u_2=0\,$ in $\Omega_2$, and $u_2{\equiv} 0$ on $({\partial\Omega}_2)\cap B_1(0)$. Obviously, the ratio $u_2(x)/|x|$ is unbounded on $\Omega_2$, i.e. {\eqref{{1.4}}} fails at the point $x_0=0\in{\partial\Omega}_2$.
\end{example}

Now consider a more general situation, when a ball $B$ in Lemmas \ref{L1.1} and \ref{L1.2} is replaced by a body of rotation  $Q$.

\begin{definition}\label{D1.7}
    Let a constant $r_0>0$ be given, and let $\psi(r)$ be a non-negative, non-decreasing function on $[0,r_0]$, with $\psi(r_0)<r_0$.  Define
    \begin{equation}\label{1.5}
        Q:=\{x=(x',x_n)\in{{\mathbb R}^n}:\;|x'|<r_0,\;0<x_n-\psi(|x'|)<r_0 \}.
    \end{equation}

    (i) We say that an open set $\Omega{\subset}{{\mathbb R}^n}$ satisfies an \emph{interior $Q$-condition} at a point $x_0\in{\partial\Omega}$ if $\Omega$ contains a body which is congruent with $Q$ with vertex at $x_0$. This means that in an appropriate coordinate system, we have $Q{\subset}\Omega$, and $x_0=0\in({\partial\Omega})\cap({\partial} Q)$.

    (ii) We say that an open set $\Omega{\subset}{{\mathbb R}^n}$ satisfies an \emph{exterior $Q$-condition} at a point $x_0\in{\partial\Omega}$ if its complement $\,\Omega^c:={{\mathbb R}^n}{\setminus}{\overline}{\Omega}\,$ satisfies an interior $Q$-condition at $x_0$.
\end{definition}

Our main results are contained in Theorems \ref{T1.8}--\ref{T1.11} below. Theorems \ref{T1.8} and \ref{T1.9} can be considered as generalizations of Lemmas \ref{L1.1} and \ref{L1.2} correspondingly, when instead of (exterior or interior) sphere conditions we impose $Q$-conditions with
    \begin{equation}\label{1.6}
        I(\psi):=\int_0^{r_0}\frac{\psi(r)\,dr}{r^2}<{\infty}.
    \end{equation}
Without loss of generality, we assume that the coordinate system is chosen in such a way that $x_0=0\in{\partial\Omega}$, $Q{\subset}\Omega$ if $\Omega$ satisfies an interior $Q$-condition, and $-Q:=\{x\in{{\mathbb R}^n}:\; -x\in Q\}{\subset} \Omega^c:={{\mathbb R}^n}{\setminus} \Omega$ if $\Omega$ satisfies an exterior $Q$-condition. Note that sphere conditions are equivalent to $Q$-condition with $\psi(r)=cr^2,\,c={{\rm const}}>0$. In this case $I(\psi)<{\infty}$ automatically. We prove Theorems \ref{T1.8} and \ref{T1.9} in Section \ref{S.3}. Another two theorems, Theorems \ref{T1.10} and \ref{T1.11},  are given here just for completeness, without proofs. They claim that the assumption $I(\psi)<{\infty}$ is sharp: if $I(\psi)={\infty}$, then the estimates in Lemmas \ref{L1.1} and \ref{L1.2} fail. Example \ref{E1.6} can serve as a clear demonstration of this fact for $\psi(r)=Kr$.

In Theorems \ref{T1.8}--\ref{T1.11}, we assume that $u\in C^2(\Omega)\cap C({\overline{\Omega}})$ is a positive solution of the inequality $Lu\le 0$ or $Lu\ge 0$ in $\Omega$, where $Lu:=\sum a_{ij}D_{ij}u$ has the form {\eqref{{1.1}}}, {\eqref{{1.2}}}, with $b_i{\equiv} 0$. Combining our techniques with others, especially those in the paper by O. A. Ladyzhenskaya and N. N. Ural'tseva \cite{LU88}, one can extend the results in Theorems \ref{T1.8}--\ref{T1.11} to more general operators $L$ in {\eqref{{1.1}}} with $b_i\in L^q,\;q>n$. We plan to do it in our subsequent work. In particular, proofs of Theorems \ref{T1.10} and \ref{T1.11} will be presented in a more general setting. On the other hand, Example \ref{E1.12} below shows that in the case $b_i\in L^n$ all the estimates under considerations fail even for flat boundary, when $\psi{\equiv} 0$.
Here we restrict ourselves to the case $b_j{\equiv} 0$ in order to expose our method in its ``pure'' form.

\begin{theorem}\label{T1.8}
    Suppose that $\Omega$ satisfies an interior $Q$-condition: $Q{\subset}\Omega$, with $I(\psi)<{\infty}$, and $0\in{\partial\Omega}$.  Then for any function $u\in C^2(\Omega)\cap C({\overline{\Omega}})$ satisfying $\,u>0,\;Lu\le 0\,$ in $\Omega$, and $u(0)=0$, we have
     \begin{equation}\label{1.7}
        \liminf_{t\to 0^+}\, t^{-1} u(t\,{\bm{{l}}})>0
        {\quad\text{{for each}}\quad} {\bm{{l}}}\in
        {{\mathbb R}^n}_+:=\{x\in{{\mathbb R}^n}:\;x_n>0\}.
    \end{equation}

     Note that from $I(\psi)<{\infty}$ it follows that $t{\bm{{l}}}\in Q{\subset}\Omega$ for small $t>0$ (Corollary \ref{C3.2} below), so that $u(t{\bm{{l}}})$ in {\eqref{{1.7}}} is well defined.
\end{theorem}

\begin{theorem}\label{T1.9}
    Suppose that $\Omega$ satisfies an exterior $Q$-condition: $-Q{\subset}\Omega^c$, with $I(\psi)<{\infty}$, and $0\in{\partial\Omega}$.  Then for any function $u\in C^2(\Omega)\cap C({\overline{\Omega}})$ satisfying $\,u>0,\;Lu\ge 0\,$ in $\Omega$, and $u=0$ on $({\partial\Omega})\cap B_{r_0}(0)$, we have
    \begin{equation}\label{1.8}
        M(r_0):=\sup_{\Omega\cap B_{r_0}(0)}
        \frac{u(x)}{|x|}<{\infty}.
    \end{equation}
\end{theorem}

The notation $M(r)$ is also used in the following
\begin{theorem}\label{T1.10}
    Suppose that $\Omega\cap B_{r_0}(0){\subset} Q$, with $I(\psi)={\infty}$, and $0\in{\partial\Omega}$. Then for any function $u\in C^2(\Omega)\cap C({\overline{\Omega}})$ satisfying $\,u>0,\;Lu\ge 0\,$ in $\Omega$, and $u=0$ on $({\partial\Omega})\cap B_{r_0}(0)$, we have $M(r)\to 0$ as $r\to 0^+$.
    Obviously, in this case the estimate {\eqref{{1.7}}} fails.
\end{theorem}

\begin{theorem}\label{T1.11}
    Suppose that $\Omega^c\cap B_{r_0}(0){\subset} -Q$, with $I(\psi)={\infty}$, and $0\in{\partial\Omega}$. Then for any function $u\in C^2(\Omega)\cap C({\overline{\Omega}})$ satisfying $\,u>0,\;Lu\le 0\,$ in $\Omega$, and $u=0$ on $({\partial\Omega})\cap B_{r_0}(0)$, we have
         \begin{equation}\label{1.9}
        \liminf_{t\to 0^+}\, t^{-1} u(t\,{\bm{{l}}})={\infty}
        {\quad\text{{for all}}\quad} {\bm{{l}}}\in {{\mathbb R}^n}_+.
    \end{equation}
\end{theorem}

In 1969--1970, similar facts were  established by B.N. Khimchenko, first in the case $L=\Delta$ \cite{Kh69}, and then for general elliptic operators $L\,$ \cite{Kh70}, under the additional assumption $\psi''\ge 0$ (in these two papers, the same author's name is spelled slightly differently). Further, is a series of joint papers by L.I. Kamynin and B.N. Khimchenko (see \cite{KH80} and references therein), these results were extended to the parabolic and degenerate elliptic equations, under a different assumption $\psi(r)=r\psi_1(r)$ with $\psi'_1\ge 0,\;\psi''_1\le 0$. Each of these assumptions, as well as our assumption {\eqref{{1.6}}}, holds true for $\psi(r):=r^{1+\alpha},\,0<\alpha<1$, so that the above mentioned result by G. Giraud \cite{G33} for ${\partial\Omega}\in C^{1,\alpha}$ is extended to general operators $L$ with bounded measurable coefficients. This case is also covered in the paper \cite{L85} by Gary M. Lieberman, in which ${\partial\Omega}$ has a Dini continuous normal.

In the papers \cite{H52}, \cite{O52}, \cite{Kh69}, \cite{Kh70}, \cite{KH80}, \cite{L85}, and many others, the estimates of such kind are proved by means of special comparison functions, which are constructed in a more or less explicit form. Our method is quite different: it does not use any explicit expressions for comparison functions, and it  does not require additional assumptions on the functions $\psi(r)$ in Definition \ref{D1.7}. Instead, we use the estimates for quotients $u_2/u_1$ of positive solutions of $Lu=0$ in a Lipschitz domain $\Omega$, which vanish on a portion of ${\partial\Omega}$. These estimates were proved by Patricia  Bauman in 1982 in her PhD thesis \cite{B82}, and published a bit later in \cite{B84}. Note that some estimates in her paper depend on the modulus of continuity of coefficients $a_{ij}$. However, it is easy to get rid of this additional dependence. In a more general parabolic case, this was done in \cite{FSY}, Theorem 4.3.

We essentially use the fact that $u(x){\equiv} x_n$ is a solution to the elliptic equation $Lu:=\sum a_{ij}D_{ij}u=0$; this is why we assume $b_i{\equiv} 0$ in {\eqref{{1.1}}}. Note that the estimates for the quotients $u_2/u_1$ are also true for solutions to the equations in the \emph{divergence} form $Lu:=\sum D_i(a_{ij}D_ju)=0$ (see \cite{CFMS}), but they are not helpful here, because linear functions do not satisfy such equations in general, and in fact, the Hopf-Oleinik estimate {\eqref{{1.3}}} fails even when the boundary is flat (see \cite{GT}, Problem 3.9).

\begin{example}\label{E1.12}
    Consider the functions
    \[ u_1(x):=\frac{x_n}{|\ln|x||}
    {\quad\emph{{and}}\quad} u_2(x):=x_n\cdot |\ln|x||\]
    in the cylinder $Q:=\{x=(x',x_n)\in{{\mathbb R}^n}:\;|x'|<1/2,\;0<x_n<1/2\}$,
    extended as $u_1=u_2=0$ on $({\partial} Q)\cap \{x_n=0\}$. Then each of these two functions can be considered as a solution to the equation
    \[ \Delta u+{\bm{{b}}}\cdot Du:= \Delta u+\sum_i b_iD_iu=0
    {\quad\emph{{in}}\quad} Q,\]
    where the vector function ${\bm{{b}}}:=-\Delta u\cdot |Du|^{-2} Du$ satisfies
    \[ |{\bm{{b}}}| = \frac{|\Delta u|}{|Du|}
    \le \frac{{\rm const}}{|x|\cdot |\ln|x||}
    \in L^n(Q)
    {\quad\emph{{for}}\quad}n\ge 2.\]
    However, the left side of {\eqref{{1.7}}} is $0$ for $u=u_1$, and the left side of {\eqref{{1.8}}} is ${\infty}$ for $u=u_2$.
\end{example}

In Section \ref{S.2}, we bring together, in a convenient form, some basic facts, including the estimated for the quotients $u_2/u_1$ of positive solutions, which are essential for our approach. Finally, in Section \ref{S.3}, we prove Theorems \ref{T1.8} and \ref{T1.9}.
\medskip

\emph{Notations}.  We use notations $N$ and $c$ for various positive constants depending only on the prescribed constants, such as $n,\,\nu$, etc., which do not depend on smoothness of coefficients $a_{ij}$. These constants may be different in different expression.
The expression $A:=B$ or $B=:A$ means ``$A=B$ by definition''.

$B_r(x_0):=\{x\in{{\mathbb R}^n}:\,|x-x_0|<r\}$ is a ball of radius $r>0$ centered at $x_0\in{{\mathbb R}^n}$. ${{\mathbb R}^n}_+:=\{x=(x_1,\ldots,x_n)\in{{\mathbb R}^n}:\,x_n>0\}$.
\medskip

\emph{Acknowledgements}. The author thanks N. V. Krylov, N. N. Ural'tseva, and H. F. Weinberger for very useful discussion of results in this paper.

{\section{{Auxiliary statements}}
\setcounter{equation}{0}}\label{S.2}

In the rest of this paper, $Lu:=\sum a_{ij}D_{ij}u$ with $a_{ij}=a_{ji}\in L^{\infty}$ satisfying the ellipticity condition {\eqref{{1.2}}} with a constant $\nu\in (0,1]$. Note that the results in this section are valid for more general operators $L$ in {\eqref{{1.1}}}, which include the lower order terms $\sum b_iD_iu$ with $b_i\in L^{\infty}$. In this case, the constants $N$ and $c$ depend also on the upper bounds for $|b_i|$.

The following theorem was proved by N. V. Krylov and the author \cite{KS80}, \cite{S80} (see also \cite{K}, Theorem IV.2.8, and \cite{GT}, Corollary 9.25).

\begin{theorem}[Interior Harnack inequality]\label{T2.1}
    Let $\Omega$ be a bounded domain in ${{\mathbb R}^n}$, such that the set
    \begin{equation}\label{2.1}
        \Omega^{\delta}:=\{x\in\Omega:\;{{\rm dist}}(x,{\partial\Omega})>\delta\}
    \end{equation}
is connected, where $\delta={{\rm const}}>0$. Then
    \begin{equation}\label{2.2}
        \sup_{\Omega^{\delta}}u\le N\cdot \inf_{\Omega^{\delta}}u,
    \end{equation}
    with a constant $N$ depending only on $n,\nu$, and $\delta/{{\rm diam}\,} \Omega$.
\end{theorem}

\begin{proof}
    In its standard form, the Harnack inequality is formulated for two concentric balls in place of $\Omega^{\delta}$ and $\Omega$, e.g. for $B_{R/8}$ and $B_R$ in \cite{S80}, Theorem 3.1. In general case, fix $x,y\in\Omega^{\delta}$, and choose a sequence $x^{(0)}=x,x^{(1)},\ldots,x^{(m)}=y$ in $\Omega^{\delta}$ such that $|x^{(k-1)}-x^{(k)}|<\delta/8$ for $k=1,2,\ldots,m$. One can do it in such a way that $m$ does  not exceed a constant $m_0$ depending only on $n$ and $\delta/{{\rm diam}\,} \Omega$. Then applying the ``standard'' Harnack inequality with $R:=\delta$, we get
    \[ u(x^{(k-1)})\le N_1 u(x^{(k)})
    {\quad\text{{for}}\quad} k=1,2,\ldots,m,\]
    where $N_1=N_1(n,\nu)\ge 1$. Therefore,
    \[ u(x)=u(x^{(0)})\le N_1u(x^{(1)})\le \ldots \le N_1^m u(x^{(m)})=N_1^m u(y),\]
    and the desired estimate {\eqref{{2.2}}} follows with $N:=N_1^{m_0}$.
\end{proof}

The following lemma will help us to reduce the proofs of our main results for operators $Lu:=\sum a_{ij}D_{ij}u$ to the case $a_{ij}\in C^{\infty}$. We can assume that $a_{ij}$ are defined on the whole space ${{\mathbb R}^n}$. Consider the convolutions $a_{ij}^{\varepsilon}:=a_{ij}*\eta^{\varepsilon}$ with kernels $\eta^{\varepsilon}$ such that
    \[ 0\le \eta^{\varepsilon}\in C^{\infty}({{\mathbb R}^n}),\quad
    \eta^{\varepsilon}(x){\equiv} 0{\quad\text{{for}}\quad} |x|\ge {\varepsilon},
    {\quad\text{{and}}\quad} \int\eta^{\varepsilon}(x)\,dx=1.\]
Then $a_{ij}^{\varepsilon}\in C^{\infty}({{\mathbb R}^n}),\; a_{ij}^{\varepsilon}=a_{ji}^{\varepsilon}$ satisfy {\eqref{{1.2}}}, and moreover,
\begin{equation}\label{2.3}
    a_{ij}^{\varepsilon}\to a_{ij}{\quad\text{{as}}\quad}{\varepsilon}\to 0^+
    {\quad\text{{a.e. in}}\quad}\Omega.
\end{equation}
This convergence follows from the properties of the Lebesgue sets (see \cite{St70}, Sec. I.1.8).

\begin{lemma}[Approximation lemma]\label{L2.2}
    Let $\Omega$ be a bounded open set in ${{\mathbb R}^n}$ satisfying an exterior cone condition at each point $x_0\in{\partial\Omega}$, i.e. an exterior $Q$-condition in Definition \ref{D1.7} with \begin{equation}\label{2.4}
    Q:=\{x=(x',x_n):\;|x|<r_0,\;x_n>K\cdot |x'|\}\end{equation}
    with constants $K>0$ and $r_0>0$. Let $u$ be a function in $C^2(\Omega)\cap C({\overline{\Omega}})$ satisfying $Lu:=\sum a_{ij}D_{ij}u\le 0$ in $\Omega$. For ${\varepsilon}>0$, consider the above approximations of $a_{ij}$ by functions $a_{ij}^{\varepsilon}\in C^{\infty}$, which satisfy {\eqref{{1.2}}} and {\eqref{{2.2}}}, and let $u^{\varepsilon}$ be a unique solution to the problem
    \begin{equation}\label{2.5}
        L^\veu^{\varepsilon}:=\sum_{i,j} a_{ij}^\veD_{ij}u^{\varepsilon}=0
        {\quad\text{{in}}\quad}\Omega,\qquad
        u^{\varepsilon}=u{\quad\emph{{on}}\quad}{\partial\Omega},
    \end{equation}
    in the class $C^{\infty}(\Omega)\cap C({\overline{\Omega}})$. Then
    \begin{equation}\label{2.6}
        \sup_{\Omega} (u^{\varepsilon}-u)\to 0
        {\quad\emph{{as}}\quad} {\varepsilon}\to 0^+.
    \end{equation}

    If $Lu=0$ in $\Omega$, then $u^{\varepsilon}\to u$ as $\,{\varepsilon}\to 0^+$ uniformly on $\Omega$.
\end{lemma}

Note that the existence of a solution $u^{\varepsilon}\in C^{\infty}(\Omega)\cap C({\overline{\Omega}})$ to the problem {\eqref{{2.5}}} (under an exterior cone condition) follows from the results by K. Miller \cite{M67}.

\begin{proof}
    From the arguments in the proof of Theorem 3 in \cite{M67} it follows that
        \[\sup_{\Omega\cap B_{\delta}(x_0)}
        |u^{\varepsilon}(x)-u(x_0)|
        \le \omega(\delta)\to 0
        {\quad\text{{as}}\quad} \delta\to 0^+,\]
    uniformly with respect to $x_0\in{\partial\Omega}$ and ${\varepsilon}>0$. Since $u\in C({\overline{\Omega}})$, this property also holds true for $u(x)$ in place of $u^{\varepsilon}(x)$. By the triangle inequality, we get
    \begin{equation}\label{2.7}
        \sup_{\Omega{\setminus}\Omega^{\delta}} |u^{\varepsilon}-u|\le 2\omega(\delta),
    \end{equation}
    where $\Omega^{\delta}$ is defined in {\eqref{{2.1}}}. Moreover, since $L^\veu^{\varepsilon}=0\ge Lu$, we also have
    \[ L^{\varepsilon}(u^{\varepsilon}-u)\ge f^{\varepsilon}:=(L-L^{\varepsilon})u
    :=\sum_{i,j}(a_{ij}-a_{ij}^{\varepsilon})D_{ij}u.\]
    Now we can use the A.D. Aleksandrov type estimate (see \cite{A67} or \cite{GT}, Theorem 9.1):
    \[\sup_{\Omega^{\delta}} (u^{\varepsilon}-u)
    \le \sup_{{\partial\Omega}^{\delta}} (u^{\varepsilon}-u)
    +N\cdot ||f^{\varepsilon}||_{L^n(\Omega^{\delta})}.\]
    By virtue of {\eqref{{2.7}}}, this yields
    \[ \sup_{\Omega} (u^{\varepsilon}-u)
    \le 2\omega(\delta)+N\cdot ||f^{\varepsilon}||_{L^n(\Omega^{\delta})}.\]
    Since $D_{ij}u$ are bounded on $\Omega^{\delta}$, and $a_{ij}^{\varepsilon}\to a_{ij}$ a.e., the last term converges to $0$ as ${\varepsilon}\to 0^+$. Hence
    \[ 0\le \limsup_{{\varepsilon}\to 0^+}\,\sup_{\Omega} (u^{\varepsilon}-u)
    \le 2\omega(\delta).\]
    The desired property {\eqref{{2.6}}} follows by sending $\delta$ to $0$.

    In the case $Lu=0$, we can apply {\eqref{{2.6}}} to both functions $u$ and $-u$, which gives the uniform convergence of $u^{\varepsilon}$ to $u$ on $\Omega$.
\end{proof}

We also need a lower estimate for positive supersolutions in $\Omega$, which are strictly positive on a Lipschitz  portion of the boundary ${\partial\Omega}$. For the proof of this estimate, it is convenient to replace the Lipschitz property of ${\partial\Omega}$ by a weaker assumption {\eqref{{2.9}}} below.

\begin{lemma}\label{L2.3}
    Let $\Omega$ be a bounded domain in ${{\mathbb R}^n}$, and let $u\in C^2(\Omega)\cap C({\overline{\Omega}})$ satisfy $u>0,\,Lu\le 0$ in $\Omega$. Suppose that
    \begin{equation}\label{2.8}
        u\ge\mu={{\rm const}}{\quad\emph{{on}}\quad} ({\partial\Omega})\cap B_{r_0}(x_0),
    \end{equation}
     where $x_0\in{\partial\Omega}$ and $r_0>0$ is a given constant. Moreover, let $\,\delta>0$ be a constant such that the set $\,\Omega^{\delta}$ in {\eqref{{2.1}}} is connected, and there are balls
    \begin{equation}\label{2.9}
        B_{\delta}(y_0){\subset} \Omega^c\cap B_{r_0/2}(x_0)
        {\quad\text{{and}}\quad}
        B_{\delta}(z_0){\subset} \Omega\cap B_{r_0/2}(x_0).
    \end{equation}
    Then
    \begin{equation}\label{2.10}
        u\ge c\mu{\quad\emph{{in}}\quad} \Omega^{\delta},
        {\quad\emph{{where}}\quad} c=c(n,\nu,\delta/{{\rm diam}\,}\Omega)>0.
    \end{equation}
\end{lemma}

\begin{proof} \emph{Step 1.} From {\eqref{{2.9}}} it follows that $\delta\le r_0/4$, and the balls $B_{3\delta}(y_0)$ and $B_{3\delta}(z_0)$ are contained in $B_{r_0}(x_0)$. Therefore, same is true for $B_{3\delta}(y)$, and by {\eqref{{2.8}}}, $u\ge\mu$ on $({\partial\Omega})\cap B_{3\delta}(y)$ for each $y$ in the segment $[y_0,z_0]$.

Next, choose a sequence of points $y_0,y_1,\ldots,y_m=z_0$ in $[y_0,z_0]$, such that $|y_{k+1}-y_k|\le \delta$ for all $k=0,1,\ldots,m-1$. Obviously, we can assume that $m$ does not exceed a constant $m_1$ depending only on $\delta/{{\rm diam}\,}\Omega$. We claim that
\begin{equation}\label{2.11}
    u\ge\theta^k\mu{\quad\text{{in}}\quad}\Omega\cap B_{\delta}(y_k)
    {\quad\text{{for}}\quad}k=0,1,\ldots,m,
\end{equation}
with a constant $\theta=\theta(n,\nu)\in (0,1)$, to be specified later. Here we impose a natural agreement that {\eqref{{2.11}}} is true automatically if $\Omega\cap B_{\delta}(y_k)$ is empty, which is the case if $k=0$. In order to use  induction, we only need to prove {\eqref{{2.11}}} with $k+1$ in place of $k$, based on the assumption that it is true for some $k<m$. For this purpose, we compare the function $u(x)$ with
    \[ v_k(x):=\theta^k\mu\cdot
    \frac{|x-y_k|^{-\gamma}-(3\delta)^{-\gamma}}
    {\delta^{-\gamma}-(3\delta)^{-\gamma}}
    {\quad\text{{in}}\quad}
    \Omega_k:=\Omega\cap
    \big(B_{3\delta}(y_k){\setminus} B_{\delta}(y_k)\big),\]
where $\gamma=\gamma(n,\nu)>0$ is a constant in Lemma \ref{L1.4}. Of course, we can skip this part if $\Omega_k$ is empty. By this lemma, $Lv_k\ge 0\ge Lu$ in $\Omega_k$. Moreover, {\eqref{{2.11}}} implies $u\ge\theta^k\mu=v_k$ on $({\partial\Omega}_k)\cap {\partial} B_{\delta}(y_k)$. We also have $u\ge 0=v_k$ on $({\partial\Omega}_k)\cap {\partial} B_{3\delta}(y_k)$, and by {\eqref{{2.8}}}, $u\ge\mu\ge v_k$ on the remaining part of ${\partial\Omega}_k$. By the comparison principle, $u\ge v_k$ in $\Omega_k$. Together with {\eqref{{2.11}}}, this gives us
    \[u\ge \theta^{k+1}\mu{\quad\text{{in}}\quad}
    \Omega\cap B_{2\delta}(y_k),{\quad\text{{if}}\quad}
    \theta:=\frac{(3/2)^{\gamma}-1}{3^{\gamma}-1}\in (0,1).\]
Finally, $|y_{k+1}-y_k|\le\delta$ implies that the set $\Omega\cap B_{\delta}(y_{k+1})$ is contained in $\Omega\cap B_{2\delta}(y_k)$, so that the inequality in {\eqref{{2.11}}} holds true for $k+1$. By induction, the proof of {\eqref{{2.11}}} is complete. In particular, taking $k=m\le m_1$, we get
    \begin{equation}\label{2.12}
    u\ge  c_1\mu{\quad\text{{on}}\quad} B_{\delta}(z_0),
    {\quad\text{{where}}\quad} c_1:=\theta^{m_1}>0.
    \end{equation}

\emph{Step 2.} For an arbitrary point $z\in \Omega^{\delta}$, and choose a sequence of points $z_0,z_1,\ldots,z_m=z$ in $\Omega^{\delta}$, such that
$|z_{k+1}-z_k|\le \delta_1:=\delta /3$ for all $k=0,1,\ldots,m-1$. Here we can assume that
$m\le m_2=m_2(n,\delta/{{\rm diam}\,}\Omega)$. Similarly to {\eqref{{2.11}}}, with $z_k$ in place of $y_k$ and $\delta_1$ in place of $\delta$, and some simplifications because of the property $B_{3\delta_1}(z_k)=B_{\delta}(z_k){\subset}\Omega$, we obtain
    \[ u\ge\theta^k c_1\mu{\quad\text{{in}}\quad}B_{\delta_1}(z_k)
    {\quad\text{{for}}\quad}k=0,1,\ldots,m.\]
In particular,
$u(z)=u(z_m)\ge \theta^m c_1\mu \ge \theta^{m_2} c_1\mu$. Since $z$ is an arbitrary point in $\Omega^{\delta}$, the desired estimate {\eqref{{2.10}}} is proved with $c:=\theta^{m_2} c_1=\theta^{m_1+m_2}$.
\end{proof}

The following theorem, which is due P. Bauman (see \cite{B84}, Theorem 2.1), is the main tool in our approach.

\begin{theorem}[Comparison theorem]\label{T2.4}
    Let ${\varphi}$ be a Lipschitz continuous function on ${{\mathbb{R}}}^{n-1}$:
    \[ |{\varphi}(x')-{\varphi}(y')|\le K\cdot |x'-y'|{\quad\emph{{for all}}\quad} x',y'\in {{\mathbb{R}}}^{n-1},\]
    with $K={{\rm const}}\ge 0$, and ${\varphi}(0)=0$. For $\,r>0$, define
    \[\Omega_r:=\{x=(x',x_n)\in{{\mathbb R}^n}:\;|x'|<r,\;0<x_n-{\varphi}(x')<r\},\]
    and $\Gamma_r:=({\partial} \Omega_r)\cap\{x_n={\varphi}(x')\}$. Let $u,v$ be functions in $C^2(\Omega_{2r})\cap C({\overline}{\Omega_{2r}})$ satisfying
    \[ u>0,\;v>0,\quad Lu=Lv=0{\quad\emph{{in}}\quad} \Omega_{2r},\]
    and $u=v=0$ on $\Gamma_{2r}$. Then
    \begin{equation}\label{2.13}
        \sup_{\Omega_r}\frac{u}{v}\le N\cdot \frac{u(0,r)}{v(0,r)},
        {\quad\emph{{where}}\quad} N=N(n,\nu,K)>0.
    \end{equation}
\end{theorem}

\begin{corollary}\label{C2.5}
    Under the assumptions of the previous theorem, we also have
    \begin{equation}\label{2.14}
    \frac{u(0,r)}{v(0,r)}\le N\cdot
    \inf_{\Omega_r}\frac{u}{v},
    {\quad\emph{{where}}\quad} N=N(n,\nu,K)>0.
    \end{equation}
\end{corollary}
\begin{proof}
    Obviously, we can interchange $u$ and $v$ in {\eqref{{2.13}}}, and then {\eqref{{2.14}}} follows from an elementary relation  $\inf(u/v)=\big(\sup(v/u)\big)^{-1}$.
\end{proof}

\begin{remark}\label{R2.6}
    In \cite{B84}, this theorem was proved with $\Omega_{8r},\,\Gamma_{8r}$ in place of $\Omega_r,\,\Gamma_r$ correspondingly. In order to apply this fact to the proof of {\eqref{{2.13}}}, consider separately each of two possible cases for $x=(x',x_n)\in\Omega_r$: (i) $x_n-{\varphi}(x')<r/8$ and (ii) $x_n-{\varphi}(x')\ge r/8$. In the case (i), from \cite{B84}, after obvious change of notations, it follows
    \[ \frac{u(x)}{v(x)}\le N_1(n,\nu,K)\cdot \frac{u(x',{\varphi}(x')+r/8)}{v(x',{\varphi}(x')+r/8)},\]
    and then by the Harnack inequality, Theorem \ref{2.1},
    \[ \frac{u(x',{\varphi}(x')+r/8)}{v(x',{\varphi}(x')+r/8)}\le  N_2(n,\nu,K)\cdot \frac{u(0,r)}{v(0,r)},\]
    so that $u/v\,(x)\le N\cdot u/v(0,r)$ with $N:=N_1N_2$. In the case (ii), we get this estimate with $N:=N_2$ by the Harnack inequality directly. Therefore, {\eqref{{2.13}}} holds true.

    The above argument also shows that in the formulation of Theorem \ref{2.4}, one can replace $2r$ by $cr$ with any absolute constant $c>1$. We will use this observation with $c=3/2$ in order to get the estimate {\eqref{{2.15}}} below.
\end{remark}

\begin{corollary}\label{C2.7}
    The estimate {\eqref{{2.13}}} in Theorem \ref{2.4} remains valid if the condition $v=0$ on $\Gamma_{2r}$ is omitted.
\end{corollary}

\begin{proof}
    Having in mind the approximation lemma (Lemma \ref{L2.3}), we can assume that $a_{ij}$ are smooth. Take a continuous function $g$ on ${\partial}\Omega_{3r/2}$ such that $0\le g\le v$ on ${\partial}\Omega_{3r/2}$, $g{\equiv} 0$ on $\Gamma_{3r/2}$ and $g{\equiv} v$ on
    \[\Gamma^*_{3r/2}:=
    \big\{x=(x,,x_n)\in{{\mathbb R}^n}:\; |x'|\le 3r/2,\;x_n-{\varphi}(x')=3r/2 \big\}.\]
    Since $a_{ij}$ are smooth, there exists a solution $v_0\in C^2(\Omega_{3r/2})\cap({\overline}{\Omega_{3r/2}})$ of the problem
    \[ Lv_0=0{\quad\text{{in}}\quad}\Omega_{3r/2},\qquad
    v_0=g{\quad\text{{on}}\quad}{\partial}\Omega_{3r/2}.\]
    By Theorem \ref{2.4}, applied to the functions $u$ and $v_0$ in $\Omega_{3r/2}$,
    \begin{equation}\label{2.15}
        \sup_{\Omega_r}\frac{u}{v_0}\le N\cdot \frac{u(0,r)}{v_0(0,r)}.
    \end{equation}
    Moreover, by the comparison principle, $0\le v_0\le v$ in $\Omega_{3r/2}\supset \Omega_r$, hence we can replace $v_0$ by $v$ in the left side. In the right side, we first apply Lemma \ref{L2.3} to the function $v_0$ in $\Omega_{3r/2}$ with $r_0:=3r/2$ and $x_0:=(0,3r/2)\in\Gamma^*_{3r/2}$, and then the Harnack inequality to the function $v$ in $\Omega_{2r}$. As a result, we get
    \[ v_0(0,r)\ge c_1\mu,{\quad\text{{where}}\quad}
    \mu:=\inf_{\Gamma^*_{3r/2}}v_0=\inf_{\Gamma^*_{3r/2}}v
    \ge c_2\cdot v(0,r),\]
    with positive constants $c_1$ and $c_2$ depending only on $n,\nu$ and $K$. Therefore, from {\eqref{{2.15}}} it follows the desired estimate {\eqref{{2.13}}}.
    \end{proof}

{\section{{Proof of Theorems \ref{T1.8} and \ref{T1.9}}}
\setcounter{equation}{0}} \label{S.3}

First of all, we write the integral condition $I(\psi)<{\infty}$ in {\eqref{{1.6}}} in an equivalent ``discrete'' form.

\begin{lemma}\label{L3.1}
    Let $\psi(r)$ be a non-negative, non-decreasing function on $[0,r_0]$, where $r_0={{\rm const}}>0$. Then $I(\psi)<{\infty}$ if and only if \begin{equation}\label{3.1}
    \sum_{k=0}^{\infty}\frac{h_k}{r_k}<{\infty}{\quad\text{{where}}\quad}
    r_k:=4^{-k}r_0,\quad h_k:=\psi(r_k).
    \end{equation}
\end{lemma}

\begin{proof}
    Since $h_{k+1}\le\psi(r)\le h_k$ on $[r_{k+1},r_k]$, and $r_k-r_{k+1}=3r_{k+1}=3r_k/4$, we obtain
    \[I(\psi)=\sum_{k=0}^{\infty}\int_{r_{k+1}}^{r_k}\frac{\psi(r)\,dr}{r^2}
    \ge\sum_{k=0}^{\infty}\frac{3r_{k+1}h_{k+1}}{r_k^2}
    =\frac{3}{16} \sum_{k=0}^{\infty}\frac{h_{k+1}}{r_{k+1}}.\]
    On the other hand, $I(\psi)\le\sum 3r_{k+1}h_k/r_{k+1}^2
    =12\sum h_k/r_k$. Therefore, $I(\psi)<{\infty}$ if and only if $\sum h_k/r_k<{\infty}$.
\end{proof}

\begin{corollary}\label{C3.2}
    If $I(\psi)<{\infty}$, then for arbitrary constant $K_0>0$, there is a constant $0<R_0\le \min(r_0,h_0)$ such that the set
    \[ V_0:=\{x=(x',x_n):\,|x|<R_0,\;x_n>K_0\,|x'|\}\]
    is contained in $Q$.
\end{corollary}

\begin{proof}
    From $I(\psi)<{\infty}$ it follows $\sum h_k/r_k<{\infty}$, hence $h_k/r_{k+1}=4h_k/r_k\to 0$ as $k\to{\infty}$. Choose an integer $k_0\ge 1$ such that $h_k/r_{k+1}\le K_0$ for all $k\ge k_0$, and set $R_0:=\min(r_{k_0},h_{k_0})$. We claim that that each $x=(x',x_n)\in V_0$ belongs to $Q$. This is obvious if $x'=0$, so we can assume $x'\ne 0$. Then there is an integer $k\ge k_0$ (depending on $x$) such that $r_{k+1}\le |x'|<r_k\le R_0$. This implies
    \[\psi(|x'|)\le \psi(r_k)=:h_k\le K_0\,r_{k+1}\le K_0\,|x'|<x_n.\]
    which means $x\in Q$.
\end{proof}

The next lemma can be considered as a very special case of Theorem \ref{T1.8}. However, this ``model'' case contains the main difficulties, so that Theorem \ref{T1.8} in full generality follows easily by the comparison principle.

\begin{lemma}\label{L3.3}
    Let $Q$ be a set defined in {\eqref{{1.5}}}, where $r_0={{\rm const}}>0$, and $\psi(r)$ is a non-negative, non-decreasing function on $[0,r_0]$, satisfying the condition $I(\psi)<{\infty}$ in {\eqref{{1.6}}}. Let $v$ be a function in $C^2(Q)\cap C({\overline}{Q})$, such that \[v>0,\quad Lv:=\sum_{i,j=1}^n a_{ij}D_{ij}v=0
    {\quad\text{{in}}\quad}Q,\]
    and $v=0$ on $\Gamma:=({\partial} Q)\cap \{x_n\in\psi(|x'|)\}$. Then
    \begin{equation}\label{3.2}
    \inf_{0<x_n \le r_0/2} \frac{v(0,x_n)}{x_n}>0.
    \end{equation}

    Note that the non-decreasing function $\psi(r)$ may be discontinuous. In order to guarantee that the set $\Gamma$ is connected, we define $\psi(r):=[\psi(r-),\psi(r+)]$ - the segment whose ends are one-sided limits of $\psi(r')$ as $r'\to r$, subject to restriction $r'<r$ or $r'>r$. Obviously, if $\psi$ is continuous at some point $r$, then this segment is reduced to the corresponding point $\psi(r)$.
\end{lemma}

\begin{proof}
    We assume that the coefficients $a_{ij}$ are smooth functions on ${{\mathbb R}^n}$. The general case follows from the approximation lemma (Lemma \ref{L2.2}), because all the estimates in the proof do not depend on this smoothness. Using notations in Lemma \ref{L3.1}, denote $\theta_k:=h_k/r_k$. By this lemma, we have $\sum\theta_k<{\infty}$. We can start our considerations with large enough $k\ge 1$. Therefore, without loss of generality, we assume that $0\le \theta_k\le{\varepsilon}_0<1$ for all $k\ge 1$, where ${\varepsilon}_0={\varepsilon}_0(n,\nu)$ is a small constant in $(0,1)$, which will be specified later.

    For integers $k\ge 1$, denote $Q_k:=Q\cup C_{r_k}$, where
    \[C_r:=\{x=(x',x_n)\in{{\mathbb R}^n}:\; |x'|<r,\;0<x_n<r\}.\]
    We will approximate the given function $v$ by solutions $v_k\in C^2(Q_k)\cap C({\overline}{Q_k})$ of the Dirichlet problem
    \[Lv_k=0{\quad\text{{in}}\quad}Q_k,\qquad
    v_k=g_k{\quad\text{{on}}\quad}{\partial} Q_k,\]
    where $g_k$ is a continuous function on ${\partial} Q_k$, defined as $g_k{\equiv} v$ on $({\partial} Q_k)\cap ({\partial} Q)$, and $g_k{\equiv} 0$ on the remaining part of ${\partial} Q_k$. Note that $Q_k$ are Lipschitz domains, hence the existence of such solutions for equations with smooth coefficients is known. It is easy to see that $Q_k\searrow Q$, and by the comparison principle $v_k\searrow v$ in $Q$ as $k\to{\infty}$.

    The following estimate is an important step in our proof:
    \begin{equation}\label{3.3}
        \sup_{C_{r_k}}\frac{v_k}{x_n}\le
        N\cdot\frac{v_k(0,r_k)}{r_k},
        {\quad\text{{where}}\quad} N=N(n,\nu)\ge 1.
    \end{equation}
    Here both functions $v_k$ and $x_n$ are positive and satisfy the equation $Lv=0$ in the domain $\Omega_{2r}:=Q_r\cap C_{2r}$, and $v_k=0$ on the set $\Gamma_{2r}:=({\partial} Q_r )\cap ({\partial} \Omega_{2r})$ with $r=r_k$. However, we cannot apply Corollary \ref{C2.7} directly, because $\Gamma_{2r}$ is not represented as the graph of a Lipschitz function. In order to fix this gap, note that $\Gamma_{2r}$ is a surface of rotation, and the function $\psi(r)$ is non-decreasing. Therefore, $\Gamma_{2r}$ is still the graph of a Lipschitz function \emph{locally} with an absolute constant $K$ in a neighborhood of each of its point $x_0$, in a rotated coordinate system centered at $x_0$. This allows us to estimate the ratio $v_k/x_n$ near $x_0$ by the same ratio at a point strictly inside of $\Omega_{2r}$, an then use the Harnack inequality in order to get {\eqref{{3.3}}} with a constant $N=N(n,\nu)\ge 1$. This argument is similar to that in Remark \ref{R2.6}. In the rest of the proof, $N$ denotes different positive constants depending only on $n$ and $\nu$.

    Next, note that $0\le x_n\le h_k:=\psi(r_k)$ on the set $({\partial} Q)\cap {\overline}{C_k}$, hence by {\eqref{{3.3}}}, $0\le v\le v_k\le N\theta_kv_k(0,r_k)$ on this set. We also have $v=v_k$ on the rest of ${\partial} Q$. By the comparison principle, this yields
    \begin{equation}\label{3.4}
        0\le v_k-v_{k+1}\le v_k-v\le
        N\theta_kv_k(0,r_k){\quad\text{{in}}\quad} Q.
    \end{equation}
    Combining the Harnack inequality with Corollary \ref{C2.5}, we get
    \begin{equation}\label{3.5}
    0<\frac{v_k(0,r_k)}{r_k}
    \le \frac{Nv_k(0,r_{r+1})}{r_{k+1}}
    \le N\mu_k,{\quad\text{{where}}\quad}
    \mu_k:=\inf_{C_{r_{k+1}}}\frac{v_k}{x_n}.
    \end{equation}
    Further, from an elementary inequality $\inf A_k-\inf B_k\le \sup (A_k-B_k)$ and $C_{r_{k+2}}{\subset} C_{r_{k+1}}$ it follows
    \[ \mu_k-\mu_{k+1} \le \inf_{C_{r_{k+2}}}\frac{v_k}{x_n}-\inf_{C_{r_{k+2}}}\frac{v_{k+1}}{x_n}
    \le \sup_{C_{r_{k+2}}}\frac{v_k-v_{k+1}}{x_n}.\]
    Here the right side can be estimated by Theorem \ref{T2.4}. In combination with {\eqref{{3.4}}} and {\eqref{{3.5}}}, this gives us
    \[ \mu_k-\mu_{k+1} \le
    \frac{N(v_k-v_{k+1})(0,r_{k+2})}{r_{k+2}}
    \le\frac{N\theta_kv_k(0,r_k)}{r_k}
    \le N\theta_k\mu_k.\]

    As we noticed in the beginning of the proof, we can assume that $\theta_k:=h_k/r_k\le{\varepsilon}_0$ for all $k$, with a convenient choice of the constant ${\varepsilon}_0={\varepsilon}_0(n,\nu)\in (0,1)$. Choose ${\varepsilon}_0$ such that in the previous expression, $\alpha_k:=N\theta_k\le N{\varepsilon}_0\le 1/2$ for all $k$. By iteration, we obtain
    \[ \mu_{k+1}\ge (1-\alpha_k)\mu_k\ge
    (1-\alpha_k)(1-\alpha_{k-1})\cdots(1-\alpha_1)\mu_1.\]
    Finally, we use the fact that convergence of the series $\sum\alpha_j=N\sum\theta_j$ is equivalent to convergence of the product $\prod(1-\alpha_j)$. More specifically, from convexity of the function $f(\alpha):=-\ln(1-\alpha)$ it follows that its values lie between $\alpha$ and $2\ln 2\cdot\alpha$ for all $\alpha\in [0,1/2]$. Hence
    \[ -\ln\mu_{k+1}\le -\ln\mu_1
    -\sum_{j=1}^k\ln(1-\alpha_j)
    \le -\ln\mu_1+2\ln 2\sum_{j=1}^{\infty}\alpha_j
    <{\infty}\]
    for all $k$. Then $v_k(0,r_{k+1})/r_{k+1}\ge\mu_k\ge{{\rm const}}>0$ for all $k$, and by the Harnack inequality, same is true for the sequence $v_k(0,r_k)/r_k$. We can also assume that $N\theta_k\le 1/2$ in {\eqref{{3.4}}}, hence $v(0,r_k)/r_k\ge v_k(0,r_k)/2r_k\ge{{\rm const}}>0$ for all $k$.

    Now we see that the ratio $v(0,x_n)/x_n$ is separated from $0$ for $x_n=r_k:=4^{-k},\,k\ge 1$. By the Harnack inequality, this is also true for $r_{k+1}\le x_n\le r_k$, and {\eqref{{3.2}}} follows.
\end{proof}

\emph{Proof of Theorem \ref{T1.8}.}
    As in the preceding proof, we can assume that $a_{ij}$ are smooth. Replacing $r_0>0$ in {\eqref{{1.5}}} by a smaller number if necessary, we can also assume that $u$ is not identically $0$ on ${\partial} Q$. Choose an arbitrary function $g\in C({\partial} Q)$, such that $0\le g\le u$ on ${\partial} Q$, $g{\equiv} 0$ on $\Gamma:=({\partial} Q)\cap \{x_n=\psi(|x'|)\}$, and $g$ is not identically $0$. Then define $v\in C^2(Q)\cap C({\overline}{Q})$ as a solution of the equation $Lu=0$ in $Q$ with the boundary data $v=g$ on ${\partial} Q$. This function $v$ automatically satisfies all the assumptions of Lemma \ref{L3.3}, and moreover, by the comparison principle, $u\ge v>0$ in $Q$. Therefore, for the proof of {\eqref{{1.7}}}, it suffices to establish a similar property for the function $v$.

    Fix an arbitrary vector ${\bm{{l}}}=(l',l_n)\in{{\mathbb R}^n}_+$, choose a constant $K_1>0$ such that $l_n>K_1|l'|$, and another constant $K_0\in (0,K_1)$. Finally take a constant $R_0\in (0,r_0]$ according to Corollary \ref{C3.2}. This guarantees that $Q$ contains the set $V_0:=\{|x|<R_0,\,x_n>K_0|x'|\}$. In turn, by our construction $V_0$ contains the set $V_1:=\{|x|<R_0/2,\,x_n>K_1|x'|\}$, and $t{\bm{{l}}}\in V_1$ for all $t$ in an interval $(0,t_0)$. By the Harnack inequality, $v(0,tl_n)\le Nv(t{\bm{{l}}})$ for all $t\in (0,t_0)$. Now the desired estimate follows from {\eqref{{3.2}}} with $x_n=tl_n$.
\hfill $\Box$
\medskip

In the rest of the paper, we skip some details of proofs which are similar to those in the proofs of Lemma \ref{L3.3} and Theorem \ref{T1.8}. In particular, we assume that $a_{ij}$ are smooth, so that the Dirichlet problem $Lu:=\sum a_{ij}D_{ij}u=0$ in $\Omega$ with the boundary condition $u=g$ on ${\partial\Omega}$ has a classical solution for any bounded Lipschitz domain $\Omega$ and any function $g\in C({\partial\Omega})$. The following lemma covers a ``model'' case for the proof of Theorem \ref{T1.9}.

\begin{lemma}\label{L3.4}
    Let $\psi(r)$ be a non-negative, non-decreasing function on $[0,r_0]$, with $I(\psi)<{\infty}$. Define
    \begin{equation}\label{3.6}
        \begin{split}
      Q^* & := \{|x'|<r_0,\,-\psi(|x'|)<x_n<r_0\},\\
      \Gamma^* & : =({\partial} Q^*)\cap \{-x_n\in\psi(|x'|)\}.
    \end{split}
    \end{equation}
    Let $w$ be a function in $C^2(Q^*)\cap C({\overline}{Q^*})$, such that
    \[w>0,\quad Lw=0{\quad\text{{in}}\quad}Q^*;\qquad
    w=0 {\quad\text{{on}}\quad} \Gamma^*.\]
     Then the ratio $\,w(x)/|x|\,$ is bounded on $Q^*$. As in Lemma \ref{L3.3}, we assume $\psi(r)=[\psi(r-),\psi(r+)]$ for $0<r<r_0$.
\end{lemma}

\begin{proof}
    We approximate $Q^*$ by a sequence of domains $Q_k^*,\,k\ge 1$, with flat boundaries in the $r_k$-neighborhood of the origin. Namely, set
    \[Q_k^*:=\{x=(x',x_n):\;|x'|<r_0,\;
    -\psi_k(|x'|)<x_n<r_0\},\]
    where $\psi_k(r){\equiv} 0$ on $[0,r_k]$, and $\psi_k(r){\equiv} \psi(r)$ on $(r_k,r_0]$. Correspondingly, the given function $w$ will be approximated by solutions $w_k\in C^2(Q_k^*)\cap({\overline}{Q_k^*})$ of the Dirichlet problems
    \[ Lw_k=0{\quad\text{{in}}\quad} Q_k^*,\qquad
    w_k=g_k{\quad\text{{on}}\quad}{\partial} Q_k^*,\]
    where the functions $g_k\in C({\partial} Q_k^*)$ are defined by the equalities $\,g_k{\equiv} w\,$ on $({\partial} Q_k^*)\cap({\partial} Q^*)$, and $g_k{\equiv} 0$ on $({\partial} Q_k^*){\setminus}({\partial} Q^*)$. We have $Q_k^*\nearrow Q^*$, and by the comparison principle $w_k\nearrow w$ in $Q^*$ as $k\to{\infty}$, if we formally extend $w_k{\equiv} 0$ on $Q^*{\setminus} Q_k^*$.
    As in the proof of Lemma \ref{L3.3}, we can assume that $\theta_k:=h_k/r_k\le {\varepsilon}_0={\varepsilon}_0(n,\nu)$ - a small constant in $(0,1)$.

    We can apply Corollary \ref{C2.7} to the functions
    \[u:=w,\quad v:=x_n+h_{k-1}{\quad\text{{in}}\quad} D_k:=\{|x'|<r_k,\;-\psi(|x'|)<x_n<r_k\}\]
    in the same way as we did it in the proof of {\eqref{{3.3}}}. These functions are positive, satisfy $Lu=Lv=0$ in a larger domain $D_{k-1}$, and $u:=w=0$ on its ``bottom'' $({\partial} D_{k-1})\cap \{-x_n\in\psi(|x'|)\}$. Therefore,
    \[ \sup_{D_k}\frac{w}{x_n+h_{k-1}}
    \le \frac{Nw(0,r_k)}{r_k},\]
    From this estimate it follows
    \[0=w_k\le w\le N\theta_{k-1}w(0,r_k){\quad\text{{on}}\quad}({\partial} Q_k^*)\cap {\overline}{D_k}.\]
    On the rest of ${\partial} Q_k^*$, we have $w_k=w$. By the comparison principle,
    \begin{equation}\label{3.7}
    0\le w_{k+1}-w_k \le w-w_k\le N\theta_{k-1}w(0,r_k)
    {\quad\text{{in}}\quad} Q_k^*.
    \end{equation}
    In particular, assuming $N\theta_{k-1}\le N{\varepsilon}_0\le 1/2$, we get $w(0,r_k)\le 2w_k(0,r_k)$.

    Further, we apply Corollary \ref{C2.7} once again, with $v{\equiv} 1$, and then use the Harnack inequality. This implies
    \begin{equation}\label{3.8}
    \sup_{D_k}w\le Nw(0,r_k)\le Nr_{k+1}M_k,
    {\quad\text{{where}}\quad}
    M_k:=\sup_{C_{r_{k+1}}}\frac{w_k}{x_n}.
    \end{equation}
    Using inequality $\sup A_k-\sup B_k\le \sup (A_k-B_k)$ and Theorem \ref{T2.4} with $u:=w_{k+1}-w_k,\;v:=x_n$ in $C_{r_{k+2}}{\subset} C_{r_{k+1}}$, we obtain
    \[ M_{k+1}-M_k\le \sup_{C_{r_{k+2}}}\frac{w_{k+1}-w_k}{x_n}
    \le \frac{N(w_{k+1}-w_k)(0,r_{k+2})}{r_{k+2}}.\]
    Together with {\eqref{{3.7}}} and {\eqref{{3.8}}}, this implies
    \[ M_{k+1}-M_k \le N\theta_{k-1}w(0,r_k)/r_{k+2}
    \le N\theta_{k-1}M_k,\]
    so that $M_{k+1}\le (1+N\theta_{k-1})M_k$. Iterating this estimate and using the fact that from convergence of the series $\sum\theta_k$ it follows convergence of the product $\prod(1+N\theta_{k-1})$, we get the estimate $M_k\le NM_1$ for all $k\ge 1$. Finally, in order to prove the boundedness of $w(x)/|x|$, it suffices to show that its supremum over the set $Q^*\cap\{r_{k+1}<|x|\le r_k\}$, which is a subset of $D_k$, does not exceed a constant uniformly for all $k$. This is an immediate consequence of {\eqref{{3.8}}}: for each $x$ in this set,
    \[\frac{w(x)}{|x|}
    \le \frac{1}{r_{k+1}}\cdot\sup_{D_k}w
    \le NM_k\le NM_1<{\infty}.\]
    Lemma is proved.
\end{proof}

\emph{Proof of Theorem \ref{T1.9}.}
    From our assumptions it follows that the set  is a subset of $Q^*$ defined in {\eqref{{3.6}}}. Replacing $r_0>0$ by a smaller number if necessary, we can assume that $u=0$ on $({\partial\Omega})\cap Q^*$. Then the function $g$ on ${\partial} Q^*$ defined by the equalities $g{\equiv} u$ on $({\partial} Q^*)\cap\Omega$, and $g{\equiv} 0$ on $({\partial} Q^*){\setminus} \Omega$, belongs to $C({\partial} Q^*)$. Assuming that $a_{ij}$ are smooth, we can define $w\in C^2(Q^8)\cap C({\overline}{Q^*})$ as a solution to the equation $Lu=0$ in $Q^*$ with the boundary condition $w=g$ on ${\partial} Q^*$. By the comparison principle, $0<u\le w$ in $Q^*\cap\Omega$. Therefore, $u(x)/|x|$ is bounded in $\Omega\cap B_{r_0}(0)$ by Lemma \ref{L3.4}
\hfill $\Box$
\medskip

\begin{thebibliography}{}

\bibitem[1]{A67}  A.D. Aleksandrov, \emph{Majorization of solutions of second-order elliptic equations}, Vestnik Leningrad Univ. \textbf{21}, no. 1 (1966), 5--25 (in Russian). English transl. in Amer. Math. Soc. Transl. (2) \textbf{68} (1968), 120--143.

\bibitem[2]{B82}  P.E. Bauman, \emph{Properties of nonnegative solutions of second-order elliptic equations and their adjoints,} Ph. D. Thesis, University of Minnesota, 1982.

\bibitem[3]{B84}  P.E. Bauman, \emph{Positive solutions of elliptic equations in non-divergence form and their adjoints}, Arkiv f\"{o}r Mathematik, \textbf{22} (1984), 153--173.

\bibitem[4]{CFMS}  L.A. Caffarelli, E.B. Fabes, S. Mortola and S. Salsa, \emph{Boundary behavior of nonnegative solutions of elliptic
operators in divergence form}, Indiana J. of Math., \textbf{30} (1981), 621--640.

\bibitem[5]{CH} R. Courant and D. Hilbert, \emph{Methods of Mathematical Physics. Volume II}. Interscience, New Tork, 1983.

\bibitem[6]{FSY} E.B. Fabes, M.V. Safonov and Yu Yuan, \emph{Behavior near the boundary of positive solution of second order parabolic equations. II}, Trans. Amer. Math. Soc. \textbf{351}, no. 12 (1999), 4947--4961.

\bibitem[7]{GT} D. Gilbarg and N.S. Trudinger, \emph{Elliptic Partial Differential Equations of Second Order}. Springer-Verlag, 1983.

\bibitem[9]{G33} G. Giraud, \emph{Probl\`emes de valeurs \`a la fronti\`ere relatifs \`a certaines donn\'es discontinues}, Bull. de la Soc. Math. de France, \textbf{61} (1933), 1--54.

\bibitem[9]{Kh70} B.N. Him\v{c}enko, \emph{On the behavior of solutions of elliptic equations near the boundary  of a domain of type $A^{(1)}$}, Dokl. Akad. Nauk SSSR \textbf{193} (1970), 304--305 (in Russian). English transl. in Soviet Math. Dokl. \textbf{11} (1970), 943--944.

\bibitem[10]{H52} E. Hopf, \emph{A remark on linear elliptic differential equations of second order,} Proc. Amer. Math. Soc., \textbf{3} (1952), 791--793.

\bibitem[11]{KH80} L.I. Kamynin and B.N. Khimchenko, \emph{Development of Aleksandrov's theory of the isotropic extremum principle}, Differents. Uravn. \textbf{16} (1980), 280--292 (in Russian). English transl. in Differential Equations \textbf{16} (1980), 181--189.

\bibitem[12]{Kh69} B.N. Khimchenko, \emph{The behavior of the superharmonic function near the boundary of a domain of type $A^{(1)}$}, Differents. Uravn. \textbf{5}  (1969), 1845--1853 (in Russian). English transl. in Differential Equations \textbf{5} (1969), 1371--1377.

\bibitem[13]{K}  N.V. Krylov, \emph{Nonlinear Elliptic and Parabolic
Equations of Second Order}, Nauka, Moscow, 1985 (in Russian). English transl.: Reidel, Dordrecht, 1987.

\bibitem[14]{KS80}  N.V. Krylov and M.V. Safonov, \emph{A certain property of solutions of parabolic equations with measurable coefficients}, Izvestia Akad. Nauk SSSR, ser. Matem., \textbf{44}(1980), 161--175 (in Russian). English transl. in Math. USSR Izvestija, \textbf{16} (1981), 151--164.

\bibitem[15]{LU88} O.A. Ladyzhenskaya and N.N. Ural'tseva, \emph{Estimates on the boundary of a domain for the first derivatives of functions satisfying an elliptic or parabolic inequality}, Trudy Mat. Inst. Steklov \textbf{179} (1988), 102--125 (in Russian). English transl. in Proc. Steklov Inst. Math. \textbf{179} (1989), 109--135.
    
\bibitem[16]{L71}  E.M. Landis, \emph{Second Order Equations of Elliptic and Parabolic Type}. Nauka, Moscow, 1971 (in Russian). English transl. in Transl. Matem. Monorgraphs \textbf{171}, Amer. Math. Soc., Providence, RI, 1998.

\bibitem[17]{L85} G.M. Lieberman, \emph{Regularized distance and its applications}, Pacific J. Math. \textbf{117} (1985), 329--352.

\bibitem[18]{M67} K. Miller, \emph{Barriers on cones for uniformly elliptic operators}, Ann. Mat. Pura Appl. (4) \textbf{76} (1967), 93--105.

\bibitem[19]{O52} O.A. Oleinik, \emph{On properties of solutions of certain boundary problems for equations of elliptic type,} Mat. Sb. (N. S.) \textbf{30} (1952), 695--702 (in Russian).

\bibitem[20]{PW} M.H. Protter and H.F. Weinberger, \emph{Maximim Principles in Differential Equations}. Englewood Cliffs, N.J., Prentice-Hall, 1967.

\bibitem[21]{S80} M.V. Safonov, \emph{Harnack inequality for elliptic equations and the H\"older property of their solutions}, Zap. Nauchn. Sem. Leningrad. Otdel. Mat. Inst. Steklov (LOMI), \textbf{96} (1980), 272--287 (in Russian). English transl. in J. Soviet Math., \textbf{96} (1983), 851--863.
    
\bibitem[22]{St70} E. Stein, \emph{Singular Integrals and Differentiability Properties of Functions}. Princeton. Princeton University Press, 1970.

\bibitem[23]{Z10} M.S. Zaremba, \emph{Sur un probl\`eme mixte relatif \`a l' \'equation de Laplace}, Bull. Intern. de l' Acad. Sci. de Cracovie. S\'erie A, Sci. Math. (1910), 313--344.

\end{thebibliography}{}

\end{document}
***********************

