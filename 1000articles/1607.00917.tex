 \documentclass{amsart}
 
\usepackage{amscd}

 

\begin{document}
\title{Deformation of Koszul algebras and the Duflo Isomorphism theorem}
\author{Murray Gerstenhaber}
\address{Department of Mathematics\\  University of Pennsylvania\\
  Philadelphia, PA 19104-6395, USA}
\email{mgersten@math.upenn.edu}
\subjclass[2010]{16S80 (primary); 16S37, 16S30 (secondary)}
\keywords{cohomology, deformation, Duflo isomorphism, Koszul algebra, Poincar{\'e}-Birkhoff-Witt theorem}

\begin{abstract}\noindent
Let $\mathfrak g$ be a finite dimensional Lie algebra over a field  $\mathbf k$, $U\mathfrak g$  be its enveloping algebra and $S\mathfrak g$ be the symmetric algebra on $\mathfrak g$. 
Extending the work of Braverman and Gaitsgory on the deformation of Koszul algebras and the   Poincar{\'e}-Birkhoff-Witt theorem we obtain a generalized Duflo isomorphism which is valid also over fields of finite characteristic: $H_{\text{Lie}}^n(\mathfrak g, S\mathfrak g) \cong H_{\text{Hoch}}^n(U\mathfrak g,U\mathfrak g)$ for all $n < \operatorname{char}\mathbf k$. This implies, in particular, that Duflo's classic theorem, which is the special case in characteristic zero of dimension zero, in fact holds in all characteristics and the generalized theorem holds whenever $\dim \mathfrak g < \operatorname{char} \mathbf k$. \end{abstract}

\keywords{}
\maketitle

\newtheorem{theorem}{Theorem}
\newtheorem{corollary}{Corollary}
\newtheorem{lemma}{Lemma}
\newtheorem{defn}{Definition}

       

\vspace{-7mm}
\medskip 

The classic Duflo isomorphism theorem  \cite{Duflo:1977} asserts that if ${\mathfrak{g}}$ is a finite dimensional real Lie algebra, $U{\mathfrak{g}}$ its universal enveloping algebra, and $S{\mathfrak{g}}$ the symmetric algebra on ${\mathfrak{g}}$, then there is an algebra isomorphism
$(S{\mathfrak{g}})^{\mathfrak{g}}  \cong Z(U{\mathfrak{g}})$, where 
$(S{\mathfrak{g}})^{\mathfrak{g}}$ is the ring of invariants of $S{\mathfrak{g}}$ under the action of ${\mathfrak{g}}$ and $Z(U{\mathfrak{g}})$ is the center of $U{\mathfrak{g}}$.  Remarkably, it was proven before the best context in which to understand it was generally available, namely, the deformation theory of algebras, cf e.g., \cite{G:Def1},  \cite{G:Def3}, \cite{G:Def4}.

Kontsevich, using his quantization of Poisson manifolds, generalized Duflo's theorem to show that ${\ensuremath{H_{\text{Lie}}}}^*({\mathfrak{g}}, S{\mathfrak{g}}) \cong {\ensuremath{H_{\text{Hoch}}}}^*(U{\mathfrak{g}},U{\mathfrak{g}})$;  for discussion and references, cf. \cite{Cattaneo et al:Deformation}, \cite{CalaqueRossi:Duflo}. The classic theorem is the special case of dimension 0, since $(S{\mathfrak{g}})^{\mathfrak{g}} = {\ensuremath{H_{\text{Lie}}}}^0({\mathfrak{g}}, S{\mathfrak{g}})$ and $Z(U{\mathfrak{g}}) = {\ensuremath{H_{\text{Hoch}}}}^0(U{\mathfrak{g}},U{\mathfrak{g}})$.  Transcendental methods, however, are generally not adaptable to finite characteristic.
Braverman and Gaitsgory, at the suggestion of Joseph Bernstein, proved essential results on the deformation of Koszul algebras and the   Poincar{\'e}-Birkhoff-Witt theorem, \cite{BravermanGaitsgory:PBW}. Extending their work, we obtain a more general Duflo isomorphism. 

\begin{theorem}\label{Duflo}
Let $\mathfrak g$ be a finite dimensional Lie algebra over a field  $\mathbf k$,  $U\mathfrak g$ be its  universal enveloping algebras, and  $S\mathfrak g$ be the symmetric algebra on $\mathfrak g$.  Then $H_{\text{Lie}}^n(\mathfrak g, S\mathfrak g) \cong H_{\text{Hoch}}^n(U\mathfrak g,U\mathfrak g)$ for all $n < \operatorname{char}\mathbf k$. The isomorphism preserves the cup product in dimensions less than ${\operatorname{ch}} k$. \end{theorem}
In particular, Duflo's classic theorem (the case $n=0$) holds over an arbitrary  field, and the full generalized  Duflo thorem holds for ${\mathfrak{g}}$ if $\dim{\mathfrak{g}}  < {\operatorname{ch}} {\ensuremath{\mathbf{k}}}$.

\section{Koszul algebras and the Poincar{\'e}-Birkhoff-Witt theorem}
Let $V$ be a vector space over a field ${\ensuremath{\mathbf{k}}}$, and $T = TV = \bigoplus_{i=0}^{\infty} T^i$ be its tensor algebra, where $T^i = V^{\otimes i}$, $R$ be a subspace of $T^2$ and $J(R)$ be the two-sided ideal of $T$ which it generates. A \emph{homogenous quadratic algebra} algebra is one of the form  $Q(V,R) = T/J(R)$.  It inherits a $Z^+$ grading from $TV$ and is generated by the subspace consisting of  its elements of degree 1, which can be identified with $V$.  When $A$ is $Z$ graded and $M$ a $Z$ graded $A$ bimodule, the space of homogeneous $n$ cochains of degree $m$ will be denoted by $C^n_m(A,M)$, and similarly for cycles and cohomology.  One of the many equivalent definitions of a Koszul algebra $A$ is that it is a Noetherian, $Z^+$ graded algebra $A= \bigoplus_{i=0}^{\infty}A_i$ over a field ${\ensuremath{\mathbf{k}}}$ with $A_0 = {\ensuremath{\mathbf{k}}}$ such that $H^i_j(A,M) = 0$ for any $Z^+$ graded bimodule $M$ whenever $i <-j$.  It follows that $A$ is of the form $Q(V,R)$ with $V= A_1$ and $\dim V < \infty$ (but not every such algebra is Koszul).  

The tensor algebra $T$ has an increasing filtration with $F^jT = \bigoplus_{i=0}^j T^i$.  Let $P$ be a subspace of $F^2T = {\ensuremath{\mathbf{k}}}\oplus V \oplus (V\otimes V)$. An \emph{inhomogeneous quadratic algebra} is one of the form $Q(V, P) = T/J(P)$  where  $J(P)$ is the two-sided ideal generated by $P$ in $T$.  It inherits a filtration from $T$. Denote its associated graded algebra by ${\operatorname{gr}} Q(V,P)$.  If $R$ is the projection of $P$ on $T^2$ then there is a natural map $Q(V,R) \to {\operatorname{gr}} Q(V,P)$.  The \emph{generalized Poincar{\'e}-Birkhoff-Witt theorem} gives the necessary and sufficient conditions for this morphism to be an injection, and hence necessarily a surjection. In the notation of Braverman and Gaitsgory, they prove that these are  
$$ \text(I): P\cap F^1T = 0, \quad \text{and (J)}: (F^1(T)\cdot P \cdot F^1(T)) \cap F^2(T) = P.$$
The necessity is evident. 

Suppose that a ${\ensuremath{\mathbf{k}}}$ algebra $A$ is deformed to $A_{\hbar}$ with multiplication 
\begin{equation} \label{star}
 a\star b = m_{\hbar}(a,b) = m_0(a,b) + \hbar m_1(a,b) + \hbar^2m_2(a,b) + \cdots,
 \end{equation}
 where $m_0(a,b) = ab$, the original multiplication, and $\hbar$ is the deformation parameter. Following \cite{BravermanGaitsgory:PBW}, call this a \emph{graded deformation} if $A_{\hbar}$ remains graded when one sets $\deg \hbar = 1$. This implies that all $m_i$ are homogeneous of degree $-i$.  If $A$ is $Z^+$ graded then it follows that for any $a,b \in A$ we will have $m_i(a,b) = 0$ for sufficiently large $i$, and therefore that $A_{\hbar}$ is defined over ${\ensuremath{\mathbf{k}}}[\hbar]$ (rather than over ${\ensuremath{\mathbf{k}}}[[\hbar]]$, as would be the  case in general). It is then meaningful to specialize $\hbar$ to any element of ${\ensuremath{\mathbf{k}}}$.  If $A$ is generated by its elements of degree 1 then one sees also that replacing the parameter $\hbar$ by $\lambda\hbar$, where $\lambda$ is any invertible element of ${\ensuremath{\mathbf{k}}}$ gives an isomorphic algebra, so the deformation is a jump deformation in the terminology of \cite{G:Def4}. Therefore, no new cohomology classes can be created by specialization of the deformation parameter to a unit.

The generalized PBW theorem
was proven by Polishchuk and Positseslky \cite{PP:Quadratic} in the case where $P \subset V \oplus (V\otimes V)$, and shortly thereafter by Braverman and Gaitsgory \cite{BravermanGaitsgory:PBW} in the general case where $P \subset F^2T$. Significantly, however,  Braverman and Gaitsgory showed that under the conditions (I) and (J), if $Q(V, R)$ is Koszul then there is a graded jump deformation of $Q(V,R)$ whose specialization at $\hbar = 1$ is $Q(V,P)$; the generalized PBW theorem follows immediately.  The classic PBW theorem is the case where $V = {\mathfrak{g}}$ for some finite-dimensional Lie algebra and $P$ is the submodule of $TV = T{\mathfrak{g}}$ spanned by all $a\otimes b -b \otimes a - [a,b], a,b \in {\mathfrak{g}}$, with  $U{\mathfrak{g}} = T{\mathfrak{g}}/J(P)$; $Q(V,P)$ is then the universal enveloping algebra $U{\mathfrak{g}}$ of ${\mathfrak{g}}$ and $Q(V,R)$ is $S{\mathfrak{g}}$.   Condition (I) is immediate and Braverman and Gaitsgory show that condition (J) is equivalent in this case to the Jacobi identity, whence the notation. The ring $S{\mathfrak{g}}$ is Koszul by the theorem of Hochschild-Kostant-Rosenberg, which asserts that setting $A = SV$ there is a linear isomorphism $H^*(A, A)\cong \bigwedge \operatorname{Der} A$. (It does not preserve the algebra structure and in general there is no skew cocycle representing any cohomology class.) Further,  $H^*(A,M) = \bigwedge V^*\otimes M$ for any $A$ module $M$, so if $M$ is $Z^+$ graded then the preceding criterion to be Koszul is satisfied. Note that the cohomology vanishes in dimensions greater than $\dim V$. The conclusion of the classic PBW theorem is that ${\operatorname{gr}} U{\mathfrak{g}}$, as a vector space, can be naturally identified with $S{\mathfrak{g}}$. This is often expressed by saying that  if one takes an ordered basis $x_1,\dots, x_N$ of ${\mathfrak{g}}$ then the $x_{i_1}x_{i_2}\cdots x_{i_r}$ with $i_1 \le i_2 \le \cdots \le  i_r$  form a basis for $U{\mathfrak{g}}$.
 
If $A = Q(V,R)$ then we can define the \emph{Koszul subcomplex} $\tilde K^{\bullet}(A)$ of the standard bar complex $B^{\bullet}(A)$ as follows. Set
\begin{multline}
K^0(A) = {\ensuremath{\mathbf{k}}}, \quad K^1(A) = V,\quad K^i(A) = \bigcap_{j=o}^{i-2}
V^{\otimes j} \otimes R \otimes V^{\otimes i-j-2}, \, i\ge2,\\ \text{and set} \quad \tilde K^i(A) = A\otimes K^i(A)\otimes A.
\end{multline}
When $A$ is Koszul,  its Koszul complex  $\tilde K^{\bullet}(A)$ is known to be a projective (in fact free) resolution of $A$; Braverman and Gaitsgory \cite{BravermanGaitsgory:PBW} show that the converse is also true.  It follows that if $Q(V,R)$ is Koszul then the cohomology class of an $n$ cocycle $f$ is completely determined by its restriction to $K^n(A)$, and therefore, in particular, by its values when all of its arguments lie in $V$.

From \cite{GS:Hodge}, if $A$ is a commutative algebra defined over the rationals, then the Gerstenhaber-Schack or `homological' idempotents\footnote{These idempotents were later labeled `Eulerian' idempotents by J.-L. Loday, \cite{Loday:Euler}, leading to their occasional misattribution; significantly, Loday proved that the decomposition extends to cyclic cohomology and, in modified form, to positive characteristics.} in the group rings of the symmetric groups split the Hochschild complex $C^{\bullet}(A,A)$ into an infinite direct sum of subcomplexes, but in dimension $n$ there are only $n$ components the definition of which requires only that we can divide by $n!$. The `top' component is always the totally skew or alternating one, obtained by applying
the skew-symmetrizing idempotent operator ${\operatorname{sk}} = (1/n!)\sum_{\sigma\in S_n} \operatorname{sgn}(\sigma)\sigma$ to cochains, where $S_n$ is the $n$-th symmetric group. In particular, the skew part of a cocycle is again a cocycle, and is not a coboundary unless it vanishes.

In $T^nV = V^{\otimes n}$, let $R$ be the span of all $v\otimes w -w\otimes v$ and let $\tau_{ij}$ denote the interchange of the $i$th and $j$th tensor factors.    Then $\tau_{i,i+1}$ sends $K^n(A)$ to its negative, whence so do all $\tau_{ij}$. Therefore, every $K^n$ is alternating or skew, and it follows in particular that if $n!$ is a unit then any class in ${\ensuremath{H_{\text{Hoch}}}}(S{\mathfrak{g}}, S{\mathfrak{g}})$ contains a (necessarily unique) skew $n$ cocycle.
Therefore, we have
\begin{lemma}\label{projection} 
Suppose that $f$ is an $n$-cocycle of $SV$, where $V$ is a finite-dimensional vector space over a field ${\ensuremath{\mathbf{k}}}$. If $n!$ is a unit then $f$ is a coboundary if and only if the restriction of  ${\operatorname{sk}} f$ to $T^nV$ vanishes. $\Box$
\end{lemma} 

\section{Lifting of cocycles}  
 Following \cite{G:Def4}, define an $n$-cocycle $f\in Z^n(A,A)$ to be \emph{liftable} to a cocycle of $A_{\hbar}$ if there is a formal power series 
 $f_{\hbar} = f + \hbar f_1 + \hbar^2 f_2 + \cdots$ which is a cocycle relative to the deformed multiplication.  Denoting the multiplication in $A_{\hbar}$ by  $m_{\hbar}$, the condition is that $\delta_{\hbar} f = -[m_{\hbar}, f_{\hbar}]_{\hbar} = 0$, where $\delta_{\hbar}$ is the coboundary operator for $A_{\hbar}$ and the subscript indicates that the bracket $[m_{\hbar}, f_{\hbar}]$ must be computed using the multiplication in $A_{\hbar}$. However,  writing $m_{\hbar} = m_0 + \hbar m_1 + \hbar^2m_2 + \cdots$ as in \eqref{star}, it is to be understood that the $m_i$ are 2-cochains of $A$  and operate using the original multiplication. With this it follows  
that for all $i$ we must have
 \begin{equation}\label{obstruction}
 [m_0, f_i] + [m_1,f_{i-1}] +\cdots+[m_i,f_0]  = 0.
 \end{equation}

It is easy to check that every cocycle of $A_{\hbar}$ is a lift of a cocycle of $A$, so the foregoing implies that  
the Hochschild cohomology of $A_{\hbar}$ can be computed from that of $A$ as the liftable cocycles modulo those which when lifted become coboundaries \emph{after the adjunction of $\hbar^{-1}$}, \cite{G:Def4}, \cite{GerstGiaquint:Weyl}. For a cocycle which is not a coboundary is never liftable to a coboundary, but  might be liftable to a cocycle which has $\hbar$ torsion, i.e, becomes a coboundary after multiplication by some positive power of $\hbar$. The primary obstruction to the lifting of a cocycle $f\in Z^n(A,A)$ is the cohomology class of its bracket with the infinitesimal of the deformation; if it vanishes then $f$ can be lifted  to first order, i.e., modulo $\hbar^2$. In general there are successive obstructions, passing each of which in turn allows extension of $f_{\hbar}$ modulo a higher power of  $\hbar$.  An $f \in Z^n$ which is already a coboundary can always be lifted to a coboundary, since if $f = \delta g$ then $\delta_{\hbar}g$ is a lift of $f$, the computation here being performed in $A_{\hbar}$.  

In general, $A_{\hbar}$ is defined over the power series ring ${\ensuremath{\mathbf{k}}}[[\hbar]]$ and without some topology it is not possible to specialize $\hbar$, but if $A$ is $Z^+$ graded and  $A_{\hbar}$ is a graded deformation, then
the cohomology of the algebra $A_1$ obtained by specializing $\hbar$ to 1 consists precisely of the liftable cocycles modulo those lifting to coboundaries. Its cohomology is thus just a subquotient of that of the original. 

When an algebra is deformed so is its bar complex.  Let $K^{\bullet}(A)$ be the Koszul complex of a Koszul algebra $Q(V, R)$, and suppose that $A$ is has a graded deformation to an algebra $A_{\hbar}$ which is the homogeneous form of a deformation $Q(V, P)$, where $P$ satisfies the conditions (I) and (J) of {Braverman-Gaitsgory}.  The deformation does not carry $K^{\bullet}(A)$ into itself since the boundary operator $\partial_{\hbar}$ has been altered, but if each $K^n(A)$ is enlarged to include all the boundaries, we will still have an exact subcomplex of the deformed bar complex. For simplicity assume that $P \subset V + (V\otimes V)$, so the infinitesimal of the deformation is a map $\alpha: R \to V$ in the notation of {Braverman-Gaitsgory}. Then no terms in ${\hbar}^2$ appear since those containing ${\hbar}$ are already boundaries.  (Braverman and Gaitsgory considered the more general case where the value of a map $R\otimes R \to {\ensuremath{\mathbf{k}}} +V +(V\otimes V)$  also has a component in ${\ensuremath{\mathbf{k}}}$. However, Hochschild cohomology can be computed using only normalized cochains, namely those which vanish whenever any argument is in ${\ensuremath{\mathbf{k}}}$, so the foregoing would continue to hold.)

Denote the resulting complex by $K^{\bullet}(A_{\hbar})$, which is still contained in $TV$ and contains $K^{\bullet}(A)$.  Then $A_{\hbar}\otimes K^{\bullet}(A_{\hbar})\otimes A_{\hbar}$ is a free resolution of  $A_{\hbar}$ so the cohomology of $\operatorname{Hom}_{\ensuremath{\mathbf{k}}}(K^{\bullet}(A_{\hbar}), A)$ is $H^*(A_{\hbar}, A_{\hbar})$, since as  vector spaces, $A_{\hbar} \cong A$.  Since in $\operatorname{Hom}_{\ensuremath{\mathbf{k}}}(K^{\bullet}(A), A)$ all coboundaries are identically
zero, it follows that if ${\overline f} \in \operatorname{Hom}_{\ensuremath{\mathbf{k}}}(K^n(A_{\hbar}), A)$ then ${\overline f}$ vanishes on any element of $\partial K^{n+1}(A)$ where $\partial$ is the original boundary operator.
Therefore, if $f \in Z^n(A,A)$ and $[\alpha, f] = \delta g$ for some $g \in C^n(A,A)$ then ${\overline f} = f| K^n(A_{\hbar})$ is a cocycle and therefore represents a cohomology class in $H^n(A_{\hbar}, A_{\hbar})$. That is, $f$ can be lifted, so its primary obstruction is its only obstruction. A similar argument shows that $f$ lifts to a coboundary if and only $f = [\alpha, g]$ for some cocycle $g$. We thus have the following fundamental lemma about the deformation of Koszul algebras.
\begin{lemma}\label{Klift}
Let $A_{\hbar}$ be a graded deformation of a Koszul algebra $A$. Then an $n$ cocycle $f$ of $A$ lifts to one of $A_{\hbar}$ if and only if its primary obstruction vanishes, and it lifts to a coboundary if and only it is the primary obstruction to the lifting of an $n-1$ cocycle. $\Box$
\end{lemma}

\section{Proof of Theorem 1}

Since $U{\mathfrak{g}}$ is a graded deformation of the Koszul algebra $S{\mathfrak{g}}$, $H^n(U{\mathfrak{g}},U{\mathfrak{g}})$ consists of the liftable $n$ cocycles $f \in Z^n(S{\mathfrak{g}}, S{\mathfrak{g}})$ modulo those lifting to coboundaries. When $n!$ is a unit, every liftable cohomology class is representable by a unique skew $f$, which may then be restricted to $T^n{\mathfrak{g}}$ and viewed a a Lie cochain, which we will tacitly do. Let $\alpha$ now be the Lie multiplication in ${\mathfrak{g}}$. 
\begin{lemma}\label{sk}
Suppose that $n!$ is a unit. If $f$ is skew, then $(-1)^{n-1}{\operatorname{sk}}([\alpha,f]) = (2/n)!\,{\ensuremath{\delta_{\text{Lie}}}}(f)$.
\end{lemma}
\noindent\textsc{Proof.} 
Writing $[a,b]$ for the Lie multiplication in ${\mathfrak{g}}$, we have 
\begin{multline}\label{dL}
({\ensuremath{\delta_{\text{Lie}}}} f)(a_0,a_1,\dots,a_n) = \sum_{i=0}^n (-1)^i[a_i, f(a_0,\dots, \check{a}, \dots, a_n)]\\ +\sum_{0 \le i < j \le n}(-1)^{i+j}f([a_i,a_j],a_0,\dots,\check{a_i}, \dots, \check{a_j},\dots, a_n), 
\end{multline}
while
\begin{multline}\label{muf}
(-1)^{n-1}[\alpha,f](a_0,\dots, a_n) =[a_0, f(a_1, \dots, a_n)] +(-1)^{n-1} [f(a_0,\dots,a_{n-1}), a_n] \\ -\sum_{i=0}^{n-1}f(a_0, \dots, a_{i-1}, [a_i, a_{i+1}], a_{i+2}, \dots, a_n),
\end{multline}
where $\check{a}$ denotes the omission of $a$.  The terms in \eqref{muf} all appear in \eqref{dL} with the correct signs. Summing \eqref{muf} over all signed permutations,  all terms of \eqref{dL} now appear with the correct signs (there is never any cancellation), and each term appears exactly $2(n-1)!$ times. 
$\Box$\medskip

Assume still that $n!$ is a unit. The desired morphism $\Phi:{\ensuremath{H_{\text{Hoch}}}}^n(U{\mathfrak{g}},U{\mathfrak{g}}) \to {\ensuremath{H_{\text{Lie}}}}^n({\mathfrak{g}}, S{\mathfrak{g}})$ is then given as follows. Represent a class in ${\ensuremath{H_{\text{Hoch}}}}^n(U{\mathfrak{g}},U{\mathfrak{g}})$  by its unique skew liftable cocycle  $f \in Z^n(S{\mathfrak{g}}, S{\mathfrak{g}})$. Denote the restriction of $f$ to $T^n{\mathfrak{g}}$ by $f_{\text{L}}$. By Lemma \ref{Klift},  $f_{\text{L}}$ is a Lie cocycle, and it is a coboundary if and only if $f$ itself lifts to a coboundary; $\Phi$ is defined by sending the class $[f]$ represented by $f$ to $n!\,[f_{\text{L}}]  \in {\ensuremath{H_{\text{Lie}}}}({\mathfrak{g}}, S{\mathfrak{g}})$.  This is a monomorphism; to see that it is onto, suppose that $F$ is a Lie cocycle.  It is in particular a map $\bigwedge^n{\mathfrak{g}} \to {\mathfrak{g}}$ which we can extend to a skew $n$ cocycle $f \in Z^n(S{\mathfrak{g}}, S{\mathfrak{g}})$ as a multiderivation, i.e., a derivation as a function of each argument; $(1/n!)f$ is the necessary preimage.  

When a Lie algebra ${\mathfrak{g}}$ acts by derivations on a ring $R$ there is a natural associative, graded commutative  product in ${\ensuremath{H_{\text{Lie}}}}({\mathfrak{g}},R)$ defined by setting, for cocycles $F^m, G^n$ of dimensions $m,n$ respectively,

\begin{multline*}\label{wedge}
(F^m \wedge G^n)(a_1,\dots, a_{m+n}) = \\
\sum _{\sigma \in {\operatorname{Sh}}_{k,m}} {\operatorname{sgn}}(\sigma) F^m (a_{\sigma 1},\dots, a_{\sigma m}) \,G^n(a_{\sigma (m+1)},\dots, a_{\sigma (m+n)}).
 \end{multline*} 
Here ${\operatorname{Sh}}_{k,m} $ is the subset of $(k,m)$ shuffles, i.e., permutations $\sigma$ of  $\{1, 2, \dots, m+n\}$ such that $\sigma 1 <\sigma 2 <\cdots< \sigma m$ and $\sigma(m+1) < \sigma(m + 2) < \cdots < \sigma(m+n)$.
Then ${\ensuremath{\delta_{\text{Lie}}}}(F\wedge G) = {\ensuremath{\delta_{\text{Lie}}}} F \wedge G +(-1)^mF\wedge {\ensuremath{\delta_{\text{Lie}}}} G$, so this `wedge' product descends to cohomology. (The proof is essentially the same as that of Lemma \ref{sk}.)  With this, $\Phi$ is an algebra morphism.  We conjecture that there should exist a natural Gerstenhaber algebra structure on ${\ensuremath{H_{\text{Lie}}}}({\mathfrak{g}}, S{\mathfrak{g}})$ with which $\Phi$ becomes an isomorphism of such algebras.

\section{Concluding remarks}
The theorem raises the question of whether the quantization of a smooth Poisson variety should require only that the characteristic be greater than the dimension and also of whether there may be any number-theoretic implications.

When there is a short proof for a previously difficult theorem it does not mean that the theorem is trivial but that we have found a context which clarifies its meaning. Deformation theory provides a context for quantization \cite{BFFLS}, \cite{DitoSternheimer:Genesis}, wave-particle duality \cite{G:Semigroups}, Sridharan's theorem \cite{GerstGiaquint:Weyl}, the Poincar{\'e}-Birkhoff-Witt theorem \cite{BravermanGaitsgory:PBW}, and now for the Duflo isomorphism theorem, which it also strengthens.

 \begin{thebibliography}{10}
 
 \bibitem{BFFLS}
F.~Bayen, M.~Flato, C.~Fr{\o}nsdal, A.~Lichnerowicz, and D.~Sternheimer.
\newblock {Deformation theory and quantization, I and II.}
\newblock{\em Ann. of Phys.}, 111:61--151, 1978.

\bibitem{BravermanGaitsgory:PBW}
A.~Braverman and D.~Gaitsgory.
\newblock{Poincar{\'e}--Birkhoff--Witt Theorem for Quadratic Algebras of Koszul Type}.
\newblock{\em J. of Algebra}, 181:315--328, 1996.
\newblock{\texttt {arXiv:hep-th 9411113v1 16 Nov 1994}}.

\bibitem{CalaqueRossi:Duflo}
D.~Calaque and C.~Rossi.
\newblock{\textsc{Lectures on Duflo isomorphisms in Lie algebras and complex geometry}}
\newblock{em EMS Series of Lectures in Mathematics}, No. 14, EMS Publishing House, 2011).
\newblock{Preliminary version available at \texttt{http://www.math.univ-montp2.fr/~calaque/publications.html}}.

\bibitem{Cattaneo et al:Deformation}
A.~Cattaneo, B.~Keller, C.~Torossian, and A.~Brugui{\`e}res.
\newblock{\textsc D{\'e}formation, Quantification, Th{\`e}orie de Lie}.
\newblock{\em Panoramas et Synth{\`e}ses}, No. 20, Soc. Math. France, 2005.

\bibitem{DitoSternheimer:Genesis}
G.~Dito and D.~Sternheimer.
\newblock{Deformation quantization: genesis, developments
and metamorphoses}, pp. 9--54 in
\newblock{G. Halbout, ed., \textsc{Deformation Quantization, IRMA Lectures in Math. Theoret. Phys.
1},}
\newblock{Walter de Gruyter, Berlin} 2002.
\newblock{\texttt {arXiv:0201168v1 [math.QA] 18 Jan 2002}}.

\bibitem{Duflo:1977}
M.~Duflo.
\newblock{Op{\'e}rateurs diff{\'e}rentiels bi-invariants sur un groupe de Lie}.
\newblock{\em Ann. Sci. {\'E}cole Norm. Sup.}, 10:265--288, 1977.

\bibitem{G:Def1}
M.~Gerstenhaber.
\newblock {On the deformation of rings and algebras}.
\newblock {\em Ann. of Math.}, 79:59--103, 1964.

\bibitem{G:Def3}
M.~Gerstenhaber.
\newblock {On the deformation of rings and algebras:III}.
\newblock {\em Ann. of Math.}, 88:1--34, 1968.

\bibitem{G:Def4}
M.~Gerstenhaber.
\newblock {On the deformation of rings and algebras IV}.
\newblock {\em Ann. of Math.}, 99:257--276, 1974.

\bibitem{G:Semigroups}
M.~Gerstenhaber.
\newblock{Semigroups, wave-particle duality, and quantization of phase space}
\newblock {\texttt{arXiv:1505.05859 May 21, 2015}},
\newblock {to appear {\em Letters Math. Phys.}}

\bibitem{GerstGiaquint:Weyl}
M.~Gerstenhaber and A.~Giaquinto.
\newblock{On the cohomology of the Weyl
algebra, the quantum plane, and the q-Weyl algebra}.
\newblock{\em J. Pure Appl. Algebra}, 218:879--887, 2014.

\bibitem{GS:Hodge}
M.~Gerstenhaber and S.~D.~Schack.
\newblock{A Hodge-type decomposition for commutative algebra cohomology}
\newblock{\em J. Pure Appl. Algebra}, 48:229-247,1987.

\bibitem{Loday:Euler}
J.-L.~Loday.
\newblock{Partition eulerienne et operations en homologie cyclique}
\newblock{ \em C. R. Acad. Sci. Paris Ser. I Math.}, 307:283-286, 1988.
 

\bibitem{PP:Quadratic}
A.~E,~Polishchuk and L.~E.~Positselsky.
\newblock{On quadratic algebras}
\newblock{preprint, Moscow University, 1990 (Russian); expanded and published as}
\newblock{\textsc{Quadratic Algebras}}
\newblock{\em University Lecture Series}, v. 37, 2005.
\newblock{\em Amer. Math. Soc., Providence, R.I.}

\end{thebibliography}

\end{document}
Quadratic algebras / Alexander Polishchuk, Leonid Positselski.
Access, holdings & availability
Library location

    Math/Physics/Astronomy Library
    Call no.: QA247 .P596 2005
    Checked out.
    Request it
    (PennKey)

[ Report an error on this page (PennKey) ]
[ Request enhanced cataloging (PennKey) ]
Go to Google Books
Record details

Author/Creator:
    Polishchuk, Alexander, 1971- 
Publication:
    Providence, R.I. : American Mathematical Society, c2005.
Series:
    University lecture series (Providence, R.I.) ; 37.
    University lecture series, 1047-3998 ; v. 37 
Format/Description:
    Book
    xii, 159 p. : ill. ; 26 cm. 
Subjects:
    Quadratic fields.
    Associative rings.
    Commutative rings.
    Stochastic processes.
Notes:
    Includes bibliographical references (p. 155-159).
Contributor:
    Positselski, Leonid, 1973- 
ISBN:
    0821838342 (acid-free paper)
OCLC:
    61247339

\section{Explicit deformation formulae} We recall here some propositions following immediately from basic results of \cite{G:Def3}, particularly Chapter II, p. 13 ff. (We give them here only in a simpler form that will be needed; they were stated there in considerably greater generality.) 

Suppose that $\Phi$ is a set of commuting derivations of a ${\ensuremath{\mathbf{k}}}$-algebra $A$. Let ${\ensuremath{\mathbf{k}}}[\Phi]$ be the subring of the ring of Hochschild 1-cochains $C^1(A,A)$. Then the subring of $C^*(A,A)$ generated by using the cup product is both closed under the composition product and is a subcomplex of $C^*(A,A)$ which we denote by $C_{\Phi}^*(A,A)$.  It inherits another product from ${\ensuremath{\mathbf{k}}}[\Phi]$ by setting $(\xi_1\smile \cdots \smile \xi_n)(\eta_1\smile \cdots \smile \eta_n) = \xi_1\eta_1\smile\cdots\smile\xi_n\eta_n$ for $\xi_1, \dots, \xi_n, \eta_1, \dots, \eta_n \in k[\Phi]$.  In particular, if $\phi, \psi \in \Phi$, then $(\phi \smile \psi)^n = \phi^n \smile \psi^n$.  The following is a special case of
\cite[Lemma 1, p. 13]{G:Def3}, where the entire proof given was, ``Compute''.  For completeness we give here the full simple computation. By a `ring ${\ensuremath{\mathbf{k}}}$ of characteristic 0' we will mean one with a unital morphism ${\ensuremath{\mathbb{Q}}} \to {\ensuremath{\mathbf{k}}}$, where ${\ensuremath{\mathbb{Q}}}$ is the rationals. 
\begin{theorem}\label{basicdef} If $\phi, \psi$ are commuting derivations of an algebra $A$ over a commutative unital ring of characteristic zero, then the formal star product product on $A$ given by
$$a\star b = e^{t\,\phi\smile\psi}(a,b)$$ 
is associative.
\end{theorem}
\noindent\textsc{Proof.} We must show that $e^{t\,\phi\smile\psi}\circ e^{t\,\phi\smile\psi} = 0$, i.e., that for all $n$ one has
\begin{equation*}
\sum_{\ell + m = n} \frac{1}{\ell!m!}[\phi^{\ell}(\phi^m\smile \psi^m)\smile \psi^{\ell} - \phi^{\ell}\smile\psi^{\ell}(\phi^m\smile\psi^m)] = 0.
\end{equation*}
Replacing $m$ by $n-\ell$ and using that $\phi^{\ell}(a,b) = \sum_{k=0}^{\ell}\binom{\ell}{k}\phi^ka\,\phi^{\ell-k}b$, we must show that 
\begin{equation}\label{kernel} 
\sum_{\ell=0}^n \binom{n}{l}[\sum_{k=0}^{\ell}\binom{l}{k}\phi^{n-\ell+k}\smile \phi^{l-k}\psi^{n-l} \smile \psi^{\ell} 
- \sum_{k=0}^{\ell} \phi^{\ell}\smile \psi^k\phi^{n-l}\smile\psi^{n-k}] = 0.
\end{equation}
Fixing  $i$ and $j$ with $0 \le i, j \le n$, there are exactly two terms in \eqref{kernel} of the form $c\phi^i\smile x \smile \psi^j$, the coefficient $c$ being some integer; since $\phi$ and $\psi$ commute, the value of $x$ in both is the same, namely $\phi^{n-i}\psi^{n-j}$.  Their respective coefficients are $\binom{n}{j}\binom{j}{i+j-n}$  and $-\binom{n}{i}\binom{j}{n-j} = -\binom{n}{j}\binom{j}{i+j-n}$, so all terms cancel. $\Box$
 
\medskip

With the above notation, any derivation of $A$ which commutes with both $\phi$ and $\psi$ clearly remains a derivation of the deformed product still commuting with $\phi$ and $\psi$, so if we have two such derivations, $\phi_2, \psi_2$ commuting with each other we can use Theorem \ref{basicdef} again. It follows that $\exp t(\phi \smile \psi + \phi_2 \smile \psi_2$ again defines an associative product. By induction we have 

\begin{corollary}
If $\phi_1, \dots, \phi_n$ are commuting derivations of an algebra $A$ over a ring ${\ensuremath{\mathbf{k}}}$ of characteristic zero and if $\{c_{ij}, \, i, j = 1,\dots, n\}$ are arbitrary elements of ${\ensuremath{\mathbf{k}}}$, then the formal star product on $A$ given by 
\begin{equation}
a\star b = \exp (t\sum_{i,j}c_{ij}\phi_i\smile \psi_j)(a,b)
\end{equation}
is associative. $\Box$
\end{corollary}

If 2 is a unit in ${\ensuremath{\mathbf{k}}}$ and $\phi, \psi$ are derivations of $A$ then $\phi \smile \psi$ is cohomologous to $\phi \wedge \psi := (\phi \smile \psi -\psi \smile \phi)/2$, so replacing the matrix $(c_{ij})$ in the foregoing by its skew symmetric part would give an equivalent deformation. 

Note now that in the polynomial ring ${\ensuremath{\mathbf{k}}}[x]$ over an arbitrary commutative unital ring ${\ensuremath{\mathbf{k}}}$, the 1-cochain $(d/dx)^n/n!$ is always well-defined: it sends $x^r$ to $\binom{r}{n}x^{r-n}$ if $r \ge n$ and to 0 otherwise. Writing $\partial$ for $d/dx$ one then has $(\partial^m/m!)(\partial^n/n!) = \partial^{mn}/m!n!$, which is all that is needed in the proof of Theorem \ref{basicdef}. Therefore we have

\begin{corollary}\label{polyring}
Let ${\ensuremath{\mathbf{k}}}$ be an arbitrary commutative unital ring.  if $\{c_{ij}, \, i, j = 1,\dots, n\}$ are arbitrary elements of ${\ensuremath{\mathbf{k}}}$, then the formal star product on $k[x_1,\dots, x_n] $ given by 
\begin{equation}
f\star g = \exp (t\sum_{i,j}c_{ij} \partial_i\smile \partial_j)(f,g)
\end{equation}
is associative, where $\partial_i = \partial/\partial x_i$. $\Box$
\end{corollary}

\medskip

Corollary \ref{polyring} clearly holds even for an infinite number of variables since any element of the ring is already contained in the subring generated by some finite number of variables. 

As stated here it is only a special case of what was proven in \cite{BravermanGaitsgory:PBW}, which considers filtered quadratic algebras, but is all that we need and is already a stronger assertion than the  Poincar{\'e}-Birkhoff-Witt theorem, which is an immediate corollary.  That $U{\mathfrak{g}}$ is a jump deformation is not explicitly mentioned in \cite{BravermanGaitsgory:PBW} but is evident since $U{\mathfrak{g}}$ is the quotient of the tensor algebra on ${\mathfrak{g}}$ by the ideal generated by all element of the form $a\otimes b -b\otimes a$ the deformed multiplication is completely determined

The Poincar{\'e}-Birkhoff-Witt theorem asserts that if ${\mathfrak{g}}$ is a Lie algebra of arbitrary dimension over a field of arbitrary characteristic then the associated graded algebra ${\operatorname{gr}}(U{\mathfrak{g}})$ of its universal enveloping algebra is the symmetric algebra $S{\mathfrak{g}}$.  It follows that these can be identified as vector spaces and therefore, that if we have and ordered basis $\{x_o\}$ of ${\mathfrak{g}}$ then $U{\mathfrak{g}}$ has a basis consisting of `pseudo monomials' $x_{i_1}x_{i_2}\cdots x_{i_n}, \, i_ \le i_2 \le \dots \le i_n$. In \cite{BravermanGaitsgory:PBW}, following the ideas of \cite{G:Def2}, Braverman and Gaitsgory,  considered filtered quadratic algebras, proving a PBW for ones of Koszul type.  (A slightly less general result had previous been obtained by Polishchuk and Positselsky, \cite{PP:X} but without the deformation methods which are essential here.) We do not need the full generality of \cite{BravermanGaitsgory:PBW} here but summarize enough to show that there is a filtered ring which is actually a deformation of the graded ring $S{\mathfrak{g}}$ and which will serve as a universal enveloping algebra for ${\mathfrak{g}}$ and which therefore is canonically isomorphic  with the classical $U{\mathfrak{g}}$.  Note that $S = S{\mathfrak{g}}$ is a polynomial ring over the ground field ${\ensuremath{\mathbf{k}}}$ whose vector space of elements of degree 1 is ${\mathfrak{g}}$.  Extending the Lie multiplication, $
\alpha$ on ${\mathfrak{g}}$ to a skew biderivation, hence a 2-cocycle, of $S{\mathfrak{g}}$ with coefficients in itself, we must show there is a one-parameter family of deformations of $S{\mathfrak{g}}$ with product of the form
\begin{equation*}
a\star b = ab + \bar\mu_1(a,b) +\hbar^2\alpha_2(a,b) + \cdots
\end{equation*}
whose infinitesimal is $\mu$.  The primary obstruction is the 3-cocycle $\mu \circ \mu$, which must be shown to be a coboundary. 

The argument uses fundamental properties of Koszul algebrasThe reason is that being a polynomial ring, $S{\mathfrak{g}}$ is Koszul

Braverman and Gaitsgory have shown that  if ${\mathfrak{g}}$ is a Lie algebra over ${\ensuremath{\mathbf{k}}}$ then its universal enveloping algebra $U{\mathfrak{g}}$ is a deformation of the symmetric algebra $S{\mathfrak{g}}$ on ${\mathfrak{g}}$, \cite{BravermanGaitsgory:PBW}. This is the essential content of the  Poincar{\'e}-Birkhoff-Witt theorem, which they generalized to quadratic algebras of Koszul type.  Denote the Lie multiplication by $\mu$, and write the deformation as 
\begin{equation*}
a\star b = ab + \bar\mu_1(a,b) +\hbar^2\mu_2(a,b) + \cdots.
\end{equation*}

\noindent Since we must have $a\star b - b\star a = ab +\hbar\mu(ab)$, it follows that all $\mu_i$ with $i \ge 2$ must be symmetric and that we must have $\mu_1 = \frac 12 \mu + $ a symmetric term.

Denote the subalgebra of invariants of $S{\mathfrak{g}}$ by $(S{\mathfrak{g}})^{\mathfrak{g}}$. One then has $\mu(a,b) = 0$ whenever $a\in (S{\mathfrak{g}})^{\mathfrak{g}}$ and $b\in S{\mathfrak{g}}$, so the star product of an invariant with an arbitrary element of $S{\mathfrak{g}}$ is its original undeformed product. In particular, $(S{\mathfrak{g}})^{\mathfrak{g}}$ remains a subalgbra of $U{\mathfrak{g}}$.

\begin{theorem}\label{Duflo}
Let  ${\mathfrak{g}}$ be a Lie algebra over a commutative unital ring ${\ensuremath{\mathbf{k}}}$ in which 2 is a unit.  Then the center of $U{\mathfrak{g}}$ is $(S{\mathfrak{g}})^{\mathfrak{g}}$.
\end{theorem}
\noindent{\textsc{Proof.} Since the invariants are contained in the center it remains to show only that there are no other central elements, i.e.,  that only invariant elements of $S{\mathfrak{g}}$ can be lifted to central elements of the deformed algebra, $U{\mathfrak{g}}$. The bracket of an element of $S{\mathfrak{g}}$ with $\mu$ is a cocycle of dimension 1. That its cohomology class vanishes means that it is an inner derivation, but $S{\mathfrak{g}}$ is commutative and therefore has no inner derivations other than 0. The element must therefore be an invariant. $\Box$
\medskip

Theorem \ref{Duflo} generalizes \cite{Duflo:1977}, as it requires neither that ${\ensuremath{\mathbf{k}}}$ be a field of characteristic 0 nor that ${\mathfrak{g}}$ be finite dimensional. The original Duflo theorem is the dimension 0 case of the assertion that  $H^*({\mathfrak{g}}, S{\mathfrak{g}}) \cong H^*(U{\mathfrak{g}}, U{\mathfrak{g}})$, cf \cite{Cattaneo et al:Deformation}, which will be addressed in another note

