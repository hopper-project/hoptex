\documentclass{amsart}

\usepackage{latexsym,amsfonts}
\usepackage{amssymb}
\usepackage{amsmath}
\usepackage{amsthm}

\newtheorem{lemma}{Lemma}[section]
\newtheorem{theorem}[lemma]{Theorem}
\newtheorem{cor}[lemma]{Corollary}
\newtheorem{propo}[lemma]{Proposition}

\theoremstyle{remark}
\newtheorem{rem}[lemma]{Remark}

          
     

            
          
           
           
          
          

\def\om2{\omega^ {-\ell}}

    
    
    
    
    
    
    

\begin{document}

\title{Scott's formula and Hurwitz groups}

\author{M.A. Pellegrini}

\address{Dipartimento di Matematica e Fisica, Universit\`a Cattolica del Sacro Cuore, Via Musei 41,
I-25121 Brescia, Italy}
\email{marcoantonio.pellegrini@unicatt.it}

\author{M.C. Tamburini Bellani}
\address{Dipartimento di Matematica e Fisica, Universit\`a Cattolica del Sacro Cuore, Via Musei 41,
I-25121 Brescia, Italy}
\email{mariaclara.tamburini@unicatt.it}

\keywords{Hurwitz generations; trilinear forms; $G_2(q)$}
\subjclass[2010]{20G40, 20F05}
 

\begin{abstract} 
This paper continues previous work, based on 
systematic use of a formula of L. Scott, to detect Hurwitz groups.
It closes the problem of determining the finite simple groups contained in
${\mathop{\rm PGL}\nolimits}_n({\mathbb{F}})$ for $n\le 7$ which are Hurwitz, where ${\mathbb{F}}$ is an algebraically closed field. 
For the groups $G_2(q)$, $q\ge 5$, and the Janko groups $J_1$
and $J_2$ it provides explicit $(2,3,7)$-generators. 

\end{abstract}

\maketitle

\section{Introduction}

Let ${\mathbb{F}}$ be  an algebraically closed field of characteristic $p\ge 0$.
The main aim of this paper, which is a continuation of \cite{TV1} and \cite{TV2},  
is to complete the list of  the finite simple groups contained in
${\mathop{\rm PGL}\nolimits}_n({\mathbb{F}})$, $n\le 7$,  which are Hurwitz. 
The crucial tool is a special case of a formula of L. Scott \cite{S} (see also \cite{hb}). 
Namely, let $H$ be a group and $\Phi: H \rightarrow {\mathop{\rm GL}\nolimits}(V)$ 
be a representation of $H$ on ${\mathbb{F}}$, with representation space $V={\mathbb{F}}^m$. For any subset $A$ of $H$ define: 
\begin{eqnarray*}
d_V^A & =& {\mathop{\rm dim}\nolimits }\{ v\in V \mid \Phi(a)v=v, \textrm{ for all } a \in A\},\\ 
\hat d_V^A & =& {\mathop{\rm dim}\nolimits }\{ v\in V \mid \Phi(a)^ T v=v, \textrm{ for all } a \in A\}.
\end{eqnarray*}
For $A=\left\{a\right\}$, set $d_V^A=d_V^a$. Then, by Scott's formula: 
\begin{equation}\label{Scott.gen}
d_V^ x+d_V^ y + d_V^ {xy}\leq m+ d_V^ H+\hat d_V^ H. 
\end{equation} 
The impact of this formula on Hurwitz generation, already suggested by Scott himself,
was first studied by A. Zalesski, who gave important applications \cite{DTZ, VZ}.

In this context, a triple $(x,y,xy)$ in ${\mathop{\rm GL}\nolimits}_n({\mathbb{F}})^3$  is called \emph{irreducible} whenever
$\left\langle x,y\right\rangle$ is an irreducible subgroup of ${\mathop{\rm GL}\nolimits}_n({\mathbb{F}})$. It is called  
\emph{rigid} when it is irreducible  and equality holds in \eqref{Scott.gen}, with respect to  the conjugation action $\Phi$ 
of $H=\left\langle x,y\right\rangle$ on  $V={\mathop{\rm Mat}\nolimits}_n({\mathbb{F}})$. Finally we will say that it is a $(2,3,7)$-triple 
when  the projective images of $x,y,xy$ have respective orders $2,3,7$.

In \cite{TV1} were classified (up to multiplication by scalars)
the admissible similarity invariants of the irreducible  $(2,3,7)$-triples
in ${\mathop{\rm GL}\nolimits}_n({\mathbb{F}})^3$, $n\le 7$. The corresponding triples  are all rigid if and only if $n\le 5$.
For $n=6,7$, the similarity invariants of the non-rigid ones (parametrized in \cite{V3} and \cite{TV2}) are  respectively: 
\begin{equation}\label{inv}
t^ 2+1, \; t^ 2+1,\; t^ 2+1; \qquad t^3-1,\; t^3-1; \qquad t^ 6+t^ 5+t^ 4+t^ 3+t^2+t+1,
\end{equation}
for $n=6$ (see  \cite[1.1.1 page 349 and 2.1.1 page 350]{TV1}), and  
\begin{equation}\label{inv7}
t+1, \;t^ 2-1, \; t^ 2-1,\; t^ 2-1; \qquad t-1,\;t^3-1,\; t^3-1; \qquad t^ 7-1,
\end{equation}
for $n=7$ (see \cite[1.1 page 351]{TV1}). In this case the irreducibility forces $p$ odd.

An irreducible triple of type \eqref{inv} generates a symplectic group, of type
\eqref{inv7} 
an orthogonal group in odd characteristic: see Section \ref{prelim} for details.

Here we prove the following:

\begin{theorem}\label{p7}
Let ${\mathbb{F}}$ be an algebraically closed field of characteristic $p>0$. Let $H_7$ 
be an irreducible subgroup of ${\mathop{\rm SL}\nolimits}_7({\mathbb{F}})$ generated by a non-rigid 
$(2,3,7)$-triple, i.e., a triple with similarity invariants \eqref{inv7}. Then $p$ is odd 
and  $H_7$ is a subgroup of  $G_2({\mathbb{F}})$.
\end{theorem}

\begin{theorem}\label{main}
Let ${\mathbb{F}}$ be an algebraically closed field of characteristic $2$. Let $H_6$ 
be an irreducible subgroup of ${\mathop{\rm SL}\nolimits}_6({\mathbb{F}})$ generated by a non-rigid 
$(2,3,7)$-triple, i.e., a triple with similarity invariants \eqref{inv}. Then $H_6$ is a subgroup of  $G_2({\mathbb{F}})$.
\end{theorem}

By an old, non-constructive, result  of G. Malle \cite{Ma}, the simple groups $G_2(q)$ are Hurwitz if, and 
only if, $q\geq 5$ (but they are $(2,3)$-generated for all $q$). 
In the last Section  we provide explicit Hurwitz generators of $G_2(q)$, $q\geq 5$, 
and of the Janko groups $J_1$ over ${\mathbb{F}}_{11}$,  $J_2$ over ${\mathbb{F}}_4$. 
Also the  Ree groups $^2G_2\left(3^{2a+1}\right)$ are Hurwitz for all $a\geq 1$.
Explicit $(2,3,7)$-generators for these groups can be found in \cite{T}.

With the already mentioned classification of the irreducible $(2,3,7)$-triples 
 and the knowledge of the maximal subgroups of 
${\mathop{\rm PSp}\nolimits}_6(q)$ and $G_2(q)$, Theorems  \ref{p7}  and \ref{main} allow to complete the
list of 
the finite simple groups contained in
${\mathop{\rm PGL}\nolimits}_n({\mathbb{F}})$, $n\le 7$, which are Hurwitz  (up to isomorphism).
They are the groups above, the symplectic groups ${\mathop{\rm PSp}\nolimits}_6(q)$, $5\le q$ odd (see \cite{TV}), 
and the following groups, all generated by rigid triples:

\begin{itemize}
\item ${\mathop{\rm PSL}\nolimits}_2(p)$ if $p\equiv 0,\pm 1\pmod 7$ and ${\mathop{\rm PSL}\nolimits}_2(p^{3})$ if $p\equiv \pm 2, \pm 3\pmod 7$
\cite{Mac}; 
\item ${\mathop{\rm PSL}\nolimits}_5(p^{n_5})$ if $p\ne 5,7$  and $n_5$ is odd,  ${\mathop{\rm PSU}\nolimits}_5(p^{n_5})$ if $p\ne 5,7$ 
and $n_5$ is even, where $n_5$ is $o(p)\pmod {5}$ times  $o(p^2)\pmod {7}$, 
$n_5=4$ if $p=7$ \cite{TZ}; 
\item ${\mathop{\rm PSL}\nolimits}_6(p^{n_6})$ if $p\ne 3$ and $n_6$ is odd, ${\mathop{\rm PSU}\nolimits}_6(p^{n_6})$ if $p\ne 3$ and
$n_6$ is even, where $n_6=o(p)\pmod {9}$ \cite{TV1};
\item ${\mathop{\rm PSL}\nolimits}_7(p^{n_7})$ if $p\ne 7$ and $n_7$ is odd, ${\mathop{\rm PSU}\nolimits}_7(p^{n_7})$ if $p\ne 7$ and
$n_7$ is even, where $n_7=o(p)\pmod {49}$ \cite{TV1}. 
\end{itemize}
Clearly rigidity restricts the isomorphism types: at most one for fixed $p$ and $n$.
For an interesting discussion of this aspect we refer to \cite{M} .

\subsection*{Acknowledgments}

We are indebted to Maxim Vsemirnov for useful suggestions and discussions.

\section{Preliminary results}\label{prelim}

We recall that ${\mathbb{F}}$ is an algebraically closed field of characteristic $p\ge 0$.
Let $(x,y,xy)$ be an irreducible triple in ${\mathop{\rm GL}\nolimits}_n({\mathbb{F}})^3$
with similarity invariants $\eqref{inv}$ if $n=6$, and $\eqref{inv7}$ if  $n=7$, 
and set $H_n=\left\langle x,y\right\rangle$.
We consider the diagonal action of ${\mathop{\rm GL}\nolimits}_n({\mathbb{F}})$ on the space ${\mathbb{F}}^n\otimes {\mathbb{F}}^n$,
identified with ${\mathop{\rm Mat}\nolimits}_n({\mathbb{F}})$, namely the action $B\mapsto gBg^T$ for all $B\in {\mathop{\rm Mat}\nolimits}_n({\mathbb{F}})$, $g\in{\mathop{\rm GL}\nolimits}_n({\mathbb{F}})$.  
The symmetric square $S=S({\mathbb{F}}^n)$ and the exterior 
square power $\Lambda^ 2=\Lambda^2({\mathbb{F}}^n)$ can be identified, respectively, with the spaces
of symmetric matrices
and antisymmetric matrices with zero-diagonal. The spaces $S$ and  $\Lambda^ 2$, being ${\mathop{\rm GL}\nolimits}_n({\mathbb{F}})$-invariant,
give rise to representations of $H_n$ on ${\mathbb{F}}$ of respective degrees $\frac{n(n+1)}{2}$ and
$\frac{n(n-1)}{2}$.
In the notation explained in the Introduction, with respect to 
the action of $H_n$ on  $S$, by point (i) of Lemma 1 of \cite{TV1} one has:
$$\hat d_{S }^ {H_n}\le d_{S}^ {H_n} \le 1 .$$
Suppose $n=6$ and $p=2$. Then $d_{S}^ x= 12$, $d_{S}^ y= 7$ and $d_{S}^ {xy}= 3$.
Since ${\mathop{\rm dim}\nolimits } S=21$, from  Scott's formula \eqref{Scott.gen} it follows  either 
$d_S^{H_6}= \hat d_S^ {H_6}= 1$ or $d_S^ {H_6}=1$ and $\hat d_S^{H_6}=0$.
In the first case $H_6\le \Omega_6({\mathbb{F}})\le {\mathop{\rm Sp}\nolimits}_6({\mathbb{F}})$, in the second $H_6\le {\mathop{\rm Sp}\nolimits}_6({\mathbb{F}})$  by points (ii)
and  (iii) of Lemma 1 mentioned above.
Next suppose $n=7$ and $p$ any. Then  $d_{S}^ x= 16$, $d_{S}^ y= 10$, $d_{S}^ {xy}= 4$.  Since ${\mathop{\rm dim}\nolimits } S=28$, 
from  \eqref{Scott.gen} we get $d_S^{H_7}+\hat d_S^ {H_7}\ge 2$. Again Lemma 1 of \cite{TV1} gives $p$ odd,
$d_S^{H_7}= \hat d_S^ {H_7}= 1$, and $H_7$ orthogonal.

We are left with  $n=6$ and $p$ odd. As in \cite{V3}, we consider the action 
of $H_6$ on $\Lambda^ 2=\Lambda^2({\mathbb{F}}^6)$.
Then ${\mathbb{F}}^6\otimes {\mathbb{F}}^6= S\oplus \Lambda^2$, whence
$\hat d_{\Lambda ^2}^{H_6}=d_{\Lambda ^2}^ {H_6^T}$. From $d_{\Lambda ^2}^ x\geq 9$, $d_{\Lambda ^2}^ y\ge 5$, $d_{\Lambda ^2}^ {xy}\ge 3$
and  ${\mathop{\rm dim}\nolimits } \Lambda^2 =15$,  we get $d_{\Lambda ^2}^ {H_6}+ \hat d_{\Lambda ^2}^{H_6}\ge 2$.
Since  $d_{\Lambda ^2}^ {H_6}\le  1$ and $d_{\Lambda ^2}^ {H_6^T}\le 1$, by the irreducibility of  $H_6$,
it follows $d_{\Lambda ^2}^ {H_6}= d_{\Lambda ^2}^ {H_6^T}= 1$. Hence $H_6$ fixes a non-zero alternating 
antisymmetric form $J$, which is non-degenerate by the irreducibility of $H_6$. 
We conclude $H_6\le {\mathop{\rm Sp}\nolimits}_6({\mathbb{F}})$.

\section{Proofs of Theorems \ref{p7} and \ref{main}}

It is well known that the space of the $m$-linear forms on $V={\mathbb{F}}^n$ ($n\geq m$) can be
identified with the skew-symmetric tensor product $\wedge^m(V^\ast)$, acting on $V^m$ via:
\begin{equation}
(v_1^\ast \wedge\ldots\wedge
v_m^\ast)(u_1,\ldots,u_m)=\det \begin{pmatrix}
v_1^\ast u_1 &  \ldots & v_1^\ast u_m\\
\vdots & \ddots & \vdots\\
v_m^\ast u_1  & \ldots & v_m^\ast u_m\\
 \end{pmatrix}=\det \begin{pmatrix}
v_k^T u_j
\end{pmatrix}.
\end{equation}
We may define a representation $\Phi: {\mathop{\rm GL}\nolimits}(V)\rightarrow {\mathop{\rm GL}\nolimits}\left(\wedge^m(
V^\ast)\right)$ by setting:
$$[\Phi(g)(v_1^\ast \wedge \ldots\wedge
v_{m}^\ast)](u_1,\ldots,u_m)=(v_1^\ast \wedge\ldots\wedge
v_{m}^\ast)(g^{-1} u_1,\ldots,g^{-1}u_m),$$
namely
$$
\Phi(g)(v_1^\ast \wedge\ldots\wedge
v_{m}^\ast)=\left(g^{-T}v_1\right)^\ast \wedge \ldots\wedge
\left(g^{-T} v_m\right)^\ast.
$$ 

We claim that the representation $g\mapsto \Phi\left(g^{-T}\right)$
is the dual of $\Phi$.
Indeed, let $\{e_1,\ldots,e_n\}$ and $\{e_1^\ast,\ldots,e_n^\ast\}$
denote basis for $V$ and its dual in $V^\ast$. Setting
$$\mathcal{B}=\left\{e_{i_1}^\ast \wedge\ldots\wedge e_{i_m}^\ast \mid\;  1\leq
i_1< i_2<
\ldots < i_m\leq n\right\}$$
we obtain a basis of $\wedge^m(V^\ast)$.
By the above
$$
\Phi(g)\left(e_{i_1}^\ast \wedge\ldots\wedge
e_{i_m}^\ast\right)=
\left( g^{-T} e_{i_1}\right)^\ast \wedge \ldots\wedge
\left( g^{-T}e_{i_m}\right)^\ast=$$ 
$$\sum_{1\le k_1\le\dots \le k_m\le n} \det \left(\gamma_{i_1,\dots ,i_m}^{k_1,\dots
,k_m}\right)\left(e_{k_1}^\ast \wedge\ldots\wedge e_{k_m}^\ast\right),
$$
where  $\gamma_{i_1,\dots ,i_m}^{k_1,\dots ,k_m}$ is the submatrix
obtained from $g^{-T}$ considering  rows $i_1,\dots ,i_m$ and columns $k_1,\dots
,k_m$.
The same calculation leads to
$$
\Phi(g^T)\left(e_{i_1}^\ast \wedge\ldots\wedge e_{i_m}^\ast\right) =
\sum_{1\le k_1\le\dots \le k_m\le n} \det \left(\gamma_{k_1,\dots ,k_m}^{i_1,\dots
,i_m}\right)\left(e_{k_1}^\ast \wedge\ldots\wedge e_{k_m}^\ast\right).$$
Identifying $\Phi(g)$ and $\Phi\left(g^T\right)$ with their matrices with respect to
$\mathcal{B}$
we have $\Phi\left(g^T\right)=\Phi(g)^T$, whence our claim.
Using this fact we can prove Theorem \ref{p7}.

\bigskip

\begin{proof}[Proof of Theorem \emph{\ref{p7}}]
Let $V={\mathbb{F}}^7$ with basis $\{e_1,\ldots,e_7\}$. 
By \cite{TV2}, we can assume that $H_7=\langle x,y\rangle$, where the similarity
invariants
of $x,y,xy$
satisfy \eqref{inv7}. As done before, we may identify the 
space of the alternating trilinear forms defined on $V$ with the space $\wedge^3
(V^\ast)$ and consider the representation $\Phi: {\mathop{\rm GL}\nolimits}(V)\rightarrow {\mathop{\rm GL}\nolimits}\left(\wedge^3
(V^\ast)\right)$ previously introduced. 
We apply formula \eqref{Scott.gen} to this action:
$$d_{\Lambda^ 3(V^\ast)}^ x+d_{\Lambda^ 3(V^\ast)}^ y+d_{\Lambda^ 3(V^\ast)}^{xy} \leq 35
+ d_{\Lambda^ 3(V^\ast)}^{H_7}+\hat d_{\Lambda^ 3(V^\ast)}^{H_7}.$$

Considering the rational form of $x$, $y$ and $xy$ we obtain $d_{\Lambda^ 3(V^\ast)}^ x=
19$, $d_{\Lambda^ 3(V^\ast)}^ y= 13$ and $d_{\Lambda^
3(V^\ast)}^{xy}=5$, whence 
$$d_{\Lambda^ 3(V^\ast)}^{H_7}+\hat d_{\Lambda^ 3(V^\ast)}^{H_7} \geq 2.$$
As $\Phi(h)^T=\Phi(h^T)$ for all $h\in H_7$ we obtain that $ \hat d_{\Lambda^
3(V^\ast)}^{H_7}= d_{\Lambda^ 3(V^\ast)}^{H_7^T}$.
Further, since $H_7$ and $H_7^T$ are absolutely irreducible, by \cite[Theorem 5(2)]{A}, 
$$0\leq d_{\Lambda^ 3(V^\ast)}^ {H_7}\le 1, \quad  0\le d_{\Lambda^ 3(V^\ast)}^{H_7^T}\leq 1.$$
It follows that  $d_{\Lambda^ 3(V^\ast)}^ {H_7}= \hat d_{\Lambda^ 3(V^\ast)}^{H_7}=1$.
Thus $H_7$ fixes a non-zero alternating trilinear form which, by \cite[Theorem 5(5)]{A}, must be a Dickson form.
We conclude that $H_7\leq G_2({\mathbb{F}})$.
\end{proof}

For $p=2$, $n=2m$, by a classical result of Dieudonn\'e \cite{Dieu} the orthogonal group
$\Omega_{2m+1}({\mathbb{F}})$
is isomorphic to ${\mathop{\rm Sp}\nolimits}_{2m}({\mathbb{F}})$. We need an explicit isomorphism.
Call  $Q$ the quadratic form  which defines $\Omega_{2m+1}({\mathbb{F}})$,  call
$g$ the associated  symmetric bilinear form  of rank $2m$ and 
$\left\langle v_0\right\rangle$ the radical of $g$. Then $\Omega_{2m+1}({\mathbb{F}})$ fixes $\left\langle v_0\right\rangle$
and induces a symplectic group on $\frac{{\mathbb{F}}^{2m+1}}{\left\langle v_0\right\rangle}$.
Since every symplectic transformation has determinant $1$,  $\Omega_{2m+1}({\mathbb{F}})$ fixes $v_0$.
Hence, with respect to any basis of ${\mathbb{F}}^{2m+1}$ having $v_0$ as first vector, $\Omega_{2m+1}({\mathbb{F}})$ consists of
matrices 
\begin{equation}\label{Omega7}
h=\begin{pmatrix} 1 & a_{h}^T \\ 0 &  h_{2m} \end{pmatrix},\ 
a_h\in {\mathbb{F}}^{2m}, \ h_{2m}\in {\mathop{\rm Sp}\nolimits}_{2m}({\mathbb{F}}).
\end{equation}
The map $h\mapsto h_{2m}$ is an isomorphism.

Now let $p=2$ and $\left(x_6, y_6, x_6y_6\right)$ be an irreducible triple with similarity invariants  \eqref{inv}.
As proved in Section \ref{prelim},  it generates a subgroup of ${\mathop{\rm Sp}\nolimits}_6({\mathbb{F}})$. 
Its preimage $\left(x_7, y_7, x_7y_7\right)$ in $\Omega_{7}({\mathbb{F}})$ has
similarity invariants  \eqref{inv7}. Since the eigenspace of $x_7y_7$ relative to $1$ has dimension 
$1$, it must coincide with $\left\langle v_0\right\rangle$. Let 
$\mathcal{B}=\{v_0, v_1, v_2,v_{-3}, v_{-1},$ $v_{-2}, v_{3}\}$ be a basis  of 
eigenvectors of $x_7y_7$, with $(x_7y_7)v_i={\varepsilon}^{-i}v_i$ for $i=0,\pm 1, \pm 2,$ $\pm 3$,
where ${\varepsilon}\in {\mathbb{F}}$ is a primitive $7$-th root of $1$.
With respect to $\mathcal{B}$ the Gram matrix $J$
of the bilinear form $g$ fixed by $\left\langle x_7,y_7\right\rangle$ is
$J=\begin{pmatrix} 0& 0\\0&J_6\end{pmatrix}$ with $J_6$ fixed
by $\left\langle x_6,y_6\right\rangle$. 
Multiplying $v_1,v_2, v_{-3}$ by scalar multiples, if necessary, 
we have $J_6=\begin{pmatrix} 0& I_3\\I_3&0\end{pmatrix}$.
Hence we may suppose:
\begin{equation}\label{shape7}
x_7=\begin{pmatrix} 1 & a_{x}^T \\ 0 &  x_6  \end{pmatrix}, \quad y_7 =
\begin{pmatrix} 1 &a_{x}^T x_6y_6\\
 0 &  y_6  \end{pmatrix},\quad 
x_7y_7 =\begin{pmatrix} 1 &
0\\ 0 &  x_6y_6  \end{pmatrix} .
\end{equation}
The conditions $x_7^2=I$, $J_6x_6=x_6^TJ_6$ and the choice of $\mathcal{B}$ give $a_{x}^Tx_6=a_{x}^T$ and 
\begin{equation}\label{shapex6}
x_6=
\begin{pmatrix} A &B\\
C&A^T \end{pmatrix},\ B=B^T, C=C^T, \quad  x_6y_6= {\mathop{\rm diag}\nolimits}\left({\varepsilon}^{-1},{\varepsilon}^{-2},{\varepsilon}^{3},{\varepsilon}^{1},{\varepsilon}^{2},{\varepsilon}^{-3} \right).
\end{equation}
We want to prove that, whenever $H_6=\left\langle x_6,y_6\right\rangle$ is absolutely
irreducible,
then $H_7=\left\langle x_7,y_7\right\rangle$ fixes a Dickson form, whence $ H_6\cong
H_7\leq G_2({\mathbb{F}})$. 

To this purpose  we need some auxiliary results, starting with:
\begin{rem}\label{quasidet} Any perfect irreducible subgroup of ${\mathop{\rm SL}\nolimits}_6({\mathbb{F}})$ containing 
an involution $x$ with similarity invariants 
$t^2+1,\ t^ 2+1,\ t^ 2+1$ cannot be orthogonal. Indeed $I-x$ has rank $3$ and so the quasideterminant
of $x$ is $(-1)^3=-1$, whence the contradiction $x\not \in {\mathop{\rm SO}\nolimits}_6({\mathbb{F}})^\prime= \Omega_6({\mathbb{F}})$ (e.g., see
\cite[p. 160]{Tay}).
\end{rem}

\begin{propo}\label{notril6}
Let $K$ be an irreducible subgroup  of ${\mathop{\rm SL}\nolimits}_6({\mathbb{F}})$ generated by 
a $(2,3,7)$-triple whose similarity
invariants are \eqref{inv}. 
Then $K$ does not fix any non-zero alternating trilinear form.
\end{propo}

\begin{proof}
By the above discussion we may suppose that $K$ is generated by $x=x_6$, $z=x_6y_6$ as in \eqref{shapex6}.
Proceeding by way of contradiction, let $f$ be a non-zero alternating trilinear form fixed  by $K$.
We have  $f(z^ {-1} v_i, z^{-1} v_j, z^ {-1} v_k)=
{\varepsilon}^ {i+j+k}f(v_i,v_j,v_k)$. Since $f$ is fixed by $z$  we get $f(v_i,v_j,v_k)=0$ whenever 
$i+j+k\not\equiv 0\pmod 7$. Thus:
$$f=\lambda (v_1^\ast \wedge v_2^\ast \wedge v_{-3}^ \ast)+\mu (v_{-1}^\ast \wedge v_{-2}^\ast \wedge v_{3}^ \ast),$$
for some $\lambda,\mu\in {\mathbb{F}}$. Assume $\mu =0$. In this case $\langle v_{-1},v_{-2},v_{3}\rangle\leq {\mathop{\rm rad}\nolimits }(f)$ and so
${\mathop{\rm rad}\nolimits }(f)$ is a non-trivial subspace of ${\mathbb{F}}^ 6$ fixed by $K$.
Since $K$ is irreducible, we obtain ${\mathop{\rm rad}\nolimits }(f) ={\mathbb{F}}^6$ and so $f=0$, a contradiction. The same holds if $\lambda=0$. 
It follows that both $\lambda$ and $\mu$ are non-zero and, multiplying by $\lambda^{-1}$, we may assume:
\begin{equation}\label{trilK}
f=v_1^\ast \wedge v_2^\ast \wedge v_{-3}^ \ast+\rho\left( v_{-1}^\ast \wedge v_{-2}^\ast \wedge v_{3}^ \ast\right),\ \rho\ne 0
\end{equation}
i.e., $f\left(v_{1},v_{2},v_{-3}\right)=1$, $f\left(v_{-1},v_{-2},v_{3}\right)=\rho$, 
$f\left(v_{i},v_{j},v_{k}\right)=0$ otherwise.
Write 
$$x=(a_{i,j}),\quad i,j\in\{1,2,-3,-1,-2,3\}.$$

\noindent \textbf{Case 1.} Suppose that $A$ in \eqref{shapex6} is diagonal.
From $f(x v_1, v_2,v_{-3} )=f(v_1, x v_2, x v_{-3})$ we get
$a_{1,1}=a_{2,2}\cdot a_{-3,-3}$. Similarly, from $f(v_1,  x v_2,  v_{-3})=
f(x v_1, v_2, x v_{-3})$ and 
$f(v_1,v_2, x v_{-3})=f( x v_1,  x v_{2}, v_{-3})$ we get, respectively, 
$a_{2,2}=a_{1,1}\cdot a_{-3,-3}$ and $a_{-3,-3}=a_{1,1}\cdot a_{2,2}$. It follows that either $a_{1,1}=a_{2,2}=a_{-3,-3}=0$ or $a_{1,1}=a_{2,2}=a_{-3,-3}=1$. 
In the first case  $A=0$ and it is clear from \eqref{shapex6} that $y=x(xy)$ cannot have order $3$. 
In the second case, $A=I$, whence the contradiction
$1={{\rm Tr}({x(xy)})}={{\rm Tr}({y})}=0$.
\medskip

\noindent\textbf{Case 2.} Some non-diagonal entry of $A$ is not zero. We may suppose that it is 
in the first column, up to the permutation $\left(v_{1}v_{i}\right)\left(v_{-1}v_{-i}\right)$ for some $i=2,-3$.
We claim that $a_{2,1} \ne 0$ if, and only if, $a_{-3,1}\ne 0$.
Indeed, suppose $a_{2,1} \ne 0$. The subspace $S_1=\{s\in {\mathbb{F}}^ 6\mid f(v_1, x v_1, s)=0\}$ is 
$x$-invariant and contains $\langle v_{-1},v_{-2},v_3,v_1\rangle$. 
If $a_{-3,1}=0$, then $v_{-3}\not\in S_1$. As $v_2 \in S_1$ the space
$S_1$ has dimension $5$ and is fixed by $K$, a contradiction. Thus $a_{-3,1}\ne 0$ and the same 
argument shows the opposite implication.  Conjugating $x$ and $z$ by 
${\mathop{\rm diag}\nolimits}\left(a_{2,1}^{-1}\cdot a_{-3,1}^{-1}, a_{2,1}, a_{-3,1}, \ a_{2,1}\cdot a_{-3,1}, a_{2,1}^{-1}, a_{-3,1}^{-1}\right)$
their shapes are preserved as well as the condition $f\left(v_1, v_2,v_{-3}\right)=1$.
Hence we may suppose
$a_{2,1}=a_{-3,1}=1$ (and $a_{-1,-2}=a_{-1,3}=1$ by \eqref{shapex6}). 

For $j\in \left\{-1, -2,3\right\}$ we obtain:
$$0 = f(x v_1,  v_1, v_j)=f(v_1, x v_1, x v_j)=a_{-3,j}+ a_{2,j},$$
whence $a_{-3,-1}=a_{2,-1}$,  $a_{-3,-2}=a_{2,-2}$ and  $a_{-3,3}=a_{2,3}$. 
After these substitutions, for $j\in \left\{1, 2,-3\right\}$ we get
$$
\rho\thinspace a_{-1,j}=f( x v_j, v_{-2}, v_3) = f( v_j, x v_{-2}, x v_3) =0,$$
whence $a_{-1,1}=a_{-1,2}=a_{-1,-3}=0$ (and $a_{-2,1}=a_{3,1}=0$).
Now $f(x v_3, v_{-2}, v_j)=f(v_3,  xv_{-2},  x v_j) $ for $j=2,-3$
gives 
$\rho\thinspace a_{-2,j}=0$, i.e., $a_{-2,2}=a_{-2,-3}=a_{3,2}=0$. Finally  
$f( x v_{-2}, v_{3}, v_{-3})=f(v_{-2},  xv_{3},  x v_{-3})$ gives $a_{3,-3}=0$.

We conclude that $\langle v_1,v_2,v_{-3} \rangle$  is an $x$-invariant subspace,
hence a $K$-invariant subspace, a contradiction.
\end{proof}

\begin{lemma}\label{orto} 
If $H_6$ is  irreducible, $\langle v_0\rangle$
has no $H_7$-invariant complement.
\end{lemma}

\begin{proof}
Any complement should coincide with the space $W=\left\langle v_1, \dots , v_{3}\right\rangle$
generated by the eigenvectors of $x_7y_7$ relative to the eigenvalues $\ne 1$. 
This happens only if $a_{x}^T =0$ in \eqref{shape7}. In this case $H_6$ preserves 
the restriction $\bar Q=Q_{|W}$, where $Q$ is the quadratic form fixed by $H_7$.
The symmetric form associated to $\bar Q$ has Gram matrix $J_6$, as above,
which is non-degenerate.
Hence $\bar Q$ is a quadratic form.
This implies that $H_6$ is an orthogonal group, in contrast with Remark \ref{quasidet}.
\end{proof}

\begin{lemma}\label{invSub}
If $H_6$ is irreducible, then $\langle v_0\rangle$ is the only proper $H_7$-invariant subspace
of ${\mathbb{F}}^7$. Similarly $W=\langle v_{1}, v_{2}, v_{-3}, v_{-1}, v_{-2}, v_3\rangle$ is the
only proper
$H_7^T$-invariant subspace
of ${\mathbb{F}}^7$.
\end{lemma}

\begin{proof}
Let $U$ be a proper $H_7$-invariant subspace.
By the irreducibility of $H_6$ we have either $U\cap W=0$
or  $U\cap W=U$. In the first case $U$ has
dimension $1$.
Hence the perfect group $H_7$ induces the identity on $U$. This gives  $U=\langle
v_0\rangle$,
the eigenspace of $x_7y_7$ relative to $1$. The second case cannot arise, since $U$ would be an
$H_7$-invariant complement of $\langle v_0\rangle$, in contrast with the previous Lemma.
The second part of the statement follows noting that $\langle v_0^T\rangle$ is the
eigenspace 
of $(x_7y_7)^T$ relative to $1$ and using the fact that $H_6^T$ is not orthogonal by Remark 
\ref{quasidet}.
\end{proof}

 

\begin{lemma}\label{Scott}
If $H_6$ is  irreducible, then
the group $H_7$  fixes a non-zero alternating trilinear form on $V$. 
\end{lemma}

\begin{proof}
Proceeding as in the proof of  Theorem \ref{p7}, we take the action of $H_7$ on the space
of trilinear forms, that we identify with $\Lambda^3(V^*)$, obtaining  
$$d_{\Lambda^ 3(V^\ast)}^ {H_7}+\hat d_{\Lambda^ 3(V^\ast)}^{H_7}=d_{\Lambda^ 3(V^\ast)}^
{H_7}+d_{\Lambda^ 3(V^\ast)}^{H_7^T} \geq 2.$$
It suffices to show that $d_{\Lambda^ 3(V^\ast)}^{H_7^T}\le 1$.
So let $\tilde f$ be a non-zero alternating trilinear form on $V$
fixed by $H_7^ T$. 
Denote by $W$ the $6$-dimensional subspace of $V$ fixed by $H_7^ T$.
The restriction $\tilde f|_W$ of $\tilde f$ to $W$ is $ H_6^ T$-invariant and so it must
be the zero form, by Proposition \ref{notril6}. 
Furthermore, the kernel of the function $\Psi: V \rightarrow \Lambda^ 2 (V^ \ast)$,
defined
as $\Psi(v)=\tilde f(v,\ast,\ast)$, is a $H_7^ T$-invariant subspace of $V$. 
By Lemma \ref{invSub} either $\ker(\Psi)=W$, or $\ker(\Psi)=0$. 
If $\ker(\Psi)=W$, then $\tilde f$ is the
zero form. It follows that $\Psi$ is
injective.
Set $z=x_7y_7$ and take 
$\mathcal{B}$ as above.
We have $z=z^ T$ and
$$\tilde f( z^ {-1}v_i,z^ {-1} v_j,z^ {-1}v_k)={\varepsilon}^{i+j+k}\tilde f(v_i,v_j,v_k)=\tilde
f(v_i,v_j,v_k),$$
whence either $i+j+k=0$ or $\tilde f(v_i,v_j,v_k)=0$. 
It follows $f(v_i,v_j,v_k)=0$ except, possibly, for 
$(i,j,k)\in \{(0,\ell,-\ell), (1,2,-3),(-1,-2,3) \mid \ell=1,2,3\}$.
By the assumption that $\tilde f|_W$ is the zero-form, 
we have $\tilde f(v_{\pm 1},v_{\pm 2},v_{\mp 3})=0$.
If $\tilde f(v_0,v_i,v_{-i})=0$, then $\tilde f(v_i,\ast,\ast)$ is the zero form, a contradiction
with the injectivity of $\Psi$. So, $\tilde f(v_0,v_i,v_{-i})\neq 0$ for all $i$'s. 
Substituting each $v_i$ with a scalar multiple, if necessary, we get
$\tilde f=\sum_{i=1}^3\left( v_0^\ast\wedge
v_i^\ast\wedge v_{-i}^\ast\right)$. We conclude that  $d_{\Lambda^3(V^\ast)}^{H_7^T}=
1$.
\end{proof}

\bigskip
\begin{proof}[Proof of Theorem {\rm \ref{main}}]
First, considering in Proposition \ref{notril6} $K=H_6^ T$, we obtain that the group $H_6^ T$ does not fix non-zero alternating trilinear forms. Hence,  Lemma \ref{Scott} implies that $H_7$ fixes a non-zero alternating trilinear form $f$  on ${\mathbb{F}}^ 7$.
By Lemma \ref{orto} the group $H_7$ acts indecomposably on $V={\mathbb{F}}^ 7$. Hence, by
\cite[(10.9)]{A} it suffices to prove that  ${\mathop{\rm rad}\nolimits }(B_{v_0})=\langle v_0\rangle$, where
$B_{v_0}$ is the symmetric form defined as $B_{v_0}(u,w)=f(v_0,u,w)$.
The radical of $B_{v_0}$ is an $H_7$-invariant subspace of $V$. 
By Lemma \ref{invSub} either ${\mathop{\rm rad}\nolimits }(B_{v_0})=\langle v_0\rangle$ or ${\mathop{\rm rad}\nolimits }(B_{v_0})=V$. 
Suppose the latter case holds, i.e.  $B_{v_0}$ is the zero-form.
Take $\bar u,\bar v,\bar w\in \overline{V}$ and define
$${\overline{f}}(\bar u,\bar v,\bar w)=f(u, v, w),$$
where
$u=\eta v_0+\overline{u}$, $v=\lambda v_0+\overline{v}$ and $w=\mu v_0+\overline{w}$.
The function ${\overline{f}}$ is well-defined as
$$
f(u,v,w)  =f(\eta v_0+\overline{u},\lambda v_0+\overline{v},\mu v_0+\overline{w} )=
f(\overline{u},\overline{v},\overline{w}).$$
For every element of $h\in H_7$,
${\overline{f}}(\overline{u},\overline{v},\overline{w})= f(u,v,w)=
f(h_6^ {-1}\overline{u},h_6^ {-1}\overline{v},h_6^{-1}\overline{w})$
 and so
$H_6$ fixes a non-zero alternating trilinear form ${\overline{f}}$ on $\overline{V}$, in contrast with Proposition \ref{notril6} (take $K=H_6$).
\end{proof}

\section{Hurwitz generators for $G_2(q)$}

In this section we set $q=p^a$, where $p$ is a prime, and consider ${\mathbb{F}}$ as the algebraic closure of the finite field ${\mathbb{F}}_q$. Our aim is to find explicit Hurwitz generators for the groups $G_2(q)$,
which are Hurwitz only for $q\geq 5$ \cite{Ma}. 

Let $p=2$ and $q\geq 8$. Our generators $x_6$, $y_6$ and their product $x_6y_6$ have similarity invariants \eqref{inv}, 
hence are conjugate to the matrices in \eqref{shapex6}, but are different. Indeed they are obtained from family (IIa) in \cite{V3} for special values of the parameters, as well as  the Hurwitz  
generators of ${\mathop{\rm PSp}\nolimits}_6(q)$, $5\le q$ odd, 
used in  \cite{TV}.  

For each  $r\in {\mathbb{F}}_q\setminus {\mathbb{F}}_4$ we 
set 
\begin{equation}\label{abc}
a=\frac{r + 1}{d},\quad
b=\frac{r^3 + r^2+ 1}{d},\quad c=\frac{r^3 + 1}{d},\quad d=r^2+r+1,
\end{equation}
\begin{equation}\label{x,y}
 x_6= \begin{pmatrix}
  0 & 0 & 1 & 0 & r & 1 \\
  0 & 0 & 0 & 1 & 0 & a \\
  1 & 0 & 0 & 0 & 1 & r \\
  0 & 1 & 0 & 0 & a & 0 \\
  0 & 0 & 0 & 0 & 0 & 1   \\
  0 & 0 & 0 & 0 & 1 & 0
    \end{pmatrix},
 \qquad
   y_6=\begin{pmatrix}
  1 & 0 & 0 & 0 & 1 & c \\
  0 & 1 & 0 & 0 & b & 1 \\
  0 & 0 & 0 & 0 & 1 & 0 \\
  0 & 0 & 0 & 0 & 0 & 1 \\
  0 & 0 & 1 & 0 & 1 & 0 \\
  0 & 0 & 0 & 1 & 0 & 1
   \end{pmatrix}
\end{equation}
and  define $H=\langle x_6,y_6\rangle$.

\begin{lemma}\label{lemma: irreducibility} 
The group $H$ is absolutely irreducible, except when:
$$r^{12} + r^9 + r^5 +  r^2 +1=0.$$
In particular, if $H$ is absolutely irreducible, then it is a subgroup of $G_2(q)$.
\end{lemma}

\begin{proof}
For the absolute irreducibility of $H$ apply \cite[Lemma 2.1]{TV} to the case  
$8 \le q$ even, $r\in {\mathbb{F}}_q\setminus {\mathbb{F}}_4$. For the second claim of the statement, apply Theorem \ref{main}. 
\end{proof}

Now we want to exclude that $H$ is contained in a maximal subgroup of $G_2(q)$. 
By \cite{C}, the absolutely irreducible maximal subgroups $M$ of $G_2(q)$ are of type:
$$M\cong {\mathop{\rm SL}\nolimits}_3(q).2,\quad M\cong {\mathop{\rm SU}\nolimits}_3(q^ 2).2,\quad M\cong G_2(q_0)\;\;({\mathbb{F}}_{q_0}<{\mathbb{F}}_q).$$

\begin{lemma}\label{PSU(3,q)} 
If $H$ is absolutely irreducible, it is not contained in ${\mathop{\rm SL}\nolimits}_3(q)$ or ${\mathop{\rm SU}\nolimits}_3(q^ 2)$.
\end{lemma}

\begin{proof}
If our claim is false,  $H$ arises from an action of ${\mathop{\rm SL}\nolimits}_3({\mathbb{F}})$ 
on the symmetric square $S({\mathbb{F}}^6)$ (see \cite[5.4.11]{KL}). But, in this action, an involution has similarity invariants $t+1$, $t+1$, $t^2+1$, $t^2+1$,
different from those of $x_6$.
\end{proof}

\begin{rem}\label{F1} 
The \emph{field of definition} of a subgroup $H$ of ${\mathop{\rm GL}\nolimits}_n(q)$  is the smallest subfield ${\mathbb{F}}_{q_1}$
of ${\mathbb{F}}_q$ such that a conjugate of $H$, under ${\mathop{\rm GL}\nolimits}_n({\mathbb{F}})$, is contained in ${\mathop{\rm GL}\nolimits}_n({\mathbb{F}}_{q_1})({\mathbb{F}}^\ast I)$, where $({\mathbb{F}}^\ast I)$ denotes the center of ${\mathop{\rm GL}\nolimits}_n({\mathbb{F}})$. This is to ensure that no conjugate of the projective image of $H$ lies in ${\mathop{\rm PGL}\nolimits}_n({\mathbb{F}}_{q_0})$ for some $q_0<q_1$. Clearly, when $H=H'$ is perfect, 
${\mathbb{F}}_{q_1}$ coincides with the smallest subfield of ${\mathbb{F}}_q$ such that a conjugate of $H$ is contained in ${\mathop{\rm SL}\nolimits}_n({\mathbb{F}}_{q_1})$,
the derived subgroup of  ${\mathop{\rm GL}\nolimits}_n({\mathbb{F}}_{q_1})\left({\mathbb{F}}^*I\right)$.
\end{rem}

\begin{theorem}\label{G2Even}
Let $x_6,y_6$ as in \eqref{x,y} with $r\in {\mathbb{F}}_{q}\setminus {\mathbb{F}}_4$, $8\le q$ even, such that:
\begin{itemize}
\item[(i)] $r^{12} + r^9 + r^5 +  r^2 +1\neq 0$;
\item[(ii)] ${\mathbb{F}}_q={\mathbb{F}}_2[ r^2+r]$.
\end{itemize}
Then $H=\langle x_6,y_6\rangle=G_2(q)$ and there exists
$r\in {\mathbb{F}}_{q}\setminus {\mathbb{F}}_4$ satisfying {\rm (i)} and {\rm (ii)}.
\end{theorem}

\begin{proof}
Condition (i) implies that $H$ is absolutely irreducible (Lemma \ref{lemma: irreducibility}) and by the same
lemma we have that $H\leq G_2(q)$.  Let $M$ be a maximal subgroup of $G_2(q)$ which contains $H$. 
Since $H$ is a Hurwitz group, it is perfect and so it is contained in the derived subgroup $M'$ of $M$. 
By  \cite[Lemma 3.2]{TV}, the minimal field of definition of $H$ is ${\mathbb{F}}_2[ r^2+r]$. 
Thus Condition (ii) gives that $M'\neq G_2(q_0)$ for any $q_0$ such that ${\mathbb{F}}_{q_0}$ is a
proper subfield of ${\mathbb{F}}_q$.
Moreover $M'\not\in \{{\mathop{\rm SL}\nolimits}_3(q),{\mathop{\rm SU}\nolimits}_3(q^2)\}$ by Lemma \ref{PSU(3,q)}. It follows that $H=G_2(q)$.
We now prove that, for $q>4$, there exists $r\in {\mathbb{F}}_{q}\setminus {\mathbb{F}}_4$, where $q=2^a$, satisfying Conditions (i) and (ii).
For each $\alpha\in {\mathbb{F}}_q$ such that ${\mathbb{F}}_2[\alpha]\neq {\mathbb{F}}_q$ there are at most two values of $r$ such that $r^ 2+r=\alpha$.
Let $N(a)$ be the number of elements $r\in {\mathbb{F}}_{2^a}$ such that ${\mathbb{F}}_2[r^ 2+r]\neq {\mathbb{F}}_{2^a}$.
Considering the possible subfields of ${\mathbb{F}}_{2^a}$, we get
$$N(a)\leq 2\left(2+2^ 2+2^3+\ldots+2^ {\lfloor\frac{a}{2} \rfloor}\right)= 2^ {2+\lfloor\frac{a}{2} \rfloor}-4.$$ 
Next, observe that if $a\geq 5$, then 
$\left(2^ {2+\lfloor\frac{a}{2} \rfloor}-4 \right)+12<2^ a$.
It follows that for $a\geq 5$, there exists $r\in {\mathbb{F}}_{2^a}\setminus {\mathbb{F}}_4$ satisfying Conditions (i) and (ii).
Finally for $q=8,16$, it suffices to take as $r$ a generator of ${\mathbb{F}}_{q}^ \ast$.
\end{proof}
 
The following Remark gives Hurwitz generators of the Janko group $J_2$ over ${\mathbb{F}}_4$.

\begin{rem}
For $q=4$, consider the matrices (belonging to family (IIa) of \cite{V3})
\begin{equation}\label{J2}
x_{J_2}=\begin{pmatrix}
0 & 0 & 1 & 0 & \omega & 0 \\
0 & 0 & 0 & 1 & 1 & \omega^2\\
1 & 0 & 0 & 0 & 0 & \omega \\
0 & 1 & 0 & 0 & \omega^2 & 1 \\
0 & 0 & 0 & 0 & 0   & 1\\
0 & 0 & 0 & 0 & 1   & 0
    \end{pmatrix},\quad
y_{J_2}=\begin{pmatrix}
  1 & 0 & 0 & 0 & \omega^2 & \omega^2\\
  0 & 1 & 0 & 0 & \omega & \omega^2 \\
  0 & 0 & 0 & 0 & 1   & 0\\
  0 & 0 & 0 & 0 & 0   & 1 \\
  0 & 0 & 1 & 0 & 1   & 0 \\
  0 & 0 & 0 & 1 & 0   & 1
  \end{pmatrix},
\end{equation}
where $\omega^2+\omega+1=0$.
By a Magma calculation the group $H=\left\langle x_{J_2} , y_{J_2}\right\rangle$ as in \eqref{J2}
 is isomorphic to $J_2$.
 \end{rem}

For the rest of this Section,  we suppose $5\le q$ odd. In this case  our generators have similarity invariants \eqref{inv7} hence are conjugate to the matices in \eqref{shape7}, but are different. Indeed we
take them from family (II) in \cite{TV2}. 

For each  $r\in {\mathbb{F}}_q$ define $H=\langle x_7,y_7\rangle$, where

\begin{equation}\label{7odd}
x_7=\begin{pmatrix}
 0 &  0 & 0 & 1 & 0 & 0 & -4\\
 0 &  0 & 0 & 0 & 1 & 0 & r \\
 0 &  0 & 0 & 0 & 0 & 1 & -3 \\
 1 &  0 & 0 & 0 & 0 & 0 & -4 \\
 0 &  1 & 0 & 0 & 0 & 0 & r \\
 0 &  0 & 1 & 0 & 0 & 0 & -3 \\
 0 &  0 & 0 & 0 & 0 & 0 & -1 
\end{pmatrix},\;
y_7=
\begin{pmatrix}
 1 & 0 & 0 & 0 & 1 & 0 & r+2\\
 0 & 1 & 0 & 0 & 2 & 0 & 2r+8\\
 0 & 0 & 1 & 1 & 0 & 0 & -4 \\ 
 0 & 0 & 0 & 0 &-1 & 0 & 0 \\
 0 & 0 & 0 & 1 &-1 & 0 & 0 \\
 0 & 0 & 0 & 0 & 0 & 0 & -1 \\ 
 0 & 0 & 0 & 0 & 0 & 1 & -1
\end{pmatrix}.
\end{equation}

\begin{lemma}\label{Irrodd}
The group $H$ is absolutely irreducible, except when:
$$d=r^2+15 r+ 100=0.$$
In particular, if $H$ is absolutely irreducible, then it is a subgroup of $G_2(q)$.
\end{lemma}

\begin{proof}
For the absolute irreducibility of $H$ see the proof of Theorem 1, page 2131, in \cite{TV2}.
For the second claim of the statement, apply Theorem   \ref{p7}. 
\end{proof}
We notice that $[x_7,y_7]=x_7y_7^{-1}x_7y_7$ has trace $3$ and characteristic polynomial 
\begin{equation}\label{char7}
\chi_{[x_7,y_7]}(t)=t^ 7 -3t^ 6  -(d-5) t^ 5 -(d+7) t^ 4 + (d+7) t^ 3 + (d-5)t^ 2+3t-1.
\end{equation}
Again we want  to exclude that $H$ 
is contained in a maximal subgroup of $G_2(q)$. For $q$ odd, the classification is due 
to P. Kleidman in \cite{K}. We summarize in Table \ref{list} the relevant information of 
\cite[Tables 8.41, 8.42 pages 397--398]{H}.

\begin{table}[!h]
\begin{itemize}
\item $G_2(q_0)$, where ${\mathbb{F}}_{q_0}$ is a subfield of ${\mathbb{F}}_q$;
\item ${2^ 3}^.{\mathop{\rm SL}\nolimits}_3(2)$,  $q=p$; 
\item ${\mathop{\rm PSL}\nolimits}_2(13)$, $q=p\equiv \pm 1,\pm 3,\pm 4\pmod{13}$ or $q=p^ 2,\ p\equiv \pm 2, \pm 5,\pm 6\pmod{13}$;
\item ${\mathop{\rm PSL}\nolimits}_2(8)$, $q=p\equiv \pm 1\pmod 9$ or $q=p^ 3,\ p= \pm 2,\pm 4\pmod{9}$;
\item  $G_2(2)= U_3(3):2$,  $q=p\ge 5$;
\item $J_1$, $q=11$;
\item ${\mathop{\rm PGL}\nolimits}_2(q)$, $p\ge 7$, $q\ge 11$; 
\item ${\mathop{\rm SL}\nolimits}_3(q):2$ and ${\mathop{\rm SU}\nolimits}_3(q^2):2$, $p=3$; 
\item $^2G_2(3^a)$, $p=3$, $a$ odd.
\end{itemize}
\caption{ Maximal irreducible subgroups of $G_2(q)$, $q=p^a$, $p\ne 2$.} \label{list}
\end{table}

Since $H$ is a Hurwitz group, it is perfect. So if $H$ is contained in a
maximal subgroup $M$ then $H\le M'$.

\begin{lemma}\label{vari}
Assume $5\le q$ odd, $H$ absolutely irreducible. Then:
\begin{itemize} 
\item[(\rm{i})] $H$ is never contained in any maximal subgroup $M$ isomorphic to one of the following:\enskip
${\mathop{\rm PSL}\nolimits}_2(13),\ {\mathop{\rm PSL}\nolimits}_2(8),\ G_2(2),\ {2^3}^.{\mathop{\rm PSL}\nolimits}_3(2),\ J_1$;
\item[(\rm{ii})] $H$ is contained in a maximal subgroup $M\cong {\mathop{\rm PSL}\nolimits}_2(8)$ iff $q=17$ and $r=1$.
\end{itemize}
\end{lemma}

\begin{proof}
We proceed case by case. If $H\le M$, then $(x_7,y_7,x_7y_7)$ must coincide with some
 $(2,3,7)$-triple $\left(g_2,g_3, g_2g_3\right)$ in $M$.
Since $[x_7,y_7]$ has trace $3$, we are interested in those triples such that $3={{\rm Tr}({[g_2,g_3]})}$. The possible values 
of these traces are read from the ordinary and modular character tables of $M$. 

\begin{itemize}
\item[(\rm{a})] Assume  $H\le M\cong{\mathop{\rm PSL}\nolimits}_2(13)$, with the above restrictions on $q$: in particular $p\ne 3$.
In this case
the order of $[g_2,g_3]$ is either $6$ or $7$ or $13$ with  trace
$1$,  $0$ or $(1\pm \sqrt{13})/2$, respectively.
Equating these values to $3$ we get either $p=2$ or $p=3=q$, in contrast with our assumptions.

\item[(\rm{b})] Assume $H\le M\cong{\mathop{\rm PSL}\nolimits}_2(8)$ with the above restrictions on $q$. 
In this case $[g_2,g_3]$ has order $9$ and its trace is a root of $t^3-3t-1$. 
Equating its trace to $3$, we get $3^3-9-1=17=0$, whence $p=17=q$.
On the other hand, for $q=17$, the element $[x_7,y_7]$ has order $9$ only when $r=1$.
In this case $H\cong {\mathop{\rm PSL}\nolimits}_2(8)$.

\item[(\rm{c})] Assume $H\le M\cong G_2(2)$,  $q=p\ge 5$. Since $H$ is perfect, we 
have that $H$ is actually contained in $M'={\mathop{\rm SU}\nolimits}_3(3)$. In this case $[g_2,g_3]$ has
order $4$. Taking $D=[x_7,y_7]^ 4$, we have $D_{7,7}=5r-7$.
If $p=5$, then $D_{7,7}=3\ne 1$. If $p\neq 5$, set
$r=\frac{8}{5}$. Then $D_{2,7}=-\frac{36}{5}\neq 0$, since $p\geq 7$.

\item[(\rm{d})] Assume $H\le M\cong {2^ 3}^.{\mathop{\rm PSL}\nolimits}_3(2)$ with $q=p\geq 5$.
If $p\neq 7$, then the trace of $[g_2,g_3]$ is in $\{0,\pm 1\}$. As in 
(a) above we  get either $p=2$ or $p=3=q$, against our assumptions.
If $q=p=7$, then  $[x_7,y_7]$ has order $\geq 19$, so it is cannot be an element
of ${2^ 3}^ .{\mathop{\rm PSL}\nolimits}_3(2)$.

\item[(\rm{e})] Assume $M\cong J_1$ with $q=11$.
The element $[g_2,g_3]$ has order $10$, $11$, $15$ or  $19$ with respective traces $9$, $7$, $5$ and $4$.
Since ${{\rm Tr}({[x_7,y_7]})}=3$ we have a contradiction.
\end{itemize}
\end{proof}

\begin{lemma}\label{SL3}
If $H$ is absolutely irreducible and $p=3$, then it is not contained in ${\mathop{\rm SL}\nolimits}_3(q)$ or ${\mathop{\rm SU}\nolimits}_3(q^ 2)$.
\end{lemma}

\begin{proof}
The Hurwitz subgroups of ${\mathop{\rm SL}\nolimits}_3(q^ 2)$ are isomorphic to ${\mathop{\rm PSL}\nolimits}_2(7)$ or ${\mathop{\rm PSL}\nolimits}_2(27)$. 
However ${\mathop{\rm PSL}\nolimits}_2(27)$ does not have irreducible representations of degree $7$.
If $H$ is  contained in ${\mathop{\rm PSL}\nolimits}_2(7)$, then $[x_7,y_7]$ has order $4$, whence  
${{\rm Tr}({[x_7,y_7]})}=-1$, a contradiction.
\end{proof}

\begin{lemma}\label{nctriples}
Assume $H$ absolutely irreducible. Then two triples $(x_7(r_1),y_7(r_1),$ $z_7(r_1))$ and $(x_7(r_2),y_7(r_2),z_7(r_2))$ 
are conjugate if, and only if, $r_1=r_2$. 
\end{lemma}

\begin{proof}
Assume that $(x_7(r_1),y_7(r_1),z_7(r_1))$ and $(x_7(r_2),y_7(r_2),z_7(r_2))$ are conjugate. 
Then the characteristic polynomial of $[x_7(r_1),y_7(r_1)]$ and $[x_7(r_2),y_7(r_2)]$ are the same, whence 
$r_1^ 2+15r_1=r_2^ 2+15r_2$. 
Suppose that $r_1\neq r_2$. Then $r_1=-15-r_2$.
Now, consider the element $w=[x_7,y_7]^ 2 y_7 (x_7y_7)^ 2 $ whose characteristic polynomial is
$$\chi_w(t)=t^ 7+f_1(r) t^ 6+f_2(r) t^ 5 - f_3(r)t^ 4 +f_3(r) t^ 3-f_2(r) t^ 2-f_1(r) t -1$$
where $f_1(r)=   r^3+25 r^2+250 r+999$.
From $f_1(r_1)=f_1(r_2)$ we get $d(2r_2+15)=0$. Since $H$ is absolutely irreducible, then
$d\neq 0$ and so
$r_2=-\frac{15}{2}$. However, in this case $r_1=-15-r_2=-\frac{15}{2}=r_2$, a contradiction.
We conclude that $r_1=r_2$.
\end{proof}

As mentioned in the Introduction, the group ${\mathop{\rm PSL}\nolimits}_2(q)$ is Hurwitz when either $q=p\equiv 0,\pm 1 \pmod 7$ or
$q=p^3$ and $p\equiv \pm 2, \pm 3\pmod 7$.
Moreover, in $\textrm{Aut}({\mathop{\rm PSL}\nolimits}_2(q))={\mathop{\rm PGL}\nolimits}_2(q)$, there are $3$ conjugacy classes of Hurwitz triples 
for $q=p\equiv \pm 1 \pmod 7$, only one otherwise.

\begin{theorem}
Let $q=p^a$, $p$ odd, $q\ge 5$, and let $x_7,y_7$ be as in \eqref{7odd} with $r \in {\mathbb{F}}_q$
such that  $r\ne 1$ if $q=17$ and $r^2+15 r +100\neq 0$. Assume further that ${\mathbb{F}}_q={\mathbb{F}}_p[r]$.
Then one of the following holds:
\begin{itemize}
\item[(\rm{i})] $H=G_2(q)$;

\item[(\rm{ii})] $q=p\equiv 0,\pm 1 \pmod 7$ or  
$q=p^3$ with $p\equiv \pm 2, \pm 3\pmod 7$ and $H$ is contained in a subgroup $M$ isomorphic to ${\mathop{\rm PGL}\nolimits}_2(q)$.
\end{itemize}
Also in case {\rm (ii)} it is always possible to choose $r$ such that  
$H= G_2(q)$.
\end{theorem}

\begin{proof}
The group $H$ is absolutely irreducible by  Lemma \ref{Irrodd} and the assumption $r^2+15 r +100\neq 0$ and so it is a subgroup of $G_2(q)$. 
Suppose that there exists a maximal subgroup $M$ of $G_2(q)$ containing $H$.
Condition ${\mathbb{F}}_q={\mathbb{F}}_p[r]$ implies that ${\mathbb{F}}_{q_0}$ is the field of definition of $H$ (see \cite[Remark 6]{TV2}). Thus
$M\neq G_2(q_0)$ for every $q_0< q$. By Lemma \ref{vari} we may also exclude $M'\in \{{\mathop{\rm PSL}\nolimits}_2(13), {\mathop{\rm PSL}\nolimits}_2(8), G_2(2), 
{2^ 3}^.{\mathop{\rm SL}\nolimits}_3(2)\}$.
When $q=3^a$, we may also exclude $M'\in \{{\mathop{\rm SL}\nolimits}_3(q), {\mathop{\rm SU}\nolimits}_3(q^2)\}$ by Lemma \ref{SL3}. If $a$ is odd, suppose $M\cong {}^2G_2(3^a)$. 
By \cite[Proposition 3.14]{B} two semisimple elements of $^2G_2(3^a)$
which have the same trace are conjugate. Now $x_7y_7$ has trace $0$ and, for $p=3$,
also $[x_7,y_7]$ has trace $0$. So $[x_7,y_7]^ 7$ must be a $3$-element and so have trace $1$. 
However this gives $r(r^2-1)=0$ and so $r\in {\mathbb{F}}_3$, a contradiction with the assumption ${\mathbb{F}}_q={\mathbb{F}}_3[r]$.
Thus $H$ is not contained in 
any subgroup of  Table \ref{list} and we conclude  that $H=G_2(q)$.

Now suppose that we are in case (\rm{ii}). Then either $H=G_2(q)$ or $H\leq M'$ with $M\cong {\mathop{\rm PGL}\nolimits}_2(q)$. 
In $G_2(q)$ there is just one conjugacy class of maximal subgroups isomorphic to ${\mathop{\rm PGL}\nolimits}_2(q)$. Thus, by the information
given  before the statement,
there are at most $3$ non-conjugate
Hurwitz triples in $G_2(q)$ which can generate ${\mathop{\rm PSL}\nolimits}_2(q)$. By Lemma \ref{nctriples},
different values of $r$ give rise to non conjugate triples. So we have to exclude at most 3
values of $r$ to avoid $H\le {\mathop{\rm PGL}\nolimits}_2(q)$. Clearly we have to exclude at most $2$ values of $r$ for the absolute irreducibility.
We also note that, if $H\le {\mathop{\rm PGL}\nolimits}_2(q)$, then $q\ne 17$. Our last claim now follows from the inequalities:
$p\ge 7>2+3$ if $q=p\equiv 0,\pm 1 \pmod 7$, and  $p^3-p>2+3$  if $q=p^3$.
\end{proof}

For sake of completeness, we  provide Hurwitz generators also for the Janko  group $J_1$.
Consider the following matrices $x_{J_1},y_{J_1}\in {\mathop{\rm SL}\nolimits}_7(11)$ (Family (I) of \cite{TV2}, with $r_3=2$ and $r_4=4$):
$$
x_{J_1}=\begin{pmatrix}
 0 &  0 &  0 &  1 &  0 &  0 & -1\\
 0 &  0 &  0 &  0 &  1 &  0 &  5\\
 0 &  0 &  0 &  0 &  0 &  1 &  2\\
 1 &  0 &  0 &  0 &  0 &  0 & -1\\
 0 &  1 &  0 &  0 &  0 &  0 &  5\\
 0 &  0 &  1 &  0 &  0 &  0 &  2\\
 0 &  0 &  0 &  0 &  0 &  0 & -1
\end{pmatrix}, \quad
y_{J_1}=\begin{pmatrix}
 1 &  0 &  0 &  0 &  4 &  0 &  3\\
 0 &  1 &  0 &  0 &  8 &  0 & -1\\
 0 &  0 &  1 &  1 &  0 &  0 &  9\\
 0 &  0 &  0 &  0 & -1 &  0 &  0\\
 0 &  0 &  0 &  1 & -1 &  0 &  0\\
 0 &  0 &  0 &  0 &  0 &  0 & -1\\
 0 &  0 &  0 &  0 &  0 &  1 & -1
\end{pmatrix}.
$$
Again, by a Magma calculation, the group $H=\left\langle x_{J_1} , y_{J_1}\right\rangle$  is isomorphic to $J_1$.

\begin{thebibliography}{20}

\bibitem{A} M. Aschbacher, Chevalley Groups of Type $G_2$ as the Group of a Trilinear
Form, \emph{J. Algebra} \textbf{109}, 193--259 (1987).

\bibitem{B} H. B\"a\"arnhielm, Recognising the small Ree groups in their
natural representations, \emph{J. Algebra} \textbf{416} (2014), 139--166. 
 
\bibitem{H} J.N. Bray, D.F. Holt \and  C.M. Roney-Dougal, \emph{The maximal subgroups of the low-dimensional finite classical groups}, London Mathematical Society Lecture Note Series, 407. Cambridge University Press, Cambridge, 2013

\bibitem{C} B.N. Cooperstein, Maximal Subgroups of $G_2(2^n)$, \emph{J. Algebra} \textbf{70}, 23--36 (1981).

\bibitem{Dieu} J. Dieudonn\'e, \emph{Sur les groupes classiques}, Actualit\'es
scientifiques et industrielles 1040, Herman, Paris, 1948.

\bibitem{DTZ}  L. Di Martino, M.C. Tamburini \and A. Zalesskii, On Hurwitz groups of low rank, \emph{Comm. Algebra},
\textbf{28}  (2000) no. 11, 5383--5404.

\bibitem{K} P.B. Kleidman, The maximal subgroups of the Chevalley groups $G_2(q)$ with $q$ odd, 
the Ree groups ${}^2G_2(q)$, and their automorphism groups, \emph{J. Algebra} \textbf{117} (1988), no. 1, 30--71. 

\bibitem{KL} P. Kleidman \and M. Liebeck, \emph{The subgroup structure of the finite classical groups},
London Mathematical Society Lecture Note Series, 129. Cambridge University Press, Cambridge, 1990.

\bibitem{Mac} A.M. Macbeath, Generators of the linear fractional groups, \emph{Proc.
Symp. Pure Math.} \textbf{12} (1969), 14--32.

\bibitem{Ma} G. Malle, Hurwitz groups and $G_2(q)$, \emph{Canad. Math. Bull.} \textbf{33} (1990), no. 3, 349--357. 

\bibitem{M} C. Marion, On triangle generation of finite groups of Lie type, \emph{J. Group Theory}, \textbf{13} (2010), 619--648.

\bibitem{S} L.L. Scott, Matrices and cohomology, \emph{Ann. Math.} \textbf{105} (1977), 473--492.

\bibitem{hb} M.C. Tamburini \and M. Vsemirnov, Hurwitz groups and Hurwitz generation.
\emph{Handbook of algebra}. Vol. 4, 385--426, Elsevier/North-Holland, Amsterdam, 2006.

\bibitem{TV1} M.C. Tamburini \and M. Vsemirnov, Irreducible $(2,3,7)$-subgroups of ${\rm PGL}_n(F),\ n\le 7$, \emph{J. Algebra} \textbf{300} (2006), 339--362.

\bibitem{TV2} M.C. Tamburini \and M. Vsemirnov, Irreducible $(2,3,7)$-subgroups of ${\rm PGL}_n(F),\ n\le 7$, II.,
\emph{J. Algebra} \textbf{321}  (2009),  no. 8, 2119--2138.

\bibitem{TV}  M.C. Tamburini Bellani \and  M. Vsemirnov, Hurwitz generation of ${\rm
PSp}_6(q)$, to appear in \emph{Comm. Algebra}.

\bibitem{TZ} M.C. Tamburini \and A. Zalesskii, Classical groups in
dimension $5$ which are Hurwitz, \emph{Proceedings of the Gainesville
Conference on Finite Groups, 2003}, Editors: C.Y. Ho, P. Sin,
P.H. Tiep, A. Turull, Walter de Gruyter, (2004), 363--371.

\bibitem{Tay} D.E. Taylor, \emph{The geometry of the classical groups}, 
Sigma Series in Pure Mathematics, 9. Heldermann Verlag, Berlin, 1992.

\bibitem{T} K.B. Tchakerian, An explicit $(2,3,7)$-generation of the simple Ree groups
${}^2G_2(3^n)$, 
\emph{C. R. Acad. Bulgare Sci.} \textbf{58} (2005), no. 7, 749--752. 

\bibitem{VZ} R. Vincent \and A.E. Zalesski, Non-Hurwitz classical groups, \emph{LMS J. Comp. Math.} \textbf{10} (2007), 21--82.

\bibitem{V3} M. Vsemirnov, Irreducible $(2,3,7)$-subgroups of ${\mathop{\rm PGL}\nolimits}_n(F),\ n\leq 7$, III (in preparation).

\end{thebibliography}
\end{document}
