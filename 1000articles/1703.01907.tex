
gsave
newpath
  20 20 moveto
  20 220 lineto
  220 220 lineto
  220 20 lineto
closepath
2 setlinewidth
gsave
  .4 setgray fill
grestore
stroke
grestore
\end{filecontents*}
\RequirePackage{fix-cm}
\documentclass[smallextended]{svjour3}       
\smartqed  
\usepackage{graphicx}
\usepackage{epsfig}
\usepackage{setspace}
\usepackage{amssymb}
\usepackage{amsmath}
\usepackage{mathrsfs}
\usepackage{here}
\usepackage{url}
\begin{document}

\title{Evaluation of some non-elementary integrals of sine, cosine and exponential integrals type}



\author{Victor Nijimbere}


\institute{Victor Nijimbbere \at
              School of mathematics and Statistics\\
              Carleton University, Ottawa ON \\
              Tel.: +1-613-520-2600 ext. 2137\\
              Fax: +1-613-520-3536\\
              \email{nijimberevictor@cmail.carleton.ca}
}

\date{Received: date / Accepted: date}


\maketitle

\begin{abstract}
The non-elementary integrals $\text{Si}_{\beta,\alpha}=\int [\sin{(\lambda x^\beta)}/(\lambda x^\alpha)] dx$ and $\text{Ci}_{\beta,\alpha}=\int [\cos{(\lambda x^\beta)}/(\lambda x^\alpha)] dx$, where $\beta\ge1$ and $\alpha\ge1$, are evaluated in terms of the hypergeometric functions $_{1}F_2$ and $_{2}F_3$ respectively, and their asymptotic expressions for $|x|\gg1$ are derived. Integrals of the form $\int [\sin^n{(\lambda x^\beta)}/(\lambda x^\alpha)] dx$ and $\int [\cos^n{(\lambda x^\beta)}/(\lambda x^\alpha)] dx$, where n is a positive integer, are expressed in terms $\text{Si}_{\beta,\alpha}$ and $\text{Ci}_{\beta,\alpha}$, and then evaluated.

On the other hand, $\text{Si}_{\beta,\alpha}$ and $\text{Ci}_{\beta,\alpha}$ are evaluated in terms of the hypergeometric function $_{2}F_2$. And so, the hypergeometric functions, $_{1}F_2$ and $_{2}F_3$, are expressed in terms of $_{2}F_2$.

The integral $\text{Ei}_{\beta,\alpha}=\int (e^{\lambda x^\beta}/x^\alpha) dx$ where $\beta\ge1$ and $\alpha\ge1$, and the logarithmic integral $\text{Li}=\int_{2}^{x} dt/\ln{t}$, are expressed in terms of $_{2}F_2$, and their asymptotic expressions are investigated. It is found that $\text{Li}\sim {x}/{\ln{x}}+\ln{\left(\frac{\ln{x}}{\ln{2}}\right)}-2-
\ln{2}\hspace{.075cm} _{2}F_{2}(1,1;2,2;\ln{2})$, where the term $\ln{\left(\frac{\ln{x}}{\ln{2}}\right)}-2-
\ln{2}\hspace{.075cm} _{2}F_{2}(1,1;2,2;\ln{2})$ is negligible if $x\sim O(10^6)$ or higher.

\keywords{Non-elementary integrals \and Sine integral \and Cosine integral \and Exponential integral \and Logarithmic integral \and Hypergeometric functions \and Asymptotic evaluation \and Fundamental theorem of calculus}
\subclass{26A36 \and 33C15 \and 30E15}
\end{abstract}

\section{Introduction}\label{intro}
\begin{definition}
An elementary function is a function of one variable constructed using that variable and constants, and by performing a finite number of repeated algebraic operations involving exponentials and logarithms.
An indefinite integral which can be expressed in terms of elementary functions is an elementary integral. If it cannot be evaluated in terms of elementary functions, then it is non-elementary\cite{MZ,R}.
\label{defn1}
\end{definition}
Liouville 1938's Theorem gives conditions to determine whether or not a given integral is elementary or non-elementary \cite{MZ,R}. It was shown in \cite {MZ,R}, using  Liouville 1938's Theorem, that $\text{Si}_{1,1}=\int (\sin{x}/x) dx$ is non-elementary. With similar arguments as in \cite {MZ,R}, One can show that $\text{Ci}_{1,1}=\int (\cos{x}/x) dx$ is also non-elementary. Using the Euler identity $e^{\pm ix}=\cos{x}\pm i\sin{x}$, and noticing that if the integral of $g(x)$ is elementary, then both its real and imaginary parts are elementary \cite{MZ}, one can readily prove that the integrals $\text{Si}_{\beta,\alpha}=\int [\sin{(\lambda x^\beta)}/(\lambda x^\alpha)] dx$ and $\beta\ge1$, and $\text{Ci}_{\beta,\alpha}=\int [\cos{(\lambda x^\beta)}/(\lambda x^\alpha)] dx$, where $\beta\ge1$ and $\alpha\ge1$, are non-elementary by using the fact that their real and imaginary parts are non-elementary.

The integrals $\int [\sin^n{(\lambda x^\beta)}/(\lambda x^\alpha)] dx$ and $\int [\cos^n{(\lambda x^\beta)}/(\lambda x^\alpha)] dx$, where n is a positive integer, are also non-elementary since they can be expressed in terms of $\text{Si}_{\beta,\alpha}$ and $\text{Ci}_{\beta,\alpha}$.

To my knowledge, no one has evaluated these integrals before. To this end, in this paper, formulas for these non-elementary integrals are expressed in terms of the hypergeometric functions  $_{1}F_2$ and $_{2}F_3$, which are entire functions on the whole complex space $\mathbb{C}$, and whose properties are well known \cite{K,ND}. And therefore, their corresponding definite integrals can be evaluated using the Fundamental Theorem of Calculus (FTC). For example, the sine integral
\begin{equation}
\text{Si}_{\beta,\alpha}=\int\limits_{A}^{B} \frac{\sin {(\lambda x^\beta)}}{(\lambda x^\alpha)} dx
\end{equation}
is evaluated for any $A$ and $B$ using the FTC.

On the other hand, the integrals  $\text{Ei}_{\beta,\alpha}=\int (e^{\lambda x^\beta}/x^\alpha) dx$ and $\int dx/\ln{x}$, can be expressed in terms of the hypergeometric function $_{2}F_2$. This is quite important since one may re-investigate the asymptotic behavior of the exponential ($\text{Ei}$) and logarithmic ($\text{Li}$) integrals using the asymptotic expression of the hypergeometric function $_{2}F_2$.


Other non-elementary integrals can be written in terms of $\text{Ei}_{\beta,\alpha}$ or $\int dx/\ln{x}$.  For instance, as a result of substitution, the integral $\int e^{\lambda e^{\beta x}} dx$ can be written in terms of $\text{Ei}_{\beta,1}=\int (e^{\lambda x^\beta}/x) dx$ and then evaluated, and using integration by parts, the integral $\int \ln(\ln{x}) dx$ can be written in terms of $\int dx/\ln{x}$ and then evaluated.

Using the Euler identity $e^{\pm ix}=\cos(x)\pm i\sin(x)$ or the hyperbolic identity $e^{\pm x}=\cosh(x)\pm\sinh(x)$, $\text{Si}_{\beta,\alpha}$ and $\text{Ci}_{\beta,\alpha}$ can be evaluated in terms $\text{Ei}_{\beta,\alpha}$. In that case, one can express the hypergeometric functions $_{1}F_2$ and $_{2}F_3$ in terms of the hypergeometric $_{2}F_2$.


\section{Evaluation of the sine integral and related integrals} \label{sine-integral}

\begin{proposition}
The function $G(x)={x}\hspace{.075cm} _{1}F_{2}\left(\frac{1}{2};\frac{3}{2},\frac{3}{2};-\frac{\lambda^2 x^2}{4}\right)$, where ${}_1F_2$ is a hypergeometric function \cite{AS} and $\lambda$ is an arbitrarily constant, is the antiderivative of the function $g(x)=\frac{\sin{(\lambda x)
}}{\lambda x}$. Thus,
\begin{equation}
\int\frac{\sin{(\lambda x)}}{\lambda x}dx={x}\hspace{.075cm} _{1}F_{2}\left(\frac{1}{2};\frac{3}{2},\frac{3}{2};-\frac{\lambda^2 x^2}{4}\right)+C.
\label{prp1-1}
\end{equation}
\label{prp1}
\end{proposition}

\begin{proof}
To prove Proposition \ref{prp1},
we expand $g(x)$ as Taylor series and integrate the series term by term.
We also use the gamma duplication formula \cite{AS}
\begin{equation}
\Gamma(2\alpha)=(2\pi)^{-\frac{1}{2}}2^{2\alpha-\frac{1}{2}}\Gamma(\alpha)\Gamma\left(\alpha+\frac{1}{2}\right), \hspace{.15cm} \alpha\in\mathbb{C},
\label{duplication}
\end{equation}
the Pochhammer's notation for the gamma function \cite{AS},
\begin{equation}
(\alpha)_n=\alpha (\alpha+1)\cdot\cdot\cdot(\alpha+n-1)=\frac{\Gamma(\alpha+n)}{\Gamma(\alpha)}, \hspace{.15cm} \alpha\in\mathbb{C},
\label{Po}
\end{equation}
and the property of the gamma function $\Gamma(\alpha+1)=\alpha\Gamma(\alpha)$. For example, $\Gamma\left(n+\frac{3}{2}\right)=\left(n+\frac{1}{2}\right)\Gamma\left(n+\frac{1}{2}\right)$ for any real $n$. We then obtain

\begin{align}
\int g(x)dx&=\int \frac{\sin{(\lambda x)}}{\lambda x}dx
\nonumber \\&=\int\frac{1}{\lambda x}\sum\limits_{n=0}^{\infty}(-1)^n\frac{(\lambda x)^{2n+1}}{(2n+1)!}dx
\nonumber\\&=\frac{x}{2}\sum\limits_{n=0}^{\infty}(-1)^n\frac{\lambda^{2n}}{(2n+1)!}\frac{x^{2n}}{n+\frac{1}{2}}+C
\nonumber\\&=\frac{x}{2}\sum\limits_{n=0}^{\infty}\frac{\Gamma\left(n+\frac{1}{2}\right)}{\Gamma(2n+2)\Gamma\left(n+\frac{3}{2}\right)}(-\lambda x^{2})^{n}+C
\nonumber\\&={x}\sum\limits_{n=0}^{\infty}\frac{\left(\frac{1}{2}\right)_n}{\left(\frac{3}{2}\right)_n\left(\frac{3}{2}\right)_n}\frac{\left(-\frac{\lambda^2 x^2}{4}\right)^{n}}{n!}+C
\nonumber\\&={x}\hspace{.075cm} _{1}F_{2}\left(\frac{1}{2};\frac{3}{2},\frac{3}{2};-\frac{\lambda^2 x^2}{4}\right)+C \nonumber \\&=
G(x)+C.
\label{prp1-2}
\end{align}
\nonumber
\end{proof}

\begin{lemma}
Assume that $G(x)$ is the antiderivative of $g(x)=\frac{\sin{x}}{x}$ ($\lambda=1$).
\begin{enumerate}
\item Then $g(x)$ is linear at $x=0$ and the point $(0,G(0))=(0,0)$ is an inflection point of the curve $Y=G(x)$.
\item And $\lim\limits_{x\rightarrow-\infty}G(x)=-\theta$ while $\lim\limits_{x\rightarrow+\infty}G(x)=\theta$, where $\theta$ is a positive finite constant.
\end{enumerate}
\label{Lm1}
\end{lemma}

\begin{proof}
\begin{enumerate}
\item 
 The series $g(x)=\frac{\sin{x}}{x}=\sum\limits_{n=0}^{\infty}(-1)^n\frac{(\lambda x)^{2n}}{(2n+1)!}$ gives $G^\prime(0)=g(0)=1$. Then, around $x=0$, $G(x)\sim x$ since $G^\prime(0)=g(0)=1$. And so, $(0,G(0))=(0,0)$. Moreover $G^{\prime\prime}(0)=g^\prime(0)=0$. Hence, by the second derivative test, the point $(0,G(0))=(0,0)$ is an inflection point of the curve $Y=G(x)$.

\item By Squeeze theorem, $\lim\limits_{x\rightarrow-\infty}g(x)=\lim\limits_{x\rightarrow+\infty}g(x)=0$, and since both $g(x)$ and $G(x)$ are analytic on $\mathbb{R}$, $G(x)$ has to constant as $x\rightarrow\pm\infty$ by Liouville Theorem (section 3.1.3 in \cite{K}). Also,
    there exists some numbers $\delta>0$ and $\epsilon$ such that if $|x|>\delta$ then $||\sin{x}|/x-{1}/{x}|<\epsilon$, and $\lim\limits_{x\rightarrow-\infty}(|\sin{x}|/x)/({1}/{x})=\lim\limits_{x\rightarrow+\infty}(|\sin{x}|/x)/({1}/{x})=\pm1$.
    This makes the function $g_1(x)=-{1}/{x}$ an envelop of $g(x)$ away from $x=0$ if $\sin{x}<0$ and $g_2(x)={1}/{x}$ an envelop of $f(x)$ away from $x=0$ if $\sin{x}>0$. Moreover, on one hand, $g_2^\prime\le G^{\prime\prime}\le g_1^\prime$ if $x<-\delta$, and and $g_1^\prime$ and $g_2^\prime$ do not change signs. While on another hand, $g_1^\prime\le G^{\prime\prime}\le g_2^\prime$ if $x>\delta$, and also $g_1^\prime$ and $g_2^\prime$ do not change signs. Therefore there exists some number $\theta>0$ such $G(x)$ oscillates about $\theta$ if $x>\delta$ and $G(x)$ oscillates about $-\theta$ if $x<-\delta$. And $|G(x)|\le\theta$ if $|x|\le\delta$.
\end{enumerate}
\end{proof}

\begin{example}
For instance, if $\lambda=1$, then
\begin{equation}
\int \frac{\sin{x}}{x} dx={x}\hspace{.075cm} _{1}F_{2}\left(\frac{1}{2};\frac{3}{2},\frac{3}{2};-\frac{x^2}{4}\right)+C.
\label{ex1-1}
\end{equation}
\label{ex1}
\end{example}
According to (\ref{ex1}), the antiderivative of $g(x)= \frac{\sin{x}}{x}$ is $G(x)={x}\hspace{.075cm} _{1}F_{2}\left(\frac{1}{2};\frac{3}{2},\frac{3}{2};-\frac{x^2}{4}\right)$, and the graph of $G(x)$ is in Figure \ref{fig1}. It is in agreement with Lemma \ref{Lm1}. It is seen in Figure \ref{fig1} that $(0,G(0))=(0,0)$ is an inflection point and that $G$ attains some constants as $x\rightarrow\pm\infty$ as predicted by Lemma \ref{Lm1}.
\begin{figure}[!h]
\centerline{\epsfig{file=F12.eps, scale=0.5}}
\begin{picture}(5,5)
\put (60,98){$G(x)$}
\put (180,8){$x$}
\put (25,45){$-\theta=-\frac{{\pi}}{2}$}
\put (270,150){$\theta=\frac{{\pi}}{2}$}
\end{picture}
\caption{The antiderivative of the function $g(x)=\frac{\sin{x}}{x}$ given by (\ref{ex1-1}).}
\label{fig1}
\normalsize
\end{figure}


\begin{lemma} Consider $G(x)$ in Proposition \ref{prp1}, and preferably assume that $\lambda>0$.
\begin{enumerate}
\item Then,
\begin{equation}
G(-\infty)=\lim_{x\rightarrow-\infty}G(x)=\lim_{x\rightarrow\infty}x\hspace{.1cm}{}_1F_1\left(\frac{1}{2};\frac{3}{2};-\frac{\lambda^2 x^2}{4}\right)=-\frac{{\pi}}{2\lambda},
\label{Lm2-1}
\end{equation}
and
\begin{equation}
G(+\infty)=\lim_{x\rightarrow+\infty}G(x)=\lim_{x\rightarrow\infty}x\hspace{.1cm}{}_1F_1\left(\frac{1}{2};\frac{3}{2};-\frac{\lambda^2 x^2}{4}\right)=\frac{{\pi}}{2\lambda}.
\label{Lm2-2}
\end{equation}
\item And by the FTC,
\begin{eqnarray}
\int\limits_{-\infty}^{\infty} \frac{\sin{(\lambda x)
}}{\lambda x} dx=G(+\infty)-G(-\infty)=\frac{{\pi}}{2\lambda}-\left(-\frac{\sqrt{\pi}}{2\lambda}\right)=\frac{{\pi}}{\lambda}.
\label{Lm2-3}
\end{eqnarray}
\end{enumerate}
\label{Lm2}
\end{lemma}


\begin{proof}
\begin{enumerate}
\item To prove (\ref{Lm2-1}) and (\ref{Lm2-2}), we use the asymptotic formula for the hypergeometric function ${}_1F_2$ which is valid for $|z|\gg 1$. It can be derived using formulas 16.11.1, 16.11.2 and 16.11.8 in \cite{ND} and is given by
\begin{eqnarray}
{}_1F_2\left(a_1;b_1,b_2;-z\right)=\Gamma(b_1)\Gamma(b_2)z^{-a_1}\hspace{6.5cm}\nonumber\\\times\left\{\sum\limits_{n=0}^{R-1}\frac{(a_1)_n }{\Gamma(b_1-a_1-n)\Gamma(b_2-a_1-n)}\frac{(-z)^{-n}}{n!}+O(|z|^{-R})\right\}\nonumber\\\frac{\Gamma(b_1)\Gamma(b_2)}{\Gamma(a_1)}
+\frac{e^{2z^{\frac{1}{2}}e^{-i\frac{\pi}{2}}}(ze^{-i\pi})^{\frac{a_1-b_1-b_2+\frac{1}{2}}{2}}}{\sqrt{\pi}}\left\{\sum\limits_{n=0}^{S-1}\frac{\mu_n}{2^{n+1}}(ze^{-i\pi})^{-n}+O(|z|^{-S})\right\}
\nonumber\\+\frac{\Gamma(b_1)\Gamma(b_2)}{\Gamma(a_1)}\frac{e^{2z^{\frac{1}{2}}e^{i\frac{\pi}{2}}}(ze^{i\pi})^{\frac{a_1-b_1-b_2+\frac{1}{2}}{2}}}{\sqrt{\pi}}\left\{\sum\limits_{n=0}^{S-1}\frac{\mu_n}{2^{n+1}}(ze^{i\pi})^{-n}+O(|z|^{-S})\right\},\hspace{.12cm}
\label{asy1-2}
\end{eqnarray}
where $a_1$, $b_1$ and $b_2$ are constants and the coefficient $\mu_n$ is given by formula 16.11.4 in \cite{ND}.

We then set $z=\frac{\lambda^2 x^2}{4}, a_1=\frac{1}{2}$, $b_1=\frac{3}{2}$ and $b_2=\frac{3}{2}$, and obtain
\begin{eqnarray}
{}_1F_2\left(\frac{1}{2};\frac{3}{2},\frac{3}{2};-\frac{\lambda^2 x^2}{4}\right)=\frac{\pi}{2}\left(\lambda^2 x^2\right)^{-
\frac{1}{2}}\left\{\sum\limits_{n=0}^{R-1}\frac{\left(\frac{1}{2}\right)_n }{n!}\left(i\frac{\lambda x}{2}\right)^{-2n}+O\left(\left|\frac{\lambda x}{2}\right|^{-2R}\right)\right\}\nonumber\\
-\frac{\sqrt{\pi}}{\lambda^2 x^2}\frac{e^{ -i{\lambda x}}}{2}\left\{\sum\limits_{n=0}^{S-1}\frac{\mu_n}{2^n}\left(-i\frac{\lambda x}{2}\right)^{-2n}+O\left(\left|\frac{\lambda x}{2}\right|^{-2S}\right)\right\}
\nonumber\\-\frac{\sqrt{\pi}}{\lambda^2 x^2}\frac{e^{ i{\lambda x}}}{2}\left\{\sum\limits_{n=0}^{S-1}\frac{\mu_n}{2^n}\left(i\frac{\lambda x}{2}\right)^{-2n}+O\left(\left|\frac{\lambda x}{2}\right|^{-2S}\right)\right\},
\label{asy1-2-1}
\end{eqnarray}
Then, for $|x|\gg1$,
\begin{equation}
\frac{\pi}{2}\left(\lambda^2 x^2\right)^{-
\frac{1}{2}}\left\{\sum\limits_{n=0}^{R-1}\frac{\left(\frac{1}{2}\right)_n }{n!}\left(i\frac{\lambda x}{2}\right)^{-2n}+O\left(\left|\frac{\lambda x}{2}\right|^{-2R}\right)\right\}\sim \frac{\pi}{2\lambda |x|},
\label{asy1-2-2}
\end{equation}
while
\begin{multline}
-\frac{\sqrt{\pi}}{\lambda^2 x^2}\frac{e^{ i{\lambda x}}}{2}\left\{\sum\limits_{n=0}^{S-1}\frac{\mu_n}{2^n}\left(-i\frac{\lambda x}{2}\right)^{-2n}+O\left(\left|\frac{\lambda x}{2}\right|^{-2S}\right)\right\}
\nonumber\\-\left\{\sum\limits_{n=0}^{S-1}\frac{\mu_n}{2^n}\left(i\frac{\lambda x}{2}\right)^{-2n}+O\left(\left|\frac{\lambda x}{2}\right|^{-2S}\right)\right\}\nonumber\\%\hspace{2cm}\\
\sim \frac{\sqrt{\pi}}{\left({\lambda x}\right)^2}\frac{e^{i{\lambda x}}+e^{ -i{\lambda x}}}{2}=
{\sqrt{\pi}}\frac{\cos{\left({\lambda x}\right)}}{\left({\lambda x}\right)^2}.\hspace{3.6cm}
\label{asy1-2-3}
\end{multline}
We then obtain,
\begin{equation}
x{}_1F_2\left(\frac{1}{2};\frac{3}{2},\frac{3}{2};-\frac{\lambda^2 x^2}{4}\right)\sim\frac{\pi}{2\lambda}\frac{x}{|x|}-\frac{\sqrt{\pi}}{\lambda}\frac{\cos{\left({\lambda x}\right)}}{\lambda x}, \hspace{.12cm} |x|\rightarrow\infty.
\end{equation}
Hence,
\begin{align}
G(-\infty)&=\lim_{x\rightarrow-\infty}x{}_1F_2\left(\frac{1}{2};\frac{3}{2},\frac{3}{2};-\frac{\lambda^2 x^2}{4}\right)
\nonumber\\&=\lim_{x\rightarrow-\infty}\left(\frac{\pi}{2\lambda}\frac{x}{|x|}-\frac{\sqrt{\pi}}{\lambda}\frac{\cos{\left({\lambda x}\right)}}{\lambda x}\right)=-\frac{\pi}{2\lambda}
\end{align}
and
\begin{align}
G(+\infty)&=\lim_{x\rightarrow+\infty}x{}_1F_2\left(\frac{1}{2};\frac{3}{2},\frac{3}{2};-\frac{\lambda^2 x^2}{4}\right)
\nonumber\\&=\lim_{x\rightarrow+\infty}\left(\frac{\pi}{2\lambda}\frac{x}{|x|}-\frac{\sqrt{\pi}}{\lambda}\frac{\cos{\left({\lambda x}\right)}}{\lambda x}\right)=\frac{\pi}{2\lambda}.
\end{align}
\item By the FTC,
\begin{align}
\int\limits_{-\infty}^{+\infty} \frac{\sin{(\lambda x)
}}{\lambda x} dx &=\lim_{y\rightarrow-\infty}\int\limits_{y}^{0} \frac{\sin{(\lambda x)
}}{\lambda x} dx
+\lim_{y\rightarrow+\infty}\int\limits_{0}^{y} \frac{\sin{(\lambda x)
}}{\lambda x}dx
\nonumber\\ &=\lim_{y\rightarrow+\infty}y\hspace{.1cm}{}_1F_2\left(\frac{1}{2};\frac{3}{2},\frac{3}{2};-\frac{\lambda^2 y^2}{4}\right)
-\lim_{y\rightarrow-\infty}y\hspace{.1cm}{}_1F_2\left(\frac{1}{2};\frac{3}{2},\frac{3}{2};-\frac{\lambda^2 y^2}{4}\right)
\nonumber\\ &=G(+\infty)-G(-\infty)=\frac{\pi}{2\lambda}-\left(-\frac{\pi}{2\lambda}\right)=\frac{\pi}{\lambda}.
\end{align}
\end{enumerate}
We now verify wether Lemma this is correct or not. We first observe that
\begin{equation}
\int\limits_{-\infty}^{+\infty} \frac{\sin{(\lambda x)
}}{\lambda x} dx=2\int\limits_{0}^{+\infty} \frac{\sin{(\lambda x),
}}{\lambda x} dx
\label{v}
\end{equation}
since the integrand is an even function. And write (\ref{v}) as a double integral.
\begin{equation}
\int\limits_{0}^{+\infty} \frac{\sin{(\lambda x)
}}{\lambda x} dx=\frac{1}{\lambda}\int\limits_{0}^{+\infty}\int\limits_{0}^{+\infty}e^{-s x} {\sin{(\lambda x)}} ds dx.
\label{v1}
\end{equation}
Then by Fubini's theorem \cite{B},
\begin{equation}
\int\limits_{0}^{+\infty}\int\limits_{0}^{+\infty}e^{-s x} {\sin{(\lambda x)
}}  ds dx=\int\limits_{0}^{+\infty}\int\limits_{0}^{+\infty}e^{-s x} {\sin{(\lambda x)}}  dx ds.
\label{v2}
\end{equation}
Now using the fact that the inside integral in (\ref{v2}) is the Laplace transform of $\sin{(\lambda x)}$ \cite{AS} yields
\begin{align}
\int\limits_{0}^{+\infty}\int\limits_{0}^{+\infty}e^{-s x} {\sin{(\lambda x)
}}ds dx&=\int\limits_{0}^{+\infty}\int\limits_{0}^{+\infty}e^{-s x} {\sin{(\lambda x)}}  dx ds \nonumber \\
& =\int\limits_{0}^{+\infty}\frac{\lambda}{s^2+\lambda^2}d s=\arctan{(+\infty)}-\arctan{0}=\frac{\pi}{2}.
\label{v3}
\end{align}
Therefore,
\begin{equation}
\int\limits_{0}^{+\infty} \frac{\sin{(\lambda x)
}}{\lambda x} dx=\frac{1}{\lambda}\int\limits_{0}^{+\infty}\int\limits_{0}^{+\infty}e^{-s x} {\sin{(\lambda x)}} ds dx=\frac{\pi}{2\lambda}.
\end{equation}
Hence, \begin{equation}
\int\limits_{-\infty}^{+\infty} \frac{\sin{(\lambda x)
}}{\lambda x} dx=2\int\limits_{0}^{+\infty} \frac{\sin{(\lambda x)
}}{\lambda x} dx=2\frac{\pi}{2\lambda}=\frac{\pi}{
\lambda}.
\end{equation}
as before. This completes the proof.
\label{lm1p}
\end{proof}

If $\lambda=1$ as in Lemma \ref{Lm1}, Lemma \ref{Lm2} gives $\lim\limits_{x\rightarrow-\infty}G(x)=-\theta=-\pi/2$ while $\lim\limits_{x\rightarrow+\infty}G(x)=\theta=\pi/2$. And these are the exact values of $G(x)$ as $x\rightarrow\pm \infty$ in Figure \ref{fig1}.

\begin{theorem}
If $\beta\ge1$ and $\alpha\ge1$, then the function
$$G(x)=\frac{x^{\beta-\alpha+1}}{\beta-\alpha+1}\hspace{.075cm} _{1}F_{2}\left(-\frac{\alpha}{2\beta}+\frac{1}{2\beta}+\frac{1}{2};-\frac{\alpha}{2\beta}+\frac{1}{2\beta}+\frac{3}{2},\frac{3}{2};-\frac{\lambda^2 x^{2\beta}}{4}\right),$$
where ${}_1F_2$ is a hypergeometric function \cite{AS} and $\lambda$ is an arbitrarily constant, is the antiderivative of the function $g(x)=\frac{\sin{(\lambda x^\beta)
}}{\lambda x^\alpha}$. Thus,
\begin{align}
\text{Si}_{\beta,\alpha}&=\int\frac{\sin{(\lambda x^\beta)}}{\lambda x^\alpha}dx\nonumber\\& =\frac{x^{\beta-\alpha+1}}{\beta-\alpha+1}\hspace{.075cm} _{1}F_{2}\left(-\frac{\alpha}{2\beta}+\frac{1}{2\beta}+\frac{1}{2};-\frac{\alpha}{2\beta}+\frac{1}{2\beta}+\frac{3}{2},\frac{3}{2};-\frac{\lambda^2 x^{2\beta}}{4}\right)+C.
\label{Thm1-1}
\end{align}
And if $\beta=\alpha$ and $\alpha\ge1$, then
\begin{equation}
\int\frac{\sin{(\lambda x^\alpha)}}{\lambda x^\alpha}dx={x}\hspace{.075cm} _{1}F_{2}\left(\frac{1}{2\alpha};\frac{1}{2\alpha}+1,\frac{3}{2};-\frac{\lambda^2 x^{2\alpha}}{4}\right)+C.
\label{Thm1-2}
\end{equation}
If $\beta\ge1$ and $\alpha\ge1$, then for $|x|\gg1$,
\begin{multline}
\frac{x^{\beta-\alpha+1}}{\beta-\alpha+1}\hspace{.075cm}_{1}F_{2}\left(-\frac{\alpha}{2\beta}+\frac{1}{2\beta}+\frac{1}{2};-\frac{\alpha}{2\beta}+\frac{1}{2\beta}+\frac{3}{2},\frac{3}{2};-\frac{\lambda^2 x^{2\beta}}{4}\right)\\\sim\frac{\left(\frac{2}{\lambda}\right)^{1+\frac{1}{\beta}-\frac{\alpha}{\beta}}}{\beta-\alpha+1}\frac{\Gamma\left(-\frac{\alpha}{2\beta}+\frac{1}{2\beta}+\frac{3}{2}\right)}
{\Gamma\left(1+\frac{\alpha}{2\beta}-\frac{1}{2\beta}\right)}\frac{\sqrt{\pi}}{2}
\frac{x^{\beta-\alpha+1}}{|x|^{\beta-\alpha+1}}
-\frac{\beta-\alpha+1}{\beta}\frac{\sqrt{\pi}}{\lambda^2}\frac{\cos{\left({\lambda x^\beta}\right)}}{x^{\beta+\alpha-1}}. 
\label{Thm1-3}
\end{multline}
And if $\beta=\alpha$ and $\alpha\ge1$, then for $|x|\gg1$,
\begin{equation}
x{}_1F_2\left(\frac{1}{2\alpha};\frac{1}{2\alpha}+1,\frac{3}{2};-\frac{\lambda^2 x^{2\alpha}}{4}\right)\sim\left(\frac{2}{\lambda}\right)^{\frac{1}{\alpha}}\frac{\Gamma\left(\frac{1}{2\alpha}+1\right)}{\Gamma\left(\frac{3}{2}-\frac{1}{2\alpha}\right)}\frac{\sqrt{\pi}}{2}
\frac{x}{|x|}
-\frac{\sqrt{\pi}}{\alpha\lambda^2}\frac{\cos{\left({\lambda x^\alpha}\right)}}{x^{2\alpha-1}}. 
\label{Thm1-4}
\end{equation}
\label{Thm1}
\end{theorem}

\begin{proof}
\begin{align}
\text{Si}_{\beta,\alpha}&=\int g(x) dx=\int \frac{\sin{(\lambda x^\beta)}}{\lambda x^\alpha}dx
\nonumber\\&=\int\frac{1}{\lambda x^\alpha}\sum\limits_{n=0}^{\infty}(-1)^n\frac{(\lambda x^\beta)^{2n+1}}{(2n+1)!}dx
\nonumber\\ &=\sum\limits_{n=0}^{\infty}(-1)^n\frac{\lambda^{2n}}{(2n+1)!}\int{x^{2\beta n+\beta-\alpha}}dx
\nonumber\\&=\lambda\sum\limits_{n=0}^{\infty}(-1)^n\frac{\lambda^{2n}}{(2n+1)!}\frac{x^{2\beta n+\beta-\alpha+1}}{2\beta n+\beta-\alpha+1}+C
\nonumber\\&=\frac{x^{\beta-\alpha+1}}{2\beta}\sum\limits_{n=0}^{\infty}(-1)^n\frac{\lambda^{2n}}{(2n+1)!}\frac{x^{2\beta n}}{n-\frac{\alpha}{2\beta}+\frac{1}{2\beta}+\frac{1}{2}}+C
\nonumber\\&=\frac{x^{\beta-\alpha+1}}{2\beta}\sum\limits_{n=0}^{\infty}\frac{\Gamma\left(n-\frac{\alpha}{2\beta}+\frac{1}{2\beta}+\frac{1}{2}\right)}{\Gamma(2n+2)\Gamma\left(n-\frac{\alpha}{2\beta}+\frac{1}{2\beta}+\frac{3}{2}\right)}(-\lambda x^{2\beta})^{n}+C
\nonumber\\&=\frac{x^{\beta-\alpha+1}}{\beta-\alpha+1}\sum\limits_{n=0}^{\infty}\frac{\left(-\frac{\alpha}{2\beta}+\frac{1}{2\beta}+\frac{1}{2}\right)_n}
{\left(\frac{3}{2}\right)_n\left(-\frac{\alpha}{2\beta}+\frac{1}{2\beta}+\frac{3}{2}\right)_n}\frac{\left(-\frac{\lambda^2 x^{2\beta}}{4}\right)^{n}}{n!}+C
\nonumber\\&=\frac{x^{\beta-\alpha+1}}{\beta-\alpha+1}\hspace{.075cm} _{1}F_{2}\left(-\frac{\alpha}{2\beta}+\frac{1}{2\beta}+\frac{1}{2};-\frac{\alpha}{2\beta}+\frac{1}{2\beta}+\frac{3}{2},\frac{3}{2};-\frac{\lambda^2 x^{2\beta}}{4}\right)+C
\nonumber\\&=G(x)+C.
\end{align}
Equation (\ref{Thm1-2}) is proved by setting $\beta=\alpha$ in (\ref{Thm1-1}).
To prove (\ref{Thm1-3}), we use the asymptotic formula for the hypergeometric function $_{1}F_{2}$, given by (\ref{asy1-2}), and proceed as in Lemma \ref{Lm2}. And (\ref{Thm1-4}) is proved by setting $\beta=\alpha$ in (\ref{Thm1-3}).
\end{proof}

One can show as above that if $\beta\ge1$ and $\alpha\ge1$, then
\begin{align}
\int \frac{\sinh{(\lambda x^\beta)}}{\lambda x^\alpha}dx
=\frac{x^{\beta-\alpha+1}}{\beta-\alpha+1}\hspace{.075cm} _{1}F_{2}\left(-\frac{\alpha}{2\beta}+\frac{1}{2\beta}+\frac{1}{2};-\frac{\alpha}{2\beta}+\frac{1}{2\beta}+\frac{3}{2},\frac{3}{2};\frac{\lambda^2 x^{2\beta}}{4}\right)+C.
\label{sinh1}
\end{align}

\begin{corollary}
Let $\beta=\alpha$. If $\alpha\ge1$, then
\begin{equation}
\int\limits_{-\infty}^{0} \frac{\sin{(\lambda x^\alpha)
}}{\lambda x^\alpha}dx=G(0)-G(-\infty)=\left(\frac{2}{\lambda}\right)^{\frac{1}{\alpha}}\frac{\Gamma\left(\frac{1}{2\alpha}+1\right)}{\Gamma\left(\frac{3}{2}-\frac{1}{2\alpha}\right)}\frac{\sqrt{\pi}}{2},
\label{Crl1-1}
\end{equation}
\begin{equation}
\int\limits_{0}^{+\infty} \frac{\sin{(\lambda x^\alpha)
}}{\lambda x^\alpha}dx=G(+\infty)-G(0)=\left(\frac{2}{\lambda}\right)^{\frac{1}{\alpha}}\frac{\Gamma\left(\frac{1}{2\alpha}+1\right)}{\Gamma\left(\frac{3}{2}-\frac{1}{2\alpha}\right)}\frac{\sqrt{\pi}}{2}
\label{Crl1-2}
\end{equation}
and
\begin{equation}
\int\limits_{-\infty}^{+\infty} \frac{\sin{(\lambda x^\alpha)
}}{\lambda x^\alpha}dx=G(+\infty)-G(-\infty)=\left(\frac{2}{\lambda}\right)^{\frac{1}{\alpha}}\frac{\Gamma\left(\frac{1}{2\alpha}+1\right)}{\Gamma\left(\frac{3}{2}-\frac{1}{2\alpha}\right)}\sqrt{\pi}.
\label{Crl1-3}
\end{equation}
\label{Crl1}
\end{corollary}
\begin{proof}
If $\beta=\alpha$, Theorem \ref{Thm1} gives
\begin{align}
G(-\infty)&=\lim_{x\rightarrow-\infty}x{}_1F_2\left(\frac{1}{2\alpha};\frac{1}{2\alpha}+1,\frac{3}{2};-\frac{\lambda^2 x^{2\alpha}}{4}\right)
\nonumber\\&=\lim_{x\rightarrow-\infty}\left(\left(\frac{2}{\lambda}\right)^{\frac{1}{\alpha}}\frac{\sqrt{\pi}}{2}\frac{\Gamma\left(\frac{1}{2\alpha}+1\right)}{\Gamma\left(\frac{3}{2}-\frac{1}{2\alpha}\right)}\frac{x}{|x|}
-\frac{\sqrt{\pi}}{\alpha\lambda^2}\frac{\cos{\left({\lambda x^\alpha}\right)}}{x^{2\alpha-1}}\right)
\nonumber\\&=-\left(\frac{2}{\lambda}\right)^{\frac{1}{\alpha}}\frac{\Gamma\left(\frac{1}{2\alpha}+1\right)}{\Gamma\left(\frac{3}{2}-\frac{1}{2\alpha}\right)}\frac{\sqrt{\pi}}{2}
\end{align}
and
\begin{align}
G(+\infty)&=\lim_{x\rightarrow+\infty}x{}_1F_2\left(\frac{1}{2\alpha};\frac{1}{2\alpha}+1,\frac{3}{2};-\frac{\lambda^2 x^{2\alpha}}{4}\right)
\nonumber\\&=\lim_{x\rightarrow+\infty}\left(\left(\frac{2}{\lambda}\right)^{\frac{1}{\alpha}}\frac{\sqrt{\pi}}{2}\frac{\Gamma\left(\frac{1}{2\alpha}+1\right)}{\Gamma\left(\frac{3}{2}-\frac{1}{2\alpha}\right)}\frac{x}{|x|}
-\frac{\sqrt{\pi}}{\alpha\lambda^2}\frac{\cos{\left({\lambda x^\alpha}\right)}}{x^{2\alpha-1}}\right)
\nonumber\\&=\left(\frac{2}{\lambda}\right)^{\frac{1}{\alpha}}\frac{\Gamma\left(\frac{1}{2\alpha}+1\right)}{\Gamma\left(\frac{3}{2}-\frac{1}{2\alpha}\right)}\frac{\sqrt{\pi}}{2}.
\end{align}
Hence, by the FTC,
\begin{multline}
\int\limits_{-\infty}^{0} \frac{\sin{(\lambda x^\alpha)
}}{\lambda x^\alpha}dx=G(0)-G(-\infty)\\=0-\left(-\left(\frac{2}{\lambda}\right)^{\frac{1}{\alpha}}\frac{\Gamma\left(\frac{1}{2\alpha}+1\right)}{\Gamma\left(\frac{3}{2}-\frac{1}{2\alpha}\right)}\frac{\sqrt{\pi}}{2}\right)
=\left(\frac{2}{\lambda}\right)^{\frac{1}{\alpha}}\frac{\Gamma\left(\frac{1}{2\alpha}+1\right)}{\Gamma\left(\frac{3}{2}-\frac{1}{2\alpha}\right)}\frac{\sqrt{\pi}}{2},
\end{multline}
\begin{multline}
\int\limits_{0}^{+\infty} \frac{\sin{(\lambda x^\alpha)
}}{\lambda x^\alpha}dx=G(+\infty)-G(0)\\=\left(\frac{2}{\lambda}\right)^{\frac{1}{\alpha}}\frac{\Gamma\left(\frac{1}{2\alpha}+1\right)}{\Gamma\left(\frac{3}{2}-\frac{1}{2\alpha}\right)}
\frac{\sqrt{\pi}}{2}-0=\left(\frac{2}{\lambda}\right)^{\frac{1}{\alpha}}\frac{\Gamma\left(\frac{1}{2\alpha}+1\right)}{\Gamma\left(\frac{3}{2}-\frac{1}{2\alpha}\right)}\frac{\sqrt{\pi}}{2}
\end{multline}
and
\begin{align}
\int\limits_{-\infty}^{+\infty} \frac{\sin{(\lambda x^\alpha)
}}{\lambda x^\alpha}dx&=G(+\infty)-G(-\infty)
\nonumber\\&=\left(\frac{2}{\lambda}\right)^{\frac{1}{\alpha}}\frac{\Gamma\left(\frac{1}{2\alpha}+1\right)}{\Gamma\left(\frac{3}{2}-\frac{1}{2\alpha}\right)}
\frac{\sqrt{\pi}}{2}-\left(-\left(\frac{2}{\lambda}\right)^{\frac{1}{\alpha}}\frac{\Gamma\left(\frac{1}{2\alpha}+1\right)}{\Gamma\left(\frac{3}{2}-\frac{1}{2\alpha}\right)}\frac{\sqrt{\pi}}{2}\right)
\nonumber\\&=\left(\frac{2}{\lambda}\right)^{\frac{1}{\alpha}}\frac{\Gamma\left(\frac{1}{2\alpha}+1\right)}{\Gamma\left(\frac{3}{2}-\frac{1}{2\alpha}\right)}\sqrt{\pi}.
\end{align}
\end{proof}

\begin{theorem}
If $\beta \ge1$ and $\alpha \ge 1$, then the FTC gives
\begin{equation}
\int\limits_{A}^{B} \frac{\sin{(\lambda x^\beta)
}}{\lambda x^\alpha}dx=G(B)-G(A),
\label{Thm2-1}
\end{equation}
for any $A$ and any $B$, and where $G$ is given in (\ref{Thm1-1}).
\label{Thm2}
\end{theorem}

\begin{proof}
Equation (\ref{Thm2-1}) holds by Theorem \ref{Thm1}, Corollary \ref{Crl1} and Lemma \ref{Lm2}. Since the FTC works for $A=-\infty$ and $B=0$ in (\ref{Crl1-1}), $A=0$ and $B=+\infty$ in (\ref{Crl1-2}) and  $A=-\infty$ and $B=+\infty$ in (\ref{Crl1-3}) by Corollary \ref{Crl1} for any $\beta=\alpha \ge1$ and by Lemma \ref{Lm2} for $\beta=\alpha=1$, then it has to work for other values of $A,B\in \mathbb{R}$ and for $\beta\ge1$ and $\alpha\ge1$ since the case with $\beta=\alpha\ge1$ is derived from the case with $\beta\ge1$ and $\alpha\ge1$.
\end{proof}

\section{Evaluation of the cosine integral and related integrals}\label{cosine-integral}
\begin{theorem}
If $\beta\ge1$, then the function $G(x)=\frac{1}{\lambda}\ln|x|-\frac{\lambda x^{2\beta}}{4\beta}\hspace{.075cm} _{2}F_{3}\left(1,1;\frac{3}{2},2,2;-\frac{\lambda^2 x^{2\beta}}{4}\right)$, where ${}_2F_3$ is a hypergeometric function \cite{AS} and $\lambda$ is an arbitrarily constant, is the antiderivative of the function $g(x)=\frac{\cos{(\lambda x^\beta)
}}{\lambda x}$. Thus,
\begin{equation}
\int\frac{\cos{(\lambda x^\beta)}}{\lambda x}dx=\frac{1}{\lambda}\ln|x|-\frac{\lambda x^{2\beta}}{4\beta}\hspace{.075cm} _{2}F_{3}\left(1,1;\frac{3}{2},2,2;-\frac{\lambda^2 x^{2\beta}}{4\beta}\right)+C.
\label{Thm3-1}
\end{equation}
\label{Thm3}
\end{theorem}

\begin{proof}
\begin{align}
\int g(x) dx&=\int \frac{\cos{(\lambda x^\beta)
}}{\lambda x}dx \nonumber\\ &=\int\frac{1}{\lambda x}\sum\limits_{n=0}^{\infty}(-1)^n\frac{(\lambda x^\beta)^{2n}}{(2n)!}dx
\nonumber\\ &=\int\frac{1}{\lambda x}dx+\frac{1}{\lambda}\int\sum\limits_{n=1}^{\infty}(-1)^n\frac{\lambda^{2n}}{(2n)!}x^{2\beta n-1}dx
\nonumber\\&=\frac{1}{\lambda}\ln|x|-\frac{1}{\lambda}\sum\limits_{n=0}^{\infty}(-1)^n\frac{\lambda^{2n+2}}{(2n+2)!}\int x^{2\beta n+2\beta-1} dx
\nonumber\\&=\frac{1}{\lambda}\ln|x|-\lambda\sum\limits_{n=0}^{\infty}(-1)^n\frac{\lambda^{2n}}{(2n+2)!}\frac{x^{2\beta n+2\beta}}{2\beta n+2\beta}+C
\nonumber\\&=\frac{1}{\lambda}\ln|x|-\frac{\lambda x^{2\beta}}{2\beta}\sum\limits_{n=0}^{\infty}
\frac{\Gamma\left(n+1\right)}{\Gamma(2n+3)\Gamma\left(n+2\right)}(-\lambda^2 x^{2\beta})^{n}+C
\nonumber\\&=\frac{1}{\lambda}\ln|x|-\frac{\lambda x^{2\beta}}{4\beta}\sum\limits_{n=0}^{\infty}\frac{(1)_n\left(1\right)_n}{\left(\frac{3}{2}\right)_n (2)_n\left(2\right)_n}\frac{\left(-\frac{\lambda^2 x^{2\beta}}{4}\right)^{n}}{n!}+C
\nonumber\\&=\frac{1}{\lambda}\ln|x|-\frac{\lambda x^{2\beta}}{4\beta}\hspace{.075cm} _{2}F_{3}\left(1,1;\frac{3}{2},2,2;-\frac{\lambda^2 x^{2\beta}}{4}\right)+C
\nonumber\\&=G(x)+C.
\end{align}
\end{proof}

\begin{theorem}
If $\beta\ge1$ and $\alpha >1$, then the function
$$G(x)=\frac{1}{\lambda}\frac{x^{1-\alpha}}{1-\alpha}-\frac{\lambda x^{2\beta-\alpha+1}}{2\beta-\alpha+1}\hspace{.075cm} _{2}F_{3}\left(1,-\frac{\alpha}{2\beta}+\frac{1}{2\beta}+1;-\frac{\alpha}{2\beta}+\frac{1}{2\beta}+2,\frac{3}{2},2;-\frac{\lambda^2 x^{2\beta}}{4}\right),$$
where ${}_2F_3$ is a hypergeometric function \cite{AS} and $\lambda$ is an arbitrarily constant, is the antiderivative of the function $g(x)=\frac{\cos{(\lambda x^\beta)
}}{\lambda x^\alpha}$. Thus,
\begin{multline}
\int\frac{\cos{(\lambda x^\beta)}}{\lambda x^\alpha}dx=\frac{1}{\lambda}\frac{x^{1-\alpha}}{1-\alpha}\\-\frac{1}{2}\frac{\lambda x^{2\beta-\alpha+1}}{2\beta-\alpha+1}\hspace{.075cm} _{2}F_{3}\left(1,-\frac{\alpha}{2\beta}+\frac{1}{2\beta}+1;-\frac{\alpha}{2\beta}+\frac{1}{2\beta}+2,\frac{3}{2},2;-\frac{\lambda^2 x^{2\beta}}{4}\right)+C.
\label{Thm4-1}
\end{multline}

And for $|x|\gg1$,
\begin{multline}
\frac{\lambda x^{2\beta-\alpha+1}}{2\beta-\alpha+1}\hspace{.075cm} _{2}F_{3}\left(1,-\frac{\alpha}{2\beta}+\frac{1}{2\beta}+1;-\frac{\alpha}{2\beta}+\frac{1}{2\beta}+2,\frac{3}{2},2;-\frac{\lambda^2 x^{2\beta}}{4}\right)\\\sim
\frac{\sqrt{\pi}\lambda}{2\beta}\Gamma\left(-\frac{\alpha}{2\beta}+\frac{1}{2\beta}+1\right)\left(\frac{2}{\lambda }\right)^{-\frac{\alpha}{\beta}+\frac{1}{\beta}+2}+\frac{\sqrt{\pi}}{\lambda\beta}x^{-\alpha+1}
+\frac{2}{\lambda^2\beta}\frac{\cos(\lambda x^{\beta})}{x^{\beta+\alpha-1}}.\\
\label{Thm4-2}
\end{multline}
\label{Thm4}
\end{theorem}

\begin{proof}
If $\beta\ge1$ and $\alpha>1$,
\begin{align}
\int g(x) dx&=\int \frac{\cos{(\lambda x^\beta)
}}{\lambda x^\alpha}dx \nonumber\\ &=\int\frac{1}{\lambda x^\alpha}\sum\limits_{n=0}^{\infty}(-1)^n\frac{(\lambda x^\beta)^{2n}}{(2n)!}dx
\nonumber\\ &=\int\frac{1}{\lambda x^\alpha}dx+\frac{1}{\lambda}\int\sum\limits_{n=1}^{\infty}(-1)^n\frac{\lambda^{2n}}{(2n)!}x^{2\beta n-\alpha}dx
\nonumber\\&=\frac{1}{\lambda}\frac{x^{1-\alpha}}{1-\alpha}-\frac{1}{\lambda}\sum\limits_{n=0}^{\infty}(-1)^n\frac{\lambda^{2n+2}}{(2n+2)!}\int x^{2\beta n+2\beta-\alpha} dx
\nonumber\\&=\frac{1}{\lambda}\frac{x^{1-\alpha}}{1-\alpha}-\lambda\sum\limits_{n=0}^{\infty}(-1)^n\frac{\lambda^{2n}}{(2n+2)!}\frac{x^{2\beta n+2\beta-\alpha+1}}{2\beta n+2\beta-\alpha+1}+C
\nonumber\\&=\frac{1}{\lambda}\frac{x^{1-\alpha}}{1-\alpha}-\frac{\lambda x^{2\beta-\alpha+1}}{2\beta}\sum\limits_{n=0}^{\infty}
\frac{\Gamma\left(n-\frac{\alpha}{2\beta}+\frac{1}{2\beta}+1\right)}{\Gamma(2n+3)\Gamma\left(n-\frac{\alpha}{2\beta}+\frac{1}{2\beta}+2\right)}(-\lambda^2 x^{2\beta})^{n}+C
\nonumber\\&=\frac{1}{\lambda}\frac{x^{1-\alpha}}{1-\alpha}-\frac{1}{2}\frac{\lambda x^{2\beta-\alpha+1}}{2\beta-\alpha+1}-\sum\limits_{n=0}^{\infty}\frac{(1)_n\left(-\frac{\alpha}{2\beta}+\frac{1}{2\beta}+1\right)_n}{\left(\frac{3}{2}\right)_n (2)_n\left(-\frac{\alpha}{2\beta}+\frac{1}{2\beta}+2\right)_n}\frac{\left(-\frac{\lambda^2 x^{2\beta}}{4}\right)^{n}}{n!}+C
\nonumber\\&=\frac{1}{\lambda}\frac{x^{1-\alpha}}{1-\alpha}-\frac{1}{2}\frac{\lambda x^{2\beta-\alpha+1}}{2\beta-\alpha+1}\hspace{.075cm} _{2}F_{3}\left(1,-\frac{\alpha}{2\beta}+\frac{1}{2\beta}+1;-\frac{\alpha}{2\beta}+\frac{1}{2\beta}+2,\frac{3}{2},2;-\frac{\lambda^2 x^{2\beta}}{4}\right)+C
\nonumber\\&=G(x)+C.
\end{align}
To prove (\ref{Thm4-2}), we use the asymptotic expression of the hypergeometric function ${}_2F_3\left(a_1,a_2;b_1,b_2,b_3;-z\right)$ for $|z|\gg 1$, where $a_1,a_2,b_1,b_2$ and $b_3$ are constants . It can be obtained using formulas 16.11.1, 16.11.2 and 16.11.8 in \cite{ND} and is given by
\begin{eqnarray}
{}_2F_3\left(a_1,a_2;b_1,b_2,b_3;-z\right)= \frac{\Gamma(b_1)\Gamma(b_2)\Gamma(b_3)}{\Gamma(a_2)}z^{-a_1}\hspace{5cm}\nonumber\\\times \left\{\sum\limits_{n=0}^{R-1}(a_1)_n\frac{\Gamma(a_1-a_2-n) }{\Gamma(b_1-a_1-n)\Gamma(b_2-a_1-n)\Gamma(b_3-a_1-n)}\frac{(-z)^{-n}}{n!}+O(|z|^{-R})\right\}\nonumber\\
+\frac{\Gamma(b_1)\Gamma(b_2)\Gamma(b_3)}{\Gamma(a_1)}z^{-a_2}\nonumber\\\times\left\{\sum\limits_{n=0}^{R-1}(a_2)_n\frac{\Gamma(a_2-a_1-n) }{\Gamma(b_1-a_2-n)\Gamma(b_2-a_2-n)\Gamma(b_3-a_2-n)}\frac{(-z)^{-n}}{n!}+O(|z|^{-R})\right\}+\nonumber\\
+\frac{\Gamma(b_1)\Gamma(b_2)\Gamma(b_3)}{\Gamma(a_1)\Gamma(a_2)}\nonumber\\\times\frac{e^{2z^{\frac{1}{2}}e^{-i\frac{\pi}{2}}}(ze^{-i\pi})^{\frac{a_1+a_2-b_1-b_2-b_3+\frac{1}{2}}{2}}}{\sqrt{\pi}}\left\{\sum\limits_{n=0}^{S-1}\frac{\mu_n}{2^{n+1}}(ze^{-i\pi})^{-n}+O(|z|^{-S})\right\}
\nonumber\\+\frac{\Gamma(b_1)\Gamma(b_2)\Gamma(b_3)}{\Gamma(a_1)\Gamma(a_2)}\nonumber\\\times \frac{e^{2z^{\frac{1}{2}}e^{i\frac{\pi}{2}}}(ze^{i\pi})^{\frac{a_1+a_2-b_1-b_2-b_3+\frac{1}{2}}{2}}}{\sqrt{\pi}}\left\{\sum\limits_{n=0}^{S-1}\frac{\mu_n}{2^{n+1}}(ze^{i\pi})^{-n}+O(|z|^{-S})\right\},
\label{asy2-3}
\end{eqnarray}
where the coefficient $\mu_n$ is given by formula 16.11.4 in \cite{ND}.

Setting $z=\frac{\lambda^2 x^{2\beta}}{4}$, $a_1=1$, $a_2=-\frac{\alpha}{2\beta}+\frac{1}{2\beta}+1$,
$b_1=-\frac{\alpha}{2\beta}+\frac{1}{2\beta}+2$, $b_2=\frac{3}{2}$ and $b_3={2}$ in (\ref{asy2-3}) yields
\begin{align}
& _{2}F_{3}\left(1,-\frac{\alpha}{2\beta}+\frac{1}{2\beta}+1;-\frac{\alpha}{2\beta}+\frac{1}{2\beta}+2,\frac{3}{2},2;-\frac{\lambda^2 x^{2\beta}}{4}\right)\nonumber \\ &\sim \frac{\sqrt{\pi}}{\lambda^2}\left(-\frac{\alpha}{\beta}+\frac{1}{\beta}+{2}\right)\frac{1}{x^{2\beta}}
+\frac{\sqrt{\pi}}{2}\Gamma\left(-\frac{\alpha}{2\beta}+\frac{1}{2\beta}+2\right)\left(\frac{2}{\lambda x^{\beta}}\right)^{-\frac{\alpha}{\beta}+\frac{1}{\beta}+2}
\nonumber \\&+\frac{2}{\lambda^3}\left(-\frac{\alpha}{\beta}+\frac{1}{\beta}+{2}\right)\frac{\cos(\lambda x^{\beta})}{x^{3\beta}}.
\label{2F3-2}
\end{align}
This gives
\begin{multline}
\frac{\lambda x^{2\beta-\alpha+1}}{2\beta-\alpha+1}\hspace{.075cm} _{2}F_{3}\left(1,-\frac{\alpha}{2\beta}+\frac{1}{2\beta}+1;-\frac{\alpha}{2\beta}+\frac{1}{2\beta}+2,\frac{3}{2},2;-\frac{\lambda^2 x^{2\beta}}{4}\right)\\\sim
\frac{\sqrt{\pi}\lambda}{2\beta-\alpha+1}\Gamma\left(-\frac{\alpha}{2\beta}+\frac{1}{2\beta}+2\right)\left(\frac{2}{\lambda }\right)^{-\frac{\alpha}{\beta}+\frac{1}{\beta}+2}+\frac{\sqrt{\pi}}{\lambda\beta}x^{-\alpha+1}
+\frac{2}{\lambda^2\beta}\frac{\cos(\lambda x^{\beta})}{x^{\beta+\alpha-1}}\\
=\frac{\sqrt{\pi}\lambda}{2\beta}\Gamma\left(-\frac{\alpha}{2\beta}+\frac{1}{2\beta}+1\right)\left(\frac{2}{\lambda }\right)^{-\frac{\alpha}{\beta}+\frac{1}{\beta}+2}+\frac{\sqrt{\pi}}{\lambda\beta}x^{-\alpha+1}
+\frac{2}{\lambda^2\beta}\frac{\cos(\lambda x^{\beta})}{x^{\beta+\alpha-1}}\nonumber,
\end{multline}
which is exactly (\ref{Thm4-2}). This completes the proof.
\end{proof}

One can show as above that if $\beta\ge1$ and $\alpha>1$, then
\begin{multline}
\int\frac{\cosh{(\lambda x^\beta)}}{\lambda x^\alpha}dx=\frac{1}{\lambda}\frac{x^{1-\alpha}}{1-\alpha}\\+\frac{1}{2}\frac{\lambda x^{2\beta-\alpha+1}}{2\beta-\alpha+1}\hspace{.075cm} _{2}F_{3}\left(1,-\frac{\alpha}{2\beta}+\frac{1}{2\beta}+1;-\frac{\alpha}{2\beta}+\frac{1}{2\beta}+2,\frac{3}{2},2;\frac{\lambda^2 x^{2\beta}}{4}\right)+C.
\label{cosh}
\end{multline}

\section{Evaluation of some integrals involving $\text{Si}_{\alpha,\beta}$ and $\text{Ci}_{\alpha,\beta}$}\label{sine-cosine}
The integral $\int\frac{\cos^n{(\lambda x^\beta)}}{\lambda x^\alpha}dx$, where $n$ is a positive integer and $\beta\ge1,\alpha\ge1$, can be written in terms of (\ref{Thm4-1}) and then evaluated.
\begin{example}
In this example, the integral $\int\frac{\cos^4{(\lambda x^\beta)}}{\lambda x^\alpha}dx$ is evaluated by linearizing the function $\cos^4{(\lambda x^\beta)}$. This gives
\begin{align}
&\int\frac{\cos^4{(\lambda x^\beta)}}{\lambda x^\alpha}dx=\frac{1}{8}\int\frac{\cos{(4\lambda x^\beta)}}{\lambda x^\alpha}dx+\frac{1}{2}\int\frac{\cos{(2\lambda x^\beta)}}{\lambda x^\alpha}dx+\frac{3}{8}\int dx=
\nonumber\\ &\frac{1}{8\lambda}\frac{x^{1-\alpha}}{1-\alpha}-\frac{1}{4}\frac{\lambda x^{2\beta-\alpha+1}}{2\beta-\alpha+1}\hspace{.075cm} _{2}F_{3}\left(1,-\frac{\alpha}{2\beta}+\frac{1}{2\beta}+1;-\frac{\alpha}{2\beta}+\frac{1}{2\beta}+2,\frac{3}{2},2;-4\lambda^2 x^{2\beta}\right)
\nonumber\\ &+\frac{1}{2\lambda}\frac{x^{1-\alpha}}{1-\alpha}-\frac{1}{2}\frac{\lambda x^{2\beta-\alpha+1}}{2\beta-\alpha+1}\hspace{.075cm} _{2}F_{3}\left(1,-\frac{\alpha}{2\beta}+\frac{1}{2\beta}+1;-\frac{\alpha}{2\beta}+\frac{1}{2\beta}+2,\frac{3}{2},2;-\lambda^2 x^{2\beta}\right)\nonumber\\ &+\frac{3x}{8}+C
\end{align}
\label{ex2}
\end{example}

If $\beta\ge1$ and $\alpha\ge1$, the integral $\int\frac{\sin^n{(\lambda x^\beta)}}{\lambda x^\alpha}dx$, where $n$ is a positive integer, can be written either in terms of (\ref{Thm1-1}) if $n$ odd, or in terms of (\ref{Thm4-1}) if $n$ even, and then evaluated.
\begin{example}
In this example, the integral $\int\frac{\sin^3{(\lambda x^\beta)}}{\lambda x^\alpha}dx$ is evaluated by linearizing the function $\sin^3{(\lambda x^\beta)}$. This gives
\begin{align}
\int\frac{\sin^3{(\lambda x^\beta)}}{\lambda x^\alpha}dx&=-\frac{1}{4}\int\frac{\sin{(3\lambda x^\beta)}}{\lambda x^\alpha}dx+\frac{3}{4}\int\frac{\sin{(\lambda x^\beta)}}{\lambda x^\alpha}dx
\nonumber\\ &=-\frac{1}{4}\frac{x^{\beta-\alpha+1}}{\beta-\alpha+1}\hspace{.075cm} _{1}F_{2}\left(-\frac{\alpha}{2\beta}+\frac{1}{2\beta}+\frac{1}{2};-\frac{\alpha}{2\beta}+\frac{1}{2\beta}+\frac{3}{2},\frac{3}{2};-\frac{9\lambda^2 x^{2\beta}}{4}\right)
\nonumber\\ &\hspace{.4cm}+\frac{3}{4}\frac{x^{\beta-\alpha+1}}{\beta-\alpha+1}\hspace{.075cm} _{1}F_{2}\left(-\frac{\alpha}{2\beta}+\frac{1}{2\beta}+\frac{1}{2};-\frac{\alpha}{2\beta}+\frac{1}{2\beta}+\frac{3}{2},\frac{3}{2};-\frac{\lambda^2 x^{2\beta}}{4}\right)+C
\end{align}
\label{ex3}
\end{example}
\begin{example}
In this example, the integral $\int\frac{\sin^4{(\lambda x^\beta)}}{\lambda x^\alpha}dx$ is evaluated by linearizing the function $\sin^4{(\lambda x^\beta)}$. This gives
\begin{align}
&\int\frac{\sin^4{(\lambda x^\beta)}}{\lambda x^\alpha}dx=\frac{1}{8}\int\frac{\cos{(4\lambda x^\beta)}}{\lambda x^\alpha}dx-\frac{1}{2}\int\frac{\cos{(2\lambda x^\beta)}}{\lambda x^\alpha}dx-\frac{3}{8}\int dx=
\nonumber\\ &\frac{1}{8\lambda}\frac{x^{1-\alpha}}{1-\alpha}-\frac{1}{4}\frac{\lambda x^{2\beta-\alpha+1}}{2\beta-\alpha+1}\hspace{.075cm} _{2}F_{3}\left(1,-\frac{\alpha}{2\beta}+\frac{1}{2\beta}+1;-\frac{\alpha}{2\beta}+\frac{1}{2\beta}+2,\frac{3}{2},2;-4\lambda^2 x^{2\beta}\right)
\nonumber\\ &-\frac{1}{2\lambda}\frac{x^{1-\alpha}}{1-\alpha}-\frac{1}{2}\frac{\lambda x^{2\beta-\alpha+1}}{2\beta-\alpha+1}\hspace{.075cm} _{2}F_{3}\left(1,-\frac{\alpha}{2\beta}+\frac{1}{2\beta}+1;-\frac{\alpha}{2\beta}+\frac{1}{2\beta}+2,\frac{3}{2},2;-\lambda^2 x^{2\beta}\right)\nonumber\\ &-\frac{3x}{8}+C
\end{align}
\label{ex4}
\end{example}


The integrals $\int\sin{\left(\frac{\lambda} {x^\mu}\right)}dx$ and $\int\cos{\left(\frac{\lambda} {x^\mu}\right)}dx$, where the constant $\mu\ge1$ can be evaluated by making the substitution $u=1/x$.
\begin{example}
\begin{enumerate}
\item Making the substitution $u=1/x$ and applying Theorem \ref{Thm1} gives
\begin{align}
&\int\sin{\left(\frac{\lambda} {x^\mu}\right)}dx=-\int\frac{\sin{\left(\lambda u^\mu\right)}}{u^2}dx
\nonumber\\ &=-\frac{\lambda u^{\mu-1}}{\mu-1}\hspace{.075cm} _{1}F_{2}\left(-\frac{1}{2\mu}+\frac{1}{2};-\frac{1}{2\mu}+\frac{3}{2},\frac{3}{2};-\frac{\lambda^2 u^{2\mu}}{4}\right)
\nonumber\\ &=-\frac{\lambda\left(\frac{1}{x}\right)^{\mu-1}}{\mu-1}\hspace{.075cm} _{1}F_{2}\left(-\frac{1}{2\mu}+\frac{1}{2};-\frac{1}{2\mu}+\frac{3}{2},\frac{3}{2};-\frac{\lambda^2 }{4 x^{2\mu}}\right)+C
\end{align}
\item Making the substitution $u=1/x$ and applying Theorem \ref{Thm4} gives
\begin{align}
&\int\cos{\left(\frac{\lambda} {x^\mu}\right)}dx=-\int\frac{\cos{\left(\lambda u^\mu\right)}}{u^2}dx
\nonumber\\ &=\frac{1}{u}+\frac{\lambda u^{2\mu-1}}{2\mu-1}\hspace{.075cm} _{2}F_{3}\left(1,-\frac{1}{2\mu}+1;-\frac{1}{2\mu}+2,\frac{3}{2},2;-\frac{\lambda^2 u^{2\mu}}{4}\right)
\nonumber\\ &=x+\frac{\lambda\left(\frac{1}{x}\right)^{2\mu-1}}{2\mu-1}\hspace{.075cm} _{2}F_{3}\left(1,-\frac{1}{2\mu}+1;-\frac{1}{2\mu}+2,\frac{3}{2},2;-\frac{\lambda^2 }{4 x^{2\mu}}\right)+C
\end{align}
\end{enumerate}
\label{ex4}
\end{example}
\section{Evaluation of exponential (Ei) and logarithmic (Li) integrals}\label{log-integral}
\begin{proposition}
If $\beta\ge1$, then for any constant $\lambda$,
\begin{equation}
\int \frac{e^{\lambda x^\beta}}{x}dx=\ln{|x|}+\frac{\lambda x^\beta}{\beta}\hspace{.075cm}   _{2}F_{2}(1,1;2,2;\lambda x^\beta)+C,
\label{prp2-1}
\end{equation}
and
\begin{equation}
\lambda x\hspace{.075cm}   _{2}F_{2}(1,1;2,2;\lambda x)\sim-2+\frac{e^{\lambda x^\beta}}{\lambda x^\beta}, \hspace{.12cm} |x|\gg1.
\label{prp2-2}
\end{equation}
\label{prp2}
\end{proposition}

\begin{proof}
\begin{align}
\int \frac{e^{\lambda x^\beta}}{x}dx&=\int\frac{1}{x}\sum\limits_{n=0}^{\infty}\frac{(\lambda x^\beta)^n}{n!}dx \nonumber\\&=\int\frac{dx}{x}+\int\sum\limits_{n=1}^{\infty}\frac{\lambda^n x^{\beta n-1}}{n!}dx
\nonumber\\&=\ln{|x|}+\sum\limits_{n=1}^{\infty}\frac{\lambda^n}{n!}\int x^{\beta n-1} dx
\nonumber\\&=\ln{|x|}+\sum\limits_{n=1}^{\infty}\frac{\lambda^n}{n!}\frac{x^{\beta n}}{\beta n}
\nonumber\\&=\ln{|x|}+\sum\limits_{n=0}^{\infty}\frac{\lambda^{n+1}}{(n+1)!}\frac{x^{\beta n+\beta}}{\beta n+\beta}
\nonumber\\&=\ln{|x|}+\frac{\lambda x^\beta}{\beta}\sum\limits_{n=0}^{\infty}\frac{\Gamma(n+1)}{\Gamma(n+2)\Gamma(n+2)}(\lambda x^\beta)^{n}+C
\nonumber\\&=\ln{|x|}+\frac{\lambda x^\beta}{\beta}\sum\limits_{n=0}^{\infty}\frac{(1)_n (1)_n}{(2)_n (2)_n}\frac{(\lambda x^\beta)^{n}}{n!}+C
\nonumber\\&=\ln{|x|}+\frac{\lambda x^\beta}{\beta}\hspace{.075cm}   _{2}F_{2}(1,1;2,2;\lambda x^\beta)+C.
\label{e-i-pro}
\end{align}

To derive the asymptotic expression of $\lambda x^\beta \hspace{.075cm} _{2}F_{2}(1,1;2,2;\lambda x^\beta)$, $|x|\gg 1$, we use the asymptotic expression of the hypergeometric function ${}_2F_2\left(a_1,a_2;b_1,b_2;z\right)$ for $|z|\gg 1$, where $z\in \mathbb{C}$, and $a_1,a_2,b_1$ and $b_2$ are constants. It can be obtained using formulas 16.11.1, 16.11.2 and 16.11.7 in \cite{ND} and is given by
\begin{eqnarray}
\hspace{-.5cm}{}_2F_2\left(a_1,a_2;b_1,b_2;z\right)=\frac{\Gamma(b_1)\Gamma(b_2)}{\Gamma(a_2)}\hspace{6cm}\nonumber\\ \times(ze^{\pm i\pi})^{-a_1}\left\{\sum\limits_{n=0}^{R-1}\frac{(a_1)_n \Gamma(a_1-a_2-n)}{\Gamma(b_1-a_1-n)\Gamma(b_2-a_1-n)_n}\frac{(ze^{\pm i\pi})^{-n}}{n!}+O(|z|^{-R})\right\}\nonumber\\+
\frac{\Gamma(b_1)\Gamma(b_2)}{\Gamma(a_1)}\hspace{6cm}\nonumber\\ \times(ze^{\pm i\pi})^{-a_2}\left\{\sum\limits_{n=0}^{R-1}\frac{(a_2)_n \Gamma(a_2-a_1-n)}{\Gamma(b_1-a_2-n)\Gamma(b_2-a_2-n)_n}\frac{(ze^{\pm i\pi})^{-n}}{n!}+O(|z|^{-R})\right\}
\nonumber\\
+\frac{\Gamma(b_1)\Gamma(b_2)}{\Gamma(a_1)\Gamma(a_2)} e^{z}z^{a_1+a_2-b_1-b_2}\left\{\sum\limits_{n=0}^{S-1}\frac{\mu_n}{2^n}z^{-n}+O(|z|^{-S})\right\},\hspace{.8cm}
\label{asy2-2}
\end{eqnarray}
where the coefficient $\mu_n$ is given by formula 16.11.4. And the upper or lower signs are chosen according as $z$ lies in the upper (above the real axis) or lower half-plane (below the real axis).

Setting $z=\lambda x^\beta, a_1=1, a_2=1, b_1=2$ and $b_2=2$ in (\ref{asy2-2}) yields
\begin{equation}
_{2}F_{2}(1,1;2,2;\lambda x^\beta)\sim\frac{-2}{\lambda x^\beta}+\frac{e^{\lambda x^\beta}}{\lambda^2 x^{2\beta}}, \hspace{.12cm} |x|\gg1.
\end{equation}
Hence,
\begin{equation}
\lambda x^\beta \hspace{.075cm} _{2}F_{2}(1,1;2,2;\lambda x)\sim-2+\frac{e^{\lambda x^\beta}}{\lambda x^\beta}, \hspace{.12cm} |x|\gg1.
\end{equation}
This ends the proof.
\end{proof}
\begin{example}
One can now evaluate $\int e^{\lambda e^{\beta x}}dx$ in terms of $_{2}F_{2}$ using the substitution $u=e^x$, and obtain
\begin{align}
\int e^{\lambda e^{\beta x} }dx=\int \frac{e^{\lambda u^\beta}}{u} du&=\ln{u}+ \frac{\lambda u^\beta}{\beta}\hspace{.075cm}   _{2}F_{2}(1,1;2,2; \lambda u^\beta)+C\nonumber \\ &=x+ \frac{\lambda e^{\beta x}}{\beta}\hspace{.075cm}   _{2}F_{2}(1,1;2,2; \lambda e^{\beta x})+C.
\label{ex6-1}
\end{align}
\label{ex6}
\end{example}

\begin{theorem} The logarithmic integral is given by
\begin{multline}
\text{Li}=\int\limits_2^x\frac{dt}{\ln{t}}=\ln{\left(\frac{\ln{x}}{\ln{2}}\right)}+\ln{x}\hspace{.075cm} _{2}F_{2}(1,1;2,2;\ln{x})-
\ln{2}\hspace{.075cm} _{2}F_{2}(1,1;2,2;\ln{2}).
\label{Thm5-1}
\end{multline}
And for $x\gg2$,
\begin{equation}
\text{Li}=\int\limits_{2}^{x}\frac{dt}{\ln{t}}\sim \frac{x}{\ln{x}}+\ln{\left(\frac{\ln{x}}{\ln{2}}\right)}-2-
\ln{2}\hspace{.075cm} _{2}F_{2}(1,1;2,2;\ln{2}).
\label{Thm5-2}
\end{equation}
\label{Thm5}
\end{theorem}

\begin{proof}
Making the substitution $u=\ln{x}$ and using (\ref{prp2-1}) gives
\begin{align}
\int\limits_{2}^{x}\frac{dx}{\ln{x}}&=\int\limits_{\ln2}^{\ln{x}}\frac{e^u}{u}du=\left[\ln{u}+u\hspace{.075cm} _{2}F_{2}(1,1;2,2;u)\right]_{\ln2}^{\ln{x}}
\nonumber\\ & =\ln{\left(\frac{\ln{x}}{2}\right)}+\ln{x}\hspace{.075cm} _{2}F_{2}(1,1;2,2;\ln{x})-
\ln{2}\hspace{.075cm} _{2}F_{2}(1,1;2,2;\ln{2}).
\end{align}
Now setting $z=\ln{x}, a_1=1$, $a_2=1$, $b_1=2$ and $b_2=2$ in (\ref{asy2-2}) yields
\begin{equation}
_{2}F_{2}(1,1;2,2;\ln{x})\sim\frac{-2}{\ln{x}}+\frac{x}{(\ln{x})^2}, \hspace{.2cm} x\gg1.
\end{equation}
This gives
\begin{equation}
\ln{x}\hspace{.075cm} _{2}F_{2}(1,1;2,2;\ln{x})\sim-2+\frac{x}{\ln{x}}, \hspace{.2cm} x\gg1.
\end{equation}
Hence for $x\gg2$,
\begin{equation}
\text{Li}=\int\limits_{2}^{x}\frac{dt}{\ln{t}}\sim \frac{x}{\ln{x}}+\ln{\left(\frac{\ln{x}}{\ln{2}}\right)}-2-
\ln{2}\hspace{.075cm} _{2}F_{2}(1,1;2,2;\ln{2}).
\end{equation}
\end{proof}
It is important to note that Theorem \ref{Thm5} adds the term $\ln{\left(\frac{\ln{x}}{\ln{2}}\right)}-2-
\ln{2}\hspace{.075cm} _{2}F_{2}(1,1;2,2;\ln{2})$ to the known asymptotic expression of the logarithmic integral (\ref{Thm6-1}), $\text{Li}\sim {x}/{\ln{x}}$ \cite{AS,ND}. And this term is negligible if $x\sim O(10^6)$ or higher.

\begin{example} One can now evaluate $\int \ln{(\ln{x})} dx$ using integration by parts.
\begin{align}
\int \ln{(\ln{x})} dx&=x\ln{(\ln{x})}-\int \frac{1}{\ln{x}} dx
\nonumber \\ &=x\ln{(\ln{x})}-\ln{(\ln{x})}-\ln{x}\hspace{.075cm} _{2}F_{2}(1,1;2,2;\ln{x})+C.
\label{ex7-1}
\end{align}
\label{ex7}
\end{example}

\begin{theorem}
If $\beta\ge1$ and $\alpha>1$, then
\begin{equation}
\text{Ei}_{\alpha,\beta}=\int \frac{e^{\lambda x^\beta}}{x^\alpha}dx=\frac{x^{1-\alpha}}{1-\alpha}+\frac{\lambda x^{\beta-\alpha+1}}{\beta-\alpha+1}\hspace{.075cm}   _{2}F_{2}\left(1,-\frac{\alpha}{\beta}+\frac{1}{\beta}+1;2,-\frac{\alpha}{\beta}+\frac{1}{\beta}+2;\lambda x^\beta\right)+C.
\label{Thm6-1}
\end{equation}
And for $|x|\gg1$,
\begin{multline}
\frac{\lambda x^{\beta-\alpha+1}}{\beta-\alpha+1} \hspace{.075cm}_{2}F_{2}\left(1,-\frac{\alpha}{\beta}+\frac{1}{\beta}+1;2,-\frac{\alpha}{\beta}+\frac{1}{\beta}+2;\lambda x\right)\\
\sim \frac{\lambda}{\beta} \Gamma\left(-\frac{\alpha}{\beta}+\frac{1}{\beta}+1\right)\left(-\frac{1}{\lambda }\right)^{-\frac{\alpha}{\beta}+\frac{1}{\beta}+1}-\frac{x^{-\alpha+1}}{\beta}
+\frac{1}{\lambda\beta}\frac{e^{\lambda x^\beta}}{x^{\beta+\alpha-1}}.
\label{Thm6-2}
\end{multline}
\label{Thm6}
\end{theorem}
\begin{proof}
If $\beta\ge1$ and $\alpha>1$, then
\begin{align}
\text{Ei}_{\beta,\alpha}&=\int \frac{e^{\lambda x^\beta}}{x^\alpha}dx=\int\frac{1}{x^\alpha}\sum\limits_{n=0}^{\infty}\frac{(\lambda x^\beta)^n}{n!}dx
\nonumber\\ &=\frac{x^{1-\alpha}}{1-\alpha}+\sum\limits_{n=1}^{\infty}\frac{\lambda^n}{n!}\int x^{\beta n-\alpha}dx
\nonumber\\ &=\frac{x^{1-\alpha}}{1-\alpha}+\sum\limits_{n=0}^{\infty}\frac{\lambda^{n+1}}{(n+1)!}\frac{ x^{\beta n+\beta-\alpha+1}}{\beta n+\beta-\alpha+1}+C
\nonumber\\ &=\frac{x^{1-\alpha}}{1-\alpha}+\frac{\lambda x^{\beta-\alpha+1}}{\beta}\sum\limits_{n=0}^{\infty}\frac{\Gamma\left(n-\frac{\alpha}{\beta}+\frac{1}{\beta}+1\right)}{\Gamma(n+2)\Gamma\left(n-\frac{\alpha}{\beta}+\frac{1}{\beta}+2\right)}
\left(\lambda x^{\beta}\right)^n+C
\nonumber\\ &=\frac{x^{1-\alpha}}{1-\alpha}+\frac{\lambda x^{\beta-\alpha+1}}{\beta-\alpha+1}
\sum\limits_{n=0}^{\infty}\frac{(1)n\left(-\frac{\alpha}{\beta}+\frac{1}{\beta}+1\right)_n}{(2)_n\left(-\frac{\alpha}{\beta}+\frac{1}{\beta}+2\right)_n}\frac{ \left(\lambda x^{\beta}\right)^n}{n!}+C
\nonumber\\ &=\frac{x^{1-\alpha}}{1-\alpha}+\frac{\lambda x^{\beta-\alpha+1}}{\beta-\alpha+1}\hspace{.075cm}   _{2}F_{2}\left(1,-\frac{\alpha}{\beta}+\frac{1}{\beta}+1;2,-\frac{\alpha}{\beta}+\frac{1}{\beta}+2;\lambda x^\beta\right)+C.
\end{align}

Now setting $a_1=1, a_2=-\frac{\alpha}{\beta}+\frac{1}{\beta}+1, b_1=2, b_2=-\frac{\alpha}{\beta}+\frac{1}{\beta}+2$ and $z=\lambda x^\beta$ in (\ref{asy2-2}) gives,
\begin{multline}
_{2}F_{2}\left(1,-\frac{\alpha}{\beta}+\frac{1}{\beta}+1;2,-\frac{\alpha}{\beta}+\frac{1}{\beta}+2;\lambda x^\beta\right)\\
\sim -\left(-\frac{\alpha}{\beta}+\frac{1}{\beta}+1\right)\frac{1}{\lambda x^\beta}
+\Gamma\left(-\frac{\alpha}{\beta}+\frac{1}{\beta}+2\right)\left(\frac{1}{\lambda x^\beta}\right)^{-\frac{\alpha}{\beta}+\frac{1}{\beta}+1}+\frac{e^{\lambda x^\beta}}{\lambda^2 x^{2\beta}}.
\end{multline}
Hence,
\begin{multline}
\frac{\lambda x^{\beta-\alpha+1}}{\beta-\alpha+1} \hspace{.075cm}_{2}F_{2}\left(1,-\frac{\alpha}{\beta}+\frac{1}{\beta}+1;2,-\frac{\alpha}{\beta}+\frac{1}{\beta}+2;\lambda x^\beta\right)\\
\sim \frac{\lambda}{\beta-\alpha+1} \Gamma\left(-\frac{\alpha}{\beta}+\frac{1}{\beta}+2\right)\left(-\frac{1}{\lambda }\right)^{-\frac{\alpha}{\beta}+\frac{1}{\beta}+1}-\frac{x^{-\alpha+1}}{\beta}
+\frac{1}{\lambda\beta}\frac{e^{\lambda x^\beta}}{x^{\beta+\alpha-1}}\\
=\frac{\lambda}{\beta} \Gamma\left(-\frac{\alpha}{\beta}+\frac{1}{\beta}+1\right)\left(-\frac{1}{\lambda }\right)^{-\frac{\alpha}{\beta}+\frac{1}{\beta}+1}-\frac{x^{-\alpha+1}}{\beta}
+\frac{1}{\lambda\beta}\frac{e^{\lambda x^\beta}}{x^{\beta+\alpha-1}},\nonumber
\end{multline}
which is (\ref{Thm6-2}). This completes the proof.
\end{proof}


\begin{theorem}
For any constants $\alpha$, $\beta$ and $\lambda$,
\begin{multline}
_{1}F_{2}\left(-\frac{\alpha}{2\beta}+\frac{1}{2\beta}+\frac{1}{2};-\frac{\alpha}{2\beta}+\frac{1}{2\beta}+\frac{3}{2},\frac{3}{2};-\frac{\lambda^2 x^{2\beta}}{4}\right)\\=\frac{1}{2}\Bigl[{}_{2}F_{2}\left(1,-\frac{\alpha}{\beta}+\frac{1}{\beta}+1;2,-\frac{\alpha}{\beta}+\frac{1}{\beta}+2;i\lambda x^\beta\right)\\+{}_{2}F_{2}\left(1,-\frac{\alpha}{\beta}+\frac{1}{\beta}+1;2,-\frac{\alpha}{\beta}+\frac{1}{\beta}+2;-i\lambda x^\beta\right)\Bigr].
\label{Thm7-1}
\end{multline}
Or,
\begin{multline}
_{1}F_{2}\left(-\frac{\alpha}{2\beta}+\frac{1}{2\beta}+\frac{1}{2};-\frac{\alpha}{2\beta}+\frac{1}{2\beta}+\frac{3}{2},\frac{3}{2};\frac{\lambda^2 x^{2\beta}}{4}\right)\\=\frac{1}{2}\Bigl[{}_{2}F_{2}\left(1,-\frac{\alpha}{\beta}+\frac{1}{\beta}+1;2,-\frac{\alpha}{\beta}+\frac{1}{\beta}+2;\lambda x^\beta\right)\\+{}_{2}F_{2}\left(1,-\frac{\alpha}{\beta}+\frac{1}{\beta}+1;2,-\frac{\alpha}{\beta}+\frac{1}{\beta}+2;-\lambda x^\beta\right)\Bigr].
\label{Thm7-2}
\end{multline}
\label{Thm7}
\end{theorem}

\begin{proof}
Using Theorem \ref{Thm6}, we obtain
\begin{multline}
\int\frac{\sin{(\lambda  x^\beta)}}{x^\alpha} dx=\frac{1}{2i}\int  \frac{e^{i\lambda x^\beta}-e^{-i\lambda x^\beta}}{x^\alpha}dx
\\=\frac{1}{2}\frac{\lambda x^{\beta-\alpha+1}}{\beta-\alpha+1}\Bigl[ \hspace{.075cm}_{2}F_{2}\left(1,-\frac{\alpha}{\beta}+\frac{1}{\beta}+1;2,-\frac{\alpha}{\beta}+\frac{1}{\beta}+2;i\lambda x^\beta\right)\\+ \hspace{.075cm}_{2}F_{2}\left(1,-\frac{\alpha}{\beta}+\frac{1}{\beta}+1;2,-\frac{\alpha}{\beta}+\frac{1}{\beta}+2;-i\lambda x^\beta\right)\Bigr]+C.
\label{eq1}
\end{multline}
Hence, comparing (\ref{Thm1-1}) with (\ref{eq1}) gives (\ref{Thm7-1}).\\
Or, on the other hand,
\begin{multline}
\int\frac{\sinh{(\lambda  x^\beta)}}{x^\alpha} dx=\frac{1}{2}\int  \frac{e^{\lambda x^\beta}-e^{-\lambda x^\beta}}{x^\alpha}dx
\\=\frac{1}{2}\frac{\lambda x^{\beta-\alpha+1}}{\beta-\alpha+1}\Bigl[ \hspace{.075cm}_{2}F_{2}\left(1,-\frac{\alpha}{\beta}+\frac{1}{\beta}+1;2,-\frac{\alpha}{\beta}+\frac{1}{\beta}+2;\lambda x^\beta\right)\\+ \hspace{.075cm}_{2}F_{2}\left(1,-\frac{\alpha}{\beta}+\frac{1}{\beta}+1;2,-\frac{\alpha}{\beta}+\frac{1}{\beta}+2;-\lambda x^\beta\right)\Bigr]+C.
\label{eq2}
\end{multline}
Hence, comparing (\ref{sinh1}) with (\ref{eq2}) gives (\ref{Thm7-2}).


\end{proof}

\begin{theorem}
For any constants $\alpha$, $\beta$ and $\lambda$,
\begin{multline}
\frac{x^{2\beta-\alpha+1}}{2\beta-\alpha+1}\hspace{.075cm} _{2}F_{3}\left(1,-\frac{\alpha}{2\beta}+\frac{1}{2\beta}+1;-\frac{\alpha}{2\beta}+\frac{1}{2\beta}+2,\frac{3}{2},2;-\frac{\lambda^2 x^{2\beta}}{4}\right)\\=
\frac{1}{i}\frac{x^{\beta-\alpha+1}}{\beta-\alpha+1}\Bigl[\hspace{.075cm}   _{2}F_{2}\left(1,-\frac{\alpha}{\beta}-\frac{1}{\beta}+1;2,-\frac{\alpha}{\beta}+\frac{1}{\beta}+2;i\lambda x^\beta\right)
\\  -\hspace{.075cm}_{2}F_{2}\left(1,-\frac{\alpha}{\beta}+\frac{1}{\beta}+1;2,-\frac{\alpha}{\beta}+\frac{1}{\beta}+2;-i\lambda x^\beta\right)\Bigr].
\label{Thm8-1}
\end{multline}
Or, \begin{multline}
\frac{x^{2\beta-\alpha+1}}{2\beta-\alpha+1}\hspace{.075cm} _{2}F_{3}\left(1,-\frac{\alpha}{2\beta}+\frac{1}{2\beta}+1;-\frac{\alpha}{2\beta}+\frac{1}{2\beta}+2,\frac{3}{2},2;\frac{\lambda^2 x^{2\beta}}{4}\right)\\=
\frac{x^{\beta-\alpha+1}}{\beta-\alpha+1}\Bigl[\hspace{.075cm}   _{2}F_{2}\left(1,-\frac{\alpha}{\beta}-\frac{1}{\beta}+1;2,-\frac{\alpha}{\beta}+\frac{1}{\beta}+2;\lambda x^\beta\right)
\\  +\hspace{.075cm}_{2}F_{2}\left(1,-\frac{\alpha}{\beta}+\frac{1}{\beta}+1;2,-\frac{\alpha}{\beta}+\frac{1}{\beta}+2;-\lambda x^\beta\right)\Bigr].
\label{Thm8-2}
\end{multline}
\label{Thm8}
\end{theorem}

\begin{proof}
Using Theorem \ref{Thm6}, we obtain
\begin{multline}
\int \frac{\cos(\lambda x^\beta)}{x^\alpha}dx=\frac{1}{2}\int\frac{e^{i\lambda x^\beta}+e^{-i\lambda x^\beta}}{x^\alpha}dx
\\=\frac{x^{1-\alpha}}{1-\alpha}-\frac{1}{2i}\frac{\lambda x^{\beta-\alpha+1}}{\beta-\alpha+1}\Bigl[ \hspace{.075cm}_{2}F_{2}\left(1,-\frac{\alpha}{\beta}+\frac{1}{\beta}+1;2,-\frac{\alpha}{\beta}+\frac{1}{\beta}+2;i\lambda x^\beta\right)\\- \hspace{.075cm}_{2}F_{2}\left(1,-\frac{\alpha}{\beta}+\frac{1}{\beta}+1;2,-\frac{\alpha}{\beta}+\frac{1}{\beta}+2;-i\lambda x^\beta\right)\Bigr]+C,
\label{eq3}
\end{multline}
Hence, comparing (\ref{Thm4-1}) with (\ref{eq3}) gives (\ref{Thm8-1}).
Or, on the other hand,
\begin{multline}
\int \frac{\cosh(\lambda x^\beta)}{x^\alpha}dx=\frac{1}{2}\int\frac{e^{\lambda x^\beta}+e^{-\lambda x^\beta}}{x^\alpha}dx
\\=\frac{x^{1-\alpha}}{1-\alpha}+\frac{1}{2}\frac{\lambda x^{\beta-\alpha+1}}{\beta-\alpha+1}\Bigl[ \hspace{.075cm}_{2}F_{2}\left(1,-\frac{\alpha}{\beta}+\frac{1}{\beta}+1;2,-\frac{\alpha}{\beta}+\frac{1}{\beta}+2;\lambda x^\beta\right)\\- \hspace{.075cm}_{2}F_{2}\left(1,-\frac{\alpha}{\beta}+\frac{1}{\beta}+1;2,-\frac{\alpha}{\beta}+\frac{1}{\beta}+2;-\lambda x^\beta\right)\Bigr]+C,
\label{eq4}
\end{multline}
Hence, comparing (\ref{cosh}) with (\ref{eq4}) gives (\ref{Thm8-2}).
\end{proof}


\section{Conclusion}\label{conclusion}
$\text{Si}_{\beta,\alpha}=\int [\sin{(\lambda x^\beta)}/{(\lambda x^\alpha)}] dx$ and $\text{Ci}_{\beta,\alpha}=\int [\cos{(\lambda x^\beta)}/{(\lambda x^\alpha)}] dx$, where $\beta\ge1$ and $\alpha\ge 1$, were expressed in terms of the hypergeometric functions ${}_1F_2$ and ${}_2F_3$ respectively, and their asymptotic expressions for $|x|\gg1$ were obtained (see Theorems \ref{Thm1}, \ref{Thm3} and \ref{Thm4}).
$\text{Ei}_{\beta,\alpha}=\int (e^{\lambda x^\beta}/x^\alpha)dx$ and $\int dx/\ln{x}$ were expressed in terms of the hypergeometric function ${}_2F_2$, and their asymptotic expressions for $|x|\gg1$ were also obtained (see Proposition (\ref{prp2}), Theorems \ref{Thm5} and  \ref{Thm6}). Therefore, their corresponding definite integrals can now be evaluated using the FTC (see for example Corollary \ref{Crl1}).

Using the Euler and hyperbolic identities $\text{Si}_{\beta,\alpha}$ and $\text{Ci}_{\beta,\alpha}$ were expressed in terms of $\text{Ei}_{\beta,\alpha}$. And hence, some expressions of the hypergeometric functions ${}_1F_2$ and ${}_2F_3$ in terms of ${}_2F_2$ were derived (see Theorems \ref{Thm7} and  \ref{Thm8}).

The evaluation of the integral $\int dx/\ln{x}$ and the asymptotic expression of the hypergeometric function ${}_2F_2$ for $|x|\gg1$ allowed us to add the term $\ln{\left(\frac{\ln{x}}{\ln{2}}\right)}-2-\ln{2}\hspace{.075cm} _{2}F_{2}(1,1;2,2;\ln{2})$ to the known asymptotic expression of the logarithmic integral, which is $\text{Li}=\int_{2}^{x} dt/\ln{t}\sim {x}/{\ln{x}}$ \cite{AS,ND}, so that it is given by $\text{Li}=\int_{2}^{x}{dt}/{\ln{t}}\sim {x}/{\ln{x}}+\ln{\left(\frac{\ln{x}}{\ln{2}}\right)}-2-
\ln{2}\hspace{.075cm} _{2}F_{2}(1,1;2,2;\ln{2})$ (Theorem \ref{Thm5}).

By substitution, the non-elementary integral $\int e^{\lambda e^ {\beta x}}dx$ was written in terms of $\text{Ei}_{\beta,1}$ and then evaluated in terms of $_{2}F_{2}$. And using integration by parts, the non-elementary integral $\int \ln(\ln{x}) dx$ was written in terms of $\int dx/\ln{x}$ and then evaluated in terms of $_{2}F_{2}$.

\begin{thebibliography}{99}

\bibitem{AS}  M. Abramowitz, I. A. Stegun, \textit{Handbook of mathematical functions with formulas, graphs and mathematical tables}, National Bureau  of Standards, 1964.
\bibitem{B}  P. Billingsley, \textit{Probability and measure}, Wiley series in Probability and Mathematical Statistics, 1986.

\bibitem{K}  S. G. Krantz, \textit{Handbook of Complex variables}, Boston, MA Birk\"{au}sser, 1999.
\bibitem{MZ} E. A. Marchisotto, G.-A. Zakeri, An invitation to integration in finite terms, College Mathematical Journal 25:4 (1994) 295--308
\bibitem{ND}  NIST Digital Library of Mathematical Functions, \url{http://dlmf.nist.gov/}.
\bibitem{R}  M. Rosenlicht, Integration in finite terms, American Mathematical Monthly 79:9 (1972) 963--972.
\end{thebibliography}








\end{document}


