
\documentclass{amsart}
\usepackage{amssymb}
\usepackage{amsfonts}

\setcounter{MaxMatrixCols}{10}


\newtheorem{theorem}{Theorem}
\theoremstyle{plain}
\newtheorem{acknowledgement}{Acknowledgement}
\newtheorem{algorithm}{Algorithm}
\newtheorem{axiom}{Axiom}
\newtheorem{case}{Case}
\newtheorem{claim}{Claim}
\newtheorem{conclusion}{Conclusion}
\newtheorem{condition}{Condition}
\newtheorem{conjecture}{Conjecture}
\newtheorem{corollary}{Corollary}
\newtheorem{criterion}{Criterion}
\newtheorem{definition}{Definition}
\newtheorem{example}{Example}
\newtheorem{exercise}{Exercise}
\newtheorem{lemma}{Lemma}
\newtheorem{notation}{Notation}
\newtheorem{problem}{Problem}
\newtheorem{proposition}{Proposition}
\newtheorem{remark}{Remark}
\newtheorem{solution}{Solution}
\newtheorem{summary}{Summary}
\numberwithin{equation}{section}


\begin{document}
\title[existence of solutions of integral system]{Measure of noncompactness
in the study of solutions for a system of integral equations}
\author{Vatan KARAKAYA}
\address{Department of Mathematical Engineering, Faculty of
Chemistry-Metallurgical, Yildiz Technical University, Istanbul, Turkey }
\email{vkkaya@yahoo.com}
\author{Mohammad MURSALEEN}
\address{Department of Mathematics, Aligarh Muslim University, Aligarh
202002, India }
\email{mursaleenm@gmail.com}
\author{Nour El Houda BOUZARA}
\address{Department of Mathematics, Faculty of Science and Letters, Yildiz
Technical University, Istanbul, Turkey }
\email{bzr.nour@gmail.com}
\subjclass{47H10, 47H08, 45G15}
\keywords{Measure of noncompactness, Fixed point, System of integral
equations.}

\begin{abstract}
In this work, we prove the existence of solutions for a tripled system of
integral equations using some new results of fixed point theory associated
with measure of noncompactness. These results extend some previous works in
the literature, since the condition under which the operator admits fixed
points is more general than the others in literature.
\end{abstract}

\maketitle

\section{INTRODUCTION AND PRELIMINARIES}

In recent years, measure of noncompactness which was introduced by
Kuratowski \cite{kuratovski} in 1930 and has provided powerful tools for
obtaining the solutions of a large variety of integral equations and
systems, see Aghajani et al. \cite{Aghajani11}, \cite{Aghajani2}, \cite{Aghajanicoupled}, Banas \cite{banas1}, Banas and Rzepka \cite{banas3},
Mursaleen and Mohiuddine. \cite{murs}, Araba et al. \cite{reza}, Deepmala
and \ Pathak \cite{deepmala}, Shaochun and Gan \cite{Shaochun}, Sikorska 
\cite{sikorska} and many others.

In this paper, we study the solvability of the following system of integral
equations

\begin{equation*}
\left\{ 
\begin{array}{c}
x\left( t\right) =g_{1}\left( t\right) +f_{1}\left( t,x\left( \xi _{1}\left(
t\right) \right) ,y\left( \xi _{1}\left( t\right) \right) ,z\left( \xi
_{1}\left( t\right) \right) ,\psi \left( \int_{0}^{q_{1}\left( t\right)
}h\left( t,s,x\left( \eta _{1}\left( s\right) \right) ,y\left( \eta
_{1}\left( s\right) \right) ,z\left( \eta _{1}\left( s\right) \right)
\right) ds\right) \right) \\ 
y\left( t\right) =g_{2}\left( t\right) +f_{2}\left( t,x\left( \xi _{2}\left(
t\right) \right) ,y\left( \xi _{2}\left( t\right) \right) ,z\left( \xi
_{2}\left( t\right) \right) ,\psi \left( \int_{0}^{q_{2}\left( t\right)
}h\left( t,s,x\left( \eta _{1}\left( s\right) \right) ,y\left( \eta
_{2}\left( s\right) \right) ,z\left( \eta _{2}\left( s\right) \right)
\right) ds\right) \right) \\ 
z\left( t\right) =g_{3}\left( t\right) +f_{3}\left( t,x\left( \xi _{3}\left(
t\right) \right) ,y_{3}\left( \xi \left( t\right) \right) ,z_{3}\left( \xi
\left( t\right) \right) ,\psi \left( \int_{0}^{q_{3}\left( t\right) }h\left(
t,s,x\left( \eta _{3}\left( s\right) \right) ,y\left( \eta _{3}\left(
s\right) \right) ,z\left( \eta _{3}\left( s\right) \right) \right) ds\right)
\right)\end{array}\right. ,
\end{equation*}by establishing some results of existence for fixed points of condensing
operators in Banach spaces.

Throughout this paper, $X$ is assumed to be a Banach space. The family of
bounded subset, closure and closed convex hull of $X$ are denoted by $\mathcal{B}_{X}$, $\overline{X}$ and $ConvX$, respectively.

We now gather some well-known definitions and results from the literature
which will be used throughout this paper.

\begin{definition}
\cite{banas2} Let $X$ be a Banach space and $\mathcal{B}_{X}$ the family of
bounded subset of $X.$ A map\begin{equation*}
\mu :\mathcal{B}_{X}\rightarrow \left[ 0,\infty \right)
\end{equation*}
which satisfies the following:

\begin{enumerate}
\item $\mu \left( A\right) =0\Leftrightarrow A$ is a precompact set,

\item $A\subset B\Rightarrow \mu \left( A\right) \leqslant \mu \left(
B\right) ,$

\item $\mu \left( A\right) =\mu \left( \overline{A}\right) ,$ $\forall A\in 
\mathcal{B}_{X},$

\item $\mu \left( ConvA\right) =\mu \left( A\right) ,$

\item $\mu \left( \lambda A+\left( 1-\lambda \right) B\right) \leqslant
\lambda \mu \left( A\right) +\left( 1-\lambda \right) \mu \left( B\right) ,$
for $\lambda \in \left[ 0,1\right] ,$

\item Let $\left( A_{n}\right) $ be a sequence of closed sets from $\mathcal{B}_{X}$\ such that $A_{n+1}\subseteq A_{n},$ $\left( n\geqslant 1\right) $
and $\lim\limits_{n\rightarrow \infty }\mu \left( A_{n}\right) =0.$ Then,
the intersection set $A_{\infty }=\bigcap\limits_{n=1}^{\infty }A_{n}$ is
nonempty and $A_{\infty }$ is precompact,
\end{enumerate}

The functional $\mu $\ is called measure of noncompactness defined on the
Banach space $X$.
\end{definition}

\begin{theorem}
\cite{lec.note} Let $C$ be a nonempty closed, bounded and convex subset of $X $. If $T:C\rightarrow C$ is a continuous mapping 
\begin{equation*}
\mu \left( TA\right) \leqslant k\mu \left( A\right) ,\text{ \ }k\in \left[
0,1\right) \text{,}
\end{equation*}then $T$ has a fixed point.
\end{theorem}

The following theorem is considered as a generalization of Darbo fixed point
theorem.

\begin{theorem}
\cite{mursaleen} Let $C\ $be a nonempty closed, bounded and convex subset of 
$X$ and $T:C\rightarrow C$ be a continuous mapping such that for any subset $A$ of $C$\begin{equation*}
\mu \left( TA\right) \leqslant \beta \left( \mu \left( A\right) \right) \mu
\left( A\right) ,
\end{equation*}where $\beta :\mathbb{R}_{+}\rightarrow \left[ 0,1\right) $ that is $\beta
\left( t_{n}\right) \rightarrow 1$ implies $t_{n}\rightarrow 0.$ Then, $T$
has at least one fixed point.
\end{theorem}

\begin{corollary}
\label{fpth}\cite{mursaleen} Let $C\ $\ be a nonempty closed, bounded and
convex subset of $X$ and $T:C\rightarrow C$ be a continuous mapping such
that for any subset $A$ of $C$\begin{equation*}
\mu \left( TA\right) \leqslant \varphi \left( \mu \left( A\right) \right) ,
\end{equation*}where $\varphi :\mathbb{R}_{+}\rightarrow \mathbb{R}_{+}$ is a nondecreasing
and upper semicontinuous functions, that is, for every $t>0,$ $\varphi
\left( t\right) <t.$ Then, $T$ has at least one fixed point.
\end{corollary}

\begin{theorem}
\label{th copy}\cite{kamenski} Let $\mu _{1},$ $\mu _{2},$ ...,$\mu _{n}$ be
measures of noncompactness in Banach spaces $E_{1},$ $E_{2},$ ...$E_{n},$ $\left( \text{respectively}\right) $.

Then the function\begin{equation*}
\widetilde{\mu }\left( X\right) =F\left( \mu _{1}\left( X_{1}\right) ,\mu
_{2}\left( X_{2}\right) ,...,\mu _{n}\left( X_{n}\right) \right) ,
\end{equation*}defines a measure of noncompactness in $E_{1}\times E_{2}\times $ ...$\times
E_{n}$ where $X_{i}$ is the natural projection of $X$\ on $E_{i},$ for $i=1,2,...,n,$ and $F$ be a convex function defined by\begin{equation*}
F:\left[ 0,\infty \right) ^{n}\rightarrow \left[ 0,\infty \right) ,
\end{equation*}such that,\begin{equation*}
F\left( x_{1},x_{2},...,x_{n}\right) =0\Leftrightarrow x_{i}=0,\text{ for }i=1,2,...,n.
\end{equation*}
\end{theorem}

\begin{example}
\cite{nour}We can notice that by taking 
\begin{equation*}
F\left( x,y,z\right) =\max \left\{ x,y,z\right\} \text{ for any }\left(
x,y,z\right) \in \left[ 0,\infty \right) ^{3},
\end{equation*}or\begin{equation*}
F\left( x,y,z\right) =x+y+z\text{ for any }\left( x,y,z\right) \in \left[
0,\infty \right) ^{3}.
\end{equation*}Then $F$ satisfies the conditions of Theorem \ref{th copy}. Thus for a
measure of noncompactness $\mu _{i}$ $\left( i=1,2,3\right) $, we have that 
\begin{equation*}
\widetilde{\mu }\left( X\right) =\max \left( \mu _{1}\left( X_{1}\right)
,\mu _{2}\left( X_{2}\right) ,\mu _{3}\left( X_{3}\right) \right) ,
\end{equation*}or 
\begin{equation*}
\widetilde{\mu }\left( X\right) =\mu _{1}\left( X_{1}\right) +\mu _{2}\left(
X_{2}\right) +\mu _{3}\left( X_{3}\right) ,
\end{equation*}defines a measure of noncompactness in the space $E\times E\times E$ where $X_{i}$, $i=1,2,3$ are the natural projections of $X$ on $E_{i}$.
\end{example}

\section{MAIN RESULTS}

\begin{theorem}
Let $A$ be a nonempty, bounded ,closed and convex subset of a Banach space $X $\ and let $\varphi :\mathbb{R}^{+}\rightarrow \mathbb{R}^{+}$ be a
nondecreasing and upper semicontinuous function such that $\varphi \left(
t\right) <t$ for all $t>0.$ Then for any measure of noncompactness $\mu $,
and continuous operators $T_{i}:A\times A\times A\rightarrow A$ $\left(
i=1,2,3\right) $ satisfying\begin{equation}
\mu \left( T_{i}\left( X_{1}\times X_{2}\times X_{3}\right) \right)
\leqslant \varphi \left( \max \left( \mu \left( X_{1}\right) ,\mu \left(
X_{2}\right) ,\mu \left( X_{3}\right) \right) \right) ,\text{ }X_{1},X_{2},X_{3}\in A,  \label{cndt}
\end{equation}there exist $x^{\ast },y^{\ast },z^{\ast }\in A$ such that\begin{equation*}
\left\{ 
\begin{array}{c}
T_{1}\left( x^{\ast },y^{\ast },z^{\ast }\right) =x^{\ast } \\ 
T_{2}\left( x^{\ast },y^{\ast },z^{\ast }\right) =y^{\ast } \\ 
T_{3}\left( x^{\ast },y^{\ast },z^{\ast }\right) =z^{\ast }\end{array}\right. .
\end{equation*}
\end{theorem}

\begin{proof}
Consider the following measure of noncompactness\begin{equation*}
\widetilde{\mu }\left( A\times A\times A\right) =\max \left( \mu \left(
X_{1}\right) ,\mu \left( X_{2}\right) ,\mu \left( X_{3}\right) \right) ,
\end{equation*}where $X_{1},X_{2},X_{3}\in A$ and the mapping $\ T:A\times A\times
A\rightarrow A,$ 
\begin{equation*}
T\left( x,y,z\right) =\left( T_{1}\left( x,y,z\right) ,T_{2}\left(
y,x,z\right) ,T_{3}\left( z,y,x\right) \right) .
\end{equation*}We have,\begin{eqnarray*}
\widetilde{\mu }\left( T\left( A\times A\times A\right) \right) &=&\widetilde{\mu }\left( \left( T_{1}\left( X_{1}\times X_{2}\times
X_{3}\right) \right) ,T_{2}\left( X_{2}\times X_{1}\times X_{3}\right)
,T_{3}\left( X_{3}\times X_{2}\times X_{1}\right) \right) \\
&=&\max \left\{ \mu \left( T_{1}\left( X_{1}\times X_{2}\times X_{3}\right)
\right) ,\mu \left( T_{2}\left( X_{2}\times X_{1}\times X_{3}\right) \right)
,\mu \left( T_{3}\left( X_{3}\times X_{2}\times X_{1}\right) \right) \right\}
\\
&\leqslant &\max \left\{ \varphi \left( \max \left( \mu \left( X_{1}\right)
,\mu \left( X_{2}\right) ,\mu \left( X_{3}\right) \right) \right) ,\varphi
\left( \max \left( \mu \left( X_{1}\right) ,\mu \left( X_{2}\right) ,\mu
\left( X_{3}\right) \right) \right) ,\right. \\
&&\left. \varphi \left( \max \left( \mu \left( X_{1}\right) ,\mu \left(
X_{2}\right) ,\mu \left( X_{3}\right) \right) \right) \right\} .
\end{eqnarray*}By hypothesis $\varphi $\ is a non-decreasing function, then\begin{eqnarray*}
\widetilde{\mu }\left( T\left( A\times A\times A\right) \right) &\leqslant
&\varphi \left[ \max \left\{ \max \left( \mu \left( X_{1}\right) ,\mu \left(
X_{2}\right) ,\mu \left( X_{3}\right) \right) ,\max \left( \mu \left(
X_{1}\right) ,\mu \left( X_{2}\right) ,\mu \left( X_{3}\right) \right)
\right. \right. , \\
&&\left. \left. \max \left( \mu \left( X_{1}\right) ,\mu \left( X_{2}\right)
,\mu \left( X_{3}\right) \right) \right\} \right] .
\end{eqnarray*}Consequently,\begin{equation*}
\widetilde{\mu }\left( T\left( A\times A\times A\right) \right) \leqslant
\varphi \left( \widetilde{\mu }\left( A\times A\times A\right) \right) .
\end{equation*}So,\begin{equation*}
\mu \left( T_{1}\left( x,y,z\right) ,T_{2}\left( y,x,z\right) ,T_{3}\left(
z,y,x\right) \right) \leqslant \varphi \left( \max \left( \mu \left(
X_{1}\right) ,\mu \left( X_{2}\right) ,\mu \left( X_{3}\right) \right)
\right) .
\end{equation*}By Corollary \ref{fpth}, we conclude that there exist $x^{\ast },y^{\ast
},z^{\ast }\in A$ such that\begin{equation*}
T\left( x^{\ast },y^{\ast },z^{\ast }\right) =\left( x^{\ast },y^{\ast
},z^{\ast }\right) .
\end{equation*}In the other hand,\begin{equation*}
T\left( x^{\ast },y^{\ast },z^{\ast }\right) =\left( T_{1}\left( x^{\ast
},y^{\ast },z^{\ast }\right) ,T_{2}\left( x^{\ast },y^{\ast },z^{\ast
}\right) ,T_{3}\left( x^{\ast },y^{\ast },z^{\ast }\right) \right) .
\end{equation*}Hence,\begin{equation*}
\left\{ 
\begin{array}{c}
T_{1}\left( x^{\ast },y^{\ast },z^{\ast }\right) =x^{\ast } \\ 
T_{2}\left( x^{\ast },y^{\ast },z^{\ast }\right) =y^{\ast } \\ 
T_{3}\left( x^{\ast },y^{\ast },z^{\ast }\right) =z^{\ast }\end{array}\right. .
\end{equation*}
\end{proof}

It is very natural to extend the above result from three dimensions to
multidimensional fixed point and in the same way we can prove the following
theorem.

\begin{theorem}
Let $A$ be a nonempty, bounded ,closed and convex subset of a Banach space $X $\ and let $\varphi :\mathbb{R}^{+}\rightarrow \mathbb{R}^{+}$ be a
nondecreasing and upper semicontinuous function such that $\varphi \left(
t\right) <t$ for all $t>0.$ Then for any measure of noncompactness $\mu \ $and for continuous operators $T_{i}:A^{n}\rightarrow \Omega $ \ $\left(
i=1,...,n\right) $ satisfying\begin{equation*}
\mu \left( T_{i}\left( X_{1}\times ...\times X_{n}\right) \right) \leqslant
\varphi \left( \max \left( \mu \left( X_{1}\right) ,...,\mu \left(
X_{n}\right) \right) \right) ,\text{ }X_{i}\in A,\text{ }i=\overline{1,n},
\end{equation*}there exist $x_{1}^{\ast },...,x_{n}^{\ast }$ such that,\begin{equation*}
\left\{ 
\begin{array}{c}
T_{1}\left( x_{1}^{\ast },...,x_{n}^{\ast }\right) =x_{1}^{\ast } \\ 
\vdots \\ 
T_{n}\left( x_{1}^{\ast },...,x_{n}^{\ast }\right) =x_{n}^{\ast }\end{array}\right. .
\end{equation*}
\end{theorem}

As a particular case we get the following corollary:

\begin{corollary}
\cite{Aghajanicoupled}Let $A$ be a nonempty, bounded ,closed and convex
subset of a Banach space $X$\ and let $\varphi :\mathbb{R}^{+}\rightarrow 
\mathbb{R}^{+}$ be a nondecreasing and upper semicontinuous function such
that $\varphi \left( t\right) <t$ for all $t>0.$ Then for any measure of
noncompactness $\mu ,$ the continuous operator $G:A\times A\times
A\rightarrow A$ satisfying\begin{equation*}
\mu \left( G\left( X_{1}\times ...\times X_{n}\right) \right) \leqslant
k\max \left( \mu \left( X_{1}\right) ,...,\mu \left( X_{n}\right) \right) ,\text{ }X_{1},...,X_{n}\in A.
\end{equation*}
\end{corollary}

And for the case $n=2,$

\begin{corollary}
\cite{Aghajanicoupled}Let $A$ be a nonempty, bounded ,closed and convex
subset of a Banach space $X$\ and let $\varphi :\mathbb{R}^{+}\rightarrow 
\mathbb{R}^{+}$ be a nondecreasing and upper semicontinuous function such
that $\varphi \left( t\right) <t$ for all $t>0.$ Then for any measure of
noncompactness $\mu ,$ the continuous operator $G:A\times A\times
A\rightarrow A$ satisfying\begin{equation*}
\mu \left( G\left( X_{1}\times X_{2}\right) \right) \leqslant k\max \left(
\mu \left( X_{1}\right) ,\mu \left( X_{2}\right) \right) ,\text{ }X_{1},X_{2}\in A.
\end{equation*}
\end{corollary}

In the following we choose for the space $X$ the space $BC\left( \mathbb{R}^{+}\right) $ i,e the space of all real functions defined, bounded and
continuous on $\mathbb{R}^{+}.$ Then, we get the following theorem.

\begin{theorem}
\label{main}Let $A$ be a nonempty, bounded, closed and convex subset of $BC\left( \mathbb{R}^{+}\right) $ and $T_{i}:A\times A\times A\rightarrow A$
be a continuous operator such that\begin{equation}
\left\Vert T_{i}\left( x,y,z\right) -T_{i}\left( u,v,w\right) \right\Vert
_{\infty }\leqslant \varphi \left( \max \left\{ \left\Vert x-u\right\Vert
_{\infty },\left\Vert y-v\right\Vert _{\infty },\left\Vert z-w\right\Vert
_{\infty }\right\} \right) ,\text{ for every }x,y,z,u,v,w\in A.
\label{cndtn}
\end{equation}where $\varphi :\mathbb{R}^{+}\rightarrow \mathbb{R}^{+}$ is a nondecreasing
and upper semicontinuous function such that $\varphi \left( t\right) <t$ for
all $t>0$. Then there exist $x^{\ast },y^{\ast },z^{\ast }\in A$ such that,\begin{equation*}
\left\{ 
\begin{array}{c}
T_{1}\left( x^{\ast },y^{\ast },z^{\ast }\right) =x^{\ast } \\ 
T_{2}\left( x^{\ast },y^{\ast },z^{\ast }\right) =y^{\ast } \\ 
T_{3}\left( x^{\ast },y^{\ast },z^{\ast }\right) =z^{\ast }\end{array}\right. .
\end{equation*}
\end{theorem}

\begin{proof}
To verify that the operator $T_{i}:A\times A\times A\rightarrow A$ satisfy
the condition $\left( \text{\ref{cndt}}\right) $ we recall the following
notions.

The measure of noncompactness on $BC\left( \mathbb{R}^{+}\right) $ for a
positive fixed $t$ on $\mathcal{B}_{BC\left( \mathbb{R}^{+}\right) }$\ is
defined as follows:\begin{equation*}
\mu \left( X\right) =\omega _{0}\left( X\right) +\lim \sup_{t\rightarrow
\infty }diamX\left( t\right) ,
\end{equation*}that is, $diamX\left( t\right) =\sup \left\{ \left\vert x\left( t\right)
-y\left( t\right) \right\vert :x,y\in X\right\} ,$ $X\left( t\right)
=\left\{ x\left( t\right) :x\in X\right\} .$and 
\begin{equation*}
\omega _{0}\left( X\right) =\lim_{T\rightarrow \infty }\omega _{0}^{K}\left(
X\right) ,
\end{equation*}\begin{equation*}
\omega _{0}^{K}\left( X\right) =\lim_{\epsilon \rightarrow 0}\omega
^{K}\left( X,\epsilon \right) ,
\end{equation*}\begin{equation*}
\omega ^{K}\left( X,\epsilon \right) =\sup \left\{ \omega ^{K}\left(
x,\epsilon \right) :x\in X\right\} ,
\end{equation*}\begin{equation*}
\omega ^{K}\left( x,\epsilon \right) =\sup \left\{ \left\vert x\left(
t\right) -x\left( s\right) \right\vert :t,s\in \left[ 0,K\right] ,\text{ }\left\vert t-s\right\vert \leqslant \epsilon \right\} \text{, for }K>0,
\end{equation*}where $\omega ^{K}\left( x,\epsilon \right) $ for $x\in X$ and $\epsilon >0,$
is the modulus of continuity of $x$ on the compact $\left[ 0,K\right] $,
where $T$ is a positive number.

We have\begin{equation*}
\left\Vert T_{i}\left( x,y,z\right) \left( t\right) -T_{i}\left(
x,y,z\right) \left( s\right) \right\Vert \leqslant \varphi \left( \max
\left\{ \left\Vert x\left( t\right) -x\left( s\right) \right\Vert
,\left\Vert y\left( t\right) -y\left( s\right) \right\Vert ,\left\Vert
z\left( t\right) -z\left( s\right) \right\Vert \right\} \right) ,
\end{equation*}by taking the supremum and using the fact that $\varphi $ is nondecreasing,
we get\begin{equation*}
\omega ^{K}\left( T_{i}\left( x,y,z\right) ,\epsilon \right) \leqslant
\varphi \left( \max \left\{ \omega ^{K}\left( x,\epsilon \right) ,\omega
^{K}\left( y,\epsilon \right) ,\omega ^{K}\left( z,\epsilon \right) \right\}
\right) .
\end{equation*}Thus,\begin{equation}
\omega _{0}\left( T_{i}\left( X_{1}\times X_{2}\times X_{3}\right) \right)
\leqslant \varphi \left( \max \left\{ \omega _{0}\left( X_{1}\right) ,\omega
_{0}\left( X_{2}\right) ,\omega _{0}\left( X_{3}\right) \right\} \right) .
\label{modulos}
\end{equation}Since in $\left( \text{\ref{cndtn}}\right) $ $x,y$ and $z$ are arbitrary and 
$\varphi $ is non-decreasing,\begin{equation*}
DiamT_{i}\left( X_{1}\times X_{2}\times X_{3}\right) \left( t\right)
\leqslant \varphi \left( \max \left\{ DiamX_{1}\left( t\right) ,\text{ }DiamX_{2}\left( t\right) ,DiamX_{3}\left( t\right) \right\} \right) .
\end{equation*}Since $X_{1}\left( t\right) ,X_{2}\left( t\right) ,X_{3}\left( t\right) $
are subspaces of $BC\left( \mathbb{R}_{+}\right) $. Then,\begin{multline*}
\lim \sup_{t\rightarrow \infty }DiamT_{i}\left( X_{1}\times X_{2}\times
X_{3}\right) \left( t\right) \leqslant \lim \sup_{t\rightarrow \infty
}\varphi \left( \max \left\{ DiamX_{1}\left( t\right) ,\text{ }DiamX_{2}\left( t\right) ,DiamX_{3}\left( t\right) \right\} \right) +\Phi
\left( \epsilon \right) \\
\leqslant \varphi \left( \max \left\{ \lim \sup_{t\rightarrow \infty
}DiamX_{1}\left( t\right) ,\lim \sup_{t\rightarrow \infty }\text{ }DiamX_{2}\left( t\right) ,\lim \sup_{t\rightarrow \infty }DiamX_{3}\left(
t\right) \right\} \right) .
\end{multline*}Using $\varphi \left( t\right) <t$ for all $t>0$ and from $\left( \text{\ref{modulos}}\right) $ and the above inequality, we get\begin{equation*}
\mu \left( T_{i}\left( X_{1}\times X_{2}\times X_{3}\right) \right)
\leqslant \varphi \left( \max \left( \mu \left( X_{1}\right) ,\mu \left(
X_{2}\right) ,\mu \left( X_{3}\right) \right) \right) ,\text{ }X_{1},X_{2},X_{3}\in A.
\end{equation*}Consequently, there exist $x^{\ast },y^{\ast },z^{\ast }\in A$ such that\begin{eqnarray*}
T\left( x^{\ast },y^{\ast },z^{\ast }\right) &=&\left( T_{1}\left( x^{\ast
},y^{\ast },z^{\ast }\right) ,T_{2}\left( x^{\ast },y^{\ast },z^{\ast
}\right) ,T_{3}\left( x^{\ast },y^{\ast },z^{\ast }\right) \right) \\
&=&\left( x^{\ast },y^{\ast },z^{\ast }\right) .
\end{eqnarray*}Thus, 
\begin{equation*}
\left\{ 
\begin{array}{c}
T_{1}\left( x^{\ast },y^{\ast },z^{\ast }\right) =x^{\ast } \\ 
T_{2}\left( x^{\ast },y^{\ast },z^{\ast }\right) =y^{\ast } \\ 
T_{3}\left( x^{\ast },y^{\ast },z^{\ast }\right) =z^{\ast }\end{array}\right. .
\end{equation*}
\end{proof}

\section{APPLICATION}

Now we will use the results of the previous section to resolve the following
system\begin{equation}
\left\{ 
\begin{array}{c}
x\left( t\right) =g_{1}\left( t\right) +f_{1}\left( t,x\left( \xi _{1}\left(
t\right) \right) ,y\left( \xi _{1}\left( t\right) \right) ,z\left( \xi
_{1}\left( t\right) \right) ,\psi \left( \int_{0}^{q_{1}\left( t\right)
}h\left( t,s,x\left( \eta _{1}\left( s\right) \right) ,y\left( \eta
_{1}\left( s\right) \right) ,z\left( \eta _{1}\left( s\right) \right)
\right) ds\right) \right) \\ 
y\left( t\right) =g_{2}\left( t\right) +f_{2}\left( t,x\left( \xi _{2}\left(
t\right) \right) ,y\left( \xi _{2}\left( t\right) \right) ,z\left( \xi
_{2}\left( t\right) \right) ,\psi \left( \int_{0}^{q_{2}\left( t\right)
}h\left( t,s,x\left( \eta _{1}\left( s\right) \right) ,y\left( \eta
_{2}\left( s\right) \right) ,z\left( \eta _{2}\left( s\right) \right)
\right) ds\right) \right) \\ 
z\left( t\right) =g_{3}\left( t\right) +f_{3}\left( t,x_{3}\left( \xi \left(
t\right) \right) ,y_{3}\left( \xi \left( t\right) \right) ,z_{3}\left( \xi
\left( t\right) \right) ,\psi \left( \int_{0}^{q_{3}\left( t\right) }h\left(
t,s,x\left( \eta _{3}\left( s\right) \right) ,y\left( \eta _{3}\left(
s\right) \right) ,z\left( \eta _{3}\left( s\right) \right) \right) ds\right)
\right)\end{array}\right. .  \label{sys}
\end{equation}

\bigskip

\bigskip

\qquad We study system $\left( \text{\ref{sys}}\right) $ under the following
assumptions:

\begin{enumerate}
\item[$\left( i\right) $] $\xi _{i},\eta _{i},q_{i}:\mathbb{R}_{+}\rightarrow \mathbb{R}_{+},$ $\left( i=1,2,3\right) ,$ are continuous
and $\xi _{i}\left( t\right) \rightarrow \infty $ as $t\rightarrow \infty .$

\item[$\left( ii\right) $] The function $\psi _{i}:\mathbb{R\rightarrow R}$, 
$\left( i=1,2,3\right) ,$ is continuous and there exist positive $\delta
_{i},\alpha _{i}$\ such that 
\begin{equation*}
\left\vert \psi _{i}\left( t_{1}\right) -\psi _{i}\left( t_{2}\right)
\right\vert \leqslant \delta _{i}\left\vert t_{1}-t_{2}\right\vert ^{\alpha
_{i}},
\end{equation*}for $\left( i=1,2,3\right) $ and\ any $t_{1},t_{2}\in \mathbb{R}_{+}$.

\item[$\left( iii\right) $] $f_{i}:\mathbb{R}_{+}\times \mathbb{R\times
R\times R\times R\times R\rightarrow R}$ are continuous, $g_{i}:\mathbb{R}_{+}\mathbb{\rightarrow R}$ are bounded and there exists non decreasing
continuous function $\Phi _{i}:\mathbb{R}_{+}\mathbb{\rightarrow R}_{+}$
with $\Phi _{i}\left( 0\right) =0,$ $\left( i=1,2,3\right) ,$ such that\begin{equation*}
\left\vert f_{i}\left( t,x_{1},x_{2},x_{3},x_{4}\right) -f_{i}\left(
t,y_{1},y_{2},y_{3},y_{4}\right) \right\vert \leqslant \left( \varphi
_{i}\left( \max \left\{ \left\vert x_{1}-y_{1}\right\vert ,\left\vert
x_{2}-y_{2}\right\vert ,\left\vert x_{3}-y_{3}\right\vert \right\} \right)
\right) +\Phi _{i}\left( \left\vert x_{4}-y_{4}\right\vert \right) .
\end{equation*}

\item[$\left( iv\right) $] The functions defined by $\left\vert f_{i}\left(
t,0,0,0,0\right) \right\vert $\ $\left( i=1,2,3\right) $ are bounded on $\mathbb{R}_{+}$, i.e,\begin{equation}
M_{i}=\sup \left\{ f_{i}\left( t,0,0,0,0\right) :t\in \mathbb{R}_{+}\right\}
<\infty .  \label{M}
\end{equation}

\item[$\left( v\right) $] $h_{i}:\mathbb{R}_{+}\mathbb{\times R}_{+}\mathbb{\times R\times R\times R\rightarrow R}$ $\left( i=1,2,3\right) $, are
continuous functions and there exists a positive constant $D$ such that $i=1,2,3,$\begin{equation}
\sup \left\{ \left\vert \int_{0}^{q_{i}\left( t\right) }h_{i}\left(
t,s,x\left( \eta \left( s\right) \right) ,y\left( \eta \left( s\right)
\right) ,z\left( \eta \left( s\right) \right) \right) ds\right\vert :\text{ }t,s\in \mathbb{R}_{+},\text{ }x,y,z\in BC\left( \mathbb{R}_{+}\right)
\right\} <D,  \label{D}
\end{equation}and\begin{equation}
\lim_{t\rightarrow \infty }\int_{0}^{q_{i}\left( t\right) }\left[
h_{i}\left( t,s,x\left( \eta \left( s\right) \right) ,y\left( \eta \left(
s\right) \right) ,z\left( \eta \left( s\right) \right) \right) -h_{i}\left(
t,s,u\left( \eta \left( s\right) \right) ,v\left( \eta \left( s\right)
\right) ,w\left( \eta \left( s\right) \right) \right) \right] ds=0,
\label{limh}
\end{equation}with respect to $x,y,z,u,v,w\in BC\left( \mathbb{R}_{+}\right) .$
\end{enumerate}

Consider the following operator, 
\begin{equation*}
T_{i}\left( x,y,z\right) =g_{i}\left( t\right) +f_{i}\left( t,x\left( \xi
_{i}\left( t\right) \right) ,y\left( \xi _{i}\left( t\right) \right)
,z\left( \xi _{i}\left( t\right) \right) ,\psi \left( \int_{0}^{q_{i}\left(
t\right) }h\left( t,s,x\left( \eta _{i}\left( s\right) \right) ,y\left( \eta
_{i}\left( s\right) \right) ,z\left( \eta _{i}\left( s\right) \right)
\right) ds\right) \right) .
\end{equation*}Solving the system $\left( \text{\ref{sys}}\right) $ is equivalent to find
the fixed points of the operator $T_{i}$. Then let verify the conditions of
Theorem \ref{main}$.$

First, since $g_{i}$ and $f_{i}$\ $\left( i=1,2,3\right) $ are continuous
then the operators $T_{i}$ are continuous.

In further, for $x,y,z\in B_{r}$ $\left( r\right) $ $\left( \text{for }r>0\right) $ let,\begin{eqnarray*}
&&\left\Vert T_{i}\left( x,y,z\right) \left( t\right) \right\Vert \\
&=&\left\Vert g_{i}\left( t\right) +f_{i}\left( t,x\left( \xi _{i}\left(
t\right) \right) ,y\left( \xi _{i}\left( t\right) \right) ,z\left( \xi
_{i}\left( t\right) \right) ,\psi \left( \int_{0}^{q_{i}\left( t\right)
}h\left( t,s,x\left( \eta _{i}\left( s\right) \right) ,y\left( \eta
_{i}\left( s\right) \right) ,z\left( \eta _{i}\left( s\right) \right)
\right) ds\right) \right) \right\Vert \\
&\leqslant &\left\Vert f_{i}\left( t,x\left( \xi _{i}\left( t\right) \right)
,y\left( \xi _{i}\left( t\right) \right) ,z\left( \xi _{i}\left( t\right)
\right) ,\psi \left( \int_{0}^{q_{i}\left( t\right) }h\left( t,s,x\left(
\eta _{i}\left( s\right) \right) ,y\left( \eta _{i}\left( s\right) \right)
,z\left( \eta _{i}\left( s\right) \right) \right) ds\right) \right) \right.
\\
&&\left. -f\left( t,0,0,0\right) +f\left( t,0,0,0\right) \right\Vert
+\left\Vert g_{i}\left( t\right) \right\Vert \\
&\leqslant &\left\Vert g_{i}\left( t\right) \right\Vert +\left\Vert f\left(
t,0,0,0\right) \right\Vert \\
&&+\varphi _{i}\left( \max \left\{ \left\vert x\left( \xi _{i}\left(
t\right) \right) \right\vert ,\left\vert y\left( \xi _{i}\left( t\right)
\right) \right\vert ,\left\vert z\left( \xi _{i}\left( t\right) \right)
\right\vert \right\} \right) \\
&&+\Phi _{i}\left( \psi \left( \int_{0}^{q_{i}\left( t\right) }h\left(
t,s,x\left( \eta _{i}\left( s\right) \right) ,y\left( \eta _{i}\left(
s\right) \right) ,z\left( \eta _{i}\left( s\right) \right) \right) ds\right)
\right) .
\end{eqnarray*}Since, $g_{i}$ are bounded, $f_{i}$ are continuous functions and using
hypothesis $\left( iv\right) $-$\left( v\right) ,$ we get\begin{eqnarray*}
\left\Vert T_{i}\left( x,y,z\right) \right\Vert _{\infty } &\leqslant
&\varphi _{i}\left( \max \left\{ \left\Vert x\right\Vert _{\infty
},\left\Vert y\right\Vert _{\infty },\left\Vert z\right\Vert _{\infty
}\right\} \right) +G+M_{i}+\Phi _{i}\left( \delta _{i}D^{\alpha _{i}}\right)
\\
&\leqslant &\varphi _{i}\left( r\right) +G+M_{i}+\Phi _{i}\left( \delta
_{i}D^{\alpha _{i}}\right) ,
\end{eqnarray*}for some $r_{0}\geqslant 0,$ we obtain $T_{i}\left( B_{r_{0}}\times
B_{r_{0}}\times B_{r_{0}}\right) \subset B_{r_{0}}.$

Moreover,

\begin{eqnarray*}
&&\left\Vert T_{i}\left( x,y,z\right) -T_{i}\left( u,v,w\right) \right\Vert
_{\infty } \\
&=&\sup_{t}\left\Vert g_{i}\left( t\right) +f_{i}\left( t,x\left( \xi
_{i}\left( t\right) \right) ,y\left( \xi _{i}\left( t\right) \right)
,z\left( \xi _{i}\left( t\right) \right) ,\psi \left( \int_{0}^{q_{i}\left(
t\right) }h\left( t,s,x\left( \eta _{i}\left( s\right) \right) ,y\left( \eta
_{i}\left( s\right) \right) ,z\left( \eta _{i}\left( s\right) \right)
\right) ds\right) \right) \right. \\
&&\left. -g_{i}\left( t\right) -f_{i}\left( t,u\left( \xi _{i}\left(
t\right) \right) ,v\left( \xi _{i}\left( t\right) \right) ,w\left( \xi
_{i}\left( t\right) \right) ,\psi \left( \int_{0}^{q_{i}\left( t\right)
}h\left( t,s,u\left( \eta _{i}\left( s\right) \right) ,v\left( \eta
_{i}\left( s\right) \right) ,w\left( \eta _{i}\left( s\right) \right)
\right) ds\right) \right) \right\Vert \\
&\leqslant &\sup_{t}\left\Vert f_{i}\left( t,x\left( \xi _{i}\left( t\right)
\right) ,y\left( \xi _{i}\left( t\right) \right) ,z\left( \xi _{i}\left(
t\right) \right) ,\psi \left( \int_{0}^{q_{i}\left( t\right) }h\left(
t,s,x\left( \eta _{i}\left( s\right) \right) ,y\left( \eta _{i}\left(
s\right) \right) ,z\left( \eta _{i}\left( s\right) \right) \right) ds\right)
\right) \right. \\
&&\left. -f_{i}\left( t,u\left( \xi _{i}\left( t\right) \right) ,v\left( \xi
_{i}\left( t\right) \right) ,w\left( \xi _{i}\left( t\right) \right) ,\psi
\left( \int_{0}^{q_{i}\left( t\right) }h\left( t,s,u\left( \eta _{i}\left(
s\right) \right) ,v\left( \eta _{i}\left( s\right) \right) ,w\left( \eta
_{i}\left( s\right) \right) \right) ds\right) \right) \right\Vert \\
&\leqslant &\sup_{t}\left\{ \varphi _{i}\left( \max \left\{ \left\vert
x\left( \xi _{i}\left( t\right) \right) -u\left( \xi _{i}\left( t\right)
\right) \right\vert ,\left\vert y\left( \xi _{i}\left( t\right) \right)
-v\left( \xi _{i}\left( t\right) \right) \right\vert ,\left\vert z\left( \xi
_{i}\left( t\right) \right) -w\left( \xi _{i}\left( t\right) \right)
\right\vert \right\} \right) \right. \\
&&\left. +\Phi _{i}\left( \left\vert \psi \left( \int_{0}^{q_{i}\left(
t\right) }h\left( t,s,x\left( \eta _{i}\left( s\right) \right) ,y\left( \eta
_{i}\left( s\right) \right) ,z\left( \eta _{i}\left( s\right) \right)
\right) ds\right) -\psi \left( \int_{0}^{q_{i}\left( t\right) }h\left(
t,s,u\left( \eta _{i}\left( s\right) \right) ,v\left( \eta _{i}\left(
s\right) \right) ,w\left( \eta _{i}\left( s\right) \right) \right) ds\right)
\right\vert \right) \right\} \\
&\leqslant &\varphi _{i}\left( \max \left\{ \left\Vert x-u\right\Vert
_{\infty },\left\Vert y-v\right\Vert _{\infty },\left\Vert z-w\right\Vert
_{\infty }\right\} \right) \\
&&+\sup_{t}\Phi _{i}\left( \delta _{i}\left\vert \int_{0}^{q_{i}\left(
t\right) }\left\{ h\left( t,s,x\left( \eta _{i}\left( s\right) \right)
,y\left( \eta _{i}\left( s\right) \right) ,z\left( \eta _{i}\left( s\right)
\right) \right) -h\left( t,s,u\left( \eta _{i}\left( s\right) \right)
,v\left( \eta _{i}\left( s\right) \right) ,w\left( \eta _{i}\left( s\right)
\right) \right) \right\} ds\right\vert ^{\alpha _{i}}\right) .
\end{eqnarray*}Consider,\begin{equation*}
\left\vert \int_{0}^{q_{i}\left( t\right) }\left\{ h\left( t,s,x\left( \eta
_{i}\left( s\right) \right) ,y\left( \eta _{i}\left( s\right) \right)
,z\left( \eta _{i}\left( s\right) \right) \right) -h\left( t,s,u\left( \eta
_{i}\left( s\right) \right) ,v\left( \eta _{i}\left( s\right) \right)
,w\left( \eta _{i}\left( s\right) \right) \right) \right\} ds\right\vert .
\end{equation*}Using the condition $\left( \text{\ref{limh}}\right) $ we get\begin{equation*}
\left\vert \int_{0}^{q_{i}\left( t\right) }\left\{ h\left( t,s,x\left( \eta
_{i}\left( s\right) \right) ,y\left( \eta _{i}\left( s\right) \right)
,z\left( \eta _{i}\left( s\right) \right) \right) -h\left( t,s,u\left( \eta
_{i}\left( s\right) \right) ,v\left( \eta _{i}\left( s\right) \right)
,w\left( \eta _{i}\left( s\right) \right) \right) \right\} ds\right\vert
\leqslant \epsilon
\end{equation*}and\begin{equation*}
\delta _{i}\left\vert \int_{0}^{q_{i}\left( t\right) }\left\{ h\left(
t,s,x\left( \eta _{i}\left( s\right) \right) ,y\left( \eta _{i}\left(
s\right) \right) ,z\left( \eta _{i}\left( s\right) \right) \right) -h\left(
t,s,u\left( \eta _{i}\left( s\right) \right) ,v\left( \eta _{i}\left(
s\right) \right) ,w\left( \eta _{i}\left( s\right) \right) \right) \right\}
ds\right\vert ^{\alpha _{i}}\leqslant \delta _{i}\epsilon ^{\alpha _{i}}.
\end{equation*}Thus\begin{equation*}
\Phi _{i}\left( \delta _{i}\left\vert \int_{0}^{q_{i}\left( t\right)
}\left\{ h\left( t,s,x\left( \eta _{i}\left( s\right) \right) ,y\left( \eta
_{i}\left( s\right) \right) ,z\left( \eta _{i}\left( s\right) \right)
\right) -h\left( t,s,u\left( \eta _{i}\left( s\right) \right) ,v\left( \eta
_{i}\left( s\right) \right) ,w\left( \eta _{i}\left( s\right) \right)
\right) \right\} ds\right\vert ^{\alpha _{i}}\right) \leqslant \Phi
_{i}\left( \delta _{i}\epsilon ^{\alpha _{i}}\right) .
\end{equation*}On the other hand $\Phi _{i}$ is continuous function and$\ \Phi _{i}\left(
0\right) =0$ and $\epsilon $ is arbitrary, then for $\epsilon \rightarrow 0,$
we get 
\begin{equation*}
\left\Vert T_{i}\left( x,y,z\right) -T_{i}\left( u,v,w\right) \right\Vert
_{\infty }\leqslant \varphi _{i}\left( \max \left\{ \left\Vert
x-u\right\Vert _{\infty },\left\Vert y-v\right\Vert _{\infty },\left\Vert
z-w\right\Vert _{\infty }\right\} \right) .
\end{equation*}Consequently, by Theorem \ref{main}\ there exist $x^{\ast },y^{\ast
},z^{\ast }$ such that\begin{equation*}
\left\{ 
\begin{array}{c}
T_{1}\left( x^{\ast },y^{\ast },z^{\ast }\right) =x^{\ast } \\ 
T_{2}\left( x^{\ast },y^{\ast },z^{\ast }\right) =y^{\ast } \\ 
T_{3}\left( x^{\ast },y^{\ast },z^{\ast }\right) =z^{\ast }\end{array}\right. .
\end{equation*}Then, we had proved the following theorem.

\begin{theorem}
Under the conditions $\left( i\right) -\left( v\right) $ the system of
integral equations $\left( \text{\ref{sys}}\right) $ has at least one
solution in the space $BC\left( \mathbb{R}_{+}\right) \times BC\left( 
\mathbb{R}_{+}\right) \times BC\left( \mathbb{R}_{+}\right) $.
\end{theorem}

\begin{example}
Let the system of integral equations\begin{equation*}
\left\{ 
\begin{array}{l}
x\left( t\right) =\frac{t^{2}}{2+2t^{4}}+\frac{x\left( \sqrt{t}\right)
+y\left( \sqrt{t}\right) +z\left( \sqrt{t}\right) }{3t^{2}+3}+\arctan
\int_{0}^{\sqrt{t}}\frac{x\left( s^{2}\right) s\left\vert \sin y\left(
s^{2}\right) \right\vert \left\vert \cos z\left( s^{2}\right) \right\vert }{e^{t}\left( 1+x^{2}\left( s^{2}\right) \right) \left( 1+\sin ^{2}y\left(
s^{2}\right) \right) \left( 1+\cos ^{2}z\left( s^{2}\right) \right) }ds \\ 
y\left( t\right) =\frac{1}{2}e^{-t^{2}}+\frac{t^{2}\left( x\left( t\right)
+y\left( t\right) +z\left( t\right) \right) }{3t^{4}+3}+\sin \int_{0}^{t}\frac{e^{s}y^{2}\left( s\right) \left( 1+\cos ^{2}x\left( s\right) \right)
\left( 1+\sin ^{2}z\left( s\right) \right) }{e^{t^{2}}\left( 1+y^{2}\left(
s\right) \right) \left( 1+\sin ^{2}x\left( s\right) \right) \left( 1+\cos
^{2}z\left( s\right) \right) }ds \\ 
z\left( t\right) =\frac{1}{2\sqrt{1+t^{4}}}+\frac{t^{3}\left( x\left(
t\right) +y\left( t\right) +z\left( t\right) \right) }{3t^{5}+3}+\cos
\int_{0}^{t^{2}}\frac{s^{2}\left\vert \cos z\left( s\right) \right\vert +\sqrt{e^{s}\left( 1+z^{2}\left( s\right) \right) \left( 1+\sin ^{2}y\left(
s\right) \right) \left( 1+\cos ^{2}x\left( s\right) \right) }}{e^{t}\left(
1+z^{2}\left( s\right) \right) \left( 1+\sin ^{2}y\left( s\right) \right)
\left( 1+\cos ^{2}x\left( s\right) \right) }ds\end{array}\right. .
\end{equation*}We notice that by taking\begin{equation*}
g_{1}\left( t\right) =\frac{t^{2}}{2+2t^{4}}\text{, }g_{2}\left( t\right) =\frac{1}{2}e^{-t^{2}},\text{ }g_{3}\left( t\right) =\frac{1}{2\sqrt{1+t^{4}}},
\end{equation*}\begin{equation*}
\begin{array}{l}
f_{1}\left( t,x,y,z,p\right) =\frac{x+y+z}{3t^{2}+3}+p \\ 
f_{2}\left( t,x,y,z,p\right) =\frac{t^{2}\left( x+y+z\right) }{3t^{4}+3}+p
\\ 
f_{3}\left( t,x,y,z,p\right) =\frac{t^{3}\left( x+y+z\right) }{3t^{5}+3}+p\end{array},
\end{equation*}\begin{eqnarray*}
h_{1}\left( t,s,x,y,z\right) &=&\frac{xs\left\vert \sin y\right\vert
\left\vert \cos z\right\vert }{e^{t}\left( 1+x^{2}\right) \left( 1+\sin
^{2}y\right) \left( 1+\cos ^{2}z\right) } \\
h_{2}\left( t,s,x,y,z\right) &=&\frac{e^{s}\left( 1+y^{2}\right) \left(
1+\sin ^{2}x\right) \left( 1+\cos ^{2}z\right) }{e^{t^{2}}\left(
1+y^{2}\right) \left( 1+\sin ^{2}x\right) \left( 1+\cos ^{2}z\right) } \\
h_{3}\left( t,s,x,y,z\right) &=&\frac{s^{2}\left\vert \cos z\right\vert +\sqrt{e^{s}\left( 1+z^{2}\right) \left( 1+\sin ^{2}y\right) \left( 1+\cos
^{2}x\right) }}{e^{t}\left( 1+z^{2}\right) \left( 1+\sin ^{2}y\right) \left(
1+\cos ^{2}x\right) }
\end{eqnarray*}and\begin{eqnarray*}
\eta _{1}\left( t\right) &=&t^{2},\text{ }\eta _{2}\left( t\right) =\eta
_{3}\left( t\right) =t \\
\xi _{1}\left( t\right) &=&\sqrt{t},\text{ }\xi _{2}\left( t\right) =\xi
_{3}\left( t\right) =t \\
q_{1}\left( t\right) &=&\sqrt{t},\text{ }q_{2}\left( t\right) =t,\text{ }q_{3}\left( t\right) =t^{2} \\
\Psi _{1}\left( t\right) &=&\arctan t,\text{ }\Psi _{2}\left( t\right) =\sin
t,\text{ }\Psi _{3}\left( t\right) =\cos t,
\end{eqnarray*}we get the system of integral equations $\left( \text{\ref{sys}}\right) .$

To solve this system we need to verify the conditions $\left( i\right)
-\left( v\right) $.

Obviously, $\xi _{i},\eta _{i},q_{i}:\mathbb{R}_{+}\rightarrow \mathbb{R}_{+} $ are continuous and $\xi ^{i}\rightarrow \infty $ as $t\rightarrow
\infty .$ In further, the functions $\psi _{i}:\mathbb{R\rightarrow R}$ are
continuous for $\delta _{i}=\alpha _{i}=1$,\ we have\begin{equation*}
\left\vert \psi _{i}\left( t_{1}\right) -\psi _{i}\left( t_{2}\right)
\right\vert \leqslant \delta _{i}\left\vert t_{1}-t_{2}\right\vert ^{\alpha
_{i}},
\end{equation*}for any $t_{1},t_{2}\in \mathbb{R}_{+}$. The conditions $\left( i\right) $
and $\left( ii\right) $ hold.

Now, let\begin{eqnarray*}
\left\vert f_{1}\left( t,x,y,z,p\right) -f_{1}\left( t,u,v,w,\rho \right)
\right\vert &=&\left\vert \frac{x+y+z}{3t^{2}+3}+p-\left( \frac{u+v+w}{3t^{2}+3}+\rho \right) \right\vert \\
&\leqslant &\frac{1}{3t^{2}+3}\left[ \left\vert x-u\right\vert +\left\vert
y-v\right\vert +\left\vert z-w\right\vert \right] +\left\vert p-\rho
\right\vert \\
&\leqslant &\frac{3}{3t^{2}+3}\max \left[ \left\vert x-u\right\vert
,\left\vert y-v\right\vert ,\left\vert z-w\right\vert \right] +\left\vert
p-\rho \right\vert \\
&\leqslant &\frac{1}{t^{2}+1}\max \left[ \left\vert x-u\right\vert
,\left\vert y-v\right\vert ,\left\vert z-w\right\vert \right] +\left\vert
p-\rho \right\vert \\
&=&\varphi _{1}\left( \max \left[ \left\vert x-u\right\vert ,\left\vert
y-v\right\vert ,\left\vert z-w\right\vert \right] \right) +\Phi \left(
\left\vert p-\rho \right\vert \right) .
\end{eqnarray*}Similarly, we prove that\begin{equation*}
\left\vert f_{2}\left( t,x,y,z,p\right) -f_{2}\left( t,u,v,w,\rho \right)
\right\vert \leqslant \varphi _{2}\left( \max \left[ \left\vert
x-u\right\vert ,\left\vert y-v\right\vert ,\left\vert z-w\right\vert \right]
\right) +\Phi \left( \left\vert p-\rho \right\vert \right)
\end{equation*}and\begin{equation*}
\left\vert f_{3}\left( t,x,y,z,p\right) -f_{3}\left( t,u,v,w,\rho \right)
\right\vert \leqslant \varphi _{3}\left( \max \left[ \left\vert
x-u\right\vert ,\left\vert y-v\right\vert ,\left\vert z-w\right\vert \right]
\right) +\Phi \left( \left\vert p-\rho \right\vert \right) .
\end{equation*}Then, $\left( iii\right) $ also holds.

In further $\left( iv\right) $ is valid. Indeed,\begin{equation*}
M_{i}=\sup \left\vert \left\{ f_{i}\left( t,0,0,0,0\right) :t\in \mathbb{R}_{+}\right\} \right\vert =0,i=1,2,3.
\end{equation*}Let us verify the last condition $\left( v\right) .$ First, note that\begin{eqnarray*}
&&\left\vert h_{1}\left( t,s,x,y,z\right) -h_{1}\left( t,s,u,v,w\right)
\right\vert \\
&=&\left\vert \frac{xs\left\vert \sin y\right\vert \left\vert \cos
z\right\vert }{e^{t}\left( 1+x^{2}\right) \left( 1+\sin ^{2}y\right) \left(
1+\cos ^{2}z\right) }-\frac{us\left\vert \sin v\right\vert \left\vert \cos
w\right\vert }{e^{t}\left( 1+u^{2}\right) \left( 1+\sin ^{2}v\right) \left(
1+\cos ^{2}w\right) }\right. \\
&\leqslant &\left\vert \frac{x}{1+x^{2}}\frac{s}{e^{t}}-\frac{u}{1+u^{2}}\frac{s}{e^{t}}\right\vert \leqslant \frac{1}{2}\frac{s}{e^{t}}+\frac{1}{2}\frac{s}{e^{t}}. \\
&\leqslant &\frac{s}{e^{t}}.
\end{eqnarray*}Hence,\begin{eqnarray*}
&&\lim_{t\rightarrow \infty }\int_{0}^{t}\left\vert h_{1}\left( t,s,x\left(
\eta \left( s\right) \right) ,y\left( \eta \left( s\right) \right) ,z\left(
\eta \left( s\right) \right) \right) -h_{1}\left( t,s,u\left( \eta \left(
s\right) \right) ,v\left( \eta \left( s\right) \right) ,w\left( \eta \left(
s\right) \right) \right) \right\vert ds \\
&\leqslant &\lim_{t\rightarrow \infty }\int_{0}^{t}\frac{s}{e^{t}}ds=0.
\end{eqnarray*}In addition,\begin{eqnarray*}
&&\left\vert h_{2}\left( t,s,x,y,z\right) -h_{2}\left( t,s,u,v,w\right)
\right\vert \\
&=&\left\vert \frac{e^{s}\left( y^{2}\right) \left( 1+\cos ^{2}x\right)
\left( 1+\sin ^{2}z\right) }{e^{t^{2}}\left( 1+y^{2}\right) \left( 1+\sin
^{2}x\right) \left( 1+\cos ^{2}z\right) }-\frac{e^{s}\left( v^{2}\right)
\left( 1+\cos ^{2}u\right) \left( 1+\sin ^{2}w\right) }{e^{t^{2}}\left(
1+v^{2}\right) \left( 1+\sin ^{2}u\right) \left( 1+\cos ^{2}z\right) }\right.
\\
&\leqslant &\left\vert \frac{y^{2}}{1+y^{2}}\frac{e^{s}}{e^{t2}}-\frac{v^{2}}{1+v^{2}}\frac{e^{s}}{e^{t2}}\right\vert \leqslant 2\frac{e^{s}}{e^{t2}}.
\end{eqnarray*}Thus,\begin{eqnarray*}
&&\lim_{t\rightarrow \infty }\int_{0}^{t}\left\vert h_{2}\left( t,s,x\left(
\eta \left( s\right) \right) ,y\left( \eta \left( s\right) \right) ,z\left(
\eta \left( s\right) \right) \right) -h_{2}\left( t,s,u\left( \eta \left(
s\right) \right) ,v\left( \eta \left( s\right) \right) ,w\left( \eta \left(
s\right) \right) \right) \right\vert ds \\
&\leqslant &\lim_{t\rightarrow \infty }\int_{0}^{t}2\frac{e^{s}}{e^{t2}}ds=0.
\end{eqnarray*}Moreover,\begin{eqnarray*}
&&\left\vert h_{3}\left( t,s,x,y,z\right) -h_{3}\left( t,s,u,v,w\right)
\right\vert \\
&=&\left\vert \frac{s^{2}\left\vert \cos z\right\vert +\sqrt{e^{s}\left(
1+z^{2}\right) \left( 1+\sin ^{2}y\right) \left( 1+\cos ^{2}x\right) }}{e^{t}\left( 1+z^{2}\right) \left( 1+\sin ^{2}y\right) \left( 1+\cos
^{2}x\right) }-\frac{s^{2}\left\vert \cos w\right\vert +\sqrt{e^{s}\left(
1+w^{2}\right) \left( 1+\sin ^{2}v\right) \left( 1+\cos ^{2}u\right) }}{e^{t}\left( 1+w^{2}\right) \left( 1+\sin ^{2}v\right) \left( 1+\cos
^{2}u\right) }\right. \\
&\leqslant &\left\vert \frac{s^{2}}{e^{t}}\left( \cos z-\cos w\right)
\right\vert \leqslant \frac{s^{2}}{e^{t}}.
\end{eqnarray*}Then,\begin{eqnarray*}
&&\lim_{t\rightarrow \infty }\int_{0}^{t}\left\vert h_{2}\left( t,s,x\left(
\eta \left( s\right) \right) ,y\left( \eta \left( s\right) \right) ,z\left(
\eta \left( s\right) \right) \right) -h_{2}\left( t,s,u\left( \eta \left(
s\right) \right) ,v\left( \eta \left( s\right) \right) ,w\left( \eta \left(
s\right) \right) \right) \right\vert ds \\
&\leqslant &\lim_{t\rightarrow \infty }\int_{0}^{t}\frac{s^{2}}{e^{t}}ds=0.
\end{eqnarray*}Furthermore, \ for any $x,y,z\in BC\left( \mathbb{R}_{+}\right) \times
BC\left( \mathbb{R}_{+}\right) \times BC\left( \mathbb{R}_{+}\right) ,$\begin{equation*}
\sup \left\{ \left\vert \int_{0}^{t}h_{i}\left( t,s,x\left( \eta \left(
s\right) \right) ,y\left( \eta \left( s\right) \right) ,z\left( \eta \left(
s\right) \right) \right) ds\right\vert ,\text{ }t,s\in \mathbb{R}_{+}\right\} <D.
\end{equation*}It is easy to see that for an $r_{0}>0,$ we have\begin{equation*}
\varphi \left( r_{0}\right) +\frac{1}{2}+\Phi \left( D\right) \leqslant
r_{0},
\end{equation*}holds and the condition $\left( v\right) $ is valid.

Finally, the system has at least one solution in $BC\left( \mathbb{R}_{+}\right) \times BC\left( \mathbb{R}_{+}\right) \times BC\left( \mathbb{R}_{+}\right) .$
\end{example}

\bigskip

\begin{thebibliography}{99}
\bibitem{mursaleen} A. Aghajani, R. Allahyari, M. Mursaleen. A
generalization of Darbo's theorem with application to the solvability of
systems of integral equations, \textit{Comput. Math. Appl.} 260 (2014)
68--77.

\bibitem{Aghajani11} A. Aghajani, J. Banas, Y. Jalilian, Existence of
solutions for a class of nonlinear Volterra singular integral equations, 
\textit{Comput. Math. Appl.} 62 (2011) 1215--1227.

\bibitem{Aghajanicoupled} Aghajani A., Haghighi A.S., Existence of solutions
for a system of integral equations via measure of noncompactness, \textit{Novi Sad J. Math}., Vol. 44, No. 1, 2014, 59-73.

\bibitem{Aghajani2} A. Aghajani, Y. Jalilian, Existence of Nondecreasing
Positive Solutions for a System of Singular Integral Equations, \textit{Mediterr. J. Math. 8 (2011)}, 563--576.

\bibitem{kamenski} R.R. Akmerov, M.I. Kamenski, A.S. Potapov, A.E. Rodkina,
B.N. Sadovskii, Measures of Noncompactness and Condensing Operators, \textit{Birkhauser-Verlag, Basel,} 1992.

\bibitem{reza} R. Araba, R. Allahyarib, A. Haghighib, Existence of solutions
of infinite systems of integral equations in two variables via measure of
noncompactness, \textit{Applied Mathematics and Computation, }vol. 246,
2014, 283--291.

\bibitem{banas1} J. Bana\'{s}, Measures of noncompactness in the study of
solutions of nonlinear differential and integral equations, \textit{Cent.
Eur. J. Math.} 10 (6) (2012) 2003--2011.

\bibitem{banas2} J. Bana\'{s}, On measures of noncompactness in Banach
spaces, Comment. Math. Univ. Carolin. 21 (1980) 131--143.

\bibitem{lec.note} J. Bana\'{s}, K. Goebel, Measures of Noncompactness in
Banach Spaces, in: \textit{Lecture Notes in Pure and Applied Mathematics},
vol. 60, Dekker, New York, 1980.

\bibitem{banamursa} J. Bana\'{s} and M. Mursaleen, "Sequence Spaces and
Measures of Noncompactness with Applications to Differential and Integral
Equations by J. Bana\'{s} and M. Mursaleen", Springer, 2014.

\bibitem{banas3} J. Bana\'{s}, R. Rzepka, An application of a measure of
noncompactness in the study of asymptotic stability, \textit{Appl. Math. Lett}. 16 (2003) 1--6.

\bibitem{berinde} V. Berinde, M. Borcut, Tripled fixed point theorems for
contractive type mappings in partially ordered metric spaces. \textit{Nonlinear Anal. TMA} 74, 4889-4897 (2011).

\bibitem{coupled} S.S. Chang, Y.J. Cho, N.J. Huang, Coupled fixed point
theorems with applications, \textit{J. Korean Math. Soc.} 33 (1996) 575--585.

\bibitem{deepmala} Deepmala, H.K. Pathak, Study on existence of solutions
for some nonlinear functional-integral equations with applications, Math.
Commun. 18(2013), 97-107.

\bibitem{nour} Karakaya V., Bouzara N.H., Dogan K., Atalan Y., Existence of
tripled fixed points for a class of condensing operators in Banach spaces,\textit{\ The Scientific World Journal}, vol. 2014, Article ID 541862, 9
pages, 2014. doi:10.1155/2014/541862.

\bibitem{kuratovski} K. Kuratowski, Sur les espaces complets, Fund. Math. 5
(1930) 301--309.

\bibitem{murs} M. Mursaleen, S.A. Mohiuddine, Applications of measures of
noncompactness to the infinite system of differential equations in $l^{p}$
space. \textit{Nonlinear Anal.} 75 (2012) 2111--2115.

\bibitem{Shaochun} Shaochun J., Gang L., A unified approach to nonlocal
impulsive differential equations with the measure of noncompactness, \textit{Advances in Difference Equations 2012}, 1-14.

\bibitem{sikorska} A. Sikorska, ExistenceTheory for Nonlinear Volterra
Integral and Differential Equations, \textit{J. of lnequal. \& Appl}., 2001,
Vol. 6, pp. 325-338.
\end{thebibliography}

\end{document}

